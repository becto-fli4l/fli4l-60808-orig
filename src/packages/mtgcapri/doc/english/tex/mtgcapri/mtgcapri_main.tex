% Synchronized to r29817
\marklabel{sec:opt-mtgcapri}
{
\section {OPT\_MTGCAPRI - ISDN-Remote-CAPI For fli4l}
}

\subsection{Introduction}

\var{OPT\_MTGCAPRI} provides an ISDN remote-CAPI for fli4l enabling ISDN-CAPI
based applications on network clients to use a CAPI without having
the need for an ISDN card in each client with fli4l's ISDN card being shared
over the network.\\
\var{OPT\_MTGCAPRI} is only the integration of the Linux CAPRI-Server by mtG
(\altlink{http://www.mtg.de/de/}) into fli4l's infrastructure.

\subsection{Copyright}
The installation manual for clients is taken from the original documentation of
the mtG-Capriserver and is Copyrighted by mtG.

\subsection{Precondition}
    \begin{itemize}
        \item Client software is only provided for Windows by mtG. \\
        \item Usage of an AVM ISDN card. In isdn.txt a
              \var{ISDN\_TYPE} greater than 100 has to be set.
   \end{itemize}


\subsection{Configuration}
\begin{description}

\config{OPT\_MTGCAPRI}{OPT\_MTGCAPRI}{OPTMTGCAPRI}

        Default Setting: \var{OPT\_MTGCAPRI}='no'

        'yes' activates the Capri server.

\config{MTGCAPRI\_PORT}{MTGCAPRI\_PORT}{MTGCAPRIPORT}

        Default Setting: \var{MTGCAPRI\_PORT}='20000'

        This may be configured freely but should not be under 10000
        in normal use cases to avoid possible conflicts. It has to be known by the client
        resp. entered in the client's capri.ini.

        \wichtig{Please pay attention to not open this port for connections from the Internet in
        base.txt!}

\config{MTGCAPRI\_TRACELEVEL}{MTGCAPRI\_TRACELEVEL}{MTGCAPRITRACELEVEL}

        Default Setting: \var{MTGCAPRI\_TRACELEVEL}='1'

        The trace level sets the logging scope.
        Possible values are:

        \begin{itemize}
            \item '0' = no logging
            \item '1' = errors
            \item '2' = like 1 + CAPI
            \item '3' = like 1 + INF
            \item '4' = errors + CAPI + INF + EntryExit
        \end{itemize}

    \wichtig{Setting the trace level to a high value should only be used in the event of an error,
        since relatively large files may arise and thus the performance of mtG-CAPRI
        decreases. This may interfere with time-critical applications (eg fax).}

\config{MTGCAPRI\_TRACEFILE}{MTGCAPRI\_TRACEFILE}{MTGCAPRITRACEFILE}

        Default Setting: \var{MTGCAPRI\_TRACEFILE}='/var/log/capri.trc'

        The trace file serves for logging the activities of the server. Name
        and path to the file may be choosen freely.

        Alternatively you may specify 'auto' here and the file will be created
        in the systems folder for persistent data automatically.
        Please pay attention to configure \var{FLI4L\_UUID} accordingly,
        because the file might become really big and /boot or even the Ramdisk
        may overflow.

\config{MTGCAPRI\_LOGFILE}{MTGCAPRI\_LOGFILE}{MTGCAPRILOGFILE}

        Default Setting: \var{MTGCAPRI\_LOGFILE}='/var/log/caprilog.txt'

        Define the name of the logfile here.

        Alternatively you may specify 'auto' here and the file will be created
        in the systems folder for persistent data automatically.
        Please pay attention to configure \var{FLI4L\_UUID} accordingly,
        because the file might become really big and /boot or even the Ramdisk
        may overflow.

\config{MTGCAPRI\_MULTIPLEBIND}{MTGCAPRI\_MULTIPLEBIND}{MTGCAPRIMULTIPLEBIND}

        Default Setting: \var{MTGCAPRI\_MULTIPLEBIND}='no'

        Allows passive binding of several clients to one phone number on the
        server. With \var{MTGCAPRI\_MULTIPLEBIND}='yes' more clients can wait for incoming
        calls on the same phone number. With \var{MTGCAPRI\_MULTIPLEBIND}='no'
        only one phone number can be assigned per client.

\config{MTGCAPRI\_USER\_N}{MTGCAPRI\_USER\_N}{MTGCAPRIUSERN}

        Default Setting: \var{MTGCAPRI\_USER\_N}='1'

        Specify the number of users here.

\config{MTGCAPRI\_USER\_x\_NAME}{MTGCAPRI\_USER\_x\_NAME}{MTGCAPRIUSERxNAME}

        This variable holds the username. It has to be the same as the user
        account on the Windows client on which the mtg-CAPRI-Client
        is installed.

\config{MTGCAPRI\_USER\_x\_SERVICE}{MTGCAPRI\_USER\_x\_SERVICE}{MTGCAPRIUSERxSERVICE}

        Default Setting: \var{MTGCAPRI\_USER\_x\_SERVICE}='all'

        Specify the services here the user can get access to.
        Possible values are: \var{all, fax23, fax4, data64, telefon}. \\
        You can specify multiple services separated by a space. \\
        Example: \var{MTGCAPRI\_USER\_x\_SERVICE}='telefon fax23'

\config{MTGCAPRI\_USER\_x\_OWN\_NUMBERS}{MTGCAPRI\_USER\_x\_OWN\_NUMBERS}{MTGCAPRIUSERxOWNNUMBERS}

        Default Setting: \var{MTGCAPRI\_USER\_x\_OWN\_NUMBERS}='all'

        Here it is determined to which phone numbers the user is allowed to bind passively.
        Possible values are:
        \begin{itemize}
            \item 'all' = all numbers allowed
            \item 'none' = all numbers denied
            \item 'partial' = all numbers mentioned in
                \jump{MTGCAPRIUSERxOWNNUMBERSLIST}{\var{MTGCAPRI\_USER\_x\_OWN\_NUMBERS\_LIST}}
                are allowed.
        \end{itemize}

\config{MTGCAPRI\_USER\_x\_OWN\_NUMBERS\_LIST}{MTGCAPRI\_USER\_x\_OWN\_NUMBERS\_LIST}{MTGCAPRIUSERxOWNNUMBERSLIST}

        Default Setting: \var{MTGCAPRI\_USER\_x\_OWN\_NUMBERS\_LIST}=''

        For the setting \var{MTGCAPRI\_USER\_x\_OWN\_NUMBERS}='partial' define the numbers
        here which the user is allowed to access.
        Multiple numbers have to be separated by spaces. \\
        Example: \var{MTGCAPRI\_USER\_x\_OWN\_NUMBERS\_LIST}='12345 12346'

\config{MTGCAPRI\_USER\_x\_INCOMING\_NUMBERS}{MTGCAPRI\_USER\_x\_INCOMING\_NUMBERS}{MTGCAPRIUSERxINCOMINGNUMBERS}

        Default Setting: \var{MTGCAPRI\_USER\_x\_INCOMING\_NUMBERS}='all'

        This determines which numbers can connect to the server from outside.
        Possible values are:
        \begin{itemize}
           \item 'all' = all numbers allowed
            \item 'none' = all numbers denied
            \item 'partial' = all numbers mentioned in
                \jump{MTGCAPRIUSERxINCOMINGNUMBERSLIST}{\var{MTGCAPRI\_USER\_x\_INCOMING\_NUMBERS\_LIST}}
                are allowed.
        \end{itemize}

\config{MTGCAPRI\_USER\_x\_INCOMING\_NUMBERS\_LIST}{MTGCAPRI\_USER\_x\_INCOMING\_NUMBERS\_LIST}{MTGCAPRIUSERxINCOMINGNUMBERSLIST}

        Default Setting: \var{MTGCAPRI\_USER\_x\_INCOMING\_NUMBERS\_LIST}=''

        When using \var{MTGCAPRI\_USER\_x\_INCOMING\_NUMBERS}='partial' define here the numbers
        that are allowed to connect to the server from outside.
        Multiple numbers have to be separated by spaces.
        Only the specified digits from the beginning of the number are compared.
        \\
        Example: \var{MTGCAPRI\_USER\_x\_INCOMING\_NUMBERS\_LIST}='0172123456 0511'

        This setting allows only connections from the phone number '0172123456 'and
        from all numbers from the area code '0511 '.

\config{MTGCAPRI\_USER\_x\_OUTGOING\_NUMBERS}{MTGCAPRI\_USER\_x\_OUTGOING\_NUMBERS}{MTGCAPRIUSERxOUTGOINGNUMBERS}

        Default Setting: \var{MTGCAPRI\_USER\_x\_OUTGOING\_NUMBERS}='all'

        This determines which numbers are LOCKED for a connection to the outside.

        Possible values are:
        \begin{itemize}
           \item 'all' = all numbers locked
            \item 'none' = all numbers open
            \item 'partial' = all numbers mentioned in
                \jump{MTGCAPRIUSERxOUTGOINGNUMBERSLIST}{\var{MTGCAPRI\_USER\_x\_OUTGOING\_NUMBERS\_LIST}}
                 are locked.
        \end{itemize}

\config{MTGCAPRI\_USER\_x\_OUTGOING\_NUMBERS\_LIST}{MTGCAPRI\_USER\_x\_OUTGOING\_NUMBERS\_LIST}{MTGCAPRIUSERxOUTGOINGNUMBERSLIST}

        Default Setting: \var{MTGCAPRI\_USER\_x\_OUTGOING\_NUMBERS\_LIST}=''

        When using \var{MTGCAPRI\_USER\_x\_OUTGOING\_NUMBERS}='partial' define here the numbers
        that are LOCKED for a connection to the outside.
        Multiple numbers have to be separated by spaces.
        Only the specified digits from the beginning of the number are compared.\\

        Example: \var{MTGCAPRI\_USER\_x\_OUTGOING\_NUMBERS\_LIST}='0900 0180'

        This setting denies all connections to phone numbers starting with '0900' or '0180'.

\config{MTGCAPRI\_USER\_x\_TIME\_XX}{MTGCAPRI\_USER\_x\_TIME\_XX}{MTGCAPRIUSERxTIMEXX}

        Default Setting: \var{MTGCAPRI\_USER\_x\_TIME\_XX}='0:0 0:0'

        Specify the times here at which the user is allowed to use the CAPI (Monday-Sunday).\\
        here some examples:
        \begin{verbatim}
            '0:0 0:0'     - no time restrictions
            '9:0 17:30'   - usage only from 09:00 to 17:30
            '24:00 24:00' - no usage
        \end{verbatim}


\end{description}

\subsection{Installation Of The mtG-CAPRI-Client-Software}
    The Client-Installation differs for a Windows95- and a WindowsNT-based system,
    due to system architecture different files are needed.

    When setting up the mtG-CAPRI client the following important point has to be noted: \\
    If there is or was already a CAPI application installed on the computer (in other words,
    a local ISDN card) and still one of the following dll files is located on the hard disk,
    these must be removed before mtG-CAPRI installation: CAPI20.DLL and CAPI2032.DLL;
    both of which are normally located in the system directory.\\

    If this is the case, one of the two following ways has to be followed:\\
    \\
    a) Deinstallation of the ISDN card and check if the files \var{capi20.dll} and \var{capi2032.dll} have been removed.\\
    b) Renaming of these dlls if they are needed again later for the activation of the local ISDN card
    and hence deactivation of the mtG-CAPRI. Renaming deactivates the drivers of the builtin card.\\

    In the normal case, those files are not overwritten by the Wise installer during installation
    (beside the version number other manufacturer data is verified). Wise Installation System would
    not perform the installation correctly.

\subsubsection{Adapting The Initialisation File capri.ini Of The mtG-CAPRI-Client}
    The file capri.ini serves for the initialisation of the mtG-CAPRI-Client, i.e. to identify the server
    to connect to and should have the following content:
\begin{verbatim}
[CAPRI]
SERVERNAME = Remote:Thor
PORTNUMBER = 20000
TRACELEVEL = 0
TRACEFILE = c:\tmp\capri.trc
FLOWCTRL = 7
\end{verbatim}

\begin{description}
\config{[CAPRI]}{CAPRI}{CAPRI}

    The header of the file must not be changed.

\config{SERVERNAME}{SERVERNAME}{SERVERNAME}

    Behind 'Remote:' the alias of the server for the TCP/IP network has to be placed
    (corresponding to the entry in the host file and the mtG-CAPRI server file capri.cfg).
\config{PORTNUMBER}{PORTNUMBER}{PORTNUMBER}

    This entry must match the one in the capri.cfg server file.

\config{TRACELEVEL}{TRACELEVEL}{TRACELEVEL}

    The value of the trace level is normally '0' for the mtG-CAPRI client.\\
    Possible values are:
    \begin{itemize}
            \item '0' = no logging
            \item '1' = errors
            \item '2' = like 1 + CAPI
            \item '3' = like 1 + INF
            \item '4' = errors + CAPI + INF + EntryExit
   \end{itemize}

    \wichtig{Setting the trace level to a high value should only be used in the event of an error,
        since relatively large files may arise and thus the performance of mtG-CAPRI
        decreases. This may interfere with time-critical applications (eg fax).}

\config{TRACEFILE}{TRACEFILE}{TRACEFILE}

    Path and name of the trace file created by mtG-CAPRI.
    This entry may be edited (here: capri.trc).

\config{FLOWCTRL}{FLOWCTRL}{FLOWCTRL}

    A flow control will be performed when sending data packets.\\
    Possible values are:\\
    \begin{itemize}
        \item 0 = no flow control
        \item 1 = wait for acknowledgment after each data packet
        \item 2 = submit a maximum of 2 data packets without acknowledgment
        \item 3 = submit a maximum of 3 data packets without acknowledgment
        \item ... other values in analogue
        \item 7 = submit a maximum of 7 data packets without acknowledgment (Default)
    \end{itemize}

    Values higher than 7 are possible, but not recommended. The CAPI specification has a value of 7.
\end{description}

\subsubsection{Testing The Client}
    The bundled software caprit32.exe (for 32bit environment) and caprit16.exe
     (for 16bit environment) can be started by double clicking.
     Normally, the message 'mtG-CAPRI test was successful' will appear.
     However, if the message 'mtG-CAPRI test failed' appears, the following points should be checked:
     \begin{itemize}
        \item Have all the steps of installation on the client and server completed correctly?
        \item Is the name of the mtG-CAPRI server written correctly in the capri.ini file of the client?
        \item In file C:\\Windows\\Hosts (Windows95) resp. ...\\System32\\Drivers\\Etc\\Hosts
            (WindowsNT® and up) the server computer must be registered with the correct alias.
        \item Is the network connection to the server established?
        \item Is the user who is logged in registered and provided with sufficient privileges
        in the authentication file of the mtG-CAPRI server?
    \end{itemize}

    A CAPI application (such as T-Online) should only be started from the client, if this
    test was successfully completed.

\subsubsection{Error Messages From CAPI Applications}
    The error messages from CAPI applications (such as T-Online, FritzFax etc.) are made for the case
    that an ISDN card with corresponding software is installed locally on the respective computer.
    Hence they are often misleading for mtg-CAPRI.\\
    Examples: \\

    The message 'driver for CAPI 2.0 not installed on this computer' means that
    CAPI (now in the interaction between client and server) is not functional.
    With mtG-CAPRI this can for example mean that anything is o.k. on the client but
    the network connection to the server is interrupted or the server is not running. \\
    T-Online i.e. produces the following error message when the user was not authenticated on the server:
    'Cannot initialise DDE (WSOCK32)'. \\
    \\
    If the server name is specified incorrectly on the client, I get the message 'Result of the connection setup:
    ISDN CAPI drivers are missing ... '. \\
    With such 'cryptic' messages it is recommended:
    \begin{itemize}
        \item on the client start the test programs caprit32.exe and start caprit16.exe first
         to see whether the connection to mtG-CAPRI server can be established or
        \item check the trace file on the server to control information it provides about problems
         as 'unauthorized user', 'no channel free', 'requested channel not available' etc.
    \end{itemize}

