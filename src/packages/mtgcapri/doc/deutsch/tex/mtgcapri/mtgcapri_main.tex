% Last Update: $Id$
\marklabel{sec:opt-mtgcapri}
{
\section {OPT\_MTGCAPRI - Die ISDN-Remote-CAPI für fli4l}
}

\subsection{Einleitung}

Das \var{OPT\_MTGCAPRI} stellt eine ISDN-Remote-CAPI auf fli4l zur Verfügung,
wodurch ISDN-CAPI basierende Anwendungen, auf verschiedenen Rechnern in einem
Netzwerk ausgeführt werden können, ohne dass in jedem eine ISDN-Karte eingebaut
sein muss. Die ISDN-Karte des Routers wird also quasi im Netzwerk geteilt. \\
Das \var{OPT\_MTGCAPRI} stellt dabei nur die Integration des
Linux-CAPRI-Servers der Firma mtG (\altlink{http://www.mtg.de/de/}) in die für
fli4l benötigte Infrastruktur dar.

\subsection{Copyright}
Die Client-Installations-Anleitung ist der Original Dokumentation des
mtG-Capriservers entnommen und unterliegt somit dem Copyright der Firma mtG.

\subsection{Voraussetzung}
    \begin{itemize}
        \item Die benötigte Clientsoftware stellt mtG nur für Windows zur Verfügung. \\
        \item Die Verwendung einer ISDN-Karten der Firma AVM. In der isdn.txt ist
              ein \var{ISDN\_TYPE} größer 100 einzutragen.
   \end{itemize}


\subsection{Konfiguration}
\begin{description}

\config{OPT\_MTGCAPRI}{OPT\_MTGCAPRI}{OPTMTGCAPRI}

        Standard-Einstellung: \var{OPT\_MTGCAPRI}='no'

        Mit der Einstellung 'yes' wird der Capriserver aktiviert.

\config{MTGCAPRI\_PORT}{MTGCAPRI\_PORT}{MTGCAPRIPORT}

        Standard-Einstellung: \var{MTGCAPRI\_PORT}='20000'

        Dieser Wert ist frei wählbar, sollte aber im Normalfall nicht unter 10000
        liegen, um mögliche Konflikte zu vermeiden. Er muss außerdem dem Client
        mitgeteilt bzw. dort in der Datei capri.ini eingetragen werden.

        \wichtig{Man sollte darauf achten, dass dieser Port in der
        base.txt nicht für Verbindungen aus dem Internet freigegeben wird!}

\config{MTGCAPRI\_TRACELEVEL}{MTGCAPRI\_TRACELEVEL}{MTGCAPRITRACELEVEL}

        Standard-Einstellung: \var{MTGCAPRI\_TRACELEVEL}='1'

        Der Trace-Level gibt den Aufzeichnungsumfang an.
        Die möglichen Werte des Trace-Levels sind:

        \begin{itemize}
            \item '0' = keine Aufzeichnung
            \item '1' = Fehler
            \item '2' = wie 1 + CAPI
            \item '3' = wie 1 + INF
            \item '4' = Fehler + CAPI + INF + EntryExit
        \end{itemize}

    \wichtig{Ein Hochsetzen des Trace-Levels sollte nur im Fehlerfall passieren,
        da relativ große Dateien entstehen können und dadurch die Performance von mtG-CAPRI,
        sinkt. Damit können unter Umständen bei zeitkritischen Anwendungen (z.B. Fax)
        Probleme entstehen.}

\config{MTGCAPRI\_TRACEFILE}{MTGCAPRI\_TRACEFILE}{MTGCAPRITRACEFILE}

        Standard-Einstellung: \var{MTGCAPRI\_TRACEFILE}='/var/log/capri.trc'

        Das Tracefile dient dem Server zur Aufzeichnung der Aktivitäten. Der Name
        und Ort der Datei ist frei wählbar.

        Alternativ kann hier 'auto' angegeben werden, sodaß das File automatisch
        im Systemordner für persistente Daten abgelegt wird.
        Bitte darauf achten, daß dann auch \var{FLI4L\_UUID} entsprechend konfiguriert wird,
        da das File sehr groß werden kann und ansonsten /boot oder gar die Ramdisk schnell
        überfüllt.

\config{MTGCAPRI\_LOGFILE}{MTGCAPRI\_LOGFILE}{MTGCAPRILOGFILE}

        Standard-Einstellung: \var{MTGCAPRI\_LOGFILE}='/var/log/caprilog.txt'

        Der Name der Log-Datei kann ebenfalls geändert werden.

        Alternativ kann hier 'auto' angegeben werden, sodaß das File automatisch
        im Systemordner für persistente Daten abgelegt wird.
        Bitte darauf achten, daß dann auch \var{FLI4L\_UUID} entsprechend konfiguriert wird,
        da das File sehr groß werden kann und ansonsten /boot oder gar die Ramdisk schnell
        überfüllt.

\config{MTGCAPRI\_MULTIPLEBIND}{MTGCAPRI\_MULTIPLEBIND}{MTGCAPRIMULTIPLEBIND}

        Standard-Einstellung: \var{MTGCAPRI\_MULTIPLEBIND}='no'

        Erlaubt passives Zuordnen (Binden) von mehreren Clients zu einer Rufnummer auf
        dem Server. Mit \var{MTGCAPRI\_MULTIPLEBIND}='yes' können mehrere Clients auf
        der gleichen Rufnummer auf eingehende Rufe warten. Mit \var{MTGCAPRI\_MULTIPLEBIND}='no'
        kann einem Client nur eine Rufnummer zugeordnet werden.

\config{MTGCAPRI\_USER\_N}{MTGCAPRI\_USER\_N}{MTGCAPRIUSERN}

        Standard-Einstellung: \var{MTGCAPRI\_USER\_N}='1'

        Hier wird die Anzahl der Benutzer festgelegt.

\config{MTGCAPRI\_USER\_x\_NAME}{MTGCAPRI\_USER\_x\_NAME}{MTGCAPRIUSERxNAME}

        Diese Variable enthält den Namen des Benutzers. Er muss dem Namen des
        Benutzeraccounts auf dem Windows-Client entsprechen, auf dem der
        mtg-CAPRI-Client installiert ist.

\config{MTGCAPRI\_USER\_x\_SERVICE}{MTGCAPRI\_USER\_x\_SERVICE}{MTGCAPRIUSERxSERVICE}

        Standard-Einstellung: \var{MTGCAPRI\_USER\_x\_SERVICE}='all'

        Hier werden die Dienste festgelegt, die der Benutzer in Anspruch nehmen darf.
        Mögliche Werte sind: \var{all, fax23, fax4, data64, telefon}. \\
        Man kann mehrere Dienste durch ein Leerzeichen getrennt angeben. \\
        Beispiel: \var{MTGCAPRI\_USER\_x\_SERVICE}='telefon fax23'

\config{MTGCAPRI\_USER\_x\_OWN\_NUMBERS}{MTGCAPRI\_USER\_x\_OWN\_NUMBERS}{MTGCAPRIUSERxOWNNUMBERS}

        Standard-Einstellung: \var{MTGCAPRI\_USER\_x\_OWN\_NUMBERS}='all'

        Hier wird festgelegt auf welche Telefon-Nummern sich der Benutzer passiv binden darf.
        Mögliche Werte sind:
        \begin{itemize}
            \item 'all' = alle Nummern sind zugelassen
            \item 'none' = alle Nummern gesperrt
            \item 'partial' = alle in
                \jump{MTGCAPRIUSERxOWNNUMBERSLIST}{\var{MTGCAPRI\_USER\_x\_OWN\_NUMBERS\_LIST}}
                angegebenen Nummern sind zugelassen.
        \end{itemize}

\config{MTGCAPRI\_USER\_x\_OWN\_NUMBERS\_LIST}{MTGCAPRI\_USER\_x\_OWN\_NUMBERS\_LIST}{MTGCAPRIUSERxOWNNUMBERSLIST}

        Standard-Einstellung: \var{MTGCAPRI\_USER\_x\_OWN\_NUMBERS\_LIST}=''

        Hier werden, bei der Einstellung \var{MTGCAPRI\_USER\_x\_OWN\_NUMBERS}='partial', die Nummern festgelegt,
        die der Benutzer benutzen darf.
        Mehrere Nummern müssen durch ein Leerzeichen getrennt werden. \\
        Beispiel: \var{MTGCAPRI\_USER\_x\_OWN\_NUMBERS\_LIST}='12345 12346'

\config{MTGCAPRI\_USER\_x\_INCOMING\_NUMBERS}{MTGCAPRI\_USER\_x\_INCOMING\_NUMBERS}{MTGCAPRIUSERxINCOMINGNUMBERS}

        Standard-Einstellung: \var{MTGCAPRI\_USER\_x\_INCOMING\_NUMBERS}='all'

        Hier wird festgelegt, welche Nummern von außen Verbindung mit dem Server aufnehmen können.
        Mögliche Werte sind:
        \begin{itemize}
            \item 'all' = alle Nummern sind zugelassen
            \item 'none' = alle Nummern gesperrt
            \item 'partial' = alle in
                \jump{MTGCAPRIUSERxINCOMINGNUMBERSLIST}{\var{MTGCAPRI\_USER\_x\_INCOMING\_NUMBERS\_LIST}}
                angegebenen Nummern sind zugelassen.
        \end{itemize}

\config{MTGCAPRI\_USER\_x\_INCOMING\_NUMBERS\_LIST}{MTGCAPRI\_USER\_x\_INCOMING\_NUMBERS\_LIST}{MTGCAPRIUSERxINCOMINGNUMBERSLIST}

        Standard-Einstellung: \var{MTGCAPRI\_USER\_x\_INCOMING\_NUMBERS\_LIST}=''

        Hier werden, bei der Einstellung \var{MTGCAPRI\_USER\_x\_INCOMING\_NUMBERS}='partial', die Nummern festgelegt,
        die von außen Verbindung mit dem Server aufnehmen können.
        Mehrere Nummern müssen durch ein Leerzeichen getrennt werden.
        Es werden nur die angegebenen Ziffern vom Anfang der Nummer verglichen.
        \\
        Beispiel: \var{MTGCAPRI\_USER\_x\_INCOMING\_NUMBERS\_LIST}='0172123456 0511'

        Diese Einstellung erlaubt nur Verbindungen von der Telefonnummer '0172123456' und
        von allen Nummern aus dem Vorwahlbereich '0511'.

\config{MTGCAPRI\_USER\_x\_OUTGOING\_NUMBERS}{MTGCAPRI\_USER\_x\_OUTGOING\_NUMBERS}{MTGCAPRIUSERxOUTGOINGNUMBERS}

        Standard-Einstellung: \var{MTGCAPRI\_USER\_x\_OUTGOING\_NUMBERS}='all'

        Hier wird festgelegt, welche Nummern für eine Verbindung nach außen GESPERRT sind.

        Mögliche Werte sind:
        \begin{itemize}
            \item 'all' = alle Nummern sind zugelassen
            \item 'none' = alle Nummern gesperrt
            \item 'partial' = alle in
                \jump{MTGCAPRIUSERxOUTGOINGNUMBERSLIST}{\var{MTGCAPRI\_USER\_x\_OUTGOING\_NUMBERS\_LIST}}
                angegebenen Nummern sind NICHT zugelassen.
        \end{itemize}

\config{MTGCAPRI\_USER\_x\_OUTGOING\_NUMBERS\_LIST}{MTGCAPRI\_USER\_x\_OUTGOING\_NUMBERS\_LIST}{MTGCAPRIUSERxOUTGOINGNUMBERSLIST}

        Standard-Einstellung: \var{MTGCAPRI\_USER\_x\_OUTGOING\_NUMBERS\_LIST}=''

        Hier werden, bei der Einstellung \var{MTGCAPRI\_USER\_x\_OUTGOING\_NUMBERS}='partial', die Nummern festgelegt,
        die für eine Verbindung nach außen GESPERRT sind.
        Mehrere Nummern müssen durch ein Leerzeichen getrennt werden.
        Es werden nur die angegebenen Ziffern vom Anfang der Nummer verglichen.\\

        Beispiel: \var{MTGCAPRI\_USER\_x\_OUTGOING\_NUMBERS\_LIST}='0900 0180'

        Diese Einstellung sperrt alle Verbindungen zu Telefonnummern, die mit '0900' oder '0180'
        beginnen.

\config{MTGCAPRI\_USER\_x\_TIME\_XX}{MTGCAPRI\_USER\_x\_TIME\_XX}{MTGCAPRIUSERxTIMEXX}

        Standard-Einstellung: \var{MTGCAPRI\_USER\_x\_TIME\_XX}='0:0 0:0'

        Hier werden die Zeiten festgelegt zu denen der Benutzer die CAPI benutzen darf (Montag-Sonntag). \\
        Hier ein paar Beispiele:
        \begin{verbatim}
            '0:0 0:0'     - keine zeitliche Einschränkung
            '9:0 17:30'   - nur von 09:00 bis 17:30 Inanspruchnahme möglich
            '24:00 24:00' - keine Inanspruchnahme möglich
        \end{verbatim}


\end{description}

\subsection{Installation der mtG-CAPRI-Client-Software}
    Bei der Client-Installation muß zwischen einem Windows95- und einem WindowsNT-basierten System
    differenziert werden, da durch Unterschiede in der Systemarchitektur verschiedene Dateien benötigt werden.

    Bei der Einrichtung des mtG-CAPRI-Client ist folgender wichtiger Punkt zu beachten: \\
    Falls auf dem Rechner schon eine CAPI-Anwendung (mit anderen Worten: eine lokale ISDN-Karte) installiert
    ist bzw. installiert war und sich noch eine der folgenden dll-Dateien auf der Festplatte befindet,
    müssen diese vor der mtG-CAPRI-Installation entfernt werden: capi20.dll und capi2032.dll;
    beide befinden sich im Normalfall im System-Verzeichnis. \\

    Sollte dies der Fall sein, ist einer der beiden folgenden Wege zu beschreiten: \\
    \\
    a) Deinstallation der ISDN-Karte und Überprüfung, ob die Dateien \var{capi20.dll} und \var{capi2032.dll} entfernt wurden. \\
    b) Umbenennen dieser Programmbibliotheken, falls man zu einem späteren Zeitpunkt die lokale ISDN-Karte wieder
    aktivieren, und damit mtG-CAPRI deaktivieren, will. Das Umbenennen führt zum Deaktivieren der Treiber der eingebauten Karte. \\

    Im Normalfall können nämlich diese Dateien vom Wise Installation System während der Installation
    nicht überschrieben werden (es werden außer der Versionsnummer auch noch andere Herstellerangaben überprüft),
    Wise Installation System könnte die Installation nicht korrekt durchführen.

\subsubsection{Anpassung der Initialisierungsdatei capri.ini des mtG-CAPRI-Client}
    Die Datei capri.ini dient der Initialisierung des mtG-CAPRI-Client, u.a. der Identifizierung des Servers,
    zu dem die Verbindung aufgebaut werden muß und sollte folgende Eintragungen haben:
\begin{verbatim}
[CAPRI]
SERVERNAME = Remote:Thor
PORTNUMBER = 20000
TRACELEVEL = 0
TRACEFILE = c:\tmp\capri.trc
FLOWCTRL = 7
\end{verbatim}

\begin{description}
\config{[CAPRI]}{CAPRI}{CAPRI}

    Die Kopfzeile der Datei darf nicht geändert werden.

\config{SERVERNAME}{SERVERNAME}{SERVERNAME}

    Hinter 'Remote:' muß der Alias des Servers für das TCP/IP Netzwerk stehen
    (entsprechend dem Eintrag der Host-Datei und der mtG-CAPRI-Server-Datei capri.cfg).

\config{PORTNUMBER}{PORTNUMBER}{PORTNUMBER}

    Dieser Eintrag muß mit dem in der Server-Datei capri.cfg übereinstimmen.

\config{TRACELEVEL}{TRACELEVEL}{TRACELEVEL}

    Der Wert des Trace-Levels beträgt im Normalfall für den mtG-CAPRI-Client '0'.\\
    Die möglichen Werte sind:
    \begin{itemize}
        \item 0 = keine Aufzeichnung
        \item 1 = Fehler
        \item 2 = wie 1 + CAPI
        \item 3 = wie 1 + INF
        \item 4 = Fehler + CAPI + INF + EntryExit
    \end{itemize}

    \wichtig{Ein Hochsetzen des Trace-Levels sollte nur im Fehlerfall passieren, d
    a relativ große Dateien entstehen können und dadurch die Performance von mtG-CAPRI sinkt.
    Damit können unter Umständen bei zeitkritischen Anwendungen (z.B. Fax) Probleme entstehen.}

\config{TRACEFILE}{TRACEFILE}{TRACEFILE}

    Pfad und Name der Trace- (Aufzeichnungs-) Datei, die von mtG-CAPRI angelegt wird.
    Der Eintrag kann editiert werden (hier: capri.trc).

\config{FLOWCTRL}{FLOWCTRL}{FLOWCTRL}

    Es wird eine Flußkontrolle beim Versenden von Datenpaketen durchgeführt. \\
    Mögliche Werte sind: \\
    \begin{itemize}
        \item 0 = keine Flußkontrolle durchführen
        \item 1 = nach jedem Datenpaket Quittung abwarten
        \item 2 = maximal 2 Datenpakete ohne Quittung absenden
        \item 3 = maximal 3 Datenpakete ohne Quittung absenden
        \item ... andere Werte analog
        \item 7 = maximal 7 Datenpakete ohne Quittung absenden (Default)
    \end{itemize}

    Es sind höhere Werte als 7 möglich, werden aber nicht empfohlen. Die CAPI-Spezifikation sieht einen Wert von 7 vor.

\end{description}

\subsubsection{Testen des Clients}
    Die mitgelieferten Programme caprit32.exe (für 32bit-Umgebung) und caprit16.exe
    (für 16bit-Umgebung) werden per Doppelklick gestartet.
    Im Normalfall erscheint die Meldung 'mtG-CAPRI Test war erfolgreich'.
    Erscheint hingegen die Meldung 'mtG-CAPRI Test schlug fehl', sollten folgende Punkte geprüft werden:
    \begin{itemize}
        \item Wurden alle Schritte der Installation auf Client und Server korrekt ausgeführt?
        \item Ist der Name des mtG-CAPRI-Servers in der Datei capri.ini auf dem Client korrekt geschrieben?
        \item In der Datei C:\\Windows\\Hosts (bei Windows95) bzw. ...\\System32\\Drivers\\Etc\\Hosts
            (bei WindowsNT®) muß der Server-Rechner mit dem korrekten Alias eingetragen sein.
        \item Steht die Netzverbindung zu dem Server?
        \item Ist der gerade angemeldete Benutzer in der Authentifizierungsdatei des mtG-CAPRI-Servers
            eingetragen und mit ausreichenden Rechten versehen?
    \end{itemize}

    Eine CAPI-Applikation (z.B. T-Online) sollte erst dann vom Client aus gestartet werden, wenn dieser
    Test erfolgreich absolviert wurde.

\subsubsection{Fehlermeldungen von CAPI-Anwendungen}
    Die Fehlermeldungen von CAPI-Anwendungen (wie z.B. T-Online, FritzFax etc.) sind für den Fall gedacht,
    dass eine ISDN-Karte mit entsprechender Software lokal im jeweiligen Rechner installiert ist.
    Ihre Texte sind daher unter mtg-CAPRI, oft irreführend. \\
    \\
    Beispiele: \\

    Die Meldung 'Treiber für CAPI 2.0 auf diesem Rechner nicht installiert' bedeutet, dass
    die CAPI (jetzt aber im Zusammenspiel von Client und Server) nicht funktionsfähig ist.
    Im Zusammenhang mit mtG-CAPRI, kann das beispielsweise heißen, dass zwar auf dem Client alles in Ordnung ist,
    die Netzverbindung zum Server aber unterbrochen oder der Server gar nicht in Betrieb ist. \\
    \\
    T-Online gibt etwa folgende Fehlermeldung aus, wenn der Benutzer nicht auf dem Server authentisiert wurde:
    'Cannot initialise DDE (WSOCK32)'. \\
    \\
    Wenn der Server-Name auf dem Client falsch angegeben ist, kommt die Meldung 'Ergebnis des Verbindungsaufbaus:
    Der für den ISDN-Betrieb nötige CAPI-Treiber fehlt oder ...'. \\
    \\
    Bei solch 'kryptischen' Meldungen wird daher empfohlen:
    \begin{itemize}
        \item vom Client zunächst die Test-Programme caprit32.exe und caprit16.exe zu starten, um zu sehen, ob die Verbindung zum mtG-CAPRI-Server hergestellt werden kann oder
        \item die Trace-Datei auf dem Server zu kontrollieren, sie gibt Aufschluß über Probleme wie 'unberechtigter Benutzer', 'kein Kanal frei', 'angeforderter Kanal nicht verfügbar' etc.
    \end{itemize}

