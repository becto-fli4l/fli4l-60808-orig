% Synchronized to r43580
\marklabel{sec:opt-mtgcapri}
{
\section {OPT\_MTGCAPRI - Interface CAPI ISDN pour fli4l}
}

\subsection{Introduction}

Le paquetage \var{OPT\_MTGCAPRI} offre une interface CAPI ISDN (en France RNIS ou numeris)
pour fli4l, il permet aux applications qui utilise le CAPI ISDN d'être exécutés sur
différents ordinateurs du réseau, sans qu'il y est de carte ISDN installée sur d'autre ordinateur.
La carte ISDN est installée sur le  routeur elle est donc partagé sur le réseau. \\
\var{OPT\_MTGCAPRI} est l'intégration d'un serveur Linux CAPRI de la société mtG
(\altlink{http://www.mtg.de/}) dans l'infrastructure de fli4l.

\subsection{Copyright}
Le manuel d'installation pour les clients a été récupéré à partir de la documentation d'origine
du serveur mtG Capri, il est soumis au Copyright de mtG.

\subsection{Condition préalable}
    \begin{itemize}
        \item Le logiciel client mtG est nécessaire, il est uniquement disponible pour Windows. \\
        \item L'utilisation une carte ISDN de la société AVM. Dans le fichier isdn.txt vous pouvez
			paramétrer la variable \var{ISDN\_TYPE} avec plus 100 types différents.
   \end{itemize}


\subsection{Configuration}
\begin{description}

\config{OPT\_MTGCAPRI}{OPT\_MTGCAPRI}{OPTMTGCAPRI}

        Paramètre par défaut~: \var{OPT\_MTGCAPRI}='no'

        Si vous mettre cette variable sur 'yes' de serveur Capri sera activé.

\config{MTGCAPRI\_PORT}{MTGCAPRI\_PORT}{MTGCAPRIPORT}

        Paramètre par défaut~: \var{MTGCAPRI\_PORT}='20000'

        La valeur du port est arbitraire, mais ne doit pas être inférieure à 10000, pour éviter
		d'éventuels conflits en utilisation normal. Il doit être reconnu et configuré dans
		le fichier capri.ini de chaque client.

        \wichtig{On doit s'assurer que ce port n'est pas ouvert dans base.txt pour
		les connexions sur Internet~!}

\config{MTGCAPRI\_TRACELEVEL}{MTGCAPRI\_TRACELEVEL}{MTGCAPRITRACELEVEL}

        Paramètre par défaut~: \var{MTGCAPRI\_TRACELEVEL}='1'

        Dans cette variable vous spécifiez le niveau de trace, les traces du programme
		seront enregistrées dans un fichier.
		Les valeurs possibles pour le niveau de trace sont~:

        \begin{itemize}
            \item '0' = aucun enregistrement
            \item '1' = Erreur
            \item '2' = comme 1 + CAPI
            \item '3' = comme 1 + INF
            \item '4' = Erreur + CAPI + INF + EntryExit
        \end{itemize}

    \wichtig{Si vous avez une erreur dans votre programme, évitez d'utiliser le niveau
		de trace élevé, car le fichier grossit relativement vite et la performance
		du mtG CAPRI diminuera. Des problèmes peuvent apparaître et cela peut interférer avec
		certaines applications (par exemple pour les fax).}

\config{MTGCAPRI\_TRACEFILE}{MTGCAPRI\_TRACEFILE}{MTGCAPRITRACEFILE}

        Paramètre par défaut~: \var{MTGCAPRI\_TRACEFILE}='/var/log/capri.trc'

        Dans cette variable vous pouvez modifier le nom du fichier qui enregistre les traces.
		Le nom et l'emplacement du fichier est arbitraire.

        Vous pouvez spécifiez 'auto' ainsi le fichier de données sera automatiquement enregistré
		dans un dossier système persistantes.
		S'il vous plaît faite attention à configurer \var{FLI4L\_UUID} en conséquence, car
		le fichier peut devenir très gros et le /boot ou même le disque virtuel peut déborder.

\config{MTGCAPRI\_LOGFILE}{MTGCAPRI\_LOGFILE}{MTGCAPRILOGFILE}

        Paramètre par défaut~: \var{MTGCAPRI\_LOGFILE}='/var/log/caprilog.txt'

        Dans cette variable vous pouvez également modifier le nom du fichier journal

        Vous pouvez spécifiez 'auto' ainsi le fichier de données sera automatiquement enregistré
		dans un dossier système persistantes.
		S'il vous plaît faite attention à configurer \var{FLI4L\_UUID} en conséquence, car
		le fichier peut devenir très gros et le /boot ou même le disque virtuel peut déborder.

\config{MTGCAPRI\_MULTIPLEBIND}{MTGCAPRI\_MULTIPLEBIND}{MTGCAPRIMULTIPLEBIND}

        Paramètre par défaut~: \var{MTGCAPRI\_MULTIPLEBIND}='no'

        Le serveur permet d'attribuer une (liaison) passive entre plusieurs clients et
		avec un numéro de téléphone. Si la variable est sur \var{MTGCAPRI\_MULTIPLEBIND}='yes'
		plusieurs clients peuvent attendre un appel entrant sur le même numéro de téléphone.
		Si la variable est sur \var{MTGCAPRI\_MULTIPLEBIND}='no' un seul numéro de téléphone
		peut être affecté par client.

\config{MTGCAPRI\_USER\_N}{MTGCAPRI\_USER\_N}{MTGCAPRIUSERN}

        Paramètre par défaut~: \var{MTGCAPRI\_USER\_N}='1'

        Dans cette variable vous activez le nombre d'utilisateurs.

\config{MTGCAPRI\_USER\_x\_NAME}{MTGCAPRI\_USER\_x\_NAME}{MTGCAPRIUSERxNAME}

        Dans cette variable vous indiquez le nom de l'utilisateur. Il doit correspondre au nom
		du compte de l'utilisateur du client Windows sur lequel le client mtG CAPRI est installé.

\config{MTGCAPRI\_USER\_x\_SERVICE}{MTGCAPRI\_USER\_x\_SERVICE}{MTGCAPRIUSERxSERVICE}

        Paramètre par défaut~: \var{MTGCAPRI\_USER\_x\_SERVICE}='all'

        Dans cette variable vous indiquez les services, que l'utilisateur peut utiliser. 
		Les valeurs possibles sont~: \var{all, fax23, fax4, data64, telefon}. \\
        Vous pouvez spécifier plusieurs services, il faut les séparés par un espace. \\
        Exemple~: \var{MTGCAPRI\_USER\_x\_SERVICE}='telefon fax23'

\config{MTGCAPRI\_USER\_x\_OWN\_NUMBERS}{MTGCAPRI\_USER\_x\_OWN\_NUMBERS}{MTGCAPRIUSERxOWNNUMBERS}

        Paramètre par défaut~: \var{MTGCAPRI\_USER\_x\_OWN\_NUMBERS}='all'

        Dans cette variable vous indiquez les numéros de téléphones, ainsi l'utilisateur sera
		autorisé à se connecté de manière passive.
		Les valeurs possibles sont~:
        \begin{itemize}
            \item 'all' = tous les numéros sont autorisés
            \item 'none' = tous les numéros sont bloqués
            \item 'partial' = tous les numéros mentionnés dans
                \jump{MTGCAPRIUSERxOWNNUMBERSLIST}{\var{MTGCAPRI\_USER\_x\_OWN\_NUMBERS\_LIST}}
                sont autorisés.
        \end{itemize}

\config{MTGCAPRI\_USER\_x\_OWN\_NUMBERS\_LIST}{MTGCAPRI\_USER\_x\_OWN\_NUMBERS\_LIST}{MTGCAPRIUSERxOWNNUMBERSLIST}

        Paramètre par défaut~: \var{MTGCAPRI\_USER\_x\_OWN\_NUMBERS\_LIST}=''

        Dans cette variable vous indiquez les numéros autorisés, si vous avez paramétré
		\var{MTGCAPRI\_USER\_x\_OWN\_NUMBERS}='partial', les numéros indiqués pourront être
		utilisés par l'utilisateur.
		Si vous indiquez plusieurs numéros, ils doivent être séparés par un espace. \\
		Exemple~: \var{MTGCAPRI\_USER\_x\_OWN\_NUMBERS\_LIST}='12345 12346'

\config{MTGCAPRI\_USER\_x\_INCOMING\_NUMBERS}{MTGCAPRI\_USER\_x\_INCOMING\_NUMBERS}{MTGCAPRIUSERxINCOMINGNUMBERS}

        Paramètre par défaut~: \var{MTGCAPRI\_USER\_x\_INCOMING\_NUMBERS}='all'

        Dans cette variable vous indiquez les autorisations de connexion téléphoniques
		externe passant par le serveur.
		Les valeurs possibles sont~:
        \begin{itemize}
            \item 'all' = tous les numéros sont autorisés
            \item 'none' = tous les numéros sont bloqués
            \item 'partial' = tous les numéros mentionnés dans
                \jump{MTGCAPRIUSERxINCOMINGNUMBERSLIST}{\var{MTGCAPRI\_USER\_x\_INCOMING\_NUMBERS\_LIST}}
                sont autorisés.
        \end{itemize}

\config{MTGCAPRI\_USER\_x\_INCOMING\_NUMBERS\_LIST}{MTGCAPRI\_USER\_x\_INCOMING\_NUMBERS\_LIST}{MTGCAPRIUSERxINCOMINGNUMBERSLIST}

        Paramètre par défaut~: \var{MTGCAPRI\_USER\_x\_INCOMING\_NUMBERS\_LIST}=''

        Dans cette variable vous indiquez les numéros autorisés, si vous avez paramétré
		\var{MTGCAPRI\_USER\_x\_INCOMING\_NUMBERS}='partial', les numéros indiqués pourront
		être utilisés pour se connecter vers un numéro externe en passant par le serveur.
		Si vous indiquez plusieurs numéros, ils doivent être séparés par un espace. \\
		Exemple~: \var{MTGCAPRI\_USER\_x\_INCOMING\_NUMBERS\_LIST}='0172123456 0511'
		
		Ce paramètre permet de se connecter au numéro de téléphone '0172123456 'et avec
		le code de zone '0511'.

\config{MTGCAPRI\_USER\_x\_OUTGOING\_NUMBERS}{MTGCAPRI\_USER\_x\_OUTGOING\_NUMBERS}{MTGCAPRIUSERxOUTGOINGNUMBERS}

        Paramètre par défaut~: \var{MTGCAPRI\_USER\_x\_OUTGOING\_NUMBERS}='all'

        Dans cette variable vous indiquez le BLOQUAGE des connexions téléphoniques externe.
		Les valeurs possibles sont~:
        \begin{itemize}
            \item 'all' = tous les numéros sont bloqués 
            \item 'none' = tous les numéros sont autorisés
            \item 'partial' = tous les numéros mentionnés dans
                \jump{MTGCAPRIUSERxOUTGOINGNUMBERSLIST}{\var{MTGCAPRI\_USER\_x\_OUTGOING\_NUMBERS\_LIST}}
                sont bloqués
        \end{itemize}

\config{MTGCAPRI\_USER\_x\_OUTGOING\_NUMBERS\_LIST}{MTGCAPRI\_USER\_x\_OUTGOING\_NUMBERS\_LIST}{MTGCAPRIUSERxOUTGOINGNUMBERSLIST}

        Paramètre par défaut~: \var{MTGCAPRI\_USER\_x\_OUTGOING\_NUMBERS\_LIST}=''

        Dans cette variable vous indiquez les numéros bloqués, si vous avez paramétré
		\var{MTGCAPRI\_USER\_x\_OUTGOING\_NUMBERS}='partial', les numéros indiqués pourront
		être bloqués pour une connexion téléphonique externe.
		Si vous indiquez plusieurs numéros, ils doivent être séparés par un espace.
		Vous pouvez indiquer uniquement le début des numéros à composer \\
		Exemple~: \var{MTGCAPRI\_USER\_x\_OUTGOING\_NUMBERS\_LIST}='0900 0180'

		Ce paramètre bloque la connexion des numéros de téléphone commençant par '0900' ou '0180'.

\config{MTGCAPRI\_USER\_x\_TIME\_XX}{MTGCAPRI\_USER\_x\_TIME\_XX}{MTGCAPRIUSERxTIMEXX}

        Paramètre par défaut~: \var{MTGCAPRI\_USER\_x\_TIME\_XX}='0:0 0:0'

        Dans cette variable vous spécifiez les heures, auxquelles l'utilisateur est autorisé
		à utiliser CAPI du (lundi-dimanche). \\ 
		Ici quelques exemples~:
        \begin{verbatim}
            '0:0 0:0'     - Aucune restriction de l'heure
            '9:0 17:30'   - Utilisation uniquement de 09:00 à 17:30
            '24:00 24:00' - Pas d'utilisation
        \end{verbatim}


\end{description}

\subsection{Installation du logiciel client mtG CAPRI}
    
	L'installation du client est différente selon le système utilisé, si vous avez un Windows95
	ou un WindowsNT de base, parce que les fichiers nécessaire à l'installation sont générés par
	l'architecture du système qui est différent.

    Lors de la configuration du client mtG CAPRI, il faut faire attention au point suivant~: \\
	S'il existe déjà une application CAPI installé sur l'ordinateur (en d'autres termes, une carte
	ISDN local), les fichiers dll sont encores installés sur le disque dur, ceux-ci doivent être
	retirés avant l'installation du mtG CAPRI: CAPI20.DLL et CAPI2032.DLL, tous les deux fichiers sont
	normalement situées dans le répertoire système. \\

    Si tel est le cas, vous pouvez utiliser l'une des deux manières suivantes~: \\
	\\
    a) Désinstallez la carte ISDN et vérifier si les fichiers \var{capi20.dll} et \var{capi2032.dll} ont été enlevés. \\
    b) Renommer les fichiers DLL nécessaires pour activer plus tard la carte ISDN locale et donc
	la désactivation du client mtG CAPRI. Lorsque vous renommez les fichiers vous désactivez les pilotes de la carte intégré. \\

    Dans un cas normal, si ces fichiers ne sont pas écrasés par une installation sage du programme
	(vous devez vérifier le numéro de version et les données du fabricant). L'installation sage du
	programme n'a pas effectué correctement l'installation.

\subsubsection{Réglage de capri.ini pour l'initialisation du client mtG CAPRI}

    Le fichier capri.ini est utilisé pour initialiser le client mtG CAPRI et aussi pour être identifié
	par le serveur pour que la connexion soit mise en place, vous devez entrer les paramètres suivants~:
\begin{verbatim}
[CAPRI]
SERVERNAME = Remote:Thor
PORTNUMBER = 20000
TRACELEVEL = 0
TRACEFILE = c:\tmp\capri.trc
FLOWCTRL = 7
\end{verbatim}

\begin{description}
\config{[CAPRI]}{CAPRI}{CAPRI}

    L'en-tête du fichier ne doit pas être modifié.

\config{SERVERNAME}{SERVERNAME}{SERVERNAME}

    Après 'Remote:' le nom de l'alias du serveur pour le réseau TCP/IP doit être placée 
	 (il correspond au paramètre du fichier hôte et du fichier capri.cfg du mtG CAPRI sur le serveur).

\config{PORTNUMBER}{PORTNUMBER}{PORTNUMBER}

    Cette valeur doit correspondre à celle du fichier capri.cfg sur le serveur.

\config{TRACELEVEL}{TRACELEVEL}{TRACELEVEL}

    Normalement la valeur du niveau de trace pour le client mtG CAPRI est '0'. \\
	Les valeurs possibles sont~:
    \begin{itemize}
        \item 0 = aucun enregistrement
        \item 1 = Erreur
        \item 2 = comme 1 + CAPI
        \item 3 = comme 1 + INF
        \item 4 = Erreur + CAPI + INF + EntryExit
    \end{itemize}

    \wichtig{Si vous avez une erreur dans votre programme, évitez d'utiliser le niveau
		de trace élevé, car le fichier grossit relativement vite et la performance
		du mtG CAPRI diminuera. Des problèmes peuvent apparaître et cela peut interférer avec
		certaines applications (par exemple pour les fax).}

\config{TRACEFILE}{TRACEFILE}{TRACEFILE}

    Le nom et le chemin du fichier de trace (pour l'enregistrement) il est créé par mtG CAPRI.
    Le paramètre peut être modifié (ici~: capri.trc).

\config{FLOWCTRL}{FLOWCTRL}{FLOWCTRL}

    Lors de l'envoi des paquets de données, un contrôle de flux est effectué. \\
    Les valeurs possibles sont~: \\
    \begin{itemize}
        \item 0 = contrôle pas effectué
        \item 1 = attendre après chaque acquittement de paquet de données
        \item 2 = maximum 2 paquets de données sans contrôle
        \item 3 = maximum 3 paquets de données sans contrôle
        \item ... d'autres valeurs analogiques
        \item 7 = maximum 7 paquets de données sans contrôle (par défaut)
    \end{itemize}

    Les valeurs supérieures à 7 sont possible, mais pas recommandé. La spécification de CAPI fournit
	une valeur de 7.

\end{description}

\subsubsection{Test du client}

	Vous pouvez démarrer par un double clic le logiciel caprit32.exe (pour un environnement 32 bits)
	et caprit16.exe (pour un environnement 16 bits). Normalement, le message 'Test mtG CAPRI a réussi'
	doit apparaître. Cependant, si le message 'Test mtG CAPRI échoué' apparaît, les points suivants doivent
	être vérifiés~:
    \begin{itemize}
        \item Toutes les étapes de l'installation du client et du serveur sont-il correctement effectuées~?
        \item Le nom du serveur mtG CAPRI est-il écrit correctement dans le fichier capri.ini du client~?
        \item Dans le fichier C:\\Windows\\Hosts (pour Windows95) ou ...\\System32\\Drivers\\Etc\\Hosts
            (pour Windows NT® et plus) le nom de l'alias de l'ordinateur serveur est-il enregistré~?
        \item La connexion réseau du serveur est-il établie~?
        \item L'utilisateur actuellement connecté a t'il les privilèges suffisants dans le fichier
			d'authentification du serveur mtG CAPRI~?
    \end{itemize}

    Si le test a été effectué avec succès, une application CAPI (par exemple T-Online) sera démarré
	à partir du client.

\subsubsection{Les messages d'erreur des applications CAPI}

    Les messages d'erreur des applications CAPI (tels que T-Online, FritzFax etc) sont affichés dans
	le cas où une carte ISDN avec le logiciel correspondant est installé localement sur l'ordinateur concerné.
	Ils sont par conséquent souvent trompeur pour mtG CAPRI. \\
    Exemple~: \\

    Le message 'pilote CAPI 2.0 n'est pas installé sur cet ordinateur' signifie que CAPI (actuellement,
	dans l'interaction entre le client et le serveur) n'est pas fonctionnelle. Le problème est en rapport
	avec mtG CAPRI, par exemple, tout est en ordre sur le client mais, la connexion réseau au serveur est
	interrompu ou le serveur est arrêté. \\
    \\
    T-Online génère un message d'erreur lorsque l'utilisateur n'a pas été authentifié sur le serveur~:
	'Vous ne pouvez pas initialiser DDE (Wsock32)'. \\
    \\
    Si le nom du serveur est spécifié de manière incorrecte sur le client, je reçois le message suivant
	'Résultat de la configuration de la connexion: Le pilotes CAPI est manquent ou ...'. \\
    \\
    Avec de tels messages "cryptés" il est donc recommandé~:
    \begin{itemize}
        \item de démarrer sur le client le logiciel de test caprit32.exe ou caprit16.exe, pour voir si la connexion
		au serveur mtG CAPRI peut être établie ou
        \item de contrôler le fichier trace sur le serveur, il donne une indication sur le problème tels que
		'utilisateur non autorisé', 'pas de canal libre', 'chaîne demandée n'est pas disponible', etc.
    \end{itemize}

