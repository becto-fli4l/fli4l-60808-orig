% Last Update: $Id$
\marklabel{sec:dnsdhcp}
{
\section{DNS\_DHCP - Dienste rund um DNS und DHCP}
}
\subsection{Hostnamen}
\subsubsection{Hosts}
\begin{description}
    \config{OPT\_HOSTS}{OPT\_HOSTS}{OPTHOSTS}
    
    Mit der optionalen Variable \var{OPT\_HOSTS} kann die Konfiguration von
    Hostname deaktiviert werden!
    
    \configlabel{HOST\_x\_NAME}{HOSTxNAME}
    \configlabel{HOST\_x\_IP4}{HOSTxIP4}
    \configlabel{HOST\_x\_IP6}{HOSTxIP6}
    \configlabel{HOST\_x\_DOMAIN}{HOSTxDOMAIN}
    \configlabel{HOST\_x\_ALIAS\_N}{HOSTxALIASN}
    \configlabel{HOST\_x\_ALIAS\_x}{HOSTxALIASx}
    \configlabel{HOST\_x\_MAC}{HOSTxMAC}
    \configlabel{HOST\_x\_MAC2}{HOSTxMAC2}
    \configlabel{HOST\_x\_DHCPTYP}{HOSTxDHCPTYP}
    \config{HOST\_N  HOST\_x\_\{attribute\}}{HOST\_N}{HOSTN}

    {Es sollten alle Rechner im LAN beschrieben werden - mit
      IP-Adresse, Namen, Aliasnamen und evtl. Mac-Adressen für die
      dhcp-Konfiguration . Dazu setzt man zunächst die Anzahl der
      Rechner mit der Variablen \var{HOST\_\-N}.

      \textbf{Hinweis: } Seit Version 3.4.0 wird der Eintrag für den
      Router aus den Angaben in der \var{$<$config$>$/base.txt} generiert.
      Sollen zusätzliche Aliasnamen aufgenommen werden, siehe auch
       \jump{HOSTNAMEALIASN}{\var{HOSTNAME\_ALIAS\_N}}.

      Anschließend werden mit den Attributen die Eigenschaften des
      Hostes definiert. Dabei sind einige Attribute Pflicht, wie z.B.
      IP-Adresse und Name, die anderen optional, d.h. man kann, aber
      man muß sie nicht spezifizieren.

      \begin{description}
      \item[NAME]         -- Name des n-ten Hostes
      \item[IP4]          -- IP-Adresse (ipv4) des n-ten Hostes
      \item[IP6]          -- IP-Adresse (ipv6) des n-ten Hostes (optional). Wenn
                             man ``auto'' verwendet, dann wird die Adresse
                             bei aktiviertem OPT\_IPV6 aus einem IPv6-Präfix
                             (mit /64er-Netzmaske) und der MAC-Adresse des
                             jeweiligen Hosts automatisch berechnet. Damit das
                             funktioniert, muss die MAC-Adresse via
                             \var{HOST\_x\_MAC} gesetzt (siehe unten) und das
                             Paket \texttt{ipv6} entsprechend konfiguriert
                             werden.
      \item[DOMAIN]       -- DNS-Domain des n-ten Hostes (optional)
      \item[ALIAS\_N]     -- Anzahl der Alias-Namen des n-ten Hostes
      \item[ALIAS\_m]     -- m-ter Alias-Name für den n-ten Host
      \item[MAC]          -- Mac Adresse des n-ten Hostes
      \item[DHCPTYP]      -- Vergabe der IP-Adresse per DHCP abhängig von
                             MAC oder NAME (optional)
      \end{description}



      In der Beispiel-Datei sind 4 Rechner konfiguriert - nämlich die
          PCs ``client1'', ``client2'', ``client3'' und ``client4''.

\begin{example}
\begin{verbatim}
         HOST_1_NAME='client1'                # 1st host: ip and name
         HOST_1_IP4='192.168.6.1'
\end{verbatim}
\end{example}

      Aliasnamen müssen mit kompletter Domain angegeben werden.

      Die MAC-Adresse ist optional und ist nur dann relevant, wenn
      fli4l zusätzlich als DHCP-Server eingesetzt wird. Dies wird in
      der Beschreibung zum optionalen Programmpaket ``\var{OPT\_\-DHCP}''
      erklärt, siehe unten. Ohne Einsatz als DHCP-Server sind
      lediglich die IP-Adresse, der Name des Rechners und eventuell
      Aliasnamen einzusetzen.  Die MAC-Adresse ist eine 48-Bit-Adresse
      und besteht aus 6 Hex-Werten, welche durch einen Doppelpunkt
      voneinander getrennt werden, z.B.

\begin{example}
\begin{verbatim}
        HOST_2_MAC='de:ad:af:fe:07:19'
\end{verbatim}
\end{example}

      \emph{Hinweis:} Wird fli4l um das IPv6-Paket ergänzt, brauchen keine
      IPv6-Adressen hinterlegt zu werden, wenn gleichzeitig die MAC-Adressen
      der Hosts vorliegen, weil das IPv6-Paket dann die IPv6-Adressen
      automatisch berechnet (modifiziertes EUI-64). Natürlich kann man aber
      den Automatismus unterbinden und feste IPv6-Adressen vorgeben, wenn man
      dies wünscht.
      }


\end{description}

\subsubsection{Extra Hosts}
  \begin{description}
  \configlabel{HOST\_EXTRA\_x\_NAME}{HOSTEXTRAxNAME}
  \configlabel{HOST\_EXTRA\_x\_IP4}{HOSTEXTRAxIP4}
  \configlabel{HOST\_EXTRA\_x\_IP6}{HOSTEXTRAxIP6}
  \config{HOST\_EXTRA\_N HOST\_EXTRA\_x\_NAME HOST\_EXTRA\_x\_IP4 HOST\_EXTRA\_x\_IP6}{HOST\_EXTRA\_N}{HOSTEXTRAN}
  {
      Mit diesen Variablen können weitere Hosts hinzugefügt werden die nicht der
      lokalen Domain angehören wie z.b. Hosts die sich auf der anderen Seite eines
      VPNs befinden.
  
  }  
\end{description}


\subsection{DNS-Server}
  \begin{description}
    \config{OPT\_DNS}{OPT\_DNS}{OPTDNS}

    {Um den DNS-Server zu aktivieren ist die Variable \var{OPT\_DNS}
    mit `yes` zu belegen.

    Werden im LAN keine Windows-Rechner verwendet oder ist bereits
      ein DNS-Server vorhanden, kann man \var{OPT\_DNS} auf `no' setzen
      und den Rest in diesem Abschnitt übergehen.

      Im Zweifel immer (Standard-Einstellung): \var{OPT\_DNS}='yes'}

\end{description}
\subsubsection{Allgemeine DNS-Optionen}

\begin{description}
    \configlabel{DNS\_LISTEN\_x}{DNSLISTENx}
    \config{DNS\_LISTEN\_N DNS\_LISTEN\_x}{DNS\_LISTEN\_N}{DNSLISTENN}

    { Wenn Sie \var{OPT\_DNS}='yes' gewählt haben, können Sie mit Hilfe von
      \var{DNS\_LISTEN\_N} die Anzahl, und mit \var{DNS\_LISTEN\_1} bis
      \var{DNS\_LISTEN\_N} lokale IPs angeben, auf denen \verb+dnsmasq+ DNS-Anfragen
      annehmen darf. Sollten Sie bei \var{DNS\_LISTEN\_N} eine 0
      eingetragen haben, beantwortet \verb+dnsmasq+ DNS-Anfragen auf allen lokalen
      IPs.
      An dieser Stelle dürfen nur IPs von existierenden Schnittstellen
      (ethernet, wlan ...) verwendet werden, es kommt sonst zu Warnmeldungen
      beim Start des Routers. Alternativ ist nun möglich hier auch ALIAS-Namen
      zu verwenden, z.\,B. \verb+IP_NET_1_IPADDR+

      Für alle hier angegebenen Adressen werden bei
      \var{PF\_INPUT\_ACCEPT\_DEF='yes'} und/oder
      \var{PF6\_INPUT\_ACCEPT\_DEF='yes'} entsprechende ACCEPT-Regeln in
      der INPUT-Kette der Firewall erzeugt. Im Falle \var{DNS\_LISTEN='0'}
      werden ebenfalls Regeln erzeugt, die den DNS-Zugriff auf \emph{allen}
      konfigurierten Schnittstellen erlauben.

      \wichtig{Falls der DNS-Server auf zur Laufzeit dynamisch hinzugefügten
      Schnittstellen horchen soll, etwa auf Netzwerk-Schnittstellen von
      VPN-Tunneln, sollten Sie dieses Array leer lassen, da andernfalls der
      DNS-Server nicht auf DNS-Anfragen antworten wird, die über den VPN-Tunnel
      gestellt werden.}

      Im Zweifelsfalle können die Standardeinstellungen übernommen werden.}

    \config{DNS\_BIND\_INTERFACES}{DNS\_BIND\_INTERFACES}{DNSBINDINTERFACES}{
      Falls Sie den DNS-Server via \var{DNS\_LISTEN\_x} nur an bestimmte
      Adressen binden möchten \emph{und} zusätzlich einen \emph{weiteren}
      DNS-Server an \emph{andere} Adressen binden möchten, können Sie durch
      diese Option den DNS-Server anweisen, sich auch wirklich \emph{nur} an
      die gelisteten Adressen zu binden. Standardmäßig bindet sich der
      DNS-Server an \emph{alle} Schnittstellen und wirft bei Adressen im
      \var{DNS\_LISTEN\_x}-Array DNS-Anfragen, die an nicht konfigurierten
      Adressen ankommen, weg. Dies hat den Vorteil, dass der DNS-Server auch
      mit zur Laufzeit dynamisch hinzugefügten Schnittstellen umgehen kann,
      aber den Nachteil, dass kein alternativer DNS-Server auf dem
      Standard-DNS-Port 53 gleichzeitig laufen kann. Ein Anwendungsfall für
      einen zweiten DNS-Server ist, wenn Sie einen Slave-DNS-Server wie
      ``yadifa'' direkt auf dem fli4l-Router betreiben möchten. Soll also der
      dnsmasq nicht exklusiv auf dem fli4l eingesetzt werden, muss die
      Einstellung `yes' gewählt und die für den dnsmasq zu nutzenden
      IP-Adressen per \var{DNS\_LISTEN} konfiguriert werden.
    }

    \config{DNS\_VERBOSE}{DNS\_VERBOSE}{DNSVERBOSE}

    { Logging von DNS-Queries: `yes' oder `no'

      Für ausführlichere Ausgaben des DNS, muß \var{DNS\_VERBOSE} auf yes
      gesetzt werden.  In diesem Fall werden DNS-Anfragen
      an den Nameserver protokolliert - und zwar über die
      syslog-Schnittstelle. Damit die Ausgaben auch sichtbar werden,
      ist dann auch die Variable \jump{OPTSYSLOGD}{\var{OPT\_SYSLOGD='yes'}}
          zu setzen, s.u.}


    \config{DNS\_MX\_SERVER}{DNS\_MX\_SERVER}{DNSMXSERVER}

      {Mit dieser Variable gibt man hier den Hostnamen für den MX-Record
      (Mail-Exchanger) für die in \var{DOMAIN\_NAME} definierte Domain an.
      Ein MTA (Mail"=Transport"=Agent, wie z.B. sendmail)
      auf einem internen Server fragt per DNS nach einem Mail-Exchanger
      für die Zieldomain der zuzustellenden Mail. Der DNS-Server liefert
      hiermit dem MTA den entsprechenden Host, der für Mails der Domain
      \var{DOMAIN\_NAME} zuständig ist.

      \achtung{Dies ist keine automatische Konfiguration für Mail-Clients,
      wie z.B. Outlook! Also bitte nicht gmx.de hier eintragen und dann
      wundern, warum Outlook nicht funktioniert.}
      }


    \configlabel{DNS\_FORBIDDEN\_x}{DNSFORBIDDENx}
    \config{DNS\_FORBIDDEN\_N DNS\_FORBIDDEN\_x}{DNS\_FORBIDDEN\_N}{DNSFORBIDDENN}

    {Hier können Sie Domains angeben, bei denen DNS-Queries vom
      DNS-Server prinzipiell als ``nicht vorhanden'' beantwortet
      werden sollen.

      Beispiel:

\begin{example}
\begin{verbatim}
        DNS_FORBIDDEN_N='1'
        DNS_FORBIDDEN_1='foo.bar'
\end{verbatim}
\end{example}

      In diesem Fall wird zum Beispiel eine Anfrage nach www.foo.bar
      mit einem Fehler beantwortet.

      Man kann damit auch ganze Top-Level-Domains verbieten:

\begin{example}
\begin{verbatim}
        DNS_FORBIDDEN_1='de'
\end{verbatim}
\end{example}

      Dann ist die Namensauflösung für sämtliche Rechner in der
      DE-Topleveldomain abgeschaltet.}


    \configlabel{DNS\_REDIRECT\_x}{DNSREDIRECTx}
    \configlabel{DNS\_REDIRECT\_x\_IP}{DNSREDIRECTxIP}
    \config{DNS\_REDIRECT\_N DNS\_REDIRECT\_x DNS\_REDIRECT\_x\_IP}{DNS\_REDIRECT\_N}{DNSREDIRECTN}

    {Hier können Domains angegeben werden, bei welchen DNS-Queries vom
      DNS-Server auf eine spezielle IP umgeleitet werden.

      Beispiel:

\begin{example}
\begin{verbatim}
        DNS_REDIRECT_N='1'
        DNS_REDIRECT_1='yourdom.dyndns.org'
        DNS_REDIRECT_1_IP='192.168.6.200'
\end{verbatim}
\end{example}

      In diesem Fall wird zum Beispiel eine Anfrage nach yourdom.dyndns.org
      mit der IP 192.168.6.200 beantwortet. Somit kann man externe Domains
      auf andere IPs umleiten.}


     \config{DNS\_BOGUS\_PRIV}{DNS\_BOGUS\_PRIV}{DNSBOGUSPRIV}

     {Setzt man diese Variable auf `yes`, werden reverse-lookups für IP-Adressen
      nach RFC1918 (Private Address Bereiche) nicht vom dnsmasq an andere DNS-Server
      weitergeleitet, sondern vom dnsmasq beantwortet.}

     \config{DNS\_FORWARD\_PRIV\_x}{DNS\_FORWARD\_PRIV\_x}{DNSFORWARDPRIVx}

     {Gelegentlich möchte man trotz aktiviertem \var{DNS\_BOGUS\_PRIV} die
     Auflösung von Adressen einiger privater Subnetze dennoch an die
     konfigurierten DNS-Server delegieren. Dies ist zum Beispiel nötig, wenn
     ein Uplink-Router private Subnetze verwaltet. Diese Array-Variable kann
     dafür genutzt werden, die privaten Subnetze zu benennen, deren Auflösung
     delegiert werden darf.}

     \config{DNS\_FILTERWIN2K}{DNS\_FILTERWIN2K}{DNSFILTERWIN2K}

     {Setzt man diese Variable auf 'yes', werden DNS-Anfragen vom Typ SOA, SRV und ANY 
     geblockt. Dienste, die diese Anfragen verwenden, werden dann nicht mehr ohne weitere
     Konfiguration funktionieren.\hfil\break
     Dazu zählen zum Beispiel:
     \begin{itemize}
     \item XMPP (Jabber)
     \item SIP
     \item LDAP
     \item Kerberos
     \item Teamspeak3 (seit Client-Version 3.0.8)
     \item Minecraft (seit Vollversion 1.3.1)
     \item Ermittlung des Domänencontrollers (Win2k)
     \end{itemize}
     Siehe hierzu auch:}
     \begin{itemize}
     \item Generelle Erklärung der DNS Record Arten:\hfil\break
           \altlink{http://en.wikipedia.org/wiki/List_of_DNS_record_types}
     \item Manpage von dnsmasq:\hfil\break
           \altlink{http://www.thekelleys.org.uk/dnsmasq/docs/dnsmasq-man.html}
     \item SRV-Record im Speziellen:\hfil\break   
           \altlink{http://de.wikipedia.org/wiki/SRV_Resource_Record} 
     \end{itemize}
     \achtung{Durch Setzen von 'no' können durch die zusätzlichen weitergeleiteten
     DNS-Anfragen ungewollte Einwahlverbindungen aufgebaut oder bestehende nicht 
     abgebaut werden. Insbesondere bei ISDN- und UMTS-Verbindungen können dadurch 
     Mehrkosten entstehen. Sie müssen selbst abwägen, was für Sie wichtiger ist.}

     \config{DNS\_FORWARD\_LOCAL}{DNS\_FORWARD\_LOCAL}{DNSFORDWARDLOCAL}

     {setzt man diese Variable auf 'yes' kann der fli4l-Router in
     einer Domäne mit DOMAIN\_NAME='example.local' konfiguriert
     werden, die wiederrum per DNS\_ZONE\_DELEGATION\_x\_DOMAIN='example.local'
     von einem anderen Nameserver aufgelöst wird.}

     \config{DNS\_LOCAL\_HOST\_CACHE\_TTL}{DNS\_LOCAL\_HOST\_CACHE\_TTL}{DNSLOCALHOSTCACHETTL}

     {Gibt die TTL (Time to live, in Sekunden) für Einträge aus den
     /etc/hosts Dateien und den per DHCP vergebenen IP-Adressen
     an. Der Standardwert für den fli4l-Router beträgt 60
     Sekunden. Standardmäßig setzt der dnsmasq die TTL für lokale
     Einträge auf 0 und deaktiviert damit faktisch das nachfolgende
     Caching der DNS Einträge. Die Idee dahinter ist das ablaufende
     DHCP Leases usw. zeitnah weitergegeben werden können. Fragt
     allerdings z.B. ein lokaler IMAP Proxy die DNS Einträge dadurch
     mehrfach pro Sekunde ab ist das eine deutliche Belastung für das
     Netzwerk. Ein Kompromiss ist daher ein relativ kurzer TTL von 60
     Sekunden. Es kann ja auch ohne die kurze TTL von 60 Sekunden
     jederzeit zu einem simplen abschalten eines Hosts kommen, so dass
     die abfragende Software sowieso mit nicht antwortetenden Hosts
     klarkommen muss.}

     \config{DNS\_SUPPORT\_IPV6}{DNS\_SUPPORT\_IPV6}{DNSSUPPORTIPV6} (optional)
     
     {setzt man diese optionale Variable auf 'yes' wird die Unterstützung für IPV6-
     Adressen des DNS-Servers aktiviert.}

  \end{description}

\subsubsection{DNS-Zonenkonfiguration}

Der dnsmasq kann auch eine DNS-Domäne eigenständig verwalten, d.h. er ist
``authoritativ'' für diese Domäne. Dazu muss man zweierlei tun: Zum einen muss
angegeben werden, welcher externe (!) DNS-Namensdienst auf den eigenen fli4l
verweist und über welche Netzwerk-Schnittstelle dies passiert. Die Angabe der
externen Referenz ist erforderlich, denn die Domäne, welche der fli4l verwaltet,
ist ja immer eine Unterdomäne einer anderen Domäne.\footnote{Wir gehen hier mal
davon aus, dass niemand einen fli4l als DNS-Rootserver verwendet...} Die Angabe
der Schnittstelle ist wichtig, weil sich der dnsmasq dort ``nach außen'' anders
verhält als auf den anderen Schnittstellen ``nach innen'': Nach außen
beantwortet der dnsmasq niemals Anfragen für Namen außerhalb der konfigurierten
eigenen Domäne. Nach innen funktioniert der dnsmasq natürlich auch als DNS-Relay
ins Internet, damit die Auflösung von nicht-lokalen Namen funktioniert.

Zum anderen muss konfiguriert werden, welche Netze nach außen via
Namensauflösung erreichbar sind. Hierbei sollten natürlich nur Netze mit
öffentlichen IP-Adressen angegeben werden, denn über private Adressen können
Hosts von außen ohnehin nicht erreicht werden.

Im Folgenden wird die Konfiguration an einem Beispiel beschrieben. Dieses
Beispiel setzt das IPv6-Paket sowie ein öffentlich geroutetes IPv6-Präfix
voraus; letzteres kann z.B. von einem 6in4-Tunnel-Provider wie Hurricane
Electric bereitgestellt werden.

\begin{description}

\config{DNS\_AUTHORITATIVE}{DNS\_AUTHORITATIVE}{DNSAUTHORITATIVE}

Die Einstellung \verb+DNS_AUTHORITATIVE='yes'+ aktiviert den authoritativen
Modus des dnsmasq. Dies reicht jedoch nicht aus, da weitere Angaben gemacht
werden müssen (s.u.).

Standard-Einstellung: \verb+DNS_AUTHORITATIVE='no'+

Beispiel: \verb+DNS_AUTHORITATIVE='yes'+

\config{DNS\_AUTHORITATIVE\_NS}{DNS\_AUTHORITATIVE\_NS}{DNSAUTHORITATIVENS}

Mit dieser Variable wird der DNS-Name konfiguriert, über den auf den fli4l von
außen mit Hilfe eines DNS-NS-Records verwiesen wird. Das kann auch ein DNS-Name
sein, der zu einem Dynamic DNS-Dienst gehört.

Beispiel: \verb+DNS_AUTHORITATIVE_NS='fli4l.noip.me'+

\config{DNS\_AUTHORITATIVE\_LISTEN}{DNS\_AUTHORITATIVE\_LISTEN}{DNSAUTHORITATIVELISTEN}

Mit dieser Variable wird konfiguriert, an welcher Adresse bzw. Schnittstelle
der dnsmasq DNS-Anfragen für die eigene Domäne authoritativ beantwortet.
Symbolische Namen wie \verb+IP_NET_2_IPADDR+, \verb+IP_NET_1_DEV+ oder
\verb+{LAN}+ sind erlaubt. Der dnsmasq kann nur an \emph{einer}
Adresse/Schnittstelle authoritativ antworten.

\wichtig{Zu beachten ist, dass dies niemals eine Adresse/Schnittstelle sein
darf, an der das eigene LAN hängt, weil sonst keine nicht-lokalen Namen mehr im
LAN aufgelöst werden können!}

Beispiel: \verb+DNS_AUTHORITATIVE_LISTEN='IP_NET_2_IPADDR'+

\configlabel{DNS\_ZONE\_NETWORK\_x}{DNSZONENETWORKx}
\config{DNS\_ZONE\_NETWORK\_N DNS\_ZONE\_NETWORK\_x}{DNS\_ZONE\_NETWORK\_N}{DNSZONENETWORKN}

Hier werden die Netzadressen angegeben, für die der dnsmasq authoritativ die
Namen auflösen soll. Dabei funktioniert sowohl die Vorwärts- (Name zu Adresse)
als auch die Rückwärtsauflösung (Adresse zu Name).

Ein komplettes Beispiel:

\begin{example}
\begin{verbatim}
        DNS_AUTHORITATIVE='yes'
        DNS_AUTHORITATIVE_NS='fli4l.noip.me'
        DNS_AUTHORITATIVE_LISTEN='IP_NET_2_IPADDR' # Uplink hängt an eth1
        DNS_ZONE_NETWORK_N='1'
        DNS_ZONE_NETWORK_1='2001:db8:11:22::/64'   # lokales IPv6-LAN
\end{verbatim}
\end{example}

Dabei wird angenommen, dass ``2001:db8:11::/48'' ein zu dem fli4l öffentlich
geroutetes IPv6-Präfix ist und dass für das LAN das Subnetz 22 gewählt wurde.

\end{description}

\subsubsection{DNS Zone Delegation}

  \begin{description}

    \configlabel{DNS\_ZONE\_DELEGATION\_x}{DNSZONEDELEGATIONx}
    \configlabel{DNS\_ZONE\_DELEGATION\_x\_UPSTREAM\_SERVER\_x}{DNSZONEDELEGATIONUPSTREAMSERVERx}
    \configlabel{DNS\_ZONE\_DELEGATION\_x\_UPSTREAM\_SERVER\_x\_IP}{DNSZONEDELEGATIONUPSTREAMSERVERxIP}
    \configlabel{DNS\_ZONE\_DELEGATION\_x\_UPSTREAM\_SERVER\_x\_quERYSOURCEIP}{DNSZONEDELEGATIONUPSTREAMSERVERxQUERYSOURCEIP}
    \configlabel{DNS\_ZONE\_DELEGATION\_x\_DOMAIN}{DNSZONEDELEGATIONxDOMAIN}
    \configlabel{DNS\_ZONE\_DELEGATION\_x\_NETWORK}{DNSZONEDELEGATIONxNETWORK}
    \config{DNS\_ZONE\_DELEGATION\_N DNS\_ZONE\_DELEGATION\_x}{DNS\_ZONE\_DELEGATION\_N}{DNSZONEDELEGATIONN}

    { Es gibt besondere Situationen, wo die Angabe eines oder mehrerer
      DNS Server sinnvoll ist, z.B. bei Einsatz von fli4l im Intranet
      ohne Internetanschluss oder einem Mix von diesen (Intranet mit
      eigenem DNS Server und zusätzlich Internetanschluss).

      Stellen wir uns folgendes Szenario vor:

      \begin{itemize}
      \item Circuit 1: Einwahl in das Internet
      \item Circuit 2: Einwahl in das Firmen-Netz 192.168.1.0 (firma.de)
      \end{itemize}


      Dann wird man \var{ISDN\_\-CIRC\_\-1\_\-ROUTE} auf `0.0.0.0'
      und \var{ISDN\_\-CIRC\_\-2\_\-ROUTE} auf `192.168.1.0'
      setzen. Bei Zugriff auf Rechner mit IP-Adresse 192.168.1.x wird
      fli4l dann den Circuit 2, sonst den Circuit 1 benutzen.  Wenn
      das Firmennetz aber nicht öffentlich ist, wird in diesem
      vermutlich ein eigener DNS Server betrieben. Nehmen wir an, die
      Adresse dieses DNS Servers wäre 192.168.1.12 und der Domainname
      wäre ``firma.de''.

      In diesem Fall gibt man an:

\begin{example}
\begin{verbatim}
        DNS_ZONE_DELEGATION_N='1'
        DNS_ZONE_DELEGATION_1_UPSTREAM_SERVER_N='1'
        DNS_ZONE_DELEGATION_1_UPSTREAM_SERVER_1_IP='192.168.1.12'
        DNS_ZONE_DELEGATION_1_DOMAIN_N='1'
        DNS_ZONE_DELEGATION_1_DOMAIN_1='firma.de'
\end{verbatim}
\end{example}

      Dann werden bei DNS Anfragen an die Domain firma.de der
      firmeninterne DNS Server benutzt. Alle anderen DNS Anfragen
      gehen wie üblich an die DNS Server im Internet.

      Ein anderer Fall:
      \begin{itemize}
      \item Circuit 1: Internet
      \item Circuit 2: Firmen-Netz 192.168.1.0 *mit* Internetanschluss
      \end{itemize}

      Hier hat man also die Möglichkeit, auf 2 Wegen in das Internet
      zu gelangen. Möchte man geschäftliches und privates trennen,
      bietet sich dann folgendes an:

\begin{example}
\begin{verbatim}
        ISDN_CIRC_1_ROUTE='0.0.0.0'
        ISDN_CIRC_2_ROUTE='0.0.0.0'
\end{verbatim}
\end{example}

      Man legt also auf beide Circuits eine Defaultroute und schaltet
      dann die Route mit dem imond-Client um - je nach Wunsch. Auch in
      diesem Fall sollte man \var{DNS\_ZONE\_DELEGATION\_\-N}
      und \var{DNS\_ZONE\_DELEGATION\_x\_DOMAIN\_x} wie oben
      beschrieben einstellen.}

      Möchte man auch die Reverse-DNS-Auflösung für ein so
      erreichbares Netz nutzen, z.B. wird ein Reverselookup von
      einigen Mailserver gemacht, gibt man in der optionalen
      Variable \var{DNS\_ZONE\_DELEGATION\_x\_NETWORK\_x}, das/die
      Netz(werke) an, für die der Reverselookup aktiviert werden soll.
      Das folgende Beispiel verdeutlicht das:

\begin{example}
\begin{verbatim}
        DNS_ZONE_DELEGATION_N='2'
        DNS_ZONE_DELEGATION_1_UPSTREAM_SERVER_N='1'
        DNS_ZONE_DELEGATION_1_UPSTREAM_SERVER_1_IP='192.168.1.12'
        DNS_ZONE_DELEGATION_1_DOMAIN_N='1'
        DNS_ZONE_DELEGATION_1_DOMAIN_1='firma.de'
        DNS_ZONE_DELEGATION_1_NETWORK_N='1'
        DNS_ZONE_DELEGATION_1_NETWORK_1='192.168.1.0/24'
        DNS_ZONE_DELEGATION_2_UPSTREAM_SERVER_N='1'
        DNS_ZONE_DELEGATION_2_UPSTREAM_SERVER_1_IP='192.168.2.12'
        DNS_ZONE_DELEGATION_2_DOMAIN_N='1'
        DNS_ZONE_DELEGATION_2_DOMAIN_1='bspfirma.de'
        DNS_ZONE_DELEGATION_2_NETWORK_N='2'
        DNS_ZONE_DELEGATION_2_NETWORK_1='192.168.2.0/24'
        DNS_ZONE_DELEGATION_2_NETWORK_2='192.168.3.0/24'
\end{verbatim}
\end{example}

      Mit der
      Konfigurationsoption \var{DNS\_ZONE\_DELEGATION\_x\_UPTREAM\_SERVER\_x\_QUERYSOURCEIP}
      kann man die IP-Adresse für die ausgehenden DNS Anfragen an den
      oder die Upstream DNS Server setzen. Das ist z.B. dann sinnvoll
      wenn man den Upstream DNS Server über ein VPN erreicht und nicht
      möchte, dass die lokale VPN Adresse vom fli4l-Router als Quell
      IP-Adresse beim Upstream DNS Server auftaucht. Ein anderer
      Anwendungsfall ist eine vom Upstream DNS Server aus gesehen
      nicht routebare IP-Adresse (die durch ein VPN Interface
      evtl. auftritt). Auch in diesem Fall ist es notwendig die vom
      dnsmasq benutzte ausgehende IP-Adresse fest auf eine vom
      fli4l-Router benutzte und vom Upstream DNS Server aus erreichbar
      IP-Adresse zu setzen.

\begin{example}
\begin{verbatim}
        DNS_ZONE_DELEGATION_N='1'
        DNS_ZONE_DELEGATION_1_UPSTREAM_SERVER_N='1'
        DNS_ZONE_DELEGATION_1_UPSTREAM_SERVER_1_IP='192.168.1.12'
        DNS_ZONE_DELEGATION_1_UPSTREAM_SERVER_1_QUERYSOURCEIP='192.168.0.254'
        DNS_ZONE_DELEGATION_1_DOMAIN_N='1'
        DNS_ZONE_DELEGATION_1_DOMAIN_1='firma.de'
        DNS_ZONE_DELEGATION_1_NETWORK_N='1'
        DNS_ZONE_DELEGATION_1_NETWORK_1='192.168.1.0/24'
\end{verbatim}
\end{example}

    \configlabel{DNS\_REBINDOK\_x\_DOMAIN}{DNSREBINDOKxDOMAIN}
    \config{DNS\_REBINDOK\_N DNS\_REBINDOK\_x\_DOMAIN}{DNS\_REBINDOK\_N}{DNSREBINDOKN}

    Der Nameserver \emph{dnsmasq} lehnt normalerweise Antworten anderer
    Nameserver ab, die IP-Adressen aus privaten Netzwerken
    enthalten. Er verhindert dadurch eine bestimmte Klasse von
    Angriffen auf das Netzwerk. Hat man allerdings eine Domain in
    einem Netzwerk mit privaten IP-Adressen und einen extra
    Nameserver, der für dieses Netz zuständig ist, liefert der genau
    die Antworten, die vom \emph{dnsmasq} abgelehnt werden
    würden. Diese Domains kann man in \var{DNS\_REBINDOK\_x}
    auflisten, die entsprechenden Antworten auf Anfragen zu der Domain
    werden dann akzeptiert.  Ein weiteres Beispiel für Nameserver, die
    private IP-Adressen als Antwort liefern, sind sogenannte
    ``Real-Time Blacklist Server''. Ein Beispiel basierend auf diesen
    könnte wie folgt aussehen:

\begin{example}
\begin{verbatim}
        DNS_REBINDOK_N='8'
        DNS_REBINDOK_1_DOMAIN='rfc-ignorant.org'
        DNS_REBINDOK_2_DOMAIN='spamhaus.org'
        DNS_REBINDOK_3_DOMAIN='ix.dnsbl.manitu.net'
        DNS_REBINDOK_4_DOMAIN='multi.surbl.org'
        DNS_REBINDOK_5_DOMAIN='list.dnswl.org'
        DNS_REBINDOK_6_DOMAIN='bb.barracudacentral.org'
        DNS_REBINDOK_7_DOMAIN='dnsbl.sorbs.net'
        DNS_REBINDOK_8_DOMAIN='nospam.login-solutions.de'
\end{verbatim}
\end{example}

\end{description}



\subsection{DHCP-Server}

  \begin{description}

    \config{OPT\_DHCP}{OPT\_DHCP}{OPTDHCP}

    {Mit \var{OPT\_DHCP} kann man einstellen, ob der DHCP-Server aktiviert wird.}

    \config{DHCP\_TYPE}{DHCP\_TYPE}{DHCPTYPE} (optional)
    
    {Mit dieser Variable legt man fest, ob man die interne DHCP-Funktion des
    dnsmasq benutzt, oder ob man auf den externen ISC-DHCPD zurückgreifen will.
    Im Falle des ISC-DHCPD entfällt der Support für DDNS.}

    \config{DHCP\_VERBOSE}{DHCP\_VERBOSE}{DHCPVERBOSE}
    
    {aktiviert zusätzliche DHCP-Ausgaben im log.}

    \config{DHCP\_LS\_TIME\_DYN}{DHCP\_LS\_TIME\_DYN}{DHCPLSTIMEDYN}

    {legt die standard Lease-Time für dynamisch vergebene IP-Adressen fest.}

    \config{DHCP\_MAX\_LS\_TIME\_DYN}{DHCP\_MAX\_LS\_TIME\_DYN}{DHCPMAXLSTIMEDYN}

    {legt die maximale Lease-Time für dynamisch vergebene IP-Adressen fest.}

    \config{DHCP\_LS\_TIME\_FIX}{DHCP\_LS\_TIME\_FIX}{DHCPLSTIMEFIX}

    {Standard Lease-Time für statisch zugeordnete IP-Adressen.}

    \config{DHCP\_MAX\_LS\_TIME\_FIX}{DHCP\_MAX\_LS\_TIME\_FIX}{DHCPMAXLSTIMEFIX}

    {legt die maximale Lease-Time für statisch zugeordnete IP-Adressen fest.}

    \config{DHCP\_LEASES\_DIR}{DHCP\_LEASES\_DIR}{DHCPLEASESDIR}

    {legt das Verzeichnis für die Leases-Datei fest. 
    Möglich ist die Angabe eines absoluten Pfades oder des Wertes \emph{auto}. 
    Bei Angabe von \emph{auto} wird die lease-Datei im Unterverzeichnis dhcp des 
    persistent-Verzeichnisses (siehe Base-Dokumentation) abgelegt.}

    \config{DHCP\_LEASES\_VOLATILE}{DHCP\_LEASES\_VOLATILE}{DHCPLEASESVOLATILE}

    {Befindet sich das Verzeichnis für die \emph{Leases} in der Ram-Disk (da der
    Router z.B. von CD oder einem anderen nicht schreibbaren Medium bootet), gibt
    der Router beim Booten eine Warnung wegen einer fehlenden \emph{Lease}-Datei aus.
    Diese Warnung entfällt, wenn man \var{DHCP\_LEASES\_VOLATILE} auf \emph{yes} setzt.}

    \config{DHCP\_DNS\_SERVERS}{DHCP\_DNS\_SERVERS}{DHCPDNSSERVERS}

    {legt die Adressen der DNS-Server fest. \\
    Mehrere DNS-Server können durch Leerzeichen getrennt angegeben werden. 
    Diese Variable ist optional. 
    Wird hier nichts eingetragen, oder die Variable weggelassen,
    wird die IP-Adresse des zugehörigen Netzes verwendet, wenn der DNS-Server auf dem Router aktiviert ist.
    Es ist auch möglich, diese Variable auf 'none' zu setzen. Dann wird kein DNS-Server übertragen.
    Diese Einstellung wird ggf. von 
    \smalljump{DHCPRANGExDNSSERVERS}{\var{DHCP\_RANGE\_x\_DNS\_SERVERS}} überschrieben.}
    
    \config{DHCP\_WINS\_SERVERS}{DHCP\_WINS\_SERVERS}{DHCPWINSSERVERS}

    {legt die Adressen der WINS-Server fest. \\
    Mehrere WINS-Server können durch Leerzeichen getrennt angegeben werden. 
    Diese Variable ist optional. 
    Wird hier nichts eingetragen, oder die Variable weggelassen,
    wird bei installiertem und aktiviertem WINS-Server die Adresse des WINS-Server 
    des SAMBA-Paketes übernommen.
    Es ist auch möglich, diese Variable auf 'none' zu setzen. Dann wird kein WINS-Server übertragen.
    Diese Einstellung wird ggf. von 
    \smalljump{DHCPRANGExWINSSERVERS}{\var{DHCP\_RANGE\_x\_WINS\_SERVERS}} überschrieben.}

    \config{DHCP\_NTP\_SERVERS}{DHCP\_NTP\_SERVERS}{DHCPNTPSERVERS}
    
    {legt die Adressen der NTP-Server fest. \\
    Mehrere NTP-Server können durch Leerzeichen getrennt angegeben werden. 
    Diese Variable ist optional. 
    Wird hier nichts eingetragen, oder die Variable weggelassen,
    wird die IP-Adresse des zugehörigen Netzes verwendet, wenn ein
    Zeitserverpaket auf dem Router aktiviert ist.
    Es ist auch möglich, diese Variable auf 'none' zu setzen. Dann wird kein NTP-Server übertragen.
    Diese Einstellung wird ggf. von 
    \smalljump{DHCPRANGExNTPSERVERS}{\var{DHCP\_RANGE\_x\_NTP\_SERVERS}} überschrieben.}
    
     \config{DHCP\_OPTION\_WPAD}{DHCP\_OPTION\_WPAD}{DHCPOPTIONWPAD}
     
     {aktiviert oder deaktiviert die Übermittlung der DHCP-OPTION 252 (Web Proxy Autodiscovery Protocol)
     womit Browser automatisiert die Proxy-Einstellungen beziehen können.
     (siehe \altlink{http://de.wikipedia.org/wiki/Web_Proxy_Autodiscovery_Protocol})
     }
     
     \config{DHCP\_OPTION\_WPAD\_URL}{DHCP\_OPTION\_WPAD\_URL}{DHCPOPTIONWPADURL}
     
     {definiert die URL der wpad.dat Datei oder wird leer gelassen um eine leere Antwort
     an den anfragenden Browser zu schicken, wodurch dieser keine weiteren regelmäßigen Anfragen durchführt.
     }
     
\end{description}
\subsubsection{Lokale DHCP-Range}
\begin{description}
    \config{DHCP\_RANGE\_N}{DHCP\_RANGE\_N}{DHCPRANGEN}

    {Anzahl der DHCP-Ranges}

    \config{DHCP\_RANGE\_x\_NET}{DHCP\_RANGE\_x\_NET}{DHCPRANGExNET}
    
    {Referenz zu einem in \var{IP\_NET\_x} definiertem Netz}
    
    \config{DHCP\_RANGE\_x\_START}{DHCP\_RANGE\_x\_START}{DHCPRANGExSTART}

    {legt die erste zu vergebende IP-Adresse fest.}

    \config{DHCP\_RANGE\_x\_END}{DHCP\_RANGE\_x\_END}{DHCPRANGExEND}

    {legt die letzte zu vergebende IP-Adresse fest. Die beiden Variablen
    \var{DHCP\_RANGE\_x\_START} und \var{DHCP\_RANGE\_x\_END} kann man auch leer
    lassen, es wird dann keine DHCP-Range angelegt und nur die weiteren Variablen
    genutzt, um einem Host der per MAC-Zuordnung seine DHCP-IP bezieht, die Werte
    der Variablen zu übergeben.}

    \config{DHCP\_RANGE\_x\_DNS\_DOMAIN}{DHCP\_RANGE\_x\_DNS\_DOMAIN}{DHCPRANGExDNSDOMAIN}

    {legt eine spezielle DNS-Domain für DHCP-Hosts dieser Range fest.
    Diese Variable ist optional. Wird hier nichts eingetragen, oder die Variable weggelassen,
    wird der Default DNS-Domain \var{DOMAIN\_NAME} verwendet.}
    
    \config{DHCP\_RANGE\_x\_DNS\_SERVERS}{DHCP\_RANGE\_x\_DNS\_SERVERS}{DHCPRANGExDNSSERVERS}

    {legt die Adressen der DNS-Server fest. \\
    Mehrere DNS-Server können durch Leerzeichen getrennt angegeben werden. 
    Diese Variable ist optional. Wird hier nichts eingetragen, oder die Variable weggelassen,
    wird die Einstellung aus \smalljump{DHCPDNSSERVERS}{\var{DHCP\_DNS\_SERVERS}} verwendet. 
    Es ist auch möglich, diese Variable auf 'none' zu setzen. Dann wird kein DNS-Server übertragen.}

    \config{DHCP\_RANGE\_x\_WINS\_SERVERS}{DHCP\_RANGE\_x\_WINS\_SERVERS}{DHCPRANGExWINSSERVERS}

    {legt die Adressen der WINS-Server fest. \\
    Mehrere WINS-Server können durch Leerzeichen getrennt angegeben werden. 
    Diese Variable ist optional. Wird hier nichts eingetragen, oder die Variable weggelassen,
    wird die Einstellung aus \smalljump{DHCPDNSSERVERS}{\var{DHCP\_WINS\_SERVERS}} verwendet. 
    Es ist auch möglich, diese Variable auf 'none' zu setzen. Dann wird kein WINS-Server übertragen.}

    \config{DHCP\_RANGE\_x\_NTP\_SERVERS}{DHCP\_RANGE\_x\_NTP\_SERVERS}{DHCPRANGExNTPSERVERS}

    {legt die Adressen der NTP-Server fest. \\
    Mehrere NTP-Server können durch Leerzeichen getrennt angegeben werden. 
    Diese Variable ist optional. Wird hier nichts eingetragen, oder die Variable einfach weggelassen,
    wird die Einstellung aus \smalljump{DHCPDNSSERVERS}{\var{DHCP\_NTP\_SERVERS}} verwendet. 
    Es ist auch möglich, diese Variable auf 'none' zu setzen. Dann wird kein NTP-Server übertragen.}

    \config{DHCP\_RANGE\_x\_GATEWAY}{DHCP\_RANGE\_x\_GATEWAY}{DHCPRANGExGATEWAY}
    
    {legt die Adresse des Gateways für diese Range fest.
    Diese Variable ist optional. Wird hier nichts eingetragen, oder die Variable einfach weggelassen,
    wird die IP-Adresse des in \var{DHCP\_RANGE\_x\_NET} referenzierten Netzes verwendet.
    Es ist auch möglich, diese Variable auf 'none' zu setzen. Dann wird kein Gatway übertragen.}

    \config{DHCP\_RANGE\_x\_MTU}{DHCP\_RANGE\_x\_MTU}{DHCPRANGExMTU}
    
    {legt die MTU für Clients in diesem Range fest.
    Diese Variable ist optional.}

     \config{DHCP\_RANGE\_x\_OPTION\_WPAD}{DHCP\_RANGE\_x\_OPTION\_WPAD}{DHCPRANGExOPTIONWPAD}
     
     {aktiviert oder deaktiviert die Übermittlung der DHCP-OPTION 252 (Web Proxy Autodiscovery Protocol)
     für diese DHCP-Range womit Browser automatisiert die Proxy-Einstellungen beziehen können.
     (siehe \altlink{http://de.wikipedia.org/wiki/Web_Proxy_Autodiscovery_Protocol})
     Diese Variable ist optional.
     }
     
     \config{DHCP\_RANGE\_x\_OPTION\_WPAD\_URL}{DHCP\_RANGE\_x\_OPTION\_WPAD\_URL}{DHCPRANGExOPTIONWPADURL}
     
     {definiert die URL der wpad.dat Datei oder wird leer gelassen um eine leere Antwort
     an den anfragenden Browser zu schicken, wodurch dieser keine weiteren regelmäßigen Anfragen durchführt.
     Diese Variable ist optional.
     }

    
    \configlabel{DHCP\_RANGE\_x\_OPTION\_x}{DHCPRANGExOPTIONx}  
    \config{DHCP\_RANGE\_x\_OPTION\_N}{DHCP\_RANGE\_x\_OPTION\_N}{DHCPRANGExOPTIONN}
    
    {gestattet die Angabe Nutzer-definierter Optionen für diesen Bereich.
    Die verfügbaren Optionen kann man dem Manual des dnsmasq entnehmen
    (\altlink{http://thekelleys.org.uk/dnsmasq/docs/dnsmasq.conf.example}).
    Sie werden ungeprüft übernommen, können also bei Fehlern zu Problemen
    mit dem DNS/DHCP-Server führen.
    Diese Variable ist optional.}


\end{description}
\subsubsection{Extra DHCP-Range}
\begin{description}
    \config{DHCP\_EXTRA\_RANGE\_N}{DHCP\_EXTRA\_RANGE\_N}{DHCPEXTRARANGEN}

    {legt die Anzahl von DHCP-Bereichen fest, die an nicht lokale Netze vergeben werden. Hierzu
    ist am Gateway zum entsprechenden Netz ein DHCP-Relay zu installieren.}

    \config{DHCP\_EXTRA\_RANGE\_x\_START}{DHCP\_EXTRA\_RANGE\_x\_START}{DHCPEXTRARANGExSTART}

    {erste zu vergebende IP-Adresse.}

    \config{DHCP\_EXTRA\_RANGE\_x\_END}{DHCP\_EXTRA\_RANGE\_x\_END}{DHCPEXTRARANGExEND}

    {letzte zu vergebende IP-Adresse.}

    \config{DHCP\_EXTRA\_RANGE\_x\_NETMASK}{DHCP\_EXTRA\_RANGE\_x\_NETMASK}{DHCPEXTRARANGExNETMASK}

    {Netzwerkmaske für diesen Bereich.}

    \config{DHCP\_EXTRA\_RANGE\_x\_DNS\_SERVERS}{DHCP\_EXTRA\_RANGE\_x\_DNS\_SERVERS}{DHCPEXTRARANGExDNSSERVERS}

    {Adressen der DNS-Server \\
    (siehe \smalljump{DHCPRANGExDNSSERVERS}{\var{DHCP\_RANGE\_x\_DNS\_SERVERS}}).}

    \config{DHCP\_EXTRA\_RANGE\_x\_WINS\_SERVERS}{DHCP\_EXTRA\_RANGE\_x\_WINS\_SERVERS}{DHCPEXTRARANGExWINSSERVERS}

    {Adressen der WINS-Server \\
    (siehe \smalljump{DHCPRANGExWINSSERVERS}{\var{DHCP\_RANGE\_x\_WINS\_SERVERS}}).}
    
    \config{DHCP\_EXTRA\_RANGE\_x\_NTP\_SERVERS}{DHCP\_EXTRA\_RANGE\_x\_NTP\_SERVERS}{DHCPEXTRARANGExNTPSERVERS}

    {Adressen der NTP-Server \\
    (siehe \smalljump{DHCPRANGExNTPSERVERS}{\var{DHCP\_RANGE\_x\_NTP\_SERVERS}}).}

    \config{DHCP\_EXTRA\_RANGE\_x\_GATEWAY}{DHCP\_EXTRA\_RANGE\_x\_GATEWAY}{DHCPEXTRARANGExGATEWAY}

    {Adresse des Default-Gateway für diesen Bereich.}

    \config{DHCP\_EXTRA\_RANGE\_x\_MTU}{DHCP\_EXTRA\_RANGE\_x\_MTU}{DHCPEXTRARANGExMTU}
    
    {legt die MTU für Clients in dieser Range fest.
    Diese Variable ist optional.}

    \config{DHCP\_EXTRA\_RANGE\_x\_DEVICE}{DHCP\_EXTRA\_RANGE\_x\_DEVICE}{DHCPEXTRARANGExDEVICE}

    {Netzwerkinterface über den dieser Bereich erreicht wird.}
\end{description}
\subsubsection{Nicht zugelassene DHCP-Clients}
\begin{description}
    \config{DHCP\_DENY\_MAC\_N}{DHCP\_DENY\_MAC\_N}{DHCPDENYMACN}

    {Anzahl der MAC-Adressen von Host, dennen der Zugriff auf DHCP-Adressen verweigert wird.}

    \config{DHCP\_DENY\_MAC\_x}{DHCP\_DENY\_MAC\_x}{DHCPDENYMACx}

    {MAC-Adresse des Hosts, dem der Zugriff auf DHCP-Adressen verweigert wird.}

  \end{description}

  \subsubsection{Unterstützung fürs Booten vom Netz}

  Der dnsmasq unterstützt Clients, die via Bootp/PXE übers Netz
  booten. Die dafür nötigen Informationen werden vom dnsmasq
  bereitgestellt und pro Subnetz und Host konfiguriert. Die dafür
  nötigen Variablen sind in den DHCP\_RANGE\_\%- und
  HOST\_\%-Abschnitten untergebracht und beschreiben das zu bootende
  File (*\_PXE\_FILENAME), den Server, der dieses File bereitstellt
  (*\_PXE\_SERVERNAME und *\_PXE\_SERVERIP) und evtl. notwendige
  Optionen (*\_PXE\_OPTIONS). Weiterhin kann man den internen
  tftp-Server aktivieren, so dass das Booten komplett von dnsmasq
  unterstützt wird.

  \begin{description}

    \configlabel{HOST\_x\_PXE\_FILENAME}{HOSTxPXEFILENAME}
    \config{HOST\_x\_PXE\_FILENAME DHCP\_RANGE\_x\_PXE\_FILENAME}{DHCP\_RANGE\_x\_PXE\_FILENAME}{DHCPRANGExPXEFILENAME}

    Hier wird das zu bootende Image angegeben. Im Falle von PXE wird
    hier der zu ladende pxe-Bootloader, wie z.B. pxegrub, pxelinux
    oder ein anderer passender Bootloader angegeben.

    \configlabel{HOST\_x\_PXE\_SERVERNAME}{HOSTxPXESERVERNAME}
    \configlabel{HOST\_x\_PXE\_SERVERIP}{HOSTxPXESERVERIP}
    \configlabel{DHCP\_RANGE\_x\_PXE\_SERVERNAME}{DHCPRANGExPXESERVERNAME}
    \config{HOST\_x\_PXE\_SERVERNAME HOST\_x\_PXE\_SERVERIP
    DHCP\_RANGE\_x\_PXE\_SERVERNAME
    DHCP\_RANGE\_x\_PXE\_SERVERIP}{DHCP\_RANGE\_x\_PXE\_SERVERIP}{DHCPRANGExPXESERVERIP}
    Name und IP des tftp-Servers, werden diese Variablen leer
    gelassen, wird der Router selbst als tftp-Server übermittelt.

    \configlabel{HOST\_x\_PXE\_OPTIONS}{HOSTxPXEOPTIONS}
    \config{DHCP\_RANGE\_x\_PXE\_OPTIONS HOST\_x\_PXE\_OPTIONS}{DHCP\_RANGE\_x\_PXE\_OPTIONS}{DHCPRANGExPXEOPTIONS}

    Einige Bootloader benötigen spezielle Optionen zum Booten. So
    erfragt zum Beispiel pxegrub über die Option 150 den Namen der
    Menu-Datei. Diese Optionen können hier angegeben werden und werden
    dann ins Konfigfile übernommen. Im Falle von pxegrub könnte das
    z.B. wie folgt aussehen:\\
    \begin{example}
      \begin{verbatim}
	HOST_x_PXE_OPTIONS='150,"(nd)/grub-menu.lst"'
      \end{verbatim}
    \end{example}

    Sind mehrere Optionen nötig, werden sie einfach mit Leerzeichen
    voneinander getrennt angegeben.

  \end{description}

\subsection {DHCP-Relay}

Das DHCP-Relay wird dann verwendet, wenn ein anderer DHCP-Server die Verwaltung
der Ranges übernimmt, der nicht direkt von den Clients erreicht werden kann.

\begin{description}

\config{OPT\_DHCPRELAY}{OPT\_DHCPRELAY}{OPTDHCPRELAY}

Dieser Wert ist auf 'yes' zu setzen, damit der Router als DHCP-Relay arbeitet.
Es darf nicht gleichzeitig ein DHCP-Server aktiv sein.

Standard-Einstellung: \var{OPT\_\-DHCPRELAY}='no'

\config{DHCPRELAY\_SERVER}{DHCPRELAY\_SERVER}{DHCPRELAYSERVER}
An dieser Stelle wird der richtige DHCP-Server eingetragen, an den die Anfragen weitergereicht
werden sollen.

\configlabel{DHCPRELAY\_IF\_N}{DHCPRELAYIFN}
\configlabel{DHCPRELAY\_IF\_x}{DHCPRELAYIFx}
\configvar{DHCPRELAY\_IF\_N DHCPRELAY\_IF\_x}
Mit \var{DHCPRELAY\_\-IF\_N} gibt man die Anzahl der Netzwerkkarten an, auf denen der Relay-Server
lauschen soll. In \var{DHCPRELAY\_IF\_x} werden dann die entsprechenden Netzwerkkarten angegeben.

Das Interface, über das die Antworten des DHCP-Servers wieder
reinkommen, muß in der Liste mit aufgeführt werden.Zusätzlich muss
sichergestellt werden, dass die Routen auf dem Rechner, auf dem der
DHCP-Server läuft, korrekt gesetzt sind. Die Antwort des DHCP-Servers
geht an die IP des Interfaces, an dem der DHCP-Client hängt. Nehmen
wir folgendes Scenario an:

\begin{itemize}
\item Relay mit zwei Interfaces
\item Interfaces zum Client: eth0, 192.168.6.1
\item Interfaces zum DHCP-Server:  eth1, 192.168.7.1
\item DHCP-Server:  192.168.7.2
\end{itemize}

Dann muss es auf dem DHCP-Server eine Route geben, über den die
Antworten an die 192.168.6.1 ihr Ziel erreichen. Ist der Router, auf
dem das Relay läuft, der default gateway für den DHCP-Server, ist
bereits alles ok. 
Ist dem nicht so, wird eine extra Route benötigt. Ist der DHCP-Server
ein fli4l-Router, würde folgender Konfig-Eintrag dieses Ziel
erreichen: IP\_ROUTE\_x='192.168.6.0/24 192.168.7.1'

Im Betrieb kann es zu Warnungen kommen, dass bestimmte Pakete ignoriert werden.
Diese Warnungen kann man ignorieren, sie stören nicht den normalen Betrieb.

Beispiel:

\begin{example}
\begin{verbatim}
        OPT_DHCPRELAY='yes'
        DHCPRELAY_SERVER='192.168.7.2'
        DHCPRELAY_IF_N='2'
        DHCPRELAY_IF_1='eth0'
        DHCPRELAY_IF_2='eth1'
\end{verbatim}
\end{example}

\end{description}

\marklabel{sec:dhcp}{
\subsection{DHCP-Client}
}

Ein DHCP-Client kann verwendet werden, um eine IP-Adresse für eine oder
mehrere Schnittstellen des Routers zu beziehen~-- dies erfolgt meist vom
Service-Provider. Diese Möglichkeit der Anbindung kommt derzeit
hauptsächlich bei Kabelmodem-Betreibern und in der Schweiz, in den
Niederlanden und in Frankreich vor. Manchmal wird diese Konfiguration
auch benötigt, wenn der Router hinter einem anderen Router eingebunden
wird, der die Adressen per DHCP verteilt (z.\,B. FritzBox).

Die Konfiguration eines Netzwerks für DHCP erfordert zwei verschiedene
Aktionen:
\begin{itemize}
\item Zum einen muss ein DHCP-Circuit konfiguriert werden, der DHCP für eine
Schnittstelle aktiviert (s.\,u.).
\item Zum anderen muss ein Netz via \var{IP\_NET\_x} definiert sein, dessen
Schnittstelle mit der Circuit-Schnittstelle übereinstimmt und das auf den
Namen des DHCP-Circuits gesetzt wird (siehe Beispiele weiter unten). Dadurch
wird erst die Verbindung zwischen Netzwerk und DHCP-Circuit hergestellt.
\end{itemize}

Die Konfiguration der Netzwerk-Schnittstelle via DHCP wird immer dann
durchgeführt, wenn der betreffende Circuit online geht; analog wird die
erhaltene Adresse wieder freigegeben, wenn der Circuit offline geht. Routen
werden gemäß der Variablen \var{CIRC\_\%\_NETS\_IPV4} gesetzt.

Das Paket wird über die Variable \var{OPT\_DHCP\_CLIENT} aktiviert:

\begin{description}
\config{OPT\_DHCP\_CLIENT}{OPT\_DHCP\_CLIENT}{OPTDHCPCLIENT}

Diese Variable muss auf 'yes' gesetzt werden, um DHCP-Circuits anlegen zu
können.

Standard-Einstellung: \verb+OPT_DHCP_CLIENT='no'+
\end{description}

Soll für ein Netz DHCP verwendet werden, so muss ein passender Circuit
konfiguriert werden, siehe Abschnitt \jump{sect:circuits}{Circuit-Konfiguration}.
Der zu verwendende Typ in \var{CIRC\_x\_TYPE} lautet ``dhcp''. Zu den
allgemeinen Circuit-Variablen kommen die folgenden hinzu, die DHCP-spezifisch
sind:

\begin{description}

\config{CIRC\_x\_DHCP\_DEV}{CIRC\_x\_DHCP\_DEV}{CIRCxDHCPDEV}

In dieser Variable ist die Netzwerk-Schnittstelle vermerkt, für die das
DHCP-Protokoll genutzt werden soll. Es ist sinnvoll, auf die Schnittstelle mit
Hilfe der zugehörigen \var{IP\_NET\_x\_DEV}-Variable zu verweisen.

Beispiel: \verb+CIRC_1_DHCP_DEV='IP_NET_1_DEV'+

\config{CIRC\_x\_DHCP\_DAEMON}{CIRC\_x\_DHCP\_DAEMON}{CIRCxDHCPDAEMON}

Das Paket kommt momentan mit zwei verschiedenen Implementierungen eines
DHCP-Clients, \texttt{dhcpcd} und \texttt{dibbler}. Momentan wird
\texttt{dhcpcd} nur für DHCPv4 und \texttt{dibbler} nur für DHCPv6 verwendet,
so dass man nicht wirklich eine Wahl hat und diese Einstellung im besten Falle
einfach nicht setzen sollte. Sollten später weitere Implementierungen
unterstützt werden, wird die Auswahl über diese Variable erfolgen.

Beispiel 1: \verb+CIRC_1_DHCP_DAEMON='dhcpcd'+
Beispiel 2: \verb+CIRC_1_DHCP_DAEMON='dibbler'+

Falls IPv6-Unterstützung aktiviert ist, muss darauf geachtet werden, dass nur
IPv6-Netze (via \var{CIRC\_x\_NET\_IPV6}) geroutet werden und die Variable
\var{CIRC\_x\_NET\_IPV4} nicht vorhanden oder leer ist. Auch ist zu beachten,
dass der DHCPv6-Server in der Regel nicht eine einzelne Adresse, sondern ein
ganzes IPv6-Subnetzpräfix zuweist (Prefix Delegation), so dass die
\var{IPV6\_NET\_x}-Variable ein Suffix enthalten sollte (siehe hierzu die
letzten beiden Beispiele weiter unten).

Es ist \emph{nicht} möglich, mit Hilfe eines einzigen Circuits sowohl IPv4-
als auch IPv6-Adressen via DHCP zu empfangen, Hierzu müssen zwei separate
DHCP-Circuits eingerichtet werden.

Standard-Einstellung:\\
\verb+CIRC_1_DHCP_DAEMON='dhcpcd'+ (für IPv4)\\
\verb+CIRC_1_DHCP_DAEMON='dibbler'+ (für IPv6)

\config{CIRC\_x\_DHCP\_HOSTNAME}{CIRC\_x\_DHCP\_HOSTNAME}{CIRCxDHCPHOSTNAME}

Manche Provider verlangen die Übermittlung eines Hostnamens. Dieser Name ist
vom Provider zu erfahren und hier anzugeben. Er muss nicht zwangsläufig mit dem
Hostnamen des Routers übereinstimmen.

Beispiel: \verb+CIRC_1_DHCP_HOSTNAME='gandalf'+

Fehlt diese Variable oder ist sie leer, wird kein Hostname zum DHCP-Server
gesandt.

Standard-Einstellung: \verb+CIRC_1_DHCP_HOSTNAME=''+

\config{CIRC\_x\_DHCP\_STARTDELAY}{CIRC\_x\_DHCP\_STARTDELAY}{CIRCxDHCPSTARTDELAY}

Mit dieser Variable kann optional der Start des DHCP-Clients verzögert werden.

In manchen Installationen (z.\,B. fli4l als DHCP-Client hinter einem Kabelmodem
oder einer FritzBox) ist es notwendig zu warten, bis der genutzte DHCP-Server
z.\,B. nach einen Stromausfall ebenfalls neu gestartet worden ist.

Falls \var{CIRC\_x\_WAIT} ebenfalls verwendet wird, muss darauf geachtet werden,
dass die Variable \var{CIRC\_x\_WAIT} größer als \var{CIRC\_x\_DHCP\_STARTDELAY}
ist, da ansonsten immer zu wenig Zeit zum Warten zur Verfügung steht.

Beispiel: \verb+CIRC_1_DHCP_STARTDELAY='20'+

Standard-Einstellung: \verb+CIRC_1_DHCP_STARTDELAY='0'+

\config{CIRC\_x\_DHCP\_ACCEPT\_CSR}{CIRC\_x\_DHCP\_ACCEPT\_CSR}{CIRCxDHCPACCEPTCSR}

Diese Variable steuert die Übernahme von zusätzlichen Routen, die der
DHCP-Server senden kann. Für gewöhnlich wird dem DHCP-Client vom DHCP-Server
nur ein Default-Router mitgeteilt. Gelegentlich passiert es jedoch, dass der
DHCP-Server gar keinen Default-Router mitteilt und/oder andere Routen
übermittelt. Das ist beispielsweise bei einem Telekom-Entertain-IPTV-Anschluss
der Fall. In diesem Fall werden diese zusätzlichen Routen ebenfalls auf dem
fli4l-Router installiert.

Diese Verarbeitung ist standardmäßig aktiviert. Vertrauen Sie Ihrem DHCP-Server
jedoch nicht, können Sie die Verarbeitung der zusätzlichen Routen deaktivieren.
Bedenken Sie jedoch, dass dann die korrekte Routing-Funktionalität nicht
garantiert werden kann.

Beispiel: \verb+CIRC_1_DHCP_ACCEPT_CSR='no'+

Standard-Einstellung: \verb+CIRC_1_DHCP_ACCEPT_CSR='yes'+
\end{description}

Neben einem oder mehreren passenden Circuits müssen weiterhin ein oder mehrere
passende Netze via \var{IP\_NET\_x} in der \texttt{config/base.txt} für den
DHCP-Betrieb vorbereitet werden. Dazu müssen diese Variablen auf den Namen des
jeweils zu verwendenden DHCP-Circuits oder eines seiner Schlagwörter (beides
jeweils in geschweiften Klammern) gesetzt werden.

Beispiel 1 (IPv4):

\begin{example}
\begin{verbatim}
    IP_NET_N='1'
    IP_NET_1='{circ1}' # alternativ '{dhcp0}'
    IP_NET_1_DEV='eth1'
    [...]
    #
    CIRC_N='1'
    #
    CIRC_1_NAME='DHCP-LAN'
    CIRC_1_TYPE='dhcp'
    CIRC_1_ENABLED='yes'
    CIRC_1_NETS_IPV4_N='1'
    CIRC_1_NETS_IPV4_1='0.0.0.0/0'
    CIRC_1_UP='yes'
    CIRC_1_WAIT='15'
    CIRC_1_DHCP_DEV='IP_NET_1_DEV'
\end{verbatim}
\end{example}

Beispiel 2 (IPv4 mit Nutzung von Schlagwörtern):

\begin{example}
\begin{verbatim}
    IP_NET_N='1'
    IP_NET_1='{dhcp-lan}'
    IP_NET_1_DEV='eth1'
    [...]
    #
    CIRC_N='1'
    #
    CIRC_1_NAME='DHCP-LAN'
    CIRC_1_TYPE='dhcp'
    CIRC_1_ENABLED='yes'
    CIRC_1_NETS_IPV4_N='1'
    CIRC_1_NETS_IPV4_1='0.0.0.0/0'
    CIRC_1_UP='yes'
    CIRC_1_WAIT='15'
    CIRC_1_DHCP_DEV='IP_NET_1_DEV'
\end{verbatim}
\end{example}

Beispiel 3 (IPv6 mit DHCPv6-Server im LAN und Referenz über den Circuit-Namen):

\begin{example}
\begin{verbatim}
    IPV6_NET_N='1'
    IPV6_NET_1='{DHCPv6-LAN}::1:0:0:0:1/64'
    IPV6_NET_1_DEV='eth1'
    [...]
    #
    CIRC_N='1'
    #
    CIRC_1_NAME='DHCPv6-LAN'
    CIRC_1_TYPE='dhcp'
    CIRC_1_ENABLED='yes'
    CIRC_1_UP='yes'
    CIRC_1_WAIT='15'
    CIRC_1_DHCP_DEV='IPV6_NET_1_DEV'
    CIRC_1_PROTOCOLS='ipv6'
\end{verbatim}
\end{example}

Beispiel 4 (IPv6 mit DHCPv6-Server auf der anderen Seite einer PPP-Verbindung):

\begin{example}
\begin{verbatim}
    IPV6_NET_N='1'
    IPV6_NET_1='{DHCPv6-manitu}::1:0:0:0:1/64'
    IPV6_NET_1_DEV='eth1'
    [...]
    #
    CIRC_N='2'
    #
    CIRC_1_NAME='DHCPv6-manitu'
    CIRC_1_TYPE='dhcp'
    CIRC_1_ENABLED='yes'
    CIRC_1_UP='yes'
    CIRC_1_WAIT='15'
    CIRC_1_DHCP_DEV='{DSL-manitu}'
    CIRC_1_PROTOCOLS='ipv6'
    #
    CIRC_2_NAME='DSL-manitu'
    CIRC_2_TYPE='ppp'
    CIRC_2_ENABLED='yes'
    CIRC_2_PPP_TYPE='ethernet'
    CIRC_2_NETS_IPV6_N='1'
    CIRC_2_NETS_IPV6_1='::/0'
    [...]
\end{verbatim}
\end{example}

In den IPv4-Beispielen wird die Default-Route über den DHCP-Circuit gelegt. Ist
das nicht erwünscht, muss \var{CIRC\_x\_NETS\_IPV4\_\%} entsprechend geändert
werden. Auch werden alle Circuits bereits beim Booten aktiviert
(\verb+CIRC_x_UP='yes'+), und der Boot-Vorgang wartet maximal 15 Sekunden
darauf, dass der Router eine IP-Adresse erfolgreich via DHCP erhalten hat. Das
ist vor allem dann nötig, wenn andere Pakete eine vorhandene Netzanbindung
während des Boot-Vorgangs erfordern.

In den IPv6-Beispielen werden \emph{keine} Default-Routen über den DHCP-Circuit
gelegt. Das liegt daran, dass bei DHCPv6 (im Gegensatz zum IPv4-Pendant) keine
Informationen zum Routing versandt werden. Informationen über Next-Hop-Router
werden hier über so genannte ``Router Advertisements'' verschickt.

\subsection {TFTP-Server}

Der TFTP-Server wird dann verwendet, wenn der fli4l per TFTP Dateien ausliefern
soll. Dies kann zum Beispiel dazu dienen, das ein Client per Netboot startet.

\begin{description}

    \config{OPT\_TFTP}{OPT\_TFTP}{OPTTFTP}
    Aktiviert den internen TFTP-Server des dnsmasq. Standard-Wert ist 'no'.

    \config{TFTP\_PATH}{TFTP\_PATH}{TFTPPATH}

    Spezifiziert das Verzeichnis, in dem die Dateien liegen, die der
    tftp-Server an die Klienten ausliefern soll. Die entsprechenden Dateien sind
    mit Hilfe eines geeigneten Programms (z.B. scp) im entsprechenden Pfad 
    abzulegen.
    
\end{description}

\subsection {YADIFA - Slave DNS Server}


\begin{description}

    \config{OPT\_YADIFA}{OPT\_YADIFA}{OPYADIFA}

    Aktiviert den YADIFA Slave DNS Server. Standard-Wert ist 'no'.

    \config{OPT\_YADIFA\_USE\_DNSMASQ\_ZONE\_DELEGATION}{OPT\_YADIFA\_USE\_DNSMASQ\_ZONE\_DELEGATION}{OPTYADIFAUSEDNSMASQZONEDELEGATION}

    Wenn diese Einstellung aktiviert wird erzeugt das yadifa
    Startscript automatisch für alle Slavezonen entsprechende Zone
    Delegation Einträge für den dnsmasq. Damit sind die Slavezonen
    auch direkt über den dnsmasq abfragbar und man benötigt im
    Prinzip keine YADIFA\_LISTEN\_x Einträge mehr. Die Anfragen werden
    dann vom dnsmasq beantwortet und einen nur auf localhost:35353
    horchenden yadifa weitergeleitet.

    \config{YADIFA\_LISTEN\_N}{YADIFA\_LISTEN\_N}{YADIFALISTENN}

    Wenn Sie \var{OPT\_YADIFA}='yes' gewählt haben, können Sie mit
    Hilfe von \var{YADIFA\_LISTEN\_N} die Anzahl, und
    mit \var{YADIFA\_LISTEN\_1} bis \var{YADIFA\_LISTEN\_N} lokale IPs
    angeben, auf denen YADIFA DNS-Anfragen annehmen darf. Eine
    Portnummer ist optional möglich, mit der Angabe 192.168.1.1:5353
    würde der YADIFA Slave DNS Server auf DNS Anfragen auf Port 5353
    horchen. Achten Sie darauf, dass der dnsmasq in diesem Fall nicht
    auf allen Schnittstellen horchen darf
    (siehe \var{DNS\_BIND\_INTERFACES}).  An dieser Stelle dürfen nur
    IPs von existierenden Schnittstellen (ethernet, wlan ...)
    verwendet werden, es kommt sonst zu Warnmeldungen beim Start des
    Routers. Alternativ ist nun möglich hier auch ALIAS-Namen zu
    verwende, z.\,B. \verb+IP_NET_1_IPADDR+
      
    \config{YADIFA\_ALLOW\_QUERY\_N}{YADIFA\_ALLOW\_QUERY\_N}{YADIFAALLOWQUERYN}
    \config{YADIFA\_ALLOW\_QUERY\_x}{YADIFA\_ALLOW\_QUERY\_x}{YADIFAALLOWQUERYX}

    Gibt IP-Adressen und Netze an denen der Zugriff auf YADIFA erlaubt
    ist. YADIFA nutzt die Angaben um den fli4l Paketfilter
    entsprechend zu konfigurieren und die Konfigurationsdateien von
    YADIFA zu erstellen. Mit dem Prefix ! wird der IP-Adresse oder dem
    Netz der Zugriff auf YADIFA verweigert.

    Der fli4l Paketfilter wird für YADIFA so konfiguriert, dass alle
    erlaubten Netze aus dieser Einstellung und der für die einzelnen
    Zonen zusammen in eine ipset Liste (yadifa-allow-query)
    aufgenommen werden. Eine Unterscheidung nach Zonen ist beim
    Paketfilter leider nicht möglich. Zusätzlich werden alle
    IP-Adressen und Netze aus dieser globalen Einstellung, denen der
    Zugriff verweigert wird, in diese Liste aufgenommen. Es ist daher
    nicht möglich den Zugriff später für einzelne Zonen wieder
    auszuweiten.

    \config{YADIFA\_SLAVE\_ZONE\_N}{YADIFA\_SLAVE\_ZONE\_N}{YADIFASLAVEZONEN}

    Gibt die Anzahl der Slave DNS Zonen an die YADIFA verwalten soll.

    \config{YADIFA\_SLAVE\_ZONE\_x}{YADIFA\_SLAVE\_ZONE\_x}{YADIFASLAVEZONEx}

    Der Name der Slave DNS Zone.

    \config{OPT\_YADIFA\_SLAVE\_ZONE\_USE\_DNSMASQ\_ZONE\_DELEGATION}{OPT\_YADIFA\_SLAVE\_ZONE\_USE\_DNSMASQ\_ZONE\_DELEGATION}{OPTYADIFASLAVEZONEUSEDNSMASQZONEDELEGATION}

    Aktiviert (='yes') oder deaktiviert (='no') die dnsmasq Zone Delegation nur für die Slavezone.

    \config{YADIFA\_SLAVE\_ZONE\_x\_MASTER}{YADIFA\_SLAVE\_ZONE\_x\_MASTER}{YADIFASLAVEZONExMASTER}

    Die IP-Adresse mit einer optionalen Portnummer des DNS Master Server.

    \config{YADIFA\_SLAVE\_ZONE\_x\_ALLOW\_QUERY\_N}{YADIFA\_SLAVE\_ZONE\_x\_ALLOW\_QUERY\_N}{YADIFASLAVEZONExALLOWQUERYN}
    \config{YADIFA\_SLAVE\_ZONE\_x\_ALLOW\_QUERY\_x}{YADIFA\_SLAVE\_ZONE\_x\_ALLOW\_QUERY\_x}{YADIFASLAVEZONExALLOWQUERYx}

    Gibt IP-Adressen und Netze an denen der Zugriff auf diese YADIFA
    DNS Zone erlaubt ist. Damit kann der Zugriff auf bestimmte DNS
    Zonen weiter eingeschränkt werden. YADIFA nutzt die Angaben um die
    Konfigurationsdateien von YADIFA zu erstellen.

    Mit dem Prefix ! wird die IP-Adresse oder das Netz der Zugriff auf
    YADIFA verweigert.

\end{description}
