% Synchronized to r30451
\marklabel{sec:opt-vbox}
{
\section {OPT\_VBOX - The ISDN-Answering Machine For fli4l}
}

\subsection{Introduction}

This package integrates the answering machine software VBOX by 
Michael ’Ghandi’ Herold into the fli4l system.

By using this software the fli4l can take the part of an answering machine with
a functionality exceeding the possibilities of a normal box by far. You may create
several voice boxes taking messages for different phone numbers (MSNs) and restrict
access to defined users. Each user can set the following items depending on daytime,
weekday and caller ID:

\begin{itemize}
    \item if a call should be answered at all
    \item after how many ringtones will the call be answered
    \item if and what announce will be used
    \item if and what signaling tone will be used
    \item if a voice message will be recorded or not
    \item how long the recorded message is allowed to be
    \item if and what timeout announce will be used
\end{itemize}

\subsection{Some Basics About Function And Resources}

The VBOX system is divided into a server taking the messages and a client to
replay and manage the messages. In OPT\_VBOX messages can be accessed via a web
interface. A working httpd Package is necessary for this. 

\subsubsection{General Requirements For Installation}

An ISDN card is mandatory, because vbox will not work with modems
(and not via a DSL line anyway, if someone should come to this thought ...).
Of course the ISDN package is required for the installation of the appropriate driver
but no circuit has to be defined if the card is used solely for vbox.

It is highly recommended to use the httpd package, since it is quite convenient
to manage stored messages.

\wichtig{Teles 16.3c will not work for vbox because the Linux driver is not voice-capable.}

\subsubsection{Conditions For Storing Messages In A RAM Disk Installation}

If the incoming messages should be be stored in a ramdisk the router requires at least
16MiB of RAM, otherwise no appreciable recording time will be accessible (see: \var{VBOX\_COMPRESSION}).
It should be clear that messages saved in a ramdisk are lost during a power failure.
If you want to be able to rely on vbox for a 100\% you should store messages only
on a hard drive. The advantage of the RAM disk storage is that the disk can remain idle.

\subsubsection{Preconditions For Saving Messages In A HD/CF Installation}

When storing messages on a hard disk you obviously need the package HD with \var{OPT\_MOUNT}='yes'.
The messages will remain stored during router reboot then. Settings for users will eventually
be reloaded during reboot with the possibility of users automatically being deleted (optional)
that are not existing in the reloaded configuration. A harddisk is also of advantage when using
a lot of different messages or with additional OPT packages due to space limitations on the boot
medium.

\subsection{Configuration}
\subsubsection{Common Settings}
\begin{description}

\config{OPT\_VBOX}{OPT\_VBOX}{OPTVBOX}

        Default Setting: \var{OPT\_VBOX}='no'

        Specifying 'yes' here activates the voicebox.

\config{VBOX\_SPOOLPATH}{VBOX\_SPOOLPATH}{VBOXSPOOLPATH}

    The parameter \var{VBOX\_SPOOLPATH} determines if the recorded messages
    should be stored on a Ramdisk or on a harddisk.

    Storing to a Ramdisk:
    \var{VBOX\_SPOOLPATH}=''
    No path is allowed here in this case!

    Storing to a harddisk:

    \var{VBOX\_SPOOLPATH}='/data/vbox' (i.e.)\\
    Precondition is an installed HD Package with \var{OPT\_MOUNT}='yes'.

    The path specified here has to point to a writable harddisk partition in ext format mounted
    to the file system, but not mounted to '/opt' (HD-Install Type B). If one of the preconditions
    is not met the installation of the vbox package will be stopped with an error message.
    If all preconditions are met, a directory \var{vbox\_spooldir} will be created in the path
    specified here to store the vbox data.

    This may be set to 'auto' to use the path defined by \var{FLI4L\_UUID}.

\config{VBOX\_SPOOLDIR\_SPACE}{VBOX\_SPOOLDIR\_SPACE}{VBOXSPOOLDIRSPACE}
    
    Specifies how much space in KiB should be estimated for all mailboxes.
    While creating the spool directory it is checked whether this place is available.

\config{VBOX\_DELETE\_OLD\_SPOOLDIRS}{VBOX\_DELETE\_OLD\_SPOOLDIRS}{VBOXDELETEOLDSPOOLDIRS}
    
    Default Setting: \var{VBOX\_DELETE\_OLD\_SPOOLDIRS}='yes'
    
    This parameter is only relevant if the incoming messages are stored on a harddisk. Since
    the spool directories of the individual users are not deleted during system reboot it may
    happen due to changing configurations with different users that old spool directories
    from now inactive users take up space unnecessarily. To prevent this, you can set this
    parameter to 'yes' to automatically clear them at every reboot.

    \wichtig{Even if the name of a once configured user only changes slightly in the configuration
    file, the automatic deletion at the next reboot will remove all spooled messages of this user
    because the user with the original name is no longer considered active. You have to be very
    careful here in order to avoid deleting messages inadvertently with this automatic.}

\config{VBOX\_COMPRESSION}{VBOX\_COMPRESSION}{VBOXCOMPRESSION}

    Default Setting: \var{VBOX\_COMPRESSION}='ulaw'

    This specifies the compression with which the messages are to be recorded.
    The higher the compression, the lower the memory consumption and the worse the quality.
    On a hard disk the messages can probably safely use the 'ulaw' mode where no
    compression will be used. The messages are stored then with the full ISDN bandwidth
    of 8kHz 8bit * = 64kbit/s. In a pure ramdisk installation without hard disk on a PC
    with 16MiB memory and about 6MiB available for recording the recording time will
    only be a little more than ten minutes .

    Those who need to save space should use the format 'adpcm-4' for a compressed recording 
    (4 bits * = 8kHz 32kbit/s -> half the memory requirement compared to 'ulaw') or
    'adpcm-3' or 'adpcm-2' with a correspondingly higher compression and poorer quality.

    An overview:

    \begin{table}[htbp]
      \begin{tabular}{lrrr}
        Mode     & Resolution       & Compression to  & approx. storage needed for 10 mins.\\
        \hline \\
        ulaw     & 8Bit             & 100\%           & 4800 kB       \\
        adpcm-4  & 4Bit             &  50\%           & 2400 kB       \\
        adpcm-3  & 3Bit             &  37\%           & 1800 kB       \\
        adpcm-2  & 2Bit             &  25\%           & 1200 kB       \\
      \end{tabular}
      \caption{Comparison of the different compressions}
    \end{table}

\config{VBOX\_FREESPACE}{VBOX\_FREESPACE}{VBOXFREESPACE}

    Default Setting: \var{VBOX\_FREESPACE}='8192'

    If less than \var{VBOX\_FREESPACE} Bytes is available for storing new messages
    new calls will not be picked up anymore. A value of '0' disables this check.
 
\config{VBOX\_LOGPATH}{VBOX\_LOGPATH}{VBOXLOGPATH}

    Default Setting: \var{VBOX\_LOGPATH}='/var/log/vbox'

    Specifies the directory where log files should be placed.

    This may be set to 'auto' to use the path defined by \var{FLI4L\_UUID}.

\config{VBOX\_USE\_VBOXD}{VBOX\_USE\_VBOXD}{VBOXUSEVBOXD}

    Default Setting: \var{VBOX\_USE\_VBOXD}='no'
    
    Messages can be replayed with other vbox clients. If this is needed specify
    'yes' in this variable and don't forget to set a password in
    \var{VBOX\_USER\_x\_VBOXD\_PASSWORD}.
    
    \wichtig{vboxd is is a service on the router. You should prefer the WebGUI instead.}

\config{VBOX\_VBOXD\_ALLOW}{VBOX\_VBOXD\_ALLOW}{VBOXVBOXDALLOW}

    Default Setting: \var{VBOX\_VBOXD\_ALLOW}='*.lan.fli4l'
    
    By the parameter \var{VBOX\_VBOXD\_ALLOW} it may be defined which computers are allowed
    to replay or manage messages with a vbox client. The default setting '*.lan.fli4l'
    stands for all computers in this DNS domain. Restrictions can be made by specifying
    single IP-addresses, host names or domain names (like *.home.lan). In case of more
    entries they have to be divided by spaces.
    
    \wichtig{When using host names of the own domain do not use use fully qualified DNS names!}

    \wichtig{To minimize the risk of a potential attack on the server vboxd keep this
    setting as restrictive as possible. If for example only VBOX\_BEEP is needed,\\
    VBOX\_VBOXD\_ALLOW should stay empty restricting access to vbox to the router itself.}

\config{VBOX\_BEEP\_HOURS}{VBOX\_BEEP\_HOURS}{VBOXBEEPHOURS}

    Default Setting: \var{VBOX\_BEEP\_HOURS}='*'
    
    Here you may specify the times in hours where an acoustic signaling should take place.
    Also ranges can be defined i.e. 8-24, with more ranges separated by spaces. A '*' is
    defined as 'always'. A detailed description of time formats can be found in the
    appendix of this documentation.
    
    \wichtig{To make VBOX\_BEEP work, VBOX\_USE\_VBOXD='yes' has to be set.}

\config{VBOX\_BEEP\_PAUSE}{VBOX\_BEEP\_PAUSE}{VBOXBEEPPAUSE}

    Default Setting: \var{VBOX\_BEEP\_PAUSE}='60'
    
    The length of the break between signals in seconds.

\config{VBOX\_DEBUGLEVEL}{VBOX\_DEBUGLEVEL}{VBOXDEBUGLEVEL}

    Default Setting: \var{VBOX\_DEBUGLEVEL}='FE'

    Specify the events to be logged to the logfiles by setting a combination of characters.
    From the original documentation:
    \begin{itemize}
        \item F - Errors that could not be recovered
        \item E - Errors that may be recovered eventually
        \item W - Warnings
        \item I - Informations
        \item D - Debugging output
        \item J - Even more debugging output
    \end{itemize}

    Logfiles are very helpful for finding errors. In the beginning you may protocol everything
    and later on when being sure that all is well 'FE' should be sufficient.

\config{VBOX\_ADMIN\_USERNAME}{VBOX\_ADMIN\_USERNAME}{VBOXADMINUSERNAME}

    The user name of the administrator already defined in the httpd configuration
    (capital letters are taken into account). This user can review all voice boxes
    in the VBOX WebGUI and thus is able to start, stop and replay messages of all boxes.
    \wichtig{The user must have the right 'vbox:all'.}

\end{description}

\subsubsection{Common User Specific Settings}

\begin{description}

\config{VBOX\_USER\_N}{VBOX\_USER\_N}{VBOXUSERN}

    The number of users that should get VBOX messages.

\config{VBOX\_USER\_x\_USERNAME}{VBOX\_USER\_x\_USERNAME}{VBOXUSERUSERNAME}

    The username of the specific user. This username is also used for authentification
    at the WebGUI. If this username already exists in the httpd configuration (capital
    letters are taken into account) the rights specified there apply (see documentation
    for the httpd package). If the username is not defined there the user only gets
    the right to access the VBOX page in the WebGUI.

\config{VBOX\_USER\_x\_PASSWORD}{VBOX\_USER\_x\_PASSWORD}{VBOXUSERPASSWORD}

    The password of the user. If the user in \var{VBOX\_USER\_x\_USER} also exists in
    the configuration of the httpd package the password specified there is used and
    the content of \var{VBOX\_USER\_x\_PASS} is meaningless. In all other cases
    this password is used for authentification at the WebGUI.
    
\config{VBOX\_USER\_x\_VBOXD\_PASSWORD}{VBOX\_USER\_x\_VBOXD\_PASSWORD}{VBOXUSERxVBOXDPASSWORD}

    Here you may specify a password for vboxd. It is used only for the login
    with a vbox client (not with the WebGUI).
    
\config{VBOX\_USER\_x\_MSN\_N}{VBOX\_USER\_x\_MSN\_N}{VBOXUSERMSNN}

    Set the number of MSNs to be monitored here. 

\config{VBOX\_USER\_x\_MSN\_x}{VBOX\_USER\_x\_MSN\_x}{VBOXUSERMSNx}

    Set the MSN here on which calls should be picked up.

\config{VBOX\_USER\_x\_MSN\_x\_START}{VBOX\_USER\_x\_MSN\_x\_START}{VBOXUSERMSNxSTART}

    Default Setting: \var{VBOX\_USER\_x\_MSN\_x\_START}='yes'

    Set this variable to specify if VBOX should be activated on boot for this MSN.
    This variable is optional. If omitted VBOX is activated on boot here. 

\config{VBOX\_USER\_x\_BEEP}{VBOX\_USER\_x\_BEEP}{VBOXUSERxBEEP}

    Default Setting: \var{VBOX\_USER\_x\_BEEP}='no'

    If an acoustic signal should signal new messages for this user this parameter has
    to be set to 'yes', if not, to 'no'.
    
    \wichtig{Note the following for this parameter: If the spool directories are situated
    on a harddisk, the disk may not enter idle mode with vboxbeep activated even if this
    is specified in the configuration of package HD because the vboxbeep daemon accesses
    the disk permanently to check for new messages.}

\config{VBOX\_USER\_x\_STD\_RINGDEF\_N}{VBOX\_USER\_x\_STD\_RINGDEF\_N}{VBOXUSERxSTDRINGDEFN}

    VBOX can pick up calls after a defined number of ringtones depending on the time. This
    variable defines the number of time ranges to be defined.

\config{VBOX\_USER\_x\_STD\_RINGDEF\_x}{VBOX\_USER\_x\_STD\_RINGDEF\_x}{VBOXUSERxSTDRINGDEFx}

    Syntax: \var{VBOX\_USER\_x\_STD\_RINGDEF\_x}=\\'TIME DAYS RINGS'

    The parameters in \var{VBOX\_USER\_x\_STD\_RINGDEF\_x} are separated by one or more spaces.
    Time ranges define daytimes (TIME), weekdays (DAYS) and after how may ring tones (RINGS)
    a call is picked up in the time period. A detailed description of time formats can be found
    in the appendix of this documentation. By the way, it is not possible to let vbox pick up
    a call before the first ring. A value of '0' for RINGS disables pick up completely.

    These settings are defaults to be used if no special treatment is defined for a caller (see
    \var{VBOX\_USER\_PROFILE\_x}), or a special treatment has no definition for RINGS.

    You may define, for example, to pick up a call after the first ring in the night but after
    the fifth ring at all other times. If callers exist that could be of such importance that
    you want to be woken at night in any case you may define a specific caller section via
    \var{VBOX\_USER\_PROFILE\_x} to overwrite the defaults from \var{VBOX\_USER\_x\_STD\_RINGDEF\_x}.
    
\end{description}

\subsubsection{User Specific Settings - Time Dependant Programming}

\begin{description}

\config{VBOX\_USER\_x\_STD\_SECDEF\_N}{VBOX\_USER\_x\_STD\_SECDEF\_N}{VBOXUSERxSTDSECDEFN}

    VBOX may not only pick up calls after different ringtone counts depending on time, but also
    decide which announcement should be played, how long the recording time should be, and so on.
    This variable specifies the number of time ranges to be defined in the following sections.

\config{VBOX\_USER\_x\_STD\_SECDEF\_x}{VBOX\_USER\_x\_STD\_SECDEF\_x}{VBOXUSERxSTDSECDEFx}

    Syntax: \var{VBOX\_USER\_x\_STD\_SECDEF\_x}=\\'TIME DAYS MESSAGE RECTIME [FLAG] [...]'
    
    The parameters in \var{VBOX\_USER\_x\_STD\_SECDEF\_x} are separated by at least one space character.
    They contain the settings for the 'STANDARD' section that always has to exist to define how
    callers should be treated for which no caller section exists or that do not transfer their ID.
    
    The specific paramenters are explained below.

    
    \var{TIME}
    
    The times when the settings should be used.

    
    \var{DAYS}
    
    The days when the settings should be used.

    
    \var{MESSAGE}
    
    Announcement for the caller. A message by this name has to be copied to
    /opt/etc/vbox/\\messages resp. config/etc/vbox/messages prior to creation of the archives.
    Spaces and tabs are not allowed. Even when using a .wav file the entry has to be defined
    with a .msg suffix. The .wav file is then copied and will be converted to a .msg file
    at boot time. The program \var{sox} needed for the conversion will be copied to the router
    automatically. The variable \var{VBOX\_WAV\_FILES} from older vbox versions hence is not
    needed anymore.

    \var{RECTIME}
    
    Maximum number of seconds used for recording. Default is 60 seconds.


    \var{FLAG}
    
    Additional flags that can be specified:
    \begin{itemize}
        \item \var{NOANSWER}: The call should not be answered at all.
        \item \var{NORECORD}: No message should be recorded.
        \item \var{NOTIMEOUTMSG}: No timeout announcement should be played.
        \item \var{NOBEEPMSG}: No Beep should be emerged.
        \item \var{NOSTDMSG}: No announcement should be played.
        \item \var{RINGS}= : Specifies the number of RING's before picking the call up. 
            This flag overwrites the RING's from section [RINGS] and the flag TOLLRINGS.
        \item \var{TOLLRINGS}= : Specifies the number of RING's before picking the call up 
            if new messages are present. This flag overwrites the RING's from section
            [RINGS] and the flag RINGS for new messages.
    \end{itemize}
\end{description}

\subsubsection{User Specific Settings - Caller Specific Configuration}

\begin{description}

\config{VBOX\_USER\_x\_PROFILE\_N}{VBOX\_USER\_x\_PROFILE\_N}{VBOXUSERxPROFILEN}

    Specifies the number of caller IDs to be assigned to a specific section.

\config{VBOX\_USER\_x\_PROFILE\_x}{VBOX\_USER\_x\_PROFILE\_x}{VBOXUSERxPROFILEx}

    Syntax: \var{VBOX\_USER\_x\_PROFILE\_x}=\\'CALLERID SECTION\_NAME DESCRIPTION'

    If VBOX recognizes a call with caller Id submission it checks for a specific 
    treatment defined for the ID. With this variable the caller gets assigned
    a section\\(see \var{VBOX\_USER\_x\_STD\_SECDEF\_x}).
    
    The first entry in each line is the caller ID number [CALLERID] with area code,
    but without a prefixing zero. Also caller ID ranges may be set here by the help
    of 'Unix-Patterns'. After some testing I can verify the following Unix-Patterns
    as operational:
    
    i.e. numbers from 55511 up to 55514: '5551[1-4]'
    
    Numbers starting with the same digits, i.e. all numbers starting with '555': '555*'
    
    You may also use the wildcard \var{PHONEBOOK}. In this case all caller IDs stored in /etc/phonebook
    are assigned to the section [SECTION\_NAME]. Obviously you will need an installed OPT phonebook on
    the router. Set \var{OPT\_TELMOND}='yes' in config/isdn.txt in addition.

    The second entry [SECTION\_NAME] is the name of the caller section and can be defined freely.
    It defines VBOX's behaviour if a call from [CALLERID] is recognized. A '-' stands for the caller
    section 'STANDARD' (\var{VBOX\_USER\_x\_STD\_SECDEF\_x}), a '*' for a caller section assgned to
    the name of the caller provided in [DESCRIPTION]. This entry may include spaces and allows that
    the WEB Gui also displays the name of the caller in the list of calls received and not only its
    number. Technically, this entry is therefore meaningless.

    \wichtig{For all callers for which no \var{VBOX\_USER\_x\_PROFILE\_x} has been defined or which
        simply submit no phone number should always the following be entered as a last
	\var{VBOX\_USER\_x\_PROFILE\_x}-line:}
        \begin{verbatim}
            '*    -    --- Unknown ---'
        \end{verbatim}
    Only then these calls will be answered on the section 'STANDARD'.
    (Instead of '--- Unknown ---' anything else may be specified.)



\config{VBOX\_USER\_x\_SECTION\_N}{VBOX\_USER\_x\_SECTION\_N}{VBOXUSERxSECTIONN}

    Define the number of sections here in which VBOX's behaviour will differ from
    \jump{VBOXUSERxSTDSECDEFx}{\var{VBOX\_USER\_x\_STD\_SECDEF\_x}} and \jump{VBOXUSERxSTDRINGDEFx}{\var{VBOX\_USER\_x\_STD\_RINGDEF\_x}}.
    
\config{VBOX\_USER\_x\_SECTION\_x\_NAME}{VBOX\_USER\_x\_SECTION\_x\_NAME}{VBOXUSERxSECTIONxNAME}

    The name of the section to be matched with\\ \jump{VBOXUSERxPROFILEx}{\var{VBOX\_USER\_x\_PROFILE\_x}}.

\config{VBOX\_USER\_x\_SECTION\_x\_SECDEF\_N}{VBOX\_USER\_x\_SECTION\_x\_SECDEF\_N}{VBOXUSERxSECTIONxSECDEFN}

    Number of time ranges for section definitions (as in \jump{VBOXUSERxSTDSECDEFN}{\var{VBOX\_USER\_x\_STD\_SECDEF\_N}})

\config{VBOX\_USER\_x\_SECTION\_x\_SECDEF\_x}{VBOX\_USER\_x\_SECTION\_x\_SECDEF\_x}{VBOXUSERxSECTIONxSECDEFx}

    Syntax: \var{VBOX\_USER\_x\_SECTION\_x\_SECDEF\_x}=\\'TIME DAYS MESSAGE RECTIME [FLAG] [...]'

    As in \jump{VBOXUSERxSTDSECDEFx}{\var{VBOX\_USER\_x\_STD\_SECDEF\_x}} settings for the behaviour of VBOX may be stored here.
    Configuration is identical to \jump{VBOXUSERxSTDSECDEFx}{\var{VBOX\_USER\_x\_STD\_SECDEF\_x}}
    and can be reviewed there.

\end{description}

\subsection{Configuration Example}

\begin{verbatim}
OPT_VBOX='yes'                           # VBOX is activated
VBOX_SPOOLPATH=''                        # Messages are stored in Ramdisk
VBOX_SPOOLDIR_SPACE='4000'               # 4000 KiB are used for the Ramdisk
VBOX_DELETE_OLD_SPOOLDIRS='yes'          # of no interest for Ramdisk
VBOX_COMPRESSION='ulaw'                  # ulaw compression is used
VBOX_FREESPACE='8192'                    # A minimum of 8MiB of free storage has to exist
                                         # for messages to be recorded
VBOX_LOGPATH='/var/log/vbox'             # Logfiles are stored in /var/log/vbox
VBOX_DEBUGLEVEL='FE'                     # Log only errors

VBOX_USER_N='1'                          # Only one user
VBOX_USER_1_USER='user1'                 # User name
VBOX_USER_1_PASS='pass1'                 # Password
VBOX_USER_1_MSN='1234'                   # Only one MSN is monitored

# Pick up calls after four rings at all times.
VBOX_USER_1_STD_RINGDEF_N     = '1' #TIME         DAYS        RINGS
VBOX_USER_1_STD_RINGDEF_1     =     '*            *           4'

# Play the standard announcementat all times and allow a recording time of 60 seconds.
VBOX_USER_1_STD_SECDEF_N  = '1'  #TIME    DAYS    MESSAGE        RECTIME [FLAG] [...]
VBOX_USER_1_STD_SECDEF_1  =      '*       *       standard.msg   60'

# All callers are assigned to the standard section
VBOX_USER_1_PROFILE_N  = '1'  #CALLERID    SECTION_NAME     DESCRIPTION
VBOX_USER_1_PROFILE_1  =      '*           -                -- unkown --'

# No specific sections defined
VBOX_USER_1_SECTION_N          = '0'  #SECTION for VBOX_USER_x_PROFILE_x
VBOX_USER_1_SECTION_1_NAME     = ''
VBOX_USER_1_SECTION_1_SECDEF_N = '0'  #TIME DAYS MESSAGE 	RECTIME [FLAG] [...]
VBOX_USER_1_SECTION_1_SECDEF_1 = ''

\end{verbatim}    

\subsection{Recording And 'Installing' New Messages}

The messages to be used in vbox-configuration have to be copied to the following directory:

opt/etc/vbox/messages.

As an alternative messages may also be stored in the configuration directory:

config/etc/vbox/messages.

The latter is very convenient, because with a router update to a new version
only the configuration directory needs to be copied.

In these directories any number of messages may be stored but only those
are added to the opt archive which are used in the configuration.

In order to operate vbox without problems the package provides a somewhat neutral
standard message you certainly would like to replace with your own announcements.
The sounds for 'beep.msg' and 'timeout.msg' may be replaced as well. However, it
is not recommended to use a tone that differs much from the known normal beeping
of answering machines, because many callers will be confused and never leave a message.

If the messages in the folders are not in ulaw, but in .au or .wav format, they are
automatically converted at the start of the router. To convert from .wav format
a tool is copied to the opt-archive that needs an additional 180KiB.

Finally, also with a computer without a sound card announcements may be recorded
by installing vbox at first and call yourself. The data stored on the router is a
message in the correct format and may be used. You may easily load the message
from the router by the help of a SCP or SFTP program.

If the messages are stored in the Ramdisk they may be found under:

/var/spool/vbox/<username>/incoming

In other cases they can be found in the directory on the harddisk you set above.

The recorded messages are stored as .msg files with rather cryptic names. 
(To know which message is in which file you should note the sequence in which the
messages were recorded.) Rename the file to something more intuitive and 
use them in your VBOX configuration then.

\subsection{For The Future}

basically I only adapted the VBOX package by Christph Peus for fli4l version 3.0.0.
I did not pay big attention to the size of the binaries which are significantly larger
than those of older versions. If I find the time and there is a need for it I will take care
of that. New features planned are sending messages via mail and remote access. (Helmut Hummel)

\subsection{Support}

It is obvious but I mention it nevertheless: 
If you have problems with the configuration or you believe you found a bug please read 
the documentation to ensure you did not oversee anything. When using complex parameters
the danger is great to have produced some errors. Maybe the original documentation
(Appendix) may be of help.

If you can't find the culprit post your question to the newsgroup spline.fli4l.opt with
a detailed description of the problem, most efficient is an excerpt from the log file. 
You  may find it in the directory specified above. 
Have fun!
