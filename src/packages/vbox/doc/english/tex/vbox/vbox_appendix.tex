% Synchronized to r29821
\section {Appendix For Package OPT\_VBOX}

\subsection{Time Settings Format Used For Configuration}

For some parameters, time and day are expected. The following
information about the format is taken from the original documentation:

\subsubsection{Time Settings}
As of vbox version v2.0.0 different time zones (eg with rings) are provided
with minute digits. Individual time entries are separated by commas, details
of start and end time by a minus sign. The hours must be given in 24-hour format
i.e. from 0 to 23 o'clock.

Supplement to the original document:
Time ranges that exceed 00:00 clock must always be defined in two parts up to
00:00 and from 00:00 on.
Example: instead of 22-06 write 22-23,00-06.

The times are internally always converted to start and end time, even
if only the start time is specified.

Take for example the following times:

20:15-21:14
This times are internally converted to 20:15:00 and to 21:14:59, ie start and
stop time will be included!
If a time specification has no minutes defined it gets the starting time
0 minutes, and for the stop time 59 minutes will be used. Internally the seconds
- which are not adjustable - are treated according to the same scheme.

Example:
\begin{itemize}
    \item 20 - Conversion to 20:00:00-20:59:59
    \item 20:15-21:14 - Conversion to 20:15:00-21:14:59
    \item 08-11 - Conversion to 08:00:00-11:59:59
    \item 12-15:30 - Conversion to 12:00:00-15:30:59
\end{itemize}

A time zone is true (matches) if the current time is greater than/equal to
the start time and less than/equal to the end time.

A '*' as the only time specification is treated as 'always ', a '-' or '!'
as the only time specification as 'never.

\subsubsection{Day Values}
Individual day values are separated by commas.
An indication of start and end times is not possible here.

The following shortcuts can be specified for days:
\begin{itemize}
    \item MO, MON - for Monday
    \item DI, TUE - for Tuesday
    \item MI, WED - for Wednesday
    \item DO, THU - for Thursday
    \item FR, FRI - for Friday
    \item SA, SAT - for Saturday
    \item SO, SUN - for Sunday
\end{itemize}

Example:
MON,TUE,WED,FRI,SAT,SUN

\subsection{Package History}
Originally this package was made by Christoph Peus for fli4l versions 1.x.x and 2.0.x. It was changed
for version 2.1.x by Gerd Walter. Arno Wetzel made the Web GUI that made the installation of vbox-clients
on other computers obsolete. I got the package from Christoph Schulz who compiled it for version 2.1.10
and built a small rudimentary remote access function which unfortunately never really worked for me. With the
stable fli4l version 3.0.0 I was motivated to renovate the package to provide a solid base for the new
release and to include some of its advanced features.

Helmut Hummel in December 2005

\subsection{The Original VBOX Documentation}

The original (german) VBOX documentation can be found in the appendix of the german documentation.
