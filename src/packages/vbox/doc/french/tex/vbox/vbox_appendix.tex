% Synchronized to r29821
\section {Annexe du paquetage OPT\_VBOX}

\subsection{Format utilisé dans la configuration pour le réglage de l'heure}

Pour certains paramètres, l'heure et le jour dans la configuration est trés importante.
Les informations suivantes sur le format de l'heure et du jour, sont prisent à partir
de la documentation original~:

\subsubsection{Paramètre pour l'heure}
Le fuseaux horaires de vbox est différents depuis la version v2.0.0 (par exemple
pour les sonnerie) il est pouvu d'un affichage en minutes. Le paramètrage de l'heure
est séparé par une virgule, un signe moins est inséré entre le début et la fin de
l'heure. Les heures sont sous le format 24 heures - c'est à dire de 0 h à 23 h.

Supplément au document original~:
Les plages horaires qui dépassent 00:00 heure doivent toujours être paramétrées en deux
parties - de 00:00 jusqu'à 00:00.
Exemple~: au lieu de 22-06 h écrire 22-23,00-06.

Les informations sur les horaires sont toujours convertie en interne avec un début et 
ne fin, même si une seule heure de début est spécifié.

Prenez par exemple l'horaire suivants~:

20:15-21:14
Cette heure est convertis en interne et est équivalent à 20:15:00-21:14:59, c'est à dire que
le début et la fin sont inclus dans la convertion~!
Si vous n'avez pas de minutes dans l'heure qui a été définie, vous allez avoir l'heure de
démarrage à 0 minute et l'heure d'arrêt à 59 minutes. En interne, les secondes ne sont
pas réglables - mais ils sont traités selon le même schéma.

Exemple~:
\begin{itemize}
    \item 20 - après conversion  20:00:00-20:59:59
    \item 20:15-21:14 - après conversion 20:15:00-21:14:59
    \item 08-11 - après conversion 08:00:00-11:59:59
    \item 12-15:30 - après conversion 12:00:00-15:30:59
\end{itemize}

Un fuseau horaire est vrai (correspond) si l'heure actuelle est supérieure/égale à
l'heure de début et inférieure/égale à l'heure de fin.

le paramètre '*' est considérée comme 'toujours', et '!' est considérée comme 'jamais'
dans le réglage de l'heure.

\subsubsection{Paramètre des jours de la semaine}
Les valeurs des jours doivent être séparées par une virgule.
Une indication de début et de fin n'est pas possible ici.

les jours peuvent être spécifiés par un raccourci~:
\begin{itemize}
    \item MO, MON - pour lundi
    \item DI, TUE - pour mardi
    \item MI, WED - pour mercredi
    \item DO, THU - pour jeudi
    \item FR, FRI - pour vendredi
    \item SA, SAT - pour samedi
    \item SO, SUN - pour dimanche
\end{itemize}

Exemple~:
MO,DI,DO,FRI,SAT,SO

\subsection{L'histoire de ce paquetage}
A l'origine les versions 1.xx et 2.0.x de ce paquetage ont été créé par Christoph Peus
pour fli4l. La version 2.1.x a été modifié par Gerd Walter. Arno Wetzel a créé l'interface
Web et créé l'installation d'un client-vbox pour des ordinateurs obsolètes. J'ai utilisé
le paquetage de Christoph Schulz, qui a compilé la version 2.1.10 et construit une petite
fonction d'accès à distance rudimentaire, malheureusement, cette fonction n'a jamais vraiment
fonctionné pour moi. Avec la version stable 3.0.0 de fli4l j'ai été motivé pour rénover
l'ensemble et j'ai fourni une base solide pour la nouvelle version et j'ai inclu certaines de
ses fonctionnalités avancées.

Helmut Hummel Décembre 2005

\subsection{Documentation originale de VBOX}

La documentation originale de VBOX peut être trouvée dans le répertoire deutsch de la documentation.
