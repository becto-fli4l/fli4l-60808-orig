% Synchronized to r30451
\marklabel{sec:opt-vbox}
{
\section {OPT\_VBOX - Répondeur pour fli4l avec ISDN}
}

\subsection{Introduction}

Ce paquetage intègre le logiciel VBOX qui fait fonction de répondeur dans le système
fli4l, il a été écrit par Michael ’Ghandi’ Herold.

Vous pouvez donc utiliser le routeur fli4l comme un répondeur, la fonctionnalité
au-delà de celle d'un répondeur normale est plus que correcte. Il est possible de mettre
en place différentes boîtes vocales prenant les messages pour différents numéros de
téléphone et peut restreindre l'accès à certains utilisateurs. Chaque utilisateur peut
régler les paramètres en fonction de la journée, de la semaine et de l'identification
de l'appelant pour~:

\begin{itemize}
    \item S'il répond à tous les appels
    \item Répondre à un appel après un nombre de sonneries
    \item Si une annonce a été enregistré, celle-ci sera utilisée
    \item Quel signal sonore sera utilisé pour un appel
    \item Si un message vocal a été enregistré ou non
    \item Le réglage du temps pour enregistrer le message vocal
    \item Si un message d'attente est enregistré, lequel sera utilisé
\end{itemize}

\subsection{Comprendre le fonctionnement / les besoins en ressources}

Le système de VBOX est divisé en deux, un serveur qui reçoit les messages et un client
avec lequel vous pouvez écouter et gérer les messages. Les messages peuvent être
récupérées par l'intermédiaire d'une interface web. Pour cela il est nécessaire
d'installer le paquetage httpd.

\subsubsection{Exigences générales pour l'installation}

Une carte ISDN (ou RNIS) est obligatoire, car vbox ne fonctionne pas avec les modems
(et pas non plus via une ligne DSL  si quelqu'un y pense ...) Bien entendu, le paquetage
ISDN est nécessaire pour l'installation du pilote approprié, mais vous ne devez pas définir
de circuit si la carte est utilisée uniquement pour vbox.

Il est fortement recommandé d'utiliser le paquetage httpd, car il est très pratique pour
gérer les messages stockés.

\wichtig{Le Teles 16.3C ne fonctionnera pas avec vbox car le pilote Linux n'est pas
capable de gérer la voix.}

\subsubsection{Condition de stockage des messages avec une installation sur un Ramdisk}

Si les messages entrants sont stockés sur un Ramdisk (ou disque virtuel), le routeur
nécessite au moins 16 Mio de Ram, sinon des messages ne seront pas accessible le temps
d'enregistrement sera trop cours (voir~: \var{VBOX\_COMPRESSION}). Il est clair que les messages
enregistrés dans un Ramdisk seront perdues lors d'une panne de courant ou un redémarrage.
Si vous voulez utiliser vbox à 100 \% vous devez stocker les messages sur un disque dur.
L'avantage de stockage sur un Ramdisk, est que le disque peut rester inactive.

\subsubsection{Conditions de stockage des messages sur une installation HD / CF}

Si vous devez stocker les messages sur un disque dur, vous devez installer le paquetage HD
et activer la variable \var{OPT\_MOUNT}='yes'. Lorsque vous redémarrez le routeur, les paramètres
des utilisateurs seront mis à jour le cas échéant, les utilisateurs existants ne seront plus
automatiquement supprimé après une nouvelle configuration (en option). Un disque dur est également
avantageux lorsque vous avez besoin de différentes annonces vocale ou d'ajouter différent paquetage-OPT
supplémentaires, car l'espace sur le support de boot risque de ne pas suffire.

\subsection{Configuration}
\subsubsection{Paramètres généraux}
\begin{description}

\config{OPT\_VBOX}{OPT\_VBOX}{OPTVBOX}

        Paramètre par défaut~: \var{OPT\_VBOX}='no'

        Si vous indiquez 'yes' dans cette variablele répondeur sera activé.

\config{VBOX\_SPOOLPATH}{VBOX\_SPOOLPATH}{VBOXSPOOLPATH}

    Dans cette variable \var{VBOX\_SPOOLPATH} vous indiquez si les messages enregistrés seront stockés
	sur un Ramdisk ou sur un disque dur.

    Enregistrer les messages sur un Ramdisk~:
    \var{VBOX\_SPOOLPATH}=''
    vous indiquez ici aucun chemin~!

    Enregistrer les messages sur un disque dur~:
    \var{VBOX\_SPOOLPATH}='/data/vbox' (z.B.)
    La condition c'est d'avoir installé le paquetage HD avec \var{OPT\_MOUNT}='yes' activé.

    Le chemin d'accès spécifié ici doit pointer vers une partition sur le disque dur, le système
	de fichiers sera accessible en écriture et monté au format ext 2/3, mais pas avec une installation
	'/opt' (HD-Install Typ B). Si l'une des conditions n'est pas remplie, l'installation de l'ensemble
	de vbox sera interrompu avec un message d'erreur. Si toutes les conditions sont remplies, le
	répertoire \var{vbox\_spooldir} sera créé dans le chemin d'accès spécifié ici pour stocker les
	données de vbox.

    Vous pouvez aussi régler la variable sur 'auto' pour utiliser le chemin d'accès défini par
	\var{FLI4L\_UUID}.

\config{VBOX\_SPOOLDIR\_SPACE}{VBOX\_SPOOLDIR\_SPACE}{VBOXSPOOLDIRSPACE}

    Vous indiquez dans cette variable l'espace en Kio pour toutes les boites vocales estimée.
	Lorsque vous créez le répertoire de spool le système vérifié également si l'espace est
	disponible.

\config{VBOX\_DELETE\_OLD\_SPOOLDIRS}{VBOX\_DELETE\_OLD\_SPOOLDIRS}{VBOXDELETEOLDSPOOLDIRS}

    Paramètre par défaut~: \var{VBOX\_DELETE\_OLD\_SPOOLDIRS}='yes'

    Ce paramètre est pertinent que si les messages entrants sont stockés sur un disque dur. Lorsque
	vous redémarrez, le répertoire utilisateur individuel ne sera pas supprimé dans le spool, avec
	le temps différents utilisateurs s'accumulent, il peut y avoir des répertoires utilisateurs
	dans le spool qui sont maintenus inactif et empêchent la vue d'ensemble, il consomme inutilement
	de l'espace et change fréquemment de configuration. Pour éviter cela, vous pouvez régler ce paramètre
	sur 'yes', de sorte qu'à chaque redémarrage ces utilisateurs inactifs seront effacés automatiquement.

    ATTENTION~: celui qui modifie de nom d'utilisateur ou qui change légèrement le fichier de
	configuration, le répertoire de cet utilisateur risque d'être supprimé automatiquement au prochain
	démarrage et tous les messages de cet utilisateur seront aussi supprimés dans le spoul, car le nom
	d'origine de l'utilisateur n'est plus considéré actif. Vous devez être très prudent, afin d'éviter,
	la suppression automatique des messages par inadvertance.

\config{VBOX\_COMPRESSION}{VBOX\_COMPRESSION}{VBOXCOMPRESSION}

    Paramètre par défaut~: \var{VBOX\_COMPRESSION}='ulaw'

    Dans cette variable vous spécifiez la compression avec laquelle les messages seront enregistrés.
	Plus la compression est élevée, moins vous consommez de mémoire et pire est la qualité du message.
	Si vous utilisez un disque dur pour les messages, vous pouvez en toute sécurité utiliser le mode
	'ulaw' aucune compression se produira. Les messages sont stockés avec toute la bande passante ISDN
	8 Bit * 8kHz = 64 kbit/s. Dans une installation avec un Ramdisk sans disque dur sur un PC et avec
	16 Mio de mémoire et 6 Mio disponible pour les enregistrements la durée d'enregistrement sera d'un
	peu plus de dix minutes.

    Si vous avez besoin d'économiser de l'espace vous pouvez utiliser le format 'adpcm-4'
	(4 Bit * 8kHz = 32kbit/s -> c'est la moitié de mémoire utilisée par rapport à 'ulaw') ou
	'adpcm-3' ou alors 'adpcm-2' vous allez avoir une compression plus élevé mais avec une
	 moins bonne qualité.

    Voici un aperçu~:

    \begin{table}[htbp]
      \begin{tabular}{lrrr}
        Mode     & Résolution & Compression à & stockage nécessaire pour 10 minutes (env.) \\
        \hline \\
        ulaw     & 8Bit             & 100\%           & 4800 kB       \\
        adpcm-4  & 4Bit             &  50\%           & 2400 kB       \\
        adpcm-3  & 3Bit             &  37\%           & 1800 kB       \\
        adpcm-2  & 2Bit             &  25\%           & 1200 kB       \\
      \end{tabular}
      \caption{Comparaison des différentes compressions}
    \end{table}

\config{VBOX\_FREESPACE}{VBOX\_FREESPACE}{VBOXFREESPACE}

    Paramètre par défaut~: \var{VBOX\_FREESPACE}='8192'

    Si vous n'avez pas assé d'octets disponible dans la variable \var{VBOX\_FREESPACE}
	pour le stockage de nouveaux messages, les nouveaux appels ne seront pas acceptées.
	La valeur '0' signifie que ce contrôle sera désactivé.

\config{VBOX\_LOGPATH}{VBOX\_LOGPATH}{VBOXLOGPATH}

    Paramètre par défaut~: \var{VBOX\_LOGPATH}='/var/log/vbox'

    Dans cette variable vous indiquez le répertoire où les fichiers de journalisation quiv
	doivent être écrits.

    Vous pouvez aussi régler la variable sur 'auto' pour utiliser le chemin d'accès défini par
	\var{FLI4L\_UUID}.

\config{VBOX\_USE\_VBOXD}{VBOX\_USE\_VBOXD}{VBOXUSEVBOXD}

    Paramètre par défaut~: \var{VBOX\_USE\_VBOXD}='no'

    Avec cette variable les messages seront accessibles avec un client vbox. Si vous le souhaitez
	il suffit d'indiquer 'yes' dans cette variable, assurez-vous d'indiquer un mot de passe dans
	la variable \var{VBOX\_USER\_x\_VBOXD\_PASSWORD}.

    \wichtig{Le démon vboxd est une appliquation sur le routeur. Il est préférable d'utiliser
	l'interface Web.}

\config{VBOX\_VBOXD\_ALLOW}{VBOX\_VBOXD\_ALLOW}{VBOXVBOXDALLOW}

    Paramètre par défaut~: \var{VBOX\_VBOXD\_ALLOW}='*.lan.fli4l'

    Dans la variable \var{VBOX\_VBOXD\_ALLOW} vous pouvez indiquer les ordinateurs qui seront
	autorisés à lire ou à gérer les messages avec le client vbox. le paramètre par défaut est
	'*.lan.fli4l' pour tous les ordinateurs dans le domaine DNS. Des restrictions peuvent être
	apportées en spécifiant unique les adresses IP, des noms d'hôtes ou des noms de domaine
	(comme *.home.lan). Si vous indiquez plusieurs paramètres, ils doivent être séparé par un espace.

    \wichtig{Si vous indiquez un nom d'hôte de votre propre domaine vous ne devez pas complétement
	paramétrer le nom du DNS~!}

    \wichtig{Pour minimiser le risque d'une attaque potentielle sur le routeur avec le démon
	vboxd, tenir ce paramètre aussi restrictif que possible. Si par exemple VBOX\_BEEP est nécessaire\\
	La variable VBOX\_VBOXD\_ALLOW doit être vide pour restreindre l'accès du routeur vbox.}

\config{VBOX\_BEEP\_HOURS}{VBOX\_BEEP\_HOURS}{VBOXBEEPHOURS}

    Paramètre par défaut~: \var{VBOX\_BEEP\_HOURS}='*'

    Dans cette variable vous pouvez indiquez une heures pour faire retentir un signal sonore.
	Vous pouvez définir la plage horaire de~: 8-24, plusieurs plages peuvent être indiquées vous
	devez les séparées par un espace. Si vous indiquez '*' cela signifie 'toujours'. Une description
	détaillée des formats de l'heure peut être trouvée dans l'annexe de cette documentation.

    \wichtig{Pour faire fonctionner la variable VBOX\_BEEP, vous devez activer la variable
	VBOX\_USE\_VBOXD='yes'.}

\config{VBOX\_BEEP\_PAUSE}{VBOX\_BEEP\_PAUSE}{VBOXBEEPPAUSE}

    Paramètre par défaut~: \var{VBOX\_BEEP\_PAUSE}='60'

    Dans cette variable vous indiquez le temps en seconde entre chaque signal sonore.

\config{VBOX\_DEBUGLEVEL}{VBOX\_DEBUGLEVEL}{VBOXDEBUGLEVEL}

    Paramètre par défaut~: \var{VBOX\_DEBUGLEVEL}='FE'

    Dans cette variable vous indiquez par une combinaison de lettres les événements qui seront
	enregistrés dans un fichier journal. Pour la documentation originale~:
    \begin{itemize}
        \item F - Les erreurs qui ne peuvent être corrigée
        \item E - Les erreurs qui peuvent être corrigées
        \item W - Les Avertissements
        \item I - Les Informations
        \item D - Les Problèmes de débogage
        \item J - Encore plus pour les problèmes de débogage
    \end{itemize}

    Le fichier journal est très utile pour trouver des erreurs. Pour commencer, vous pouvez
	indiquer toutes les valeurs, quand vous ètes sûr que tout va bien, 'FE' doit être suffisant. 

\config{VBOX\_ADMIN\_USERNAME}{VBOX\_ADMIN\_USERNAME}{VBOXADMINUSERNAME}

    Dans cette variable vous indiquez le nom d'utilisateur de l'administrateur, il est peut être
	déjà défini dans la configuration httpd (les majuscules sont prises en compte). L'utilisateur
	pourra examiner toutes les boîtes vocales VBOX dans le WebGUI (ou l'interface Web) il est donc
	en mesure de démarrer, arrêter, de lire les messages de toutes les boîtes.

    \wichtig{L'utilisateur peut avoir tout les droits avec 'vbox:all'.}

\end{description}

\subsubsection{Paramètres utilisateur spécifique}

\begin{description}

\config{VBOX\_USER\_N}{VBOX\_USER\_N}{VBOXUSERN}

    Dans cette variable vous indiquez le nombre d'utilisateurs qui doivent posséder une boite vocal.

\config{VBOX\_USER\_x\_USERNAME}{VBOX\_USER\_x\_USERNAME}{VBOXUSERUSERNAME}

	Dans cette variable vous indiquez le nom de l'utilisateur spécifique à la boite. Ce nom
	d'utilisateur est également utilisé pour s'authentifier dans l'interface Web. Si ce nom
	d'utilisateur existe déjà dans la configuration du démon httpd (les majuscules sont prises
	en compte) les droits spécifiés sont appliqués (voir la documentation du paquetage httpd).
	Si le nom d'utilisateur n'est pas défini, l'utilisateur a les droits d'accés seulement pour
	la page VBOX dans l'interface Web.

\config{VBOX\_USER\_x\_PASSWORD}{VBOX\_USER\_x\_PASSWORD}{VBOXUSERPASSWORD}

    Dans cette variable vous indiquez le mot de passe de l'utilisateur. Si le nom d'utilisateur
	dans \var{VBOX\_USER\_x\_USERNAME} existe également dans la configuration du paquetage httpd
	avec son mot de passe, le contenu de la variable \var{VBOX\_USER\_x\_PASSWORD} n'a pas de
	sens. Dans tous les autres cas, ce mot de passe est utilisé pour s'authentifier sur l'interface
	Web.

\config{VBOX\_USER\_x\_VBOXD\_PASSWORD}{VBOX\_USER\_x\_VBOXD\_PASSWORD}{VBOXUSERxVBOXDPASSWORD}

    Dans cette variable vous pouvez indiquer un mot de passe pour vboxd. Il sera utilisé uniquement
	pour la connexion avec un client vbox (pas avec l'interface Web).

\config{VBOX\_USER\_x\_MSN\_N}{VBOX\_USER\_x\_MSN\_N}{VBOXUSERMSNN}

    Dans cette variable vous indiquez le nombre de numéro de téléphone à surveiller.

\config{VBOX\_USER\_x\_MSN\_x}{VBOX\_USER\_x\_MSN\_x}{VBOXUSERMSNx}

    Dans cette variable vous indiquez le numéro de téléphone qui sera autorisé pour les appels
	téléphonique.

\config{VBOX\_USER\_x\_MSN\_x\_START}{VBOX\_USER\_x\_MSN\_x\_START}{VBOXUSERMSNxSTART}

    Paramètre par défaut~: \var{VBOX\_USER\_x\_MSN\_x\_START}='yes'

    Dans cette variable vous pouvez spécifier si VBOX doit être activé au démarrage pour ce numéro
	de téléphone. Cette variable est optionnelle. Si vous n'indiquez rien VBOX démarre qu'en même
	avec ce numéro.

\config{VBOX\_USER\_x\_BEEP}{VBOX\_USER\_x\_BEEP}{VBOXUSERxBEEP}

    Paramètre par défaut~: \var{VBOX\_USER\_x\_BEEP}='no'

    Si vous souhaitez un signal sonore lorsque vous recevez un nouveau message pour cet utilisateur
	vous devez paramètrer la variable sur 'yes', sinon, laissez sur 'no'.

    \wichtig{Notez le point suivant pour ce paramètre~: si le répertoire du spool est situé sur un
	disque dur, le disque ne pourra pas entrer en mode veille avec vboxbeep activé même si cela est
	spécifié dans la configuration du package HD, parce que le démon vboxbeep accède au disque en
	permanence pour vérifier les nouveaux messages.}

\config{VBOX\_USER\_x\_STD\_RINGDEF\_N}{VBOX\_USER\_x\_STD\_RINGDEF\_N}{VBOXUSERxSTDRINGDEFN}

    VBOX peut prendre les appels après un nombre défini de sonneries et en fonction de l'heure
	Dans cette variable vous définissez le nombre de plages horaires.

\config{VBOX\_USER\_x\_STD\_RINGDEF\_x}{VBOX\_USER\_x\_STD\_RINGDEF\_x}{VBOXUSERxSTDRINGDEFx}

    Syntaxe~: \var{VBOX\_USER\_x\_STD\_RINGDEF\_x}='TIME DAYS RINGS'

    Les paramètres de la variable \var{VBOX\_USER\_x\_STD\_RINGDEF\_x} doivent être séparés par
	un ou plusieurs espaces. La plages horaire peut être définie à certain moment de la journée
	(TIME), à certain jour de la semaine (DAYS), à un certain nombre de sonneries (RINGS), un appel
	avec ces paramètre sera accepté. Une description détaillée des formats d'heure peut être trouvée
	dans l'annexe de cette documentation. En outre, dans cette variable il n'est pas possible que
	vbox reponde à un appel à la première sonnerie. Avec la valeur '0' pour RINGS vbox ne repondra
	pas du tout à l'appel.

    Ces paramètres sont des paramètres par défaut, si vous n'utilisez aucun traitement spécial pour
	la configuration (voir \var{VBOX\_USER\_PROFILE\_x}), ou si vous utilisez un traitement spécial
	vous ne devez pas paramétrer RINGS.

    Par exemple vous pouvez définir, que vbox prendra l'appel après la première sonnerie dans la
	nuit, mais dans les autres moments vbox prend les appels après la cinquième sonnerie. Si vous
	avez vraiment des appels téléphonique important la nuit et que vous voulez dormir, vous pouvez
	définir une section d'appel spécifique via \var{VBOX\_USER\_PROFILE\_x} pour écraser les paramètres
	par défaut de \var{VBOX\_USER\_x\_STD\_RINGDEF\_x}.

\end{description}

\subsubsection{Paramètres utilisateur spécifique - Dépend du temps de programmation}

\begin{description}

\config{VBOX\_USER\_x\_STD\_SECDEF\_N}{VBOX\_USER\_x\_STD\_SECDEF\_N}{VBOXUSERxSTDSECDEFN}

    VBOX peut non seulement prendre les appels à tout moments de la journée, après un nombre de
	sonnerie défini, il peut aussi décider du message à diffusé, de régler le temps d'enregistrement,
	et d'autre chose. Avec cette variable vous pouvez indiquer le nombre de plage horaire et des
	autres paramètres avec la variable suivante.

\config{VBOX\_USER\_x\_STD\_SECDEF\_x}{VBOX\_USER\_x\_STD\_SECDEF\_x}{VBOXUSERxSTDSECDEFx}

    Syntaxe~: \var{VBOX\_USER\_x\_STD\_SECDEF\_x}='TIME DAYS MESSAGE RECTIME [FLAG] [...]'

    Les paramètres de la variable \var{VBOX\_USER\_x\_STD\_SECDEF\_x} doivent être séparés par
	un ou plusieurs espaces. Il contient les paramètres de la section 'STANDARD', ils doivent
	toujours être présent, car ils définissent comment les appels doivent être traités, si ils sont
	manquant aucune section des appels ne sera disponible ou aucune section des appels pourra
	être définie, car le numéro de téléphone ne sera pas transmis.

    Les paramenters spécifiques sont décrites ci-dessous.


    \var{TIME}

    Les horaires au cours des quelles les paramètres doivent être utilisés.


    \var{DAYS}

    Les jours au cours des quelles les paramètres doivent être utilisés.


    \var{MESSAGE}

    Le message d'annonce du répondeur. Le nom du fichier pour le message d'annonce doit être
	copié dans le répertoire opt/etc/vbox/messages ou dans config/etc/vbox/messages. Les espaces
	et les onglets ne sont pas autorisés. Même si vous utilisez un fichier wav, ce fichier sera
	transformé en .msg à l'installation. Le fichier wav qui est copié sera converti en un fichier
	.msg au moment du démarrage. Le programme \var{sox} est nécessaire pour la conversion,
	il sera copié sur le routeur automatiquement. La variable \var{VBOX\_WAV\_FILES} qui faisait
	parti des versions antérieures de vbox n'est plus nécessaire.


    \var{RECTIME}

    Nombre de secondes d'enregistrement maximum. La valeur par défaut est de 60 secondes.


    \var{FLAG}

    D'autres flags (ou indicateurs) peuvent être définis~:
    \begin{itemize}
        \item \var{NOANSWER}~: Ne repondra pas à d'appel tél.
        \item \var{NORECORD}~: Tous les messages ne seront pas enregistrés.
        \item \var{NOTIMEOUTMSG}~: Il n'y aura aucun message d'attente tél.
        \item \var{NOBEEPMSG}~: Il n'y aura aucun signal sonore.
        \item \var{NOSTDMSG}~: Il n'y aura aucune annonce tél délivrée.
        \item \var{RINGS}=~: Indique le nombre de sonnerie avant de répondre à un appel.
            Ce flag écrase la sonnerie dans la section [RINGS] et le flag TOLLRINGS.
        \item \var{TOLLRINGS}=~: Indique le nombre de sonnerie avant de prendre l'appel,
			si un nouveau message est disponible. Ce flag écrase la sonnerie dans la section
			[RINGS] et le flag RINGS pour les nouveaux messages.
    \end{itemize}
\end{description}

\subsubsection{Paramètres utilisateur spécifique - Configuration pour des appels spécifique}

\begin{description}

\config{VBOX\_USER\_x\_PROFILE\_N}{VBOX\_USER\_x\_PROFILE\_N}{VBOXUSERxPROFILEN}

    Dans cette variable vous indiquez le nomble de numéro de téléphone qui doit être configuré
	dans la section spécifique.

\config{VBOX\_USER\_x\_PROFILE\_x}{VBOX\_USER\_x\_PROFILE\_x}{VBOXUSERxPROFILEx}

    Syntaxe~: \var{VBOX\_USER\_x\_PROFILE\_x}='CALLERID SECTION\_NAME DESCRIPTION'

    Si un appel est transmis VBOX vérifie si ce numéro de téléphone a un traitement spécial de
	prévu. Avec cette variable vous indiquez le numéro de l'appelant qui ensuite sera assigner
	à une section. (voir \var{VBOX\_USER\_x\_STD\_SECDEF\_x}).

    Le premier paramètre de chaque ligne est le numéro de téléphone de l'appelant [CALLERID]
	AVEC CODE RÉGION, MAIS SANS LE ZÉRO. Vous pouvez indiquer une plage de numéros avec l'aide
	d'un modèle-Unix. Après avoir testé plusieurs modèle-Unix avec des degrés de succès divers,
	les méthodes suivantes pour une plage de numéros semblent fonctionner~:

    Plusieurs numéros adjacents, par exemple de 55511 à 55514~: '5551[1-4]'

    Les numéros commençant par les mêmes chiffres, par exemple avec le 555xx~: '555*'

    Vous pouvez également utiliser les caractères génériques avec \var{PHONEBOOK}. Tous
	les numéros de téléphone qui sont stockés dans le répertoire /etc/phonebook seront
	assignés dans la section [SECTION\_NAME]. Cela nécessite bien sûr que l'annuaire
	téléphonique soit présent sur le routeur. Par conséquent, dans le fichier /config/isdn.txt
	il est nécessaire que la variable \var{OPT\_TELMOND}='yes' soit activée.

    Le deuxième paramètre [SECTION\_NAME] est le nom de la section de l'appelant, il peut
	être définie librement. Il définit le comportement de VBOX si le numéro d'appel est reconnu
	à partir de [CALLERID]. Le symbole '-' est synonyme à l'appelant dans la section 'STANDARD' de
	la variable (\var{VBOX\_USER\_x\_STD\_SECDEF\_x}), le symbole '*' est synonyme à l'appelant
	donc le nom de l'appelant peut être spécifié dans la section [DESCRIPTION]. Ce paramètre peut
	contenir des espaces et permet d'afficher également le numéro de téléphone mais aussi le nom de
	l'appelant dans la liste des appels reçus sur l'interface Web. Techniquement, cet article est
	donc dénué de sens.

    \wichtig{Pour tous les appelants, qu'ils n'ont pas été définis dans la variable \var{VBOX\_USER\_x\_PROFILE\_x}
	ou le numéro de téléphone de appelant qui a été oublié ou perdu, vous pouvez l'indiquer dans
	\var{VBOX\_USER\_x\_PROFILE\_x} et l'enregistrer en dernière ligne comme ceci~:}
		\begin{verbatim}
            '*    -    --- Unknown ---'
        \end{verbatim}
	Ainsi le répondeur pourra déchocher à l'appel en utilisant la section 'STANDARD'.
	(Au lieu de '--- Unknown ---' vous pouvez spécifier un autre nom.)

\config{VBOX\_USER\_x\_SECTION\_N}{VBOX\_USER\_x\_SECTION\_N}{VBOXUSERxSECTIONN}

    Dans cette variable vous indiquez le nombre de comportement de VBOX pour les sections
    \jump{VBOXUSERxSTDSECDEFx}{\var{VBOX\_USER\_x\_STD\_SECDEF\_x}} et \jump{VBOXUSERxSTDRINGDEFx}{\var{VBOX\_USER\_x\_STD\_RINGDEF\_x}}.

\config{VBOX\_USER\_x\_SECTION\_x\_NAME}{VBOX\_USER\_x\_SECTION\_x\_NAME}{VBOXUSERxSECTIONxNAME}

    Dans cette variable vous indiquez le nom qui sera associé à la section
	\jump{VBOXUSERxPROFILEx}{\var{VBOX\_USER\_x\_PROFILE\_x}}.

\config{VBOX\_USER\_x\_SECTION\_x\_SECDEF\_N}{VBOX\_USER\_x\_SECTION\_x\_SECDEF\_N}{VBOXUSERxSECTIONxSECDEFN}

    Dans cette variable vous indiquez le nombre de plages horaires pour la section
	(comme dans \jump{VBOXUSERxSTDSECDEFN}{\var{VBOX\_USER\_x\_STD\_SECDEF\_N}})

\config{VBOX\_USER\_x\_SECTION\_x\_SECDEF\_x}{VBOX\_USER\_x\_SECTION\_x\_SECDEF\_x}{VBOXUSERxSECTIONxSECDEFx}

    Syntaxe~: \var{VBOX\_USER\_x\_SECTION\_x\_SECDEF\_x}='TIME DAYS MESSAGE RECTIME [FLAG] [...]'

    Comme dans \jump{VBOXUSERxSTDSECDEFx}{\var{VBOX\_USER\_x\_STD\_SECDEF\_x}} vous pouvez indiquer ici
	les paramètres pour le comportement de VBOX. La configuration est identique à
	\jump{VBOXUSERxSTDSECDEFx}{\var{VBOX\_USER\_x\_STD\_SECDEF\_x}} que vous pouvez revoir.
\end{description}

\subsection{Exemple de configuration}

\begin{verbatim}
OPT_VBOX='yes'                           # VBOX est activé
VBOX_SPOOLPATH=''                        # Les messages sont stockés dans le Ramdisk
VBOX_SPOOLDIR_SPACE='4000'               # 4000 Kio sont utilisés pour le Ramdisk
VBOX_DELETE_OLD_SPOOLDIRS='yes'          # Sans intérêt pour Ramdisk
VBOX_COMPRESSION='ulaw'                  # Compression ulaw est utilisé
VBOX_FREESPACE='8192'                    # Un minimum de 8 Mio de stockage est utilisé
                                         # pour enregistrer les messages sur le répondeur
VBOX_LOGPATH='/var/log/vbox'             # Enregistre le fichier journal dans /var/log/vbox
VBOX_DEBUGLEVEL='FE'                     # Seulement les erreurs dans le fichier journal 

VBOX_USER_N='1'                          # Seulement un utilisateur
VBOX_USER_1_USER='user1'                 # Nom de l'utilisateur
VBOX_USER_1_PASS='pass1'                 # Mot de passe
VBOX_USER_1_MSN='1234'                   # Seul un numéro est surveillé

# Accepter tous les appels n'importe quand après 4 sonneries.
VBOX_USER_1_STD_RINGDEF_N  = '1' #TIME         DAYS        RINGS
VBOX_USER_1_STD_RINGDEF_1  =     '*            *           4'

# Annonce vocale à tout moment, avec un temps d'enregistrement de 60 secondes.
VBOX_USER_1_STD_SECDEF_N  = '1'  #TIME        DAYS        MESSAGE             RECTIME [FLAG] [...]
VBOX_USER_1_STD_SECDEF_1  =      '*           *           standard.msg        60'

# Tous les appels sont affectés à une section standard
VBOX_USER_1_PROFILE_N  = '1'  #CALLERID    SECTION_NAME     DESCRIPTION
VBOX_USER_1_PROFILE_1  =      '*           -                -- unkown --'

# Aucune section spéciale définie
VBOX_USER_1_SECTION_N          = '0'  #SECTION for VBOX_USER_x_PROFILE_x
VBOX_USER_1_SECTION_1_NAME     = ''
VBOX_USER_1_SECTION_1_SECDEF_N = '0'  #TIME        DAYS        MESSAGE             RECTIME [FLAG] [...]
VBOX_USER_1_SECTION_1_SECDEF_1 = ''

\end{verbatim}

\subsection{Enregistrer et 'installer' un nouveau message sur le répondeur}

Les messages vocales qui sont utilisés par vbox doivent être copiés dans le répertoire fli4l suivant~:

opt/etc/vbox/messages

une autre alternative peut également être utilisé pour stocker les messages c'est dans le répertoire
de configuration~:

config/etc/vbox/messages

Ce dernier est très pratique, lorsque vous devez mettre à jour votre routeur (nouvelle version) vous avez
juste à copier le répertoire de configuration.

Dans ce répertoire, vous pouvez enregistrer autant de messages que vous voulés, mais seul ceux paramétrés
dans l'archive-opt seront utilisés dans la configuration.

Afin de faire fonctionner vbox sans problèmes, le paquetage est fournit avec un message standard plutôt
neutre, vous auriez certainement souhaitez remplacer ce message par votre propre annonce. Les sons pour
le 'beep.msg' et le 'timeout.msg'(annonce de temps dépassé) peuvent être remplacés aussi. Cependant, il
n'est pas recommandé d'utiliser un son diffèrent par rapport au bip bien connu des répondeurs, car de
nombreux appelants seront un peut confus par rapport au son émis et ne laisseront pas de message.

Si les messages dans le répertoire ne sont pas en ulaw, mais en .au ou .wav, ils seront automatiquement
convertis au démarrage du routeur. Pour convertir le format wav un outil est copié depuis l'archive-opt
il nécessite 180 Kio supplémentaire sur le support de média.

pour finir, si vous avez un ordinateur sans carte son et si vous avez installation vbox vous pouvez
vous appeler pour enregistrer un message. Les données stockées sur le routeur c'est à dire les messages
sont dans un format correct et peuvent être utilisées. Vous pouvez facilement charger les messages qui
son sur le routeur à l'aide du programme SCP ou SFTP.

Si les messages sont stockés sur le Ramdisk ils seront situés dans le répertoire~:

/var/spool/vbox/<username>/incoming

Sinon, dans le répertoire spécifié du disque dur.

Les messages enregistrés sont stockés dans des fichiers .msg avec des noms très cryptiques.
(Pour savoir quel message est dans le fichier vous devez noter l'ordre dans lequel les messages
ont été enregistrés). Renommez le fichier avec un nom plus significatif, vous pourrez ensuite les
utiliser dans votre configuration VBOX.

\subsection{Dans le future}

j'ai adapté le paquetage VBOX créé par Christph Peus pour la version fli4l 3.0.0. Je n'ai pas
prêté une grande attention à la taille des fichiers binaires qui sont beaucoup plus grandes
que ceux des anciennes versions. Si je trouve le temps et il est nécessaire pour cela je vais
m'en occupé. Les nouvelles fonctionnalités prévues c'est l'envoient des messages par courrier
et accès à distance. (Helmut Hummel)

\subsection{Support}

C'est évident, mais je le mentionne quand même~:
Si vous avez des problèmes avec la configuration ou si vous croyez avoir trouvé un bug
s'il vous plaît lire la documentation pour s'assurer que vous n'avez rien oublié. Lorsque
vous utilisez des paramètres complexes, il est facile de faire des erreurs. La documentation
originale avec (l'annexe) peut être utile.

Si vous ne trouvez pas la solution, vous pouvez poster votre question sur le forum de discussion
de spline.fli4l.opt avec une description détaillée du problème, le plus efficace c'est un extrait
du fichier journal. Vous pouvez le trouver dans le répertoire indiqué ci-dessus. Amusez-vous~!

