% Last Update: $Id$
\section {Anhang zum OPT\_VBOX}

\subsection{Format der in der Konfiguration verwendeten Zeitangaben}

Bei einigen Parametern werden Zeit- und Tagesangaben erwartet. Die folgenden 
Informationen zum Format dieser Angaben habe ich unverändert aus der 
Original-Dokumentation des Autors übernommen:

\subsubsection{Zeitangaben}
Ab v2.0.0 von vbox können die verschiedenen Zeitzonen (z.B. bei den  Rings) auch 
mit einer Minutenangabe versehen werden.
Einzelne Zeitangaben werden durch Kommata getrennt, Angaben von Start- und Endzeit 
durch ein Minuszeichen. Die Stundenangaben müssen im 24-Stunden-Format 
- also von 0 Uhr bis 23 Uhr - gemacht werden.

Ergänzung zur Original-Doku: 
Zeitbereiche, die 00:00 Uhr überschreiten, müssen immer 
in zwei Teilen - bis 00:00 Uhr und ab 00:00 Uhr - definiert werden. 
Beispiel: statt 22-06 muß es heissen: 22-23,00-06  (cp)

Die Zeitangaben werden intern immer in Start- und Endzeit umgerechnet, auch dann, 
wenn nur eine Startzeit angegeben ist.

Nehmen wir zum Beispiel folgende Zeitangaben:

20:15-21:14
Diese Zeit wird intern in 20:15:00-21:14:59 umgerechnet, d.h. Start- und Endzeit 
sind inklusive!
Wenn bei einer Zeitangabe keine Minuten angebenen sind, wird bei der Startzeit 
0 Minuten und bei der Endzeit 59 Minuten benutzt. Intern werden die Sekunden 
- die nicht einstellbar sind - nach dem gleichen Schema behandelt.

Beispiel:
\begin{itemize}
    \item 20 - Umrechnung nach 20:00:00-20:59:59
    \item 20:15-21:14 - Umrechnung nach 20:15:00-21:14:59
    \item 08-11 - Umrechnung nach 08:00:00-11:59:59
    \item 12-15:30 - Umrechnung nach 12:00:00-15:30:59
\end{itemize}

Eine Zeitzone zählt als zutreffend (match) wenn die aktuelle Zeit größer/gleich 
der Startzeit uunndd kleiner/gleich der Endzeit ist.

Ein '*' als einzige Zeitangabe wird als immer behandelt, ein '-' oder '!' 
als einzige Zeitangabe als nie.


\subsubsection{Tagesangaben}
Einzelne Tagesangaben werden durch Kommanta getrennt. 
Eine Angabe von Start- und Ende ist hier nicht möglich.

Folgende Tageskürzel können angegeben werden:
\begin{itemize}
    \item MO, MON - für Montag
    \item DI, TUE - für Dienstag
    \item MI, WED - für Mittwoch
    \item DO, THU - für Donnerstag
    \item FR, FRI - für Freitag
    \item SA, SAT - für Samstag
    \item SO, SUN - für Sonntag
\end{itemize}

Beispiel:
MO,DI,DO,FRI,SAT,SO

\subsection{Die Geschichte dieses Paketes}
Ursprünglich stammt dieses Paket von Christoph Peus, der es für die fli4l Versionen 1.x.x und
2.0.x erstellt hat. Später wurde es von Gerd Walter für die 2.1.x Versionen angepasst. Zwischenzeitlich
hat Arno Wetzel dann die Weboberfläche gebaut, womit sich die Installation eines vbox-clients auf einem
anderen Rechner erübrigt. Ich bekam das Paket von Christoph Schulz, der es für die 2.1.10 kompiliert
und auch eine kleine rudimentäre Fernabfragefunktion eingebaut hatte, die bei mir leider nie so richtig 
funktioniert hat. Mit dem Release der neune fli4l sable Version 3.0.0 war ich dann motiviert genug
das Paket einer Grundrenovierung zu unterziehen, um eine solide Basis für die neue fli4l Version zu schaffen
und auch einige der erweiterten Features derselben zu nutzen.

Helmut Hummel im Dezember 2005

\subsection{Die Original VBOX Dokumentation}

\verbatimfile{vbox.txt}
