% Last Update: $Id$
\marklabel{sec:opt-vbox}
{
\section {OPT\_VBOX - Der ISDN-Anrufbeantworter für fli4l}
}

\subsection{Einleitung}

Dieses Paket integriert die Anrufbeantworter-Software VBOX von 
Michael ’Ghandi’ Herold in das fli4l-System.

Man kann den fli4l-Router damit also auch als Anrufbeantworter verwenden, 
wobei die Funktionalität über die eines normalen Anrufbeantworters weit hinausgeht. 
Es ist möglich, verschiedene Voiceboxen einzurichten, die an verschiedene Telefonnummern 
(MSNs) gerichtete Anrufe speichern und nur bestimmten Benutzern den Zugriff erlauben. 
Auch läßt sich für jeden Benutzer einzeln in Abhängigkeit von Tageszeit, Wochentag und 
der Rufnummer des Anrufenden einstellen:

\begin{itemize}
    \item ob ein Anruf überhaupt beantwortet wird
    \item nach wie vielen Klingelzeichen ein Anruf beantwortet wird
    \item ob eine Ansage und falls ja welche Ansage verwendet wird
    \item ob ein Signalton und falls ja welcher Signalton verwendet wird
    \item ob nach der Ansage eine Nachricht aufgesprochen werden kann
    \item wie lang eine Nachricht maximal werden darf
    \item ob eine Timeout-Ansage (und falls ja, welche) verwendet wird
\end{itemize}

\subsection{Grundlegendes zur Funktionsweise / Resourcenbedarf}

Das VBOX-System ist aufgeteilt in einen Server, der die Nachrichten entgegen nimmt
und einen Client, mit dem man die Nachrichten abhören und verwalten kann.
Nachrichten können über ein Webinterface abgehört werden. Dafür ist ein
funktionierendes httpd Paktet nötig. 

\subsubsection{Allgemeine Installationsvoraussetzungen}

Eine ISDN-Karte ist Pflicht, denn vbox läuft nicht mit Modems 
(und über eine DSL-Leitung sowieso nicht, falls jemand auf diesen Gedanken kommen sollte...)
Für die Installation des passenden Treibers wird natürlich das ISDN-Paket benötigt, 
wobei aber kein Circuit definiert werden muß, wenn die Karte ausschließlich für vbox verwendet wird.

Es ist sehr zu empfehlen das httpd Paket zu verwenden, da damit eine recht bequeme
Verwaltung der gespeicherten Nachrichten möglich ist.

\wichtig{Die Teles 16.3c kommt für vbox nicht in Frage, da der Linux-Treiber nicht Voice-fähig ist.}

\subsubsection{Voraussetzungen für das Speichern der Nachrichten in einer Ramdisk-Installation}

Wenn die eingehenden Nachrichten auf einer Ramdisk gespeichert werden sollen, 
benötigt der Router mindestens 16MiB Ram, da sonst keine nennenswerte Aufzeichnungsdauer 
erreichbar ist (siehe: \var{VBOX\_COMPRESSION}).
Es dürfte klar sein, dass auf einer Ramdisk gespeicherte Nachrichten bei einem Stromausfall 
verloren gehen. Wer sich auf seinen AB 100\% verlassen können muß, sollte also von diesem Paket 
Abstand nehmen oder die Nachrichten auf einer Festplatte speichern lassen. Der Vorteil der
Speicherung in der Ramdisk ist, dass die Festplatte im Ruhezustand bleiben kann.

\subsubsection{Voraussetzungen für das Speichern der Nachrichten auf einer HD / CF-Installation}

Wenn die Nachrichten auf einer Festplatte gespeichert werden sollen, benötigt man natürlich 
das HD-Paket mit \var{OPT\_MOUNT}='yes'. Bei einem Neustart des Routers bleiben die Nachrichten 
dann erhalten. Die Einstellungen für bereits vorhandene User werden bei einem Neustart jedoch 
gegebenenfalls aktualisiert, und in einer neuen Konfiguration nicht mehr vorhandene User 
automatisch gelöscht (optional). Eine Festplatte ist auch dann vorteilhaft, 
wenn man viele verschiedene Ansagen oder zusätzliche OPT-Pakete benötigt, da der Platz auf dem
Bootmedium dann eventuell nicht mehr ausreicht.

\subsection{Konfiguration}
\subsubsection{allgemeine Einstellungen}
\begin{description}

\config{OPT\_VBOX}{OPT\_VBOX}{OPTVBOX}

        Standard-Einstellung: \var{OPT\_VBOX}='no'

        Mit der Einstellung 'yes' wird der Anrufbeantworter aktiviert.

\config{VBOX\_SPOOLPATH}{VBOX\_SPOOLPATH}{VBOXSPOOLPATH}

    Mit dem Parameter \var{VBOX\_SPOOLPATH} wird festgelegt, ob die aufgesprochenen Nachrichten 
    auf einer Ramdisk oder auf einer Festplatte gespeichert werden.

    Speichern auf einer Ramdisk:
    \var{VBOX\_SPOOLPATH}=''
    Es darf hier also kein Pfad angegeben werden!

    Speichern auf einer Festplatte:

    \var{VBOX\_SPOOLPATH}='/data/vbox' (z.B.)
    Voraussetzung ist hier das hd-Paket mit \var{OPT\_MOUNT}='yes'

    Der angegebene Pfad muss auf einer Festplatte im Filesystem einer 
    schreibbar gemounteten ext2/3 Partition liegen, jedoch nicht in '/opt' (HD-Install Typ B).
    Wird eine dieser Bedingungen nicht erfüllt, wird die Installation des vbox-Pakets 
    mit einer entsprechenden Fehlermeldung abgebrochen. Sind alle Bedingungen erfüllt, wird 
    in dem angegebenen Pfad das Verzeichnis \var{vbox\_spooldir} angelegt, in dem die eigentlichen 
    vbox-Daten abgelegt werden.

    Hier kann auch 'auto' stehen, dann wird der automatisch gefundene Pfad benutzt, der durch \var{FLI4L\_UUID} definiert ist.

\config{VBOX\_SPOOLDIR\_SPACE}{VBOX\_SPOOLDIR\_SPACE}{VBOXSPOOLDIRSPACE}
    
    Gibt an wieviel Platz in KiB für für alle Mailboxen veranschlagt werden soll. 
    Beim Anlegen des Spoolverzeichnisses wird dann geprüft, ob dieser Platz auch zur Verfügung steht.

\config{VBOX\_DELETE\_OLD\_SPOOLDIRS}{VBOX\_DELETE\_OLD\_SPOOLDIRS}{VBOXDELETEOLDSPOOLDIRS}
    
    Standard-Einstellung: \var{VBOX\_DELETE\_OLD\_SPOOLDIRS}='yes'
    
    Dieser Parameter ist nur von Bedeutung, wenn die eingehenden Nachrichten auf einer 
    Festplatte abgelegt werden. Da die Spoolverzeichnisse der einzelnen user dann beim Neustart 
    nicht gelöscht werden, kann es bei häufig wechselnden Konfigurationen mit verschiedenen
    usern vorkommen, dass sich im Laufe der Zeit alte Spoolverzeichnisse von inzwischen inaktiven
    usern ansammeln, die die Übersicht behindern und unnötig Platz verbrauchen. 
    Um das zu verhindern, kann man diesen Parameter auf 'yes' setzen, so dass bei jedem Neustart 
    automatisch aufgeräumt wird.

    VORSICHT: wer den Namen eines einmal konfigurierten users in der Konfigurationsdatei auch nur 
    geringfügig ändert, löscht damit beim nächsten Neustart automatisch alle für diesen user 
    gespoolten Nachrichten, weil der user mit dem ursprünglichen Namen als nicht mehr aktiv gewertet wird. 
    Man muß hier also sehr aufpassen, dass man mit dieser Automatik nicht ungewollt Nachrichten löscht.

\config{VBOX\_COMPRESSION}{VBOX\_COMPRESSION}{VBOXCOMPRESSION}

    Standard-Einstellung: \var{VBOX\_COMPRESSION}='ulaw'
    
    Hier lässt sich einstellen, mit welcher Kompression die Nachrichten aufgezeichnet werden sollen. 
    Je höher die Kompression, desto geringer der Speicherplatzverbrauch und desto schlechter die Qualität.
    Wer die Nachrichten auf einer Festplatte ablegen lässt, kann sich wahrscheinlich unbesorgt für den Modus 
    'ulaw' entscheiden, bei dem keine Kompression erfolgt. Die Nachrichten werden dann mit der vollen 
    ISDN-Bandbreite von 8Bit * 8kHz = 64kbit/s gespeichert. Bei einer reinen Ramdisk-Installation ohne 
    Festplatte auf einem PC mit 16MiB Speicher reichen die dann für die Aufzeichnung zur Verfügung stehenden 
    ca. 6MiB allerdings nur für etwas mehr als zehn Minuten Aufzeichnungsdauer.

    Wer Platz sparen muß, sollte sich daher für eine komprimierte Aufzeichung im Format 'adpcm-4' 
    (4 Bit * 8kHz = 32kbit/s -> halber Speicherbedarf gegenüber 'ulaw') 
    oder für 'adpcm-3' bzw. 'adpcm-2' mit entsprechend höherer Kompression und 
    schlechtere Qualität entscheiden.

    Hier eine Übersicht:

    \begin{table}[htbp]
      \begin{tabular}{lrrr}
        Modus    & Sample-Auflösung & Kompression auf & Speicherbedarf für 10min (ca.) \\
        \hline \\
        ulaw     & 8Bit             & 100\%           & 4800 kB       \\
        adpcm-4  & 4Bit             &  50\%           & 2400 kB       \\
        adpcm-3  & 3Bit             &  37\%           & 1800 kB       \\
        adpcm-2  & 2Bit             &  25\%           & 1200 kB       \\
      \end{tabular}
      \caption{Die verschiedenen Kompressionsraten im Vergleich}
    \end{table}

\config{VBOX\_FREESPACE}{VBOX\_FREESPACE}{VBOXFREESPACE}

    Standard-Einstellung: \var{VBOX\_FREESPACE}='8192'

    Wenn zum Speichern einer neuen Nachricht weniger als \var{VBOX\_FREESPACE} Bytes 
    freier Speicher zur Verfügung stehen, wird der Anruf gar nicht erst angenommen. 
    Ein Wert von '0' bewirkt, dass diese Überprüfung abgeschaltet wird.
 
\config{VBOX\_LOGPATH}{VBOX\_LOGPATH}{VBOXLOGPATH}

    Standard-Einstellung: \var{VBOX\_LOGPATH}='/var/log/vbox'

    Gibt das Verzeichnis an, in das die Logfiles geschrieben werden sollen.

    Hier kann auch 'auto' stehen, dann wird der automatisch gefundene Pfad benutzt, der durch \var{FLI4L\_UUID} definiert ist.

\config{VBOX\_USE\_VBOXD}{VBOX\_USE\_VBOXD}{VBOXUSEVBOXD}

    Standard-Einstellung: \var{VBOX\_USE\_VBOXD}='no'
    
    Nachrichten mit können mit beliebigen vbox-clients abgerufen werden. 
    Wenn dies gewünscht ist, dann einfach hier auf 'yes' stellen und in
    \var{VBOX\_USER\_x\_VBOXD\_PASSWORD} unbedingt das Passwort setzen.
    
    \wichtig{Der vboxd ist ein Serverdienst. Wenn möglich, sollte statt dessen lieber 
    die Weboberfläche verwendet werden.}

\config{VBOX\_VBOXD\_ALLOW}{VBOX\_VBOXD\_ALLOW}{VBOXVBOXDALLOW}

    Standard-Einstellung: \var{VBOX\_VBOXD\_ALLOW}='*.lan.fli4l'
    
    Mit dem \var{VBOX\_VBOXD\_ALLOW} Parameter kann definiert werden, von welchen Rechnern 
    der Anrufbeantworter mit Hilfe eines vbox-clients abgehört bzw. verwaltet werden darf. 
    Die Defaulteinstellung '*.lan.fli4l' steht für jeder Rechner innerhalb dieser DNS-Domain. 
    Einschränkungen sind durch Angabe einzelner IP-Adressen, Hostnamen oder Domainnamen 
    (in der Form *.home.lan) möglich. Mehrere Einträge sind mit einem Leerzeichen zu trennen.
    
    \wichtig{Bei Angabe von Hostnamen der eigenen Domain ist nur der Hostname, nicht der 
    volle DNS-Name anzugeben!}

    \wichtig{Um die Risiken eines potentiellen Angriffs auf den Serverdienst vboxd gering zu 
    halten, sollte diese Einstellung so restriktiv wie möglich gehalten werden. Wenn 
    beispielsweise nur VBOX\_BEEP benötigt wird, sollte VBOX\_VBOXD\_ALLOW leer bleiben, wodurch
    der Zugriff auf den vboxd nur vom Router selbst möglich ist.}

\config{VBOX\_BEEP\_HOURS}{VBOX\_BEEP\_HOURS}{VBOXBEEPHOURS}

    Standard-Einstellung: \var{VBOX\_BEEP\_HOURS}='*'
    
    Hier lassen sich die Zeiten, zu denen das akustische Signal gesendet werden soll, 
    stundenweise angeben. Es lassen sich auch Bereiche angeben, z.B. 8-24, wobei auch mehrere 
    Angaben durch Komma getrennt werden können. Ein '*' steht für 'immer'. Eine genaue 
    Erklärung des Formats der Zeitangaben findet sich im Anhang am Ende dieser Doku.
    
    \wichtig{Damit VBOX\_BEEP funktioniert, muss VBOX\_USE\_VBOXD='yes' eingestellt werden.}

\config{VBOX\_BEEP\_PAUSE}{VBOX\_BEEP\_PAUSE}{VBOXBEEPPAUSE}

    Standard-Einstellung: \var{VBOX\_BEEP\_PAUSE}='60'
    
    Die Länge der Pause zwischen den einzelnen Signaltönen in Sekunden.

\config{VBOX\_DEBUGLEVEL}{VBOX\_DEBUGLEVEL}{VBOXDEBUGLEVEL}

    Standard-Einstellung: \var{VBOX\_DEBUGLEVEL}='FE'

    Hier kann durch eine Buchstabenkombination angegeben werden, welche Ereignisse in 
    den log-files protokolliert werden sollen. Aus der Original-Doku:
    \begin{itemize}
        \item F - Fehler die nicht behoben werden können
        \item E - Fehler die eventuell behoben werden können
        \item W - Warnungen
        \item I - Informationen
        \item D - Debugging Ausgaben
        \item J - Noch mehr Debugging Ausgaben
    \end{itemize}

    Für den Anfang sollte man ruhig alles einschalten, weil die log-files bei der Fehlersuche 
    sehr hilfreich sein können. Läuft das System erstmal ordentlich, sollte 'FE' ausreichend sein.

\config{VBOX\_ADMIN\_USERNAME}{VBOX\_ADMIN\_USERNAME}{VBOXADMINUSERNAME}

    Der Benutzername des in der httpd Konfiguration bereits eingetragen 
    (Groß- und Kleinschreibung wird unterschieden) Administrators. Dieser Benutzer
    sieht in der VBOX WebGUI stehts alle Sprachboxen und kann somit auch alle 
    starten, stoppen und Nachrichten derselben anhören.
    \wichtig{Der Benutzer muss die Berechtigung 'vbox:all' besitzen.}

\end{description}

\subsubsection{Userspezifische Einstellungen - Allgemeine Angaben}

\begin{description}

\config{VBOX\_USER\_N}{VBOX\_USER\_N}{VBOXUSERN}

    Die Anzahl der User, die mit über VBOX Nachrichten bekommen sollen.

\config{VBOX\_USER\_x\_USERNAME}{VBOX\_USER\_x\_USERNAME}{VBOXUSERUSERNAME}

    Der Benutzername den der User bekommen soll. Dieser Benutzername wird auch zur
    Authentifizierung an der Weboberfläche verwendet. Ist dieser Benutzername in der httpd 
    Konfiguration bereits eingetragen (Groß- und Kleinschreibung wird unterschieden), so 
    gelten die Berechtigungen, die dort angegeben sind (siehe Dokumentation des httpd Paketes).
    Wird der dieser Benutzername dort nicht verwendet, bekommt dieser User ausschliesslich die
    Berechtigung die VBOX Seite aufzurufen.

\config{VBOX\_USER\_x\_PASSWORD}{VBOX\_USER\_x\_PASSWORD}{VBOXUSERPASSWORD}

    Das Passwort des Users. Ist der Benutzername in \var{VBOX\_USER\_x\_USER} auch
    in der Konfiguration des httpd Paketes eingetragen, so wird das dort angegebene
    Passwort verwendet und der Inhalt von \var{VBOX\_USER\_x\_PASS} ist bedeutungslos.
    Ansonsten gilt dieses Passwort zur Authentifizierung auf der Weboberfläche.
    
\config{VBOX\_USER\_x\_VBOXD\_PASSWORD}{VBOX\_USER\_x\_VBOXD\_PASSWORD}{VBOXUSERxVBOXDPASSWORD}

    Hier kann ein Passwort für den vboxd angegeben werden. Es wird beim Login
    über einen vbox-client (nicht der Weboberfläche) verwendet.
    
\config{VBOX\_USER\_x\_MSN\_N}{VBOX\_USER\_x\_MSN\_N}{VBOXUSERMSNN}

    Hier die Anzahl der zu überwachenden MSNs angegeben. 

\config{VBOX\_USER\_x\_MSN\_x}{VBOX\_USER\_x\_MSN\_x}{VBOXUSERMSNx}

    Hier wird die MSN eingetragen, auf der Anrufe entgegengenommen werden soll.

\config{VBOX\_USER\_x\_MSN\_x\_START}{VBOX\_USER\_x\_MSN\_x\_START}{VBOXUSERMSNxSTART}

    Standard-Einstellung: \var{VBOX\_USER\_x\_MSN\_x\_START}='yes'

    Hier kann angegeben werden, ob VBOX für diese MSN beim Booten aktiviert 
    werden soll. Diese Variable ist optional, wird sie weggelassen, ist VBOX für 
    diese MSN nach dem Booten aktiv.

\config{VBOX\_USER\_x\_BEEP}{VBOX\_USER\_x\_BEEP}{VBOXUSERxBEEP}

    Standard-Einstellung: \var{VBOX\_USER\_x\_BEEP}='no'
    
    Wenn mit einem akustischen Signal angezeigt werden soll, dass für diesen User 
    neue Nachrichten vorhanden sind, muss dieser Parameter auf 'yes' gesetzt werden, 
    ansonsten auf 'no'.
    
    \wichtig{Bei diesem Parameter ist folgendes zu beachten: Wenn die Spoolverzeichnisse 
    auf einer Festplatte liegen, kann sich die Platte bei aktivem vboxbeep auch dann 
    nicht ausschalten, wenn dies in der Konfiguration des HD-Pakets so eingestellt wird, 
    weil der vboxbeep-daemon ständig auf die Platte zugreifen muß, um den Eingang neuer 
    Nachrichten zu überprüfen.}

\config{VBOX\_USER\_x\_STD\_RINGDEF\_N}{VBOX\_USER\_x\_STD\_RINGDEF\_N}{VBOXUSERxSTDRINGDEFN}

    VBOX kann zu veschiedenen Uhr- und Tageszeiten nach einer jeweils definierten Anzahl 
    von Klingelzeichen den Anruf annehmen. Diese Variable gibt an wie viele Zeitbereiche
    im Folgenden definiert werden.

\config{VBOX\_USER\_x\_STD\_RINGDEF\_x}{VBOX\_USER\_x\_STD\_RINGDEF\_x}{VBOXUSERxSTDRINGDEFx}

    Syntax: \var{VBOX\_USER\_x\_STD\_RINGDEF\_x}='TIME DAYS RINGS'
    
    Die in \var{VBOX\_USER\_x\_STD\_RINGDEF\_x} angegebenen Parameter werden durch ein oder mehrere 
    Leerzeichen voneinander getrennt. Es kann für bestimmte Zeitbereiche, d.h. für bestimmte 
    Tageszeiten (TIME) an bestimmten Wochentagen (DAYS) eingestellt werden, nach wie vielen 
    Klingelzeichen (RINGS) ein Anruf während dieser Zeiten überhaupt angenommen wird. Eine genaue 
    Erklärung des Formats der Zeitangaben findet sich im Anhang am Ende dieser Doku. Es ist 
    übrigens nicht möglich, den AB vor dem ersten Klingeln abheben zu lassen. Ein Wert von 
    '0' für RINGS führt dazu, dass der Anruf gar nicht beantwortet wird.

    Diese Einstellungen sind Standard-Einstellungen, die verwendet werden, wenn für einen Anrufer 
    keine Sonderbehandlung konfiguriert wurde (siehe \var{VBOX\_USER\_PROFILE\_x}), oder eine 
    Sonderbehandlung keine eigene Definition von RINGS besitzt.

    Es lässt sich beispielsweise einstellen, dass der AB während der Nachtstunden normalerweise 
    schon nach dem ersten Klingeln rangeht, damit man nicht aufgeweckt wird, zu allen anderen 
    Zeiten aber erst nach dem fünften Klingeln. Gibt es aber Anrufer, deren Anrufe so wichtig 
    sein könnten, dass man auf jeden Fall wachgeklingelt werden möchte, so kann man für diese 
    Anrufer mittels \var{VBOX\_USER\_PROFILE\_x} eine eigene Anrufersektion definieren, in der 
    man die Standard-Einstellung aus \var{VBOX\_USER\_x\_STD\_RINGDEF\_x} wieder überschreiben kann.
    
\end{description}

\subsubsection{Userspezifische Einstellungen - Zeitabhängige Programmierung}

\begin{description}

\config{VBOX\_USER\_x\_STD\_SECDEF\_N}{VBOX\_USER\_x\_STD\_SECDEF\_N}{VBOXUSERxSTDSECDEFN}

    VBOX kann zu veschiedenen Uhr- und Tageszeiten nicht nur nach einer unterschiedlichen 
    Anzahl von Klingeltönen den Anruf annehmen, sondern auch welche Nachrichten abgespielt werden,
    wie lange die Aufnahmezeit ist, und so weiter. Diese Variable gibt an wie viele Zeitbereiche
    für solche Definitionen im Folgenden definiert werden.

\config{VBOX\_USER\_x\_STD\_SECDEF\_x}{VBOX\_USER\_x\_STD\_SECDEF\_x}{VBOXUSERxSTDSECDEFx}

    Syntax: \var{VBOX\_USER\_x\_STD\_SECDEF\_x}='TIME DAYS MESSAGE RECTIME [FLAG] [...]'
    
    Auch die Parameter \var{VBOX\_USER\_x\_STD\_SECDEF\_x} werden durch mindestens ein Leerzeichen
    voneinander getrennt. Sie enthalten die Einstellungen für die 'STANDARD'-Sektion, die immer vorhanden sein 
    müssen, weil sie definiert, wie Anrufer behandelt werden sollen, für die keine Anrufersektion 
    vorhanden ist oder für die keine Anrufersektion definiert werden kann, weil sie ihre Rufnummer 
    nicht übermitteln.
    
    Im Folgenden werden die einzelnen Paramenter erläutert.

    
    \var{TIME}
    
    Zeiten, zu denen die Einstellungen benutzt werden sollen.

    
    \var{DAYS}
    
    Tage, an denen die Einstellungen benutzt werden sollen.

    
    \var{MESSAGE}
    
    Nachricht die als Ansagetext gespielt werden soll. Eine Ansage mit diesem Namen muß 
    vor dem Erstellen der Archive nach opt/etc/vbox/messages bzw. config/etc/vbox/messages
    kopiert werden. Leerzeichen und Tabulatoren sind nicht erlaubt. Auch wenn eine .wav
    Datei verwendet wird, muss an dieser Stelle die Datei mit der Endung .msg angegeben werden.
    In das Verzeichnis wird dann die .wav kopiert, die dann beim Booten in eine .msg umgewandelt
    wird. Das für die Umwandlung benötigte \var{sox} Programm wird automatisch auf den Router kopiert.
    Die in älteren vbox Versionen vorhandene Variable \var{VBOX\_WAV\_FILES} wird also nicht 
    mehr benötigt.

    \var{RECTIME}
    
    Anzahl der Sekunden die maximal aufgezeichnet werden sollen. Voreinstellung ist 60 Sekunden.


    \var{FLAG}
    
    Zusätzliche Flags die angegeben werden können:
    \begin{itemize}
        \item \var{NOANSWER}: Der Anruf soll nicht beantwortet werden.
        \item \var{NORECORD}: Es soll keine Nachricht aufgezeichnet werden.
        \item \var{NOTIMEOUTMSG}: Es soll keine Timeout-Nachricht gespielt werden.
        \item \var{NOBEEPMSG}: Der Signalton soll nicht gespielt werden.
        \item \var{NOSTDMSG}: Der Ansagetext soll nicht gespielt werden.
        \item \var{RINGS}= : Gibt an nach wievielen RING's der Anruf beantwortet werden soll. 
            Dieses Flag überschreibt die RING's aus der Sektion [RINGS] und das Flag TOLLRINGS.
        \item \var{TOLLRINGS}= : Gibt an, nach wievielen RING's der Anruf beantwortet werden soll, 
            wenn neue Nachrichten vorhanden sind. Dieses Flag überschreibt die RING's aus der Sektion 
            [RINGS] und das Flag RINGS bei neuen Nachrichten.
    \end{itemize}
\end{description}

\subsubsection{Userspezifische Einstellungen - Anruferspezifische Konfiguration}

\begin{description}

\config{VBOX\_USER\_x\_PROFILE\_N}{VBOX\_USER\_x\_PROFILE\_N}{VBOXUSERxPROFILEN}

    Hiermit wir die Anzahl der Rufnummern definiert, die einer bestimmten Sektion
    zugeordnet werden sollen.

\config{VBOX\_USER\_x\_PROFILE\_x}{VBOX\_USER\_x\_PROFILE\_x}{VBOXUSERxPROFILEx}

    Syntax: \var{VBOX\_USER\_x\_PROFILE\_x}='CALLERID SECTION\_NAME DESCRIPTION'

    Registriert VBOX einen Anruf, bei dem die Rufnummer des Anrufers übermittelt wird, 
    so wird geprüft, ob für diesen Anrufer bzw. seine Nummer eine Sonderbehandlung 
    vorgesehen ist. Mit dieser Variablen wird dem Anrufer eine bestimmte Sektion (siehe
    \var{VBOX\_USER\_x\_STD\_SECDEF\_x}) zugewiesen.
    
    Der erste Eintrag in jeder Zeile ist die Nummer des / der Anrufenden [CALLERID]
    MIT VORWAHL, JEDOCH OHNE DIE FÜHRENDE NULL. Hier können auch Rufnummernbereiche angegeben 
    werden, und zwar laut original-Doku des VBOX-Autors mit Hilfe eines 'Unix-Pattern'. 
    Nachdem ich verschiedene Unix-Pattern mit unterschiedlichem Erfolg getestet habe, scheinen 
    folgende Methoden der Bereichsangabe zu funktionieren:
    
    Mehrere angrenzende Nummern, z.B. von 55511 bis 55514: '5551[1-4]'
    
    Alle gleich beginnenden Nummern, z.B. alles mit '555' beginnend: '555*'
    
    Des Weiteren kann statt einer Nummer der Platzhalter \var{PHONEBOOK} verwendet werden.
    Dann werden alle Telefonummern, die in /etc/phonebook gespeichert sind der Sektion
    [SECTION\_NAME] zugeordnet. Voraussetzung dafür ist natürlich, dass das phonebook auch 
    auf dem Router vorhanden ist. Dazu ist notwendig, in der config/isdn.txt \var{OPT\_TELMOND}='yes' 
    zu setzen.

    Der zweite Eintrag [SECTION\_NAME] ist der frei wählbare Name der Anrufersektion, 
    in der definiert wird, wie der Anrufbeantworter sich verhalten soll, wenn ein Anruf von [CALLERID]
    eingeht. Dabei steht ein '-' für die Anrufersektion 'STANDARD' (\var{VBOX\_USER\_x\_STD\_SECDEF\_x}), 
    ein '*' für eine Anrufersektion, die auf den Namen des Anrufers lautet, der in [DESCRIPTION] angegeben
    werden kann. Dieser Eintrag darf ausnahmsweise Leerzeichen enthalten und ermöglicht es, dass 
    die WEB-Gui in der Liste der eingegangenen Anrufe nicht nur die Rufnummer, sondern auch den Namen des 
    Anrufers anzeigen kann. Technisch ist dieser Eintrag also bedeutungslos.

    \wichtig{Für alle Anrufer, für die kein \var{VBOX\_USER\_x\_PROFILE\_x} definiert wurde oder die 
        einfach gar keine Rufnummer übermitteln, sollte als letzte \var{VBOX\_USER\_x\_PROFILE\_x}-Zeile 
        immer folgendes eingetragen werden:}
        \begin{verbatim}
            '*    -    --- Unknown ---'
        \end{verbatim}
    Nur dann werden diese Anrufe über die Sektion 'STANDARD' beantwortet. 
    (Statt '--- Unknown ---' kann hier natürlich irgendwas stehen.)



\config{VBOX\_USER\_x\_SECTION\_N}{VBOX\_USER\_x\_SECTION\_N}{VBOXUSERxSECTIONN}

    Hiermit wir die Anzahl der Bereiche definiert, in denen VBOX von dem in
    \jump{VBOXUSERxSTDSECDEFx}{\var{VBOX\_USER\_x\_STD\_SECDEF\_x}} und \jump{VBOXUSERxSTDRINGDEFx}{\var{VBOX\_USER\_x\_STD\_RINGDEF\_x}} Verhalten
    abweicht.    
    
\config{VBOX\_USER\_x\_SECTION\_x\_NAME}{VBOX\_USER\_x\_SECTION\_x\_NAME}{VBOXUSERxSECTIONxNAME}

    Der Name des Bereiches, der mit dem in \jump{VBOXUSERxPROFILEx}{\var{VBOX\_USER\_x\_PROFILE\_x}} angegebenen übereinstimmen
    muss.

\config{VBOX\_USER\_x\_SECTION\_x\_SECDEF\_N}{VBOX\_USER\_x\_SECTION\_x\_SECDEF\_N}{VBOXUSERxSECTIONxSECDEFN}

    Anzahl der Zeitbereiche für Sectionsdefinitionen (analog zu \jump{VBOXUSERxSTDSECDEFN}{\var{VBOX\_USER\_x\_STD\_SECDEF\_N}})

\config{VBOX\_USER\_x\_SECTION\_x\_SECDEF\_x}{VBOX\_USER\_x\_SECTION\_x\_SECDEF\_x}{VBOXUSERxSECTIONxSECDEFx}

    Syntax: \var{VBOX\_USER\_x\_SECTION\_x\_SECDEF\_x}='TIME DAYS MESSAGE RECTIME [FLAG] [...]'

    Wie auch in \jump{VBOXUSERxSTDSECDEFx}{\var{VBOX\_USER\_x\_STD\_SECDEF\_x}} können hier genaue Einstellungen zum Verhalten
    von VBOX hinterlegt werden. Die Konfiguration ist identisch zu der in \jump{VBOXUSERxSTDSECDEFx}{\var{VBOX\_USER\_x\_STD\_SECDEF\_x}}
    und kann dort nachgelesen werden.

\end{description}

\subsection{Konigurationsbeispiel}

\begin{verbatim}
OPT_VBOX='yes'                           # VBOX ist eingeschaltet
VBOX_SPOOLPATH=''                        # Nachrichten werden in der Ramdisk gespeichert
VBOX_SPOOLDIR_SPACE='4000'               # Es wird 4000 KiB für die Ramdisk verwendet
VBOX_DELETE_OLD_SPOOLDIRS='yes'          # Ist für Ramdisk uninteressant
VBOX_COMPRESSION='ulaw'                  # Es wird ulaw Kompression verwendet
VBOX_FREESPACE='8192'                    # Es muss mindestens 8MiB freier Speicher vorhanden
                                         # sein, damit Nachrichten aufgezeichnet werden
VBOX_LOGPATH='/var/log/vbox'             # Logfiles werden in /var/log/vbox gespeichert
VBOX_DEBUGLEVEL='FE'                     # Nur Fehler in's Logfile schreiben

VBOX_USER_N='1'                          # Nur ein User
VBOX_USER_1_USER='user1'                 # Username
VBOX_USER_1_PASS='pass1'                 # Password
VBOX_USER_1_MSN='1234'                   # Nur eine MSN wird überwacht

# Zu jeder Uhrzeit Anrufe nach 4 Klingeltönen annehmen.
VBOX_USER_1_STD_RINGDEF_N     = '1' #TIME         DAYS        RINGS
VBOX_USER_1_STD_RINGDEF_1     =     '*            *           4'

# Zu jeder Uhrzeit die Standardansage abspielen und 60 Sekunden Aufnahmezeit erlauben.
VBOX_USER_1_STD_SECDEF_N  = '1'  #TIME        DAYS        MESSAGE             RECTIME [FLAG] [...]
VBOX_USER_1_STD_SECDEF_1  =      '*           *           standard.msg        60'

# Alle Anrufe werden der Standardsection zugeordnet
VBOX_USER_1_PROFILE_N  = '1'  #CALLERID    SECTION_NAME     DESCRIPTION
VBOX_USER_1_PROFILE_1  =      '*           -                -- unkown --'

# Keine Speziellen Sektionen definiert
VBOX_USER_1_SECTION_N          = '0'  #SECTION for VBOX_USER_x_PROFILE_x
VBOX_USER_1_SECTION_1_NAME     = ''
VBOX_USER_1_SECTION_1_SECDEF_N = '0'  #TIME        DAYS        MESSAGE             RECTIME [FLAG] [...]
VBOX_USER_1_SECTION_1_SECDEF_1 = ''

\end{verbatim}    

\subsection{Aufnehmen und 'Installieren' neuer Ansagen}

Die Ansagen, die in der vbox-Konfiguration verwendet werden können, müssen im fli4l-Verzeichnisbaum 
in folgendes Verzeichnis kopiert werden:

opt/etc/vbox/messages

Alternativ können die Ansagen aber auch im Konfigurationsverzeichnis untergebracht werden:

config/etc/vbox/messages

Letzteres ist sehr praktisch, weil man bei einem Routerupdate auf eine neue Version
nur das Konfigurations-Verzeichnisses kopieren muss.

In diesen Verzeichnissen können beliebig viele Nachrichten sein, es werden nur diejenigen
dem opt-Archiv hinzugefügt, die auch in der Konfiguration benutzt werden.

Um den AB möglichst problemlos in Betrieb nehmen zu können, bringt das Paket eine einigermassen neutrale 
Standard-Ansage mit, die man aber sicherlich durch eigene Ansagen ersetzen bzw. ergänzen möchte. Auch der 
Signalton, der immer 'beep.msg' heissen muss, und die Timeout-Ansage ('timeout.msg'), die auf das Ende 
der Aufzeichnungsdauer hinweist, können durch eigene Signale bzw. Texte ersetzt werden. 
Es ist allerdings nicht zu empfehlen, einen Signalton zu verwenden, der sich wesentlich von dem 
bekannten Piepen normaler Anrufbeantworter unterscheidet, weil das viele Anrufer derart in Verwirrung 
stürzt, dass sie lieber keine Nachricht hinterlassen.

Wenn die Nachrichten in den Verzeichnissen nicht im ulaw, sondern im .au oder .wav Format vorliegen,
werden sie beim Start des Routers automatsch konvertiert. Zur Konvertierung aus dem .wav Format wird
automatisch ein Tool in das opt-Archiv gepackt werden, dass zusätzlich noch ca. 180KiB benötigt.

Schließlich gibt es auch die Möglichkeit, ohne einen mit Soundkarte und Mikro ausgestatteten Rechner 
Ansagen garantiert im richtigen Format aufzunehmen, indem man vbox erst mal normal installiert und 
sich selbst anruft. Die auf dem Router gespeicherten Nachrichten sind im richtigen Format
und können verwendet werden. Man kann sie am einfachsten mit einem SCP oder SFTP Programm vom Router
holen.

Sind die Nachrichten in der Ramdisk gespeichert, befinden sie sich unter:

/var/spool/vbox/<username>/incoming

Ansonsten in dem angegebenen Verzeichniss der Festplatte.

Dort liegen die aufgesprochenen Ansagen nun als .msg Dateien mit recht kryptischen Namen. 
(Um zu wissen, welche Ansage in welcher Datei steckt, sollte man sich die Reihenfolge der 
aufgesprochenen Ansagen merken.) Man sollte den Dateien nun etwas sprechendere Namen geben und 
kann sie dann in der VBOX-Konfiguration verwenden.

\subsection{Ausblick}

Im Wesentlichen habe ich erst mal nur das VBOX-Paket von Christph Peus für die fli4l Version 3.0.0
angepasst. Dabei habe ich zunächst erst mal keine Rücksicht auf die Größe der Binaries genommen,
was sich im Vergleich zur älteren Version doch deutlich bemerkbar macht. Wenn ich Zeit finde
und deutlicher Bedarf besteht, werde ich mich dessen mal annehmen. Als neue Features sind dann
auch Mailversand der Nachrichten und Fernabfrage geplant. (Helmut Hummel)

\subsection{Support}

Es ist ja eigentlich selbstverständlich, aber ich erwähne es vorsichtshalber trotzdem: 
wenn Ihr mit der Konfiguration nicht zurecht kommt oder glaubt, einen bug gefunden zu haben, 
schaut bitte zunächst nochmal in dieser Anleitung nach, ob Ihr nichts übersehen habt. 
Gerade bei den etwas komplexeren Parametern macht man leicht Flüchtigkeitsfehler. 
Eventuell kann auch die Original-Doku (Im Anhang) helfen.

Wenn Ihr wirklich nicht weiterkommt, postet eure Frage in der Newsgruppe spline.fli4l.opt mit einer
möglichst genaue Beschreibung des Problems, am besten inkl. entsprechenden Auszügen aus dem Logfile. 
Dieses findet man auf dem Router in dem angegebenen Verzeichnis. 
Viel Spass!
