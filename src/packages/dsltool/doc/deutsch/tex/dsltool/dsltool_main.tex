% Last Update: $Id$
\section {DSLTOOL - DSL-Modem-Daten erfassen und graphisch Anzeigen}

\subsection {Beschreibung}
Das Paket DSLTOOL sammelt mit Hilfe des Daemons 'collectd' Systemdaten und
speichert diese in rrd-Datenbanken ab.
Im Webinterface des fli4l-Routers sind dann die daraus mit Hilfe von rrdtool
erzeugten Grafiken abruf- bzw. einsehbar.
\\
Es werden zum Beispiel die folgenden Daten erfasst und dargestellt:
\begin{itemize}
  \item Bit Allozierung
  \item Rauschabstand
  \item Dämpfung
  \item Sendeleistung
  \item Fehler-Sekunden
  \item Frame Error Counter
  \item CRC Error Counter
  \item Header Error Counter
\end{itemize}

\subsection {Unterstützte DSL-Modems}

  Im Moment werden nur die unter \smalljump{DSLTOOLMODEM}
  {\var{DSLTOOL\_x\_MODEM}} aufgelisteten DSL-Modems vom DSL-Tool unterstützt.
  
  Im Wiki {[\ref{wiki}]} kann man Hinweise zur Konfiguration bestimmter Modems
  finden oder auch hinzufügen.

  Sollte Ihr spezielles DSL-Modem nicht unterstützt werden so bitte ich um eine
  eMail um abzuklären ob eine Unterstützung möglich ist.

  Um die Möglichkeiten des DSL-Tools ohne unterstütztes DSL-Modem auszuprobieren
  kann \smalljump{DSLTOOLMODEM}{\var{DSLTOOL\_x\_MODEM}} auf 'demo-adsl' oder
  'demo-vdsl' gesetzt werden.

\subsection {Konfiguration des Paketes DSLTOOL}

  Die Konfiguration erfolgt, wie bei allen fli4l Paketen, durch Anpassung der \\
  Datei \var{Pfad/fli4l-\version/$<$config$>$/dsltool.txt}
  an die eigenen Anforderungen.

\begin{description}

\config {OPT\_DSLTOOL}{OPT\_DSLTOOL}{OPTDSLTOOL}

  Die Einstellung \var{'no'} deaktiviert das OPT\_DSLTOOL vollständig. Es werden
  keine Änderungen am fli4l Archiv \var{rootfs.img} bzw. dem Archiv \var{opt.img}
  vorgenommen. Weiterhin überschreibt das OPT\_DSLTOOL grundsätzlich keine anderen
  Teile der fli4l Installation.\\
  Um OPT\_DSLTOOL zu aktivieren, ist die Variable \var{OPT\_DSLTOOL} auf
  \var{'yes'} zu setzen.

\config {DSLTOOL\_N}{DSLTOOL\_N}{DSLTOOLN}

  Legt die Anzahl der abzufragenden DSL-Modems fest.

\config {DSLTOOL\_x\_MODEM}{DSLTOOL\_x\_MODEM}{DSLTOOLMODEM}

  Wählt den verwendeten DSL-Modem-Typ aus. Die Variable kann folgende Werte
  annehmen:
\begin{description}
\item[amazon] Infineon \smalljump{appendix:amazon}{Amazon SE}

  Modems basierend auf dem Infineon/Lantiq Amazon SE Chipsatz

\item[ar7] TI \smalljump{appendix:ar7}{AR7}

  Modems/Router basierend auf der AR7 Chipsatz-Familie von Texas Instruments

\item[avm-tr064] AVM \smalljump{appendix:avm-tr064}{Fritz!Box}

  AVM Fritz!Box Router (Firmware $\ge$ 5.50)

  Die Variable \smalljump{DSLTOOLPORT}{\var{DSLTOOL\_x\_PORT}} muss
  auf \var{'49000'} gesetzt werden.
  
  Wenn auf der Fritz!Box kein Username gesetzt ist, muss die Variable 
  \smalljump{DSLTOOLUSER}{\var{DSLTOOL\_x\_USER}}
  auf \var{'dslf-config'} gesetzt werden.

\item[bc63] Broadcom \smalljump{appendix:bc63}{bc63}

  Modems/Router basierend auf Broadcom bc63xx Chipsatz

\item[conexant] \smalljump{appendix:conexant}{Conexant}

  Modems/Router basierend auf Conexant Chipsatz

\item[openwrt] \smalljump{appendix:openwrt}{OpenWrt}

  Modems/Router basierend auf Broadcom bc631xx Chipsatz mit OpenWrt

\item[speedtouch] Thomson \smalljump{appendix:speedtouch}{Speedtouch}

  ALCATEL/Thomson 5x6 und 7x6 Modems/Router mit Firmware-Version 5.x und 6.x

\item[trendchip] \smalljump{appendix:trendchip}{Trendchip}

  Modems basierend auf dem Trendchip Chipsatz

\item[vigor] DrayTek \smalljump{appendix:vigor}{Vigor}

  DrayTek Vigor Modems

\item[vinax] Infineon \smalljump{appendix:vinax}{Vinax}

  Modems basierend auf dem Infineon/Lantiq Vinax Chipsatz

\item[demo-adsl]
\item[demo-vdsl] Demo-Modem (liefert Dummy-Werte)

  Die Variable \smalljump{DSLTOOLPROTOCOL}{\var{DSLTOOL\_x\_PROTOCOL}}
  muss auf \var{'demo'} gesetzt werden.

  Die Variablen \smalljump{DSLTOOLHOST}{\var{DSLTOOL\_x\_HOST}},
  \smalljump{DSLTOOLUSER}{\var{DSLTOOL\_x\_USER}} und
  \smalljump{DSLTOOLPASS}{\var{DSLTOOL\_x\_PASS}} werden beim Demo-Modem
  zwar nicht ausgewertet, dürfen aber nicht leer sein.
\end{description}

\config {DSLTOOL\_x\_PROTOCOL}{DSLTOOL\_x\_PROTCOL}{DSLTOOLPROTOCOL}

  Mit dieser optionalen Einstellung wird das verwendete Protokoll eingestellt.
  Gültige Wert sind \var{'telnet'} (default), \var{'http'} und \var{'demo'} 

\config {DSLTOOL\_x\_PORT}{DSLTOOL\_x\_PORT}{DSLTOOLPORT}

  Mit dieser optionalen Variablen wird der verwendete TCP-Port eingestellt.
  Wenn die Variable nicht vorhanden ist, wird der Standardport des jeweiligen 
  Protokolls verwendet (z.B. telnet: 23, http: 80). 
  
\config {DSLTOOL\_x\_ETHTYPE}{DSLTOOL\_x\_ETHTYPE}{DSLTOOLETHTYPE}

  Mit dieser optionalen Einstellung wird der Ethernet Typ eingestellt.
  Gültige Wert sind \var{'IPv4'} (default), \var{'IPv6'} und \var{'auto'}.

\config {DSLTOOL\_x\_HOST}{DSLTOOL\_x\_HOST}{DSLTOOLHOST}

  Hostname oder IP-Adresse des DSL-Modem's.

  Beispiel:

\begin{example}
\begin{verbatim}
    DSLTOOL_HOST='192.168.1.254'
\end{verbatim}
\end{example}

  \achtung{Achtung}, ein Netzwerk zum DSL-Modem muss konfiguriert sein.

  Z.B. in \var{base.txt} \var{IP\_NET\_3='192.168.1.1/24'} und
  \var{IP\_NET\_3\_DEV='eth3'},
  es genügt nicht nur \var{PPPOE\_ETH='eth3'} in \var{dsl.txt} zu setzen.
  Bitte vergessen Sie nicht, die Firewall Regeln anzupassen, damit das DSL-Modem
  auch erreicht wird (siehe \smalljump{appendix:iptables}{Beispiel}).

\config {DSLTOOL\_x\_USER}{DSLTOOL\_x\_USER}{DSLTOOLUSER}

  Der User-Name für die Anmeldung am DSL-Modem.

  Beispiel:

\begin{example}
\begin{verbatim}
    DSLTOOL_USER='Admin'
\end{verbatim}
\end{example}

\config {DSLTOOL\_x\_PASS}{DSLTOOL\_x\_PASS}{DSLTOOLPASS}

  Das Passwort für die Anmeldung am DSL-Modem.

  Beispiel:

\begin{example}
\begin{verbatim}
    DSLTOOL_PASS='Admin'
\end{verbatim}
\end{example}

\config {DSLTOOL\_x\_RRD}{DSLTOOL\_x\_RRD}{DSLTOOLRRD}

  Die Einstellung \var{'yes'} aktiviert die Datenaufzeichnung über collectd
  aus dem RRDTOOL Paket.
  Das RRDTOOL Paket muss mit \var{OPT\_RRDTOOL='yes'} aktiviert und
  die Option \var{RRDTOOL\_UNIXSOCK='yes'} gesetzt sein.

\config {DSLTOOL\_x\_DEBUG}{DSLTOOL\_x\_DEBUG}{DSLTOOLDEBUG}

  Mit der Einstellung \var{'yes'} wird eine Debug-Möglichkeit aktiviert
  Dazu muss tcpdump aus dem TOOLS Paket mit \var{OPT\_TCPDUMP='yes'} aktiviert sein.

  Im Webinterface kann über den Debug Tab eine Datenabfrage angestossen
  und heruntergeladen werden.

  Alternativ kann die Datenabfrage an der SSH-Konsole mit
  \var{/usr/bin/dsltool-dump.sh} gestartet werden.
  Die Daten werden in \var{/tmp/dsltool.tgz} gespeichert.

  In der Datei \var{dsltool.tgz} sind zu Analysezwecken die laufende Konfiguration
  des DSLTOOL Pakets, eine tcpdump-Aufzeichnung des Modem-Abfrage, sowie die
  Ausgabedaten enthalten.

  Da sowohl der Loginname für das Modem als auch das Passwort im Klartext in der
  Konfiguration und im Dumpfile enthalten sind sollte man das Passwort zu
  Debuggingzwecken ändern.

\config {DSLTOOL\_x\_LOG}{DSLTOOL\_x\_LOG}{DSLTOOLLOG}

  Mit der Einstellung \var{'yes'} wird das Schreiben von Logausgaben in eine Datei
  oder syslog (\smalljump{DSLTOOLSYSLOG}{\var{DSLTOOL\_x\_SYSLOG}} aktiviert.

\config {DSLTOOL\_x\_SYSLOG}{DSLTOOL\_x\_SYSLOG}{DSLTOOLSYSLOG}

  Mit der Einstellung \var{'yes'} wird in syslog protokolliert.
  Dazu muss die Variable \var{DSLTOOL\_x\_LOG='yes'} gesetzt sein.

\end{description}

