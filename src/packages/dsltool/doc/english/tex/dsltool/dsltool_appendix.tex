% Synchronized to r48723
\section{Tested DSL modems}
  Reports about additional successfully tested DSL modems are welcome.
  
\marklabel{appendix:amazon}{
\subsection{amazon}}
\begin{itemize}
\item Allnet ALL 0333 CJ
\end{itemize}

\marklabel{appendix:ar7}{
\subsection{ar7}}
\begin{itemize}
\item Funkwerk M22
\item Sphairon AR860
\item D-Link DSL-T380
\end{itemize}

\marklabel{appendix:avm-tr064}{
\subsection{avm-tr064}}
\begin{itemize}
\item Fritz!Box 3272 FW 6.30
\end{itemize}

\marklabel{appendix:bc63}{
\subsection{bc63}}
\begin{itemize}
\item D-Link DSL-321B
(Hardware Revision D\emph{x})
\item Zyxel VMG1312-B30A
\end{itemize}

\marklabel{appendix:conexant}{
\subsection{conexant}}
\begin{itemize}
\item Sphairon AR800
\end{itemize}

\marklabel{appendix:openwrt}{
\subsection{openwrt}}
\begin{itemize}
\item Technicolor DGA 4132
\end{itemize}

\marklabel{appendix:speedtouch}{
\subsection{speedtouch}}
\begin{itemize}
\item ALCATEL/Thomson Speedtouch 516i V6 FW 5.4.0.14
\item ALCATEL/Thomson Speedtouch 585i V6 FW 6.1.0.5
\item ALCATEL/Thomson Speedtouch 536i V6 FW 6.2.15.5
\end{itemize}

\marklabel{appendix:trendchip}{
\subsection{trendchip}}
\begin{itemize}
\item D-Link DSL-321B
(Hardware Revision Z\emph{x})
\end{itemize}

\marklabel{appendix:vigor}{
\subsection{vigor}}
\begin{itemize}
\item Vigor 130
\end{itemize}

\marklabel{appendix:vinax}{
\subsection{vinax}}
\begin{itemize}
\item Sphairon Speedlink 1113
\end{itemize}

\marklabel{sec:dsltoolexamples}{
\section{Examples}} 

\marklabel{appendix:iptables}{
\subsection{Package filter}}

\begin{example}
\begin{verbatim}
    # pppoe.txt
    CIRC_1_PPP_ETHERNET_DEV='ethY'

    # base.txt
    IP_NET_x='10.0.1.1/24' # internal net
    IP_NET_x_DEV='ethX'
    IP_NET_x_NAME='lan-admin'
    IP_NET_y='10.0.2.1/24' # modem 
    IP_NET_y_DEV='ethY'
    IP_NET_y_NAME='lan-modem'
\end{verbatim}
\end{example}

\marklabel{sec:dsltoolappendix}{
\section{Appendix}}
 
\subsection{Credits}

  The idea for DSLTOOL is based on the DSL modem tool {[\ref{dmt}]} written by
  Andreas Matthöfer, which requires Windows and is closed -source.

  The data is recorded by the help of collectd {[\ref{collectd}]} and displayed with 
  rrdtool {[\ref{rrdtool}]}.
  
  The spectrum graphs are created with cairo/pango {[\ref{cairo},\ref{pango}]}.
  
\subsection{References}

\newcounter{ref}
\begin{list}{\textbf{[\arabic{ref}]}}{\usecounter{ref}}
  
  \item \label{wiki}
  \altlink{https://ssl.nettworks.org/wiki/display/f/dsltool+-+Tipps}

  \item \label{dmt}
  \altlink{http://dmt.mhilfe.de/}
  
  \item \label{collectd}
  \altlink{http://www.collectd.org/}
  
  \item \label{rrdtool}
  \altlink{http://oss.oetiker.ch/rrdtool/}
  
  \item \label{cairo}
  \altlink{http://www.cairographics.org/}
  
  \item \label{pango}
  \altlink{http://www.pango.org/}
  
\end{list}
