% Synchronized to r53574
\section {DSLTOOL - DSL modem data recording and graphical display}

\subsection {Description}
The package DSLTOOL records data from a DSL modem by the help to the 'collectd'
daemon and stores it in a rrd databases.
The Web-GUI of the fli4l router allows to display the generated graphs.
\\
Among others the following data will be recorded and displayed:
\begin{itemize}
  \item Bit allocation
  \item Signal noise ration
  \item Attenuation
  \item Transmit power
  \item Errored Seconds
  \item Frame Error Counter
  \item CRC Error Counter
  \item Header Error Counter
\end{itemize}

\subsection {Supported DSL modems}

  At this time only the DSL modems listed in \smalljump{DSLTOOLMODEM}
  {\var{DSLTOOL\_x\_MODEM}} are supported.

  At the Wiki {[\ref{wiki}]} hints for configuration of particular modems
  can be found or added.

  If your DSL modem is not supported, please send an eMail to check
  if supporting this modem type is possible.

  To test the features of the DSL tools without having a supported DSL modem,
  it is possible to set \smalljump{DSLTOOLMODEM}{\var{DSLTOOL\_x\_MODEM}} to
  'demo-adsl' or 'demo-vdsl' mode.

\subsection {Configuration of the DSLTOOL package}

  The configuration is made, as of all fli4l packages, by adjusting the file\\
  \var{path/fli4l-\version/$<$config$>$/dsltool.txt} to meet your own demands.

\begin{description}

\config {OPT\_DSLTOOL}{OPT\_DSLTOOL}{OPTDSLTOOL}

  The setting \var{'no'} deactivates OPT\_DSLTOOL completely. There will be no changes
  made on the fli4l boot medium or the archive \var{opt.img}.
  OPT\_DSLTOOL does not overwrite other parts of the fli4l installation.
  To activate OPT\_DSLTOOL set the variable \var{OPT\_DSLTOOL} to \var{'yes'}..

\config {DSLTOOL\_N}{DSLTOOL\_N}{DSLTOOLN}

  Defines the number of used DSL modems.

\config {DSLTOOL\_x\_MODEM}{DSLTOOL\_x\_MODEM}{DSLTOOLMODEM}

  Selects the DSL modem type. The variable can be set to the following values:

\begin{description}
\item[amazon] Infineon \smalljump{appendix:amazon}{Amazon SE}

  Modems/router based on Infineon/Lantiq Amazon SE chipset

\item[ar7] TI \smalljump{appendix:ar7}{AR7}

  Modems/router based on Texas Instruments AR7 chipset family

\item[avm-tr064] AVM \smalljump{appendix:avm-tr064}{Fritz!Box}

  AVM Fritz!Box Router (Firmware $\ge$ 5.50)

  If no username is configured on the Fritz!Box then the variable 
  \smalljump{DSLTOOLUSER}{\var{DSLTOOL\_x\_USER}}
  needs to be set to \var{'dslf-config'}.

\item[bc63] Broadcom \smalljump{appendix:bc63}{bc63}

  Modems/router based on Broadcom bc63xx chipset

\item[conexant] \smalljump{appendix:conexant}{Conexant}

  Modems/router based on Conexant chipset

\item[openwrt] \smalljump{appendix:openwrt}{OpenWrt}

  Modems/router based on Broadcom bc631xx chipset with OpenWrt

\item[speedtouch] Thomson \smalljump{appendix:speedtouch}{Speedtouch}

  ALCATEL/Thomson 5x6 and 7x6 modems/router with firmware version 5.x and 6.x

\item[trendchip] \smalljump{appendix:trendchip}{Trendchip}

  Modems based on Trendchip chipset

\item[vigor] DrayTek \smalljump{appendix:vigor}{Vigor}

  DrayTek Vigor modems

\item[vinax] Infineon \smalljump{appendix:vinax}{Vinax}

  Modems/router based on Infineon/Lantiq Vinax chipset

\item[demo-adsl]
\item[demo-vdsl] Demo modem (gives sample values)


  The variable \smalljump{DSLTOOLPROTOCOL}{\var{DSLTOOL\_x\_PROTOCOL}}
  must be set to \var{'demo'}.

  In demo mode the variables \smalljump{DSLTOOLHOST}{\var{DSLTOOL\_x\_HOST}},
  \smalljump{DSLTOOLUSER}{\var{DSLTOOL\_x\_USER}} and
  \smalljump{DSLTOOLPASS}{\var{DSLTOOL\_x\_PASS}} are not evaluated
  but may not be empty.
\end{description}

\config {DSLTOOL\_x\_PROTOCOL}{DSLTOOL\_x\_PROTCOL}{DSLTOOLPROTOCOL}

  This optional setting defines the protocol used.
  Valid values are \var{'telnet'} (default) and \var{'demo'} 

\config {DSLTOOL\_x\_PORT}{DSLTOOL\_x\_PORT}{DSLTOOLPORT}

  This optional variable defines the TCP port used.
  If this variable is not present, the default port of the corresponding 
  protocol is used (e.g. telnet: 23, http: 80). 
  
\config {DSLTOOL\_x\_ETHTYPE}{DSLTOOL\_x\_ETHTYPE}{DSLTOOLETHTYPE}

  This optional setting defines the ethernet type.
  Valid values are \var{'IPv4'} (default), \var{'IPv6'} and \var{'auto'}.


\config {DSLTOOL\_x\_HOST}{DSLTOOL\_x\_HOST}{DSLTOOLHOST}

  Hostname or IP Address of the DSL modem.

  Example:

\begin{example}
\begin{verbatim}
    DSLTOOL_HOST='192.168.1.254'
\end{verbatim}
\end{example}

  \achtung{Attention}, a network route to the DSL modem must be configured.

  E.g. set \var{IP\_NET\_3='192.168.1.1/24'} and \var{IP\_NET\_3\_DEV='eth3'}
  in \var{base.txt}. It is not sufficient to set \var{PPPOE\_ETH='eth3'} only in \var{dsl.txt}.
  Don't forget to adapt the firewall rules to allow communication with the DSL modem
  (see \smalljump{appendix:iptables}{example}).

\config {DSLTOOL\_x\_USER}{DSLTOOL\_x\_USER}{DSLTOOLUSER}

  The user name for the login to the DSL modem.

  Example:

\begin{example}
\begin{verbatim}
    DSLTOOL_USER='Admin'
\end{verbatim}
\end{example}

\config {DSLTOOL\_x\_PASS}{DSLTOOL\_x\_PASS}{DSLTOOLPASS}

  The password for the login to the DSL modem.

  Example:

\begin{example}
\begin{verbatim}
    DSLTOOL_PASS='Admin'
\end{verbatim}
\end{example}

\config {DSLTOOL\_x\_RRD}{DSLTOOL\_x\_RRD}{DSLTOOLRRD}

  The setting \var{'yes'} activates the data recording with the collectd daemon
  from the RRDTOOL package.
  The RRDTOOL package must be activated with \var{OPT\_RRDTOOL='yes'} and
  the option \var{RRDTOOL\_UNIXSOCK='yes'} must be set.

\config {DSLTOOL\_x\_DEBUG}{DSLTOOL\_x\_DEBUG}{DSLTOOLDEBUG}

  The setting \var{'yes'} activates a debug option. To use it, the
  program tcpdump (to be found in the TOOLS package) needs to be activated
  by specifying \var{OPT\_TCPDUMP='yes'} in the TOOLS package's configuration file.

  Data recording may be started using the Web-GUI's debug tab and the
  data recorded will be downloaded.

  The data recording can be started from the SSH-shell as well by executing
  \var{/usr/bin/dsltool-dump.sh}.
  The data recorded will be stored in the file \var{/tmp/dsltool.tgz}.

  The file \var{dsltool.tgz} will log the actual configuration of the DSLTOOL package,
  a tcpdump capture of the modem communication and the output data for later analysis.

  The DSL modem's login name and password are stored in readable format
  in both configuration and capture file, hence the password should be changed for
  debugging purposes.

\config {DSLTOOL\_x\_LOG}{DSLTOOL\_x\_LOG}{DSLTOOLLOG}

  The setting  \var{'yes'} activates the output of log messages to a file
  or syslog (\smalljump{DSLTOOLSYSLOG}{\var{DSLTOOL\_x\_SYSLOG}}.

\config {DSLTOOL\_x\_SYSLOG}{DSLTOOL\_x\_SYSLOG}{DSLTOOLSYSLOG}

  The setting \var{'yes'} activates logging to syslog.
  The variable \var{DSLTOOL\_x\_LOG='yes'} must be set.

\end{description}

