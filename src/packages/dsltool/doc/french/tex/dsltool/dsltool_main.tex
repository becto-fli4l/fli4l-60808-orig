% Synchronized to r53574
\section {DSLTOOL~- Traitement des données et graphiques pour modem DSL}

\subsection {Description}
Le paquetage DSLTOOL recueille les données du modem à l'aide du démon 'collectd'
et stock ces informations dans une base de donnée RRD. Le paquetage rrdtool
affichera les graphiques sur les informations générées dans l'interface Web
du routeur fli4l.
\\
Voici les données enregistrées qui peuvent être affichées~:
\begin{itemize}
  \item L'allocation de Bit
  \item La Marge de bruit
  \item L'atténuation
  \item La puissance d'émission
  \item Les erreurs par seconde
  \item Un compteur d'erreur de trame
  \item Un compteur d'erreur CRC
  \item Un compteur d'erreur d'en-tête
\end{itemize}

\subsection {Les modems DSL supportés}

  Les modems prisent en charges à ce jour sont répertoriés dans la liste
  \smalljump{DSLTOOLMODEM}{\var{DSLTOOL\_x\_MODEM}}.

  Dans le wiki {[\ref{wiki}]}, vous pouvez voir la configuration ou ajouter
  une configuration de modems.

  Si votre modem DSL n'est pas supporté, s'il vous plaît envoyez nous un
  courriel pour vérifier si ce type de modem est possible d'être ajouté
  à la liste.

  Vous avez un outil DSL pour tester les fonctionnalités, sans avoir de modem
  DSL pris en charge. Dans la variable \smalljump{DSLTOOLMODEM}{\var{DSLTOOL\_x\_MODEM}}
  il est possible de définir le mode 'demo-adsl' ou 'demo-vdsl'.

\subsection {Configuration du paquetage DSLTOOL}

  La configuration se fait comme les autres paquetages fli4l, en paramètrant\\
  le fichier \var{path/fli4l-\version/$<$config$>$/dsltool.txt} selon votre
  propre configuration.

\begin{description}

\config {OPT\_DSLTOOL}{OPT\_DSLTOOL}{OPTDSLTOOL}

  La valeur \var{'no'} dans cette variable désactive complètement le paquetage
  OPT\_DSLTOOL. Il n'y aura aucun changement sur le support de boot de l'archive
  fli4l \var{rootfs.img} n'y dans l'archive \var{opt.img}. Pour finir OPT\_DSLTOOL
  n'écrase aucune partie de l'installation fli4l.\\
  Pour activer la variable OPT\_DSLTOOL dans \var{OPT\_DSLTOOL} vous devez
  placer la valeur \var{'yes'}.

\config {DSLTOOL\_N}{DSLTOOL\_N}{DSLTOOLN}

  Dans cette variable vous indiquez le nombre de modem DSL que vous utilisé

\config {DSLTOOL\_x\_MODEM}{DSLTOOL\_x\_MODEM}{DSLTOOLMODEM}

  Dans cette variable vous sélectionnez le type de modem DSL, vous pouvez
  indiquez les valeurs suivante~:

\begin{description}

\item[amazon] Infineon \smalljump{appendix:amazon}{Amazon SE}

  Modem/routeur basé sur le chipset Infineon/Lantiq Amazon SE

\item[ar7] TI \smalljump{appendix:ar7}{AR7}

  Modem/routeur basé sur la famille de chipset Texas Instruments AR7

\item[avm-tr064] AVM \smalljump{appendix:avm-tr064}{Fritz!Box}

  Routeur AVM Fritz!Box (Firmware $\ge$ 5.50)

  Si aucun nom d'utilisateur n'est configuré sur le routeur Fritz!Box,
  vous devez indiquer le paramètre \var{'dslf-config'} dans la variable
  \smalljump{DSLTOOLUSER}{\var{DSLTOOL\_x\_USER}}.

\item[bc63] Broadcom \smalljump{appendix:bc63}{bc63}

  Modem/routeur basé sur le chipset Broadcom bc63xx

\item[conexant] \smalljump{appendix:conexant}{Conexant}

  Modem/routeur basé sur le chipset Conexant
  
\item[openwrt] \smalljump{appendix:openwrt}{OpenWrt}

  Modem/routeur basé sur le chipset Broadcom bc631xx avec OpenWrt  

\item[speedtouch] Thomson \smalljump{appendix:speedtouch}{Speedtouch}

  Modem/routeur ALCATEL/Thomson 5x6 et 7x6 avec la version du firmware 5.x ou 6.x

\item[trendchip] \smalljump{appendix:trendchip}{Trendchip}

  Modem/routeur basé sur le chipset Trendchip

\item[vigor] DrayTek \smalljump{appendix:vigor}{Vigor}

  Modem DrayTek Vigor

\item[vinax] Infineon \smalljump{appendix:vinax}{Vinax}

  Modem/routeur basé sur le chipset Infineon/Lantiq Vinax

\item[demo-adsl]
\item[demo-vdsl] Modem de démo (fournit une valeur factice)

  Vous devez indiquer dans La variable \smalljump{DSLTOOLPROTOCOL}{\var{DSLTOOL\_x\_PROTOCOL}}
  le paramétré \var{'demo'}.

  Les variables \smalljump{DSLTOOLHOST}{\var{DSLTOOL\_x\_HOST}},
  \smalljump{DSLTOOLUSER}{\var{DSLTOOL\_x\_USER}} et
  \smalljump{DSLTOOLPASS}{\var{DSLTOOL\_x\_PASS}} ne sont pas exploitées avec
  un modem demo mais ne peuvent pas être vide.

\end{description}

\config {DSLTOOL\_x\_PROTOCOL}{DSLTOOL\_x\_PROTCOL}{DSLTOOLPROTOCOL}

  Dans cette variable optionnelle vous indiquez le protocole à utiliser.
  Les valeurs possibles sont \var{'telnet'} (par défaut) et \var{'demo'}

\config {DSLTOOL\_x\_PORT}{DSLTOOL\_x\_PORT}{DSLTOOLPORT}

  Dans cette variable optionnelle vous indiquez le port TCP à utiliser.
  Si la variable n'est pas présente ou pas configurée, le port par défaut
  pour chaque protocole sera utilisé (par exemple telnet~: 23, http~: 80).

\config {DSLTOOL\_x\_ETHTYPE}{DSLTOOL\_x\_ETHTYPE}{DSLTOOLETHTYPE}

  Dans cette variable optionnelle vous indiquez le type d'Ethernet.
  Les valeurs possibles sont \var{'IPv4'} (par défaut), \var{'IPv6'} et \var{'auto'}.

\config {DSLTOOL\_x\_HOST}{DSLTOOL\_x\_HOST}{DSLTOOLHOST}

  Vous indiquez dans cette variable le nom d'hôte ou l'adresse IP du modem

  Exemple~:

\begin{example}
\begin{verbatim}
    DSLTOOL_HOST='192.168.1.254'
\end{verbatim}
\end{example}

  \achtung{Attention}, le réseau doit être configuré pour le modem DSL

  Par exemple dans le fichier \var{base.txt}, il y a \var{IP\_NET\_3='192.168.1.1/24'}
  et \var{IP\_NET\_3\_DEV='eth3'}, il ne suffit pas d'indiquer \var{PPPOE\_ETH='eth3'}
  dans le fichier \var{dsl.txt}. Il ne faut pas oublier de configurer les règles
  du pare-feu pour communiquer avec le modem DSL (voir \smalljump{appendix:iptables}{exemple}).

\config {DSLTOOL\_x\_USER}{DSLTOOL\_x\_USER}{DSLTOOLUSER}

  Vous indiquez dans cette variable le nom d'utilisateur (ou login) pour la
  connexion au modem DSL.

  Exemple~:

\begin{example}
\begin{verbatim}
    DSLTOOL_USER='Admin'
\end{verbatim}
\end{example}

\config {DSLTOOL\_x\_PASS}{DSLTOOL\_x\_PASS}{DSLTOOLPASS}

  Vous indiquez dans cette variable le mot de passe pour la connexion 
  au modem DSL.

  Exemple~:

\begin{example}
\begin{verbatim}
    DSLTOOL_PASS='Admin'
\end{verbatim}
\end{example}

\config {DSLTOOL\_x\_RRD}{DSLTOOL\_x\_RRD}{DSLTOOLRRD}

  La valeur \var{'yes'} dans cette variable active l'enregistrement des
  informations avec le démon collectd du paquetage RRDTOOL. Il faut bien sur
  que le paquetage RRDTOOL soit activé avec la variable \var{OPT\_RRDTOOL='yes'}
  et la variable optionnelle \var{RRDTOOL\_UNIXSOCK='yes'}.

\config {DSLTOOL\_x\_DEBUG}{DSLTOOL\_x\_DEBUG}{DSLTOOLDEBUG}

  Si vous indiquez \var{'yes'} dans cette variable, vous activez l'option de débogage.
  Pour cela, il est nécessaire d'utiliser tcpdump du paquetage tools et d'activer
  la variable \var{OPT\_TCPDUMP='yes'}.

  Vous pouvez activer l'enregistrement et le téléchargement les données dans l'onglet
  débogage de l'interface Web du routeur.

  Vous pouvez aussi utiliser la console SSH avec la commande \var{/usr/bin/dsltool-dump.sh}
  pour enregistrer les données. Les données seront stockées dans le fichier \var{/tmp/dsltool.tgz}.

  Le fichier \var{dsltool.tgz} du paquetage DSLTOOL sera ensuite utilisé pour être
  analysé en fonction de la configuration en cours, l'outil tcpdump capture
  les requêtes du modem ainsi que les données sortante.

  Le nom et le mot de passe pour la connexion au modem sont paramétrés dans un format
  lisible dans le fichier de configuration, ils seront aussi lisible dans le fichier
  de données pour le débogage. Vous devez par la suite modifier le mot de passe pour
  votre sécurité.

\config {DSLTOOL\_x\_LOG}{DSLTOOL\_x\_LOG}{DSLTOOLLOG}

  Si vous indiquez le paramètre \var{'yes'} vous activez les messages de journalisation
  qui seront enregistrés dans un fichier, pour cela il faut que syslog soit activé avec
  la variable (\smalljump{DSLTOOLSYSLOG}{\var{DSLTOOL\_x\_SYSLOG}}.

\config {DSLTOOL\_x\_SYSLOG}{DSLTOOL\_x\_SYSLOG}{DSLTOOLSYSLOG}

  Vous devez indiquez le paramètre \var{'yes'} pour activer syslog.
  il faut aussi activer la variable \var{DSLTOOL\_x\_LOG='yes'}.

\end{description}

