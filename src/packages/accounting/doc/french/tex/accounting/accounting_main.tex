% Synchronized to r29817
\marklabel{sec:accounting}
{
\section {ACCOUNTING~- Détail sur l'enregistrement du trafic}
}

Cette OPT vise à détecter et analyser le trafic de chaque client aussi
précisément que possible~:
\begin{itemize}
\item Les données seront stockées octet par octet une fois par jour.
\item Le trafic est enregistré sur toutes les interfaces.
\item Les requêtes seront affichés sous forme de rapports quotidiens ou mensuels
sur l'interface web. Il est même possible de définir librement la période.
\item Le trafic entrant et sortant est affiché séparément ou sous la forme d'une somme.
\item Les valeurs peuvent être affichés en octets, kilo-octets, méga-octets ou giga-octets.
\end{itemize}

\subsection{Configuration (Standard)}
Les variables suivantes doivent être configurés dans le fichier accounting.txt~:

\begin{description}
\config{OPT\_ACCOUNTING}{OPT\_ACCOUNTING}{OPTACCOUNTING}
La configuration par défaut est \verb*?'no'?.
\verb*?'yes'? active le paquetage.

\config{ACCOUNTING\_DIR}{ACCOUNTING\_DIR}{ACCOUNTINGDIR}
La configuration par défaut est \verb*?'/boot/data/accounting'?.
Vous spécifiez ici la direction pour les données. Il ne faut pas l'utiliser
un disque RAM. Si vous souhaitez récupérer les données dans un disque RAM 
(peut-être pour sauvegarder les accès du disque dur) jeter un oeil sur les
paquetages OPT\_SARFILE ou OPT\_CPMVRMLOG pour sauvegarder et restaurer vos
précieuses données.\newline
Le répertoire \verb*?'/var/run/accounting'? peut pas être considérée comme un
répertoire de données ou être utilisé comme cible pour sauvegarder et restaurer
des opérations.

\config{ACCOUNTING\_INT}{ACCOUNTING\_INT}{ACCOUNTINGINT}
La configuration par défaut est \verb*?'pppoe'?.
Vous indiquez dans cette variable l'interface utilisée par fli4l pour l'accés à
Internet. Vous pouvez indiquer plusieurs interfaces, ils seront séparées par
un espace.
Exemple~:
\begin{example}
\begin{verbatim}
    ACCOUNTING_INT='pppoe'                 # DSL
    ACCOUNTING_INT='circuit-1 circuit-2'   # ISDN
    ACCOUNTING_INT='IP_NET_x_DEV'          # Ethernet
\end{verbatim}
\end{example}

\config{ACCOUNTING\_CRON}{ACCOUNTING\_CRON}{ACCOUNTINGCRON}
La configuration par défaut est \verb*?'55 * * * *'?.
Ce paramètre est facultatif et peut être ignoré. Avec ce réglage accounting.sh
sera exécuté à intervalle définis. Pour de plus amples informations sur la
syntaxe lire s'il vous plaît la documentation du paquetage easycron. Il convient
de s'assurer que les données du trafic entre deux mises à jour ne dépasse pas
4Go. Une mise à jour avant la fin de la journée est utile, car le trafic sera
affecté à cette journée. Le réglage par défaut devrait donc être suffisante
pour indiquer une bande passante de 9Mbit/s. Cette option nécessite que le
paquetage easycron soit installé.

\config{ACCOUNTING\_LEARNIPS}{ACCOUNTING\_LEARNIPS}{ACCOUNTINGLEARNIPS}
La configuration par défaut est \verb*?'no'?.
Ce paramètre est facultatif et peut être ignoré. Si vous placez la valeur sur
\verb*?'yes'? le script acclearnips.sh sera copié sur le routeur fli4l et
exécutera \smalljump{ACCOUNTINGLEARNIPSINTERVAL}{\var{ACCOUNTING\_LEARNIPS\_INTERVAL}}
toutes les minutes. Le cache ARP du routeur sera surveillé et une règle
accounting sera créée pour les nouvelles adresses IP. Cette option nécessite
que le paquetage easycron soit installé.
\end{description}

\subsection{Configuration (Experts)}
vous devez configurer les variables suivantes du fichier accounting.txt,
uniquement dans des cas exceptionnels.

\begin{description}

\config{ACCOUNTING\_VPNINT}{ACCOUNTING\_VPNINT}{ACCOUNTINGVPNINT}
La configuration par défaut est \verb*?''?.
Ce paramètre est facultatif et peut être ignoré. Vous indiquez dans cette
variable les interfaces VPN (tun0, tun1, ...) à surveiller. Vous pouvez
indiquer plusieurs interfaces, ils seront séparées par un espace.

\config{ACCOUNTING\_LEARNIPS\_INTERVAL}{ACCOUNTING\_LEARNIPS\_INTERVAL}{ACCOUNTINGLEARNIPSINTERVAL}
La configuration par défaut est \verb*?'5'?.
Ce paramètre est facultatif et peut être ignoré. Vous indiquez dans cette
variable l'intervalle temps pour l'exécution de acclearnips.sh (voir
\smalljump{ACCOUNTINGLEARNIPS}{\var{ACCOUNTING\_LEARNIPS?}}).
Si cette variable est vide, un intervalle temps de 5 minutes sera utilisé
par défaut.

\config{ACCOUNTING\_LEARNFROMINT}{ACCOUNTING\_LEARNFROMINT}{ACCOUNTINGLEARNFROMINT}
Ce paramètre est facultatif et peut être ignoré. En utilisant ce paramètre,
vous pouvez spécifier les interfaces à partir de laquelle les adresses IP seront
étudiées. Par défaut, toutes les interfaces sont surveillés. Cela peut être utile
si fli4l est utilisé comme un routeur Ethernet et que l'adresse IP de votre
fournisseur est connus. Vous pouvez indiquer plusieurs interfaces, ils seront
séparées par un espace.

\config{ACCOUNTING\_METHOD}{ACCOUNTING\_METHOD}{ACCOUNTINGMETHOD}
La configuration par défaut est \verb*?'new'?.
Ce paramètre est facultatif et peut être ignoré. Avec la méthode accounting
précédente (\verb*?'old'?) tout le trafic était acheminé à travers la chaîne
de règle accounting, les règles supplémentaires était utilisées pour exclure
le trafic entre les réseaux masqués. la chaîne de règle Postrouting était lu.
Mais cela ne fonctionnerait qu'avec
\begin{example}
\begin{verbatim}
    POSTROUTING_LIST_1='IP_NET_1 MASQUERADE'
\end{verbatim}
\end{example}
dans le fichier base.txt mais pas avec
\begin{example}
\begin{verbatim}
    POSTROUTING_LIST_1='if:any:pppoe MASQUERADE'.
\end{verbatim}
\end{example}
Avec la nouvelle méthode (\verb*?'new'?), le trafic est transmit uniquement à
l'interface qui est définie dans \smalljump{ACCOUNTINGINT}{\var{ACCOUNTING\_INT}}
pour la chaîne de règle accounting. Maintenant la chaîne peut être lu
\begin{example}
\begin{verbatim}
    POSTROUTING_LIST_1='if:any:pppoe MASQUERADE'
\end{verbatim}
\end{example}
cela n'est plus un problème.

\config{ACCOUNTING\_LOCALTRAF}{ACCOUNTING\_LOCALTRAF}{ACCOUNTINGLOCALTRAF}
La configuration par défaut est \verb*?'no'?.
Ce paramètre est facultatif et peut être ignoré. Normalement, seul le trafic
routé est compté par accounting. Si un proxy est installé sur fli4l le trafic
ne seront pas routées et les valeurs accounting resteront à zéro. Vous pouvez
définir la variable \verb*?ACCOUNTING_LOCALTRAF? sur \verb*?'yes'? pour éviter
cela. Le trafic du routeur vers le client et vice versa sera géré avec la chaîne
de la règle accounting. L'inconvénient, c'est que tout le trafic entre le routeur
et le client sera considéré comme du trafic Internet. Cela inclut les mises à
jour, SSH, SCP, FTP, httpd, et ainsi de suite. La différence peut être
rapidement négative.

\config{ACCOUNTING\_MAXINT}{ACCOUNTING\_MAXINT}{ACCOUNTINGMAXINT}
La configuration par défaut est \verb*?'4294967296'?.
Ce paramètre est facultatif et doit être utilisé dans de cas particulièrement
rares. La valeur indique le montant maximal en octets donc l'interface peut
resevoir avant que le débordement soit atteint. Cette variable doit être
spécifié que si l'interface n'atteintle PAS une maximum de 4294967296 bytes
(4Go). Si vous n'êtes pas sûr, ne toucher pas ce paramètre.

\config{ACCOUNTING\_DEBUG\_INT}{ACCOUNTING\_DEBUG\_INT}{ACCOUNTINGDEBUGINT}
La configuration par défaut est \verb*?'no'?.
Ce paramètre est facultatif et peut être ignoré. Si vous placez la valeur sur
\verb*?'yes'? un fichier appelé int.log sera créé pour enregistrer le calcul
des données de l'interface.
\end{description}

\subsection{Les droits du Httpd}
Accounting supporte l'attribution des droits du httpd.
Exemple pour httpd.txt~:
\begin{example}
\begin{verbatim}
    ...
    HTTPD_USER_N='2'
    HTTPD_USER_1_USERNAME='admin'
    HTTPD_USER_1_PASSWORD='secret'
    HTTPD_USER_1_RIGHTS='all'
    HTTPD_USER_2_USERNAME='accounting'
    HTTPD_USER_2_PASSWORD='0815'
    HTTPD_USER_2_RIGHTS='accounting:view'
    ...
\end{verbatim}
\end{example}
L'utilisateur admin a tous les droits, y compris les droits accounting,
l'utilisateur accounting ne dispose que des droits accounting.

\subsection{Utilisation du navigateur}
L'utilisation du navigateur web (httpd) devrait être évident. Je voudrais faire
un bref commentaire à ce sujet. Vous pouvez cliquer dans la colonne à gauche
du mois, du nom d'hôte, de l'adresses IP, du nom de l'interfaces et de la date 
pour naviguer. En cliquant sur rafraîchir accounting.sh s'exécute.

\subsection{Données et index}
Dans le répertoire de données, un répertoire est créé pour chaque année et
les données sont stockées dans des fichiers mensuels. Les adresses IP et
des interfaces attribuées obtennent un numéro unique et seront enregistrés
dans le fichier index.acc.

Exemple~:
\begin{example}
\begin{verbatim}
    ACCOUNTING_HOST_n='3'
    ACCOUNTING_HOST_1_IP='192.168.6.1'
    ACCOUNTING_HOST_1_NAME='client1'
    ACCOUNTING_HOST_2_IP='192.168.6.2'
    ACCOUNTING_HOST_2_NAME='client2'
    ACCOUNTING_HOST_3_IP='192.168.6.3'
    ACCOUNTING_HOST_3_NAME='client3'
    ACCOUNTING_INT_n='3'
    ACCOUNTING_INT_1_NAME='ppp0'
    ACCOUNTING_INT_2_NAME='eth0'
    ACCOUNTING_INT_3_NAME='lo'
\end{verbatim}
\end{example}
La variable \var{ACCOUNTING\_HOST\_x\_NAME} est facultatif. Les noms seront
affichés sur l'interface web si la résolution de noms à été indexé. Le fichier
index.acc peut être édité directement sur le routeur. Vous pouvez choisir
l'éditeur que vous aimez dans le paquetage tools.

\begin{description}
\item[Attention] Les données sont enregistrées par un numéro d'index. Des
modifications du numéro d'index sur l'affectation des adresses IP ou du des
noms d'interfaces mèneront à des données non valides.
\end{description}

\subsection{FAQ}
\begin{description}
\item[Question] Quelle est la différence~?
\item[Répondre] Seul le trafic routé par les clients inscrits seront détectés
par les règles accounting. L'interface Internet capte tout le trafic vers / à
partir d'Internet. La différence peut par exemple être le trafic entre un client
non enregistré, un proxy, le serveur de messagerie, le téléchargement d'outils
sur le routeur et le TCP/IP général.
\end{description}

\begin{description}
\item[Question] Depuis que j'ai installé OPT\_ACCOUNTING, le routeur sélectionne
régulièrement un accés Internet, même si aucun ordinateur dans le réseau local
n'est démarré. Comment puis-je arrêter ça~?
\item[Répondre] Mettez la valeur dans la variable \verb*?DNS_BOGUS_PRIV? sur
'yes' dans le fichier base.txt. (Il s'agit de la valeur par défaut.)
\end{description}

\begin{description}
\item[Question] Le trafic sur l'interface ppp0 est trés élevé de plusieurs
gigaoctet de données, pourquoi une telle différence~?
\item[Répondre] Le problème peut survenir si vous utilisez la variable
\verb*?DIALMODE='auto'? dans base.txt et la variable \verb*?PPPOE_HUP_TIMEOUT='0'?
dans dsl.txt en même temps. Pour être en ligne 24h/24 la variable
\verb*?PPPOE_HUP_TIMEOUT? doit avoir une valeur élevée (par exemple 86400), et
adresse régulièrement des requêtes au fournisseur. Depuis fli4l-2.1.12 vous
pouvez définir dans la variable \verb*?PPPOE_HUP_TIMEOUT='never'? cela empêche
fli4l d'arrêter la connexion. Pour moi, je préfère éviter ce paramètre et
utiliser la méthode mentionnée précédemment.
\end{description}
