% Synchronized to r43684

\section {MAILSEND - Envoi de courriel}

\subsection {Description}

    Le paquetage MAILSEND est une API (ou application) qui sert à envoyer du courriel,
	il est utile pour les autres paquetages.

    Le paquetage EASYCRON est nécessaire lors de l'installation, vous devez donc activer
	la variable (\var{OPT\_EASYCRON='yes'}).

\subsection {Configuration du paquetage MAILSEND}

    La configuration se fait comme tous les autres paquetages fli4l, en paramétrant
	le fichier\\
	\var{chemin/fli4l-\version/\emph{\flq config\frq}/mailsend.txt} selon vos besoins.

\subsubsection {Généralité}
\begin{description}

\config {OPT\_MAILSEND}{OPT\_MAILSEND}{OPTMAILSEND}

    Si vous indiquez \var{'no'} l'OPT\_MAILSEND sera complètement désactivé.
	Il n'y aura aucun changement dans le support de boot \var{rootfs.img} et dans
	l'archive \var{opt.img} de fli4l.
	De plus l'installation de l'OPT\_MAILSEND ne remplacera jamais aucune autre partie
	de l'installation fli4l.

    Pour activer l'OPT\_MAILSEND vous devez indiquer dans la variable \var{OPT\_MAILSEND}
	le paramètre \var{'yes'}.

\config {MAILSEND\_SPOOL\_DIR}{MAILSEND\_SPOOL\_DIR}{MAILSENDSPOOLDIR}

    Vous indiquez dans cette variable optionnelle le répertoire spool (ou mémoire
	tampon). Le répertoire spool doit est placé sur une partition persistante.

    Le paramètre par défaut est \var{'/data/spool/mail'}.

\config {MAILSEND\_LOGFILE}{MAILSEND\_LOGFILE}{MAILSENDLOGFILE}

    Dans cette variable optionnelle vous pouvez paramétrer le fichier journal.
	Si vous voulez que le système enregistre le fichier journal vous devez indiquer
	\var{'syslog'}.

    Si vous ne voulez pas de fichier journal vous n'indiquez rien \var{'{}'}
	dans la variable (paramètre par défaut).

\config {MAILSEND\_SEND\_DELAYED}{MAILSEND\_SEND\_DELAYED}{MAILSENDSENDDELAYED}

    Si vous indiquez \var{'yes'} dans cette variable, le courriel sera envoyé plus tard,
	en utilisant une tâche du programme cron ou en exécutant un événement avec ip-up.
	Si vous indiquez \var{'no'} le courriel sera envoyé immédiatement.

\config {MAILSEND\_CHARSET}{MAILSEND\_CHARSET}{MAILSENDCHARSET}

    Dans cette variable optionnelle vous indiquez le jeu de caractères pour l'encodage
	des caractères. Le paramètre utilisé par défaut est \var{'ISO-8859-1'}, vous pouvez
	aussi indiquer par exemple \var{'UTF-8'}.

\config {MAILSEND\_N}{MAILSEND\_N}{MAILSENDN}

    Dans cette variable vous indiquez le nombre de compte de messagerie à installer.

\end{description}

\subsubsection {Compte de messagerie}

    Dans les variables suivantes vous allez paramétrer les comptes de messagerie, chaque
	compte de messagerie est à configurer séparément.

\begin {description}

\config {MAILSEND\_\emph{x}\_ACCOUNT}{MAILSEND\_\emph{x}\_ACCOUNT}{MAILSENDACCOUNT}

    Dans cette variable vous indiquez un nom logique du compte de messagerie que vous
	allez utiliser.

	L'utilisation de plusieurs comptes de messagerie seront distingués par le nom,
	par conséquent, ce nom doit être différent pour chaque compte installé.

	Si vous n'indiquez pas de nom pour le compte \var{'par défaut'}, le compte avec
	l'index \var{'1'} sera adressé, se sera également son nom.

\config {MAILSEND\_\emph{x}\_AUTH}{MAILSEND\_\emph{x}\_AUTH}{MAILSENDAUTH}

    Dans cette variable vous indiquez la méthode d'authentification.
	Les valeurs possibles sont~:

    \begin {itemize}
      \item [\var{'none'}] pas d'authentification
      \item [\var{'pop'}] POP3 avant le SMTP
      \item [\var{'plain'}] mot de passe en clair
      \item [\var{'cram-md5'}] valeurs de hachage MD5
      \item [\var{'login'}] mot de passe en clair (similaire à \var{'plain'})
    \end {itemize}

\config {MAILSEND\_\emph{x}\_CRYPT}{MAILSEND\_\emph{x}\_CRYPT}{MAILSENDCRYPT}

    Dans cette variable vous indiquez la méthode de cryptage.
	Les valeurs possibles sont~:

    \begin {itemize}
      \item [\var{'none'}] pas de cryptage
      \item [\var{'TLS'}] SSL/TLS
      \item [\var{'STARTTLS'}] TLS avec STARTTLS
    \end {itemize}

\config {MAILSEND\_\emph{x}\_FROM}{MAILSEND\_\emph{x}\_FROM}{MAILSENDFROM}

    Vous indiquez dans cette variable optionnelle l'adresse de l'expéditeur.

    Si cette variable est vide, vous pouvez aussi utiliser la variable
    \smalljump{MAILSENDSMTPUSERNAME}{\var{MAILSEND\_\emph{x}\_SMTP\_USERNAME}}.

    L'adresse de l'expéditeur peut être écrasé lorque vous utilisez
	\smalljump{sec:mailsendapi}{API} en ligne de commande.

\end{description}

\subsubsection {Serveur SMTP}
\begin {description}

\config {MAILSEND\_\emph{x}\_SMTP\_SERVER}{MAILSEND\_\emph{x}\_SMTP\_SERVER}{MAILSENDSMTPSERVER}

    Dans cette variable vous indiquez le serveur SMTP, vous pouvez paramétrer
	le nom d'hôte ou l'adresse IP.

\config {MAILSEND\_\emph{x}\_SMTP\_PORT}{MAILSEND\_\emph{x}\_SMTP\_PORT}{MAILSENDSMTPPORT}

    Vous indiquez dans cette variable optionnelle le numéro de port du SMTP,
	la valeur par défaut du port SMTP est \var{'25'}.

\config {MAILSEND\_\emph{x}\_SMTP\_USERNAME}{MAILSEND\_\emph{x}\_SMTP\_USERNAME}{MAILSENDSMTPUSERNAME}

    Dans cette variable vous indiquez le nom d'utilisateur pour l'authentification
	du SMTP.

\config {MAILSEND\_\emph{x}\_SMTP\_PASSWORD}{MAILSEND\_\emph{x}\_SMTP\_PASSWORD}{MAILSENDSMTPPASSWORD}

    Dans cette variable vous indiquez le mot de passe pour l'authentification
	du SMTP.

\end {description}

\subsubsection {Optionnel~: serveur POP3}

    Ces paramètres sont nécessaires seulement si vous avez indiquez la valeur \var{'pop'}
	dans la variable \smalljump{MAILSENDAUTH}{\var{MAILSEND\_\emph{x}\_AUTH}}.

\begin {description}
\config {MAILSEND\_\emph{x}\_POP3\_SERVER}{MAILSEND\_\emph{x}\_POP3\_SERVER}{MAILSENDPOP3SERVER}

    Vous indiquez dans cette variable le serveur POP3, vous pouvez paramétrer
	le nom d'hôte ou l'adresse IP.

\config {MAILSEND\_\emph{x}\_POP3\_PORT}{MAILSEND\_\emph{x}\_POP3\_PORT}{MAILSENDPOP3PORT}

    Vous indiquez dans cette variable optionnelle le numéro de port du POP3,
	la valeur par défaut du port POP3 est \var{'110'}.

\config {MAILSEND\_\emph{x}\_POP3\_USERNAME}{MAILSEND\_\emph{x}\_POP3\_USERNAME}{MAILSENDPOP3USERNAME}

    Vous indiquez dans cette variable le nom d'utilisateur du POP3 avant
	l'authentification du SMTP.

\config {MAILSEND\_\emph{x}\_POP3\_PASSWORD}{MAILSEND\_\emph{x}\_POP3\_PASSWORD}{MAILSENDPOP3PASSWORD}

    Vous indiquez dans cette variable le mot de passe du POP3 avant
	l'authentification du SMTP.

\end {description}

\subsubsection {Fonctionnalités avancées}

\paragraph {Certificat du serveur}

	Avec ces variables vous pouvez tester le certificat du serveur.

\begin {description}

\config {MAILSEND\_\emph{x}\_TEST\_SCERT}{MAILSEND\_\emph{x}\_TEST\_SCERT}{MAILSENDTESTSCERT}

    Si vous indiquez \var{'yes'} dans cette variable, le certificat \smalljump{MAILSENDSCERT}{MAILSEND\_\emph{x}\_SCERT}
	du serveur SMTP sera testé.

\config {MAILSEND\_\emph{x}\_SCERT}{MAILSEND\_\emph{x}\_SCERT}{MAILSENDSCERT}

    Dans cette variable vous indiquez le nom du fichier y compris le chemin du
	certificat X.509 du serveur avec l'extention PEM. Par exemple
	\var{'/etc/mailsend/server-cert.pem'}

    Le chemin du fichier \var{'/etc/mailsend/\emph{server}.pem'} est recommandé,
	car le certificat est placé dans le répertoire de configuration sous
	\var{\emph{\flq config\frq}/etc/mailsend/}, il sera utilisé à l'installation.

\end {description}

\paragraph {Certificat/clé client}
\begin {description}

\config {MAILSEND\_\emph{x}\_USE\_CCERT}{MAILSEND\_\emph{x}\_USE\_CCERT}{MAILSENDUSECCERT}

    Si le paramètre est sur \var{'yes'}, le SMTP devient alors client, la variable
	\smalljump{MAILSENDCCERT}{MAILSEND\_\emph{x}\_CCERT} pour le certificat et la variable
	\smalljump{MAILSENDCKEY}{MAILSEND\_\emph{x}\_CKEY} pour la clé privée seront utilisées
	pour l'authentification.

\config {MAILSEND\_\emph{x}\_CCERT}{MAILSEND\_\emph{x}\_CCERT}{MAILSENDCCERT}

    Dans cette variable vous indiquez le nom du fichier du certificat X.509 avec
	l'extention PEM, y compris le chemin d'accés pour le client. \\
	Par exemple \var{'/etc/mailsend/client-cert.pem'}

\config {MAILSEND\_\emph{x}\_CKEY}{MAILSEND\_\emph{x}\_CKEY}{MAILSENDCKEY}

    Dans cette variable vous indiquez le nom du fichier de la clé X.509 avec
	l'extention PEM, y compris le chemin d'accés pour le client. \\
	Par exemple \var{'/etc/mailsend/client-key.pem'}

\end {description}

\marklabel{sec:mailsendapi}{
\subsection {API}}
\subsubsection{\var{mailsend}}

    L'API (ou l'application) \var {/usr/local/bin/mailsend} peut être utilisée
	en ligne de commande pour envoyer un courriel, par défaut le texte et
	les paramètres seront entre guillemet, vous pouvez aussi utiliser un pipe
	ou tube (séparateur vertical), (voir l'annexe \smalljump{sec:mailsendexample}{Exemple}).

    L'API peut être utilisée avec les options suivantes.

\begin {itemize}
  \item [\var{-t},] \var{-{}-to} \var{"\emph{\flq address\frq}"}
    \\Requis~: lors de l'envoi d'un courriel, si vous utilisez une ou
	plusieurs adresses de réception vous devez les séparer par une virgule.
	\\Vous pouvez définir cette option plusieurs fois.

  \item [\var{-s},] \var{-{}-subject} \var{"\emph{\flq subject\frq}"}
    \\Requis~: vous devez indiquer l'objet du courriel.

  \item [\var{-A},] \var{-{}-account} \var{"\emph{\flq account\frq}"}
    \\Optionnel~: vous pouvez indiquer le \smalljump{MAILSENDACCOUNT}{nom du compte}.
	Si l'option est omise, le \smalljump{MAILSENDACCOUNT}{nom} \var{par défaut}
	sera utilisé.

  \item [\var{-f},] \var{-{}-from} \var{"\emph{\flq address\frq}"}
    \\Optionnel~: vous pouvez indiquer l'adresse de l'expéditeur.
	Si l'option est omise, l'adresse de la variable
	\smalljump{MAILSENDFROM}{\var{MAILSEND\_\emph{x}\_FROM}} sera utilisée.

  \item [\var{-c},] \var{-{}-cc} \var{"\emph{\flq address\frq}"}
    \\Optionnel~: vous pouvez indiquer une ou plusieurs adresses de destination CC,
	vous devez les séparer par une virgule, une copie du courriel sera ainsi
	envoyée aux destinataires.
	\\Vous pouvez également définir cette option plusieurs fois.

  \item [\var{-b},] \var{-{}-bcc} \var{"\emph{\flq address\frq}"}
    \\Optionnel~: vous pouvez indiquer une ou plusieurs adresses de destination BCC,
	vous devez les séparer par une virgule, une copie du courriel sera ainsi
	envoyée à chaque destinataires (le non des autres destinataires sera cachés).
	\\Vous pouvez également définir cette option plusieurs fois.

  \item [\var{-r},] \var{-{}-reply-to} \var{"\emph{\flq address\frq}"}
    \\Optionnel~: vous pouvez indiquer une ou plusieurs adresses avec une réponse,
	vous devez les séparer par une virgule, le courriel envoyé au destinataire
	demandera une réponse de celui-ci.
	\\Vous pouvez également définir cette option plusieurs fois.

  \item [\var{-B},] \var{-{}-body} \var{"\emph{\flq file\frq}"}
    \\Optionnel~: vous pouvez mettre le texte du message dans un fichier
	en indiquant son (chemin absolu).
	\\Si cette option est manquante, le texte du message sera envoyé avec un paramètre
	ou avec un pipe, (voir \smalljump{sec:mailsendexample}{Exemple}).

  \item [\var{-a},] \var{-{}-attach-to} \var{"\emph{\flq file\frq}"}
    \\Optionnel~: vous pouvez ajouter au courriel une pièce jointe en indiquant
	son (chemin absolu).
	\\Vous pouvez définir cette option plusieurs fois.
	\\Vous pouvez intégrer plusieurs pièces jointe, les fichiers seront separés
	par un point-virgule (;).

  \item [\var{-C},] \var{-{}-charset} \var{"\emph{\flq charset\frq}"}
    \\Optionnel~: vous pouvez indiquer un jeu de caractères pour l'encodage
	des caractères. Si l'option est omise, le jeu de caractères de la variable
	\smalljump{MAILSENDCHARSET}{\var{MAILSEND\_CHARSET}} sera utilisé \var{par défaut}.
\end{itemize}

\subsubsection{\var{mailsend} mailfile}

	Alternativement, un mailfile (ou fichier de messagerie) compatible avec la norme
	\smalljump{rfc5322}{RFC5322} peut être envoyé. Tous les paramètres intégrés à
	ce fichier seront utilisés pour l'envoi du courriel.

	Les paramètres (tels que \var{-{}-to}) ne doivent pas être utilisés~!

\begin {itemize}
  \item [\var{-m},] \var{-{}-mailfile} \var{"\emph{\flq file\frq}"}
    \\Requis~: le nom du fichier de messagerie avec son (chemin absolu).

  \item [\var{-A},] \var{-{}-account} \var{"\emph{\flq account\frq}"}
    \\Optionnel~: vous pouvez indiquer le \smalljump{MAILSENDACCOUNT}{nom du compte}
	avec lequel le courriel sera envoyé. Si l'option est omise,
	le \smalljump{MAILSENDACCOUNT}{nom} \var{par défaut} sera utilisé.

\end{itemize}
