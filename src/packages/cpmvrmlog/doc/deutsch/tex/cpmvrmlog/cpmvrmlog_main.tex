% Last Update: $Id$
\section {OPT\_CPMVRMLOG - kopieren/verschieben/löschen/sichern von LOG-Dateien}

OPT\_CPMVRMLOG erweitert fli4l um eine Funktion um z.b. LOG-Dateien zu kopieren,
zu verschieben, zu löschen oder zu sichern.


\textbf{Disclaimer: }\emph{Der Autor gibt weder eine Garantie auf die
Funktionsfähigkeit des OPT\_\-CPMVRMLOG, noch haftet er für Schäden, z.B.
Datenverlust, die durch den Einsatz von OPT\_\-CPMVRMLOG entstehen.}


\marklabel{sec:konfigcpmvrmlog}{
\subsection {Konfiguration des OPT\_CPMVRMLOG}
}

Die Konfiguration erfolgt, wie bei allen fli4l Opts, durch Anpassung der Datei
\var{Pfad/fli4l-\version/$<$config$>$/cpmvrmlog.txt} an die eigenen Anforderungen.


\begin{description}

\config {OPT\_CPMVRMLOG}{OPT\_CPMVRMLOG}{OPTCPMVRMLOG}

  Default: \var{OPT\_CPMVRMLOG='no'}

  Die Einstellung \var{'no'} deaktiviert das OPT\_CPMVRMLOG vollständig. Es werden
  keine Änderungen an dem fli4l Bootmedium bzw. dem Archiv \var{opt.img}
  vorgenommen. Weiter überschreibt das OPT\_CPMVRMLOG grundsätzlich keine anderen
  Teile der fli4l Installation.\\
  Um OPT\_CPMVRMLOG zu aktivieren, ist die Variable \var{OPT\_CPMVRMLOG} auf 
  \var{'yes'} zu setzen. Sollte das nötige Paket OPT\_EASYCRON nicht aktiv sein,
  wird beim generieren der neuen fli4l-Dateien nach dem Aufruf von mkfli4l.sh
  oder mkfli4l.bat eine Warnmeldung ausgegeben.

  
\config {CPMVRMLOG\_VERBOSE}{CPMVRMLOG\_VERBOSE}{CPMVRMLOGVERBOSE}

  Default: \var{CPMVRMLOG\_VERBOSE='yes'}

  \var{CPMVRMLOG\_VERBOSE} schaltet die Ausgabe von zusätzlichen Meldungen
  an bzw. aus.\\

\config {CPMVRMLOG\_COMPRESS}{CPMVRMLOG\_COMPRESS}{CPMVRMLOGCOMPRESS}

  Default: \var{CPMVRMLOG\_COMPRESS='yes'}
  
  \var{CPMVRMLOG\_COMPRESS} aktiviert die Kompression der Dateien, die kopiert bzw. 
  verschoben werden. Diese Einstellung wird bei 'backup' Operationen ignoriert.\\
 
\config {CPMVRMLOG\_DEVRAM\_CHECK}{CPMVRMLOG\_DEVRAM\_CHECK}{CPMVRMLOGDEVRAMCHECK}

  Default: \var{CPMVRMLOG\_DEVRAM\_CHECK='yes'}

  \var{CPMVRMLOG\_DEVRAM\_CHECK} schaltet die regelmäßige Prüfung des freien Platzes
  in der Ramdisk des rootfs an bzw. aus.\\
  
\config {CPMVRMLOG\_DEVRAM\_FREE}{CPMVRMLOG\_DEVRAM\_FREE}{CPMVRMLOGDEVRAMFREE}

  Default: \var{CPMVRMLOG\_DEVRAM\_FREE='250'}

  \var{CPMVRMLOG\_DEVRAM\_FREE} legt den minimalen Wert an freien Blöcken in der
  Ramdisk fest. Sollte der Wert unterschritten werden, werden die Aktionen 'move'
  oder 'remove' durchgeführt.\\

\config {CPMVRMLOG\_N}{CPMVRMLOG\_N}{CPMVRMLOGN}

  Default: \var{CPMVRMLOG\_N='1'}

  \var{CPMVRMLOG\_N} legt die Anzahl der aktiven nachfolgenden Aktionsabschnitte fest.\\


\config {CPMVRMLOG\_x\_ACTION}{CPMVRMLOG\_x\_ACTION}{CPMVRMLOGxACTION}

  \var{CPMVRMLOG\_x\_ACTION} legt fest, ob kopiert (copy), gelöscht (delete). 
  verschoben (move) oder gesichert (backup) werden soll.
  Die mit der Aktion 'backup' gesicherten (kopierten) Dateien werden beim 
  Systemstart wiederhergestellt. \var{CPMVRMLOG\_COMPRESS} und
  \var{CPMVRMLOG\_MAXCOUNT} werden ignoriert.\\
 
\config {CPMVRMLOG\_x\_SOURCE}{CPMVRMLOG\_x\_SOURCE}{CPMVRMLOGxSOURCE}

  \var{CPMVRMLOG\_x\_SOURCE} ist der Name inclusive komplettem Pfad der
  zu bearbeitenden Datei oder des Verzeichnisses. \\
  
\config {CPMVRMLOG\_x\_DESTINATION}{CPMVRMLOG\_x\_DESTINATION}{CPMVRMLOGxDESTINATION}

  \var{CPMVRMLOG\_x\_DESTINATION} legt den Zielpfad fest (nur bei 'copy', 'move'
  und 'backup'). Wenn \var{CPMVRMLOG\_x\_SOURCE} ein Verzeichnisname ist, wird 
  hier das Verzeichnis angegeben, in welches das Quell-Verzeichnis kopiert oder 
  verschoben werden soll.\\

\config {CPMVRMLOG\_x\_CUSTOM}{CPMVRMLOG\_x\_CUSTOM}{CPMVRMLOGxCUSTOM}

  über \var{CPMVRMLOG\_x\_CUSTOM} kann nach der ausgeführten Aktion ein 
  zusätzliches Programm oder ein Befehl ausgeführt
  werden. So ist es z.B. unbedingt nötig, per 'killall -HUP syslogd' dem syslogd
  das Signal '-HUP' zu schicken, damit dieser nach einem 'delete' oder 'move' 
  eine neue Logdatei anlegt und benutzt. Anstelle eines einzelnen Shell-Befehls 
  kann auch eine durch ';' getrennte Befehlskette angegeben werden.\\

\config {CPMVRMLOG\_x\_MAXCOUNT}{CPMVRMLOG\_x\_MAXCOUNT}{CPMVRMLOGxMAXCOUNT}

  \var{CPMVRMLOG\_x\_MAXCOUNT} legt die maximale Anzahl der Archiv-Dateien 
  fest. Wird ignoriert, wenn \var{CPMVRMLOG\_x\_ACTION} den Wert 'backup' 
  oder 'delete' hat. Wenn diese Variable einen Wert größer 0 hat, wird den 
  verarbeiteten Dateien oder Verzeichnissen ein numerischer Suffix angehängt.
  Bei bereits vorhandenen Kopien wird der Suffix
  um eins hochgezählt, die älteste Version wird (wenn der Wert von 
  \var{CPMVRMLOG\_x\_MAXCOUNT} erreicht ist) gelöscht.\\
  Wird ein ganzer Verzeichnisbaum kopiert oder verschoben, erhalten nur die 
  Dateien und Verzeichnisse auf der obersten Ebene diesen Suffix.\\
  Wenn für \var{CPMVRMLOG\_x\_MAXCOUNT} der Wert '-1' angegeben wird, erhält die
  kopierte bzw. verschobene Datei einen Zeitstempel im Format yyyy-mm-dd-hh:mm 
  angehängt. Diese Variante eignet sich besonders zum zeitlich unbeschränkten,
  regelmäßigen Sichern von Logdateien. Der Administrator ist in diesem Fall 
  selbst für die Überwachung des freien Platzes auf dem verwendeten Speichermedium 
  (Festplatte, Speicherkarte, ...) verantwortlich.

\config {CPMVRMLOG\_x\_CRONTIME}{CPMVRMLOG\_x\_CRONTIME}{CPMVRMLOGxCRONTIME}

  \var{CPMVRMLOG\_x\_CRONTIME} legt in CRON-Syntax (siehe Paket EASYCRON) 
  das Intervall fest, in dem die Aktion ausgeführt wird. Diese Variable kann 
  bei der Aktion 'backup' leer bleiben; in diesem Fall werden die angegebenen 
  Dateien nur beim Shutdown des Routers weggesichert.\\

\end{description}
  

\subsection{Literatur}

\marklabel{url:CPMVRMLOGhpauthor}{
 Homepage des Authors:  \altlink{http://www.lan4me.de/}
 }

\marklabel{url:CPMVRMLOGfli4lnews}{
 fli4l Newsgroups und ihre Spielregeln: \altlink{http://www.fli4l.de/hilfe/newsgruppen/}
}
