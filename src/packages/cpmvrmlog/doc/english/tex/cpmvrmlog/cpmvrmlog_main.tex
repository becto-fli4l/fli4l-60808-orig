% Synchronized to r30050

\section {OPT\_CPMVRMLOG - copy/move/delete logfiles}

OPT\_CPMVRMLOG enables fli4l to copy, move or delete files to other destinations (for example logfiles).

\textbf{Disclaimer: }\emph{The author gives neither a guarantee of functionality 
nor is he liable for any damage or the loss of data when using OPT\_\-CPMVRMLOG.}


\marklabel{sec:konfigcpmvrmlog}{
\subsection {Configuration Of OPT\_CPMVRMLOG}
}

Like always when using fli4l, the configuration is done by adjusting the file\\
\var{fli4l-\version/$<$config$>$/cpmvrmlog.txt} to your own demands.

\begin{description}

\config {OPT\_CPMVRMLOG}{OPT\_CPMVRMLOG}{OPTCPMVRMLOG}

  Default: \var{OPT\_CPMVRMLOG='no'}

  The setting \var{'no'} deactivates OPT\_CPMVRMLOG completely. There will be no changes made
  to the fli4l boot medium or the archive \var{opt.img}.
  OPT\_CPMVRMLOG does not overwrite other parts of the fli4l installation.

  To activate OPT\_CPMVRMLOG set the variable \var{OPT\_CPMVRMLOG} to \var{'yes'}.
  If the package OPT\_EASYCRON is not activated, a warning will be presented while
  building the new fli4l files because OPT\_CPMVRMLOG depends on this package.
  
\config {CPMVRMLOG\_VERBOSE}{CPMVRMLOG\_VERBOSE}{CPMVRMLOGVERBOSE}

  Default: \var{CPMVRMLOG\_VERBOSE='yes'}

  \var{CPMVRMLOG\_VERBOSE} enables or disables additional messages (i.e. for debugging).\\

\config {CPMVRMLOG\_COMPRESS}{CPMVRMLOG\_COMPRESS}{CPMVRMLOGCOMPRESS}

  Default: \var{CPMVRMLOG\_COMPRESS='yes'}
  
  \var{CPMVRMLOG\_COMPRESS} activates compression for the files being copied or moved.
  Ignored on 'backup' operations.\\
 
\config {CPMVRMLOG\_DEVRAM\_CHECK}{CPMVRMLOG\_DEVRAM\_CHECK}{CPMVRMLOGDEVRAMCHECK}

  Default: \var{CPMVRMLOG\_DEVRAM\_CHECK='yes'}

  \var{CPMVRMLOG\_DEVRAM\_CHECK} enables the periodical check for free space in the 
  ramdisk or rootfs. Possible values are \var{'on'} or \var{'off'}.\\
  
\config {CPMVRMLOG\_DEVRAM\_FREE}{CPMVRMLOG\_DEVRAM\_FREE}{CPMVRMLOGDEVRAMFREE}

  Default: \var{CPMVRMLOG\_DEVRAM\_FREE='250'}

  \var{CPMVRMLOG\_DEVRAM\_FREE} specifies the minimum value for free blocks in the ramdisk.
  If less space is detected, the actions 'move' or 'remove' are processed.\\

\config {CPMVRMLOG\_N}{CPMVRMLOG\_N}{CPMVRMLOGN}

  Default: \var{CPMVRMLOG\_N='1'}

  \var{CPMVRMLOG\_N} sets the number of action blocks following.\\

\config {CPMVRMLOG\_x\_ACTION}{CPMVRMLOG\_x\_ACTION}{CPMVRMLOGxACTION}


  sets if copy, delete, move or backup should be used.
  Files saved by the action 'backup' are restored during the next boot. \var{CPMVRMLOG\_COMPRESS} 
  and \var{CPMVRMLOG\_MAXCOUNT} are ignored.\\
 
\config {CPMVRMLOG\_x\_SOURCE}{CPMVRMLOG\_x\_SOURCE}{CPMVRMLOGxSOURCE}

  specifies the name of the file or folder to be processed (incl. full path).\\
  
\config {CPMVRMLOG\_x\_DESTINATION}{CPMVRMLOG\_x\_DESTINATION}{CPMVRMLOGxDESTINATION}

  sets the destination path (only for 'copy', 'move' and 'backup').
  If \var{CPMVRMLOG\_x\_SOURCE} is a directory name the destination
  directory to where the source folder should be copied or moved has to be
  specified here.\\

\config {CPMVRMLOG\_x\_CUSTOM}{CPMVRMLOG\_x\_CUSTOM}{CPMVRMLOGxCUSTOM}

  With \var{CPMVRMLOG\_x\_CUSTOM} an additional command can be defined after the 
  execution of the action. For example: it is absolutely necessary to do a 
  'killall -HUP syslogd' to tell the syslogd to create a new logfile after a 
  'delete' or 'move'. More than one command is possible by separating the commands 
  with ';'.\\

\config {CPMVRMLOG\_x\_MAXCOUNT}{CPMVRMLOG\_x\_MAXCOUNT}{CPMVRMLOGxMAXCOUNT}

  sets the maximum number of archive files. It will be ignored if
  \var{CPMVRMLOG\_x\_ACTION} is set to 'backup' or 'delete'. Values greater than 0 
  add a numeric suffix to the files or folders being processed.
  Suffixes on existing copies are increased, the oldest version will be deleted (if
  \var{CPMVRMLOG\_x\_MAXCOUNT} is reached).\\
  When copying or moving trees of folders, the suffix is only added to the files and 
  folders at the top level.\\
  If the value of \var{CPMVRMLOG\_x\_MAXCOUNT} is set to '-1', the copied or moved
  files get a timestamp added (yyyy-mm-dd-hh:mm). This is useful for unlimited regular
  saving of logfiles. The administrator is responsible for monitoring the free space
  on the destination device. 

\config {CPMVRMLOG\_x\_CRONTIME}{CPMVRMLOG\_x\_CRONTIME}{CPMVRMLOGxCRONTIME}

  \var{CPMVRMLOG\_x\_CRONTIME} Sets the interval in which the action should be executed.
  Use CRON syntax here (for details see package OPT\_EASYCRON). The variable may be empty
  on action 'backup'. In this case the files are saved during router shutdown.\\

\end{description}
  

\subsection{Literature}

\marklabel{url:CPMVRMLOGhpauthor}{
 The author's homepage:  \altlink{http://www.lan4me.de/}
 }

\marklabel{url:CPMVRMLOGfli4lnews}{
 fli4l Newsgroups and the rules: \altlink{http://www.fli4l.de/hilfe/newsgruppen/}
}
