% Synchronized to r30050
\section {OPT\_CPMVRMLOG~- Copier/Déplacer/Supprimer les fichiers LOG}

Avec le paquetage OPT\_CPMVRMLOG vous étendez les fonctions de fli4l,
par exemple pour copier, pour déplacer, pour supprimer ou sauvegarder
les fichiers LOG (ou journal).

\textbf{Disclaimer: }\emph{L'auteur ne garantie pas la fonctionnalité et n'ai pas
non plus responsable de tous les dommages que peut occasionner le paquetage
OPT\_\-CPMVRMLOG. Y compris la perte de données, découlant d'une mauvaise
utilisation de l'OPT\_\-CPMVRMLOG.}

\marklabel{sec:konfigcpmvrmlog}{
\subsection {Configuration de l'OPT\_CPMVRMLOG}
}

La configuration se fait, comme pour tous les opt fli4l, en adaptant le fichier
\var{Chemin/fli4l-\version/$<$config$>$/cpmvrmlog.txt} selon vos besoins.

\begin{description}

\config {OPT\_CPMVRMLOG}{OPT\_CPMVRMLOG}{OPTCPMVRMLOG}

  Par défaut~: \var{OPT\_CPMVRMLOG='no'}

  La valeur \var{'no'} dans cette variable désactive complètement le paquetage
  OPT\_CPMVRMLOG. Il n'y aura aucun changement sur le support de boot de
  l'archive \var{rootfs.img} n'y dans l'archive \var{opt.img} de fli4l. Pour
  finir OPT\_CPMVRMLOG n'écrase aucune partie de l'installation fli4l.\\
  Pour activer la variable OPT\_CPMVRMLOG dans \var{OPT\_CPMVRMLOG} vous devez
  placer la valeur \var{'yes'}. Si le paquetage OPT\_EASYCRON est installé mais
  n'ai pas actif un message d'avertissement sera affiché lors de la construction
  de fli4l, en utilisant le programme mkfli4l.sh, ou mkfli4l.bat.\\

\config {CPMVRMLOG\_VERBOSE}{CPMVRMLOG\_VERBOSE}{CPMVRMLOGVERBOSE}

  Par défaut~: \var{CPMVRMLOG\_VERBOSE='yes'}

  Avec la variable \var{CPMVRMLOG\_VERBOSE} vous activez ou désactivez les
  messages supplémentaires.\\

\config {CPMVRMLOG\_COMPRESS}{CPMVRMLOG\_COMPRESS}{CPMVRMLOGCOMPRESS}

  Par défaut~: \var{CPMVRMLOG\_COMPRESS='yes'}

  Avec la variable \var{CPMVRMLOG\_COMPRESS} vous activez ou désactivez la
  compression des fichiers qui seront copiés ou déplacés. Ce paramètre est
  ignoré lorsqu'une opérations de sauvegarde est effectuée.\\

\config {CPMVRMLOG\_DEVRAM\_CHECK}{CPMVRMLOG\_DEVRAM\_CHECK}{CPMVRMLOGDEVRAMCHECK}

  Par défaut~: \var{CPMVRMLOG\_DEVRAM\_CHECK='yes'}

  Avec la variable \var{CPMVRMLOG\_DEVRAM\_CHECK} vous activez ou désactivez un
  examen réguliè de l'espace libre dans le disque RAM du rootfs.\\

\config {CPMVRMLOG\_DEVRAM\_FREE}{CPMVRMLOG\_DEVRAM\_FREE}{CPMVRMLOGDEVRAMFREE}

  Par défaut~: \var{CPMVRMLOG\_DEVRAM\_FREE='250'}

  Avec la variable \var{CPMVRMLOG\_DEVRAM\_FREE} vous définissez la valeur
  minimum dans le disque RAM de bloc libre. Si la valeur tombe en dessous de
  la valeur indiquée, le 'déplacement' ou la 'suppression' des données seront
  effectuées.\\

\config {CPMVRMLOG\_N}{CPMVRMLOG\_N}{CPMVRMLOGN}

  Par défaut~: \var{CPMVRMLOG\_N='1'}

  Avec la variable \var{CPMVRMLOG\_N} vous spécifiez le nombre d'actions
  à activer.\\

\config {CPMVRMLOG\_x\_ACTION}{CPMVRMLOG\_x\_ACTION}{CPMVRMLOGxACTION}

  Avec la variable \var{CPMVRMLOG\_x\_ACTION} vous indiquez les actions à
  effectuer, copier (copy) supprimer (delete), déplacer (move), ou sauvegarder
  (backup). Après l'action 'backup' les fichiers (copiés) pourront être
  restaurés au redémarrage du système. les variables \var{CPMVRMLOG\_COMPRESS}
  et \var{CPMVRMLOG\_MAXCOUNT} sont ignorées pendant un backup.\\

\config {CPMVRMLOG\_x\_SOURCE}{CPMVRMLOG\_x\_SOURCE}{CPMVRMLOGxSOURCE}

  Avec la variable \var{CPMVRMLOG\_x\_SOURCE} vous indiquez le chemin complet,
  y compris le nom du fichier ou du répertoire source.\\

\config {CPMVRMLOG\_x\_DESTINATION}{CPMVRMLOG\_x\_DESTINATION}{CPMVRMLOGxDESTINATION}

  Avec la variable \var{CPMVRMLOG\_x\_DESTINATION} vous indiquez le chemin de
  destination, uniquement (pour 'copy', 'move' et 'backup'). Si dans la variable
  \var{CPMVRMLOG\_x\_SOURCE} vous avez indiqué un nom de répertoire, ce répertoire
  devrat être spécifié ici pour pouvoir copier ou déplacer les données dans ce
  répertoire.\\

\config {CPMVRMLOG\_x\_CUSTOM}{CPMVRMLOG\_x\_CUSTOM}{CPMVRMLOGxCUSTOM}

  Avec la variable \var{CPMVRMLOG\_x\_CUSTOM} vous indiquez une commande
  supplémentaire après l'exécution d'une action. Par exemple, il est absolument
  nécessaire d'effectuer 'killall -HUP syslogd' après 'delete' ou 'move', c'est
  pour dire au démon syslogd de créer un nouveau fichier log (ou journal). Il
  est possible d'ajouter plus d'une commande, il faut pour cela séparer les
  commandes avec un point virgule ';'.\\

\config {CPMVRMLOG\_x\_MAXCOUNT}{CPMVRMLOG\_x\_MAXCOUNT}{CPMVRMLOGxMAXCOUNT}

  Avec la variable \var{CPMVRMLOG\_x\_MAXCOUNT} vous indiquez le nombre maximum
  de fichiers à archiver. Cette valeur est ignoré si la variable
  \var{CPMVRMLOG\_x\_ACTION} est paramétré sur 'backup' ou 'delete'. Si cette
  variable a une valeur supérieure à 0, un suffixe numérique sera ajouté aux
  fichiers ou aux dossiers en cours de traitement. Le suffixe des copies
  existantes sera incrémentée, si (la valeur dans \var{CPMVRMLOG\_x\_MAXCOUNT}
  est atteinte) la plus ancienne version sera supprimée.\\
  Lors de la copie ou du déplacement d'un dossiers, le suffixe sera ajouté au
  niveau supérieur du dossier de l'arborescence.\\ Si vous indiquez la valeur
  '-1' dans la variable \var{CPMVRMLOG\_x\_MAXCOUNT}, les fichiers copiés ou
  déplacé seront horodater dans le format yyyy-mm-dd-hh:mm. Cette configuration
  est particulièrement adaptée pour des sauvegardes illimitées et régulières
  du fichier log. Dans ce cas, l'administrateur est responsable de la
  surveillance de l'espace utile sur le support de stockage (disque dur,
  carte mémoire, ...).\\

\config {CPMVRMLOG\_x\_CRONTIME}{CPMVRMLOG\_x\_CRONTIME}{CPMVRMLOGxCRONTIME}

  Avec la variable \var{CPMVRMLOG\_x\_CRONTIME} vous pouvez indiquer une
  syntaxe CRON (voir le paquetage EASYCRON) pour effectuer une action dans un
  intervalle temps. Cette variable peut rester vide si vous utilisez l'action
  'backup'.\\

\end{description}

\subsection{Littérature}

\marklabel{url:CPMVRMLOGhpauthor}{
Page d'accueil de l'auteur~: \altlink{http://www.lan4me.de/}
 }

\marklabel{url:CPMVRMLOGfli4lnews}{
Les newsgroups de fli4l et leurs règles~: \altlink{http://www.fli4l.de/hilfe/newsgruppen/}
}
