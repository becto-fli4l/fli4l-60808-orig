% Last Update: $Id$
\marklabel{sec:faxrcv}
{
\section{FAXRCV}
}

\subsection {OPT\_FAXRCV - Faxempfang mit der AVM-Capi }
\configlabel{OPT\_FAXRCV}{OPTFAXRCV}

  Dieses Paket ermöglicht ihnen den Faxempfang mit einer CAPI-fähigen
  ISDN-Karte. Welche Treiber das sind, erfahren Sie aus der ISDN-Doku.

  Außerdem sollten Sie bezüglich Ihrer Hardware folgende Hinweise beachten:

  Das Empfangen von Faxen mit passiven ISDN-Karten ist ein verhältnismäßig
  CPU-lastiger Vorgang.
  Nach meinen Erfahrungen funktioniert es mit einem Pentium absolut reibungslos.

\begin{description}

\config{FAXRCV\_START\_AT\_BOOT}{FAXRCV\_START\_AT\_BOOT}{FAXRCVSTARTATBOOT}
    {Wenn diese Variable auf 'yes' gesetzt ist, wird der Faxempfang während
    des Bootvorgangs aktiviert. Ist sie hingegen auf 'no', muss der Faxempfang
    zunächst über die Weboberfläche oder an der Konsole mittels
    'faxrcv.sh start' gestartet werden.

        Standard-Einstellung: \var{FAXRCV\_START\_AT\_BOOT}='yes'}

\config{FAXRCV\_N}{FAXRCV\_N}{FAXRCVN}

    {Hier wird die Anzahl der einzurichtenden capifaxrcvd-Dienste
    angegeben, d.h. auf wie vielen MSN Sie Faxe empfangen wollen.
    Dies ist z.B. sinnvoll, wenn Sie für verschiedene Benutzer
    separate MSN und Verzeichnisse angeben wollen. Wollen Sie nur
    einen capifaxrcvd starten, geben Sie '1' ein!

        Standard-Einstellung: \var{FAXRCV\_N}='1'}

\config{FAXRCV\_x\_CSID}{FAXRCV\_x\_CSID}{FAXRCVxCSID}

    {Hier wird die \dq{}Station ID\dq{} angegeben, die der Daemon an anrufende
    Faxgeräte sendet. Dies sollte die eigene Nummer sein, wobei der
    capifaxrcvd offensichtlich nur Ziffern und einige Sonderzeichen
    übertragen kann.

    Eine sinnvolle CSID wäre also bspw. eine im unten angegebenen Format...
    Leerzeichen sind durch '-' zu ersetzen (123-456)!

    Danke an Jan Zude für einige \dq{}Forschungsarbeit\dq{} in diesem Bereich.

        Standard-Einstellung: \var{FAXRCV\_x\_CSID}='+49(00)00000'}

\config{FAXRCV\_x\_NUMBER}{FAXRCV\_x\_NUMBER}{FAXRCVxNUMBER}

    {Hier wird die MSN angegeben, auf der der Daemon Faxe empfangen soll.
    Im Normalfall ist das die eigene Nummer ohne Vorwahl.

    \achtung{Telefonanlagen können diese Nummer in manchen Fällen
    verändern!} Wenn Sie sich nicht sicher sind, können Sie auf Ihrer
    Faxnummer anrufen - im telmond-log steht dann Ihre MSN als angerufene
    Nummer.

        Standard-Einstellung: \var{FAXRCV\_x\_NUMBER}='0'}

\config{FAXRCV\_x\_DIRECTORY}{FAXRCV\_x\_DIRECTORY}{FAXRCVxDIRECTORY}

    {Hier wird das Verzeichnis angegeben, in dem der capifaxrcvd Ihre
    eingehenden Faxe speichern soll.

        Standard-Einstellung:
            \var{FAXRCV\_x\_DIRECTORY}='/data/fax'}

\config{FAXRCV\_x\_TEMP}{FAXRCV\_x\_TEMP}{FAXRCVxTEMP}

    {Hier wird das Verzeichnis angegeben, in dem der capifaxrcvd temporäre 
	Dateien ablegt (z.B. beim Anzeigen oder Download von Faxen).

        Standard-Einstellung:
            \var{FAXRCV\_x\_TEMP}='/tmp/fax'}

\end {description}

\subsection{Benachrichtigungs-Mails}

\begin {description}

\config {FAXRCV\_x\_MAIL} {FAXRCV\_x\_MAIL}{FAXRCVMAIL}

  Wenn auf \var{'yes'} gesetzt werden Benachrichtigungs-Mails über das Konto 
  \smalljump{FAXRCVMAILACCOUNT}{\var{FAXRCV\_x\_MAIL\_ACCOUNT}} an die Adresse 
  \smalljump{FAXRCVMAILTO}{\var{FAXRCV\_x\_MAIL\_TO}} gesendet.
  Das MAILSEND Paket muss mit \var{OPT\_MAILSEND='yes'} aktiviert sein.
  
  (Vorgabe \var{'no'}). 
  
\config {FAXRCV\_x\_MAIL\_ACCOUNT} {FAXRCV\_x\_MAIL\_ACCOUNT}{FAXRCVMAILACCOUNT}

  Optionaler MAILSEND Kontoname über den Benachrichtigungs-Mails verschickt werden.
  Wenn der Kontoname nicht angegeben wird, so wird das Konto \var{'default'}
  verwendet.

\config {FAXRCV\_x\_MAIL\_TO} {FAXRCV\_x\_MAIL\_TO}{FAXRCVMAILTO}

  Die eMail Adresse an die Benachrichtigungs-Mails verschickt werden.
  Es können eine oder mehrere, durch Komma getrennte eMail Adressen angegeben werden 

\config {FAXRCV\_x\_MAIL\_ATTACH} {FAXRCV\_x\_MAIL\_ATTACH}{FAXRCVMAILATTACH}
  An die Benachrichtigungs-Mail kann das empfangene Fax in verschiedenen 
  Dateiformaten angehängt werden.
  Das verwendete Dateiformat wird durch Setzen dieser Variable angegeben 
  (Vorgabe \var{'pdf'}). 
   
  Zur Auswahl stehen dabei:
  \begin{itemize}
    \item[none] kein Anhang
    \item[sff] Structured Fax Format
    \item[tiff] Tagged Image File Format
    \item[pdf] Portable Document Format
  \end{itemize}
  
\end {description}

\subsection{Notify-Befehle}

\begin {description}

\config{FAXRCV\_x\_NTFYCMD\_N}{FAXRCV\_x\_NTFYCMD\_N}{FAXRCVxNTFYCMDN}

    {Wie viele Notify-Befehle sollen ausgeführt werden (siehe nächster Punkt)?

        Standard-Einstellung: \var{FAXRCV\_x\_NTFYCMD\_N}='0'}

\config{FAXRCV\_x\_NTFYCMD\_x}{FAXRCV\_x\_NTFYCMD\_x}{FAXRCVxNTFYCMDx}

    {Dieser Befehl wird ausgeführt, wenn ein Fax eintrifft. Als Parameter
    werden der absolute Pfad inklusive Name des empfangenen Faxes und die
    msn (\var{FAXRCV\_x\_NUMBER}), auf der das Fax empfangen wurde, übergeben.
    Im Normalfall kann diese Variable einfach leer gelassen werden!
    In diesem Zusammenhang \var{OPT\_CAPIFAXBLINK} (s.u.) beachten.

        Standard-Einstellung: \var{FAXRCV\_x\_NTFYCMD\_x}=\dq{}}

\end {description}

\subsection{Imonc}

\begin {description}

\config{FAXRCV\_IMONC\_LIST}{FAXRCV\_IMONC\_LIST}{FAXRCVIMONCLIST}

    {Hier werden die Nummern (x) derjenigen \var{FAXRCV\_N} angegeben,
    die über Imonc erreichbar sein sollen. Wenn überhaupt keine Liste der
    empfangenen Faxe für den Imonc angelegt werden soll, ist hier '0'
    anzugeben.
    Mehrere Einträge müssen durch Leerzeichen getrennt werden,
    außerdem ist in diesem Fall zusätzlich die Variable
    \var{FAXRCV\_IMONC\_DIR} zu definieren.

        Standard-Einstellung:
            \var{FAXRCV\_IMONC\_LIST}='1', d.h. diejenige Telefonnummer,
                die durch \var{FAXRCV\_1\_NUMBER} definiert ist.}

\config{FAXRCV\_IMONC\_DIR}{FAXRCV\_IMONC\_DIR}{FAXRCVIMONCDIR}

    {Diese Variable wird nur benötigt, wenn mehrere MSN über den Imonc
    erreichbar gemacht werden und muss von Hand in die Konfigurationsdatei
    nachgetragen werden.
    Sie muss ein Verzeichnis enthalten, in dem das Logfile für den Imonc
    vorgehalten werden soll. Dieses Verzeichnis darf mit keinem der
    \var{FAXRCV\_x\_DIRECTORY} übereinstimmen und sollte auf einem
    Festspeicher liegen, da sonst nach einem Neustart nicht mehr auf die
    Faxe zugegeriffen werden kann.

        Beispiel:
            \var{FAXRCV\_IMONC\_DIR}='/data/fax/imonc'

        Standard-Einstellung:
            Diese Variable ist in der Standard-Config-Datei nicht enthalten.}

\end{description}

\subsection {Weboberfläche}

    Das Paket bringt außerdem eine Weboberfläche für den mini-httpd mit.
    Die Weboberfläche wird beim Setzen \var{OPT\_HTTPD='yes'} 
    automatisch mit aktiviert.

    Die Berechtigungsstufe für den httpd kann für das Ansehen, Löschen und Ein-
    bzw. Ausschalten separat vergeben werden. Bei \var{HTTPD\_USER\_x\_RIGHTS} muss
    dann \dq{}isdnfax:view\dq{}, \dq{}isdnfax:delete\dq{} bzw. \dq{}isdnfax:startstop\dq{}
    angegeben werden. Ein User mit Rights \dq{}all\dq{} darf natürlich alles :)
    
    Über die Weboberfläche besteht auch die Möglichkeit die Faxe direkt
    zu betrachten. Das Programm \var{sfftobmp} wird genutzt um aus den sff-Faxdateien
    dynamisch JPG-Bilddateien zu erstellen.
    Um dieses Feature zu nutzen, benötigen Sie etwas mehr Platz im Faxverzeichnis
    (etwa 1mb, wenn Sie ein 4-seitiges Fax betrachten wollen). Außerdem dauert
    die Konvertierung einige Sekunden. Nicht ungeduldig sein und zweimal klicken...

\begin{description}

\config {FAXRCV\_DOWNLOAD} {FAXRCV\_DOWNLOAD}{FAXRCVDOWNLOAD}

    Das beim Abruf über die Weboberfläche verwendete Dateiformat 
    wird durch Setzen dieser Variable angegeben (Vorgabe \var{'pdf'}). 
   
    Zur Auswahl stehen dabei:
    \begin{itemize}
        \item[sff] Structured Fax Format
        \item[tiff] Tagged Image File Format
        \item[pdf] Portable Document Format
    \end{itemize}

\end {description}

\subsection {Faxsignalisierung}

    Zusammen mit dem Paket HWSUPP kann das Eintreffen eines Faxes über eine 
    blinkede LED angezeigt werden.
    
    Die LED wird ausgeschaltet wenn über die Weboberfläche das Fax angezeigt, 
    heruntergeladen oder gelöscht wird.
    
    Aternativ kann auf der SSH-Konsole mit \texttt{faxrcv\_setleds off} die LED
    ausgeschaltet werden.

\begin{description}

\config {HWSUPP\_LED\_x}{HWSUPP\_LED\_x}{}
    Mit \var{HWSUPP\_LED\_x='faxrcv'} wird ein eingehendes Fax über diese 
    LED angezeigt.
    
\config {HWSUPP\_BUTTON\_x}{HWSUPP\_BUTTON\_x}{}
    Bei \var{HWSUPP\_BUTTON\_x='faxrcv'} kann die blinkede LED 
    durch Tastendruck ausgeschaltet werden.

\end{description}

\subsection {Faxabruf }

    Sie können die Faxe entweder über den Imonc (siehe \var{FAXRCV\_IMONC\_LIST}),
    oder über die Weboberfläche (\altlink{http://fli4l/} bzw. 
    http://$<$namedesrouters$>$/) unter dem Punkt \dq{}Faxe\dq{}
    abrufen. Das Dateiformat (.sff) wird bspw. von Fritz!Fax und IrfanView
    unterstützt.
 
\subsection {Autor }

    Das Paket wurde von Felix Eckhofer (\email{felix@fli4l.de}) zusammengestellt.
