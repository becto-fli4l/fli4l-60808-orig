% Synchronized to r54828
\marklabel{sec:faxrcv}
{
\section{FAXRCV}
}

\subsection {OPT\_FAXRCV - Receiving Faxes By the Help Of The AVM-Capi }
\configlabel{OPT\_FAXRCV}{OPTFAXRCV}

  This package enables fli4l to receive Faxes with a CAPI-capable
  ISDN Card. Read the documentation to learn which drivers are needed.\\

  Cling to the following hints concerning your hardware:\\

  Receiving Faxes with passive ISDN adapters will generate heavy CPU load.
  According to my experience a Pentium will work without problems.

\begin{description}

\config{FAXRCV\_START\_AT\_BOOT}{FAXRCV\_START\_AT\_BOOT}{FAXRCVSTARTATBOOT}
    {If this variable is set to 'yes' Fax receiving will be activated during boot. If set to
    'no', Fax receiving has to be started on the Web GUI or console via 'faxrcv.sh start'.

        Default Setting: \var{FAXRCV\_START\_AT\_BOOT}='yes'}

\config{FAXRCV\_N}{FAXRCV\_N}{FAXRCVN}

    {Set the number of capifaxrcvd services to be configured here (on how many MSN's
    should Faxes be received). This is useful i.e. if separate MSN's and directories
    should be set for various users. If only one capifaxrcvd should be started, set
    '1' here.

        Default Setting: \var{FAXRCV\_N}='1'}

\config{FAXRCV\_x\_CSID}{FAXRCV\_x\_CSID}{FAXRCVxCSID}

    {Specify the Station ID that the daemon should transmit to
    Fax callers. This should be your own Fax number. Keep in mind that
    capifaxrcvd is only able to handle digits and some special chars.

    A CSID like this would make sense: (123-456). 

        Default Setting: \var{FAXRCV\_x\_CSID}='+49(00)00000'}

    \achtung{Please don't use spaces, replace them by dashes!}

\config{FAXRCV\_x\_NUMBER}{FAXRCV\_x\_NUMBER}{FAXRCVxNUMBER}

    {Specify the MSN on which the daemon should receive Faxes here.
    Usually this is your own number without area prefix.

    \achtung{Phone boxes may change this number in some cases!} If not sure
    make a call with your own Fax number and see which number appears in the
    telmond-log as the caller ID.

        Default Setting: \var{FAXRCV\_x\_NUMBER}='0'}

\config{FAXRCV\_x\_DIRECTORY}{FAXRCV\_x\_DIRECTORY}{FAXRCVxDIRECTORY}

    {Set the directory here to which capifaxrcvd should save received Faxes.

        Default Setting:
            \var{FAXRCV\_x\_DIRECTORY}='/data/fax'}

\config{FAXRCV\_x\_TEMP}{FAXRCV\_x\_TEMP}{FAXRCVxTEMP}

    {Set the directory here to which capifaxrcvd should save temporary files
	    (e.g. files used for displaying or downloading).

        Default Setting:
            \var{FAXRCV\_x\_TEMP}='/tmp/fax'}

\end {description}

\subsection{Notification mails}

\begin {description}

\config {FAXRCV\_x\_MAIL} {FAXRCV\_x\_MAIL}{FAXRCVMAIL}

  If set to \var{'yes'} notification mails will be sent to the address in
  \smalljump{FAXRCVMAILTO}{\var{FAXRCV\_x\_MAIL\_TO}} via the account 
  configured in \smalljump{FAXRCVMAILACCOUNT}{\var{FAXRCV\_x\_MAIL\_ACCOUNT}} 
  The package MAILSEND must be activated with \var{OPT\_MAILSEND='yes'}.
  
  (default \var{'no'}). 

\config {FAXRCV\_x\_MAIL\_ACCOUNT} {FAXRCV\_x\_MAIL\_ACCOUNT}{FAXRCVMAILACCOUNT}

  Optional MAILSEND account name used to transmit notification mails.
  If the account name is not given, the account \var{'default'} is used.

\config {FAXRCV\_x\_MAIL\_TO} {FAXRCV\_x\_MAIL\_TO}{FAXRCVMAILTO}

  The email address receiving notification mails is to be entered here.
  One or more comma separated recipient addresses can be entered here.

\config {FAXRCV\_x\_MAIL\_ATTACH} {FAXRCV\_x\_MAIL\_ATTACH}{FAXRCVMAILATTACH}

  The received fax can be attached to the notification mail in different 
  file formats.
  The used format is set by this variable (default \var{'pdf'}). 
   
  The format can be one of:
  \begin{itemize}
    \item[none] no attachment
    \item[sff] Structured Fax Format
    \item[tiff] Tagged Image File Format
    \item[pdf] Portable Document Format
  \end{itemize}
  
\end {description}

\subsection{Notify-command}

\begin {description}

\config{FAXRCV\_x\_NTFYCMD\_N}{FAXRCV\_x\_NTFYCMD\_N}{FAXRCVxNTFYCMDN}

    {How many Notify-commands should be executed (see below)?

        Default Setting: \var{FAXRCV\_x\_NTFYCMD\_N}='0'}

\config{FAXRCV\_x\_NTFYCMD\_x}{FAXRCV\_x\_NTFYCMD\_x}{FAXRCVxNTFYCMDx}

    {This command is executed if a Fax has arrived. As a parameter
    the absolute path including the name of the fax file and the
    MSN (\var{FAXRCV\_x\_NUMBER}) with which the Fax was received will
    be provided. You may leave this variable empty. See\var{OPT\_CAPIFAXBLINK}
    in this context.

        Default Setting: \var{FAXRCV\_x\_NTFYCMD\_x}=''}

\end {description}

\subsection{Imonc}

\begin {description}

\config{FAXRCV\_IMONC\_LIST}{FAXRCV\_IMONC\_LIST}{FAXRCVIMONCLIST}

    {Provide the numbers of \var{FAXRCV\_N} that should be reached via imonc.
    If no list of received Faxes should be generated for imonc set '0' here.
    Entries have to be divided by spaces. Remeber that you will have to define
    their \var{FAXRCV\_IMONC\_DIR}.

        Default Setting:
            \var{FAXRCV\_IMONC\_LIST}='1', e.g. the phone number
                defined by \var{FAXRCV\_1\_NUMBER}.}

\config{FAXRCV\_IMONC\_DIR}{FAXRCV\_IMONC\_DIR}{FAXRCVIMONCDIR}

    {This variable is needed only if more than one MSN should be reachable via
    Imonc and has to be put in the config file manually.
    It should contain a directory where the logfile is present for Imonc.
    This directory must differ from all settings in \var{FAXRCV\_x\_DIRECTORY}
    and should be located on a permanent storage because otherwise access to all
    Faxes will be lost after a reboot.

        Beispiel:
            \var{FAXRCV\_IMONC\_DIR}='/data/fax/imonc'

        Default Setting:
            This variable is not contained in the standard config file.}

\end{description}

\subsection {Web GUI}

    The package also provides a Web GUI for mini-httpd.
    The Web GUI is automatically activated with \var{OPT\_HTTPD='yes'}.
    
    The rights for viewing, deleting and switching Fax receiving on and off can be defined
    separately for the httpd. In \var{HTTPD\_USER\_x\_RIGHTS} you have to provide \verb?isdnfax:view?,
    \verb?isdnfax:delete? resp. \verb?isdnfax:startstop? then. A user with rights \verb?all?
    is allowed to.. guess what :)
    

    Direct Fax viewing in the Web GUI is also possible here. 
    The programm \var{sfftobmp} is used to convert sff-Faxfiles to JPEG files
    dynamically.
    To use this feature you will need some additional storage (about one MB for
    a Fax containing four pages). The conversion will take some seconds. So for
    obvious reasons don't be impatient and don't click twice...
    

\begin{description}

    \config {FAXRCV\_DOWNLOAD} {FAXRCV\_DOWNLOAD}{FAXRCVDOWNLOAD}

    The download file format is set by this variable (default \var{'pdf'}). 
   
    It can be choosen from:
    \begin{itemize}
        \item[sff] Structured Fax Format
        \item[tiff] Tagged Image File Format
        \item[pdf] Portable Document Format
    \end{itemize}

\end{description}

\subsection {Fax Signaling}

    In conjunction with the package HWSUPP the arrival of a fax is signaled by 
    a blinking LED.
    
    The LED is switched off by displaying, downloading or deleting a fax on 
    the Web GUI.
    
    Aternatively on the SSH shell the command \texttt{faxrcv\_setleds off} 
    can be used to switch the LED off.

\begin{description}

\config {HWSUPP\_LED\_x}{HWSUPP\_LED\_x}{}
    With \var{HWSUPP\_LED\_x='faxrcv'} an incomming fax will be signaled with 
    this LED.
    
\config {HWSUPP\_BUTTON\_x}{HWSUPP\_BUTTON\_x}{}
    With \var{HWSUPP\_BUTTON\_x='faxrcv'} the blinking LED can be switched off 
    by pressing this button.

\end{description}

\subsection {Fax Download}

    You may download the Faxes either with Imonc (see \var{FAXRCV\_IMONC\_LIST}),
    or from the Web GUI (\altlink{http://fli4l/} resp. 
    http://$<$router\_name$>$/) in the menu Faxes.
    The sff file format is supported in Fritz!Fax and IrfanView, for example.


\subsection {Author}

    This package was integrated by Felix Eckhofer (\email{felix@fli4l.de}).
