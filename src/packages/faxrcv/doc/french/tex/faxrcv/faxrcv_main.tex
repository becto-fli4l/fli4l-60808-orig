% Synchronized to r54828
\marklabel{sec:faxrcv}
{
\section{FAXRCV}
}

\subsection{OPT\_FAXRCV - Réception de fax avec AVM Capi}
\configlabel{OPT\_FAXRCV}{OPTFAXRCV}

   Ce paquetage vous permet de réceptionner les fax avec le pilote CAPI et
   un système ISDN (en france RNIS). Vous devez lire la documentation de
   l'ISDN pour savoir quel sont les pilotes nécessaires.

  Vous devez vous référencer à votre matériel concernant les paramètres suivants~:

  La réception de fax avec un adaptateur ISDN passif va générer de lourde charge
  CPU. Selon mon expérience un Pentium fonctionnera sans problèmes.

\begin{description}

\config{FAXRCV\_START\_AT\_BOOT}{FAXRCV\_START\_AT\_BOOT}{FAXRCVSTARTATBOOT}

    {Si vous avez activez cette variable avec le paramètre 'yes', vous pouvez
	utiliser la réception de fax lors du démarrage du routeur. Si cette variable
	est sur 'no', la réception de fax peut être démarré par l'interface Web
	ou via la console avec la commande 'faxrcv.sh'.

        Paramètre par défaut~: \var{FAXRCV\_START\_AT\_BOOT}='yes'}

\config{FAXRCV\_N}{FAXRCV\_N}{FAXRCVN}

    {Dans cette variable vous définissez le nombre de services capifaxrcvd qui
	doit être configuré. C'est à dire, combien souhaitez-vous de MSN pour recevoir
	les fax. C'est utile si par exemple pour chaque connexion MSN vous configurez
	un répertoire séparé par utilisateur. Si vous voulez démarrer un seul capifaxrcvd,
	entrez '1'~!

        Paramètre par défaut~: \var{FAXRCV\_N}='1'}

\config{FAXRCV\_x\_CSID}{FAXRCV\_x\_CSID}{FAXRCVxCSID}

    {Dans cette variable vous définissez "l'ID de la station" pour que le démon
	transmet les fax du correspondant. Cela doit être votre numéro de fax. Gardez
	à l'esprit que capifaxrcvd est seulement capable de gérer des chiffres et
	des caractères spéciaux.

    Un CSID est par exemple écrit dans le format (123-456) les espaces sont remplacés
	par le signe '-'~!

        Paramètre par défaut~: \var{FAXRCV\_x\_CSID}='+49(00)00000'}

		\achtung{S'il vous plaît n'utilisez pas les espaces, remplacez-les par des tirets}

\config{FAXRCV\_x\_NUMBER}{FAXRCV\_x\_NUMBER}{FAXRCVxNUMBER}

    {Dans cette variable vous définissez le MSN sur lequel le démon devra recevoir
	les fax. Habituellement, cela est votre propre numéro sans le préfixe de zone.

    \achtung{Dans certains cas le système téléphonique peut modifier ce numéro~!} Si
	vous n'êtes pas sûr, vous pouvez appeler sur votre numéro de fax - Vous pouvez
	voir dans le journal telmond les numéros MSN appelé.

        Paramètre par défaut~: \var{FAXRCV\_x\_NUMBER}='0'}

\config{FAXRCV\_x\_DIRECTORY}{FAXRCV\_x\_DIRECTORY}{FAXRCVxDIRECTORY}

    {Dans cette variable vous définissez le répertoire dans lequel capifaxrcvd
	enregistre vos fax entrants.

        Paramètre par défaut~: \var{FAXRCV\_x\_DIRECTORY}='/data/fax'}

\config{FAXRCV\_x\_TEMP}{FAXRCV\_x\_TEMP}{FAXRCVxTEMP}

    {Dans cette variable vous indiquez le répertoire dans lequel capifaxrcvd enregistrera
	les fichiers temporaires (par exemple pour la visualisation ou le téléchargement des fax).

        Paramètre par défaut~: \var{FAXRCV\_x\_TEMP}='/tmp/fax'}

\end {description}

\subsection{Notification par courriel}

\begin {description}

\config {FAXRCV\_x\_MAIL} {FAXRCV\_x\_MAIL}{FAXRCVMAIL}

  Si vous indiquez \var{'yes'} dans cette variable une notification par courriel
  sera envoyée à l'adresse configurée dans \smalljump{FAXRCVMAILTO}{\var{FAXRCV\_x\_MAIL\_TO}}
  via le compte configuré dans \smalljump{FAXRCVMAILACCOUNT}{\var{FAXRCV\_x\_MAIL\_ACCOUNT}}.
  Le paquetage MAILSEND doit être activé avec \var{OPT\_MAILSEND='yes'}.

  (Par défaut \var{'no'}).

\config {FAXRCV\_x\_MAIL\_ACCOUNT} {FAXRCV\_x\_MAIL\_ACCOUNT}{FAXRCVMAILACCOUNT}

   Variable optionnelle, vous pouvez indiquer le nom du compte à utiliser pour transmettre
   les notifications par courriel. Si le nom du compte est pas indiqué, le compte \var{'default'}
   configuré dans MAILSEND sera utilisé.

\config {FAXRCV\_x\_MAIL\_TO} {FAXRCV\_x\_MAIL\_TO}{FAXRCVMAILTO}

  Dans cette variable vous indiquez l'adresse email (ou courriel) de réception.
  Vous pouvez indiquer une ou plusieurs adresses de destinataires, vous devez
  les séparées par une virgule.

\config {FAXRCV\_x\_MAIL\_ATTACH} {FAXRCV\_x\_MAIL\_ATTACH}{FAXRCVMAILATTACH}

  Le fax reçue peut être joint à la notification par courriel, dans différents formats
  de fichiers.
  Vous devez définir le format à utiliser dans cette variable, (par défaut \var{'pdf'}).

  Les formats peuvent être les suivants~:
  \begin{itemize}
    \item[none] no attachment
    \item[sff] Structured Fax Format
    \item[tiff] Tagged Image File Format
    \item[pdf] Portable Document Format
  \end{itemize}

\end {description}

\subsection{Notification par commande}

\begin {description}

\config{FAXRCV\_x\_NTFYCMD\_N}{FAXRCV\_x\_NTFYCMD\_N}{FAXRCVxNTFYCMDN}

    {Dans cette variable vous définissez le nombre de commandes qui doit être
	exécuté (voir ci-dessous).

        Paramètre par défaut~: \var{FAXRCV\_x\_NTFYCMD\_N}='0'}

\config{FAXRCV\_x\_NTFYCMD\_x}{FAXRCV\_x\_NTFYCMD\_x}{FAXRCVxNTFYCMDx}

    {Dans cette variable vous définissez la commande à exécuter quand un fax arrive.
	Le paramètre doit être un chemin absolu, y compris le nom du fichier fax et le MSN
	(\var{FAXRCV\_x\_NUMBER}) de l'espéditeur du fax envoyé. Normalement, cette
	variable peut être vide. Dans ce contexte, voir \var{OPT\_CAPIFAXBLINK} (ci-dessous)

        Paramètre par défaut~: \var{FAXRCV\_x\_NTFYCMD\_x}=''}

\end {description}

\subsection{Imonc}

\begin {description}

\config{FAXRCV\_IMONC\_LIST}{FAXRCV\_IMONC\_LIST}{FAXRCVIMONCLIST}

    {Dans cette variable vous définissez le nombre (x) de liste par rapport à \var{FAXRCV\_N}
	qui seront accessibles via l'interface imonc. Si vous ne créez aucune liste de
	réception de fax pour imonc vous indiquez ici '0'. Si vous en rentrez plusieurs
	vous devez les séparer par un espace, dans ce cas vous devez également défini
	la variable \var{FAXRCV\_IMONC\_DIR}.

        Paramètre par défaut~:
            \var{FAXRCV\_IMONC\_LIST}='1', par exemple le numéro de téléphone
				défini dans \var{FAXRCV\_1\_NUMBER}.}

\config{FAXRCV\_IMONC\_DIR}{FAXRCV\_IMONC\_DIR}{FAXRCVIMONCDIR}

    {Cette variable n'est nécessaire que si plusieurs MSN sont utilisés via imonc,
	ils doivent être ajoutés manuellement dans le fichier de configuration.
	Il doit contenir un répertoire pour conserver le fichier journal de imonc.
	Ce répertoire doit être différent du paramètres de la variable \var{FAXRCV\_x\_DIRECTORY}
	et doit être situé sur un support permanent, sinon l'accès à toutes les données
	seront perdus après un redémarrage du routeur.

        Exemple~:
            \var{FAXRCV\_IMONC\_DIR}='/data/fax/imonc'

        Paramètre par défaut~:
            Cette variable n'est pas incluse par défaut dans le fichier de configuration.}

\end{description}

\subsection {Web GUI (ou interface Web)}

    Ce paquetage met à disposition une interface Web qui fonctionne avec mini-httpd.
    L'interface Web sera automatiquement activée si la variable \var{OPT\_HTTPD='yes'}
	du paquetage httpd est activée.

    Les droit d'autorisation pour la suppression, l'affichage et la réception des fax, peuvent
	être définies séparément dans le paquetage httpd. Si la variable \var{HTTPD\_USER\_x\_RIGHTS} est
	paramétrée, vous pouvez indiquer \verb?isdnfax:view?, ou \verb?isdnfax:delete?, ou \verb?isdnfax:startstop?.
	Un utilisateur avec tous des droits \verb?all? est autorisé à ..., vous devinez quoi :)

\begin{description}

    \config {FAXRCV\_DOWNLOAD} {FAXRCV\_DOWNLOAD}{FAXRCVDOWNLOAD}

    Dans cette variable vous indiquez le format du fichier pour le téléchargement
	(par défaut \var{'pdf'}).

    Vous pouvez choisir les formats suivants~:
    \begin{itemize}
        \item[sff] Structured Fax Format
        \item[tiff] Tagged Image File Format
        \item[pdf] Portable Document Format
    \end{itemize}

\end{description}

    L'arrivé d'un fax peut être signalé par un clignotement d'une LED, mais cela doit être
	en collaboration avec le package HWSUPP.

	Le voyant s'éteintra quand vous allez afficher, télécharger ou supprimer le fax via
	l'interface Web.

	Autre variante, la LED peut être éteinte en utilisant la console SSH avec la commande
	\texttt{faxrcv\_setleds off}.

\begin{description}

\config {HWSUPP\_LED\_x}{HWSUPP\_LED\_x}{}

	Avec le paramètre \var{HWSUPP\_LED\_x='faxrcv'} la réception d'un fax sera signalée
	par une LED.

\config {HWSUPP\_BUTTON\_x}{HWSUPP\_BUTTON\_x}{}

	Avec le paramètre \var{HWSUPP\_BUTTON\_x='faxrcv'} le clignotement de la LED peut
	être désactivé en appuyant sur un bouton.

\end{description}

\subsection {Téléchargement des fax}

    Vous pouvez télécharger les fax soit avec imonc (voir \var{FAXRCV\_IMONC\_LIST}),
	soit via l'interface web (\altlink{http://fli4l/} ou http://$<$nomdurouteur$>$/),
	cliquez ensuite dans la section "Fax". Le format de fichier (.sff) est pris en
	charge par exemple avec Fritz!Fax et IrfanView.

\subsection {Auteur}

    Le paquetage a été compilé par~: Felix Eckhofer (\email{felix@fli4l.de}).
