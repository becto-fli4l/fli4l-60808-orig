% Synchronized to r29817
\marklabel{sec:opt-lnk}
{
\section {OPT\_LNK - Creation Of Links}
}

\subsection {Introduction}

With this package it is possible to make links to files/folders during
the boot process. This is particularly useful for volatile configuration
and log files such as the imonc telephone directory \texttt{/etc/phonebook}
to prevent losing all changes after the next reboot. Even complete
directories can be ``moved'', i.e. \texttt{/var/log}.

This package is an adaptation of Alexander Krause's OPT\_LNK package to
fli4l 3.x.

\subsection {Examples}

To link the directory \texttt{/var/log} to the harddisk, change the variables
in \texttt{config/lnk.txt} like this:

\begin{small}
\begin{example}
\begin{verbatim}
OPT_LNK='yes' # 'yes' installs the package to the router
LNK_N='1'                        # Number of links, here '1'
LNK_1_OPT='-fs'                  # Link 1: Options
LNK_1_DST='/var/log'             # Link 1: Target
LNK_1_SRC='/data/var/log'        # Link 1: Source
\end{verbatim}
\end{example}
\end{small}

\wichtig{The target of the link (here: \texttt{/var/log}) must be on a writable
file system in order to create the folder; so in this case \texttt{/var} has to be
a writable directory. If not, the link can not be created. Of course the target
(here: \texttt{/data/var/log}) must exist at the time of link creation.
}

\achtung{If the target (here: \texttt{/var/log}) already exists at the time of
link creation, it will be deleted! (this is no problem in the example as at the
time of link creation the directory is empty anyway.)}

To make the \texttt{imonc} phone book persistent, at first copy it to a directory
on your harddisk (i.e. to \texttt{/data/etc/phonebook}) and create an additional link:

\begin{small}
\begin{example}
\begin{verbatim}
OPT_LNK='yes' # 'yes' installs the package to the router
LNK_N='2'                        # Number of links, here '2'
LNK_1_OPT='-fs'                  # Link 1: Options
LNK_1_DST='/var/log'             # Link 1: Target
LNK_1_SRC='/data/var/log'        # Link 1: Source
LNK_2_OPT='-fs'                  # Link 2: Options
LNK_2_DST='/etc/phonebook'       # Link 2: Target
LNK_2_SRC='/data/etc/phonebook'  # Link 2: Source
\end{verbatim}
\end{example}
\end{small}

\subsection {Configuration}

\begin{description}
\config{OPT\_LNK}{OPT\_LNK}{OPTLNK}{
Setting this to ``yes'', activates the package.
}
\config{LNK\_N}{LNK\_N}{LNKN}{
This variable holds the number of links to be created.

\achtung {In the following variables \var{x} has to be replaced by an index number!}
}
\config{LNK\_x\_OPT}{LNK\_x\_OPT}{LNKxOPT}{
With this variable you may change the options transferred to the program ``ln''
during linking of \smalljump{LNKxSRC}{Source} and \smalljump{LNKxDST}{target}.
The default ``-fs'' is correct in most use cases and should not be changed.
}
\config{LNK\_x\_DST}{LNK\_x\_DST}{LNKxDST}{
This variable holds the target of the link. By the name provided here you may
acces the \smalljump{LNKxSRC}{Source} after link creation. As a rule of thumb
the target referenced here should not exist.

\achtung {Attention: If the target exists it will be deleted without warning!}
}
\config{LNK\_x\_SRC}{LNK\_x\_SRC}{LNKxSRC}{
This variable holds the source for the link to be created. This may be a file
or a directory.

\wichtig {The source mentioned here has to exist at the time of link creation!}
}
\end{description}
