% Do not remove the next line
% Synchronized to r29817
\marklabel{sec:opt-lnk}
{
\section {OPT\_LNK~- Création de liens}
}

\subsection {Introduction}

Avec ce paquetage, il est possible d'associer un fichier/dossier spécifique
pendant le processus de boot \footnote{cette liaison est appelé \emph{Link} il
est issu du même nom que le paquetage}. C'est particulièrement est utile pour
la configuration normalement volatile, des fichiers log (ou journal), du
répertoire téléphonique imonc \texttt{/etc/phonebook}, car à chaque redémarrage
les modifications des fichiers seront perdues. Et aussi tous se qui tourne
"autour" du répertoire \texttt{/var/log}.

Ce paquetage est une adaptation du paquetage OPT\_LNK de Alexander Krause
pour fli4l 3.x.

\subsection {Exemples}

Pour mapper (ou établir un lien) le répertoire \texttt{/var/log} sur votre disque
dur, configurer les variables du fichier \texttt{config/lnk.txt} comme ci-dessous~:

\begin{small}
\begin{example}
\begin{verbatim}
OPT_LNK='yes' # avec 'yes' le paquetage sera installé, avec 'no' il n'est pas installé
LNK_N='1'                        # Nombre de liens, ici '1'
LNK_1_OPT='-fs'                  # Link 1: Option
LNK_1_DST='/var/log'             # Link 1: Cible
LNK_1_SRC='/data/var/log'        # Link 1: Source
\end{verbatim}
\end{example}
\end{small}

\wichtig{La cible du lien (ici~: \texttt{/var/log}) sera créée, se qui veut dire
que dans cet exemple le répertoire \texttt{/var} doit être accessible en écriture.
Si ce n'est-ce pas le cas, le lien ne sera pas créé. De même, la source du lien
(ici~: \texttt{/data/var/log}) doit bien sûr être présent au moment de la connexion.
}

\achtung{Si la cible (ici~: \texttt{/var/log}) existe déjà avant de lien, il sera
supprimé~! (De toute façon, ce n'est pas un problème dans l'exemple, puisque
ce répertoire au moment de la création du lien sera vide.)}

Nous allons poursuivre, et copier l'annuaire téléphonique persistante de
\texttt{imonc}, dans un répertoire de votre disque dur (ici~: \texttt{/data/etc/phonebook}),
pour cela on ajoute les liens supplémentaires~:

\begin{small}
\begin{example}
\begin{verbatim}
OPT_LNK='yes' # avec 'yes' le paquetage sera installé, avec 'no' il n'est pas installé
LNK_N='2'                        # Nombre de liens, ici '2'
LNK_1_OPT='-fs'                  # Link 1: Option
LNK_1_DST='/var/log'             # Link 1: Cible
LNK_1_SRC='/data/var/log'        # Link 1: Source
LNK_2_OPT='-fs'                  # Link 2: Option
LNK_2_DST='/etc/phonebook'       # Link 2: Cible
LNK_2_SRC='/data/etc/phonebook'  # Link 2: Source
\end{verbatim}
\end{example}
\end{small}

\subsection {Configuration}

\begin{description}
\config{OPT\_LNK}{OPT\_LNK}{OPTLNK}{
Si cette valeur est "yes", le package est activée.
}
\config{LNK\_N}{LNK\_N}{LNKN}{
Dans cette variable vous indiquez le nombre de liens à créer.

\achtung {Dans les variables suivantes le \var{x} sera remplacé par l'indexation~!}
}
\config{LNK\_x\_OPT}{LNK\_x\_OPT}{LNKxOPT}{
Dans cette variable vous pouvez modifier les options qui seront transmis lors
d'une connexion \smalljump{LNKxSRC}{Sourse} et \smalljump{LNKxDST}{Cible}
du programme "ln". La valeur par défaut "-fs" est correcte dans la plupart
des cas et ne doit pas être modifiée.
}
\config{LNK\_x\_DST}{LNK\_x\_DST}{LNKxDST}{
Dans cette variable vous indiquez la cible du lien à créer. Au sujet du nom à
donné ici, la \smalljump{LNKxSRC}{Source} sera accessible après la connexion.
En général, la cible n'existe pas.

\achtung {Attention~: Si la cible existe, elle sera supprimée sans avertissement~!}
}
\config{LNK\_x\_SRC}{LNK\_x\_SRC}{LNKxSRC}{
Dans cette variable vous indiquez la source qui sera appliquée après la
connexion. Cela peut être à la fois un fichier et un répertoire.

\wichtig {La source mentionnée doit être présente au moment de la connexion~!}
}
\end{description}
