% Synchronized to r60510

\section {VIRT -- Support For Virtualization}

This package supports the use of fli4l as a virtual machine. Using a 64
bit kernel for the x86\_64 architecture is required.

For virtualization of a fli4l using \emph{Xen}, \emph{KVM}, \emph{VMware} or \emph{Hyper-V} this package
loads the necessary kernel modules. Furthermore you can set additional options,
which may be useful for the use as virtual machine.

\subsection {Xen}

\begin{description}
\config{OPT\_XEN}{OPT\_XEN}{OPTXEN}

The activation of this variable ensures that the Xen-specific kernel modules
are loaded on fli4l. This is necessary if the fli4l system is virtualized using Xen.

The following drivers are loaded:

\begin{itemize}
   \item netxen\_nic
   \item xen-blkfront
   \item xen-kbdfront
   \item xen-netfront
\end{itemize}

Default Setting: \verb+OPT_XEN='no'+

Example: \verb+OPT_XEN='yes'+

\end{description}

\subsection{Virtio}

\begin{description}
\config{OPT\_VIRTIO}{OPT\_VIRTIO}{OPTVIRTIO}

The activation of this variable ensures that the KVM-specific kernel modules
are loaded on fli4l. This is necessary if the fli4l system is virtualized using KVM.

The following drivers are loaded:

\begin{itemize}
   \item virtio\_balloon
   \item virtio\_blk
   \item virtio\_net
   \item virtio\_pci
\end{itemize}

Default Setting: \verb+OPT_VIRTIO='no'+

Example: \verb+OPT_VIRTIO='yes'+

\config{VIRTIO\_QEMU\_GUEST\_AGENT}{VIRTIO\_QEMU\_GUEST\_AGENT}{VIRTIOQEMUGUESTAGENT}

You can start the QEMU Guest 
Agent\footnote{See \altlink{https://wiki.libvirt.org/page/Qemu_guest_agent}} 
on the virtualized fli4l with this option. This way the host can execute
certain management functions, which need support from within the guest
system. Retrieving statistics or triggering a clean shutdown or suspend
is possible for example.

This option requires correspondent configuration of the virtual machine on the 
host side. See the documentation of KVM,
virt-manager\footnote{See \altlink{https://virt-manager.org/}} or
Proxmox\footnote{See \altlink{https://pve.proxmox.com/wiki/Qemu-guest-agent}} 
for that.

Default Setting: \verb+VIRTIO_QEMU_GUEST_AGENT='no'+

Example: \verb+VIRTIO_QEMU_GUEST_AGENT='yes'+

\end{description}

\subsection {VMware}

\begin{description}
\config{OPT\_VMWARE}{OPT\_VMWARE}{OPTVMWARE}

Activating this variable triggers loading of the specific kernel modules required 
for virtualizing fli4l using VMware. 

The following drivers are loaded:

\begin{itemize}
   \item vmw\_pvscsi
   \item mptsas
   \item mptspi
   \item ahci
   \item ata\_piix
   \item vmxnet3
   \item e1000e
   \item e1000
   \item pcnet32
\end{itemize}

Default setting: \verb+OPT_VMWARE='no'+

Example: \verb+OPT_VMWARE='yes'+

\end{description}

\subsection {Hyper-V}

\begin{description}
\config{OPT\_HYPERV}{OPT\_HYPERV}{OPTHYPERV}

Activating this variable ensures that Hyper-V specific kernel modules 
are loaded. This is necessary for the fli4l system running virtualized 
using Hyper-V.

The following drivers are loaded: 

\begin{itemize}
    \item pci\_hyperv
    \item hv\_storvsc
    \item hv\_utils
    \item hv\_balloon
    \item hv\_sock
    \item hv\_netvsc
\end{itemize}

Default setting: \verb+OPT_HYPERV='no'+

Example: \verb+OPT_HYPERV='yes'+

\end{description}
