% Synchronized to r58996

\section{VIRT -- Supporte la virtualisation}

Ce paquetage charge les modules dans le kernel de fli4l pour la
virtualisation XEN ou KVM. Cela nécessite l'utilisation du kernel
64 bits avec l'architecture x86-64.

% #N.A.# ##TRANSLATE## : FFL-2731
Pour virtualier fli4l vous pouvez utiliser les technologies suivantes
\emph {Xen}, \emph {KVM} ou \emph{VMware} les modules nécessaires à
la virtualisation seront chargés dans le kernel. En outre, vous pouvez
définir d'autres options, ils seront utiles ou pratiques pour 
une machine virtuelle.
% End of TRANSLATE

\subsection{Xen}

\begin{description}
\config{OPT\_XEN}{OPT\_XEN}{OPTXEN}

L'activation de cette variable garantit le chargement des modules dans
le kernel de fli4l, ils sont spécifiques à Xen. Cela est nécessaire lors de
la virtualisation du système fli4l en utilisant Xen.

Les pilotes suivants seront chargés :

\begin{itemize}
   \item netxen\_nic
   \item xen-blkfront
   \item xen-kbdfront
   \item xen-netfront
\end{itemize}

Paramètre par défaut : \verb+OPT_XEN='no'+

Exemple : \verb+OPT_XEN='yes'+

\end{description}

\subsection{Virtio}

\begin{description}
\config{OPT\_VIRTIO}{OPT\_VIRTIO}{OPTVIRTIO}

L'activation de cette variable garantit le chargement des modules dans
le kernel de fli4l, ils sont spécifiques à KVM. Cela est nécessaire
lors de la virtualisation du système fli4l en utilisant KVM.

Les pilotes suivants seront chargés :

\begin{itemize}
   \item virtio\_balloon
   \item virtio\_blk
   \item virtio\_net
   \item virtio\_pci
\end{itemize}

Paramètre par défaut : \verb+OPT_VIRTIO='no'+

Exemple : \verb+OPT_VIRTIO='yes'+

L'option suivante active QEMU Guest Agent ou (l’agent invité QEMU)
sur fli4l virtualisé\footnote{voir \altlink{https://wiki.libvirt.org/page/Qemu_guest_agent}}.
De cette manière, l'hôte invité peut exécuter certaines fonctions
de gestion et exécuter des commandes dans le système virtualisé.
Par exemple, la récupération de statistique, arrêter correctement
ou suspendre le système.

Cette option nécessite une configuration appropriée de la machine
côté hôte. Veuillez vous reporter à la documentation de KVM,
virt-manager\footnote{voir \altlink{https://virt-manager.org/}} ou
Proxmox\footnote{voir \altlink{https://pve.proxmox.com/wiki/Qemu-guest-agent}}.

Paramètre par défaut : \verb+VIRTIO_QEMU_GUEST_AGENT='no'+

Exemple : \verb+VIRTIO_QEMU_GUEST_AGENT='yes'+

\end{description}

\subsection {VMware}

\begin{description}
\config{OPT\_VMWARE}{OPT\_VMWARE}{OPTVMWARE}

L'activation de cette variable garantit le chargement des modules dans
le kernel de fli4l, ils sont spécifiques à VMware. Cela est nécessaire
lors de la virtualisation du système fli4l en utilisant VMware.

Les pilotes suivants seront chargés :

\begin{itemize}
   \item vmw\_pvscsi
   \item mptsas
   \item mptspi
   \item ahci
   \item ata\_piix
   \item vmxnet3
   \item e1000e
   \item e1000
   \item pcnet32
\end{itemize}

Paramètre par défaut : \verb+OPT_VMWARE='no'+

Exemple : \verb+OPT_VMWARE='yes'+

\end{description}

% #N.A.# ##TRANSLATE## : FFL-2731
\subsection {Hyper-V}

\begin{description}
\config{OPT\_HYPERV}{OPT\_HYPERV}{OPTHYPERV}

Die Aktivierung dieser Variable sorgt dafür, dass auf dem fli4l die für
Hyper-V spezifischen Kernelmodule geladen werden. Das ist erforderlich,
wenn das fli4l-System mittels Hyper-V virtualisiert wird.

Es werden die folgenden Treiber geladen:

\begin{itemize}
    \item pci\_hyperv
    \item hv\_storvsc
    \item hv\_utils
    \item hv\_balloon
    \item hv\_sock
    \item hv\_netvsc
\end{itemize}

Standard-Einstellung: \verb+OPT_HYPERV='no'+

Beispiel: \verb+OPT_HYPERV='yes'+

\end{description}
% End of TRANSLATE
