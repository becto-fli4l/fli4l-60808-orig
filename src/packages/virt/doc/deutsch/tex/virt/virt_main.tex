% Last Update: $Id$
\section {VIRT -- Unterstützung für Virtualisierung}

Dieses Paket unterstützt den Einsatz von fli4l als virtuelle Maschine.
Voraussetzung hierfür ist der Einsatz eines 64-Bit-Kernels für die
x86-64-Architektur.

Für die Virtualisierung eines fli4l mittels \emph{Xen}, \emph{KVM},
\emph{VMware} oder \emph{Hyper-V} lädt es die nötigen Kernelmodule.
Darüberhinaus können weitere Optionen gesetzt werden, die für den
Einsatz als VM sinnvoll oder praktisch sind.

\subsection {Xen}

\begin{description}
\config{OPT\_XEN}{OPT\_XEN}{OPTXEN}

Die Aktivierung dieser Variable sorgt dafür, dass auf dem fli4l die
Xen-spezifischen Kernelmodule geladen werden. Das ist erforderlich, wenn
das fli4l-System mittels Xen virtualisiert wird.

Es werden die folgenden Treiber geladen:

\begin{itemize}
   \item netxen\_nic
   \item xen-blkfront
   \item xen-kbdfront
   \item xen-netfront
\end{itemize}

Standard-Einstellung: \verb+OPT_XEN='no'+

Beispiel: \verb+OPT_XEN='yes'+

\end{description}

\subsection {Virtio}

\begin{description}
\config{OPT\_VIRTIO}{OPT\_VIRTIO}{OPTVIRTIO}

Die Aktivierung dieser Variable sorgt dafür, dass auf dem fli4l die
KVM-spezifischen Kernelmodule geladen werden. Das ist erforderlich, wenn
das fli4l-System mittels KVM virtualisiert wird.

Es werden die folgenden Treiber geladen:

\begin{itemize}
   \item virtio\_balloon
   \item virtio\_blk
   \item virtio\_net
   \item virtio\_pci
\end{itemize}

Standard-Einstellung: \verb+OPT_VIRTIO='no'+

Beispiel: \verb+OPT_VIRTIO='yes'+

\config{VIRTIO\_QEMU\_GUEST\_AGENT}{VIRTIO\_QEMU\_GUEST\_AGENT}{VIRTIOQEMUGUESTAGENT}

Mit dieser Option kann auf dem virtualisierten fli4l der QEMU Guest
Agent\footnote{Siehe \altlink{https://wiki.libvirt.org/page/Qemu_guest_agent}} 
gestartet werden. Auf diese Weise kann der Virtualisierungshost gewisse
Managementfunktionen ausführen, die Unterstützung aus dem Gastsystem
heraus erfordern. Beispielsweise können Statistikdaten abgerufen oder
sauberes Shutdown und Suspend vom Host ausgelöst werden.

Diese Option erfordert eine entsprechende hostseitige Einrichtung der
virtuellen Maschine. Hierzu sei auf die Dokumentation von KVM,
virt-manager\footnote{Siehe \altlink{https://virt-manager.org/}} oder
Proxmox\footnote{Siehe \altlink{https://pve.proxmox.com/wiki/Qemu-guest-agent}} 
verwiesen.

Standard-Einstellung: \verb+VIRTIO_QEMU_GUEST_AGENT='no'+

Beispiel: \verb+VIRTIO_QEMU_GUEST_AGENT='yes'+

\end{description}

\subsection {VMware}

\begin{description}
\config{OPT\_VMWARE}{OPT\_VMWARE}{OPTVMWARE}

Die Aktivierung dieser Variable sorgt dafür, dass auf dem fli4l die für
den Betrieb unter VMware nötigen Kernelmodule geladen werden. Das ist
erforderlich, wenn das fli4l-System mittels VMware virtualisiert wird.

Es werden die folgenden Treiber geladen:

\begin{itemize}
   \item vmw\_pvscsi
   \item mptsas
   \item mptspi
   \item ahci
   \item ata\_piix
   \item vmxnet3
   \item e1000e
   \item e1000
   \item pcnet32
\end{itemize}

Standard-Einstellung: \verb+OPT_VMWARE='no'+

Beispiel: \verb+OPT_VMWARE='yes'+

\end{description}

\subsection {Hyper-V}

\begin{description}
\config{OPT\_HYPERV}{OPT\_HYPERV}{OPTHYPERV}

Die Aktivierung dieser Variable sorgt dafür, dass auf dem fli4l die für
Hyper-V spezifischen Kernelmodule geladen werden. Das ist erforderlich,
wenn das fli4l-System mittels Hyper-V virtualisiert wird.

Es werden die folgenden Treiber geladen:

\begin{itemize}
    \item pci\_hyperv
    \item hv\_storvsc
    \item hv\_utils
    \item hv\_balloon
    \item hv\_sock
    \item hv\_netvsc
\end{itemize}

Standard-Einstellung: \verb+OPT_HYPERV='no'+

Beispiel: \verb+OPT_HYPERV='yes'+

\end{description}
