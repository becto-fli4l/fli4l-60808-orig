% Last Update: $Id$
\marklabel{sec:opt-lcd }
{
  \section {LCD - Anzeige von Statusinformationen über LC-Display}
}

\subsection{Einleitung}


    Mit diesem Paket ist es möglich, ein LCD-Modul an den Parallelport des
    fli4l Rechners anzuschließen. Auch können nun serielle LCD-Module
    der Firma Matrix-Orbital verwendet werden.
    Es gibt ausserdem Filter für spezielle Displays.

    Auf diesem Display werden Informationen wie Datum, Uhrzeit, die aktuellen
    Loadwerte und natürlich auch der ISDN- oder DSL-Durchsatz für Up- und
    Download in kb/s und ein Balken angegeben.


\subsection{Konfiguration}\index{OPT\_LCD}


    Möchte man das LCD-Paket einsetzen, sind noch folgende Variablen zu
    setzen:

\begin{example}
\begin{verbatim}
        OPT_LCD='yes'  (Standard-Einstellung: OPT_LCD='no')
\end{verbatim}
\end{example}


\begin{description}
  \config{LCD\_COLS}{LCD\_COLS}{LCDCOLS} - Anzahl der Zeichen pro Zeile.
  Das Kernelmodul unterstützt momentan Werte von 16, 20, 24, 32 und
  40. Module mit 8 oder 27 Zeichen funktionieren üblicherweise ebenfalls.
  Am Kernelmodul wird dann von fli4l 16 bzw. 40 übergeben.

\config{LCD\_LINES}{LCD\_LINES}{LCDLINES} - Anzahl der Zeilen. 

    Mögliche Werte: 1, 2 und 4.
    
    Achtung: Displays am Parallelport mit zwei Controllerchips (4x16,
    4x40 etc.)  müssen hier mit 2 Zeilen definiert werden! Die
    Koordinaten der anzuzeigenden Werte können normal angegeben
    werden. Der LCD-Treiber entscheidet anhand der Koordinate und der
    Anzahl der Zeilen, welcher der beiden Kontroller anzusprechen ist.

\config{LCD\_ADDRESS}{LCD\_ADDRESS}{LCDADDRESS}

      Die IO-Adresse des LPT-Ports, z.B. '0x278'
      
      Wird ein serielles Display von Matrix-Orbital verwendet, ist
      hier die verwendete serielle Schnittstelle einzutragen, z.B.
      'com1' oder 'com2'. Mit 'null' kann man LAN-Only einstellen, siehe: \var{LCD\_LANIP}
      
      Es ist möglich, die Ausgabe auf den Bildschirm auszugeben:
      'console' oder tty1 wählen den Hauptbildschirm. Dies kann allerdings
      zur Vermischung mit den normalen Ausgaben führen und ist daher nicht
      zu empfehlen. 'tty2', 'tty3' ... 'tty9' selektieren andere virtuelle 
      Konsolen, die mit ALT-F2 ... zu erreichen sind. ALT-F1 schaltet dann
      wieder auf den Hauptbildschirm zurück.

        \wichtig{
        Bisher wurden nur parallele Schnittstellen auf dem Mainboard
        oder auf ISA-Schnittstellenkarten unterstützt. PCI-Karten mit
        parallelen Schnittstellen konnten nicht verwendet werden.
        Diese Version hier erlaubt auch die Konfiguration von parallelen
        Schnittstellen auf bestimmten PCI-Karten mit NETMOS-Chips.
        Hier zu muss man sich mittels }

\begin{example}
\begin{verbatim}
        cat /proc/pci
\end{verbatim}
\end{example}

        \emph{die erkannten PCI-Geräte anzeigen lassen. Hier sucht man das Gerät mit
        der passenden Vendor-ID und Device-ID und wählt als io-Adresse den oder
        die folgenden Einträge aus:}

        \begin{itemize}
        \item Nm9705CV  (Vendor id=9710, Device id=9705, Port1 1. Eintrag)
        \item Nm9735CV  (Vendor id=9710, Device id=9735, Port1 3. Eintrag)
        \item Nm9805CV  (Vendor id=9710, Device id=9805, Port1 1. Eintrag)
        \item Nm9715CV  (Vendor id=9710, Device id=9815, Port1 1. Eintrag, Port2 3. Eintrag)
        \item Nm9835CV  (Vendor id=9710, Device id=9835, Port1 3. Eintrag)
        \item Nm9755CV  (Vendor id=9710, Device id=9855, Port1 1. Eintrag, Port2 3. Eintrag)
        \end{itemize}

        \emph{Die Konfigurationsmöglichkeit wurde eingebaut, ohne entsprechende
        Hardware zum Testen zur Verfügung zu haben. Daher ist das als
        experimentelles Feature zu betrachten. Bei Fehlern bitte ausführliche
        Informationen in die Newsgroup posten!}

\config{LCD\_LANIP}{LCD\_LANIP}{LCDLANIP} (optionale Variable)

      Die IP-Adresse eines 16x2 Displays an einem Pollin AVR-NET-IO im LAN
      oder andere Displays an ethersex - siehe weitere Variablen
      (experimentell) 

\config{LCD\_LANTYPE}{LCD\_LANTYPE}{LCDLANTYPE} (optionale Variable)
      
      Typ der Firmware.
      Zur Auswahl stehen: 'pollin' (default) - Original-Firmware AVR-NET-IO
      'ethersex' - Firmware von www.ethersex.de mit aktivem LCD
      
\config{LCD\_LANUSER}{LCD\_LANUSER}{LCDLANUSER} (optionale Variable)
      
      Authentifizierung am ethersex, wenn PAM auf ecmd/tcp konfiguriert wurde.
      Hier: Benutzername

\config{LCD\_LANPASS}{LCD\_LANPASS}{LCDLANPASS} (optionale Variable)
      
      Passwort zu \var{LCD\_LANUSER}
      
\configlabel{LCD\_TIME\_LONG}{LCDTIMELONG}
\configlabel{LCD\_TIME\_SHORT}{LCDTIMESHORT}
\configvar {LCD\_TIME\_LONG   LCD\_TIME\_SHORT}

      Dieses sind zwei Timer-Werte für den IO-Port zum LCD-Display.
      Werden die Variablen leer gelassen, gelten folgende Standardwerte:

\begin{example}
\begin{verbatim}
        LCD_TIME_LONG='100'
        LCD_TIME_SHORT='40'
\end{verbatim}
\end{example}

      Sollte es Probleme mit dem LCD-Display geben, z.B. wenn wild verteilte
      Zeichen erscheinen, sollte man diese Werte erhöhen, z.B.

\begin{example}
\begin{verbatim}
        LCD_TIME_LONG='120'
        LCD_TIME_SHORT='60'
\end{verbatim}
\end{example}

      Bei seriellen Displays von Matrix-Orbital sind diese Variablen
      ohne Bedeutung

\config{LCD\_ADDR\_TYPE}{LCD\_ADDR\_TYPE}{LCDADDRTYPE}
 - Adressierungsart des LCD-Controllers.

\begin{example}
\begin{verbatim}
        LCD_ADDR_TYPE='0'     # Fuer HD44780 und kompatible Kontroller
        LCD_ADDR_TYPE='1'     # Fuer HD66712 und kompatible Kontroller
        LCD_ADDR_TYPE='2'     # Obsolet, Funktionalität ist jetzt
                              # in '0' eingebaut

\end{verbatim}
\end{example}
    
    Bei seriellen Displays von Matrix-Orbital ist diese Variable ohne Bedeutung

\config{LCD\_WINAMP}{LCD\_WINAMP}{LCDWINAMP}

Es existieren verschiedene Verdratungsvarianten bei fertig bestückten
LCD-Displays von Kernel-Concepts, die normale und die
Winamp-Verdrahtung. Neuere Displays kommen mit letzterer Variante, dann
muss hier ein yes eingetragen werden.

\config{LCD\_FILTER}{LCD\_FILTER}{LCDFILTER}

 - Filter für spezielle Displays.
Zur Zeit gibt es Filter für
- ipc\_a78 Displays

\begin{example}
\begin{verbatim}
        LCD_FILTER='mo2ipc_a78'     # Fuer mo2ipc_a78 Displays
\end{verbatim}
\end{example}

\config{LCD\_START\_MSG}{LCD\_START\_MSG}{LCDSTARTMSG}

       Die hier eingetragene Meldung wird beim Systemstart kurz nach Laden der
       Treiber auf das Display ausgegeben. Sie sollte die Länge einer Zeile 
       nicht überschreiten, da anderenfalls die vollständige Ausgabe des Textes 
       nicht garantierte werden kann.

\config{LCD\_STOP\_MSG}{LCD\_STOP\_MSG}{LCDSTOPMSG}

       Die hier eingetragene Meldung wird beim Herunterfahren des Systems
       auf das Display ausgegeben. Sie sollte die Länge einer Zeile 
       nicht überschreiten, da anderenfalls die vollständige Ausgabe des Textes 
       nicht garantierte werden kann.

\config{LCD\_REBOOT\_MSG}{LCD\_REBOOT\_MSG}{LCDREBOOTMSG}

       Die hier eingetragene Meldung wird beim Rebooten des Systems
       auf das Display ausgegeben. Sie sollte die Länge einer Zeile 
       nicht überschreiten, da anderenfalls die vollständige Ausgabe des Textes 
       nicht garantierte werden kann.

\config{LCD\_START\_ISDN\_RATE}{LCD\_START\_ISDN\_RATE}{LCDSTARTISDNRATE}

      Gibt an, ob das Programm isdn\_rate gestartet werden soll.

\config{LCD\_TYPE\_N}{LCD\_TYPE\_N}{LCDTYPEN}

      Ulf Lanz hat das Ausgabeformat im isdn\_rate-Programm so variabel
      gestaltet, dass jeder Anwender die Anzeige seinen Wünschen gemäß
      zusammenstellen kann.

      Bei \var{LCD\_\-TYPE\_\-N} ist die Anzahl der auszugebenden Datentypen anzugeben.
      Diese Datentypen werden immer angezeigt, egal ob man Online oder
      Offline ist.


\config{LCD\_TYPE\_x}{LCD\_TYPE\_x}{LCDTYPEx}

        \var{LCD\_\-TYPE\_\-x} gibt den Datentyp, die Spalte und die Zeile an, wo die
        gewünschte Information auf dem Display erscheinen soll. Der
        Datentyp ist numerisch codiert. Die möglichen Werte sind in
        Table \ref{tab:lcd-type-x} aufgeführt.

        \begin{table}[htbp]
          \begin{small}
          \begin{center}
            \begin{tabular}{rlr}
               Typ &     Information   &             Zeichenbreite\\


                0 &       local date dd.mm.yyyy           & 10 \\
                1 &       local date dd.mm.yy             &  8 \\
                2 &       local time hh:mm:ss             &  8 \\

                3 &       remote date dd.mm.yyyy          & 10 \\
                4 &       remote date dd.mm.yy            &  8 \\
                5 &       remote time hh:mm:ss            &  8 \\

                6 &       isdn status channel 1           &  7 \\
                7 &       isdn status channel 2           &  7 \\
                8 &       dsl status                      &  7 \\

                9 &       isdn circuit name channel 1     & 16 \\
                10 &      isdn circuit name channel 2     & 16 \\
                11 &      dsl circuit name                & 16 \\

                12 &      isdn input rate bar             &  8 \\
                13 &      isdn output rate bar            &  8 \\
                14 &      dsl input rate bar              &  8 \\
                15 &      dsl output rate bar             &  8 \\

                16 &      isdn input rate                 &  5 \\
                17 &      isdn output rate                &  5 \\
                18 &      dsl input rate                  &  9 \\
                19 &      dsl output rate                 &  9 \\

                20 &      isdn charge channel 1           &  6 \\
                21 &      isdn charge channel 2           &  6 \\
                22 &      dsl charge                      &  6 \\

                23 &      isdn ip address channel 1       & 15 \\
                24 &      isdn ip address channel 2       & 15 \\
                25 &      dsl ip address                  & 15 \\

                26 &      load 1                          &  5 \\
                27 &      load 2                          &  5 \\

                28 &      phone                           & 16 \\

                29 &      isdn online time channel 1      &  8 \\
                30 &      isdn online time channel 2      &  8 \\
                31 &      dsl online time                 &  8 \\

                32 &      isdn quantity in channel 1      &  8 \\
                33 &      isdn quantity in channel 2      &  8 \\
                34 &      dsl quantity in                 &  8 \\

                35 &      isdn quantity out channel 1     &  8 \\
                36 &      isdn quantity out channel 2     &  8 \\
                37 &      dsl quantity out                &  8 \\

                38 &      cpu usage                       &  4 \\

                39 &      fixed text                  & max 20 \\
                40 &      text -$>$ /etc/lcd\_text1.txt  & max 20 \\
                41 &      text -$>$ /etc/lcd\_text2.txt  & max 20 \\
                42 &      text -$>$ /etc/lcd\_text3.txt  & max 20 \\
                43 &      text -$>$ /etc/lcd\_text4.txt  & max 20 \\

                44 &      router uptime                   & 10 \\

            \end{tabular}
            \caption{Übersicht über mögliche Werte für LCD\_TYPE\_x}
            \marklabel{tab:lcd-type-x}{}
          \end{center}
          \end{small}
        \end{table}




        Die beiden nächsten Zahlen in \var{LCD\_\-TYPE\_\-1} geben die Position an.
        Format: ``Spalte Zeile'', wobei beide Zahlen bei 0 beginnen.

      Beispiel:

\begin{example}
\begin{verbatim}
        LCD_TYPE_1='4 10 1'   # Status in der 2. Zeile, ab Spalte 11
                    | |  |
                    | |  \--    Zeile im Display 
                    | \-----    Spalte im Display
                    \-------    Anzeigentyp laut Tabelle
\end{verbatim}
\end{example}


        Für Type 39 (fixed text) ist obiges Format noch um den Text zu
        erweitern der angezeigt werden soll.

      Beispiel:

\begin{example}
\begin{verbatim}
        LCD_TYPE_2='39 10 1 Hallo'   # Text "Hallo" in der 2. Zeile 
                                     # ab Spalte 11
\end{verbatim}
\end{example}

      
      Die Typen 40 - 43 holen sich den Text, der angezeigt werden
      soll, aus den Dateien, die in der Typenliste erwähnt sind. Diese
      Dateien werden während der Laufzeit jede Sekunde eingelesen und
      angezeigt.  Sie können von anderen Programmen (z.B. telmond)
      geändert werden.  Damit kann man sich z.B. auf dem Display
      anzeigen lassen, dass man neue \mbox{E-Mails} hat, auch wenn man Offline
      ist (MyJack).  Die Standardtexte für die Datentypen 40-43 werden
      mit folgenden Variablen definiert (und beim Starten des Systems
      in temporärren Dateien zwischengespeichert, deren Namen einfach
      immer durch Einfügen des Indizes in die Zeichenkette
      ``/etc/lcd\_text$<$Zahl$>$.txt'' definiert wird):

\begin{example}
\begin{verbatim}
        LCD_VAR_TEXT1='Text 1'  # -> /etc/lcd_text1.txt
        LCD_VAR_TEXT2='Text 2'  # -> /etc/lcd_text2.txt
        LCD_VAR_TEXT3='Text 3'  # -> /etc/lcd_text3.txt
        LCD_VAR_TEXT4='Text 4'  # -> /etc/lcd_text4.txt
\end{verbatim}
\end{example}

    Ab der Version 1.6.2 ist es auch möglich, in Abhängigkeit vom Onlinestatus
    verschiedene Texte anzeigen zu lassen. Z.B. kann nun während man Online
    ist, die aktuelle Onlinezeit angezeigt werden und wenn man Offline ist,
    wird an gleicher Stelle das Datum und die Uhrzeit angezeigt.
    Dafür sind folgende Variablen hinzugekommen:

\config{LCD\_TYPE\_ONLINE\_N}{LCD\_TYPE\_ONLINE\_N}{LCDTYPEONLINEN}

      Bei \var{LCD\_\-TYPE\_\-ONLINE\_\-N} ist die Anzahl der auszugebenden Datentypen
      anzugeben. Diese Datentypen werden nur angezeigt wenn man Online ist.


\config{LCD\_TYPE\_ONLINE\_x}{LCD\_TYPE\_ONLINE\_x}{LCDTYPEONLINEx}

      \var{LCD\_\-TYPE\_\-ONLINE\_\-x} gibt den Datentyp, die Spalte und die Zeile an,
      wo die gewünschte Information auf dem Display erscheinen soll. Der
      Datentyp ist numerisch codiert.
      Das Format und die Typen entsprechen denen aus der Tabelle bei
      'LCD\_TEXT\_x'.

      Beispiel:
\begin{example}
\begin{verbatim}
        LCD_TYPE_ONLINE_1=' 8  0 0'     # dsl status
\end{verbatim}
\end{example}


\config{LCD\_TYPE\_OFFLINE\_N}{LCD\_TYPE\_OFFLINE\_N}{LCDTYPEOFFLINEN}

      Bei \var{LCD\_\-TYPE\_\-OFFLINE\_\-N} ist die Anzahl der auszugebenden Datentypen
      anzugeben. Diese Datentypen werden nur angezeigt wenn man Offline ist.


\config{LCD\_TYPE\_OFFLINE\_x}{LCD\_TYPE\_OFFLINE\_x}{LCDTYPEOFFLINEx}

        \var{LCD\_\-TYPE\_\-OFFLINE\_\-x} gibt den Datentyp, die Spalte und die Zeile an,
        wo die gewünschte Information auf dem Display erscheinen soll. Der
        Datentyp ist numerisch codiert.
        Das Format und die Typen entsprechen denen aus der Tabelle bei
        'LCD\_TEXT\_x'.

        Beispiel:
\begin{example}
\begin{verbatim}
        LCD_TYPE_OFFLINE_1=' 0  0 0'  # local date
\end{verbatim}
\end{example}

\configlabel{LCD\_DSL\_SPEED\_OUT}{LCDDSLSPEEDOUT}
\config{LCD\_DSL\_SPEED\_IN LCD\_DSL\_SPEED\_OUT}{LCD\_DSL\_SPEED\_IN}{LCDDSLSPEEDIN}
        \var{LCD\_\-DSL\_\-SPEED\_\-IN} und \var{LCD\_\-DSL\_\-SPEED\_\-OUT} 
        dienen zur Skalierung der Balkenanzeige (Typen 14 und 15). Hier werden die maximalen
        Übertragungsraten der DSL-Verbindung angegeben. Prinzipiell kann jeder ganzzahlige 
        Wert angegeben werden. Es kann sogar sinnvoll sein, etwas höhere Werte anzugeben,
        um ein '+' in der letzten Stelle zu vermeiden.
        Bitte auch beachten, dass die tatsächliche Rate etwas höher als der Name des 
        Anschlusses sugeriert, also DSL1000 hat eine Downloadrate (inbound) von 1024 kilobits/s. 

        Beispiel für DSL-Verbindung mit 1024/128 kilobit/s:
\begin{example}
\begin{verbatim}
        LCD_DSL_SPEED_IN='1024' # Bitrate for DSL inbound
        LCD_DSL_SPEED_OUT='128' # Bitrate for DSL outbound
\end{verbatim}
\end{example}

        Für ISDN sind die Werte nicht relevant. 
        
\end{description}

\subsection{isdn\_rate}
 Das Programm \glqq{}isdn\_rate\grqq{} ist der eigentliche Kern des LCD-Paketes. Es erfasst den Status der Circuits
 und gibt die in der Konfigurationsdatei festgelegten Datentypen an den entsprechenden Positionen im LCD
 aus.
 isdn\_rate wird wie folgt aufgerufen:
 
\begin{example}
\begin{verbatim}
 isdn_rate [-ip router-ip] [-port imond-port] [-telmond-port telmond-port]
           [-type hitachi|matrix-orbital|tty] [-config configfilename]
\end{verbatim}
\end{example}

 Die optionalen Parameter haben folgende Bedeutung:

\begin{description}
\item[-ip router-ip]

        Mit \var{-ip} wird die Adresse des Routers bestimmt, dessen IMOND die Daten liefern soll.
        Wird der Parameter weggelassen, wird 127.0.0.1 (localhost) verwendet.
        Es ist möglich, statt der Adresse auch den Namen anzugeben.
        
\item[-port imond-port]
        \var{-port} bestimmt den Port auf dem der IMOND die Daten liefert.          
        Standardmässig wird 5000 verwendet.
        
\item[-telmond-port telmond-port]
        \var{-telmond-port} bestimmt den Port auf dem der TELMOND die Daten liefert.
        5001 ist hier der Defaultwert.
        
\item[-type hitachi|matrix-orbital|tty]
        \var{-type} legt den Typ des Displays fest.
        \var{hitachi} selektiert Displays mit HD44780 kompatible Anzeigen.
        \var{matrix-orbital} entsprechend für Matrix-Orbital-Displays.
        \var{tty} ermöglicht die Ausgabe auf der Konsole.
        WICHTIG: Alle Ausgaben des isdn\_rate erfolgen immer auf stdout. Eine
        Ausgabeumleitung auf die entsprechende Schnittstelle ist also notwendig.
        tty ist der Standardwert.

\item[-config configfilename]
        \var{-config} bestimmt den Pfad und den Namen der lcd.conf-Datei. Diese wird vom
        rc410.lcd-Script erstellt. Der Normalwert ist /var/run/lcd.conf.
        
isdn\_rate gibt es auch in einer Version die unter Windows lauffähig ist. Hierzu muss die Datei 
/var/run/lcd.conf entweder nach dem Start des Routers von diesem in das isdn\_rate-Verzeichnis kopiert werden
oder sie muss von Hand erstellt werden.
Der Aufruf könnte dann wie folgt aussehen:
\begin{example}
\begin{verbatim}
 isdn_rate -ip fli4l -config lcd.conf
\end{verbatim}
\end{example}
               
\end{description}

\subsection{Anschlußbelegung eines LCD-Moduls am Parallelport}
 
\begin{example}
\begin{verbatim}
   13 _____________________________ 1 Draufsicht auf den
      \ o o o o o o o o o o o o o /   Parallelport, Rück-
       \ o o o o o o o o o o o o /    seite PC
     25 ------------------------- 14
\end{verbatim}
\end{example}
   

 Der Anschluß eines LCD-Moduls an den Router wird folgendermaßen aufgetrennt:

 
\begin{example}
\begin{verbatim}
 
 Parallelport-Pin   Beschreibung   LCD-Modul    LCD-Pin
         18-25      GND                             --|
                    GND                          1  --|- Brücke
                    R/W                          5  --|
                    +5V                          2
             1      STROBE         EN(1)         6
             2      D0             D0            7
             3      D1             D1            8
             4      D2             D2            9
             5      D3             D3           10
             6      D4             D4           11
             7      D5             D5           12
             8      D6             D6           13
             9      D7             D7           14
            14      Autofeed       RS            4
            17      Select In      EN(2)         ? (für LCDs mit 2 Controller)

  Bei Display mit Hintergrundbeleuchtung:
                                   HG+          15 (mit Vorwiderstand ca. 20Ohm)
                                   GND          16
\end{verbatim}
\end{example}

  An Pin 3 kann der Abgriff eines $>$= 20kOhm Potis zwischen +5V und GND 
  geschaltet werden. Damit kann der Kontrast des Displays reguliert werden.
  Bei meinem Display (Conrad) liegt Pin 3 direkt an Masse und man kann
  alles einwandfrei erkennen.

\begin{example}
\begin{verbatim}
  
  +5V ---+
         /
         \ <--+
         /    |
         \    |
  GND ---+    +--- VL (Pin 3 - driver input)
\end{verbatim}
\end{example}



\subsection{Anschluß eines 4x40 Displays}

  Da sich der Anschluß eines 4x40 Displays stark von anderen Displays
  unterscheidet, hier ein Beispiel (Conrad - NLC-40x4x05):

\begin{example}
\begin{verbatim}

Parallelport-Pin   Beschreibung   LCD-Modul           LCD-Pin
        18-25                                             --|
                   GND                                13  --|- Brücke
                   R/W                                10  --|
                   +5V                                14
            1      STROBE         EU (Enable-Upper)    9
            2      D0             D0                   8
            3      D1             D1                   7
            4      D2             D2                   6
            5      D3             D3                   5
            6      D4             D4                   4
            7      D5             D5                   3
            8      D6             D6                   2
            9      D7             D7                   1
           14      Autofeed       RS                  11
           17      Select In      ED (Enable-Down)    15
\end{verbatim}
\end{example}


  An Pin 12 kann der Abgriff eines $>$= 20kOhm Potis zwischen +5V und GND 
  geschaltet werden. Damit kann der Kontrast des Displays reguliert werden.
  Es kann aber auch reichen, Pin 12 direkt an Masse zu legen um alles
  einwandfrei erkennen zu können.

\begin{example}
\begin{verbatim}

  +5V ---+
         /
         \ <--+
         /    |
         \    |
  GND ---+    +--- VL (Pin 12 - driver input)
\end{verbatim}
\end{example}


  \begin{itemize}
  \item Die Leitung ED wird an Pin 17 des parallelen Ports angeschlossen.
  
  \item Das Display wird in der lcd.txt als 2x40 Display definiert.
  
  \item Bei den Typendefinitionen für isdn\_rate wird aber die 4x40 als
    Zeilen-/Spaltengröße angesehen.
  \end{itemize}


  Leider gibt es keinen Standard, was die Pinbelegung des Parallelports
  auf dem Motherboard betrifft. Für die interne Verwendung von LCD-Modulen
  muß man also die Anschlüsse anhand des zum Motherboard mitgelieferten
  Slotblechadapters durchmessen.

  Die erforderliche Spannungsversorgung kann man leider nicht dem Parallelport
  entnehmen, da die Stromaufnahme eines LCD-Modules zu hoch ist. Geeignet dafür
  sind die Anschlüsse für Maus (PS/2), Tastatur (DIN, PS/2), Gameport, USB oder ein
  freier Anschluß vom PC-Netzteil. Da einige Soundkartenhersteller am Gameport
  spezielle Signale generieren, kann keine Garantie übernommen, dass es in jeder
  Kombination funktioniert. Daher gilt hier: Immer vorher messen!

\subsection{Winamp-Verdrahtung eines LCD-Moduls}
 
  Es existieren verschiedene Verdratungsvarianten bei fertig bestückten
  LCD-Displays von Kernel-Concepts, die normale und die
  Winamp-Verdrahtung. Neuere Displays kommen mit letzterer Variante.
 
\begin{example}
\begin{verbatim}
   13 _____________________________ 1 Draufsicht auf den
      \ o o o o o o o o o o o o o /   Parallelport, Rück-
       \ o o o o o o o o o o o o /    seite PC
     25 ------------------------- 14
\end{verbatim}
\end{example}
   

 Der Anschluß eines LCD-Moduls an den Router wird bei der Winamp-Verdrahtung
 folgendermaßen aufgetrennt:

 
\begin{example}
\begin{verbatim}
 
 Parallelport-Pin   Beschreibung   LCD-Modul    LCD-Pin
         18-25      GND                          1
            14      Autofeed       R/W           5
                    +5V                          2
             1      STROBE         EN(1)         6
             2      D0             D0            7
             3      D1             D1            8
             4      D2             D2            9
             5      D3             D3           10
             6      D4             D4           11
             7      D5             D5           12
             8      D6             D6           13
             9      D7             D7           14
            16      Init           RS            4

  Bei Display mit Hintergrundbeleuchtung:
                    +5V            HG+          15
                                   GND          16 (mit Regelwiderstand 100Ohm)
\end{verbatim}
\end{example}

  An Pin 3 kann der Abgriff eines $>$= 10kOhm Potis zwischen +5V und GND 
  geschaltet werden. Damit kann der Kontrast des Displays reguliert werden.

\begin{example}
\begin{verbatim}
  
  +5V ---+
         /
         \ <--+
         /    |
         \    |
  GND ---+    +--- VL (Pin 3 - driver input)
\end{verbatim}
\end{example}


\subsection{Tips und Tricks - Zusammengefasst aus Beiträgen von Robert Resch}

\begin{description}
\item [Anschluss von 2 Displays]

   Mit Hilfe des 2. EN Signals ist es möglich, 2 baugleiche Displays
   parallel zu betreiben. Dabei wird Pin 6 des einen Displays an 
   Pin 1 (EN1) des Par-Ports angeschlossen und an Pin 6 des 2. Display
   wird Pin 17 (EN2) angeschlossen. Alle anderen Leitungen werden
   parallel angeschlossen.


 \item  [Zwei Displayseiten an einem Display]

   Es sind jetzt auch 2 Displayseiten mit folgender Schaltung möglich:
\begin{example}
\begin{verbatim}

   25-pol. Sub-D        LCD

   1 -------|
            |
             \
              \-------- Pin 6

            |
   17-------|

          Schalter
\end{verbatim}
\end{example}

   Dabei wird der gemeinsame Anschluss des Wechslers an Pin 6 des
   Displays angeschlossen. An den beiden anderen Kontakten werden
   die beiden EN-Leitungen angeschlossen.


 \item   [Ansteuerung beider Varianten]

  \sloppypar{Das EN2-Signal wird generiert, sobald eine Zeile z mit
  \var{LCD\_LINES} $< z < 2*$\var{LCD\_LINES}
  angesprochen wird.
  Wenn eine Zeilenzahl noch größer verwendet wird, werden
  beide Displays angesprochen (um die definierbaren Zeichen
  wie z.B. die Balkenanzeige von isdn\_rate) auf beiden Displays
  anzeigen zu können. Beide Display-Seiten können eigene
  definierte Zeichen haben.}

  Bei 4x40 wird also in der $<$config$>$/lcd.txt ein 2x40 Display angegeben.
  Die Zeilen werden aber trotzdem mit 0-3 angesprochen.
  Die Zeilennummern 4 und 5 werden in diesem Fall an beide
  Display-Hälften geschickt. Zeile 4 geht an Zeile 0 und 2,
  Zeile 5 an Zeile 1 und 3.

  Bei z.B. 2 parallel geschalteten 4x20 Displays sieht die
  Ansteuerung folgendermassen aus:

  \begin{itemize}
  \item Zeilen 0-3 werden auf Display 1 dargestellt
    
  \item  Zeilen 4-7 werden auf Display 2 dargestellt
    
  \item Zeilen 8-11 werden auf beiden Displays dargestellt
  \end{itemize}


 \item [Überschwingungen]

  Bei einem langen Kabel oder auch bei bestimmten Parallelports können
  überschwingende Signale Probleme verursachen. Entweder
  das Kabel kürzen oder, wenn das nicht geht, eine Terminierung machen. Dazu
  nimmt man pro Datenleitung (10 Stück!) je einen 10kOhm Widerstand gegen die
  +5V. Das sollte die Schwinger stabilisieren.

\end{description}


\subsection{Danksagung}

  Dank geht an:

  \begin{itemize}
  \item Nils Färber (\email{nils@kernelconcepts.de}) für den Treiber
   
  \item Jürgen Bauer (\email{jb@idowa.net}) für die erste Version von isdn\_rate
   
  \item Frank Meyer (\email{frank@fli4l.de}) für das imond Interface und fli4l :)
   
  \item Ulf Lanz (\email{u.lanz@odn.de}) für die variable Ausgabe von isdn\_rate
   
  \item Robert Resch (\email{rresch@gmx.de}) für Bugfixes und Erweiterungen des
     lcd-Treibers
   
  \item Stefan Krister (\email{Stefan.Krister@keimfarben.de}) für die erste
     Überarbeitung der Dokumentation
   
   \item Nicole Hornung (\email{fli4l@xerotech.de}) für ein paar Verbesserungen
     an der Doku
   
   \item Gerrit Lammert (\email{gerrit@fli4l.de}) für das Umsetzen der Text-Doku
     in die HTML-Doku
  \end{itemize}


    Für Fragen, Anregungen Kritik usw.:

    Schreibt eine \mbox{E-Mail} an Gernot Miksch \email{ibgm@gmx.de}

