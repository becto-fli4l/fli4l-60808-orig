% Synchronized to r44350
\marklabel{sec:opt-lcd }
{
  \section {LCD - Displaying Status Informations On A LC-Display}
}

\subsection{Introduction}


    This package enables a LCD module connected to the parallel port of 
    fli4l. Serial LCD modules by Matrix-Orbital can be used too.
    Some more filters for special displays exist.

    Informations like date, time, values of actual load, ISDN- or DSL 
    up- and download rates/bars in kb/s can be shown on the display.


\subsection{Konfiguration}\index{OPT\_LCD}


    To use the package lcd configure the following variables:

\begin{example}
\begin{verbatim}
        OPT_LCD='yes'  (Default setting: OPT_LCD='no')
\end{verbatim}
\end{example}


\begin{description}
  \config{LCD\_COLS}{LCD\_COLS}{LCDCOLS} - Amount of characters per line.
  The kernel module supports values of 16, 20, 24, 32 and 40 at the moment. 
  Modules with 8 or 27 charcters per line usually will work as well.
  fli4l will transfer 16 res. 40 to the kernel module.

\config{LCD\_LINES}{LCD\_LINES}{LCDLINES} - Amount of lines. 

    Possible values: 1, 2 and 4.
    
    Attention: Displays on parallel ports with two controller chips (4x16,
    4x40 etc.)  have to be defined with two lines! The coordinates of the 
    values to be shown can be specified as usual. The LCD driver decides 
    which of the two controllers has to be addressed depending on coordinates.
    
\config{LCD\_ADDRESS}{LCD\_ADDRESS}{LCDADDRESS}

      IO address of the LPT port, i.e. '0x278'
      
      If a serial display by Matrix-Orbital is used specify the serial port 
      to be used here i.e. 'com1' or 'com2'. With 'null' LAN-only is set, see: 
      \var{LCD\_LANIP}
      
      Displaying on a monitor:
      'console' or tty1 choose the main screen. This may lead to output mixed 
      with normal messages and hence is not advised. 'tty2', 'tty3' ... 'tty9' 
      select other virtual consoles reached by typing ALT-F2 ... F9. ALT-F1 
      returns to the main screen.

        \wichtig{
        parallel interfaces on motherboards or on ISA adapters are fully 
        supported. PCI adapters with parallel ports provided by NETMOS-Chips
        can be also used.}

\begin{example}
\begin{verbatim}
        cat /proc/pci
\end{verbatim}
\end{example}

        \emph{lists PCI devices discovered. Select the device with matching 
        Vendor-ID and Device-ID and choose one of the following IO addresses:}

        \begin{itemize}
        \item Nm9705CV  (Vendor id=9710, Device id=9705, Port1 1st entry)
        \item Nm9735CV  (Vendor id=9710, Device id=9735, Port1 3rd entry)
        \item Nm9805CV  (Vendor id=9710, Device id=9805, Port1 1st entry)
        \item Nm9715CV  (Vendor id=9710, Device id=9815, Port1 1st entry, Port2 3rd entry)
        \item Nm9835CV  (Vendor id=9710, Device id=9835, Port1 3rd entry)
        \item Nm9755CV  (Vendor id=9710, Device id=9855, Port1 1st entry, Port2 3rd entry)
        \end{itemize}

        \emph{Consider this as an experimental feature because of the lack of 
        matching hardware to test. Please report errors on the newsgroups!}

\config{LCD\_LANIP}{LCD\_LANIP}{LCDLANIP} (optional variable)

      IP address of a 16x2 displays on a Pollin AVR-NET-IO in LAN
      or other displays on ethersex - see further variables
      (experimental) 

\config{LCD\_LANTYPE}{LCD\_LANTYPE}{LCDLANTYPE} (optional variable)
      
      Type of firmware.
      Choose: 'pollin' (default) - Original firmware AVR-NET-IO
      'ethersex' - firmware by www.ethersex.de with active LCD
      
\config{LCD\_LANUSER}{LCD\_LANUSER}{LCDLANUSER} (optional variable)
      
      Authentification for ethersex if PAM is set to ecmd/tcp.
      Here: Username

\config{LCD\_LANPASS}{LCD\_LANPASS}{LCDLANPASS} (optional variable)
      
      Password belonging to \var{LCD\_LANUSER}
      
\configlabel{LCD\_TIME\_LONG}{LCDTIMELONG}
\configlabel{LCD\_TIME\_SHORT}{LCDTIMESHORT}
\configvar {LCD\_TIME\_LONG   LCD\_TIME\_SHORT}

      These are two timer values for the IO port of the LCD-Display.
      If blank the following defaults will be used:

\begin{example}
\begin{verbatim}
        LCD_TIME_LONG='100'
        LCD_TIME_SHORT='40'
\end{verbatim}
\end{example}

      If problems occur, for example distorted output on the LCD display 
      try to use higher values:
      
\begin{example}
\begin{verbatim}
        LCD_TIME_LONG='120'
        LCD_TIME_SHORT='60'
\end{verbatim}
\end{example}

      These variables have no meaning for serial displays by Matrix-Orbital.
      
\config{LCD\_ADDR\_TYPE}{LCD\_ADDR\_TYPE}{LCDADDRTYPE}
 - Way of addressing the LCD controller.

\begin{example}
\begin{verbatim}
        LCD_ADDR_TYPE='0'     # For HD44780 and compatible
        LCD_ADDR_TYPE='1'     # For HD66712 and compatible
        LCD_ADDR_TYPE='2'     # Obsolete
        
\end{verbatim}
\end{example}
    
    These variables have no meaning for serial displays by Matrix-Orbital.

\config{LCD\_WINAMP}{LCD\_WINAMP}{LCDWINAMP}

Ready-to-use LCD displays by Kernel-Concepts exist in different wiring 
variants: normal and Winamp-wiring. Newer displays use the latter, specify
'yes' in this case.

\config{LCD\_FILTER}{LCD\_FILTER}{LCDFILTER}

 - Filter for special displays.
At the moment filters exist for
- ipc\_a78 displays

\begin{example}
\begin{verbatim}
        LCD_FILTER='mo2ipc_a78'     # For mo2ipc_a78 displays
\end{verbatim}
\end{example}

\config{LCD\_START\_MSG}{LCD\_START\_MSG}{LCDSTARTMSG}

       The message specified here will be shown on the display during system startup 
       shortly after driver loading. It should not exceed the length of one line 
       because longer texts may not be shown completely.

\config{LCD\_STOP\_MSG}{LCD\_STOP\_MSG}{LCDSTOPMSG}

       The message specified here will be shown on the display during system shutdown. 
       It should not exceed the length of one line because longer texts may not be 
       shown completely.

\config{LCD\_REBOOT\_MSG}{LCD\_REBOOT\_MSG}{LCDREBOOTMSG}

       The message specified here will be shown on the display during system reboot. 
       It should not exceed the length of one line because longer texts may not be 
       shown completely.
       
\config{LCD\_START\_ISDN\_RATE}{LCD\_START\_ISDN\_RATE}{LCDSTARTISDNRATE}

      Specifies if program isdn\_rate should be started.

\config{LCD\_TYPE\_N}{LCD\_TYPE\_N}{LCDTYPEN}

      Output format of program isdn\_rate can be largely customized to the personal 
      taste of the user.

      \var{LCD\_\-TYPE\_\-N} sets the number of data types to be shown.
      Data types are always shown, independent of being online or not.


\config{LCD\_TYPE\_x}{LCD\_TYPE\_x}{LCDTYPEx}

        \var{LCD\_\-TYPE\_\-x} specifies data type and column/row of the 
        display in which the information should appear. Data types are coded 
        numerically. Possible values: see table \ref{tab:lcd-type-x}.

        \begin{table}[htbp]
          \begin{small}
          \begin{center}
            \begin{tabular}{rlr}
               Typ &     Information   &             Characters\\


                0 &       local date dd.mm.yyyy           & 10 \\
                1 &       local date dd.mm.yy             &  8 \\
                2 &       local time hh:mm:ss             &  8 \\

                3 &       remote date dd.mm.yyyy          & 10 \\
                4 &       remote date dd.mm.yy            &  8 \\
                5 &       remote time hh:mm:ss            &  8 \\

                6 &       isdn status channel 1           &  7 \\
                7 &       isdn status channel 2           &  7 \\
                8 &       dsl status                      &  7 \\

                9 &       isdn circuit name channel 1     & 16 \\
                10 &      isdn circuit name channel 2     & 16 \\
                11 &      dsl circuit name                & 16 \\

                12 &      isdn input rate bar             &  8 \\
                13 &      isdn output rate bar            &  8 \\
                14 &      dsl input rate bar              &  8 \\
                15 &      dsl output rate bar             &  8 \\

                16 &      isdn input rate                 &  5 \\
                17 &      isdn output rate                &  5 \\
                18 &      dsl input rate                  &  9 \\
                19 &      dsl output rate                 &  9 \\

                20 &      isdn charge channel 1           &  6 \\
                21 &      isdn charge channel 2           &  6 \\
                22 &      dsl charge                      &  6 \\

                23 &      isdn ip address channel 1       & 15 \\
                24 &      isdn ip address channel 2       & 15 \\
                25 &      dsl ip address                  & 15 \\

                26 &      load 1                          &  5 \\
                27 &      load 2                          &  5 \\

                28 &      phone                           & 16 \\

                29 &      isdn online time channel 1      &  8 \\
                30 &      isdn online time channel 2      &  8 \\
                31 &      dsl online time                 &  8 \\

                32 &      isdn quantity in channel 1      &  8 \\
                33 &      isdn quantity in channel 2      &  8 \\
                34 &      dsl quantity in                 &  8 \\

                35 &      isdn quantity out channel 1     &  8 \\
                36 &      isdn quantity out channel 2     &  8 \\
                37 &      dsl quantity out                &  8 \\

                38 &      cpu usage                       &  4 \\

                39 &      fixed text                  & max 20 \\
                40 &      text -$>$ /etc/lcd\_text1.txt  & max 20 \\
                41 &      text -$>$ /etc/lcd\_text2.txt  & max 20 \\
                42 &      text -$>$ /etc/lcd\_text3.txt  & max 20 \\
                43 &      text -$>$ /etc/lcd\_text4.txt  & max 20 \\

                44 &      router uptime                   & 10 \\

            \end{tabular}
            \caption{Overview of possible values for LCD\_TYPE\_x}
            \marklabel{tab:lcd-type-x}{}
          \end{center}
          \end{small}
        \end{table}




        The following two digits in \var{LCD\_\-TYPE\_\-1} set the position.
        Format: ``column row'', whereas both digits start with '0'.

      Example:

\begin{example}
\begin{verbatim}
        LCD_TYPE_1='4 10 1'   # Status in 2nd row, starting from column 11
                    | |  |
                    | |  \--    row in display 
                    | \-----    column in display
                    \-------    Information type according to table
\end{verbatim}
\end{example}


        For Type 39 (fixed text) expand format above with a text to be 
        displayed.

      Example:

\begin{example}
\begin{verbatim}
        LCD_TYPE_2='39 10 1 Hello'   # Text "Hello" in 2nd row 
                                     # starting at column 11
\end{verbatim}
\end{example}

      
      Types 40 - 43 fetch text to be displayed from the files mentioned 
      in the type list. These files are read and displayed every second 
      during runtime. They can be read and changed by other programs as 
      well, (i.e. telmond). Possible usecases are messages for incoming 
      \mbox{E-Mails} even while being offline (MyJack). Standard texts 
      for data types 40-43 are definded by the following variables (and 
      are spooled to temporary files during system startup, names are
      generated by pasting of indexes into the string
      ``/etc/lcd\_text$<$Zahl$>$.txt''):

\begin{example}
\begin{verbatim}
        LCD_VAR_TEXT1='Text 1'  # -> /etc/lcd_text1.txt
        LCD_VAR_TEXT2='Text 2'  # -> /etc/lcd_text2.txt
        LCD_VAR_TEXT3='Text 3'  # -> /etc/lcd_text3.txt
        LCD_VAR_TEXT4='Text 4'  # -> /etc/lcd_text4.txt
\end{verbatim}
\end{example}

    Texts can also be displayed depending on online state (i.e. online time 
    only while being online and while being offline date and time at the same 
    place).
    Configure the following variables:

\config{LCD\_TYPE\_ONLINE\_N}{LCD\_TYPE\_ONLINE\_N}{LCDTYPEONLINEN}

      Set the amount of data types to be shown in \var{LCD\_\-TYPE\_\-ONLINE\_\-N}. 
      They will only be shown while being online.


\config{LCD\_TYPE\_ONLINE\_x}{LCD\_TYPE\_ONLINE\_x}{LCDTYPEONLINEx}

      \var{LCD\_\-TYPE\_\-ONLINE\_\-x} Data type, column and row where to 
      show the information on the display. Data types are coded numerically.
      Format and type match the ones from the 'LCD\_TEXT\_x' table.

      Example:
\begin{example}
\begin{verbatim}
        LCD_TYPE_ONLINE_1=' 8  0 0'     # dsl status
\end{verbatim}
\end{example}


\config{LCD\_TYPE\_OFFLINE\_N}{LCD\_TYPE\_OFFLINE\_N}{LCDTYPEOFFLINEN}

      \var{LCD\_\-TYPE\_\-OFFLINE\_\-N} sets the number of data types to show.
      These data types are only displayed while being offline.


\config{LCD\_TYPE\_OFFLINE\_x}{LCD\_TYPE\_OFFLINE\_x}{LCDTYPEOFFLINEx}

        \var{LCD\_\-TYPE\_\-OFFLINE\_\-x} Data type, column and row where to 
      show the information on the display. Data types are coded numerically.
      Format and type match the ones from the 'LCD\_TEXT\_x' table.

        Example:
\begin{example}
\begin{verbatim}
        LCD_TYPE_OFFLINE_1=' 0  0 0'  # local date
\end{verbatim}
\end{example}

\configlabel{LCD\_DSL\_SPEED\_OUT}{LCDDSLSPEEDOUT}
\config{LCD\_DSL\_SPEED\_IN LCD\_DSL\_SPEED\_OUT}{LCD\_DSL\_SPEED\_IN}{LCDDSLSPEEDIN}
        \var{LCD\_\-DSL\_\-SPEED\_\-IN} und \var{LCD\_\-DSL\_\-SPEED\_\-OUT} 
        is used for scaling bar display (type 14 and 15). Maximum data transfer rates of a 
        DSL connection are set here. In principle any integer value can be specified. 
        It may be even useful to set slightly higher values to avoid a plus sign at the end 
        of the bar.
        Please note that connection names may be slightly different from real data transfer 
        rates, for example DSL1000 has a maximum download rate (inbound) of 1024 kilobits/s. 

        Example for a DSL connection with 1024/128 kilobit/s:
\begin{example}
\begin{verbatim}
        LCD_DSL_SPEED_IN='1024' # Bitrate for DSL inbound
        LCD_DSL_SPEED_OUT='128' # Bitrate for DSL outbound
\end{verbatim}
\end{example}

        These values are irrelevant for ISDN connections. 
        
\end{description}

\subsection{isdn\_rate}
 The program \glqq{}isdn\_rate\grqq{} is the core of package LCD. It monitors circuit states 
 and puts out the data types mentioned in the config file at the matching position in the LCD
 display.
 isdn\_rate is executed as follows:
 
\begin{example}
\begin{verbatim}
 isdn_rate [-ip router-ip] [-port imond-port] [-telmond-port telmond-port]
           [-type hitachi|matrix-orbital|tty] [-config configfilename]
\end{verbatim}
\end{example}

 Optional parameters have the following meanings:

\begin{description}
\item[-ip router-ip]

        IMOND and/or TELMOND running on a router are used as data sources. 
        \var{-ip} sets the corresponding router IP address.
        If the parameter is omitted 127.0.0.1 (localhost) will be used.
        The name of the router may be used instead of the address as well.
        
\item[-port imond-port]
        \var{-port} sets the port IMOND can be reached at.          
        Default port is 5000.
        
\item[-telmond-port telmond-port]
        \var{-telmond-port} sets the port TELMOND can be reached at.
        Default port is 5001.
        
\item[-type hitachi|matrix-orbital|tty]
        \var{-type} sets the display type.
        \var{hitachi} selects displays compatible to HD44780.
        \var{matrix-orbital} Matrix-Orbital displays.
        \var{tty} directs output to console.
        IMPORTANT: All output of isdn\_rate is directed to stdout. Redirection to 
        according interfaces is needed because of this.
        Default value is tty.

\item[-config configfilename]
        \var{-config} sets the path and the name of the lcd.conf file to be created by 
        the script rc410.lcd. Default value is /var/run/lcd.conf.
        
A windows version of isdn\_rate exists. To use it the file /var/run/lcd.conf either 
has to be copied manually into the isdn\_rate directory after router startup or has to 
be created by hand.
Execution could work like this:
\begin{example}
\begin{verbatim}
 isdn_rate -ip fli4l -config lcd.conf
\end{verbatim}
\end{example}
               
\end{description}

\subsection{Pin wiring of a LCD module connected to the parallel port}
 
\begin{example}
\begin{verbatim}
   13 _____________________________ 1 Front view of a 
      \ o o o o o o o o o o o o o /   parallel port, rear
       \ o o o o o o o o o o o o /    side of a PC
     25 ------------------------- 14
\end{verbatim}
\end{example}
   

 Connecting a LCD module:

 
\begin{example}
\begin{verbatim}
 
 Parallelport-Pin   Beschreibung   LCD-Modul    LCD-Pin
         18-25      GND                             --|
                    GND                          1  --|- Bridge
                    R/W                          5  --|
                    +5V                          2
             1      STROBE         EN(1)         6
             2      D0             D0            7
             3      D1             D1            8
             4      D2             D2            9
             5      D3             D3           10
             6      D4             D4           11
             7      D5             D5           12
             8      D6             D6           13
             9      D7             D7           14
            14      Autofeed       RS            4
            17      Select In      EN(2)         ? (for LCD with 2 controllers)

  Display with backlight:
                                   HG+          15 (with series resistor ~ 20 Ohm)
                                   GND          16
\end{verbatim}
\end{example}

  Connect a $>$= 20kOhm Poti between +5V and GND on pin 3 to control display contrast. 
  For my display (Conrad) Pin 3 is directly connected to ground.

\begin{example}
\begin{verbatim}
  
  +5V ---+
         /
         \ <--+
         /    |
         \    |
  GND ---+    +--- VL (Pin 3 - driver input)
\end{verbatim}
\end{example}



\subsection{Connection of a 4x40 displays}

  Connection of a 4x40 displays differs vastly from other displays -
  see this Example (Conrad - NLC-40x4x05):

\begin{example}
\begin{verbatim}

Parallel port pin   Description   LCD module           LCD pin
        18-25                                             --|
                   GND                                13  --|- bridge
                   R/W                                10  --|
                   +5V                                14
            1      STROBE         EU (Enable-Upper)    9
            2      D0             D0                   8
            3      D1             D1                   7
            4      D2             D2                   6
            5      D3             D3                   5
            6      D4             D4                   4
            7      D5             D5                   3
            8      D6             D6                   2
            9      D7             D7                   1
           14      Autofeed       RS                  11
           17      Select In      ED (Enable-Down)    15
\end{verbatim}
\end{example}


  Connect a $>$= 20kOhm Poti between +5V and GND on pin 12 to control display 
  contrast. It may be enough to connect pin 12 directly to ground to achieve 
  readable display.
  
\begin{example}
\begin{verbatim}

  +5V ---+
         /
         \ <--+
         /    |
         \    |
  GND ---+    +--- VL (Pin 12 - driver input)
\end{verbatim}
\end{example}


  \begin{itemize}
  \item Connect ED wire to pin 17 of the parallel port.
  
  \item Define the display as 2x40 in lcd.txt.
  
  \item As type definition for isdn\_rate set 4x40 for rown-/column size instead.
  \end{itemize}


  Unfortunately no standard exists for the pin wiring of a motherboard's internal parallel 
  port connector. For internal use of LCD modules the pin connection has to be found in 
  another way.

  Power supply for a LCD module can't be taken directly from the parallel port because 
  of its current draw being too high. Use mouse (PS/2), keyboard (DIN, PS/2), game port, 
  USB or PC power supply instead. Game ports may be complicated as well. Always use a 
  voltmeter at first! No guarantees can be given, your mileage may vary.
  
\subsection{Winamp-wiring of a LCD module}
 
\begin{example}
\begin{verbatim}
   13 _____________________________ 1 Front view of a 
      \ o o o o o o o o o o o o o /   parallel port, rear
       \ o o o o o o o o o o o o /    side of a PC
     25 ------------------------- 14
\end{verbatim}
\end{example}
   

 Connecting a LCD module with Winamp wiring:

 
\begin{example}
\begin{verbatim}
 
 Parallel port pin  description    LCD module   LCD pin
         18-25      GND                          1
            14      Autofeed       R/W           5
                    +5V                          2
             1      STROBE         EN(1)         6
             2      D0             D0            7
             3      D1             D1            8
             4      D2             D2            9
             5      D3             D3           10
             6      D4             D4           11
             7      D5             D5           12
             8      D6             D6           13
             9      D7             D7           14
            16      Init           RS            4

  Displays with backlight:
                    +5V            HG+          15
                                   GND          16 (with regulating series resistor 100 Ohm)
\end{verbatim}
\end{example}

  Connect a $>$= 10kOhm Poti between +5V and GND on pin 3 to control display contrast. 
  For my display (Conrad) Pin 3 is directly connected to ground.

\begin{example}
\begin{verbatim}
  
  +5V ---+
         /
         \ <--+
         /    |
         \    |
  GND ---+    +--- VL (Pin 3 - driver input)
\end{verbatim}
\end{example}


\subsection{Tips and Tricks}

\begin{description}
\item [Connecting 2 displays]

   By aid of the 2nd EN signal it is possible to use 2 identical displays 
   in parallel zu. To accomplish this Pin 6 of the first display is connected 
   to pin 1 (EN1) of the parallel port while pin 6 of the 2nd display is 
   wired to pin 17 (EN2). All other wires are connected in parallel.


 \item  [Two sided displays]

   For two sided displays make the following connections:
\begin{example}
\begin{verbatim}

   25-pol. Sub-D        LCD

   1 -------|
            |
             \
              \-------- Pin 6

            |
   17-------|

          switch
\end{verbatim}
\end{example}

   Pin 6 of the display is connected to the common side of the switch. The two 
   EN wires are connected to the two pins of the switching side.


 \item   [Driving both variants]

  \sloppypar{EN2 signal is generated when a row z is accessed with
  \var{LCD\_LINES} $< z < 2*$\var{LCD\_LINES}.
  If a row number higher than that is used both displays are adressed 
  to get definable chars (i.e. isdn\_rate's bar display) on both displays. 
  Both display sides can have their own defined chars.}

  Hence for a 4x40 a 2x40 display is specified in $<$config$>$/lcd.txt.
  Rows are accessed with 0-3. row numbers 4 and 5 are sent to both displays 
  in this case. Row 4 is sent to row 0 and 2,
  row 5 is sent to row 1 and 3.

  Addressing i.e. 2 parallel 4x20 displays looks like this:

  \begin{itemize}
  \item row 0-3 are shown on display 1
    
  \item  row 4-7 are shown on display 2 
    
  \item row 8-11 are shown on both displays 
  \end{itemize}


 \item [Oscillation loops]

  With long cables or certain parallel ports oscillation loops can cause troubles. 
  Either use shorter cables or if this is not feasible use a termination. Take 
  a 10kOhm resistor for each of the 10 data wires connected to +5V. 
  
\end{description}

