% Do not remove the next line
% Synchronized to r44350

\marklabel{sec:opt-lcd }
{
  \section {LCD - Affichage des informations sur écran LCD}
}

\subsection{Introduction}


    Avec ce paquetage, vous pouvez relier un module-LCD au port parallèle de
    l'ordinateur fli4l. Maintenant, il est possible d'utiliser les modules-LCD
    sur le port série de la société Matrix-Orbital. En outre, il existe aussi
    un filtre spécial pour écran.

    On peut afficher sur cet écran, les informations suivantes~: la date, l'heure,
    les valeurs de la charge CPU et naturellement le débit UP et Down de l'ISDN
    ou de la DSL en Kb/s, ces valeurs seront indiqués par une barre.


\subsection{Configuration}\index{OPT\_LCD}

    Si vous voulez utiliser le paquetage LCD, vous devez d'abord définir les
    variables suivantes~:

\begin{example}
\begin{verbatim}
        OPT_LCD='yes'

       (Configuration par défaut : OPT_LCD='no')
\end{verbatim}
\end{example}


\begin{description}
  \config{LCD\_COLS}{LCD\_COLS}{LCDCOLS} - Nombre de caractères par ligne.

  Les modules pris actuellement en charge par le kernel, pour le nombre de
  caractère sont 16, 20, 24, 32 et 40, en général les modules avec 8 et 27
  caractères fonctionnent également. Vous devez installer les module 16 ou 40
  caractères pour les faire fonctionner avec le kernel de fli4l.

\config{LCD\_LINES}{LCD\_LINES}{LCDLINES} - Nombre de lignes.

    Valeurs possibles~: 1, 2 et 4.

    Attention~: les écrans (4x16, 4x40 etc.) pour port Parallèle qui ont
    deux puces contrôleurs, doivent être définis sur 2 lignes~! Les valeurs
    pour les coordonnées affichage sont indiquées normalement. Le pilote-LCD
    décide ensuite avec lequel des deux contrôleurs il doit afficher des
    coordonnés par rapport au nombre de lignes.

\config{LCD\_ADDRESS}{LCD\_ADDRESS}{LCDADDRESS}

      Adresse-IO du Port-LPT, exemple '0x278'

      Si vous utilisez un écran Matrix-Orbital sur le port série, l'interface
      série utilisée doit être indiquer ici, par exemple 'com1' ou 'com2'. Avec
      le paramètre 'null', la configuration de écran passera uniquement par le
      LAN (ou réseau local), voir~: \var{LCD\_LANIP}.

      Il est possible de transférer la diffusion des messages sur l'écran de
      l'ordinateur~: la 'console' ou tty1 doit être choisi comme écran principal.
      Cependant, il n'est pas recommandé d'utiliser 'tty1' car l'affichage des
      messages pourrait se mélanger avec l'affichage normal de fli4l. Vous pouvez
      sélectionner d'autres consoles virtuelles exemple 'tty2', 'tty3' ... 'tty9'
      qui pourront être atteint avec les touches ALT-F2, ALT-F3 .... Avec la touche
      ALT-F1 vous pouvez revenir à nouveau sur l'écran principal.

        \wichtig{
        Jusqu'à présent on utilisait, seulement les interfaces parallèles sur les
        cartes-mères ou sur les cartes-ISA. Les interfaces parallèles sur les cartes-PCI
        ne pouvaient pas être utilisées. Avec cette version vous pouvez configurer
        les interfaces parallèles en utilisant les cartes-PCI avec la PUCE-NETMO.
        Au moyen de la commande }

\begin{example}
\begin{verbatim}
        cat /proc/pci
\end{verbatim}
\end{example}

        \emph{Les Périphériques-PCI reconnus seront affichés. On cherche ensuite le
        Périphérique approprié avec l'identification Vendor-ID et identification
        Device-ID, pour être ensuite utilisé comme adresse IO, voici les entrées
        suivantes~:}

        \begin{itemize}
        \item Nm9705CV  (Vendor id=9710, Device id=9705, Port1 1. Entrée)
        \item Nm9735CV  (Vendor id=9710, Device id=9735, Port1 3. Entrée)
        \item Nm9805CV  (Vendor id=9710, Device id=9805, Port1 1. Entrée)
        \item Nm9715CV  (Vendor id=9710, Device id=9815, Port1 1. Entrée, Port2 3. Entrée)
        \item Nm9835CV  (Vendor id=9710, Device id=9835, Port1 3. Entrée)
        \item Nm9755CV  (Vendor id=9710, Device id=9855, Port1 1. Entrée, Port2 3. Entrée)
        \end{itemize}

        \emph{La configuration à été installée sans disposer du matériel correspondant
        pour les tests. Ainsi, veuillez considérer cette fonctionnalité comme
        expérimentale. En cas d'erreur, veuillez s'il vous plaît, nous envoyer
        un message détaillé de l'erreur sur le newsgroup~!}

\config{LCD\_LANIP}{LCD\_LANIP}{LCDLANIP} (Variable optionnelle )

      On indique dans cette variable l'adresse-IP pour les cartes contrôleurs AVR-NET-IO
      de Pollin Electronic avec un écran 2x16 dans un LAN (ou réseau local) ou
      pour les cartes contrôleurs avec firmware ethersex - voir les variables
      suivantes. (cette variable est expérimental)

\config{LCD\_LANTYPE}{LCD\_LANTYPE}{LCDLANTYPE} (optionale Variable)

      On indique dans cette variable le type de Firmware.
      Les choix sont~: 'pollin' (par défaut) - Original-Firmware AVR-NET-IO,
      'ethersex' - Firmware de www.ethersex.de avec LCD actif

\config{LCD\_LANUSER}{LCD\_LANUSER}{LCDLANUSER} (optionale Variable)

      Authentification pour ethersex, quand PAM et configuré sur le protocole
      ecmd/tcp.
      Vous indiquez ici~: Le nom d'utilisateur

\config{LCD\_LANPASS}{LCD\_LANPASS}{LCDLANPASS} (optionale Variable)

      Vous indiquez ici le Mot de passe en relation avec \var{LCD\_LANUSER}

\configlabel{LCD\_TIME\_LONG}{LCDTIMELONG}
\configlabel{LCD\_TIME\_SHORT}{LCDTIMESHORT}
\configvar {LCD\_TIME\_LONG   LCD\_TIME\_SHORT}

      Avec ces deux variables, vous pouvez indiquer les valeurs de l'horloge du
      port-IO pour l'écran LCD. Si les variables sont laissées vides, les paramètres
      suivants sont pris par défaut~:

\begin{example}
\begin{verbatim}
        LCD_TIME_LONG='100'
        LCD_TIME_SHORT='40'
\end{verbatim}
\end{example}

      Si vous avez des problèmes d'affichage avec l'écran LCD, par exemple si vous
      voyez des caractères bizarres qui apparaissent, vous devez augmenter ces
      valeurs, par exemple.

\begin{example}
\begin{verbatim}
        LCD_TIME_LONG='120'
        LCD_TIME_SHORT='60'
\end{verbatim}
\end{example}

      Ces variables sont sans importances, si vous utilisez un LCD Matrix-Orbital
      sur port série.

\config{LCD\_ADDR\_TYPE}{LCD\_ADDR\_TYPE}{LCDADDRTYPE}
 - Type d'adressage pour le contrôleur-LCD.

\begin{example}
\begin{verbatim}
        LCD_ADDR_TYPE='0'     # Pour HD44780 et Controlleur compatible
        LCD_ADDR_TYPE='1'     # Pour HD66712 et Controlleur compatible
        LCD_ADDR_TYPE='2'     # Obsolète, cette fonctionnalité est maintenant
                              # disponible avec l'installation '0'
\end{verbatim}
\end{example}

    Ces variables sont sans importances, si vous utilisez un LCD Matrix-Orbital
    sur port série.

\config{LCD\_WINAMP}{LCD\_WINAMP}{LCDWINAMP}

Il existe différente variante de câblage pour les écrans-LCD, prêt équipé autour
du concept kernel, le câblage normal et le câblage Winamp. Cette dernière variante
est utilisée pour les nouveaux écrans LCD, vous devez indiquer ici yes.

\config{LCD\_FILTER}{LCD\_FILTER}{LCDFILTER}

- Filtre pour écrans spéciaux.
A l'heure actuelle, il existe des filtres pour
- écran ipc\_a78

\begin{example}
\begin{verbatim}
        LCD_FILTER='mo2ipc_a78'     # Pour écran mo2ipc_a78
\end{verbatim}
\end{example}

\config{LCD\_START\_MSG}{LCD\_START\_MSG}{LCDSTARTMSG}

       Le message enregistré ici sera affiché sur l'écran, lors du démarrage du
       système peu après le chargement des pilotes. Le message ne doit pas dépasser
       la longueur d'une ligne, dans le cas contraire nous ne garantissons pas
       l'affichage complet du texte.

\config{LCD\_STOP\_MSG}{LCD\_STOP\_MSG}{LCDSTOPMSG}

       Le message enregistré ici sera affiché sur l'écran, lors de l'arrêt du
       système. Le message ne doit pas dépasser la longueur d'une ligne, dans
       le cas contraire nous ne garantissons pas l'affichage complet du texte.

\config{LCD\_REBOOT\_MSG}{LCD\_REBOOT\_MSG}{LCDREBOOTMSG}

       Le message enregistré ici sera affiché sur l'écran, lors du redémarrage du
       système. Le message ne doit pas dépasser la longueur d'une ligne, dans le
       cas contraire nous ne garantissons pas l'affichage complet du texte.

\config{LCD\_START\_ISDN\_RATE}{LCD\_START\_ISDN\_RATE}{LCDSTARTISDNRATE}

      Dans cette variable on indique, si le programme isdn\_rate doit être démarré.

\config{LCD\_TYPE\_N}{LCD\_TYPE\_N}{LCDTYPEN}

      Ulf Lanz a créé un format de distribution pour le programme-isdn\_rate.
      Ainsi chaque utilisateur peut composer son message selon ses désirs.

      Dans la variable \var{LCD\_\-TYPE\_\-N} vous indiquez le nombre de message
      ou de types de données à afficher. Ces types de données sont toujours
      affichés, que se soit en on-line ou en off-line.

\config{LCD\_TYPE\_x}{LCD\_TYPE\_x}{LCDTYPEx}

        Dans la variable \var{LCD\_\-TYPE\_\-x} vous indiquez le type de données
        souhaitées à affichage, dans les lignes et les colonnes, ces informations
        apparaîtront sur l'écran. Les types de données sont codés numériquement.
        Les valeurs possibles sont mentionnées dans le Tableau \ref{tab:lcd-type-x}.


        \begin{table}[htbp]
          \begin{small}
          \begin{center}
            \begin{tabular}{rlr}
               Type &     Information   &        Nombre de caractères \\


                0 &       local date dd.mm.yyyy           & 10 \\
                1 &       local date dd.mm.yy             &  8 \\
                2 &       local time hh:mm:ss             &  8 \\

                3 &       remote date dd.mm.yyyy          & 10 \\
                4 &       remote date dd.mm.yy            &  8 \\
                5 &       remote time hh:mm:ss            &  8 \\

                6 &       isdn status channel 1           &  7 \\
                7 &       isdn status channel 2           &  7 \\
                8 &       dsl status                      &  7 \\

                9 &       isdn circuit name channel 1     & 16 \\
                10 &      isdn circuit name channel 2     & 16 \\
                11 &      dsl circuit name                & 16 \\

                12 &      isdn input rate bar             &  8 \\
                13 &      isdn output rate bar            &  8 \\
                14 &      dsl input rate bar              &  8 \\
                15 &      dsl output rate bar             &  8 \\

                16 &      isdn input rate                 &  5 \\
                17 &      isdn output rate                &  5 \\
                18 &      dsl input rate                  &  9 \\
                19 &      dsl output rate                 &  9 \\

                20 &      isdn charge channel 1           &  6 \\
                21 &      isdn charge channel 2           &  6 \\
                22 &      dsl charge                      &  6 \\

                23 &      isdn ip address channel 1       & 15 \\
                24 &      isdn ip address channel 2       & 15 \\
                25 &      dsl ip address                  & 15 \\

                26 &      load 1                          &  5 \\
                27 &      load 2                          &  5 \\

                28 &      phone                           & 16 \\

                29 &      isdn online time channel 1      &  8 \\
                30 &      isdn online time channel 2      &  8 \\
                31 &      dsl online time                 &  8 \\

                32 &      isdn quantity in channel 1      &  8 \\
                33 &      isdn quantity in channel 2      &  8 \\
                34 &      dsl quantity in                 &  8 \\

                35 &      isdn quantity out channel 1     &  8 \\
                36 &      isdn quantity out channel 2     &  8 \\
                37 &      dsl quantity out                &  8 \\

                38 &      cpu usage                       &  4 \\

                39 &      fixed text                  & max 20 \\
                40 &      text -$>$ /etc/lcd\_text1.txt  & max 20 \\
                41 &      text -$>$ /etc/lcd\_text2.txt  & max 20 \\
                42 &      text -$>$ /etc/lcd\_text3.txt  & max 20 \\
                43 &      text -$>$ /etc/lcd\_text4.txt  & max 20 \\

                44 &      router uptime                   & 10 \\

            \end{tabular}
            \caption{aperçu des valeurs pour la variable LCD\_TYPE\_x}
            \marklabel{tab:lcd-type-x}{}
          \end{center}
          \end{small}
        \end{table}




        Les deux chiffres suivants, indiqués dans la variable \var{LCD\_\-TYPE\_\-1}
        donnent la position du message. Au format~: "colonne ligne", les deux
        chiffres commencent à 0 (0 compte pour 1).

      Exemple~:

\begin{example}
\begin{verbatim}
        LCD_TYPE_1='4 10 1'    # Message, 2ème ligne, à la 11ème colonne
                    | |  |
                    | |  \--    Ligne sur l'écran
                    | \-----    Colonne sur l'écran
                    \-------    Type de message selon tableau
\end{verbatim}
\end{example}


        Pour le Type 39 (Fixed texte) si vous voulez encore rajouter du texte,
        le format de celui-ci sera affiché sur l'écran LCD.

      Exemple~:

\begin{example}
\begin{verbatim}
        LCD_TYPE_2='39 10 1 Hallo'    # Texte "Hallo" sur la 2ème ligne
                                      # à la 11ème colonne
\end{verbatim}
\end{example}


      Les types 40 - 43 affichent le texte, à partir des fichiers types enregistrés
      dans le répertoire. Les fichiers lu et affichés, sont générés toutes les
      secondes. vous pouvez les changer par d'autre programme (par exemple telmond).
      On peut afficher de nouveaux courriels sur l'écran même si l'on est en mode
      off-line (MyJack). Les fichiers textes, définis dans la variable précédant
      pour les types de données 40 à 43, (ces fichiers lors de la mise en marche
      du système seront sauvegardés dans un dossier temporaire, le nom du fichier
      sera toujours défini simplement en ajoutant un index dans la chaîne de
      caractères "/etc/lcd\_text$<$Zahl$>$.txt")~:

\begin{example}
\begin{verbatim}
        LCD_VAR_TEXT1='Text 1'  # -> /etc/lcd_text1.txt
        LCD_VAR_TEXT2='Text 2'  # -> /etc/lcd_text2.txt
        LCD_VAR_TEXT3='Text 3'  # -> /etc/lcd_text3.txt
        LCD_VAR_TEXT4='Text 4'  # -> /etc/lcd_text4.txt
\end{verbatim}
\end{example}

    Depuis la version 1.6.2, il est possible d'afficher des textes différents en
    fonction du statut on-line off-line. Par exemple, on peut afficher un texte
    défini pendant le temps on-line et afficher la date et l'heure lorsque la
    connexion passe en off-line, les caractères seront affichés à la même place
    sur l'écran. Pour cela les variables suivantes seront ajoutées~:

\config{LCD\_TYPE\_ONLINE\_N}{LCD\_TYPE\_ONLINE\_N}{LCDTYPEONLINEN}

     Dans la variable \var{LCD\_\-TYPE\_\-ONLINE\_\-N} vous indiquez le nombre
     de types de données à afficher. Ces types de données seront seulement
     affichés si l'on est on-line.


\config{LCD\_TYPE\_ONLINE\_x}{LCD\_TYPE\_ONLINE\_x}{LCDTYPEONLINEx}

      Dans la variable \var{LCD\_\-TYPE\_\-ONLINE\_\-x} vous indiquez le type
      de données, dans la colonne et la ligne, où seront affichage les informations
      souhaitées sur l'écran LCD. Le type de données sont numérotés. Voir le
      format des types correspondant dans le tableau 'LCD\_TEXT\_x' ci-dessus.

      Exemple:
\begin{example}
\begin{verbatim}
        LCD_TYPE_ONLINE_1='8 0 0'     # dsl status
\end{verbatim}
\end{example}


\config{LCD\_TYPE\_OFFLINE\_N}{LCD\_TYPE\_OFFLINE\_N}{LCDTYPEOFFLINEN}

      Dans la variable \var{LCD\_\-TYPE\_\-OFFLINE\_\-N} vous indiquez le nombre
      de types de données à afficher. Ces types de données seront seulement
      affichés si l'on est off-line.


\config{LCD\_TYPE\_OFFLINE\_x}{LCD\_TYPE\_OFFLINE\_x}{LCDTYPEOFFLINEx}

        Dans la variable \var{LCD\_\-TYPE\_\-OFFLINE\_\-x} vous indiquez le
        type de données, dans la colonne et la ligne, où seront affichage les
        informations souhaitées sur l'écran LCD. Le type de données sont
        numérotés. Voir le format des types correspondant dans le tableau
        'LCD\_TEXT\_x' ci-dessus.

        Exemple:
\begin{example}
\begin{verbatim}
        LCD_TYPE_OFFLINE_1='0 0 0'    # local date
\end{verbatim}
\end{example}

\configlabel{LCD\_DSL\_SPEED\_OUT}{LCDDSLSPEEDOUT}
\config{LCD\_DSL\_SPEED\_IN LCD\_DSL\_SPEED\_OUT}{LCD\_DSL\_SPEED\_IN}{LCDDSLSPEEDIN}

        Avec les variables \var{LCD\_\-DSL\_\-SPEED\_\-IN} et \var{LCD\_\-DSL\_\-SPEED\_\-OUT}

        vous pouvez avoir une échelle de barre d'affichage pour les (type 14 et 15).
        Vous indiquez ici la vitesse de transfert maximum de la connexion DSL.
        En principe vous pouvez spécifier la valeur totale. Il est même logique de
        donner une valeur '+' haute par rapport à la valeur indiquée. Veuillez
        également noter que les taux réels sont généralement un peu plus élevés
        par rapport au débit recommandé par les fournisseurs, donc avec DSL1000,
        vous avez un taux de téléchargement (en arrivée) de 1024 kilobits/s.

        Exemple pour une connexion-DSL 1024/128 kilobit/s~:
\begin{example}
\begin{verbatim}
        LCD_DSL_SPEED_IN='1024' # Bitrate for DSL inbound
        LCD_DSL_SPEED_OUT='128' # Bitrate for DSL outbound
\end{verbatim}
\end{example}

        Pour une connexion ISDN ces valeurs ne sont pas importantes.

\end{description}

\subsection{isdn\_rate}

 Le programme, \flqq{}isdn\_rate\frqq{} est le véritable c\oe{}ur du paquetage LCD.
 on enregistre l'état des circuits dans le fichier de configuration et on défini
 les emplacements correspondants pour le type de donnée sur l'écran LCD. Le
 programme isdn\_rate est appelé comme indiqué ci-dessous~:

\begin{example}
\begin{verbatim}
 isdn_rate [-ip router-ip] [-port imond-port] [-telmond-port telmond-port]
           [-type hitachi|matrix-orbital|tty] [-config config filename]
\end{verbatim}
\end{example}

 Définition Des paramètres optionnels~: 

\begin{description}
\item[-ip router-ip]
        Avec \var{-ip} on détermine l'adresse-IP du routeur, pour que IMOND puisse
        envoyer les données. Si le paramètre est laissé vide, l'adresse 127.0.0.1
        (localhost) est utilisée par défaut. Il est possible de donner le nom du
        routeur à la place de l'adresse-IP.

\item[-port imond-port]
        Avec \var{-port} on détermine le port sur lequel IMOND envoie les données. 
        Configuration par défaut 5000.

\item[-telmond-port telmond-port]
        Avec \var{-telmond-port} on détermine le port sur lequel TELMOND envoie 
        les données. Configuration par défaut 5001.

\item[-type hitachi|matrix-orbital|tty]
        Avec \var{-type} on règle le type d'écran LCD.
        Sélectionnez \var{hitachi} pour les écrans compatibles HD44780.
        Sélectionnez \var{matrix-orbital} pour les écrans matrix-orbital.
        Sélectionnez \var{tty} pour permettre de sortir sur la console du routeur fli4l.
        IMPORTANT~: avec le programme isdn\_rate on doit toujours indiquer une sortie.
        Il est donc nécessaire de paramétrer une interface de sortie. tty est la
        configuration par défaut.

\item[-config configfilename]
        Avec \var{-config} on détermine le chemin et le nom du fichier lcd.conf.
        Ce fichier est utilisé par le script rc410.lcd. La valeur par défaut
        est /var/run/lcd.conf.

Il y a aussi une version isdn\_rate qui fonctionne sous Windows. A cet égard,
le fichier /var/run/lcd.conf doit être copié après le démarrage du routeur dans
le répertoire-isdn\_rate ou doit être créé à la main.L'appel du programme pourrait
alors ressembler à ceci~:

\begin{example}
\begin{verbatim}
 isdn_rate -ip fli4l -config lcd.conf
\end{verbatim}
\end{example}
               
\end{description}

\subsection{Branchement d'un module-LCD sur le port Parallèle}
 
\begin{example}
\begin{verbatim}
   13 _____________________________ 1  Vue du port
      \ o o o o o o o o o o o o o /    Parallèle, au 
       \ o o o o o o o o o o o o /     dos du PC
     25 ------------------------- 14  

\end{verbatim}
\end{example}


 Le raccordement d'un écran LCD au routeur se fait de la manière suivante~:


\begin{example}
\begin{verbatim}

 Broches-port-parallèle   Description    Module-LCD   Broche-LCD
         18-25            GND                             --|
                          GND                          1  --|- pont
                          R/W                          5  --|
                          +5V                          2
             1            STROBE         EN(1)         6
             2            D0             D0            7
             3            D1             D1            8
             4            D2             D2            9
             5            D3             D3           10
             6            D4             D4           11
             7            D5             D5           12
             8            D6             D6           13
             9            D7             D7           14
            14            Autofeed       RS            4
            17            Select In      EN(2)         ? (Pour LCDs avec 2 contrôleurs)

  En cas d'affichage rétro-éclairé :
                                   HG+          15 (avec une valeur environ 20 Ohm)
                                   GND          16
\end{verbatim}
\end{example}

  La broche 3 peut être branchée sur un potentiomètre de $>$= 20 kOhm entre +5V
  et GND (masse). Ainsi, vous pouvez régler le contraste de l'écran LCD. Sur
  mon écran LCD (Conrad), la broche 3 se trouve directement reliée à la masse,
  cela fonctionne très bien.

\begin{example}
\begin{verbatim}

  +5V ---+
         /
         \ <--+
         /    |
         \    |
  GND ---+    +--- VL (Broche 3 - driver input)
\end{verbatim}
\end{example}



\subsection{Branchement d'un écran 4x40}

  Le branchement d'un écran LCD 4x40 se différencie par rapport aux autres
  écrans LCD, ici un exemple du (NLC-40x4x05 - de Conrad)~:

\begin{example}
\begin{verbatim}

Broches-port-parallèle   Description    Module-LCD         Broche-LCD
        18-25                                                   --|
                         GND                                13  --|- pont
                         R/W                                10  --|
                         +5V                                14
            1            STROBE         EU (Enable-Upper)    9
            2            D0             D0                   8
            3            D1             D1                   7
            4            D2             D2                   6
            5            D3             D3                   5
            6            D4             D4                   4
            7            D5             D5                   3
            8            D6             D6                   2
            9            D7             D7                   1
           14            Autofeed       RS                  11
           17            Select In      ED (Enable-Down)    15
\end{verbatim}
\end{example}

  La broche 12 peut être branchée sur un potentiomètre de $>$= 20 kOhm entre
  +5V et GND (masse). Ainsi, vous pouvez régler le contraste de l'écran LCD.
  Cependant, vous pouvez toujours brancher la broche 12 directement à la masse
  cela fonctionne très bien.

\begin{example}
\begin{verbatim}

  +5V ---+
         /
         \ <--+
         /    |
         \    |
  GND ---+    +--- VL (Broche 12 - driver input)
\end{verbatim}
\end{example}


  \begin{itemize}
  \item Le circuit ED est connecté à la broche 17 du port parallèle.

  \item L'écran est défini dans le fichier lcd.txt comme un écran 2x40.

  \item Pour définir le type dans isdn\_rate on garde quand même 4x40 comme
        valeur de lignes et de colonnes.
  \end{itemize}


  Pour l'utilisation interne des modules LCD. Il n'y a malheureusement,
  aucun schéma standard qui correspond à la connexion du port parallèle des
  cartes-mères. Il faut donc contrôler les broches du connecteur qui est
  livré avec la carte-mère.

  Une alimentation électrique est nécessaire, on ne peut malheureusement pas
  utiliser l'alimentation du port parallèle, étant donné que la consommation
  électrique d'un module LCD est trop élevée. Les connecteurs suivants seront
  plus adaptés, la souris (PS/2), le clavier (DIN, PS/2), le port jeux ou un
  connecteur libre de l'alimentation du PC. Certains fabricants de cartes
  son génèrent les signaux spéciaux sur port jeux. Je n'ai aucune garantie
  que chacune les distributions fonctionnent. Alors, il est important de~:
  Toujours mesurer la tension avant le branchement~!

\subsection{Câblage Winamp pour les Modules LCD}

  Par rapport à la conception du Kernel il existe plusieurs variantes de câblages
  qui équipe des écrans LCD, le câblage normale et le câblage de Winamp. les
  nouveaux écrans Winamp sont câblés de cette manière.

\begin{example}
\begin{verbatim}
   13 _____________________________ 1 Vue du port
      \ o o o o o o o o o o o o o /   Parallèle, au
       \ o o o o o o o o o o o o /    dos du PC
     25 ------------------------- 14
\end{verbatim}
\end{example}

 Raccordement de la manière suivante d'un module LCD au routeur avec
 le câblage Winamp~:

\begin{example}
\begin{verbatim}

Broches-port-parallèle   Description   Module-LCD    Broche-LCD
         18-25           GND                          1
            14           Autofeed       R/W           5
                         +5V                          2
             1           STROBE         EN(1)         6
             2           D0             D0            7
             3           D1             D1            8
             4           D2             D2            9
             5           D3             D3           10
             6           D4             D4           11
             7           D5             D5           12
             8           D6             D6           13
             9           D7             D7           14
            16           Init           RS            4

  En cas d'affichage rétro-éclairé :
                    +5V            HG+          15
                                   GND          16 (avec une résistance de 100 Ohm)
\end{verbatim}
\end{example}

  La broche 3 peut être branchée sur un potentiomètre de $>$= 10 kOhm entre +5V
  et GND (masse). Ainsi, vous pouvez régler le contraste de l'écran LCD.

\begin{example}
\begin{verbatim}

  +5V ---+
         /
         \ <--+
         /    |
         \    |
  GND ---+    +--- VL (Broche 3 - driver input)
\end{verbatim}
\end{example}


\subsection{Trucs et astuce - Résumé avec la collaboration de Robert Resch}

\begin{description}
\item [Raccordement de 2 écrans]

  A l'aide de 2 signaux EN, il est possible d'exploiter en même temps, 2
  écrans en parallèle. La broche 6 de l'écran est connectée à la broche 1
  (EN1) du port Parallèle et la broche 6 du 2ème écran est connectée à
  la broche 17 (EN2). Toutes les autres broches sont connectées en
  parallèle sur les 2 écrans.


 \item  [Deux pages d'affichage sur un écran]

  Maintenant, il est aussi possible d'afficher 2 pages sur un écran
  avec le circuit suivant~:
\begin{example}
\begin{verbatim}

   25-pol. Sub-D        LCD

   1 -------|
            |
             \
              \-------- Pin 6
               
            |
   17-------|

          Commutateur
\end{verbatim}
\end{example}

  On raccorde le commutateur à la broche 6 de l'écran. Les deux autres
  contacts sont raccordés sur les deux signaux-EN broche 1 et 17 du
  port parallèle.

 \item   [Gestion des deux variantes]

  \sloppypar{Le signal-EN2 est généré dès qu'une ligne z dans
  \var{LCD\_LINES} $< z < 2*$\var{LCD\_LINES} est demandé. Si une plus
  grande ligne de caractère est utilisée, les deux écrans sont alimentés
  (pour pouvoir afficher les caractères bien définis par exemple les barres
  de débit avec isdn\_rate sur les deux écrans). Avec les deux pages sur le
  même écran, ils peuvent avoir leurs propres caractères bien définis.}

  Avec l'écran 4x40 on configure dans le fichier $<$config$>$/lcd.txt comme
  un écran 2x40. Les lignes sont cependant représentées par 0-3. Les numéros
  de ligne 4 et 5 sont dans ce cas affichées sur les deux moitiés d'écran.
  La ligne 4 va à la ligne 0 et 2, ligne 5, à la ligne 1 et 3.

  Par exemple avec 2 écrans 4x20 commutés en parallèle la gestion se présente
  de la manière suivante~:

  \begin{itemize}
  \item Les lignes de 0 à 3 sont affichées sur l'écran 1

  \item Les lignes de 4 à 7 sont affichées sur l'écran 2

  \item Les lignes de 8 à 11 sont affichées sur les deux écrans
  \end{itemize}


 \item [A propos de l'oscillation]

  Si vous avez des problèmes d'oscillation du signal, cela vient peut-être
  d'un câble trop long ou du port parallèle. Raccourcissent le câble, si
  cela ne va toujours pas, vous devez créer une terminaison de ligne. Pour
  ce faire, on place une résistance de 10 kOhm sur la ligne de données
  (raccord 10!) et le +5V. Cela devrait stabiliser l'oscillation.

\end{description}


\subsection{Remerciement}

  Merci à~:

  \begin{itemize}
  \item Nils Färber (\email{nils@kernelconcepts.de}) pour le pilote

  \item Jürgen Bauer (\email{jb@idowa.net}) pour la première version de isdn\_rate

  \item Frank Meyer (\email{frank@fli4l.de}) pour l'interface imond et Fli4l~:)

  \item Ulf Lanz (\email{u.lanz@odn.de}) pour la distribution des variables isdn\_rate

  \item Robert Resch (\email{rresch@gmx.de}) pour la correction des bugs et
    des améliorations du pilote-lcd

  \item Stefan Krister (\email{Stefan.Krister@keimfarben.de}) pour le premier
    arrangement de la documentation

   \item Nicole Hornung (\email{fli4l@xerotech.de}) pour quelques améliorations
     de la Documentation 

   \item Gerrit Lammert (\email{gerrit@fli4l.de}) pour la transformation de la
     Doc-texte en Doc-HTML
  \end{itemize}


    Pour toutes questions, suggestions, critique, etc.~:

    Envoyer un courriel à Gernot Miksch \email{ibgm@gmx.de}

