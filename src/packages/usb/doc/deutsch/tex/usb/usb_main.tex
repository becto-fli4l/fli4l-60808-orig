% Last Update: $Id$
\marklabel{sec:opt-usb }
{
\section {USB - Support für USB-Geräte}
}
\begin{description}

\config{OPT\_USB}{OPT\_USB}{OPTUSB}

        Hier wird die grundsätzliche Unterstützung von USB-Geräten ein-
        beziehungsweise ausgeschaltet. Erst wenn hier 'yes' eingetragen
        wird, können USB-Geräte überhaupt verwendet werden. Sollten Sie
        also in der base.txt ein USB-Gerät ausgewählt haben, so müssen Sie
        hier zwingend 'yes' eintragen. Andernfalls wird das Gerät nicht
        verwendet. Mit der Aktivierung ist auch der Support für USB Sticks,
        externen Laufwerken und Tastaturen eingeschaltet.

        Standard-Einstellung:  \var{OPT\_USB}='no'

\config{USB\_EXTRA\_DRIVER\_N}{USB\_EXTRA\_DRIVER\_N}{USBEXTRADRIVERN}

        Anzahl der zusätzlich zu ladenden Treiber. 

        Standard-Einstellung: \var{USB\_EXTRA\_DRIVER\_N}='0'

\config{USB\_EXTRA\_DRIVER\_x}{USB\_EXTRA\_DRIVER\_x}{USBEXTRADRIVERx}

        Treiber der geladen werden soll.

        Mögliche Werte im Moment
        \begin{itemize}
	    \item printer - Unterstützung für USB-Drucker
	    \item belkin\_sa - USB Belkin Serial converter
	    \item cyberjack - REINER SCT cyberJack pinpad/e-com USB Chipcard Reader
	    \item digi\_acceleport - Digi AccelePort USB-2/USB-4 Serial Converter
	    \item empeg - USB Empeg Mark I/II
	    \item ftdi\_sio - USB FTDI Serial Converter
	    \item io\_edgeport - Edgeport USB Serial
	    \item io\_ti - Edgeport USB Serial
	    \item ipaq - USB PocketPC PDA
	    \item ir-usb - USB IR Dongle
	    \item keyspan - Keyspan USB to Serial Converter
	    \item keyspan\_pda - USB Keyspan PDA Converter
	    \item kl5kusb105 - KLSI KL5KUSB105 chipset USB->Serial Converter
	    \item kobil\_sct - KOBIL USB Smart Card Terminal (experimental)
	    \item mct\_u232 - Magic Control Technology USB-RS232 converter
	    \item omninet - USB ZyXEL omni.net LCD PLUS
	    \item pl2303 - Prolific PL2303 USB to serial adaptor
	    \item visor - USB HandSpring Visor / Palm OS
	    \item whiteheat - USB ConnectTech WhiteHEAT
        \end{itemize}

        Standard-Einstellung: \var{USB\_EXTRA\_DRIVER\_x}=''

\config{USB\_EXTRA\_DRIVER\_x\_PARAM}{USB\_EXTRA\_DRIVER\_x\_PARAM}{USBEXTRADRIVERxPARAM}

        Parameter für den zusätzlichen Treiber. Im Normalfall muss hier
        nichts eingegeben werden.

        Standard-Einstellung: \var{USB\_EXTRA\_DRIVER\_x\_PARAM}=''

\config{USB\_MODEM\_WAITSECONDS}{USB\_MODEM\_WAITSECONDS}{USBMODEMWAITSECONDS}

        Standard-Einstellung: \var{USB\_MODEM\_WAITSECONDS}='21'

        Leider braucht das Speedtouch USB Modem eine halbe 
        Ewigkeit bis es bereit ist. In den meisten Fällen reichen 
        die 21~Sekunden, die als Standardeinstellung genommen werden, 
        für die Initialisierung aus. Manchmal hat man
        das Glück das man den Wert auch halbieren kann und das
        Speedtouch USB Modem bereits nach 10~Sekunden einsatzbereit ist, dann
        kann man hier halt 10~Sekunden eintragen. Wenn man Pech hat
        muss man den Wert erhöhen. Hier hilft leider nur probieren und
        austesten.

\end{description}

\subsection{Probleme mit USB-Geräten}

Es kann bei einigen USB-Geräten zu Problemen kommen. Das kann verschiedene
Ursachen haben, wie Beispielsweise der Treiber-Software oder dem
USB-Controller.

\subsection{Hinweise zur Benutzung}

Es ist darauf zu achten, dass die USB-Unterstützung hardwareseitig aktiviert 
ist. Insbesondere bei Onboard-USB-Kontrollern ist das wichtig. So wird z. B. 
ein WRAP ohne USB-Anschluss ausgeliefert. USB kann hier durch ein Zusatzmodul 
nachgerüstet werden und ist aus diesem Grund im BIOS standardmäßig deaktiviert. 


\subsection{Mounten von USB-Geräten}

Eingesteckte USB-Geräte werden zwar automatisch erkannt, müssen aber 'von Hand' 
sowohl an- als auch abgemeldet werden. Beim Einstecken z. B. eines USB-Stick 
wird dieser als SCSI-Device erkannt. Aus diesem Grund erfolgt der Zugriff über 
das Device sd\verb=#= bei SuperFloppy-Geräten bzw. über sd\verb=#=$<$Partitionsnummer$>$ bei
Geräten mit einer Partitionstabelle. USB-Sticks werden wie Festplatten 
behandelt, also bei zwei USB-Anschlüssen sda1 und sdb1 angesprochen. 
USB-Floppies hingegen werden durch sda bzw. sdb angesprochen, also ohne Angabe
einer Partitionsnummer.

Somit kann ein USB-Stick durch das Kommando

mount /dev/sda1 /mnt

nach /mnt gemountet werden. Analog dazu durch

mount /dev/sdb1 /mnt

für das zweite USB-Gerät. Die Geräte werden in der Reihenfolge des Einsteckens
benannt, also erstes USB-Gerät = sda, zweites USB-Gerät = sdb etc. pp. Es lässt 
sich somit nicht fix definieren, welcher der USB-Ports welche 'Bezeichnung' 
hat, da diese von der Reihenfolge des Einsteckens der Geräte abhängt. 
Die Abmeldung der angemeldeten USB-Geräte erfolgt durch

umount /mnt

Bei gleichzeitiger Verwendung mehrerer USB-Geräte sollte es unbedingt vermieden
werden, alles in ein Ziel zu mounten. Aus diesem Grund bietet es sich an, 
unterhalb von /mnt weitere Verzeichnisse anzulegen, in welche die Geräte dann
gemountet werden können. Dies kann z. B. wie folgt erledigt werden:

mkdir /mnt/usba
mkdir /mnt/usbb

Beim Mounten der Geräte werden dann diese Verzeichnisse als Ziel angegeben:

mount /dev/sda1 /mnt/usba
mount /dev/sdb1 /mnt/usbb

Somit ist der Inhalt der USB-Geräte unter /mnt/usba bzw. /mnt/usbb zu finden.
Die Abmeldung erfolgt dann durch

umount /mnt/usba
umount /mnt/usbb

Wenn mehrere Partitionen je USB-Gerät existieren, müssen die Verzeichnisse 
unterhalb von /mnt entsprechend strukturiert werden.
