% Synchronized to r46491
\marklabel{sec:opt-usb }
{
\section {USB - Support For USB Devices}
}
\begin{description}

\config{OPT\_USB}{OPT\_USB}{OPTUSB}

        Supprt for USB devices in general is activated here. Only if 'yes' 
        is entered here USB devices will get usable at all. If you have 
        chosen an USB device in base.txt entering 'yes' here is mandatory. 
        Otherwise the device can't be used. This activates support for USB sticks,
        external USB drives and USB keyboards as well.

        Default setting:  \var{OPT\_USB}='no'

\config{USB\_EXTRA\_DRIVER\_N}{USB\_EXTRA\_DRIVER\_N}{USBEXTRADRIVERN}

        Number of drivers to load. 

        Default setting: \var{USB\_EXTRA\_DRIVER\_N}='0'

\config{USB\_EXTRA\_DRIVER\_x}{USB\_EXTRA\_DRIVER\_x}{USBEXTRADRIVERx}

        Driver that should be loaded.

        Possible values at the moment
        \begin{itemize}
	    \item printer - support for USB printers
	    \item belkin\_sa - USB Belkin Serial converter
	    \item cyberjack - REINER SCT cyberJack pinpad/e-com USB Chipcard Reader
	    \item digi\_acceleport - Digi AccelePort USB-2/USB-4 Serial Converter
	    \item empeg - USB Empeg Mark I/II
	    \item ftdi\_sio - USB FTDI Serial Converter
	    \item io\_edgeport - Edgeport USB Serial
	    \item io\_ti - Edgeport USB Serial
	    \item ipaq - USB PocketPC PDA
	    \item ir-usb - USB IR Dongle
	    \item keyspan - Keyspan USB to Serial Converter
	    \item keyspan\_pda - USB Keyspan PDA Converter
	    \item kl5kusb105 - KLSI KL5KUSB105 chipset USB->Serial Converter
	    \item kobil\_sct - KOBIL USB Smart Card Terminal (experimental)
	    \item mct\_u232 - Magic Control Technology USB-RS232 converter
	    \item omninet - USB ZyXEL omni.net LCD PLUS
	    \item pl2303 - Prolific PL2303 USB to serial adaptor
	    \item visor - USB HandSpring Visor / Palm OS
	    \item whiteheat - USB ConnectTech WhiteHEAT
        \end{itemize}

        Default setting: \var{USB\_EXTRA\_DRIVER\_x}=''

\config{USB\_EXTRA\_DRIVER\_x\_PARAM}{USB\_EXTRA\_DRIVER\_x\_PARAM}{USBEXTRADRIVERxPARAM}

        Parameters for the driver. Usually you won't need this.

        Default setting: \var{USB\_EXTRA\_DRIVER\_x\_PARAM}=''

\config{USB\_MODEM\_WAITSECONDS}{USB\_MODEM\_WAITSECONDS}{USBMODEMWAITSECONDS}

        Default setting: \var{USB\_MODEM\_WAITSECONDS}='21'

        Unfortunately Speedtouch USB Modems need a long time to get 
        ready. In most cases the default setting of about 21 seconds is sufficient 
        for initialisation. Sometimes the value can be halved and the
        Speedtouch USB modem is ready after 10 seconds. If this is the case you 
        could set the value to 10 seconds. If you are unlucky the value has to 
        be raised. Only testing can help you here.

\end{description}

\subsection{Problems With USB Devices}

There may be problems with some USB devices. This can have different
causes, for example the driver software or the USB controller.

\subsection{Hints For Use}

It is important to ensure that the hardware USB support is enabled. Especially 
with onboard USB controllers that is important. For example a WRAP is shipped 
without USB port. USB is added only by an additional module and is therefore 
disabled in the BIOS by default.

\subsection{Mounting Of USB Devices}

Plugged USB devices will be detected automatically but must be mounted and 
unmounted 'by hand'. When plugging an USB stick it is recognized as a SCSI 
block device. For this reason is accomplished via device sd\verb=#= for 
SuperFloppy devices or sd\verb=#=$<$Partition-number$>$ for devices with 
a partition table. USB drives are treated as hard disks and addressed as 
sda1 and sdb1 if plugged in two USB ports. USB floppies are addressed via 
sda or sdb without specifying a partition number.

Thus a first USB stick can be mounted to /mnt by executing\\

mount /dev/sda1 /mnt\\

A second one at the same would need \\

mount /dev/sdb1 /mnt\\

to be mounted. Devices are numbered in the sequence of plugging - first USB 
device = sda, second USB device = sdb a.s.o. This means that device numbering 
is not fixed but depending on the sequence of plugin. Unmounting is done via \\

umount /mnt\\

If using more that one USB device never mount all devices to the same directoy. 
Create directories under /mnt for each device to be mounted. Make the directories 
by executing:\\

mkdir /mnt/usba\\
mkdir /mnt/usbb\\

Then specify the directories as destinations when mounting:\\

mount /dev/sda1 /mnt/usba\\
mount /dev/sdb1 /mnt/usbb\\

The content of the USB devices can be found at /mnt/usba res. /mnt/usbb.
Unmounting can be done via \\

umount /mnt/usba\\
umount /mnt/usbb\\

If an USB device has more partitions the device directories under /mnt have 
to reflect this structure in subdirectories.
