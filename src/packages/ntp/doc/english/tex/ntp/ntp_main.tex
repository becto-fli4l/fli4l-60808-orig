% Synchronized to r39854
\marklabel{sec:opt-ntp}
{
\section {NTP - Network Time Protocol Server/Client}
}

OPT\_NTP provides the Network Time Protocol (NTP) for fli4l.
This is not to be confused with the \emph{normal} Time Protocol.
The protocols are not compatible and thus new client programs
understanding the NTP are required.

OPT\_NTP works in both server and client mode. In client mode it
synchronizes fli4l's system time with Internet time references (Time
Servers) or uses the time information of a connected Radio clock.

In server mode OPT\_NTP acts as a time server for the local
net (LAN). NTP uses port 123.

\marklabel{sec:konfigntp}{
\subsection {Configuration Of The Package NTP}
}

Like with all fli4l packages configuration is done by adapting the file\\
\var{Path/fli4l-\version/$<$config$>$/ntp.txt} to your needs.

\begin{description}

\config {OPT\_NTP}{OPT\_NTP}{OPTNTP}

  {Default: \var{OPT\_NTP='no'}

  Setting this to \var{'no'} completely deactivates OPT\_NTP. No changes
  to the fli4l boot medium resp. archive \var{opt.img} are made.
  OPT\_NTP does not overwrite any other part of the fli4l installation.
  To activate OPT\_NTP set \var{OPT\_NTP} to \var{'yes'}.}

\config {NTP\_REFCLOCK\_TYPE}{NTP\_REFCLOCK\_TYPE}{NTPREFCLOCKTYPE}

  This variable determines the type of a locally connected reference clock.
  Possible values are listed in Tab.~\ref{table:ntp:refclocks}. If no reference clock
  is available and hence system time is only synchronized by other network time servers
  specify ``none'' here.

  \begin{table}[ht!]
    \centering
    \small
    \caption{Supported Reference Clocks}
    \label{table:ntp:refclocks}
    \begin{tabular}{|l|p{7.5cm}|}
      \hline
      Shortcut & Model \\
      \hline
      none              & no Reference Clock \\
      mouseclock-nts    & mouseCLOCK NTS \\
      mouseclock-usb-ii & mouseCLOCK USB v2.0\\
      sure              & Sure RPC DCF77 \\
      neoclock4x        & NeoClock4X DCF77 \\
      hopf-seriell      & hopf DCF77/GPS (serial) \\
      \hline
    \end{tabular}
  \end{table}

\config {NTP\_REFCLOCK\_DEVICE}{NTP\_REFCLOCK\_DEVICE}{NTPREFCLOCKDEVICE}

  This variable specifies the interface to which the reference clock is connected.
  Typically, this is \texttt{/dev/ttySx} for the (x+1)-th RS232-Interface or
  \texttt{/dev/ttyUSBx} for the (x+1)-th USB-Interface, i.e. \texttt{/dev/ttyS0} or
  \texttt{/dev/ttyUSB1}.

\config {NTP\_SERVER\_N}{NTP\_SERVER\_N}{NTPSERVERN}

  This variable specifies the the number of servers for time synchronisation.

\config {NTP\_SERVER\_x\_HOST}{NTP\_SERVER\_x\_HOST}{NTPSERVERxHOST}

  This variable contains the IP address or the FQDN of the server to be used
  for time synchronisation.

\config {NTP\_SERVER\_x\_TYPE}{NTP\_SERVER\_x\_TYPE}{NTPSERVERxTYPE}

  This variable controls the type of relationship between fli4l and the
  time Server. When setting the value to \var{'peer'}, the time is synchronized
  with the defined server. This is used if more local NTP servers exist
  (usually using time sources with similar Stratum) and that should be
  synchronized with each other. Setting this value to \var{'server'}
  specifies the defined time server as the source for the router's NTP server.
  This is for using external time servers (which usually have time sources with
  a higher stratum).

\config {NTP\_SERVER\_x\_BURST}{NTP\_SERVER\_x\_BURST}{NTPSERVERxBURST}

  This variable is optional.\\
  \var{'yes'} speeds up the synchronization with the defined NTP Server
  by using eight NTP adjustment packages.
\config {NTP\_SERVER\_x\_IBURST}{NTP\_SERVER\_x\_IBURST}{NTPSERVERxIBURST}

  This variable is optional.\\
  \var{'yes'} speeds up the synchronization with the defined NTP Server
  by using 16 NTP adjustment packages.

\config {NTP\_SERVER\_x\_PREFER}{NTP\_SERVER\_x\_PREFER}{NTPSERVERxPREFER}

  This variable is optional.\\
  With \var{'yes'} this NTP server is preferred over other NTP servers with
  the same Stratum.

\config {NTP\_SERVER\_x\_MINPOLL}{NTP\_SERVER\_x\_MINPOLL}{NTPSERVERxMINPOLL}

  This variable is optional.\\
  This variable defines the minimum time interval between
  NTP time synchronization packets. Values between 4 (15~s) and 6 (64~s)
  are allowed.
\config {NTP\_SERVER\_x\_MAXPOLL}{NTP\_SERVER\_x\_MAXPOLL}{NTPSERVERxMAXPOLL}

  This variable is optional.\\
  This variable defines the maximum time interval between
  NTP time synchronization packets. Values between 10 (1024~s) and 17
  (131 702~s = around 36.4~h) are allowed.
\config {NTP\_SERVER\_x\_VERSION}{NTP\_SERVER\_x\_VERSION}{NTPSERVERxVERSION}

  This variable is optional.\\
  This variable specifies the NTP packet's version number to be used. Values
  between 1 and 4 are allowed.

\config {NTP\_LOCAL\_RTC}{NTP\_LOCAL\_RTC}{NTPLOCALRTC}

  With \var{'yes'} the clock of the BIOS is also used as a time source.
  This enables the NTP daemon to continue to work as a time server even if
  none of the specified time servers can be reached or the connected
  radio clock does not provide a valid time anymore.

\config {NTP\_LOCAL\_RTC\_STRATUM}{NTP\_LOCAL\_RTC\_STRATUM}{NTPLOCALRTCSTRATUM}

  Defines the priority of the local clock within the NTP hierarchy.
  The higher the value the more inaccurate is the accuracy of the time source
  used. A server using the time of the reference clock has a stratum of
  \var{'1'}. The BIOS clock should normally have a value from \var{'10'}
  to \var{'12'} assigned.

\config {NTP\_ALLOW\_IPV4\_N}{NTP\_ALLOW\_IPV4\_N}{NTPALLOWIPV4N}

  This variable specifies the number of IPv4 networks which are allowed to
  access the router via NTP.

\config {NTP\_ALLOW\_IPV4\_x}{NTP\_ALLOW\_IPV4\_x}{NTPALLOWIPV4x}

  This variable holds a reference to an IPv4 network that is allowed to access the
  NTP server, for example, \var{IP\_NET\_1}.

\config {NTP\_ALLOW\_IPV4\_x\_PEERING}{NTP\_ALLOW\_IPV4\_x\_PEERING}{NTPALLOWIPV4xPEERING}

  This variable is optional.\\
  \var{'yes'} allows peering of fli4l with NTP servers in the referenced IPv4 network.

\config {NTP\_ALLOW\_IPV6\_N}{NTP\_ALLOW\_IPV6\_N}{NTPALLOWIPV6N}

  This variable specifies the number of IPv6 networks which are allowed to
  access the router via NTP.

\config {NTP\_ALLOW\_IPV6\_x}{NTP\_ALLOW\_IPV6\_x}{NTPALLOWIPV6x}

  This variable holds a reference to an IPv6 network that is allowed to access the
  NTP server, for example, \var{IPV6\_NET\_1}.

\config {NTP\_ALLOW\_IPV6\_x\_PEERING}{NTP\_ALLOW\_IPV6\_x\_PEERING}{NTPALLOWIPV6xPEERING}

  This variable is optional.\\
  \var{'yes'} allows peering of fli4l with NTP servers in the referenced IPv6 network.

\config {NTP\_CHECK\_STATUS}{NTP\_CHECK\_STATUS}{NTPCHECKSTATUS}

  Setting this variable to \var{'yes'} will display information on the time
  synchronization in the WebGUI in case of \var{OPT\_HTTPD='yes'}. With an activated
  \var{RRDTOOL\_NTP} graphs for the time drift are generated in addition.

\config {NTP\_SHOW\_STATUS\_VIA\_LED}{NTP\_SHOW\_STATUS\_VIA\_LED}{NTPSHOWSTATUSVIALED}

  Specifying \var{'yes'} here will indicate the state of NTP synchronisation
  on the defined LEDs given that the package ``hwsupp'' defines any LEDs.
  The following assignment applies:
  \begin{itemize}
  \item If LED 1 blinks no correct year is provided, else it is off.
  \item If LED 2 is on fli4l's time is in sync and is off if not and the NTP
	daemon is in the process of adjusting it in steps.
  \item If LED 3 is on the connected radio clock is working. It blinks if
        the radio clock does not provide a valid time but synchronisation
        can be accomplished with other NTP servers. It is off if the NTP
        server has no valid time source at all.
  \end{itemize}

\end{description}

\marklabel{sec:ntpsupport}{
\subsection{Support}
}
Yo will only get support within the \jump{url:ntpfli4lnews}{fli4l-Newsgroups}.

\marklabel{url:ntpfli4lnews}{
 fli4l Newsgroups and their rules: \altlink{http://www.fli4l.de/hilfe/newsgruppen-irc-forum/}
}
