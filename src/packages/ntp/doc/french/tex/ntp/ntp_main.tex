% Synchronized to r39854

\marklabel{sec:opt-ntp}
{
\section {NTP - Network Time Protocol serveur/client}
}

Le paquetage OPT\_NTP Network Time Protocol (NTP) étend les fonctionnalités
du routeur fli4l. Il ne doit pas être confondu avec le Protocol \emph{normal} Time.
Les deux protocoles ne sont pas compatibles, si vous avez un nouveau programme
client qui utilise le NTP vous pouvez l'utiliser.

Le paquetage OPT\_ntp fonctionne aussi bien en mode serveur qu'en mode client.
Le routeur fli4l met à l'heure l'ordinateur en fonction du client, par l'intermédiaire
du (serveur de temps) sur Internet ou en utilisant les informations d'une horloge
radio-pilotée sur le routeur.

La fonction du serveur et de l'OPT\_NTP sert à mettre à l'heure les ordinateurs
dans le réseau local (LAN). Le NTP fonctionne sur port 123.

\marklabel{sec:konfigntp}{
\subsection {Configuration du paquetage NTP}
}

La configuration est réalisée, comme tous les paquetage de fli4l, en adaptant
le fichier\\
\var{répertoire/fli4l-\version/$<$config$>$/ntp.txt} selon vos besoins.

\begin{description}

\config {OPT\_NTP}{OPT\_NTP}{OPTNTP}

  {Par défaut~: \var{OPT\_NTP='no'}

  Si vous indiquez \var{'no'} vous déactivez complètement l'OPT\_NTP. Il n'y aura
  aucune modification sur le démarrage du médias ou dans l'archive \var{opt.img}
  de fli4l. L'OPT\_NTP ne remplace jamais d'autres parties de l'installation de
  fli4l. Pour activer l'OPT\_NTP vous devez placer la variable \var{OPT\_NTP}
  sur \var{'yes'}.}

\config {NTP\_REFCLOCK\_TYPE}{NTP\_REFCLOCK\_TYPE}{NTPREFCLOCKTYPE}

  Dans cette variable vous indiquez la référence du type d'horloge radio-pilotée
  localement. Les valeurs autorisées sont répertorié dans le Tab.~\ref{table:ntp:refclocks}.
  Si aucune référence pour l'horloge n'est disponible et si vous voulez synchroniser
  l'horloge du système à l'aide d'un serveur de temps, vous devez indiquer "none" dans
  cette variable.

  \begin{table}[ht!]
    \centering
    \small
    \caption{Référence des horloges prise en charge}
    \label{table:ntp:refclocks}
    \begin{tabular}{|l|p{7.5cm}|}
      \hline
      Référence & Model \\
      \hline
      none              & aucune référence \\
      mouseclock-nts    & mouseCLOCK NTS \\
      mouseclock-usb-ii & mouseCLOCK USB v2.0\\
      sure              & Sure RPC DCF77 \\
      neoclock4x        & NeoClock4X DCF77 \\
      hopf-seriell      & hopf DCF77/GPS (seriell) \\
      \hline
    \end{tabular}
  \end{table}

\config {NTP\_REFCLOCK\_DEVICE}{NTP\_REFCLOCK\_DEVICE}{NTPREFCLOCKDEVICE}

  Dans cette variable vous indiquez l'interface à laquelle la référence de l'horloge
  est connecté. Précisément vous avez \texttt{/dev/ttySx} pour (x+1)-ème interface
  RS232 ou \texttt{/dev/ttyUSBx} pour (x+1)-ème interface USB. Par exemple,
  \texttt{/dev/ttyS0} ou \texttt{/dev/ttyUSB1}.

\config {NTP\_SERVER\_N}{NTP\_SERVER\_N}{NTPSERVERN}

  Dans cette variable vous indiquez le nombre de serveur de temps que vous voulez utiliser.

\config {NTP\_SERVER\_x\_HOST}{NTP\_SERVER\_x\_HOST}{NTPSERVERxHOST}

  Dans cette variable vous indiquez l'adresse IP ou le serveur FQDN qui sera utilisé
  pour la synchronisation de l'heure sur le LAN.

\config {NTP\_SERVER\_x\_TYPE}{NTP\_SERVER\_x\_TYPE}{NTPSERVERxTYPE}

  Cette variable contrôle le type de relation entre fli4l et le serveur de temps.
  En indiquant la valeur \var{'peer'}, l'heure sera aligné mutuellement au serveur
  défini. Ceci est utilisé lorsque vous avez plusieurs serveurs NTP locaux (habituellement
  la source de temps utilise un stratum similaire) ceux-ci, devraient être synchronisés
  entre eux. Si on met au contraire la valeur \var{'server'}, le serveur défini sera
  la source de temps pour le serveur NTP sur le routeur. Ceci est utilisé si vous
  (habituellement la source de temps utilise un stratum similaire) souhaitez vous connecter
  à un serveur de temps distant.

\config {NTP\_SERVER\_x\_BURST}{NTP\_SERVER\_x\_BURST}{NTPSERVERxBURST}

  La configuration de cette variable est optionnelle.\\
  Si vous indiquez \var{'yes'} la synchronisation avec le serveur NTP défini sera
  accéléré. Huit synchroniséations seront utilisées pour les paquets NTP.

\config {NTP\_SERVER\_x\_IBURST}{NTP\_SERVER\_x\_IBURST}{NTPSERVERxIBURST}

  La configuration de cette variable est optionnelle.\\
  Si vous indiquez \var{'yes'} la synchronisation avec le serveur NTP défini sera
  accéléré. Seize synchroniséations seront utilisées pour les paquets NTP.

\config {NTP\_SERVER\_x\_PREFER}{NTP\_SERVER\_x\_PREFER}{NTPSERVERxPREFER}

  La configuration de cette variable est optionnelle.\\
  Si vous indiquez \var{'yes'} Le serveur NTP sera préféré à d'autres serveurs NTP
  dans le même stratum.

\config {NTP\_SERVER\_x\_MINPOLL}{NTP\_SERVER\_x\_MINPOLL}{NTPSERVERxMINPOLL}

  La configuration de cette variable est optionnelle.\\
  Avec cette variable vous définissez l'intervalle minimum entre les paquets NTP
  pour la synchronisation du temps. Les valeurs autorisée sont comprises entre
  4 (15~s) et 6 (64~s).

\config {NTP\_SERVER\_x\_MAXPOLL}{NTP\_SERVER\_x\_MAXPOLL}{NTPSERVERxMAXPOLL}

  La configuration de cette variable est optionnelle.\\
  Avec cette variable vous définissez l'intervalle maximum entre les paquets NTP
  pour la synchronisation du temps. Les valeurs autorisée sont comprises entre
  10 (1.024~s) et 17 (131.702~s, ce qui équivaut environ à 36,4~h).

\config {NTP\_SERVER\_x\_VERSION}{NTP\_SERVER\_x\_VERSION}{NTPSERVERxVERSION}

  La configuration de cette variable est optionnelle.\\
  Avec cette variable vous indiquez le numéro de version qui est utilisé pour
  les paquets NTP. Les valeurs que vous pouvez indiquer sont de 1 à 4.

\config {NTP\_LOCAL\_RTC}{NTP\_LOCAL\_RTC}{NTPLOCALRTC}

  Si vous indiquez \var{'yes'} l'horloge du BIOS sera utilisé comme une source de
  temps. Cela permet au démon NTP de travailler comme un serveur de temps, si aucun
  des serveurs de temps spécifié peut être atteint ou si l'horloge radio-pilotée
  ne peux fournir une heure valide.

\config {NTP\_LOCAL\_RTC\_STRATUM}{NTP\_LOCAL\_RTC\_STRATUM}{NTPLOCALRTCSTRATUM}

  Dans cette variable vous indiquez la priorité de l'horloge locale à l'intérieur de
  la hiérarchie NTP. Plus la valeur est précise, plus la source de l'horloge utilisée
  est exact. Donc un serveur de temps utilise la référence de statum \var{'1'}. Pour
  l'horloge du BIOS, la valeur est normalement de \var{'10'} voir \var{'12'}.

\config {NTP\_ALLOW\_IPV4\_N}{NTP\_ALLOW\_IPV4\_N}{NTPALLOWIPV4N}

  Dans cette variable vous indiquez le nombre de réseaux IPv4 qui seront autorisés à accéder
  au routeur via le protocole NTP.

\config {NTP\_ALLOW\_IPV4\_x}{NTP\_ALLOW\_IPV4\_x}{NTPALLOWIPV4x}

  Dans cette variable vous indiquez le réseau IPv4 qui aura accès au serveur NTP,
  par exemple \var{IP\_NET\_1}.

\config {NTP\_ALLOW\_IPV4\_x\_PEERING}{NTP\_ALLOW\_IPV4\_x\_PEERING}{NTPALLOWIPV4xPEERING}

  La configuration de cette variable est optionnelle.\\
  Si vous indiquez \var{'yes'} la comparaison mutualisé avec les serveurs NTP et fli4l sera
  permise pour le réseau IPv4 référencé.

\config {NTP\_ALLOW\_IPV6\_N}{NTP\_ALLOW\_IPV6\_N}{NTPALLOWIPV6N}

  Dans cette variable vous indiquez le nombre de réseaux IPv6 qui seront autorisés à accéder
  au routeur via le protocole NTP.

\config {NTP\_ALLOW\_IPV6\_x}{NTP\_ALLOW\_IPV6\_x}{NTPALLOWIPV6x}

  Dans cette variable vous indiquez le réseau IPv6 qui aura accès au serveur NTP,
  par exemple \var{IPV6\_NET\_1}.

\config {NTP\_ALLOW\_IPV6\_x\_PEERING}{NTP\_ALLOW\_IPV6\_x\_PEERING}{NTPALLOWIPV6xPEERING}

  La configuration de cette variable est optionnelle.\\
  Si vous indiquez \var{'yes'} la comparaison mutualisé avec les serveurs NTP et fli4l sera
  permise pour le réseau IPv6 référencé.

\config {NTP\_CHECK\_STATUS}{NTP\_CHECK\_STATUS}{NTPCHECKSTATUS}

  En placant cette variable sur \var{'yes'}, et si le paquetage \var{OPT\_HTTPD='yes'} est
  activé les informations délivrées par la synchronisation de l'heure seront affichées dans
  le WebGUI. En outre avec le paquetage \var{RRDTOOL\_NTP} un graphique actif sera généré,
  pour les écarts de temps.

\config {NTP\_SHOW\_STATUS\_VIA\_LED}{NTP\_SHOW\_STATUS\_VIA\_LED}{NTPSHOWSTATUSVIALED}

  En placant cette variable sur \var{'yes'}, et si le paquetage "hwsupp" est activé le status
  de la synchronisation NTP sera affiché par des LEDs commandées. L'affectation des LEDs sera
  la suivante~:
  \begin{itemize}
  \item LED 1 clignote, tant que l'année n'est pas réglée correctement, ne peux pas
		être autrement.
  \item LED 2 éclairée, lorsque l'heure avec fli4l est synchrone et éteinte lorsque l'heure
		n'est pas synchronisée, mais le démon NTP l'adaptera progressivement.
  \item LED 3 éclairée quand l'horloge radio-pilotée fonctionne. Elle clignote lorsque
		l'horloge radio-pilotée fourni une heure valide et la synchronisation avec un autre
		serveur NTP peut être établie. Elle est exécutée si le serveur NTP n'a pas de source
		de temps.
  \end{itemize}

\end{description}

\marklabel{sec:ntpsupport}{
\subsection{Aide}
}
Un aide pour ce programme sera assurée uniquement dans les \jump{url:ntpfli4lnews}{Newsgroups fli4l}.

\marklabel{url:ntpfli4lnews}{
Les newsgroups fli4l et sont règlement~: \altlink{http://www.fli4l.de/fr/aide/newsgroup-forum/}
}

