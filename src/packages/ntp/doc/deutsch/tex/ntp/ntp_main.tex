% Last Update: $Id$
\marklabel{sec:opt-ntp}
{
\section {NTP - Network Time Protocol Server/Client}
}

OPT\_NTP erweitert fli4l um das Network Time Protocol (NTP). 
Dies ist nicht mit dem \emph{normalen} Time Protokoll zu
verwechseln. Die Protokolle sind nicht
kompatibel und somit werden gegebenenfalls neue Client-Programme, die NTP
beherrschen, benötigt.

OPT\_NTP arbeitet sowohl im  Server- als auch im Client Modus. In der
Funktion des Clients gleicht es die Zeit des fli4l mit Zeitreferenzen (Time
Server) im Internet ab oder nutzt die Zeitinformationen einer angeschlossenen
Funkuhr.

In der Funktion des Servers dient OPT\_NTP als Zeitreferenz für das lokale
Netzwerk (LAN). NTP arbeitet auf Port 123.

\marklabel{sec:konfigntp}{
\subsection {Konfiguration des Paketes NTP}
}

Die Konfiguration erfolgt, wie bei allen fli4l Paketen, durch Anpassung der Datei\\
\var{Pfad/fli4l-\version/$<$config$>$/ntp.txt} an die eigenen Anforderungen.

\begin{description}

\config {OPT\_NTP}{OPT\_NTP}{OPTNTP}

  {Default: \var{OPT\_NTP='no'}

  Die Einstellung \var{'no'} deaktiviert das OPT\_NTP vollständig. Es werden
  keine Änderungen an dem fli4l-Bootmedium bzw. dem Archiv \var{opt.img}
  vorgenommen. Weiter überschreibt das OPT\_NTP grundsätzlich keine anderen
  Teile der fli4l-Installation.
  Um OPT\_NTP zu aktivieren, ist die Variable \var{OPT\_NTP} auf 
  \var{'yes'} zu setzen.}

\config {NTP\_REFCLOCK\_TYPE}{NTP\_REFCLOCK\_TYPE}{NTPREFCLOCKTYPE}

  Mit dieser Variable wird die Typenbezeichnung der verwendeten lokal
  angeschlossenen Referenzuhr eingestellt. Die erlaubten Werte sind in
  Tab.~\ref{table:ntp:refclocks} aufgelistet. Ist keine Referenzuhr vorhanden
  und soll die Systemzeit somit allein mit Hilfe anderer Zeitserver im Netzwerk
  synchronisiert werden, ist ``none'' auszuwählen.

  \begin{table}[ht!]
    \centering
    \small
    \caption{Unterstützte Referenzuhren}
    \label{table:ntp:refclocks}
    \begin{tabular}{|l|p{7.5cm}|}
      \hline
      Kürzel & Modell \\
      \hline
      none              & keine Referenzuhr \\
      mouseclock-nts    & mouseCLOCK NTS \\
      mouseclock-usb-ii & mouseCLOCK USB v2.0\\
      sure              & Sure RPC DCF77 \\
      neoclock4x        & NeoClock4X DCF77 \\
      hopf-seriell      & hopf DCF77/GPS (seriell) \\
      \hline
    \end{tabular}
  \end{table}

\config {NTP\_REFCLOCK\_DEVICE}{NTP\_REFCLOCK\_DEVICE}{NTPREFCLOCKDEVICE}

  Diese Variable gibt die Schnittstelle an, an der die Referenzuhr
  angeschlossen ist. Typischerweise ist dies \texttt{/dev/ttySx} für die
  (x+1)-te RS232-Schnittstelle oder \texttt{/dev/ttyUSBx} für die (x+1)-te
  USB-Schnittstelle, beispielsweise \texttt{/dev/ttyS0} oder
  \texttt{/dev/ttyUSB1}.

\config {NTP\_SERVER\_N}{NTP\_SERVER\_N}{NTPSERVERN}

  Diese Variable spezifiziert die Anzahl der Server zum Zeitabgleich.

\config {NTP\_SERVER\_x\_HOST}{NTP\_SERVER\_x\_HOST}{NTPSERVERxHOST}

  Diese Variable nimmt die IP-Adresse oder den FQDN des Servers, der zum
  Zeitabgleich genutzt wird, auf.
  
\config {NTP\_SERVER\_x\_TYPE}{NTP\_SERVER\_x\_TYPE}{NTPSERVERxTYPE}

  Diese Variable kontrolliert die Art der Beziehung zwischen dem fli4l und dem
  Zeitserver. Setzt man den Wert auf \var{'peer'}, wird mit dem definierten
  Server gegenseitig die Zeit abgeglichen. Dies verwendet man, wenn man mehrere
  lokale NTP-Server hat (die in der Regel Zeitquellen mit ähnlichem
  Stratum nutzen) und diese sich untereinander abgleichen sollen. Setzt man den
  Wert hingegen auf \var{'server'}, ist der definierte Server Zeitquelle für
  den NTP-Server auf dem Router. Dies wird benutzt, wenn man fremde Zeitserver
  (die in der Regel Zeitquellen mit höherem Stratum nutzen) nutzen möchte.
 
\config {NTP\_SERVER\_x\_BURST}{NTP\_SERVER\_x\_BURST}{NTPSERVERxBURST}

  Diese Konfigurationvariable ist optional.\\
  Mit \var{'yes'} wird die Synchronisation mit dem definierten
  NTP-Server beschleunigt. Es werden acht NTP-Abgleichpakete verwendet.

\config {NTP\_SERVER\_x\_IBURST}{NTP\_SERVER\_x\_IBURST}{NTPSERVERxIBURST}

  Diese Konfigurationvariable ist optional.\\
  Mit \var{'yes'} wird die Synchronisation mit dem definierten
  NTP-Server beschleunigt. Es werden 16 NTP-Abgleichpakete verwendet.

\config {NTP\_SERVER\_x\_PREFER}{NTP\_SERVER\_x\_PREFER}{NTPSERVERxPREFER}

  Diese Konfigurationvariable ist optional.\\
  Mit \var{'yes'} wird dieser NTP-Server anderen NTP-Servern mit gleichem
  Stratum vorgezogen.

\config {NTP\_SERVER\_x\_MINPOLL}{NTP\_SERVER\_x\_MINPOLL}{NTPSERVERxMINPOLL}

  Diese Konfigurationvariable ist optional.\\
  Diese Variable definert den Mindestzeitabstand zwischen
  NTP-Zeitabgleichspaketen. Es sind Werte zwischen 4 (15~s) und 6 (64~s)
  erlaubt.

\config {NTP\_SERVER\_x\_MAXPOLL}{NTP\_SERVER\_x\_MAXPOLL}{NTPSERVERxMAXPOLL}

  Diese Konfigurationvariable ist optional.\\
  Diese Variable definert den Maximalzeitabstand zwischen
  NTP-Zeitabgleichspaketen. Es sind Werte zwischen 10 (1.024~s) und 17
  (131.702~s, entspricht ca. 36,4~h) erlaubt.

\config {NTP\_SERVER\_x\_VERSION}{NTP\_SERVER\_x\_VERSION}{NTPSERVERxVERSION}

  Diese Konfigurationvariable ist optional.\\
  Diese Variable legt die Versionsnummer fest, die in den NTP-Paketen verwendet
  wird. Es sind Werte zwischen 1 und 4 erlaubt.

\config {NTP\_LOCAL\_RTC}{NTP\_LOCAL\_RTC}{NTPLOCALRTC}

  Mit \var{'yes'} wird die Uhr des BIOS zusätzlich als Zeitquelle genutzt.
  Dies macht es dem NTP-Dämon möglich, weiter als Zeitserver zu arbeiten, wenn
  keiner der definierten Zeitserver erreichbar ist oder die angeschlossene 
  Funkuhr keine gültige Zeit mehr liefert.

\config {NTP\_LOCAL\_RTC\_STRATUM}{NTP\_LOCAL\_RTC\_STRATUM}{NTPLOCALRTCSTRATUM}

  Definiert die Priorität der lokalen Uhr innerhalb der NTP-Hierarchie.
  Je höher der Wert desto ungenauer ist die Genauigkeit der genutzten
  Zeitquelle. Ein Server der die Zeit der Referenzuhr nutzt hat ein Stratum von
  \var{'1'}. Für die BIOS-Uhr sollte hier im Normalfall ein Wert von \var{'10'}
  bis \var{'12'} gesetzt werden.

\config {NTP\_ALLOW\_IPV4\_N}{NTP\_ALLOW\_IPV4\_N}{NTPALLOWIPV4N}

  Diese Variable gibt die Anzahl der IPv4-Netze an, die auf den Router via NTP
  zugreifen dürfen.
  
\config {NTP\_ALLOW\_IPV4\_x}{NTP\_ALLOW\_IPV4\_x}{NTPALLOWIPV4x}

  Diese Variable enthält eine Referenz auf ein IPv4-Netzwerk, das auf den
  NTP-Server zugreifen darf, beispielsweise \var{IP\_NET\_1}.

\config {NTP\_ALLOW\_IPV4\_x\_PEERING}{NTP\_ALLOW\_IPV4\_x\_PEERING}{NTPALLOWIPV4xPEERING}

  Diese Konfigurationvariable ist optional.\\
  Mit \var{'yes'} wird der gegenseitige Abgleich des fli4l mit NTP-Servern
  im referenzierten IPv4-Netz erlaubt.

\config {NTP\_ALLOW\_IPV6\_N}{NTP\_ALLOW\_IPV6\_N}{NTPALLOWIPV6N}

  Diese Variable gibt die Anzahl der IPv6-Netze an, die auf den Router via NTP
  zugreifen dürfen.

\config {NTP\_ALLOW\_IPV6\_x}{NTP\_ALLOW\_IPV6\_x}{NTPALLOWIPV6x}

  Diese Variable enthält eine Referenz auf ein IPv6-Netzwerk, das auf den
  NTP-Server zugreifen darf, beispielsweise \var{IPV6\_NET\_1}.

\config {NTP\_ALLOW\_IPV6\_x\_PEERING}{NTP\_ALLOW\_IPV6\_x\_PEERING}{NTPALLOWIPV6xPEERING}

  Diese Konfigurationvariable ist optional.\\
  Mit \var{'yes'} wird der gegenseitige Abgleich des fli4l mit NTP-Servern
  im referenzierten IPv6-Netz erlaubt.

\config {NTP\_CHECK\_STATUS}{NTP\_CHECK\_STATUS}{NTPCHECKSTATUS}

  Setzt man diese Variable auf \var{'yes'}, werden bei \var{OPT\_HTTPD='yes'}
  Informationen zu der Zeitsynchronisation in der WebGUI
  ausgegeben. Bei zusätzlich aktivem \var{RRDTOOL\_NTP} werden Graphen mit
  Verläufen der Zeitabweichung erzeugt.
  
\config {NTP\_SHOW\_STATUS\_VIA\_LED}{NTP\_SHOW\_STATUS\_VIA\_LED}{NTPSHOWSTATUSVIALED}

  Setzt man diese Variable auf\var{'yes'}, wird der NTP-Syn\-chronisationsstatus
  über die definierten LEDs dargestellt, wenn im Paket ``hwsupp'' die
  Ansteuerung von LEDs definiert ist. Dabei gilt die folgende Zuordnung:
  \begin{itemize}
  \item LED 1 blinkt, solange das Jahr nicht korrekt gesetzt wurde, und ist
        ansonsten aus.
  \item LED 2 leuchtet, wenn die Uhrzeit des fli4l synchron ist, und ist aus,
        wenn die Uhrzeit nicht synchron ist, sondern der NTP-Dämon dabei ist,
        sie schrittweise anzupassen.
  \item LED 3 leuchtet, wenn die angeschlossene Funkuhr arbeitet. Sie blinkt,
        wenn die Funkuhr zwar keine gültige Zeit liefert, aber die
        Synchronisation über andere NTP-Server hergestellt werden kann. Sie
        ist aus, wenn der NTP-Server überhaupt keine Zeitquelle zur Verfügung
        hat.
  \end{itemize}

\end{description}
  
\marklabel{sec:ntpsupport}{
\subsection{Support}
}
Support wird nur im Rahmen der \jump{url:ntpfli4lnews}{fli4l-Newsgroups}
geleistet. 

\marklabel{url:ntpfli4lnews}{
 fli4l Newsgroups und ihre Spielregeln: \altlink{http://www.fli4l.de/hilfe/newsgruppen-irc-forum/}
}
