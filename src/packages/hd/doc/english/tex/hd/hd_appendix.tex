% Synchronized to r30143

\marklabel{sec:hd-errors}
{
  \section{HD - Possible Errors Concerning Hardisks/CompactFlashs}
}    
    \textbf{Problem:}

    \begin{itemize}
    \item the router does not recognize the harddisk at all.
    \end{itemize}

    Possible Causes:

    \begin{itemize}
    \item the router lacks drivers for the hd-controller - additional 
      drivers for the controller may be needed. Configure \var{OPT\_\-HDDRV} 
      in this case.
    \item BIOS entry for the disk is wrong.
    \item Controller is defective or switched off.
    \item wrong disk is configured for the installation 
    \item Controller is not supported by fli4l. Some controllers may 
      need special drivers not included with fli4l 
    \end{itemize}

    \textbf{Problem:}
    \begin{itemize}
    \item Installation routine stops
    \item after a remote update of the opt-archive the router refuses to boot
    \item Error messages occur while partitioning or formatting the harddisk
    \end{itemize}

    Possible Causes:
    \begin{itemize}
    \item IDE harddisks can suffer from cables too long or unmatching 
    \item older harddisks may suffer from wrong PIO mode or transfer rate settings 
      in Bios (or controller) eventually being too fast for the disk.
    \item IDE-Chipset not suitable
    \end{itemize}

    Remarks:
    \begin{itemize}
    \item Problems with DMA eventually can be solved by setting
      \verb*?LIBATA_NODMA='no'?. (The default is 'yes'). 
      This activates DMA with ATA devices.
    \end{itemize}

    \textbf{Problem:}
    \begin{itemize}
    \item fli4l doesn't boot from harddisk after the installation
    \end{itemize}
    
    Possible Cause:
    \begin{itemize}
    \item If booting from a CF module fails check if the CF was recognized as LBA 
    or LARGE by the Bios. Correct setting for modules smaller than 512MB is 
    NORMAL or CHS.
    \item an Adaptec 2940 Controller with an old BIOS is used and extended 
    mapping for harddisks over 1GB is active. Update the Bios of the SCSI controller 
    or switch mapping.\\
    \achtung{ By switching the mapping all data on the disk will be lost! }
    \end{itemize}

    \textbf{Problem:}
    \begin{itemize}
    \item Windows error message while preparing of a CF-card: \glqq{}Medium in drive (X:) 
    contains no FAT. [Cancel]\grqq{}.
    \end{itemize}

    Possible Cause:
    \begin{itemize}
    \item The card was removed from the reader too early / without unmounting.
    Windows did not finish writing and the file system is damaged. Prepare the CompactFlash again 
    at the fli4l via HD-install.
    \end{itemize}
