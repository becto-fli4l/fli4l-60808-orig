% Do not remove the next line
% Synchronized to r48680

\section{IPv6~- Internet protocole version 6}

\subsection{Introduction}

Ce paquetage permet au routeur fli4l avec bien des égards de rendre compatible
l'IPv6. Les informations qui sont incluses dans le paquetage IPv6 pour le routeur
sont les adresses IPv6, la gestion des (sous-)réseaux IPv6, la route IPv6
prédéfinie et les règles de pare-feu.
Enfin, il est possible de construire un tunnel automatiquement
avec des fournisseurs IPv6. Maintenant cela fonctionne correctement, mais,
seulement avec des tunnels 6in4, le fournisseur "Hurricane Electric" prend en
charge cette technologie. Les autres technologies comme (AYIYA, 6to4, Teredo) ne
sont pas encore prisent en charge.

IPv6 est le successeur du protocole Internet IPv4. Il a été principalement conçu
pour augmenter la quantité relativement faible des adresses Internet formelles~:
IPv4 supporte environ \(2^{32}\) d'adresses, \footnote{c'est seulement approximatif,
car certaines adresses ont un objectif bien spécifique, comme le Broadcasting et
le Multicasting,} avec IPv6 on a déjà \(2^{128}\) d'adresses. Avec la communication
IPv6, on peut attribuer une adresse unique pour chaque hôte, et nous ne sommes
plus sur des techniques telles que le NAT, le PAT, le Masquerading, etc.

Outre cet aspect, les sujets comme l'autoconfiguration et la sécurité ont aussi
joué un rôle lors du développement du protocole IPv6. Ces questions seront
traitées dans les sections suivantes.

Le plus gros problème avec IPv6 est sa distribution~: Actuellement, l'IPv6~-- par
rapport à IPv4~-- est très peu utilisés. La raison est que le protocole IPv6 et
IPv4 ne sont pas techniquement compatibles l'un avec l'autre et par conséquent
tous les composants matériels et logiciels, qui sont impliqués dans la
transmission de paquets sur Internet pour l'IPv6 doit être installé. Certains
services comme le DNS (Domain Name System) pour IPv6 doivent être ouverts en
conséquence.

Un cercle vicieux s'ouvre alors~: la faible propagation des IPv6 chez les
fournisseurs d'accès Internet amène l'indifférence de la part des fabricants à
équiper les routeurs d'un dispositif pour le fonctionnement IPv6, cela signifie
que les fournisseurs d'accès ont peur de la transition vers IPv6, parce qu'ils
craignent qu'un tel effort ne vos pas la peine. Ce n'est que lentement que le
vent tourne en faveur de l'IPv6, car des réserves d'adresses IPv4 s'épuisent.
\footnote{Maintenant les derniers blocs d'adresses IPv4 ont été attribués par
l'IANA.}

\subsection{Format de l'adresse}

Une adresse IPv6 se compose de huit valeurs de deux octets, elles sont classées
en hexadécimal~:

\emph{Exemple 1~:} \verb*?2001:db8:900:551:0:0:0:2?

\emph{Exemple 2~:} \verb*?0:0:0:0:0:0:0:1? (IPv6-Loopback-Adresse)

Pour réduite l'encombrement des adresses, on peut fusionnée une suite de zéros
successifs, en les supprimant et en ajoutant seulement une paire de deux points.
Les adresses ci-dessus peuvent également être écrites comme ceci~:

\emph{Exemple 1 (compacté)~:} \verb*?2001:db8:900:551::2?

\emph{Exemple 2 (compacté)~:} \verb*?::1?

Une telle réduction est uniquement autorisée d'une fois, pour éviter toute
ambiguïté. L'adresse \verb*?2001:0:0:1:2:0:0:3? peut être réduite comme
ceci \verb*?2001::1:2:0:0:3? ou \verb*?2001:0:0:1:2::3?, mais pas comme ceci
\verb*?2001::1:2::3?, parce que, il serait maintenant difficile de savoir comment
les quatre zéros doivent être répartis sur les zones de réductions.

Une autre ambiguïté existe, si une adresse IPv6 doit être combinée avec un port
(TCP ou UDP)~: dans ce cas, il ne faut pas joindre le port directement avec les
deux-points à l'adresse, parce que ces deux-points seront intégrés à
l'intérieur de l'adresse et donc dans certains cas, il serait difficile de
savoir si la spécification du port est peut-être ou pas un composant de
l'adresse. Il faut donc, dans ce cas mettre l'adresse IPv6 entre deux crochets.
Cette syntaxe est demandée dans les URL (par exemple lorsque l'utilisation doit
indiquer une adresse IPv6 au format numérique dans le navigateur Web).

\emph{Exemple 3~:} \verb*?[2001:db8:900:551::2]:1234?

Voici l'adresse sans mettre les crochets \verb*?2001:db8:900:551::2:1234?,
correspond à l'adresse intégrale \verb*?2001:db8:900:551:0:0:2:1234? vous voyez
quelle ne possède aucune indication de port.

\subsection{Configuration}

\subsubsection{Paramètres généraux}

Les paramètres généraux contiennent d'abord, l'activation du support IPv6,
d'autre part l'attribution optionnelle d'une adresse IPv6 sur le routeur.

\begin{description}
\config{OPT\_IPV6}{OPT\_IPV6}{OPTIPV6}{
Avec cette variable, vous pouvez activer le support IPv6.

Configuration par défaut~:
}
\verb*?OPT_IPV6='no'?

\config{HOSTNAME\_IP6}{HOSTNAME\_IP6}{HOSTNAMEIP6}{(optionnelle)
Cette variable règle explicitement l'adresse IPv6 du routeur. Si la variable
n'est pas définie, l'adresse IPv6 est placé sur la configuration de la premier
adresse du sous-réseau IPv6 (\var{IPV6\_NET\_x}, voir ci-dessous).

Exemple~:
}
\verb*?HOSTNAME_IP6='IPV6_NET_1_IPADDR'?
\end{description}

\subsubsection{Configuration du sous-réseau}

Dans ce paragraphe, nous allons décrire la configuration d'un ou plusieurs
sous-réseaux IPv6. Un sous-réseau IPv6 est une adresse IPv6 étendue qui est
spécifiée par un préfixe et qui est liée à une interface réseau spécifique. Les
autres paramètres concernent l'édition du préfixe et le service DNS dans le
sous-réseau, ainsi que le nom du routeur optionnel à l'intérieur du sous-réseau.

\begin{description}
\config{IPV6\_NET\_N}{IPV6\_NET\_N}{IPV6NETN}{
Dans cette variable, vous indiquez le nombre de sous-réseaux IPv6 à utiliser.
Vous devez définir Au moins un sous-réseau IPv6, pour utiliser l'IPv6 dans le
réseau local.

Configuration par défaut~:
}
\verb*?IPV6_NET_N='0'?

\config{IPV6\_NET\_x}{IPV6\_NET\_x}{IPV6NETx}{
Dans cette variable, vous indiquez l'adresse IPv6, contenu dans le sous-réseau
IPv6 du routeur, ainsi que la taille du masque de sous-réseau en utilisant la
notation CIDR. Si le sous-réseau est un routage public, il provient en générale
d'Internet ou d'un prestataire de tunnel.

\wichtig{Si vous activez la configuration automatique sans-état dans le même
sous-réseau (voir la section \var{IPV6\_NET\_x\_ADVERTISE} ci-dessous), la
longueur du préfixe du sous-réseau doit faire 64 bits!}

\wichtig{Si le sous-réseau est connecté à un tunnel (voir \var{IPV6\_NET\_x\_TUNNEL}
ci-dessous), vous devez indiquer seulement une partie de l'adresse du routeur,
mais \emph{pas} le préfixe du sous-réseau associé au tunnel (qui se trouve
dans \var{IPV6\_TUNNEL\_x\_PREFIX}), avec ce préfixe, l'adresse pourra être
combiné~! Dans la version précédente du paquetage IPv6, la variable
\var{IPV6\_TUNNEL\_x\_PREFIX} n'existait pas, le préfixe et le sous-réseau de
l'adresse du routeur étaient ensemble dans la variable \var{IPV6\_NET\_x}.
Toutefois, cela ne s'applique pas si le préfixe du sous-réseau est assigné
dynamiquement par le fournisseur à la construction du tunnel. De plus, la
longueur du préfixe du sous-réseau (dans ce cas~: /48) est cachée, si bien que
le routage prédéfini ne peut pas être correctement réglé et que la route vers
les destinations spécifiques conduit alors à des effets étranges.}

Exemples~:
}

\begin{example}
\begin{verbatim}
IPV6_NET_1='2001:db8:1743:42::1/64'       # sans Tunnel~: adresse complette
IPV6_NET_1_TUNNEL=''

IPV6_NET_2='0:0:0:42::1/64'               # avec Tunnel~: adresse partielle
IPV6_NET_2_TUNNEL='1'
IPV6_TUNNEL_1_PREFIX='2001:db8:1743::/48' # voir section "configuration du Tunnel"
\end{verbatim}
\end{example}

\config{IPV6\_NET\_x\_DEV}{IPV6\_NET\_x\_DEV}{IPV6NETxDEV}{
Avec cette variable, vous indiquez le nom de l'interface du sous-réseau IPv6
sur laquelle l'adresse IPv6 sera associée. Cette interface réseau n'entre
\emph{pas} en collision avec l'interface réseau qui a été attribuée dans la
configuration de base (\texttt{base.txt}), les deux adresses IPv4 et IPv6
pourront être affectées sur cette interface réseau.

Exemple~:
}
\verb*?IPV6_NET_1_DEV='eth0'?

\config{IPV6\_NET\_x\_TUNNEL}{IPV6\_NET\_x\_TUNNEL}{IPV6NETxTUNNEL}{
Dans cette variable, vous indiquez un sous-réseau IPv6 spécifique, à l'index
du tunnel. Le préfixe du tunnel spécifié sera combiné avec l'adresse du routeur
pour obtenir l'adresse IPv6 complète pour le routeur. Si la variable est vide
ou non définie, aucun sous-réseau ne fera partie du tunnel, dans la variable
\var{IPV6\_NET\_x} vous devez indiquer une 'adresse IPv6 complète pour le routeur,
y compris le masque de réseau (voir plus haut).

Un tunnel peut être attribué à plusieurs sous-réseaux, le préfixe du sous-réseau
du tunnel est généralement assez grand pour qu'il puisse être divisé en plusieurs
sous-réseaux (/56 ou plus). Bien sûr, ce n'est pas possible dans l'autre sens,
attribuer un sous-réseau à plusieurs préfixes du sous-réseau du tunnel, car
l'adresse du sous-réseau serait ambiguë.

Exemple~:
}
\verb*?IPV6_NET_1_TUNNEL='1'?

\config{IPV6\_NET\_x\_ADVERTISE}{IPV6\_NET\_x\_ADVERTISE}{IPV6NETxADVERTISE}{
Avec cette variable, vous déterminez si le préfixe du sous-réseau sera distribué
par "l'intermédiaire du routeur" dans le LAN. Cela est utilisé pour une
"stateless autoconfiguration" (ou configuration automatique sans état) et ne doit
pas être confondu avec le DHCPv6. Les valeurs possibles sont "yes" ou "no".

Il est recommandé d'activer ce paramètre, à moins que toutes les adresses dans
le réseau soient affectées statiquement ou qu'un autre routeur est déjà compétent
pour notifier le préfixe du sous-réseau.

\wichtig{La distribution automatique des sous-réseaux fonctionne seulement si
le sous-réseau est un réseau /64, c.-à-d., si la longueur du préfixe du
sous-réseau est de 64 bits! La raison est que les hôtes du réseau calculent
l'adresse IPv6 à partir du préfixe et de leur adresse MAC, si l'hôte ne
partage pas les 64 bits cela ne fonctionne pas. Si la configuration automatique
échoue, il faut vérifier le préfixe du sous-réseau, il a peut-être été spécifié
de manière incorrecte (par exemple /48).}

Configuration par défaut~:
}
\verb*?IPV6_NET_1_ADVERTISE='yes'?

\config{IPV6\_NET\_x\_ADVERTISE\_DNS}{IPV6\_NET\_x\_ADVERTISE\_DNS}{IPV6NETxADVERTISEDNS}{
Avec cette variable vous déterminez si le service DNS local sur le sous-réseau
IPv6 sera distribué par "l'intermédiaire du routeur". Cela ne fonctionne que si
la fonction IPv6 du service DNS est activé par le biais de la variable
\var{DNS\_SUPPORT\_IPV6}='yes'. Les valeurs possibles sont "yes" ou "no".

Configuration par défaut~:
}
\verb*?IPV6_NET_1_ADVERTISE_DNS='no'?

\config{IPV6\_NET\_x\_NAME}{IPV6\_NET\_x\_NAME}{IPV6NETxNAME}{(optionnelle)
Dans cette variable, vous pouvez paramètrer un nom d'hôte spécifique pour chaque
interface du sous-réseau IPv6 du routeur.

Exemple~:
}
\verb*?IPV6_NET_1_NAME='fli4l-subnet1'?

\end{description}

\subsubsection{Configuration d'un Tunnel}

Dans ce paragraphe nous allons présenter la configuration d'un tunnel IPv6-6in4
Un tel tunnel est utile lorsque votre propre fournisseur d'accès Internet ne
supporte pas l'IPv6 par défaut. Ainsi, nous pouvons faire un tunnel-broker avec
un hôte bien précis sur Internet, avec le soi-disant PoP (Point of Presence), il
faut construit une connexion bidirectionnelle via IPv4, les paquets IPv6 seront
ensuite empaquetés et acheminés (d'où 6 "in" 4 parce que les paquets IPv6 sont
encapsulés dans les paquets IPv4). \footnote{Il s'agit de l'IPv4 protocole 41,
"encapsulation IPv6".} Pour que le tunnel fonctionne, il faut configurer les
routeurs avec le paquetage IPv6 des deux côtés de la connexion Internet. Le
premier paragraphe décrit la configuration, le deuxième paragraphe décrit la
connexion.

\begin{description}
\config{IPV6\_TUNNEL\_N}{IPV6\_TUNNEL\_N}{IPV6TUNNELN}{
Avec cette variable vous indiquez le nombre de tunnels 6in4 à mettre en place.

Exemple~:
}
\verb*?IPV6_TUNNEL_N='1'?

\config{IPV6\_TUNNEL\_x\_TYPE}{IPV6\_TUNNEL\_x\_TYPE}{IPV6TUNNELxTYPE}{
Avec cette variable, vous déterminez le type de tunnel. Actuellement, les
valeurs possibles sont~: "raw" pour un tunnel qui envoi des paquets "brut",
"static" pour un tunnel statique et "he" pour un tunnel du fournisseur Hurricane
Electric. Au sujet du tunnel Heartbeat voir le paragraphe plus bas.

Exemple~:
}
\verb*?IPV6_TUNNEL_1_TYPE='he'?

\config{IPV6\_TUNNEL\_x\_DEFAULT}{IPV6\_TUNNEL\_x\_DEFAULT}{IPV6TUNNELxDEFAULT}{
Avec cette variable, vous déterminez si les paquets IPv6 qui ne sont pas
adressés au niveau local ou aux réseaux locaux, doivent être routés sur un autre
tunnel. Il ne peut y avoir qu'un seul tunnel (parce que seulement une route par
défaut peut exister). Les valeurs possibles sont "yes" ou "no".

\wichtig{le tunnel doit exactement être une passerelle par défaut pour les
données IPv6 sortantes, car la communication avec des hôtes IPv6 ne serait pas
possible autrement sur Internet~! L'utilisation exclusive du tunnel pas par défaut,
n'est utile que si le trafic IPv6 sortant est envoyé via une route par défaut
configurée séparément et qui n'est pas en rapport avec un tunnel. Voir
l'introduction du paragraphe "configuration de route" et aussi la description
de la variable \var{IPV6\_ROUTE\_x} ci-dessous.}

Configuration par défaut~:
}
\verb*?IPV6_TUNNEL_1_DEFAULT='no'?

\config{IPV6\_TUNNEL\_x\_PREFIX}{IPV6\_TUNNEL\_x\_PREFIX}{IPV6TUNNELxPREFIX}{
Avec cette variable, vous indiquez le préfixe IPv6 du sous-réseau du tunnel
dans la notation CIDR, c.-à-d. que vous indiquez la longueur du préfixe, mais
aussi l'adresse IPv6. Cette information est précisée dans la convention du
fournisseur de tunnel. En ce qui concerne certains fournisseurs de tunnel, si le
préfixe est réaffecté à chaque construction du tunnel, alors cette information
sera inutile. (Actuellement, de tels fournisseurs ne sont pas supportés).

\wichtig{Cette variable \emph{peut} restée vide, si le tunnel n'a pas de préfixe
de sous-réseau attribué. Toutefois, ce tunnel ne peut pas être affecté à un
sous-réseau IPv6 par la variable (\var{IPV6\_NET\_x}), parce que les adresses
IPv6 dans le sous-réseau ne peuvent pas être calculées. Il est logique d'une
telle configuration ne soit que provisoire, en attendant l'activation du tunnel
et avant que le fournisseur de tunnel attribue un préfixe du sous-réseau.}

Exemples~:
}

\begin{example}
\begin{verbatim}
IPV6_TUNNEL_1_PREFIX='2001:db8:1743::/48'      # /48-sous-réseau
IPV6_TUNNEL_2_PREFIX='2001:db8:1743:5e00::/56' # /56-sous-réseau
\end{verbatim}
\end{example}

\config{IPV6\_TUNNEL\_x\_LOCALV4}{IPV6\_TUNNEL\_x\_LOCALV4}{IPV6TUNNELxLOCALV4}{
Dans cette variable, vous indiquez l'adresse IPv4 locale du tunnel ou le paramètre
'dynamic' si l'adresse IPv4 est allouée dynamiquement par le circuit WAN
actif. S'il s'agit d'un tunnel Heartbeat (voir \var{IPV6\_TUNNEL\_x\_TYPE}
ci-dessus).

Exemple~:
}

\begin{example}
\begin{verbatim}
IPV6_TUNNEL_1_LOCALV4='172.16.0.2'
IPV6_TUNNEL_2_LOCALV4='dynamic'
\end{verbatim}
\end{example}

\config{IPV6\_TUNNEL\_x\_REMOTEV4}{IPV6\_TUNNEL\_x\_REMOTEV4}{IPV6TUNNELxREMOTEV4}{
Dans cette variable, vous indiquez l'adresse IPv4 distant du tunnel. Cette
information est habituellement déterminée par le fournisseur du tunnel.

Exemple (Correspond au PoP deham01 d'Easynet)~:
}

\begin{example}
\begin{verbatim}
IPV6_TUNNEL_1_REMOTEV4='212.224.0.188
\end{verbatim}
\end{example}

\wichtig{Si la variable \var{PF\_INPUT\_ACCEPT\_DEF} est sur "no", c.-à-d que
le pare-feu IPv4 est configuré manuellement, une règle est nécessaire pour
accepter tous les paquets IPv6-in-IPv4 (Protocole-IP 41) de l'extrémité du
tunnel. Surnommé point d'arrêt du tunnel, la règle correspondante est indiqué
ci-dessous~:}

\begin{example}
\begin{verbatim}
PF_INPUT_x='prot:41 212.224.0.188 ACCEPT'
\end{verbatim}
\end{example}

\config{IPV6\_TUNNEL\_x\_LOCALV6}{IPV6\_TUNNEL\_x\_LOCALV6}{IPV6TUNNELxLOCALV6}{
Dans cette variable, vous indiquez l'adresse IPv6 local du tunnel avec le masque
de sous-réseau, en utilisant la notation CIDR. Cette information est donnée par
le fournisseur d'accès du tunnel. Lors d'une nouvelle configuration du tunnel,
les fournisseurs de tunnel l'attribuent à chaque extrémité du tunnel. Cette
information est inutile, (actuellement les fournisseurs ne supportent pas encore
cette fonction).

Exemple~:
}
\verb*?IPV6_TUNNEL_1_LOCALV6='2001:db8:1743::2/112'?

\config{IPV6\_TUNNEL\_x\_REMOTEV6}{IPV6\_TUNNEL\_x\_REMOTEV6}{IPV6TUNNELxREMOTEV6}{
Dans cette variable, vous indiquez l'adresse IPv6 distante du tunnel. Cette
information est donnée par le fournisseur d'accès du tunnel. Le masque de sous-réseau
n'est pas nécessaire, car il est récupéré dans La variable \var{IPV6\_TUNNEL\_x\_LOCALV6}.
Lors d'une nouvelle configuration du tunnel, les fournisseurs de tunnel
l'attribuent à chaque extrémité du tunnel. Cette information est inutile,
(actuellement les fournisseurs ne supportent pas encore cette fonction).

Exemple~:
}
\verb*?IPV6_TUNNEL_1_REMOTEV6='2001:db8:1743::1'?

\config{IPV6\_TUNNEL\_x\_DEV}{IPV6\_TUNNEL\_x\_DEV}{IPV6TUNNELxDEV}{
(optionnelle) Dans cette variable, vous indiquez le nom de l'interface réseau du
tunnel à produire. Si vous avez plusieurs tunnels, ils doivent être nommés
différemment, de sorte que tout fonctionne. Si la variable n'est pas définie,
un nom pour le tunnel sera généré automatiquement ("v6tun" + index Tunnel).

Exemple~:
}
\verb*?IPV6_TUNNEL_1_DEV='6in4'?

\config{IPV6\_TUNNEL\_x\_MTU}{IPV6\_TUNNEL\_x\_MTU}{IPV6TUNNELxMTU}{
(optionnelle) Dans cette variable, vous indiquez la taille du MTU (Maximum
Transfert Unit) en octets, c.-à-d. le plus grand paquet qui peut être envoyé sur
le tunnel. en régle général cette information est précisée par le fournisseur de
tunnel. Le réglage par défaut si non spécifié est de "1280", il doit être
compatible avec tous les tunnels.

Configuration par défaut~:
}
\verb*?IPV6_TUNNEL_1_MTU='1280'?

\end{description}

Certains fournisseurs de tunnel exigent un signe de vie qui soit en permanence
envoyée sur le routeur du fournisseur de tunnel, pour s'assurer que l'hôte
sollicite le tunnel, bien que celui-ci n'est pas utilisé. En plus le soi-disant
protocole Heartbeat ("battement de coeur") est utilisé. Les fournisseurs exigent
généralement une ouverture de session réussie avec identifiant et mot de passe
pour empêcher les abus. Si vous utilisez un tunnel Heartbeat, alors les
informations appropriées doivent être renregistré, elles sont décrites plus bas.

\begin{description}
\config{IPV6\_TUNNEL\_x\_USERID}{IPV6\_TUNNEL\_x\_USERID}{IPV6TUNNELxUSERID}{
Dans cette variable, vous indiquez le nom d'utilisateur, nécessaires pour
la connexion au tunnel.

Exemple~:
}
\verb*?IPV6_TUNNEL_1_USERID='USERID'?

\config{IPV6\_TUNNEL\_x\_PASSWORD}{IPV6\_TUNNEL\_x\_PASSWORD}{IPV6TUNNELxPASSWORD}{
Dans cette variable, vous indiquez le mot de passe pour le nom d'utilisateur
spécifié ci-dessus. Il ne doit pas contenir d'espaces.

Exemple~:
}
\verb*?IPV6_TUNNEL_1_PASSWORD='passwort'?

\config{IPV6\_TUNNEL\_x\_TUNNELID}{IPV6\_TUNNEL\_x\_TUNNELID}{IPV6TUNNELxTUNNELID}{
Dans cette variable, vous indiquez, l'indenfication du tunnel.

Exemple~:
}
\verb*?IPV6_TUNNEL_1_TUNNELID='TunnelID'?

\config{IPV6\_TUNNEL\_x\_TIMEOUT}{IPV6\_TUNNEL\_x\_TIMEOUT}{IPV6TUNNELxTIMEOUT}{
(optionnelle) Dans cette variable, vous indiquez le temps d'attente en seconde,
avant la construction du tunnel. La valeur par défaut dépend du fournisseur
d'accès du tunnel.

Exemple~:
}
\verb*?IPV6_TUNNEL_1_TIMEOUT='30'?
\end{description}

\subsubsection{Configuration des routes}

Les routes sont des chemins pour rediriger les paquets IPv6. Cela signifie que le
routeur doit savoir ou envoyer les paquets entrants, il s'appuie sur une table
de routage pour trouver exactement les informations. Pour les paquets IPv6, il
est important de savoir où sont envoyés les paquets qui ne font pas partie du
réseau local. Pour cela, une route par défaut doit être configurée pour envoyer
tous les paquets à l'autre extrémité du tunnel IPv6. Vous pouvez également ajouter
d'autres routes qui relient les sous-réseaux IPv6 les uns aux autres.

\begin{description}
\config{IPV6\_ROUTE\_N}{IPV6\_ROUTE\_N}{IPV6ROUTEN}{
Dans cette variable vous indiquez le nombre de routes IPv6. En général, aucune
route supplémentaire IPv6 n'est nécessaire.

Configuration par défaut~:
}
\verb*?IPV6_ROUTE_N='0'?

\config{IPV6\_ROUTE\_x}{IPV6\_ROUTE\_x}{IPV6ROUTEx}{
Dans cette variable, vous indiquez la route sous la forme 'Réseau de destination
Passerelle', le réseau de destination est écrit en utilisant la notation CIDR.
Vous devez indiquer \var{::/0} pour la route par défaut du réseau de destination.
Cependant, il n'est pas nécessaire de configurer la route par défaut qui passe
par le tunnel (voir l'introduction de ce paragraphe).

Exemple~:
}
\verb*?IPV6_ROUTE_1='2001:db8:1743:44::/64 2001:db8:1743:44::1'?
\end{description}

\subsubsection{IPv6-Firewall}

Comme pour les réseaux IPv4, les réseaux IPv6 ont besoin d'un pare-feu, ainsi
le monde extérieur ne pourra pas joindre les ordinateurs du réseau local.
Cela est d'autant plus important, car chaque ordinateur est remplacé dans le cas
normal, d'une adresse IPv6 unique, cette adresse qui peut être affectée à
l'ordinateur de façon permanente, car elle est basée sur l'adresse MAC de la
carte d'interface réseau. \footnote{Une exception existe, si "Privacy extension"
est activé pour les hôtes du LAN, alors une partie de l'adresse IPv6 sera générée
de façon aléatoire. Ces adresses par définition, ne sont pas connues du monde
extérieur et donc la configuration du firewall sera partiellement ou pas du tout
pertinente.} Par conséquent, le pare-feu interdira toute demande provenant de
l'extérieur, dans ce paragraphe vous allez voir comment ouvrir les entrées
correspondantes petit à petit~-- selon vos besoins~--.

La configuration du pare-feu IPv6, correspond grosso modo à la configuration du
pare-feu IPv4. Les différences particulières seront examinées séparément.

\begin{description}

\config{PF6\_LOG\_LEVEL}{PF6\_LOG\_LEVEL}{PF6LOGLEVEL} La configuration du
système de journalisation dans la variable \var{PF6\_LOG\_LEVEL} est utilisée
pour toutes les chaînes ci-dessous sans distinction, leur contenu peut être
réglé sur l'une des valeurs suivantes~: debug, info, notice, warning, err, crit,
alert, emerg.

\config{PF6\_INPUT\_POLICY}{PF6\_INPUT\_POLICY}{PF6INPUTPOLICY}{
Cette variable définit la politique par défaut pour les paquets entrants sur le
routeur avec la (chaîne INPUT). Les valeurs possibles sont "REJECT" (par défaut,
rejette tous les paquets), "DROP" (rejette en secret tous les paquets), "ACCEPT"
(accepte tous les paquets). Pour une description plus détaillée, voir la
documentation de la variable \var{PF\_INPUT\_POLICY}

Configuration par défaut~:
}
\verb*?PF6_INPUT_POLICY='REJECT'?

\config{PF6\_INPUT\_ACCEPT\_DEF}{PF6\_INPUT\_ACCEPT\_DEF}{PF6INPUTACCEPTDEF}{
Dans cette variable vous pouvez activer les règles prédéfinies pour la
chaîne INPUT du pare-feu IPv6. Les valeurs possibles sont "yes" ou "no".

La règle par défaut pour l'ouverture entrante du trafic pings-ICMPv6 (un ping
par seconde en tant que limite), ainsi que pour les paquets NPD (Neighbour
Discovery Procotol) sur le pare-feu, qui sont nécessaires pour l'auto-configuration
sans état des réseaux IPv6. La communication localhost et la réponse des
paquets entre la communication d'origine locale, sont également autorisés.
Enfin, le pare-feu IPv4 est réglé de telle sorte que pour chaque tunnel IPv6
encapsulé dans le paquet IPv4, la communication avec l'extrémité du tunnel sera
acceptée.

Configuration par défaut~:
}
\verb*?PF6_INPUT_ACCEPT_DEF='yes'?

\config{PF6\_INPUT\_LOG}{PF6\_INPUT\_LOG}{PF6INPUTLOG}{
Cette variable active le fichier journal il enregistre tous les paquets entrants
et rejetés.Les valeurs possibles sont "yes" ou "no". Pour une description plus
détaillée, voir la documentation de la variable \var{PF\_INPUT\_LOG}.

Configuration par défaut~:
}
\verb*?PF6_INPUT_LOG='no'?

\config{PF6\_INPUT\_LOG\_LIMIT}{PF6\_INPUT\_LOG\_LIMIT}{PF6INPUTLOGLIMIT}{
On configure avec cette variable une limite pour le fichier journal de la
chaîne INPUT du pare-feu IPv6, pour garder le fichier journal en lecture. Pour
une description plus détaillée, voir la documentation de la variable
\var{PF\_INPUT\_LOG\_LIMIT}.

Configuration par défaut~:
}
\verb*?PF6_INPUT_LOG_LIMIT='3/minute:5'?

\config{PF6\_INPUT\_REJ\_LIMIT}{PF6\_INPUT\_REJ\_LIMIT}{PF6INPUTREJLIMIT}{
On configure avec cette variable une limite pour le rejet des paquets TCP entrants.
Les paquets TCP dépassant cette limite, seront rejetés avec la méthode douce
(DROP). Pour une description plus détaillée, voir la documentation de la variable
\var{PF\_INPUT\_REJ\_LIMIT}.

Configuration par défaut~:
}
\verb*?PF6_INPUT_REJ_LIMIT='1/second:5'?

\config{PF6\_INPUT\_UDP\_REJ\_LIMIT}{PF6\_INPUT\_UDP\_REJ\_LIMIT}{PF6INPUTUDPREJLIMIT}{
On configure avec cette variable une limite pour le rejet des paquets UDP entrants.
Les paquets UDP dépassant cette limite, seront rejetés avec la méthode douce (DROP).
Pour une description plus détaillée, voir la documentation de la variable
\var{PF\_INPUT\_UDP\_REJ\_LIMIT}.

Configuration par défaut~:
}
\verb*?PF6_INPUT_UDP_REJ_LIMIT='1/second:5'?

\config{PF6\_INPUT\_ICMP\_ECHO\_REQ\_LIMIT}{PF6\_INPUT\_ICMP\_ECHO\_REQ\_LIMIT}{PFI6NPUTICMPECHOREQLIMIT}{
Avec cette variable, vous définissez la façon de répondre à une demande de requête
écho ICMPv6 commune. La fréquence et la limite de restriction est décrite
analogiquement comme ceci 'n/unité de temps\-rafales' par exemple, '3/minute:5'.
Une fois que la limite est dépassée, le paquet est tout simplement ignoré (DROP).
S'il la variable est vide, la valeur par défaut utilisé sera la suivante
'1/seconde:5' si la variable contient 'none', alors, aucune limite ne sera effectuée.

Configuration par défaut~:
}
\verb*?PF6_INPUT_ICMP_ECHO_REQ_LIMIT='1/second:5'?

\config{PF6\_INPUT\_ICMP\_ECHO\_REQ\_SIZE}{PF6\_INPUT\_ICMP\_ECHO\_REQ\_SIZE}{PF6INPUTICMPECHOREQSIZE}{
Avec cette variable, vous définissez la taille (en octets) que peut recevoir la
demande de requête écho ICMPv6. Ce chiffre vient "ajouter" des donnés à l’en-tête
du paquet à prendre en considération. La valeur par défaut est de 150 octets.

Configuration par défaut~:
}
\verb*?PF6_INPUT_ICMP_ECHO_REQ_SIZE='150'?

\config{PF6\_INPUT\_N}{PF6\_INPUT\_N}{PF6INPUTN}{
Dans cette variable vous indiquez le nombre de règles du pare-feu IPv6 pour les
paquets entrants (chaîne INPUT). Par défaut, deux règles sont activées~:
la première permet l'accès au routeur par tous des hôtes locaux via
l'adresse du niveau de lien et la seconde permet la communication des hôtes du
premier sous-réseau IPv6 défini avec le routeur.

Si plusieurs sous-réseaux IPv6 locaux sont définis, la seconde règle doit
être reproduite au temps de fois que nécessaire. Voir le fichier de configuration.

Exemple~:
}
\verb*?PF6_INPUT_N='2'?

\config{PF6\_INPUT\_x}{PF6\_INPUT\_x}{PF6INPUTx}{
Dans cette variable, vous indiquez la règle pour la chaîne INPUT du pare-feu
IPv6. Pour une description plus détaillée, voir la documentation de la variable
\var{PF\_INPUT\_x}.

Les différences par rapport au pare-feu IPv4~:
\begin{itemize}
\item Au lieu de \var{IP\_NET\_x} vous devez mettre \var{IPV6\_NET\_x}.
\item Au lieu de \var{IP\_ROUTE\_x} vous devez mettre \var{IPV6\_ROUTE\_x}.
\item Les adresses IPv6 doivent être placées entre deux crochets. (y compris
    le masque de sous réseau, s'il est disponible).
\item Tous les adresses IPv6 que vous indiquez (y compris les variables
    \var{IPV6\_NET\_x}, etc.) doivent être placées entre deux crochets, si
    l’adresse est suivi d’un port ou d’une plage de ports.
\end{itemize}

Exemple~:
}

\begin{example}
\begin{verbatim}
PF6_INPUT_1='[fe80::0/10] ACCEPT'
PF6_INPUT_2='IPV6_NET_1 ACCEPT'
PF6_INPUT_3='tmpl:samba DROP NOLOG'
\end{verbatim}
\end{example}

\config{PF6\_INPUT\_x\_COMMENT}{PF6\_INPUT\_x\_COMMENT}{PF6INPUTxCOMMENT}{
Dans cette variable, vous pouvez indiquer une description ou un commentaire
associé à la règle INPUT.

Exemple~:
}
\verb*?PF6_INPUT_3_COMMENT='no samba traffic allowed'?

\config{PF6\_FORWARD\_POLICY}{PF6\_FORWARD\_POLICY}{PF6FORWARDPOLICY}{
Avec cette variable vous définissez la stratégie par défaut pour les paquets
transmis par le routeur avec la (chaîne FORWARD). Les valeurs possibles sont
"REJECT" (par défaut, rejette tous les paquets), "DROP" (rejette en secret tous
les paquets), "ACCEPT" (accepte tous les paquets). Pour une description plus
détaillée, voir la documentation de la variable \var{PF\_FORWARD\_POLICY}.

Configuration par défaut~:
}
\verb*?PF6_FORWARD_POLICY='REJECT'?

\config{PF6\_FORWARD\_ACCEPT\_DEF}{PF6\_FORWARD\_ACCEPT\_DEF}{PF6FORWARDACCEPTDEF}{
Cette variable active les règles prédéfinies pour la chaîne FORWARD du pare-feu
IPv6. Les valeurs possibles sont "yes" ou "no".

Ouverture des règles par défaut sur le pare-feu pour les ping ICMPv6 sortants
(un ping par seconde comme limite). Les paquets de réponses au ping seront
également autorisés.

Configuration par défaut~:
}
\verb*?PF6_FORWARD_ACCEPT_DEF='yes'?

\config{PF6\_FORWARD\_LOG}{PF6\_FORWARD\_LOG}{PF6FORWARDLOG}{
Cette variable active le fichier journal il enregistre tous les paquets entrants
et rejetés. Les valeurs possibles sont "yes" ou "no". Pour une description plus
détaillée, voir la documentation de la variable \var{PF\_FORWARD\_LOG}.

Configuration par défaut~:
}
\verb*?PF6_FORWARD_LOG='no'?

\config{PF6\_FORWARD\_LOG\_LIMIT}{PF6\_FORWARD\_LOG\_LIMIT}{PF6FORWARDLOGLIMIT}{
On configure avec cette variable une limite pour le fichier journal de la chaîne FORWARD
du pare-feu IPv6, pour garder le fichier journal en lecture. Pour une description
plus détaillée, voir la documentation de la variable \var{PF\_FORWARD\_LOG\_LIMIT}.

Configuration par défaut~:
}
\verb*?PF6_FORWARD_LOG_LIMIT='3/minute:5'?

\config{PF6\_FORWARD\_REJ\_LIMIT}{PF6\_FORWARD\_REJ\_LIMIT}{PF6FORWARDREJLIMIT}{
On configure avec cette variable une limite pour le rejet des paquets TCP entrants.
Les paquets TCP dépassant cette limite, seront rejetés en secret avec (DROP).
Pour une description plus détaillée, voir la documentation de la variable
\var{PF\_FORWARD\_REJ\_LIMIT}.

Configuration par défaut~:
}
\verb*?PF6_FORWARD_REJ_LIMIT='1/second:5'?

\config{PF6\_FORWARD\_UDP\_REJ\_LIMIT}{PF6\_FORWARD\_UDP\_REJ\_LIMIT}{PF6FORWARDUDPREJLIMIT}{
On configure avec cette variable une limite pour le rejet des paquets UDP entrants.
Les paquets UDP dépassant cette limite, seront rejetés avec la méthode douce
(DROP). Pour une description plus détaillée, voir la documentation de la variable
\var{PF\_FORWARD\_UDP\_REJ\_LIMIT}.

Configuration par défaut~:
}
\verb*?PF6_FORWARD_UDP_REJ_LIMIT='1/second:5'?

\config{PF6\_FORWARD\_N}{PF6\_FORWARD\_N}{PF6FORWARDN}{
Dans cette variable vous indiquez le nombre de règles du pare-feu IPv6 pour les
paquets entrants (chaîne FORWARD). Par défaut, deux règles sont activées~: la
première empêche la transmission de tous les paquets samba dans d'autre réseaux
qui ne proviennent pas du réseau local et la seconde permet la communication à
partir des hôtes du premier sous-réseau IPv6 défini dans le routeur.

Si plusieurs sous-réseaux IPv6 locaux sont définis, la seconde règle doit
être reproduite au temps de fois que nécessaire. Voir le fichier de configuration.

Exemple~:
}
\verb*?PF6_FORWARD_N='2'?

\config{PF6\_FORWARD\_x}{PF6\_FORWARD\_x}{PF6FORWARDx}{
Dans cette variable, vous indiquez la règle pour la chaîne FORWARD du pare-feu
IPv6. Pour une description plus détaillée, voir la documentation de la variable
\var{PF\_FORWARD\_x}.

Les différences par rapport au pare-feu IPv4~:
\begin{itemize}
\item Au lieu de \var{IP\_NET\_x} vous devez mettre \var{IPV6\_NET\_x}.
\item Au lieu de \var{IP\_ROUTE\_x} vous devez mettre \var{IPV6\_ROUTE\_x}.
\item Les adresses IPv6 doivent être placées entre deux crochets. (y compris
    le masque de sous réseau, s'il est disponible).
\item Tous les adresses IPv6 que vous indiquez (y compris les variables
    \var{IPV6\_NET\_x}, etc.) doivent être placées entre deux crochets si
    l’adresse est suivi d’un port ou d’une plage de ports.
\end{itemize}

Exemple~:
}

\begin{example}
\begin{verbatim}
PF6_FORWARD_1='tmpl:samba DROP'
PF6_FORWARD_2='IPV6_NET_1 ACCEPT'
\end{verbatim}
\end{example}

\config{PF6\_FORWARD\_x\_COMMENT}{PF6\_FORWARD\_x\_COMMENT}{PF6FORWARDxCOMMENT}{
Dans cette variable, vous pouvez indiquer une description ou un commentaire
associé à la règle FORWARD.

Exemple~:
}
\verb*?PF6_FORWARD_1_COMMENT='no samba traffic allowed'?

\config{PF6\_OUTPUT\_POLICY}{PF6\_OUTPUT\_POLICY}{PF6OUTPUTPOLICY}{
Cette variable définit la stratégie par défaut pour les paquets sortants du
routeur(chaîne OUTPUT). Les valeurs possibles sont "REJECT" (par défaut, pour
tous les paquets), "DROP" (rejette secrètement tous les paquets) et "ACCEPT"
(accepte tous les paquets). Pour plus de détails, reportez-vous à la
documentation de la variable \var{PF\_OUTPUT\_POLICY}.

Configuration par défaut~:
}
\verb*?PF6_OUTPUT_POLICY='REJECT'?

\config{PF6\_OUTPUT\_ACCEPT\_DEF}{PF6\_OUTPUT\_ACCEPT\_DEF}{PF6OUTPUTACCEPTDEF}{
Cette variable active les règles pré-réglées pour la chaîne OUTPUT du pare-feu
IPv6. Les valeurs possibles sont "yes" ou "no". À l'heure actuelle, il n'existe
pas de règle prédéfinie.

Configuration par défaut~:
}
\verb*?PF6_OUTPUT_ACCEPT_DEF='yes'?

\config{PF6\_OUTPUT\_LOG}{PF6\_OUTPUT\_LOG}{PF6OUTPUTLOG}{
Cette variable permet l'enregistrement tous les paquets sortants rejetés. Les
valeurs possibles sont "yes" ou "no". Pour plus de détails, reportez-vous à
la documentation de la variable \var{PF\_OUTPUT\_LOG}.

Configuration par défaut~:
}
\verb*?PF6_OUTPUT_LOG='no'?

\config{PF6\_OUTPUT\_LOG\_LIMIT}{PF6\_OUTPUT\_LOG\_LIMIT}{PF6OUTPUTLOGLIMIT}{
On configure avec cette variable une limite pour le journal de la chaîne
OUTPUT du pare-feu IPv6, pour garder le fichier journal en lecture. Pour une
description plus détaillée de la documentation voir la variable
\var{PF\_OUTPUT\_LOG\_LIMIT}.

Configuration par défaut~:
}
\verb*?PF6_OUTPUT_LOG_LIMIT='3/minute:5'?

\config{PF6\_OUTPUT\_REJ\_LIMIT}{PF6\_OUTPUT\_REJ\_LIMIT}{PF6OUTPUTREJLIMIT}{
On configure avec cette variable une limite pour le rejet des paquets TCP sortants.
Les paquets TCP dépassant cette limite, seront rejetés en secret avec (DROP).
Pour une description plus détaillée, voir la documentation de la variable
\var{PF\_OUTPUT\_REJ\_LIMIT}.

Configuration par défaut~:
}
\verb*?PF6_OUTPUT_REJ_LIMIT='1/second:5'?

\config{PF6\_OUTPUT\_UDP\_REJ\_LIMIT}{PF6\_OUTPUT\_UDP\_REJ\_LIMIT}{PF6OUTPUTUDPREJLIMIT}{
On configure avec cette variable une limite pour le rejet des paquets UDP sortants.
Les paquets UDP dépassant cette limite, seront rejetés en secret avec (DROP).
Pour une description plus détaillée, voir la documentation de la variable
\var{PF\_OUTPUT\_UDP\_REJ\_LIMIT}.

Configuration par défaut~:
}
\verb*?PF6_OUTPUT_UDP_REJ_LIMIT='1/second:5'?

\config{PF6\_OUTPUT\_N}{PF6\_OUTPUT\_N}{PF6OUTPUTN}{
Dans cette variable vous indiquez le nombre de règles du pare-feu IPv6 pour les
paquets entrants (chaîne OUTPUT). Par défaut, deux règles sont activées~: la
première permet l'accès au routeur par tous des hôtes locaux via l'adresse du
niveau de lien et la seconde permet la communication des hôtes du premier
sous-réseau IPv6 défini avec le routeur.

Si plusieurs sous-réseaux IPv6 locaux sont définis, la seconde règle doit
être reproduite au temps de fois que nécessaire. Voir le fichier de configuration.

Exemple~:
}
\verb*?PF6_OUTPUT_N='1'?

\config{PF6\_OUTPUT\_x}{PF6\_OUTPUT\_x}{PF6OUTPUTx}{
Dans cette variable, vous indiquez la règle pour la chaîne OUTPUT du pare-feu
IPv6. Pour une description plus détaillée, voir la documentation de la variable
\var{PF\_OUTPUT\_x}.

Les différences par rapport au pare-feu IPv4~:
\begin{itemize}
\item Au lieu de \var{IP\_NET\_x} vous devez mettre \var{IPV6\_NET\_x}.
\item Au lieu de \var{IP\_ROUTE\_x} vous devez mettre \var{IPV6\_ROUTE\_x}.
\item Les adresses IPv6 doivent être placées entre deux crochets. (y compris
    le masque de sous réseau, s'il est disponible).
\item Tous les adresses IPv6 que vous indiquez (y compris les variables
    \var{IPV6\_NET\_x}, etc.) doivent être placées entre deux crochets si
    l’adresse est suivi d’un port ou d’une plage de ports.
\end{itemize}

Exemple~:
}

\begin{example}
\begin{verbatim}
PF6_OUTPUT_1='tmpl:ftp IPV6_NET_1 ACCEPT HELPER:ftp'
\end{verbatim}
\end{example}

\config{PF6\_OUTPUT\_x\_COMMENT}{PF6\_OUTPUT\_x\_COMMENT}{PF6OUTPUTxCOMMENT}{
Dans cette variable, vous pouvez indiquer une description ou un commentaire
associé à la règle OUTPUT.

Exemple~:
}
\verb*?PF6_OUTPUT_3_COMMENT='no samba traffic allowed'?

\config{PF6\_USR\_CHAIN\_N}{PF6\_USR\_CHAIN\_N}{PF6USRCHAINN}{
Dans cette variable, vous indiquez le nombre de chaînes, qui seront définies par
l'utilisateur dans la table du pare-feu IPv6. Pour une description plus
détaillée, voir la documentation de la variable \var{PF\_USR\_CHAIN\_N}.

Configuration par défaut~:
}
\verb*?PF6_USR_CHAIN_N='0'?

\config{PF6\_USR\_CHAIN\_x\_NAME}{PF6\_USR\_CHAIN\_x\_NAME}{PF6USRCHAINxNAME}{
Dans cette variable, vous indiquez le nom personnalisé de la table du pare-feu
IPv6. Pour une description plus détaillée, voir la documentation de la variable
\var{PF\_USR\_CHAIN\_x\_NAME}

Exemple~:
}
\verb*?PF6_USR_CHAIN_1_NAME='usr-myvpn'?

\config{PF6\_USR\_CHAIN\_x\_RULE\_N}{PF6\_USR\_CHAIN\_x\_RULE\_N}{PF6USRCHAINxRULEN}{
Dans cette variable, vous indiquez le nombre de règles personnalisées pour
pare-feu IPv6 associé à la table du pare-feu IPv6. Pour une description plus
détaillée, voir la documentation de la variable \var{PF\_USR\_CHAIN\_x\_RULE\_N}.

Exemple~:
}
\verb*?PF6_USR_CHAIN_1_RULE_N='0'?

\config{PF6\_USR\_CHAIN\_x\_RULE\_x}{PF6\_USR\_CHAIN\_x\_RULE\_x}{PF6USRCHAINxRULEx}{
dans cette variable, vous indiquez la règle définie par l'utilisateur de la table
du pare-feu IPv6. Pour une description plus détaillée, voir la documentation de
la variable \var{PF\_USR\_CHAIN\_x\_RULE\_x}

Les différences par rapport au pare-feu IPv4~:
\begin{itemize}
\item Au lieu de \var{IP\_NET\_x} vous devez mettre \var{IPV6\_NET\_x}.
\item Au lieu de \var{IP\_ROUTE\_x} vous devez mettre \var{IPV6\_ROUTE\_x}.
\item Les adresses IPv6 doivent être placées entre deux crochets. (y compris
    le masque de sous réseau, s'il est disponible).
\item Tous les adresses IPv6 que vous indiquez (y compris les variables
    \var{IPV6\_NET\_x}, etc.) doivent être placées entre deux crochets si
    l’adresse est suivi d’un port ou d’une plage de ports.
\end{itemize}
}

\config{PF6\_USR\_CHAIN\_x\_RULE\_x\_COMMENT}{PF6\_USR\_CHAIN\_x\_RULE\_x\_COMMENT}{PF6USRCHAINxRULExCOMMENT}{
Dans cette variable, vous pouvez indiquer une description ou un commentaire
associé à la règle.

Exemple~:
}
\verb*?PF6_USR_CHAIN_1_RULE_1_COMMENT='some user-defined rule'?

\config{PF6\_POSTROUTING\_N}{PF6\_POSTROUTING\_N}{PF6POSTROUTINGN}{
Dans cette variable vous indiquez le nombre de règles du pare-feu IPv6 pour le
masquage des paquets (chaîne POSTROUTING). Pour plus de détails, reportez-vous
à la documentation de la variable \var{PF\_POSTROUTING\_N}.

Exemple~:
}
\verb*?PF6_POSTROUTING_N='2'?

\configlabel{PF6\_POSTROUTING\_x\_COMMENT}{PF6POSTROUTINGxCOMMENT}
\config{PF6\_POSTROUTING\_x PF6\_POSTROUTING\_x\_COMMENT}{PF6\_POSTROUTING\_x}{PF6POSTROUTINGx}
\mbox{}\newline
Vous indiquez dans ces variables la liste de règles qui décrivent les paquets
IPv6 qui seront masqués par le routeur (ou transmis non masqué). Pour plus de
détails, reportez-vous à la documentation de la variable \var{PF\_POSTROUTING\_x}

\config{PF6\_PREROUTING\_N}{PF6\_PREROUTING\_N}{PF6PREROUTINGN}{
Dans cette variable, vous indiquez le nombre de règles du pare-feu IPv6 pour
transmettre les paquets vers une autre destination (chaîne PREROUTING). Pour
plus de détails, reportez-vous à la documentation de la variable \var{PF\_PREROUTING\_N}.

Exemple~:
}
\verb*?PF6_PREROUTING_N='2'?

\configlabel{PF6\_PREROUTING\_x\_COMMENT}{PF6PREROUTINGxCOMMENT}
\config{PF6\_PREROUTING\_x PF6\_PREROUTING\_x\_COMMENT}{PF6\_PREROUTING\_x}{PF6PREROUTINGx}
\mbox{}\newline
Vous indiquez dans ces variables la liste de règles qui décrivent la transmission
des paquets IPv6 du routeur vers une autre destination. Pour plus de détails,
reportez-vous à la documentation de la variable \var{PF\_PREROUTING\_x}.

\end{description}

\subsection{WebGUI}

Ce paquetage installe un menu supplémentaire dans le mini-HTTPD pour le
"filtrage de paquets (IPv6)", sous lequel vous pourrez voir les enregistrements
du filtrage de paquets de votre configuration IPv6.
