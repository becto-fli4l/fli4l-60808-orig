% Last Update: $Id$
\section{IPv6 - Unterstützung von IPv6-Tunneln durch IPv4-Netzwerke}

\subsection{Einleitung}

Dieses Paket unterstützt das Aufbauen von Tunneln zu IPv6-Anbietern. Dies ist
immer dann interessant bzw. nötig, wenn man gern IPv6-Verbindungen ins Internet
nutzen möchte, der eigene Internet-Anbieter aber keine native IPv6-Unterstützung
anbietet. Momentan erlaubt das Paket die Nutzung von so genannten 6in4-Tunneln,
wie sie etwa der Anbieter ``Hurricane Electric'' unterstützt. Andere
Technologien (AYIYA, 6to4, Teredo) werden zur Zeit nicht unterstützt.

Sie müssen für die Nutzung von Tunneln die generelle IPv6-Unterstützung mit
Hilfe der Variable \verb+OPT_IPV6='yes'+ aktivieren.

\subsection{Konfiguration}

\subsubsection{Tunnel-Konfiguration}

\wichtig{Dieser Abschnitt ist falsch, weil veraltet! Für die Nutzung von
6in4-Tunneln müssen Circuits vom Typ ``tun6in4'' konfiguriert werden! Schauen
Sie sich hierzu die Beispiel-Konfigurationen in der
\texttt{\textup{config/ipv6.txt}} an!}

Dieser Abschnitt stellt die Konfiguration von 6in4-IPv6-Tunneln vor. Ein
solcher Tunnel bietet sich an, wenn der eigene Internet-Anbieter kein IPv6
von Haus aus unterstützt. In diesem Fall wird mit einem bestimmten
Internet-Host eines Tunnel-Brokers, dem so genannten PoP (Point of Presence),
via IPv4 eine bidirektionale Verbindung aufgebaut, über die dann alle
IPv6-Pakete verpackt geroutet werden (deswegen 6 ``in'' 4, weil die
IPv6-Pakete innerhalb von IPv4-Paketen gekapselt werden).\footnote{Es handelt
sich um das IPv4-Protokoll 41, ``IPv6 encapsulation''.} Damit das funktioniert,
muss zum einen der Tunnel aufgebaut und zum anderen der Router so konfiguriert
werden, dass die IPv6-Pakete, die ins Internet sollen, auch über den Tunnel
geroutet werden. Der erste Teil wird in diesem Abschnitt konfiguriert, der
zweite Teil wird im nächsten Abschnitt beschrieben.

\begin{description}
\config{IPV6\_TUNNEL\_N}{IPV6\_TUNNEL\_N}{IPV6TUNNELN}{
Diese Variable enthält die Anzahl der aufzubauenden 6in4-Tunnel.

Beispiel:
}
\verb*?IPV6_TUNNEL_N='1'?

\config{IPV6\_TUNNEL\_x\_TYPE}{IPV6\_TUNNEL\_x\_TYPE}{IPV6TUNNELxTYPE}{
Diese Variable bestimmt den Typ des Tunnels. Momentan werden die Werte ``raw''
für ``rohe'' Tunnel, ``static'' für statische Tunnel und ``he'' für Tunnel des
Anbieters Hurricane Electric unterstützt. Mehr zu Heartbeat-Tunneln steht im
nächsten Absatz.

Beispiel:
}
\verb*?IPV6_TUNNEL_1_TYPE='he'?


\config{IPV6\_TUNNEL\_x\_DEFAULT}{IPV6\_TUNNEL\_x\_DEFAULT}{IPV6TUNNELxDEFAULT}{
Diese Variable legt fest, ob IPv6-Pakete, die nicht an das lokale bzw. die
lokalen Netze adressiert sind, über diesen Tunnel geroutet werden sollen. Es
kann nur einen solchen Tunnel geben (weil nur eine Default-Route existieren
kann). Mögliche Werte sind ``yes'' und ``no''.

\wichtig{Genau ein Tunnel sollte ein Default-Gateway für nausgehende IPv6-Daten
sein, da andernfalls eine Kommunikation mit IPv6-Hosts im Internet nicht
möglich ist! Die ausschließliche Verwendung von Nicht-Default-Tunneln ist nur
sinnvoll, wenn ausgehende IPv6-Daten über eine separat konfigurierte
Default-Route geschickt werden, die nicht mit einem Tunnel zusammenhängt. Siehe
hierzu auch die Einleitung zum Unterabschnitt ``Routen-Konfiguration'' sowie
die Beschreibung der Variable \var{IPV6\_ROUTE\_x} weiter unten.}

Standard-Konfiguration:
}
\verb*?IPV6_TUNNEL_1_DEFAULT='no'?

\config{IPV6\_TUNNEL\_x\_PREFIX}{IPV6\_TUNNEL\_x\_PREFIX}{IPV6TUNNELxPREFIX}{
Diese Variable enthält den IPv6-Subnetzpräfix des Tunnels in CIDR-Notation,
d.h. es wird sowohl eine IPv6-Adresse als auch die Länge des Präfixes angegeben.
Diese Angabe wird in der Regel vom Tunnelanbieter vorgegeben. Bei
Tunnel\-anbietern, die den Präfix beim Tunnelaufbau jedes Mal neu vergeben, ist
diese Angabe unnötig. (Momentan werden solche Anbieter aber noch nicht
unterstützt.) Diese Variable muss auch bei rohen (``raw'') Tunneln leer bleiben.

\wichtig{Diese Variable \emph{darf} leer bleiben, wenn dem Tunnel noch kein
Subnetz-Präfix zugewiesen worden ist. Allerdings kann dieser Tunnel dann nicht
einem IPv6-Subnetz (\var{IPV6\_NET\_x}) zugeordnet werden, weil die
IPv6-Adressen im Subnetz nicht berechnet werden können. Sinnvoll ist eine
solche Konfiguration also nur übergangsweise, etwa wenn der Tunnel einige Zeit
aktiv sein muss, bevor der Tunnelanbieter einem ein Subnetz-Präfix zuweist.}

Beispiele:
}

\begin{example}
\begin{verbatim}
IPV6_TUNNEL_1_PREFIX='2001:db8:1743::/48'      # /48-Subnetz
IPV6_TUNNEL_2_PREFIX='2001:db8:1743:5e00::/56' # /56-Subnetz
\end{verbatim}
\end{example}

\config{IPV6\_TUNNEL\_x\_LOCALV4}{IPV6\_TUNNEL\_x\_LOCALV4}{IPV6TUNNELxLOCALV4}{
Diese Variable enthält die lokale IPv4-Adresse des Tunnels oder den Wert
`dynamic', wenn die dynamisch zugewiesene IPv4-Adresse des aktiven WAN-Circuits
verwendet werden soll. Letzteres ist nur sinnvoll, wenn es sich um einen
Heartbeat-Tunnel handelt (siehe \var{IPV6\_TUNNEL\_x\_TYPE} weiter unten).

Beispiele:
}

\begin{example}
\begin{verbatim}
IPV6_TUNNEL_1_LOCALV4='172.16.0.2'
IPV6_TUNNEL_2_LOCALV4='dynamic'
\end{verbatim}
\end{example}

\config{IPV6\_TUNNEL\_x\_REMOTEV4}{IPV6\_TUNNEL\_x\_REMOTEV4}{IPV6TUNNELxREMOTEV4}{
Diese Variable enthält die entfernte IPv4-Adresse des Tunnels. Diese Angabe
wird in der Regel vom Tunnel-Anbieter vorgegeben.

Beispiel (entspricht dem PoP deham01 von Easynet):
}

\begin{example}
\begin{verbatim}
IPV6_TUNNEL_1_REMOTEV4='212.224.0.188'
\end{verbatim}
\end{example}

\wichtig{Wenn \var{PF\_INPUT\_ACCEPT\_DEF} auf ``no'' steht, d.h. wenn die
IPv4-Firewall manuell konfiguriert wird, dann wird eine Regel benötigt, die
alle IPv6-in-IPv4-Pakete (IP-Protokoll 41) vom Tunnelendpunkt akzeptiert. Für
den o.g. Tunnelendpunkt sähe die entsprechende Regel wie folgt aus:}

\begin{example}
\begin{verbatim}
PF_INPUT_x='prot:41 212.224.0.188 ACCEPT'
\end{verbatim}
\end{example}

\config{IPV6\_TUNNEL\_x\_LOCALV6}{IPV6\_TUNNEL\_x\_LOCALV6}{IPV6TUNNELxLOCALV6}{
Diese Variable legt die lokale IPv6-Adresse des Tunnels inklusive verwendeter
Netzmaske in CIDR-Notation fest. Diese Angabe wird vom Tunnelanbieter
vorgegeben. Bei Tunnelanbietern, welche die Tunnelendpunkte beim Tunnelaufbau
jedes Mal neu vergeben, ist diese Angabe unnötig. (Momentan werden solche
Anbieter aber noch nicht unterstützt.)

Beispiel:
}
\verb*?IPV6_TUNNEL_1_LOCALV6='2001:db8:1743::2/112'?

\config{IPV6\_TUNNEL\_x\_REMOTEV6}{IPV6\_TUNNEL\_x\_REMOTEV6}{IPV6TUNNELxREMOTEV6}{
Diese Variable legt die entfernte IPv6-Adresse des Tunnels fest. Diese Angabe
wird vom Tunnelanbieter vorgegeben. Eine Netzmaske wird nicht benötigt, da sie
der Variable \var{IPV6\_TUNNEL\_x\_LOCALV6} entnommen wird. Bei
Tunnelanbietern, welche die Tunnelendpunkte beim Tunnelaufbau jedes Mal neu
vergeben, ist diese Angabe unnötig. (Momentan werden solche Anbieter aber noch
nicht unterstützt.)

Beispiel:
}
\verb*?IPV6_TUNNEL_1_REMOTEV6='2001:db8:1743::1'?

\config{IPV6\_TUNNEL\_x\_DEV}{IPV6\_TUNNEL\_x\_DEV}{IPV6TUNNELxDEV}{
(optional) Diese Variable enthält den Namen der zu erstellenden
Tunnel-Netzwerkschnittstelle. Verschiedene Tunnel müssen unterschiedlich
benannt werden, damit alles funktioniert. Falls die Variable nicht definiert
ist, wird ein Tunnelname automatisch generiert (``v6tun'' + Tunnelindex).

Beispiel:
}
\verb*?IPV6_TUNNEL_1_DEV='6in4'?

\config{IPV6\_TUNNEL\_x\_MTU}{IPV6\_TUNNEL\_x\_MTU}{IPV6TUNNELxMTU}{
(optional) Diese Variable enthält die Größe der MTU (Maximum Transfer Unit) in
Bytes, d.h. des größten Pakets, das noch getunnelt werden kann. Diese Angabe
wird in der Regel vom Tunnelanbieter vorgegeben. Die Standard-Einstellung,
falls nichts angegeben wird, lautet ``1280'' und sollte mit allen Tunneln
funktionieren.

Standard-Konfiguration:
}
\verb*?IPV6_TUNNEL_1_MTU='1280'?

\end{description}

Einige Tunnelanbieter verlangen, dass über den Tunnel permanent ein
Lebenszeichen vom Router an den Anbieter gesandt wird, um zu verhindern, dass
ein Host einen Tunnel in Anspruch nimmt, obwohl er ihn nicht nutzt. Dazu wird
ein so genanntes Heartbeat-Protokoll (dt. ``Herzschlag'') verwendet. Zusätzlich
verlangen Anbieter in der Regel eine erfolgreiche Anmeldung mit Benutzernamen
und Passwort, um Missbrauch zu vermeiden. Soll ein solcher Heartbeat-Tunnel
genutzt werden, dann müssen entsprechende Angaben gemacht werden, die im Folgenden
beschrieben werden.

\begin{description}
\config{IPV6\_TUNNEL\_x\_USERID}{IPV6\_TUNNEL\_x\_USERID}{IPV6TUNNELxUSERID}{
Diese Variable enthält den Namen des Benutzers, der beim Tunnel-Login
erforderlich ist.

Beispiel:
}
\verb*?IPV6_TUNNEL_1_USERID='USERID'?

\config{IPV6\_TUNNEL\_x\_PASSWORD}{IPV6\_TUNNEL\_x\_PASSWORD}{IPV6TUNNELxPASSWORD}{
Diese Variable enthält das Passwort für den oben angegebenen Benutzernamen. Es
darf keine Leerzeichen enthalten.

Beispiel:
}
\verb*?IPV6_TUNNEL_1_PASSWORD='passwort'?

\config{IPV6\_TUNNEL\_x\_TUNNELID}{IPV6\_TUNNEL\_x\_TUNNELID}{IPV6TUNNELxTUNNELID}{
Diese Variable enthält den Identifikator des Tunnels.

Beispiel:
}
\verb*?IPV6_TUNNEL_1_TUNNELID='TunnelID'?

\config{IPV6\_TUNNEL\_x\_TIMEOUT}{IPV6\_TUNNEL\_x\_TIMEOUT}{IPV6TUNNELxTIMEOUT}{
(optional) Diese Variable enthält die Zeitspanne in Sekunden, die beim
Tunnelaufbau maximal gewartet wird. Der Standard-Wert ist abhängig vom
eingestellten Tunnelanbieter.

Beispiel:
}
\verb*?IPV6_TUNNEL_1_TIMEOUT='30'?
\end{description}
