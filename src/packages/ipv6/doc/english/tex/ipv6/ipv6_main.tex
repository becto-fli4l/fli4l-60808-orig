% Synchronized to r48680
\section{IPv6 - Internet Protocol Version 6}

\subsection{Introduction}

This package enables fli4l to support IPv6 in many ways. This includes 
informations about the IPv6 address of the router's IPv6 (sub) networks managed 
by it, predefined IPv6 routes and firewall rules regarding IPv6 packets. 
Last but no least it is possible to automatically establish tunnels 
to IPv6 providers. This currently works only with so-called 6in4 tunneling 
as supported by the provider ``Hurricane Electric''. Other technologies (AYIYA, 
6to4, Teredo) are not supported at the moment.

IPv6 is the successor of the internet protocol IPv4. It was developed mainly 
to enlarge the relatively small amount of unique internet addresses: While 
IPv4 supports approximately \(2^{32}\) addresses, \footnote{only nearly because 
some addresses are used for specialized purposes, such as for broadcast and
multicasting} in IPv6 \(2^{128}\) are possible. Each communicating IPv6 host 
can thus have a unique address and no longer has to rely on techniques such 
as NAT, PAT, masquerading a.s.o.

Besides this aspect topics such as self-configuration and safety concerns also 
played a role in the development of the IPv6 protocol. This will be carried out 
in later sections.

The biggest problem with IPv6 is its spread: Currently IPv6~--compared with 
IPv4~-- is rarely used. The reason is that IPv6 and IPv4 are technically 
incompatible and thus all software and hardware components involved in packet 
forwarding in the internet have to be retrofitted for IPv6. Certain services 
such as DNS (Domain Name System) must be extended in accordance with IPv6.

This is a vicious circle: The low spreading of IPv6 with service providers 
on the internet leads to disinterest on the part of router manufacturers to 
equip their devices with IPv6 functionality, which in turn means that service 
providers fear the transition to IPv6 because they fear that it's not worth 
the effort. Only slowly the mood is changing in favour of IPv6 mostly because 
of the increasing pressure from shortness of IPv4 address supplies. \footnote{The 
last IPv4 address blocks have been assigned by the IANA by now.}

\subsection{Address Format}

An IPv6 address consists of eight two-byte values listed 
in hexadecimal notation:

\emph{Example 1:} \verb*?2001:db8:900:551:0:0:0:2?

\emph{Example 2:} \verb*?0:0:0:0:0:0:0:1? (IPv6-loopback-address)

To make the addresses a little clearer, successive zeros are merged by removing 
them, and only two successive colons remain. The above addresses may therefore
be rewritten as:

\emph{Example 1 (compact):} \verb*?2001:db8:900:551::2?

\emph{Example 2 (compact):} \verb*?::1?

Such a reduction is only allowed once to avoid ambiguities. The address 
\verb*?2001:0:0:1:2:0:0:3? can thus either be shortened to \verb*?2001::1:2:0:0:3? 
or \verb*?2001:0:0:1:2::3? but not to \verb*?2001::1:2::3? because then it would 
be unclear how the four zeros belong to the contracted regions.

Another ambiguity exists when an IPv6 address is to be combined with a port
(TCP or UDP): In this case you can not add the port immediately followed by 
a colon and value because the colon is already used within the address and 
it is therefore unclear in some cases whether the port specification is 
perhaps an address component. In such cases the IPv6 address is enclosed 
in square brackets. This is the syntax also required in URLs (for example if
a numerical IPv6 address should be used in the web browser).

\emph{Example 3:} \verb*?[2001:db8:900:551::2]:1234? 

Without the use of braces the following address is created
\verb*?2001:db8:900:551::2:1234?, which without shortening equals to the address
\verb*?2001:db8:900:551:0:0:2:1234? and thus has no port appendend.

\subsection{Configuration}

\subsubsection{General Settings}

The general settings include at first the activation of IPv6 support and then  
the optional assignment of an IPv6 address to the router.

\begin{description}
\config{OPT\_IPV6}{OPT\_IPV6}{OPTIPV6}{
Activates IPv6 support.

Default setting:
}
\verb*?OPT_IPV6='no'?

\config{HOSTNAME\_IP6}{HOSTNAME\_IP6}{HOSTNAMEIP6}{(optional)
This variable sets the IPv6-address of the routers explicitely. If the variable is 
not set the IPv6-address will be set to the first configured IPv6-subnet
(\var{IPV6\_NET\_x}, see below).

Example:
}
\verb*?HOSTNAME_IP6='IPV6_NET_1_IPADDR'?
\end{description}

\subsubsection{Subnet Configuration}

This section describes the configuration of one or more IPv6 subnets. 
An IPv6 subnet is an IPv6 address space specified by a so-called prefix
and is bound to a certain network interface. Further settings include the
announcing of the prefix and of the DNS service within the subnet
and optional a router name within that subnet.

\begin{description}
\config{IPV6\_NET\_N}{IPV6\_NET\_N}{IPV6NETN}{
This variable contains the number of used IPv6 subnets. At least one
IPv6 subnet should be defined in order to use IPv6 in the local network.

Default setting:
}
\verb*?IPV6_NET_N='0'?

\config{IPV6\_NET\_x}{IPV6\_NET\_x}{IPV6NETx}{
This variable contains the IPv6 address of the router and the size of the 
network mask in CIDR notation for a specific IPv6 subnet. If the subnet is 
to be routed publicly it usually is obtained from the Internet or tunnel 
provider.

\wichtig{If the stateless auto-configuration should be activated in the subnet
then the length of the sub\-net prefix has to be 64 bits!  
(see the section on \var{IPV6\_NET\_x\_ADVERTISE} below)}

\wichtig{If the subnet is connected to a tunnel (see \var{IPV6\_NET\_x\_TUNNEL} 
below) then only the part of the router address is specified here that is 
\emph{not} assigned to the tunnel's subnet prefix (to be found in 
\var{IPV6\_TUNNEL\_x\_PREFIX}) because that prefix and the address 
are combined! In previous versions of the IPv6 package the variable 
\var{IPV6\_TUNNEL\_x\_PREFIX} did not exist and subnet prefix and 
router address were combined together in \var{IPV6\_NET\_x}. 
That does not work if the subnet prefix associated by the tunnel 
provider is first handed out when instantiating the dynamic tunnel.
In addition the length of the subnet prefix (in this case /48) is not 
known and certain predefined routes can not be set properly. This leads 
to strange effects while routing to some specific destinations.
The configuration was separated to avoid such effects.}

Examples:
}

\begin{example}
\begin{verbatim}
IPV6_NET_1='2001:db8:1743:42::1/64'       # without tunnel: complete address
IPV6_NET_1_TUNNEL=''

IPV6_NET_2='0:0:0:42::1/64'               # with tunnel: partial address
IPV6_NET_2_TUNNEL='1'
IPV6_TUNNEL_1_PREFIX='2001:db8:1743::/48' # see "Tunnel Configuration"
\end{verbatim}
\end{example}

\config{IPV6\_NET\_x\_DEV}{IPV6\_NET\_x\_DEV}{IPV6NETxDEV}{This 
variable contains for a specific IPv6 subnet the name of the network 
interface to which the IPv6 network is bound. This does \emph{not}
collide with the network interfaces assigned in the base configuration 
(\texttt{base.txt}) because a network interface can have both IPv4 
and IPv6 addresses assigned.

Example:
}
\verb*?IPV6_NET_1_DEV='eth0'?

\config{IPV6\_NET\_x\_TUNNEL}{IPV6\_NET\_x\_TUNNEL}{IPV6NETxTUNNEL}{
This variable contains the index of the associated tunnel for a specific 
IPv6 subnet. The prefix of the tunnel is combined with the router address 
to get the complete IPv6 address of the router. If the variable is empty 
or undefined then no tunnel belongs to the subnet and in \var{IPV6\_NET\_x} 
the full IPv6 address of the router including network mask has to be 
specified (see above).

A tunnel can be assigned to multiple subnets because a tunnel subnet is 
usually large enough to be split in multiple subnets (/56 or greater). 
It is not possible to assign multiple tunnel subnet prefixes to a subnet 
because the address of the subnet would be ambiguous in this case.

Example:
}
\verb*?IPV6_NET_1_TUNNEL='1'?

\config{IPV6\_NET\_x\_ADVERTISE}{IPV6\_NET\_x\_ADVERTISE}{IPV6NETxADVERTISE}{
This variable determines whether the chosen subnet prefix is advertised on the
LAN via ``Router Advertisements''. This is used for the so-called ``Stateless
auto-configuration'' and should not be confused with DHCPv6. Possible values
are ``yes'' and ``no''.

It is recommended to enable this setting, unless all addresses in the network 
are statically assigned or another router is already responsible to advertise
the subnet prefix.

\wichtig{Automatic distribution of the subnet will only work if the subnet is 
a /64 network, i.e. if the length of the subnet prefix is 64 bits! The reason 
for this is that the other hosts in the network compute their IPv6 address 
from the prefix and their host MAC addresses which will not work if the host 
part is not 64 bits. If the self-configuration fails the subnet prefix should 
be checked for incorrect length (for example as /48).}

Default setting:
}
\verb*?IPV6_NET_1_ADVERTISE='yes'?

\config{IPV6\_NET\_x\_ADVERTISE\_DNS}{IPV6\_NET\_x\_ADVERTISE\_DNS}{IPV6NETxADVERTISEDNS}{
This variable determines whether the local DNS service in IPv6 subnets
should be advertised by ``Router Advertisements''. This will only works if 
the IPv6 functionality of the DNS service is activated using 
\var{DNS\_SUPPORT\_IPV6}='yes'. Possible values are ``yes'' and ``no''.

Default setting:
}
\verb*?IPV6_NET_1_ADVERTISE_DNS='no'?

\config{IPV6\_NET\_x\_NAME}{IPV6\_NET\_x\_NAME}{IPV6NETxNAME}{(optional)
This variable specifies an interface-specific hostname for the router in
this IPv6-subnet.

Example:
}
\verb*?IPV6_NET_1_NAME='fli4l-subnet1'?

\end{description}

\subsubsection{Tunnel Configuration}

This section introduces the configuration of 6in4 IPv6 tunnels. Such a 
tunnel is useful in case that your own ISP does no support IPv6 natively. 
With an internet host of a tunnel broker known as a PoP (Point of Presence)
a bi-directional link via IPv4 can be established. All IPv6 packets will 
be routed encapsulated (therefore 6 ``in'' 4 because of the IPv6 packets 
are encapsulated in IPv4 packets). \footnote {This is known as IPv4 protocol 41,
 ``IPv6 encapsulation''.} First the tunnel will be established and in a second 
step the router is configured in a way that the IPv6 packets to the Internet 
are routed through the tunnel. The first part is covered in this section and 
the second part is described in the next section.

\begin{description}
\config{IPV6\_TUNNEL\_N}{IPV6\_TUNNEL\_N}{IPV6TUNNELN}{
This variable contains the number of 6in4 tunnels to be established.

Example:
}
\verb*?IPV6_TUNNEL_N='1'?

\config{IPV6\_TUNNEL\_x\_TYPE}{IPV6\_TUNNEL\_x\_TYPE}{IPV6TUNNELxTYPE}{
This variable determines the type of the tunnel. Currently, the values ``raw'', 
``static'', and ``he'' for tunnels of provider Hurricane Electric are supported. 
More about heartbeat-tunnels in the next paragraph.

Example:
}
\verb*?IPV6_TUNNEL_1_TYPE='he'?


\config{IPV6\_TUNNEL\_x\_DEFAULT}{IPV6\_TUNNEL\_x\_DEFAULT}{IPV6TUNNELxDEFAULT}{
This variable determines whether IPv6 packets which are not addressed to local 
networks are allowed to be routed through this tunnel. Only one tunnel can do 
this (because there is only one default route). Possible values are ``yes'' and ``no''.

\wichtig{Exactly one tunnel should be a default gateway for outbound IPv6 
otherwise no communication with IPv6 hosts on the Internet is possible! 
The use of only non-default tunnels is only useful if outgoing IPv6 traffic 
is sent on a separately configured default route, which is not related to a tunnel. 
See the introduction to the subsection ``Route Configuration'' and the description 
of the variable \var{IPV6\_ROUTE\_x} below.}

Default setting:
}
\verb*?IPV6_TUNNEL_1_DEFAULT='no'?

\config{IPV6\_TUNNEL\_x\_PREFIX}{IPV6\_TUNNEL\_x\_PREFIX}{IPV6TUNNELxPREFIX}{
This variable contains the IPv6-subnet-prefix of the tunnel in CIDR-Notation which
means an IPv6-address is provided as well as the length of the prefix. This 
settings are usually given to you by the tunnel provider. This setting is not 
necessary for tunnel providers giving a new prefix each time a tunnel is established. 
(Such providers are not supported at the moment.) This variable has to be empty with 
``raw'' tunnels as well.

\wichtig{This variable \emph{may be} empty if the tunnel has no subnet-prefix 
assigned. But in turn the tunnel can not be assigned to a IPv6-subnet (\var{IPV6\_NET\_x}) 
because IPv6-addresses in the subnet ca not be computed. Such a configuration makes 
only sense on an interim basis if the tunnel has to be active for some time before 
the tunnel provider assigns a subnet-prefix.}

Examples:
}

\begin{example}
\begin{verbatim}
IPV6_TUNNEL_1_PREFIX='2001:db8:1743::/48'      # /48-subnet
IPV6_TUNNEL_2_PREFIX='2001:db8:1743:5e00::/56' # /56-subnet
\end{verbatim}
\end{example}

\config{IPV6\_TUNNEL\_x\_LOCALV4}{IPV6\_TUNNEL\_x\_LOCALV4}{IPV6TUNNELxLOCALV4}{
This variable contains the local IPv4-address of the tunnel or the value `dynamic' 
if the dynamic IPv4-address of the active WAN-Circuits should be used. The latter 
makes only sense for a heartbeat-tunnel (see \var{IPV6\_TUNNEL\_x\_TYPE} below).

Examples:
}

\begin{example}
\begin{verbatim}
IPV6_TUNNEL_1_LOCALV4='172.16.0.2'
IPV6_TUNNEL_2_LOCALV4='dynamic'
\end{verbatim}
\end{example}

\config{IPV6\_TUNNEL\_x\_REMOTEV4}{IPV6\_TUNNEL\_x\_REMOTEV4}{IPV6TUNNELxREMOTEV4}{
This variable contains the remote IPv4-address of the tunnel. Usually this 
value is given to you by the tunnel provider.

Example (as used by PoP deham01 by Easynet):
}

\begin{example}
\begin{verbatim}
IPV6_TUNNEL_1_REMOTEV4='212.224.0.188'
\end{verbatim}
\end{example}

\wichtig{If \var{PF\_INPUT\_ACCEPT\_DEF} is set to ``no'' (the IPv4-firewall is 
configured  manually) you will need a firewall rule to accept all IPv6-in-IPv4 
packets (IP-Protokoll 41) from the tunnel endpoint. For the tunnel endpoint above 
the rule would be like this:}

\begin{example}
\begin{verbatim}
PF_INPUT_x='prot:41 212.224.0.188 ACCEPT'
\end{verbatim}
\end{example}

\config{IPV6\_TUNNEL\_x\_LOCALV6}{IPV6\_TUNNEL\_x\_LOCALV6}{IPV6TUNNELxLOCALV6}{
This variable sets the local IPv6 address of the tunnel including the used
netmask in CIDR notation. This information is predetermined from the tunnel 
provider. This information is unnecessary for tunnel providers which re-assign 
the tunnel endpoints for establishing the tunnel new each time (Such Providers 
are not yet supported).

Example:
}
\verb*?IPV6_TUNNEL_1_LOCALV6='2001:db8:1743::2/112'?

\config{IPV6\_TUNNEL\_x\_REMOTEV6}{IPV6\_TUNNEL\_x\_REMOTEV6}{IPV6TUNNELxREMOTEV6}{
This variable specifies the remote IPv6 address of the tunnel. This information
is set by the tunnel provider. A netmask is not needed because it is taken from the 
variable \var{IPV6\_TUNNEL\_x\_LOCALV6}. This information is unnecessary for tunnel 
providers assigning new tunnel endpoints for every tunnel established (Such providers 
are not yet supported).

Example:
}
\verb*?IPV6_TUNNEL_1_REMOTEV6='2001:db8:1743::1'?

\config{IPV6\_TUNNEL\_x\_DEV}{IPV6\_TUNNEL\_x\_DEV}{IPV6TUNNELxDEV}{
(optional) This variable contains the name of the network interface of the tunnel 
to be created. Different tunnels have to be named differently to make everything work. 
If the variable is not defined a tunnel name is generated automatically (``v6tun'' 
+ tunnel index).

Example:
}
\verb*?IPV6_TUNNEL_1_DEV='6in4'?

\config{IPV6\_TUNNEL\_x\_MTU}{IPV6\_TUNNEL\_x\_MTU}{IPV6TUNNELxMTU}{
(optional) This variable contains the size of the MTU (Maximum Transfer Unit)
in bytes, i.e. the the size of the largest packet that can still be tunneled. 
This information is generally predetermined by the tunnel provider. If nothing is 
specified the standard setting is ``1280'' and should work with all tunnels.

Default setting:
}
\verb*?IPV6_TUNNEL_1_MTU='1280'?

\end{description}

To prevent a host blocking a tunnel although it does not need it at all some providers 
require permanently sending packets over the tunnel to the provider to prove it is still 
``alive''. For this purpose a so-called heartbeat protocol is used. Providers usually 
require a successful login with user name and password in order to avoid abuse. If 
such a heartbeat tunnel is used then appropriate information has to be passed which will 
be described below.

\begin{description}
\config{IPV6\_TUNNEL\_x\_USERID}{IPV6\_TUNNEL\_x\_USERID}{IPV6TUNNELxUSERID}{
This variable holds the username needed for the tunnel login.

Example:
}
\verb*?IPV6_TUNNEL_1_USERID='USERID'?

\config{IPV6\_TUNNEL\_x\_PASSWORD}{IPV6\_TUNNEL\_x\_PASSWORD}{IPV6TUNNELxPASSWORD}{
This variable contains the password for the username above. It can't contain spaces.

Example:
}
\verb*?IPV6_TUNNEL_1_PASSWORD='passwort'?

\config{IPV6\_TUNNEL\_x\_TUNNELID}{IPV6\_TUNNEL\_x\_TUNNELID}{IPV6TUNNELxTUNNELID}{
This variable contains the identificator for the tunnel.

Example:
}
\verb*?IPV6_TUNNEL_1_TUNNELID='TunnelID'?

\config{IPV6\_TUNNEL\_x\_TIMEOUT}{IPV6\_TUNNEL\_x\_TIMEOUT}{IPV6TUNNELxTIMEOUT}{
(optional) This variable contains the maximum waiting time in seconds for the tunnel 
to establish. The default setting depends on the tunnel provider.

Example:
}
\verb*?IPV6_TUNNEL_1_TIMEOUT='30'?
\end{description}

\subsubsection{Route Configuration}

Routes are paths for IPv6 packets. To know the direction in which it should send an 
incoming packet the router does rely on a routing table in which this information can be 
found. In the case of IPv6 it is important to know where IPv6 packets have to be sent that 
are not bound to the local network. A default route is automatically configured that sends 
all packets to the other end of an IPv6 tunnel if \var{IPV6\_TUNNEL\_x\_DEFAULT} is set for 
the relevant tunnel. Other routes can be configured here (i.e. to interconnect parallel 
IPv6 subnets).

\begin{description}
\config{IPV6\_ROUTE\_N}{IPV6\_ROUTE\_N}{IPV6ROUTEN}{
This variable sets the number of IPv6 routes to define. Usually additional IPv6 routes 
are not needed.

Default setting:
}
\verb*?IPV6_ROUTE_N='0'?

\config{IPV6\_ROUTE\_x}{IPV6\_ROUTE\_x}{IPV6ROUTEx}{
This variable holds the route in form of `target-net gateway'. The target net has to be 
specified in CIDR-notation. For the default route the target net has to be \var{::/0}. 
It is not necessary to configure default routes here that cross a tunnel (see 
introduction on this paragraph).

Example:
}
\verb*?IPV6_ROUTE_1='2001:db8:1743:44::/64 2001:db8:1743:44::1'?
\end{description}

\subsubsection{IPv6 Firewall}

As for IPv4 a firewall is needed for IPv6 networks in order to avoid that everyone 
can reach each machine in the local network from outside. This is even more 
important because every computer in IPv6 normally has a globally unique address 
which can be permanently assigned to the machine since it is computed from the 
MAC address of the network card. \footnote {An exception exists if the LAN hosts 
have activated so-called ``Privacy Extensions'' because then a part of the IPv6 
address is generated randomly. These addresses are by definition not known 
outside and thus only partially or not at all relevant for the firewall configuration.} 
Therefore the firewall blocks all requests from outside at first and can then 
be opened by corresponding entries in this section bit by bit as needed.

The configuration of an IPv6 firewall corresponds widely to that of an IPv4 
firewall. Special features and differences will be explained separately.

\begin{description}
\config{PF6\_LOG\_LEVEL}{PF6\_LOG\_LEVEL}{PF6LOGLEVEL} For all chains following
the protocol setting in \var{PF6\_LOG\_LEVEL} is active. Possible values are:
debug, info, notice, warning, err, crit, alert, emerg.

\config{PF6\_INPUT\_POLICY}{PF6\_INPUT\_POLICY}{PF6INPUTPOLICY}{
This variable sets the default strategy for all incoming packets for the router 
(INPUT-Chain). Possible values are ``REJECT'' (default: rejects all packets), 
``DROP'' (discards all packets without further notice) and ``ACCEPT'' (accepts 
all packets). For a detailed description see the documentation of \var{PF\_INPUT\_POLICY}.

Default setting:
}
\verb*?PF6_INPUT_POLICY='REJECT'?

\config{PF6\_INPUT\_ACCEPT\_DEF}{PF6\_INPUT\_ACCEPT\_DEF}{PF6INPUTACCEPTDEF}{
This variable activates the predefined rules for the INPUT-chain of the 
IPv6-Firewall. Possible values are ``yes'' and ``no''.

The default rules open the firewall for incoming ICMPv6 pings (one ping per 
second as a limit) as well as for NDP packets (Neighbor Discovery Procotol) 
needed for stateless auto-configuration of IPv6 networks. Connections from 
localhost and response packets to locally initiated connections are also allowed. 
Finally the IPv4 firewall is adapted so that for each tunnel IPv6-in-IPv4 
encapsulated packets are accepted by the end of the tunnel.

Default setting:
}
\verb*?PF6_INPUT_ACCEPT_DEF='yes'?

\config{PF6\_INPUT\_LOG}{PF6\_INPUT\_LOG}{PF6INPUTLOG}{
This variable activates logging of all rejected incoming packets. Possible 
values are ``yes'' and ``no''. For a detailed description see 
documentation of \var{PF\_INPUT\_LOG}.

Default setting:
}
\verb*?PF6_INPUT_LOG='no'?

\config{PF6\_INPUT\_LOG\_LIMIT}{PF6\_INPUT\_LOG\_LIMIT}{PF6INPUTLOGLIMIT}{
This variable configures the log-limit of the INPUT-chains of the IPv6-firewall 
to keep logfiles readable. For a detailed description see 
documentation of \var{PF\_INPUT\_LOG\_LIMIT}.

Default setting:
}
\verb*?PF6_INPUT_LOG_LIMIT='3/minute:5'?

\config{PF6\_INPUT\_REJ\_LIMIT}{PF6\_INPUT\_REJ\_LIMIT}{PF6INPUTREJLIMIT}{
This variable sets the limit for rejection of incoming TCP-packets. If such a 
packet exceeds this limit the packet will be dropped silently (DROP). For a 
detailed description see the documentation of \var{PF\_INPUT\_REJ\_LIMIT}.

Default setting:
}
\verb*?PF6_INPUT_REJ_LIMIT='1/second:5'?

\config{PF6\_INPUT\_UDP\_REJ\_LIMIT}{PF6\_INPUT\_UDP\_REJ\_LIMIT}{PF6INPUTUDPREJLIMIT}{
This variable sets the limit for rejection of incoming UDP-packets. If such a 
packet exceeds this limit the packet will be dropped silently (DROP). 
For a detailed description see the documentation of \var{PF\_INPUT\_UDP\_REJ\_LIMIT}.

Default setting:
}
\verb*?PF6_INPUT_UDP_REJ_LIMIT='1/second:5'?

\config{PF6\_INPUT\_ICMP\_ECHO\_REQ\_LIMIT}{PF6\_INPUT\_ICMP\_ECHO\_REQ\_LIMIT}{PFI6NPUTICMPECHOREQLIMIT}{
Defines how often the router should react to ICMPv6-echo-queries. The frequency is 
written as `n/time period' with bursts i.E. '3/minute:5' (in analogy to the 
limit-restriction). The packet will be ignored (DROP) if the limit is reached. If empty 
the default setting '1/second:5' will be used, if set to 'none' no limitations are 
accomplished.

Default setting:
}
\verb*?PF6_INPUT_ICMP_ECHO_REQ_LIMIT='1/second:5'?

\config{PF6\_INPUT\_ICMP\_ECHO\_REQ\_SIZE}{PF6\_INPUT\_ICMP\_ECHO\_REQ\_SIZE}{PF6INPUTICMPECHOREQSIZE}{
Defines the maximum size of a revceived ICMPv6-echo-request (in Bytes). Beside the 
data the packet-header size has to be considered. The default value is 150 Bytes.

Default setting:
}
\verb*?PF6_INPUT_ICMP_ECHO_REQ_SIZE='150'?

\config{PF6\_INPUT\_N}{PF6\_INPUT\_N}{PF6INPUTN}{
This variable contains the number of IPv6-firewall rules for incoming packets 
(INPUT-Chain). Per default two rules are activated: the first allows all local 
hosts to access the router on so-called link-level addresses, the second allows 
hosts from the first defined IPv6-subnet to access the router.

In case of multiple local IPv6-subnets defined the second rule has to be cloned 
respectively. See the configuration file for details.

Example:
}
\verb*?PF6_INPUT_N='2'?

\config{PF6\_INPUT\_x}{PF6\_INPUT\_x}{PF6INPUTx}{
This variable specifies a rule for the INPUT-chain of the der IPv6-firewall.
For a detailed description see the documentation of 
\var{PF\_INPUT\_x}.

Differences regarding the IPv4-firewall:
\begin{itemize}
\item \var{IPV6\_NET\_x} has to be used instead of \var{IP\_NET\_x}.
\item \var{IPV6\_ROUTE\_x} has to be used instead of \var{IP\_ROUTE\_x}.
\item IPv6-addresses must be enclosed in square brackets (including
    the network mask, if present).
\item All IPv6 address strings (including \var{IP\_NET\_x} etc.) must
    be enclosed in square brackets if a port or a port range follows.
\end{itemize}

Examples:
}

\begin{example}
\begin{verbatim}
PF6_INPUT_1='[fe80::0/10] ACCEPT'
PF6_INPUT_2='IPV6_NET_1 ACCEPT'
PF6_INPUT_3='tmpl:samba DROP NOLOG'
\end{verbatim}
\end{example}

\config{PF6\_INPUT\_x\_COMMENT}{PF6\_INPUT\_x\_COMMENT}{PF6INPUTxCOMMENT}{
This variable holds a description or a comment for the input rule it belongs to.

Example:
}
\verb*?PF6_INPUT_3_COMMENT='no samba traffic allowed'?

\config{PF6\_FORWARD\_POLICY}{PF6\_FORWARD\_POLICY}{PF6FORWARDPOLICY}{
This variable sets the default policy for packets to be forwarded by the router 
(FORWARD-Chain). Possible values are ``REJECT'' (default, rejects 
all packets), ``DROP'' (ignores all packets without further notice) and ``ACCEPT''
(accepts all packets). For a detailed description see the documentation 
of \var{PF\_FORWARD\_POLICY}.

Default setting:
}
\verb*?PF6_FORWARD_POLICY='REJECT'?

\config{PF6\_FORWARD\_ACCEPT\_DEF}{PF6\_FORWARD\_ACCEPT\_DEF}{PF6FORWARDACCEPTDEF}{
This variable activates the predefined rules for the\\ FORWARD-chain of the 
IPv6-firewall. Possible values are ``yes'' and ``no''.

The predefined rules open the firewall for outgoing ICMPv6-pings
(one ping per second as a limit). Response packets to already allowed 
connections will also be allowed.

Default setting:
}
\verb*?PF6_FORWARD_ACCEPT_DEF='yes'?

\config{PF6\_FORWARD\_LOG}{PF6\_FORWARD\_LOG}{PF6FORWARDLOG}{
This variable activates logging of all rejected forwarding packets. Possible 
values are ``yes'' and ``no''. For a detailed description see the documentation 
of \var{PF\_FORWARD\_LOG}.

Default setting:
}
\verb*?PF6_FORWARD_LOG='no'?

\config{PF6\_FORWARD\_LOG\_LIMIT}{PF6\_FORWARD\_LOG\_LIMIT}{PF6FORWARDLOGLIMIT}{
This variable configures the log limit for the FORWARD-chain of the 
IPv6-firewall to keep it readable. For a detailed description see the
documentation of \var{PF\_FORWARD\_LOG\_LIMIT}.

Default setting:
}
\verb*?PF6_FORWARD_LOG_LIMIT='3/minute:5'?

\config{PF6\_FORWARD\_REJ\_LIMIT}{PF6\_FORWARD\_REJ\_LIMIT}{PF6FORWARDREJLIMIT}{
This variable sets the limit for the rejection of forwarding TCP-packets. 
If a packet exceeds this limit it will be dropped without further notice 
(DROP). For a detailed description see the documentation 
of \var{PF\_FORWARD\_REJ\_LIMIT}.

Default setting:
}
\verb*?PF6_FORWARD_REJ_LIMIT='1/second:5'?

\config{PF6\_FORWARD\_UDP\_REJ\_LIMIT}{PF6\_FORWARD\_UDP\_REJ\_LIMIT}{PF6FORWARDUDPREJLIMIT}{
This variable sets the limit for the rejection of forwarding UDP-packets. If a 
packet exceeds this limit it will be dropped without further notice 
(DROP). For a detailed description see the documentation 
of \var{PF\_FORWARD\_UDP\_REJ\_LIMIT}.

Default setting:
}
\verb*?PF6_FORWARD_UDP_REJ_LIMIT='1/second:5'?

\config{PF6\_FORWARD\_N}{PF6\_FORWARD\_N}{PF6FORWARDN}{
This variable contains the number of IPv6-firewall rules for packets to be 
forwarded (FORWARD-chain). Two rules are activated as a default : the first 
denies forwarding of all local samba packets to non-local nets and the 
second allows this for all other local packets from the first defined IPv6-subnet.

If more local IPv6-dubnets are defined the last rule has to be cloned accordingly. 
See also the configuration file.

Example:
}
\verb*?PF6_FORWARD_N='2'?

\config{PF6\_FORWARD\_x}{PF6\_FORWARD\_x}{PF6FORWARDx}{
This variable specifies a rule for the FORWARD-chain of the IPv6-fire\-wall. 
For a detailed description see the documentation of \var{PF\_FORWARD\_x}.

Differences regarding the IPv4-firewall:
\begin{itemize}
\item \var{IPV6\_NET\_x} has to be used instead of \var{IP\_NET\_x}.
\item \var{IPV6\_ROUTE\_x} has to be used instead of \var{IP\_ROUTE\_x}.
\item IPv6-addresses must be enclosed in square brackets (including
   the network mask, if present).
\item All IPv6 address strings (including \var{IP\_NET\_x} etc.) must
   be enclosed in square brackets if a port or a port range follows.
\end{itemize}

Examples:
}

\begin{example}
\begin{verbatim}
PF6_FORWARD_1='tmpl:samba DROP'
PF6_FORWARD_2='IPV6_NET_1 ACCEPT'
\end{verbatim}
\end{example}

\config{PF6\_FORWARD\_x\_COMMENT}{PF6\_FORWARD\_x\_COMMENT}{PF6FORWARDxCOMMENT}{
This variable holds a description or a comment for the forward rule it belongs to.

Example:
}
\verb*?PF6_FORWARD_1_COMMENT='no samba traffic allowed'?

\config{PF6\_OUTPUT\_POLICY}{PF6\_OUTPUT\_POLICY}{PF6OUTPUTPOLICY}{
This variable sets the default strategy for outgoing packets from the router
(OUTPUT chain). Possible values are ``REJECT'' (standard, denies all packets),
``DROP'' (discards all packets without further notification) and ``ACCEPT''
(accepts all packages). For a more detailed description see the documentation of the
Variable \var{PF\_OUTPUT\_POLICY}.

Default setting:
}
\verb*?PF6_OUTPUT_POLICY='REJECT'?

\config{PF6\_OUTPUT\_ACCEPT\_DEF}{PF6\_OUTPUT\_ACCEPT\_DEF}{PF6OUTPUTACCEPTDEF}{
This variable enables the default rules for the OUTPUT chain of the
IPv6 firewall. Possible values are ``yes'' or ``no''. Currently, there are no
preset rules.

Default setting:
}
\verb*?PF6_OUTPUT_ACCEPT_DEF='yes'?

\config{PF6\_OUTPUT\_LOG}{PF6\_OUTPUT\_LOG}{PF6OUTPUTLOG}{
This variable enables logging of all rejected outgoing packets.
Possible values are ``yes'' or ``no''. For a more detailed description see
the documentation of variable \var{PF\_OUTPUT\_LOG}.

Default setting:
}
\verb*?PF6_OUTPUT_LOG='no'?

\config{PF6\_OUTPUT\_LOG\_LIMIT}{PF6\_OUTPUT\_LOG\_LIMIT}{PF6OUTPUTLOGLIMIT}{
This variable configures the log limit for the OUTPUT chain of the IPv6 firewall,
to keep the log file readable. For a more detailed description see the
documentation of variable \var{PF\_OUTPUT\_LOG\_LIMIT}.

Default setting:
}
\verb*?PF6_OUTPUT_LOG_LIMIT='3/minute:5'?

\config{PF6\_OUTPUT\_REJ\_LIMIT}{PF6\_OUTPUT\_REJ\_LIMIT}{PF6OUTPUTREJLIMIT}{
This variable configures the limit for the rejection of outgoing
TCP packets. If a packet exceeds this limit the packet is discarded
quietly (DROP). For a more detailed description see the
documentation of variable \var{PF\_OUTPUT\_REJ\_LIMIT}.

Default setting:
}
\verb*?PF6_OUTPUT_REJ_LIMIT='1/second:5'?

\config{PF6\_OUTPUT\_UDP\_REJ\_LIMIT}{PF6\_OUTPUT\_UDP\_REJ\_LIMIT}{PF6OUTPUTUDPREJLIMIT}{
This variable configures the limit for the rejection of outgoing
UDP packets. If a packet exceeds this limit the packet is discarded
quietly (DROP). For a more detailed description see the
documentation of variable \var{PF\_OUTPUT\_UDP\_REJ\_LIMIT}.

Default setting:
}
\verb*?PF6_OUTPUT_UDP_REJ_LIMIT='1/second:5'?

\config{PF6\_OUTPUT\_N}{PF6\_OUTPUT\_N}{PF6OUTPUTN}{
This variable contains the number of IPv6 firewall rules for incoming
packets (OUTPUT chain). By default, two rules are activated: the first
allows all local hosts to access the router via so-called
Link-level addresses and the second allows router access for hosts from the
first defined IPv6 subnet.

If several local IPv6 subnets are defined, the second rule must be
added repeatedly. See the configuration file.

Example:
}
\verb*?PF6_OUTPUT_N='1'?

\config{PF6\_OUTPUT\_x}{PF6\_OUTPUT\_x}{PF6OUTPUTx}{
This variable specifies a rule for the OUTPUT chain of the IPv6 Firewall.
For a more detailed description, see the documentation of the variable
\var{PF\_OUTPUT\_x}.

Differences from IPv4 firewall:
\begin{itemize}
\item \var{IPV6\_NET\_x} has to be used instead of \var{IP\_NET\_x}.
\item \var{IPV6\_ROUTE\_x} has to be used instead of \var{IP\_ROUTE\_x}.
\item IPv6 addresses must be enclosed in square brackets (including
    the network mask, if present).
\item All IPv6 address strings (including \var{IP\_NET\_x} etc.) must
    be enclosed in square brackets if followed by a port or a port range.
\end{itemize}

Examples:
}

\begin{example}
\begin{verbatim}
PF6_OUTPUT_1='tmpl:ftp IPV6_NET_1 ACCEPT HELPER:ftp'
\end{verbatim}
\end{example}

\config{PF6\_OUTPUT\_x\_COMMENT}{PF6\_OUTPUT\_x\_COMMENT}{PF6OUTPUTxCOMMENT}{
This variable contains a description or comment to the associated
OUTPUT rule.

Example:
}
\verb*?PF6_OUTPUT_3_COMMENT='no samba traffic allowed'?

\config{PF6\_USR\_CHAIN\_N}{PF6\_USR\_CHAIN\_N}{PF6USRCHAINN}{
This variable holds the number of IPv6-firewall tables defined by the user. 
For a detailed description see the documentation of \var{PF\_USR\_CHAIN\_N}.

Default setting:
}
\verb*?PF6_USR_CHAIN_N='0'?

\config{PF6\_USR\_CHAIN\_x\_NAME}{PF6\_USR\_CHAIN\_x\_NAME}{PF6USRCHAINxNAME}{
This variable contains the name of the according user defined IPv6-Firewall table. 
For a detailed description see the documentation of \var{PF\_USR\_CHAIN\_x\_NAME}.

Example:
}
\verb*?PF6_USR_CHAIN_1_NAME='usr-myvpn'?

\config{PF6\_USR\_CHAIN\_x\_RULE\_N}{PF6\_USR\_CHAIN\_x\_RULE\_N}{PF6USRCHAINxRULEN}{
This variable contains the number of IPv6-firewall rules in the according 
user defined IPv6-firewall table. For a detailed description see the documentation of 
\var{PF\_USR\_CHAIN\_x\_RULE\_N}.

Example:
}
\verb*?PF6_USR_CHAIN_1_RULE_N='0'?

\config{PF6\_USR\_CHAIN\_x\_RULE\_x}{PF6\_USR\_CHAIN\_x\_RULE\_x}{PF6USRCHAINxRULEx}{
This variable specifies a rule for the user defined IPv6-firewall table. For a 
detailed description see the documentation of \var{PF\_USR\_CHAIN\_x\_RULE\_x}.

Differences regarding the IPv4-firewall:
\begin{itemize}
\item \var{IPV6\_NET\_x} has to be used instead of \var{IP\_NET\_x}.
\item \var{IPV6\_ROUTE\_x} has to be used instead of \var{IP\_ROUTE\_x}.
\item IPv6-addresses must be enclosed in square brackets (including
	the network mask, if present).
\item All IPv6 address strings (including \var{IP\_NET\_x} etc.) must
    be enclosed in square brackets if a port or a port range follows.
\end{itemize}
}

\config{PF6\_USR\_CHAIN\_x\_RULE\_x\_COMMENT}{PF6\_USR\_CHAIN\_x\_RULE\_x\_COMMENT}{PF6USRCHAINxRULExCOMMENT}{
This variable holds a description or a comment for the rule it belongs to.

Example:
}
\verb*?PF6_USR_CHAIN_1_RULE_1_COMMENT='some user-defined rule'?

\config{PF6\_POSTROUTING\_N}{PF6\_POSTROUTING\_N}{PF6POSTROUTINGN}{
This variable contains the number of IPv6 firewall rules for masking
(POSTROUTING chain). For a more detailed description, see the documentation of
variable \var{PF\_POSTROUTING\_N}.

Example:
}
\verb*?PF6_POSTROUTING_N='2'?

\configlabel{PF6\_POSTROUTING\_x\_COMMENT}{PF6POSTROUTINGxCOMMENT}
\config{PF6\_POSTROUTING\_x PF6\_POSTROUTING\_x\_COMMENT}{PF6\_POSTROUTING\_x}{PF6POSTROUTINGx}
\mbox{}\newline
A list of rules that describe which IPv6 packets are masked by the router
(or will be forwarded unmasked). For a more detailed description
see the documentation of variable \var{PF\_POSTROUTING\_x}.

\config{PF6\_PREROUTING\_N}{PF6\_PREROUTING\_N}{PF6PREROUTINGN}{
This variable contains the number of IPv6 firewall rules for forwarding to
a different destination (PREROUTING chain). For a more detailed description
see the documentation of variable \var{PF\_PREROUTING\_N}.

Example:
}
\verb*?PF6_PREROUTING_N='2'?

\configlabel{PF6\_PREROUTING\_x\_COMMENT}{PF6PREROUTINGxCOMMENT}
\config{PF6\_PREROUTING\_x PF6\_PREROUTING\_x\_COMMENT}{PF6\_PREROUTING\_x}{PF6PREROUTINGx}
\mbox{}\newline
A list of rules to set the IPv6 packets that should be forwarded by the
router to another destination. For a more detailed description
see the documentation of variable \var{PF\_PREROUTING\_x}.

\end{description}

\subsection{Web-GUI}

The package installs a menu entry ``Packet Filter (IPv6)'' in Mini-HTTPD to 
review entries of the packet filter configured for IPv6 .
