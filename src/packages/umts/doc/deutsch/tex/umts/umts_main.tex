% Last Update: $Id$
\marklabel{sec:opt-umts}
{
\section {UMTS - Anbindung mittels UMTS an das Internet}
}

Dieses Paket stellt die Funktion zur Verfügung, den fli4l-Router mittels UMTS
an das Internet anzubinden. Für den Betrieb sind unter anderem auch weitere
optionale Pakete erforderlich.

Ein UMTS-Zugang wird generell als spezieller PPP-Circuit konfiguriert (siehe
\jump{sect:ppp-circuits}{Circuits vom Typ ``ppp''}), d.\,h. es gilt:

\begin{example}
\begin{verbatim}
    CIRC_x_TYPE='ppp'
    CIRC_x_PPP_TYPE='umts'
\end{verbatim}
\end{example}

Weitere, UMTS-spezifische (Circuit-)Einstellungen werden in den folgenden
Abschnitten erläutert.

\subsection{Allgemeine UMTS-Konfiguration}

\begin{description}

\config{OPT\_UMTS}{OPT\_UMTS}{OPTUMTS}

Diese Variable aktiviert die Unterstützung für UMTS. Damit auch tatsächlich
eine UMTS-Verbindung genutzt werden kann, muss mindestens ein PPP-Circuit den
Typ ``umts'' besitzen, d.\,h. es muss zusätzlich gelten

\begin{example}
\begin{verbatim}
    CIRC_x_TYPE='ppp'
    CIRC_x_PPP_TYPE='umts'
\end{verbatim}
\end{example}

(wobei ``x'' einen gültigen Circuit-Index darstellt).

Standard-Einstellung: \verb+OPT_UMTS='no'+

Beispiel: \verb+OPT_UMTS='yes'+

\config{UMTS\_DEBUG}{UMTS\_DEBUG}{UMTSDEBUG}

Sollen während der UMTS-Initialisierung zusätzliche Informationen auf der
Boot-Konsole ausgegeben werden, muss man \var{UMTS\_DEBUG} auf 'yes' setzen.

Standard-Einstellung: \verb+UMTS_DEBUG='no'+

Beispiel: \verb+UMTS_DEBUG='yes'+

\config{UMTS\_PIN}{UMTS\_PIN}{UMTSPIN}

Diese Variable enthält die Pin für die SIM-Karte. Erlaubt sind eine
vierstellige Nummer oder das Wort 'disabled'.

Standard-Einstellung: \verb+UMTS_PIN='disabled'+

Beispiel: \verb+UMTS_PIN='1234'+

\config{UMTS\_GPRS\_UMTS}{UMTS\_GPRS\_UMTS}{UMTSGPRSUMTS}

Diese Variable definiert, welche Übertragungsart genutzt werden soll. Die
erlaubten Werte sind \verb+gprs+ (nur GPRS), \verb+umts+ (nur UMTS) oder
\verb+both+ (GPRS oder UMTS).

Standard-Einstellung: \verb+UMTS_GPRS_UMTS='both'+

Beispiel: \verb+UMTS_GPRS_UMTS='umts'+

\config{UMTS\_ADAPTER}{UMTS\_ADAPTER}{UMTSADAPTER}

Hier wird eingetragen, ob es sich bei dem UMTS-Gerät um eine PCMCIA-Karte (Wert
\verb+pcmcia+), einen USB-Adapter (Wert \verb+usbstick+) , um ein per
USB-Kabel angeschlossenes Telefon (Wert \verb+usbphone+) , um einen auf dem 
GOBI1000-Chipsatz (Wert \verb+gobi1000+) basierenden Adapter oder um einen 
auf dem GOBI2000-Chipsatz (Wert \verb+gobi2000+) basierenden Adapter handelt.

Diese Variable ist optional. Bei Nichtvorhandensein der Variable werden nur die
benötigten Dateien für einen USB-Adapter kopiert.

Standard-Einstellung: \verb+UMTS_ADAPTER='usbstick'+

Beispiel: \verb+UMTS_ADAPTER='usbphone'+

\wichtig{Alle folgenden Variablen sind optional und nur notwendig wenn die
automatische Erkennung versagt!}

\config{UMTS\_IDVENDOR}{UMTS\_IDVENDOR}{UMTSIDVENDOR}

Diese Variable definiert die Hersteller-ID des UMTS-Geräts nach Einschalten des
Adapters. Bei einem GOBI-Chipsatz ist die Angabe nötig.

Beispiel: \verb+UMTS_IDVENDOR='12d1'+

\config{UMTS\_IDDEVICE}{UMTS\_IDDEVICE}{UMTSIDDEVICE}

Diese Variable definiert die Produkt-ID des UMTS-Geräts nach Einschalten des
Adapters. Bei einem GOBI-Chipsatz ist die Angabe nötig.

Beispiel: \verb+UMTS_IDDEVICE='1446'+

\config{UMTS\_IDVENDOR2}{UMTS\_IDVENDOR2}{UMTSIDVENDOR2}

Diese Variable definiert die Hersteller-ID des UMTS-Geräts nach Initialisierung
des Adapters. Die Angabe ist nur notwendig, wenn sich diese ID nach der
Initialisierung ändert. Bei einem GOBI-Chipsatz ist die Angabe nötig.

Beispiel: \verb+UMTS_IDVENDOR2='12d1'+

\config{UMTS\_IDDEVICE2}{UMTS\_IDDEVICE2}{UMTSIDDEVICE2}

Diese Variable definiert die Produkt-ID des UMTS-Geräts nach Initialisierung
des Adapters. Die Angabe ist nur notwendig, wenn sich diese ID nach der
Initialisierung ändert. Bei einem GOBI-Chipsatz ist die Angabe nötig.

Beispiel: \verb+UMTS_IDDEVICE2='1436'+

\config{UMTS\_DRV}{UMTS\_DRV}{UMTSDRV}

Diese Variable definiert den Treiber zum Ansteuern des Adapters. Wenn sie fehlt,
wird \verb+usbserial+ angenommen.

Beispiel: \verb+UMTS_DRV='option'+

\config{UMTS\_SWITCH}{UMTS\_SWITCH}{UMTSSWITCH}

Diese Variable definiert Parameter für usb-modeswitch\footnote{siehe
\altlink{http://www.draisberghof.de/usb_modeswitch/}} zum Initialisieren des
Modems. Es sollten bis auf wenige Ausnahmen alle auf der Webseite genannten
Modems automatisch erkannt werden.

Beispiel: \verb+UMTS_SWITCH='-v 0x0af0 -p 0x6971 -M 555...000 -s 10'+

\config{UMTS\_DEV}{UMTS\_DEV}{UMTSDEV}

Bei Problemen kann hier die Datenschnittstelle für den PPP-Dämon angegeben
werden. Für die Adapter sind das meist folgende:

\begin{tabular}{llll}
\emph{Adapter-Typ}  &\emph{Gerät} \\
\verb+ttyUSB0+      &usbstick \\
\verb+ttyS2+        &pcmcia \\
\verb+ttyACM0+      &usbphone \\
\end{tabular}

\config{UMTS\_CTRL}{UMTS\_CTRL}{UMTSCTRL}

Einige Adapter haben mehrere Schnittstellen, über die das Modem gesteuert wird.
Ist nur eine vorhanden können Statusinformationen nur im 'Offline'-Zustand
ausgelesen werden. Bei einer Option Fusion UMTS Quad lautet die Schnittstelle
beispielsweise ttyUSB2. Diese zusätzliche Schnittstelle kann hier angegeben
werden.

Beispiel: \verb+UMTS_CTRL='ttyUSB2'+

\end{description}

\subsection{UMTS-Circuit-Konfiguration}

\begin{description}

\config{CIRC\_x\_UMTS\_DIALOUT}{CIRC\_x\_UMTS\_DIALOUT}{CIRCxUMTSDIALOUT}

Diese Variable definiert die zu wählende Nummer zum Herstellen der Verbindung.

Standard-Einstellung: \verb+CIRC_x_UMTS_DIALOUT='*99***1\#'+

\config{CIRC\_x\_UMTS\_APN}{CIRC\_x\_UMTS\_APN}{CIRCxUMTSAPN}

Diese Variable definiert den Namen des zu nutzenden Access Points.

Standard-Einstellung: \verb+CIRC_x_UMTS_APN='web.vodafone.de'+

\end{description}

Für einige deutsche Netzbetreiber bzw. Provider können die Einwahldaten
(die über die Variablen \var{CIRC\_x\_UMTS\_APN}, \var{CIRC\_x\_PPP\_USERID}
und \var{CIRC\_x\_PPP\_PASSWORD} konfiguriert werden) Tabelle
\ref{umts:credentials} entnommen werden.\footnote{Quelle:
\altlink{http://www.teltarif.de/mobilfunk/internet/einrichtung.html}}

\begin{table}
\begin{tabular}{llll}
\emph{Anbieter}     &\emph{APN}            &\emph{Benutzername} &\emph{Passwort} \\
T-Mobile            &internet.t-mobile     &beliebig     &beliebig \\
Vodafone            &web.vodafone.de       &beliebig     &beliebig \\
E-Plus              &internet.eplus.de     &eplus        &gprs \\
O2 (Vertragskunden) &internet              &beliebig     &beliebig \\
O2 (Prepaid-Kunden) &pinternet.interkom.de &beliebig     &beliebig \\
Alice               &internet.partner1     &beliebig     &beliebig \\
\end{tabular}
\caption{Einwahldaten einiger deutscher Netzbetreiber/Provider}\label{umts:credentials}
\end{table}

