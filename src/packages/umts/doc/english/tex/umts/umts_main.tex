% Synchronized to r43711
\marklabel{sec:opt-umts}
{
\section {UMTS - Internet Connection Via UMTS}
}

Connecting a fli4l to the Internet via UMTS.
For proper operation other packages may be needed.

\subsection{Configuration}

\begin{description}

\config{OPT\_UMTS}{OPT\_UMTS}{OPTUMTS}

	Default setting:  \var{OPT\_UMTS}='no'

	'yes' activates the package.

\config{UMTS\_DEBUG}{UMTS\_DEBUG}{UMTSDEBUG}

	Default setting:  \var{UMTS\_DEBUG}='no'

	If pppd should output additional debug information set
	\var{UMTS\_DEBUG} to 'yes'. In this case pppd will write 
	additional informations to the syslog interface.

	IMPORTANT: To see this messages via syslogd \var{OPT\_SYSLOGD} 
	has to be set to 'yes'.


\config{UMTS\_PIN}{UMTS\_PIN}{UMTSPIN}

	Default setting:  \var{UMTS\_PIN}='disabled'

	Pin for the SIM card 

	Give a four digit number or the word 'disabled'

\config{UMTS\_DIALOUT}{UMTS\_DIALOUT}{UMTSDIALOUT}

	Default setting:  \var{UMTS\_DIALOUT}='*99***1\#'

	Dialing parameters for the connection

\config{UMTS\_GPRS\_UMTS}{UMTS\_GPRS\_UMTS}{UMTSGPRSUMTS}

	Default setting:  \var{UMTS\_GPRS\_UMTS}='both'

	Which transfer mode should be used

	Possible values (both, gprs, umts) 


\config{UMTS\_APN}{UMTS\_APN}{UMTSAPN}

	Default setting:  \var{UMTS\_APN}='web.vodafone.de'

\config{UMTS\_USER}{UMTS\_USER}{UMTSUSER}

	Default setting:  \var{UMTS\_USER}='anonymer'

\config{UMTS\_PASSWD}{UMTS\_PASSWD}{UMTSPASSWD}

	Default setting:  \var{UMTS\_PASSWD}='surfer'

	Specify data needed for dial-in.

	Set username and password for the provider used. 
	\var{UMTS\_USER} is the username,
	\var{UMTS\_PASSWD} the password.

	Name of the APN (Acess Provider Node) for some german providers
	\begin{table}
	\textbf{Dial-in data of some german providers}

	\vspace{1ex}
	\begin{tabular}{llll}
	Provider            &APN                   &Username     &Password \\
	T-Mobile            &internet.t-mobile     &arbitrary    &arbitrary \\
	Vodafone            &web.vodafone.de       &arbitrary    &arbitrary \\
	E-Plus              &internet.eplus.de     &eplus        &gprs \\
	O2 (Vertragskunden) &internet              &arbitrary    &arbitrary \\
	O2 (Prepaid-Kunden) &pinternet.interkom.de &arbitrary    &arbitrary \\
	Alice               &internet.partner1     &arbitrary    &arbitrary \\
	\end{tabular}
	\end{table}

	\begin{itemize}
	\item \altlink{http://www.teltarif.de/mobilfunk/internet/einrichtung.html}
	\end{itemize}


\config{UMTS\_NAME}{UMTS\_NAME}{UMTSNAME}

	Default setting:  \var{UMTS\_NAME}='UMTS'

	Set a name for the  circuit here - maximum 15 chars.
	It will be shown in the imon-client imonc. Blanks are not 
	allowed.
 	
\config{UMTS\_HUP\_TIMEOUT}{UMTS\_HUP\_TIMEOUT}{UMTSHUPTIMEOUT}

	Default setting:  \var{UMTS\_TIMEOUT}='600'

	Specify a hangup time in seconds here if no traffic is detected 
	over the UMTS connection. A timeout '0' is equal to no timeout.

\config{UMTS\_TIMES}{UMTS\_TIMES}{UMTSTIMES}

	Default setting:  \var{UMTS\_TIMES}='Mo-Su:00-24:0.0:Y'

	Times mentioned here determine when the circuit becomes active and 
	at what costs. This allows to use different circuits and default routes 
	at different times (Least-Cost-Routing). The daemon imond will 
	control	routing.

\config{UMTS\_CHARGEINT}{UMTS\_CHARGEINT}{UMTSCHARGEINT}

	Default setting:  \var{UMTS\_CHARGEINT}='60'

	Charge-Interval: Timespan in seconds, that will be used 
	for calculating online costs.

\config{UMTS\_USEPEERDNS}{UMTS\_USEPEERDNS}{UMTSUSEPEERDNS}

	Default setting:  \var{UMTS\_USEPEERDNS}='yes'

	Use the provider's DNS server or not.

\config{UMTS\_FILTER}{UMTS\_FILTER}{UMTSFILTER}

	Default setting:  \var{UMTS\_FILTER}='yes'

	fli4l automatically hangs up if no traffic is going over the ppp0 
	interface in the hangup timeout time. Unfortunately also data transfers 
	from outside count as relevant traffic i.e. P2P-clients like eDonkey. 
	Since you will be nearly permanently contacted from outside nowadays 
	it may happen that fli4l never hangs up an UMTS connection. 

	Option \var{UMTS\_FILTER} is helping here. If set to 'yes' only traffic 
	generated by your own machine is monitored and external traffic will be 
	ignored completely. Since incoming traffic normally leads to a reaction 
	from the router or the machines behind it (i.e. denying or dropping the 
	packets) some additional outgoing packets will be ignored too.


\config{UMTS\_ADAPTER}{UMTS\_ADAPTER}{UMTSADAPTER}

	(optional)

	Specify here if the hardware is a PCMCIA card, an USB adapter or 
	a phone connected via an USB cable.

	If omitting this variable only the files necessary for an USB 
	adapter will be copied.

	Possible values: (pcmcia,usbstick,usbphone)

\textbf{All variables that follow are optional and only needed if the automatic detection fails.}

\config{UMTS\_IDVENDOR}{UMTS\_IDVENDOR}{UMTSIDVENDOR}

	(optional) \var{UMTS\_IDVENDOR}='xxxx'

	Vendor ID after plugging/starting the adapter

\config{UMTS\_IDDEVICE}{UMTS\_IDDEVICE}{UMTSIDDEVICE}

	(optional) \var{UMTS\_IDDEVICE}='xxxx'

	Product ID after plugging/starting the adapter

\textbf{Specifying the following two parameters is only needed if the ID changes after initialisation}
\config{UMTS\_IDVENDOR2}{UMTS\_IDVENDOR2}{UMTSIDVENDOR2}

	(optional) \var{UMTS\_IDVENDOR2}='xxxx'

	Vendor ID after initialisation of the adapter

\config{UMTS\_IDDEVICE2}{UMTS\_IDDEVICE2}{UMTSIDDEVICE2}

	(optional) \var{UMTS\_IDDEVICE2}='xxxx'

	Product ID after initialisation of the adapter

\config{UMTS\_DRV}{UMTS\_DRV}{UMTSDRV}

	(optional) \var{UMTS\_DRV}='xxxx'

	Driver for the adapter, if omitted 'usbserial' is used

\config{UMTS\_SWITCH}{UMTS\_SWITCH}{UMTSSWITCH}

	(optional) \var{UMTS\_SWITCH}='-v 0x0af0 -p 0x6971 -M 555...000 -s 10'

	Parameters for usb-modeswitch initialisation of the modem (see Website 
	usb-modeswitch). With a few exceptions all modems mentioned on the website should be 
	recognized automatically.
	
	\begin{itemize}
	\item \altlink{http://www.draisberghof.de/usb_modeswitch/}
	\end{itemize}

\config{UMTS\_DEV}{UMTS\_DEV}{UMTSDEV}

	(optional)

	In case of problems the data interface for pppd can be set here. 
	The most usual for adapters are:

	\begin{verbatim}
	ttyUSB0 for usbstick
	ttyS2   for pcmcia
	ttyACM0 for usbphone
	\end{verbatim}

\config{UMTS\_CTRL}{UMTS\_CTRL}{UMTSCTRL}

	(optional)

	Some adapter have more interfaces for modem control. If only one is 
	existig status informations can only be read in	'Offline' state. 
	For Option Fusion UMTS Quad the interface is i.e. ttyUSB2.

\end{description}
