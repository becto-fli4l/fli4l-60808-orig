% Last Update: $Id$
\marklabel{sec:tools}
{
\section{TOOLS - Zusätzliche Werkzeuge zum Debugging}
}

    Das Paket TOOLS liefert eine Reihe von Unix Programmen, die zumeist für
    Administrations- und Debugzwecke gedacht sind. Andere Programme wie wget
    werden z.B. dafür verwendet, die erste (Werbungs-)Seite einiger Provider
    abzufangen. Mit dem Wert 'yes' wird das jeweilige Programm mit auf den
    fli4l-Router kopiert. Die Standardeinstellung ist 'no'. Die Programme werden
    nur kurz vorgestellt, wie sie zu bedienen sind, entnehme man bitte den man
    Pages einer beliebigen Unix/ Linux Distribution oder online unter:
    \altlink{http://www.linuxmanpages.com}

\subsection{Netzwerk-Tools}

\begin{description}

\config{OPT\_BMON}{OPT\_BMON}{OPTBMON} Bandbreitenmonitor

    Das Programm \emph{bmon} ist ein Monitor- und Debugging-Werkzeug um
    netzwerkbezogene Statistiken zu erheben und diese in einer
    nutzerfreundlichen Weise visuell aufbereitet anzuzeigen. Es werden
    verschiedene Ausgabemethoden unterstützt, unter anderem auch eine auf curses
    basierende interaktive Schnittstelle und eine programmierbare Ausgabe für
    Skripte.

\config{OPT\_CURL}{OPT\_CURL}{OPTCURL} Datenübertragungswerkzeug

    Das Programm curl erlaubt den Datentransfer von oder zu einem Server mit 
    einer Reihe von unterstützten Protokollen. Dazu gehören u.a. FTP, HTTP(S),
    SCP, SFTP und TFTP.

    Außerdem bietet das Programm Unterstützung für Benutzer-Authentifizierung, 
    Datentransfer via Proxy, FTP-Upload, HTTP-POST-Anfragen, SSL-Verbindungen,
    Cookies, Wiederaufnahme abgebrochener Dateiübertragungen und mehr.

    \wichtig{Um mit curl TLS-Verbindungen aufbauen zu können, sollten im Paket
    CERT via \var{CERT\_X509\_MOZILLA='yes'} die Mozilla-X.509-Wurzelzertifikate
    installiert werden.}

\config{OPT\_DIG}{OPT\_DIG}{OPTDIG} Schweizer Taschenmesser fürs DNS

    Der Befehl dig erlaubt es, vielfältige DNS-Abfragen durchzuführen.

    \config{OPT\_FTP}{OPT\_FTP}{OPTFTP} FTP-Client

    Mit dem Programm ftp können eine FTP-Verbindung zu einem FTP-Server
    aufgebaut und Dateien zwischen Router und FTP-Server übertragen werden.

\config{FTP\_PF\_ENABLE\_ACTIVE}{FTP\_PF\_ENABLE\_ACTIVE}{FTPPFENABLEACTIVE}

    Die Einstellung \verb+FTP_PF_ENABLE_ACTIVE='yes'+ fügt dem Paketfilter eine
    Regel hinzu, die auf dem Router initiiertes aktives FTP ermöglicht.
    Bei \verb+FTP_PF_ENABLE_ACTIVE='no'+ muss eine solche
    Regel (falls gewünscht) manuell zum \var{PF\_OUTPUT\_\%}-Array hinzugefügt
    werden, ein Beispiel ist in diesem \jump{sec:masqueradingmodule}{Abschnitt}
    zu finden.

    Passives FTP ist immer möglich, hierfür ist weder diese Variable noch eine
    explizite Paketfilter-Regel notwendig.

\config{OPT\_IFTOP}{OPT\_IFTOP}{OPTIFTOP} Netzwerküberwachung

    Mit dem Programm iftop wird eine Auflistung alle aktiven
    Netzwerkverbindung und deren Durchsatz direkt auf dem fli4l
    angezeigt.

    Das Programm iftop wird nach dem Anmelden auf dem fli4l-Router
    durch Eingabe von iftop gestartet.

\config{OPT\_IMONC}{OPT\_IMONC}{OPTIMONC} Textorientiertes Steuerprogramm für imond

    Dieses Programm liefert ein textorientiertes Frontend für den Router, um
    den imond zu steuern.

\config{OPT\_IPERF}{OPT\_IPERF}{OPTIPERF} Performancemessung im Netzwerk

    Mit dem Programm iperf kann eine Performancemessung des Netzwerks
    durchgeführt werden. Dazu wird das Programm auf den beiden beteiligten
    Testsystemen gestartet. Auf dem Server wird das Programm mit

\begin{example}
\begin{alltt}
fli4l-server \version~\# iperf -s
------------------------------------------------------------
Server listening on TCP port 5001
TCP window size: 85.3 KByte (default)
------------------------------------------------------------
\end{alltt}
\end{example}

    gestartet. Der Server wartet dann auf eine Verbindung vom Client. Der
    Client wird durch

\begin{example}
\begin{alltt}
fli4l-client \version~\# iperf -c 1.2.3.4
------------------------------------------------------------
Client connecting to 1.2.3.4, TCP port 5001
TCP window size: 16.0 KByte (default)
------------------------------------------------------------
[  3] local 1.2.3.5 port 50311 connected with 1.2.3.4 port 5001
[ ID] Interval       Transfer     Bandwidth
[  3]  0.0-10.0 sec    985 MBytes    826 Mbits/sec
\end{alltt}
\end{example}

    gestartet. Sofort startet die Performancemessung und zeigt die ersten
    Ergebnisse an. iperf kennt noch eine Reihe weiterer Optionen, für Details
    schauen Sie sich bitte die Informationen auf der Homepage
    \altlink{http://iperf.sourceforge.net/} an.

\config{OPT\_LNSTAT}{OPT\_LNSTAT}{OPTLNSTAT} Erweiterte Netzwerk-Informationen

    Dieses OPT installiert \texttt{lnstat}, ein Werkzeug zum Aufbereiten der
    Informationen in  \texttt{/proc/net/stat}. Der Name steht für ``Linux
    Network Statistics''.

    Beispiele für \texttt{lnstat} aus dem Handbuch:
    \begin{example}
    \begin{alltt}
    \# \textbf{lnstat -d}
        \textsl{Get a list of supported statistics files.}
    \# \textbf{lnstat -k arp_cache:entries,rt_cache:in_hit,arp_cache:destroys}
        \textsl{Select the specified files and keys.}
    \# \textbf{lnstat -i 10}
        \textsl{Use an interval of 10 seconds.}
    \# \textbf{lnstat -f ip_conntrack}
        \textsl{Use only the specified file for statistics.}
    \# \textbf{lnstat -s 0}
        \textsl{Do not print a header at all.}
    \# \textbf{lnstat -s 20}
        \textsl{Print a header at start and every 20 lines.}
    \# \textbf{lnstat -c -1 -i 1 -f rt_cache -k entries,in_hit,in_slow_tot}
        \textsl{Display statistics for keys entries, in_hit and in_slow_tot of field
        rt_cache every second.}
    \end{alltt}
    \end{example}

\config{OPT\_NETCAT}{OPT\_NETCAT}{OPTNETCAT} Übertragen von Daten an TCP basierte Server

\config{OPT\_NETIO}{OPT\_NETIO}{OPTNETIO} Performancemessung im Netzwerk

    Analog zu ``iperf'' kann mit ``netio'' eine Performancemessung des
    Netzwerks durchgeführt werden. Dazu wird das Programm auf den beiden 
    beteiligten Testsystemen gestartet. Auf dem Server wird das Programm via
    \verb+netio -s -t+ (für Datenaustausch via TCP) bzw. \verb+netio -s -u+
    (für Datenaustausch via UDP) gestartet. Auf dem Client wird dann via
    \verb+netio <Server> -t+ bzw. \verb+netio <Server> -u+ der zugehörige
    ``netio''-Clientprozess gestartet, der dann das Programm auf dem Server
    kontaktiert und eine Messung vornimmt.

\config{OPT\_NGREP}{OPT\_NGREP}{OPTNGREP} Ein grep der direkt auf einem Netzwerkdevice arbeiten kann.

\config{OPT\_NMAP}{OPT\_NMAP}{OPTNMAP} Portscanner

    Mithilfe von nmap können Betriebssysteme auf offene Ports untersucht werden.
    Weiterhin liefert das Programm auf Wunsch weitere Informationen z.B. zu
    MAC-Adressen oder dem verwendeten Betriebssystem.

\config{OPT\_NTTCP}{OPT\_NTTCP}{OPTNTTCP} Netzwerktest

    Mit dem Programm NTTCP kann man die Netzwerkgeschwindigkeit
    testen. Dazu wird auf einer Seite ein Server gestartet und auf
    einer anderen Seite ein entsprechende Client.

    Den Server startet man durch Eingabe von nttcp -i -v. Der Server
    wartet dann auf eine Testanforderung des Clients. Um jetzt z.B die
    Geschwindigkeit zu testen gibt man auf dem Client nttcp -t $<$IP
    Adresse des Servers$>$ ein.

    So sieht ein gestarteter nttcp Server aus:

\begin{example}
\begin{alltt}
fli4l-server \version~\# nttcp -i -v
nttcp-l: nttcp, version 1.47
nttcp-l: running in inetd mode on port 5037 - ignoring options beside -v and -p
\end{alltt}
\end{example}

    So sieht ein Test mit einem nttcp Client aus:

\begin{example}
\begin{alltt}
fli4l-client \version~\# nttcp -t 192.168.77.77
l~~8388608~~~~4.77~~~~0.06~~~~~14.0713~~~1118.4811~~~~2048~~~~429.42~~~34133.3
1~~8388608~~~~4.81~~~~0.28~~~~~13.9417~~~~239.6745~~~~6971~~~1448.21~~~24896.4
\end{alltt}
\end{example}

    Die Hilfeseite von nttcp zeigt alle weiteren Parameter:

\begin{verbatim}
Usage: nttcp [local options] host [remote options]
       local/remote options are:
        -t      transmit data (default for local side)
        -r      receive data
        -l#     length of bufs written to network (default 4k)
        -m      use IP/multicasting for transmit (enforces -t -u)
        -n#     number of source bufs written to network (default 2048)
        -u      use UDP instead of TCP
        -g#us   gap in micro seconds between UDP packets (default 0s)
        -d      set SO_DEBUG in sockopt
        -D      don't buffer TCP writes (sets TCP_NODELAY socket option)
        -w#     set the send buffer space to #kilobytes, which is
                dependent on the system - default is 16k
        -T      print title line (default no)
        -f      give own format of what and how to print
        -c      compares each received buffer with expected value
        -s      force stream pattern for UDP transmission
        -S      give another initialisation for pattern generator
        -p#     specify another service port
        -i      behave as if started via inetd
        -R#     calculate the getpid()/s rate from # getpid() calls
        -v      more verbose output
        -V      print version number and exit
        -?      print this help
        -N      remote number (internal use only)
        default format is: %9b%8.2rt%8.2ct%12.4rbr%12.4cbr%8c%10.2rcr%10.1ccr
\end{verbatim}

\config{OPT\_RTMON}{OPT\_RTMON}{OPTRTMON}
        Installiert ein tool, dass Änderungen der Routingtabelle
        überwacht. Primäre Verwendung: Debugging

\config{OPT\_SOCAT}{OPT\_SOCAT}{OPTSOCAT}

    Das Programm ``socat'' ist quasi eine verbesserte und mit mehr Funktionen
    ``vollgestopfte'' Version des \jump{OPTNETCAT}{``netcat''-Programms}. Mit
    ``socat'' können nicht nur diverse Netzwerk-Verbindungen aufgebaut bzw.
    entgegengenommen werden, sondern auch Daten an UNIX-Sockets, Geräte,
    FIFOs etc. gesandt bzw. von dort ausgelesen werden. Insbesondere können
    Quellen und Ziele \emph{verschiedener} Typen miteinander verbunden werden:
    Ein Beispiel wäre etwa ein via TCP auf einem Port horchender
    Netzwerk-Server, der empfangene Daten in einen lokalen FIFO schreibt bzw.
    Daten aus dem FIFO ausliest und diese dann übers Netzwerk an den Client
    schickt. Siehe \altlink{http://www.dest-unreach.org/socat/doc/socat.html}
    für mehr Informationen sowie Anwendungsbeispiele.

\config{OPT\_TCPDUMP}{OPT\_TCPDUMP}{OPTTCPDUMP} debug

    Mit dem Programm tcpdump kann Netzwerkverkehr beobachtet, ausgewertet
    mitgeschnitten werden. Mehr dazu unter z.B. Google mit den Suchworten
    \grqq{}tcpdump man\grqq{}

    tcpdump $<$parameter$>$

\config{OPT\_TRACEPATH}{OPT\_TRACEPATH}{OPTTRACEPATH} Ermittlung der PMTU

    Mit dem Programm \texttt{tracepath} kann die so
    genannte ``Path MTU'' ermittelt werden. Das ist die maximal verwendbare
    Paketgröße auf dem Pfad vom fli4l-Router zum Zielhost. Größere Pakete
    müssen entweder fragmentiert werden (IPv4) oder werden verworfen (IPv6).
    Typischerweise sorgt sich der Linux-Kernel um die Ermittlung der korrekten
    Path MTU (Stichwort ``Path MTU Discovery''). Gelegentlich ist es jedoch
    nützlich, diese Path MTU herauszufinden, um Probleme im Netzwerk zu finden.
    
    Ein Beispiel für IPv4:
    \begin{example}
    \begin{alltt}
    sandbox 4.0.0-r46077M \# tracepath -4 fli4l.de
     1?: [LOCALHOST]                                         pmtu 1500
     1:  fritz.box                                             0.703ms 
     1:  fritz.box                                             0.588ms 
     2:  a89-182-53-190.net-htp.de                             0.702ms pmtu 1492
     2:  a81-14-248-243.net-htp.de                            33.692ms 
     3:  a81-14-249-82.net-htp.de                             32.089ms asymm  4 
     4:  xe-4-1-2.edge4.Berlin1.Level3.net                    35.936ms
     5:  SYSELEVEN-G.edge4.Berlin1.Level3.net                 74.944ms asymm  8
     6:  ecix.dus.octalus.in-berlin.de                        49.693ms asymm  7
     7:  virtualhost.in-berlin.de                             50.269ms reached
         Resume: pmtu 1492 hops 7 back 57
    \end{alltt}
    \end{example}
    Hier erfolgt die Internet-Anbindung über eine DSL-Verbindung zwischen einer
    AVM Fritz!Box und einem BRAS des Internet-Providers htp. Da für DSL in
    Deutschland PPP-over-Ethernet (PPPoE) verwendet wird, ist die Path MTU
    nur 1492 Byte groß.

    Ein Beispiel für IPv6:
    \begin{example}
    \begin{alltt}
    sandbox 4.0.0-r46077M \# tracepath -6 fli4l.de
     1?: [LOCALHOST]                        0.046ms pmtu 1280
     1:  gw-1362.ham-01.de.sixxs.net                          43.586ms
     1:  gw-1362.ham-01.de.sixxs.net                          42.832ms
     2:  2001:6f8:862:1::c2e9:c729                            43.565ms asymm  1
     3:  2001:6f8:862:1::c2e9:c72c                            44.313ms asymm  2
     4:  30gigabitethernet4-3.core1.fra1.he.net               64.501ms asymm  6
     5:  no reply
     6:  virtualhost.in-berlin.de                             65.949ms reached
         Resume: pmtu 1280 hops 6 back 56
    \end{alltt}
    \end{example}
    Hier erfolgt die Internet-Anbindung über einen 6in4-Tunnel von SixXS. In
    der Konfiguration des Tunnels ist eine Tunnel-MTU von 1280 fest eingestellt.
    Da dies die kleinste erlaubte MTU für IPv6-Verbindungen ist, wird sie
    nirgendwo anders auf dem Pfad zum Zielhost weiter reduziert.

\config{OPT\_DHCPDUMP}{OPT\_DHCPDUMP}{OPTDHCPDUMP} DHCP packet dumper

    Mit dem Programm dhcpdump können DHCP Pakete genauer analysiert werden.
    Das Programm setzt auf tcpdump auf und erzeugt leicht lesbare Ausgaben.

    Benutzung:

\begin{verbatim}
    dhcpdump -i interface [-h regular-expression]
\end{verbatim}

    Gestartet wird das Programm also bspw. mit folgendem Aufruf:

\begin{verbatim}
    dhcpdump -i eth0
\end{verbatim}

    Falls gewünscht, kann mit Hilfe regulärer Ausdrücke auch direkt auf eine
    bestimmte MAC-Adresse gefiltert werden. Der Aufruf sieht dann so aus:

\begin{verbatim}
    dhcpdump -i eth0 -h ^00:a1:c4
\end{verbatim}

    Die Ausgabe könnte dann z.B. so aussehen:

\begin{example}
\begin{verbatim}
         TIME: 15:45:02.084272
           IP: 0.0.0.0.68 (0:c0:4f:82:ac:7f) > 255.255.255.255.67 (ff:ff:ff:ff:ff:ff)
           OP: 1 (BOOTPREQUEST)
        HTYPE: 1 (Ethernet)
         HLEN: 6
         HOPS: 0
          XID: 28f61b03
         SECS: 0
        FLAGS: 0
       CIADDR: 0.0.0.0
       YIADDR: 0.0.0.0
       SIADDR: 0.0.0.0
       GIADDR: 0.0.0.0
       CHADDR: 00:c0:4f:82:ac:7f:00:00:00:00:00:00:00:00:00:00
        SNAME: .
        FNAME: .
       OPTION:  53 (  1) DHCP message type         3 (DHCPREQUEST)
       OPTION:  54 (  4) Server identifier         130.139.64.101
       OPTION:  50 (  4) Request IP address        130.139.64.143
       OPTION:  55 (  7) Parameter Request List      1 (Subnet mask)
                                                     3 (Routers)
                                                    58 (T1)
                                                    59 (T2)
\end{verbatim}
\end{example}

\config{OPT\_WGET}{OPT\_WGET}{OPTWGET} http/ftp Client

    Mit dem Programm wget können Daten von einem Webserver im Batch
    abgerufen werden. Praktisch ist aber (und deswegen ist wget im
    fli4l-Paket dabei), dass man damit Umlenkungen des Providers auf
    den eigenen Webserver nach einem Verbindungsaufbau auf einfache
    Weise abfangen kann, z.B. für Freenet. Wie das geht, hat Steffen
    Peiser in einem Mini-HowTo erklärt.

    Siehe:
\altlink{http://www.fli4l.de/hilfe/howtos/einsteiger/wget-und-freenet/}

    \wichtig{Um mit wget TLS-Verbindungen aufbauen zu können, sollten im Paket
    CERT via \var{CERT\_X509\_MOZILLA='yes'} die Mozilla-X.509-Wurzelzertifikate
    installiert werden.}

\end{description}

\subsection{Die Hardware-Erkennung}

Oftmals weiß man nicht genau, welche Hardware im eigenen Rechner
steckt bzw. welche Treiber man nun genau für seine Netzwerkkarte oder
seinen USB-Chipsatz verwenden soll. Die Hardware kann an der Stelle
helfen. Sie liefert eine Liste von Geräten im Rechner und wenn möglich
den dazugehörenden Treiber. Man kann dabei auswählen, ob die Erkennung
gleich beim Booten erfolgen soll (was sich vor einer Erstinstallation
empfiehlt) oder später bei laufendem Rechner bequem über das
Web-Interfaces getriggert werden soll. Die Ausgabe könnte dabei
z.B. wie folgt aussehen:

\begin{example}
\begin{alltt}
fli4l \version # cat /bootmsg.txt
#
# PCI Devices and drivers
#
Host bridge: Advanced Micro Devices [AMD] CS5536 [Geode companion] Host Bridge (rev 33)
Driver: 'unknown'
Entertainment encryption device: Advanced Micro Devices [AMD] Geode LX AES Security Block
Driver: 'geode_rng'
Ethernet controller: VIA Technologies, Inc. VT6105M [Rhine-III] (rev 96)
Driver: 'via_rhine'
Ethernet controller: VIA Technologies, Inc. VT6105M [Rhine-III] (rev 96)
Driver: 'via_rhine'
Ethernet controller: VIA Technologies, Inc. VT6105M [Rhine-III] (rev 96)
Driver: 'via_rhine'
Ethernet controller: Atheros Communications, Inc. AR5413 802.11abg NIC (rev 01)
Driver: 'unknown'
ISA bridge: Advanced Micro Devices [AMD] CS5536 [Geode companion] ISA (rev 03)
Driver: 'unknown'
IDE interface: Advanced Micro Devices [AMD] CS5536 [Geode companion] IDE (rev 01)
Driver: 'amd74xx'
USB Controller: Advanced Micro Devices [AMD] CS5536 [Geode companion] OHC (rev 02)
Driver: 'ohci_hcd'
USB Controller: Advanced Micro Devices [AMD] CS5536 [Geode companion] EHC (rev 02)
Driver: 'ehci_hcd'
\end{alltt}
\end{example}


Hier stecken also im wesentlichen 3 Netzwerkkarten drin, die vom
'via\_rhine'-Treiber verwaltet werden und eine Atheros-Wlan-Karte, die
vom madwifi-Treiber verwaltet wird (der Name wird noch nicht korrekt
aufgelöst).

\begin{description}

  \config{OPT\_HW\_DETECT}{OPT\_HW\_DETECT}{OPTHWDETECT} Diese
  Variable sorgt dafür, dass die für die Hardware-Erkennung Dateien auf
  dem Router landen. Man kann sich die Ergebnisse dann entweder nach
  dem Booten auf der Konsole ansehen, wenn man
  \var{HW\_DETECT\_AT\_BOOTTIME} auf 'yes' gesetzt hat oder im
  Web-Interface ansehen, wenn man \jump{OPTHTTPD}{\var{OPT\_HTTPD}} auf 'yes' gesetzt
  hat. Im Web-Interface kann man sich natürlich auch den Inhalt von
  '/bootmsg.txt' ansehen, wenn man schon einen funktionierenden
  Netzzugang hat.

  \config{HW\_DETECT\_AT\_BOOTTIME}{HW\_DETECT\_AT\_BOOTTIME}{HWDETECTATBOOTTIME} Startet die
  Hardware-Erkennung beim Booten. Die Erkennung läuft im Hintergrund
  (sie dauert ein wenig) und schreibt dann ihre Ergebnisse auf die
  Konsole und nach '/bootmsg.txt'.

\config{OPT\_LSPCI}{OPT\_LSPCI}{OPTLSPCI} Auflisten aller PCI-Geräte

\config{OPT\_I2CTOOLS}{OPT\_I2CTOOLS}{OPTI2CTOOLS} Tools für
I\textsuperscript{2}C Zugriffe.

\config{OPT\_IWLEEPROM}{OPT\_IWLEEPROM}{OPTIWLEEPROM} Tool zum Zugriff auf das
EEPROM von Intel und Atheros WLAN Karten.

Wird benötigt um z.B. bei ath9k Karten die Reg-Domain passend zu setzen
(siehe \altlink{http://blog.asiantuntijakaveri.fi/2014/08/one-of-my-atheros-ar9280-minipcie-cards.html}).

\config{OPT\_ATH\_INFO}{OPT\_ATH\_INFO}{OPTATHINFO} Tool zur Hardwarediagnose von
WLAN Karten mit Atheros Chipsatz.

Mithilfe dieses Tools können z.B. bei ath5k WLAN Karten detaillierte
Informationen über die verwendete Hardware gewonnen werden. Dazu gehören z.B.
der verwendete Chipsatz oder Angaben zur Kalibrierung.

\config{OPT\_FLASHROM}{OPT\_FLASHROM}{OPTFLASHROM} Tool zum flashen von Chipsätzen.

Mithilfe dieses Tools können z.B. PC Engines Boards mit einem neuen BIOS oder einer neuen 
Firmware versorgt werden. Näheres dazu findet sich auf \altlink{http://www.flashrom.org}.

\end{description}

\subsection{Dateien-Tools}

\begin{description}

\config{OPT\_E3}{OPT\_E3}{OPTE3} Ein Editor für fli4l

    Dies ist ein sehr kleiner, in Assembler geschriebener Editor. Er stellt
    verschiedene Editor-Modi zur Verfügung, die andere (,,große'') Editoren
    nachstellen. Um einen bestimmten Modus zu wählen, reicht es e3 mit dem
    richtigen Befehl zu starten. Eine Kurzübersicht der Tastenbelegung
    bekommt man, wenn man e3 ohne Parameter startet oder Alt+H drückt (außer
    im VI-Modus, dort muß man im CMD-Modus ,,:h'' eintippen). Zu beachten ist
    auch, dass das Caret-Zeichen (\^{ }) für die Ctrl-/Strg-Taste steht.

    \begin{tabular}{ll}
      Befehl & Modus \\
      \hline
      e3 / e3ws & WordStar, JOE \\
      e3vi      & VI, VIM \\
      e3em      & Emacs \\
      e3pi      & Pico \\
      e3ne      & NEdit
    \end{tabular}

\config{OPT\_MTOOLS}{OPT\_MTOOLS}{OPTMTOOLS} Die mtools stellen eine Reihe von
     DOS-ähnlichen Befehlen zum vereinfachten Umgang (Kopieren, Formatieren, etc.)
     mit DOS-Datenträgern bereit.

     Die genaue Syntax der Befehle kann in der Dokumentation von mtools
     nachgeschlagen werden:\\
     \altlink{http://www.gnu.org/software/mtools/manual/mtools.html}

\config{OPT\_SHRED}{OPT\_SHRED}{OPTSHRED}

        Installiert das Programm \emph{shred} auf dem Router, ein Programm zum gründlichen Löschen von Blockgeräten.

\config{OPT\_YTREE}{OPT\_YTREE}{OPTYTREE} Datei-Manager

    Installiert Datei-Manager Ytree auf dem Router.

\end{description}

\subsection {Entwickler-Tools}

\begin{description}

\config{OPT\_OPENSSL}{OPT\_OPENSSL}{OPTOPENSSL}

    Mit dem Programm openssl können z.B. Test der Cryptobeschleuniger durchgeführt werden.

\begin{example}
\begin{alltt}
openssl speed -evp des -elapsed
openssl speed -evp des3 -elapsed
openssl speed -evp aes128 -elapsed
\end{alltt}
\end{example}

\config{OPT\_STRACE}{OPT\_STRACE}{OPTSTRACE} debug

    Mit dem Programm strace können die Funktionsaufrufe, der Ablauf eines
    Programmes beobachtet werden

    strace $<$programm$>$

\config{OPT\_REAVER}{OPT\_REAVER}{OPTREAVER} Brute force Angriff aus Wifi WPS PINs

    Testet alle möglichen WPS PINS aus um das WPA Passwort zu
    ermitteln. Details für die Verwendung auf der Kommandozeile bitte
    nachlesen unter: \\
    \altlink{http://code.google.com/p/reaver-wps/}

\config{OPT\_VALGRIND}{OPT\_VALGRIND}{OPTVALGRIND}
        Installiert Valgrind auf dem Router.

\end{description}
