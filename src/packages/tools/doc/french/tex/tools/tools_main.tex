% Do not remove the next line
% Synchronized to r57922

\marklabel{sec:tools}
{
\section{TOOLS - Outils supplémentaires pour le débogage}
}

    Le paquetage Tools fournit un certain nombre de programmes Unix, pour
    l'administration et aussi pour le débogage. D'autres programmes sont
    intégrer comme wget, par ex. pour intercepter la première page (publicité)
    de certains fournisseurs d'accès. Si vous indiquez 'yes' le programme choisi
    sera installé dans le routeur fli4l. Le paramètre par défaut est 'no'. Voici
    une brève présentation des programmes, sur la façon de les utiliser, s.v.p.
    utiliser la commande man pour avoir plus informations sur les commandes des
    programmes de la distribution Unix/Linux, ou voir le site~:
    \altlink{http://www.linuxmanpages.com}

\subsection{Outils pour le réseau}

\begin{description}

\config{OPT\_BMON}{OPT\_BMON}{OPTBMON} Moniteur de bande passante

    Le programme \emph{bmon} est un outil de surveillance et de débogage,
    il collecte les statistiques liées au réseau et les afficher de manière
    conviviale. Il propose différentes méthodes de sortie, notamment une
    interface utilisateur avec des curseurs interactifs et une sortie texte
    programmable avec l'utilisation de scripts.

\config{OPT\_CURL}{OPT\_CURL}{OPTCURL} Outil pour le transfert de données

    Le programme cURL permet le transfert de données depuis ou vers un serveur
	il supporte un certain nombre de protocoles. Il comprend entre autres les protocoles
	FTP, HTTP(S), SCP, SFTP et TFTP.

    Le programme fournit également une authentification de l'utilisateur, le transfert
	de données via un proxy, le téléchargement FTP, des requêtes POST pour HTTP,
	une connexion SSL, les cookies, la reprise du transfert de fichiers interrompus,
	et encore plus.

	\wichtig{Pour établir une connexion TLS, les certificats X.509 doivent être installés
	à la racines de Mozilla, pour cela vous devez activer la variable \var{CERT\_X509\_MOZILLA='yes'}
	dans le paquetage CERT.}

\config{OPT\_DIG}{OPT\_DIG}{OPTDIG} Couteau suisse pour DNS

    Le programme dig vous permet d'effectuer différentes requêtes DNS.

\config{OPT\_FTP}{OPT\_FTP}{OPTFTP} Pour utiliser un client FTP

    Avec le programme FTP, vous pouvez mettre en place une connexion FTP en
    utilisant un serveur FTP et transmettre des fichiers entre le serveur FTP
    et le routeur.

\config{FTP\_PF\_ENABLE\_ACTIVE}{FTP\_PF\_ENABLE\_ACTIVE}{FTPPFENABLEACTIVE}

    Si vous activez la variable \verb+FTP_PF_ENABLE_ACTIVE='yes'+ une règle
    sera ajoutée dans le filtrage de paquets pour le routeur elle permettra
    d'initialiser un FTP actif. Si la variable \verb+FTP_PF_ENABLE_ACTIVE='no'+
    est déactivée, vous pouvez créer cette règle manuellement dans
    \var{PF\_OUTPUT\_\%} elle sera ajoutée à la liste. Vous pouvez trouver
    un exemple dans cette \jump{sec:masqueradingmodule}{section}.

    Un FTP passif est toujours possible, il n'est pas nécessaire d'utiliser
    cette variable ni de créer une règle de filtrage de paquets.

\config{OPT\_IFTOP}{OPT\_IFTOP}{OPTIFTOP} Pour la surveillance du réseau

    Avec le programme iftop, vous pouvez créer une liste de toutes les connexions
    réseaux, il affichera le débit direct de fli4l.

    Le programme iftop démarre, après avoir installé iftop sur le routeur fli4l.

\config{OPT\_IMONC}{OPT\_IMONC}{OPTIMONC} Pour la gestion du programme imond par ligne de commande

    Ce programme est utilisé pour le contrôle de fli4l par ligne de commande sur
    le routeur, afin de gèrer imond.

\config{OPT\_IPERF}{OPT\_IPERF}{OPTIPERF} Pour mesurer la performance du réseau

    Avec le programme iperf, vous pouvez effectuer des mesures sur la performance
    du réseau. En outre, la commande doit être lancée sur les deux systèmes
    serveur/client pour le test. Voici la commande du programme sur le serveur.

\begin{example}
\begin{alltt}
fli4l-server \version~\# iperf -s
------------------------------------------------------------
Server listening on TCP port 5001
TCP window size: 85.3 KByte (default)
------------------------------------------------------------
\end{alltt}
\end{example}

    Ensuite pour démarrer, le serveur attend une connexion du client. Voici
    la commande sur le client avec l'adresse IP du serveur.

\begin{example}
\begin{alltt}
fli4l-client \version~\# iperf -c 1.2.3.4
------------------------------------------------------------
Client connecting to 1.2.3.4, TCP port 5001
TCP window size: 16.0 KByte (default)
------------------------------------------------------------
[  3] local 1.2.3.5 port 50311 connected with 1.2.3.4 port 5001
[ ID] Interval       Transfer     Bandwidth
[  3]  0.0-10.0 sec    985 MBytes    826 Mbits/sec
\end{alltt}
\end{example}

    Les mesures de performances démarre immédiatement et affiche les premiers
    résultats. iperf utilise un certain nombre d'options. Pour plus de détails
    sur ces options, visiter s'il vous plaît la page d'accueil du lien
    \altlink{http://iperf.sourceforge.net/}.

\config{OPT\_LNSTAT}{OPT\_LNSTAT}{OPTLNSTAT} Information sur les réseaux avancés

    Si vous instalez l'OPT \texttt{lnstat}, avec cet outil vous pouvez éditer les informations
	du réseau dans le dossier \texttt{/proc/net/stat}. Le nom signifie "Linux Network Statistics"
	(ou Statistique du Réseau Linux). La commande lnstat permet un affichage de manière conviviale
	des informations du réseau.

	Exemple du manuel des commandes de \texttt{lnstat}~:

    \begin{example}
    \begin{alltt}
    \# \textbf{lnstat -d}
        \textsl{Get a list of supported statistics files.}
    \# \textbf{lnstat -k arp_cache:entries,rt_cache:in_hit,arp_cache:destroys}
        \textsl{Select the specified files and keys.}
    \# \textbf{lnstat -i 10}
        \textsl{Use an interval of 10 seconds.}
    \# \textbf{lnstat -f ip_conntrack}
        \textsl{Use only the specified file for statistics.}
    \# \textbf{lnstat -s 0}
        \textsl{Do not print a header at all.}
    \# \textbf{lnstat -s 20}
        \textsl{Print a header at start and every 20 lines.}
    \# \textbf{lnstat -c -1 -i 1 -f rt_cache -k entries,in_hit,in_slow_tot}
        \textsl{Display statistics for keys entries, in_hit and in_slow_tot of field
        rt_cache every second.}
    \end{alltt}
    \end{example}

\config{OPT\_NETCAT}{OPT\_NETCAT}{OPTNETCAT} Pour le transfert de données, basé sur un serveur TCP

\config{OPT\_NETIO}{OPT\_NETIO}{OPTNETIO} Mesure les performances du réseau

    Le programme "netio" est similaire à "iperf" il mesure la performance du réseau.
	Vous devez démarrer le programme sur les deux système à tester. Sur le serveur
	la commande sera \verb+netio -s -t+ (pour l'échange de données via le TCP) ou
	\verb+netio -s -u+ (pour l'échange de données via l'UDP). Sur le client la commande
	sera \verb+netio <serveur> -t+ ou \verb+netio <serveur> -u+ le processus client
	démarre, il contacte ensuite le serveur et effectue les mesures.

\config{OPT\_NGREP}{OPT\_NGREP}{OPTNGREP} Grep peut être utilisé directement sur le périphérique réseau

\config{OPT\_NMAP}{OPT\_NMAP}{OPTNMAP} Scanner de ports

	Vous pouvez utiliser Nmap pour scanner les ports ouverts sur un système d'exploitation.
	Le programme fournit aussi des informations supplémentaires, par ex. les adresses MAC
	ou la version du système d'exploitation utilisé.

\config{OPT\_NTTCP}{OPT\_NTTCP}{OPTNTTCP} Pour tester le réseau

    Avec le programme NTTCP, on peut tester la vitesse du réseau. Pour ce faire, 
    on démarre le programme sur le serveur et de l'autre côté, sur le client correspondant.

    On lance le serveur avec la commande nttcp -i -v. Puis, le serveur attend une demande
    de test du client. Maintenant pour tester la vitesse, on entre par exemple
    sur le client la commande nttcp -t $<$Adresse IP du Serveur$>$

    Démarrer le serveur avec nttcp comme ceci~:

\begin{example}
\begin{alltt}
fli4l-server \version~\# nttcp -i -v
nttcp-l: nttcp, version 1.47
nttcp-l: running in inetd mode on port 5037 - ignoring options beside -v and -p
\end{alltt}
\end{example}

    Test le client avec nttcp comme ceci~:

\begin{example}
\begin{alltt}
fli4l-client \version~\# nttcp -t 192.168.77.77
l~~8388608~~~~4.77~~~~0.06~~~~~14.0713~~~1118.4811~~~~2048~~~~429.42~~~34133.3
1~~8388608~~~~4.81~~~~0.28~~~~~13.9417~~~~239.6745~~~~6971~~~1448.21~~~24896.4
\end{alltt}
\end{example}

    Vous pouvez voir ci-dessous tous les paramètres nttcp~:

\begin{verbatim}
Usage: nttcp [local options] host [remote options]
       local/remote options are:
        -t      transmit data (default for local side)
        -r      receive data
        -l#     length of bufs written to network (default 4k)
        -m      use IP/multicasting for transmit (enforces -t -u)
        -n#     number of source bufs written to network (default 2048)
        -u      use UDP instead of TCP
        -g#us   gap in micro seconds between UDP packets (default 0s)
        -d      set SO_DEBUG in sockopt
        -D      don't buffer TCP writes (sets TCP_NODELAY socket option)
        -w#     set the send buffer space to #kilobytes, which is
                dependent on the system - default is 16k
        -T      print title line (default no)
        -f      give own format of what and how to print
        -c      compares each received buffer with expected value
        -s      force stream pattern for UDP transmission
        -S      give another initialisation for pattern generator
        -p#     specify another service port
        -i      behave as if started via inetd
        -R#     calculate the getpid()/s rate from # getpid() calls
        -v      more verbose output
        -V      print version number and exit
        -?      print this help
        -N      remote number (internal use only)
        default format is: %9b%8.2rt%8.2ct%12.4rbr%12.4cbr%8c%10.2rcr%10.1ccr
\end{verbatim}

\config{OPT\_RTMON}{OPT\_RTMON}{OPTRTMON} Pour le débogage

    Si vous installez cette outil, il surveillera les changements du tableau
    de routage. L'utilisation initial est~: le débogage

\config{OPT\_SOCAT}{OPT\_SOCAT}{OPTSOCAT}

    Le programme "socat" est une version plus ou moins améliorée du
    \jump{OPTNETCAT}{programme "netcat"} avec plus de fonctionnalités.
    En utilisant "socat" vous pouvez non seulement établir ou accepter
    différents types de connexions réseau, mais aussi d'envoyer des données
    ou lire des données avec les Sockets UNIX, les périphériques, FIFO, et
    ainsi de suite. Au sujet des sources et des destinations particulières,
    \emph{différent} types de connexions peuvent être utilisées~: l'exemple
    suivant serait un serveur réseau qui écoute sur un port TCP et qui écrit
    les données reçues dans une mémoire FIFO local ou de lire les données dans
    la mémoire FIFO, puis de les transmettre via le réseau à un client. Vous
    pouvez allez sur le site \altlink{http://www.dest-unreach.org/socat/doc/socat.html}
    pour avoir plus d'exemples sur les applications et sur la documentation.

\config{OPT\_TCPDUMP}{OPT\_TCPDUMP}{OPTTCPDUMP} Pour le déboguage réseau

    Avec le programme tcpdump on peut observer en détail le trafic du réseau
    et d'analyser les paquets. Pour en savoir plus, faite une recherche
    par ex. sur Google ou avec la commande \flqq{}tcpdump man\frqq{}.

    tcpdump $<$paramètre$>$

\config{OPT\_TRACEPATH}{OPT\_TRACEPATH}{OPTTRACEPATH} Détermine le PMTU

	Le programme \texttt{tracepath} peut soi-disant
	déterminer le MTU. C'est la taille maximale du paquet que le routeur fli4l utilise
	pour le transfère de l'hôte vers le destinataire. Les plus grands paquets seront
	fragmentés pour (l'IPv4) et rejetés pour (l'IPv6). En règle générale, le Kernel Linux
	détermine le Path MTU correctement (mots-clés "Path MTU Discovery"). Cependant, il est
	parfois utile de savoir utiliser le Path MTU pour trouver les problèmes dans le réseau.

	Un exemple pour l'IPv4~:

    \begin{example}
    \begin{alltt}
    sandbox 4.0.0-r46077M \# tracepath -4 fli4l.de
     1?: [LOCALHOST]                                         pmtu 1500
     1:  fritz.box                                             0.703ms 
     1:  fritz.box                                             0.588ms 
     2:  a89-182-53-190.net-htp.de                             0.702ms pmtu 1492
     2:  a81-14-248-243.net-htp.de                            33.692ms 
     3:  a81-14-249-82.net-htp.de                             32.089ms asymm  4 
     4:  xe-4-1-2.edge4.Berlin1.Level3.net                    35.936ms
     5:  SYSELEVEN-G.edge4.Berlin1.Level3.net                 74.944ms asymm  8
     6:  ecix.dus.octalus.in-berlin.de                        49.693ms asymm  7
     7:  virtualhost.in-berlin.de                             50.269ms reached
         Resume: pmtu 1492 hops 7 back 57
    \end{alltt}
    \end{example}

	Ci-dessus, une connexion Internet DSL via une Fritz!Box de AVM et le B-RAS du fournisseur
	d'accès Internet htp. En Allemagne une connexion DSL utilise le protocole (PPPoE) via PPP
	sur Ethernet, le path MTU est de 1492 octets.

	Un exemple pour l'IPv6~:

    \begin{example}
    \begin{alltt}
    sandbox 4.0.0-r46077M \# tracepath -6 fli4l.de
     1?: [LOCALHOST]                        0.046ms pmtu 1280
     1:  gw-1362.ham-01.de.sixxs.net                          43.586ms
     1:  gw-1362.ham-01.de.sixxs.net                          42.832ms
     2:  2001:6f8:862:1::c2e9:c729                            43.565ms asymm  1
     3:  2001:6f8:862:1::c2e9:c72c                            44.313ms asymm  2
     4:  30gigabitethernet4-3.core1.fra1.he.net               64.501ms asymm  6
     5:  no reply
     6:  virtualhost.in-berlin.de                             65.949ms reached
         Resume: pmtu 1280 hops 6 back 56
    \end{alltt}
    \end{example}

	Ci-dessus, une connexion Internet via un tunnel 6in4 de SixXS. Dans la configuration du tunnel
	le MTU est fixé à 1280 octets. C'est le plus petit MTU possible pour une connexion IPv6, il
	ne sera plus réduit nulle part ailleurs entre l'hôle et le destinataire.

\config{OPT\_DHCPDUMP}{OPT\_DHCPDUMP}{OPTDHCPDUMP} Pour analyser les paquets DHCP

    Avec le programme dhcpdump on peut analysé les paquets DHCP en détail.
    Le programme est basé sur le programme tcpdump, la sortie génèrée des paquets
    est plus facilement lisible.

    Utilisation~:

\begin{verbatim}
    dhcpdump -i interface [-h expression régulière]
\end{verbatim}

    Vous démarrez le programme par exemple avec la commande suivante~:

\begin{verbatim}
    dhcpdump -i eth0
\end{verbatim}

    Si vous le souhaitez, vous pouvez également filtrer directement une adresse
    MAC spécifique, en utilisant une expression régulière. La commande ressemble
    à ceci~:

\begin{verbatim}
    dhcpdump -i eth0 -h ^00:a1:c4
\end{verbatim}

    La réponse pourrait alors, par exemple ressembler à ceci~:

\begin{example}
\begin{verbatim}
         TIME: 15:45:02.084272
           IP: 0.0.0.0.68 (0:c0:4f:82:ac:7f) > 255.255.255.255.67 (ff:ff:ff:ff:ff:ff)
           OP: 1 (BOOTPREQUEST)
        HTYPE: 1 (Ethernet)
         HLEN: 6
         HOPS: 0
          XID: 28f61b03
         SECS: 0
        FLAGS: 0
       CIADDR: 0.0.0.0
       YIADDR: 0.0.0.0
       SIADDR: 0.0.0.0
       GIADDR: 0.0.0.0
       CHADDR: 00:c0:4f:82:ac:7f:00:00:00:00:00:00:00:00:00:00
        SNAME: .
        FNAME: .
       OPTION:  53 (  1) DHCP message type         3 (DHCPREQUEST)
       OPTION:  54 (  4) Server identifier         130.139.64.101
       OPTION:  50 (  4) Request IP address        130.139.64.143
       OPTION:  55 (  7) Parameter Request List      1 (Subnet mask)
                                                     3 (Routers)
                                                    58 (T1)
                                                    59 (T2)
\end{verbatim}
\end{example}

\config{OPT\_WGET}{OPT\_WGET}{OPTWGET} Client http/ftp

    Avec le programme wget on peut télécharger des données sur un serveur Web
    avec un fichier batch de lancement, il travail en arrière plan. Il est pratique
    (c'est pour cela qu'il est dans le paquetage fli4l), on peut télécharger
    d'une manière simple la page web du fournisseur d'accés Internet et là placer
    sur sont propre serveur web avec un lien. Par exemple sur le site de Freenet,
    Steffen Peiser a décrit les commande dans ce mini HOWTO.

    Voir~: \altlink{http://www.fli4l.de/fr/aide/guide-pratique/debutant/wget-und-freenet/}

	\wichtig{Pour établir une connexion TLS, les certificats X.509 doivent être installés
	à la racines de Mozilla, pour cela vous devez activer la variable \var{CERT\_X509\_MOZILLA='yes'}
	dans le paquetage CERT.}

\end{description}

\subsection{Outils pour la détection du matériel}

En général, on ne sait jamais exactement le matériel qui est installé dans son
propre routeur. Le matériel installé peut nous aider à configurer exactement
le pilote de la carte réseau ou du chipset USB. Pour nous fournir la liste
des périphériques et si possible des pilotes correspondants, nous avons le
choix de visualiser ces informations, soit sur la console, juste après le
démarrage (recommandé pour une première installation) ou plus facilement,
par l'intermédiaire de l'interface Web de votre ordinateur. Vous pouvez
voir ci-dessous un exemple des informations fournies, avec la commande~:

\begin{example}
\begin{alltt}
fli4l \version # cat /bootmsg.txt

#
# PCI Devices and drivers
#
Host bridge: Advanced Micro Devices [AMD] CS5536 [Geode companion] Host Bridge (rev 33)
Driver: 'unknown'
Entertainment encryption device: Advanced Micro Devices [AMD] Geode LX AES Security Block
Driver: 'geode_rng'
Ethernet controller: VIA Technologies, Inc. VT6105M [Rhine-III] (rev 96)
Driver: 'via_rhine'
Ethernet controller: VIA Technologies, Inc. VT6105M [Rhine-III] (rev 96)
Driver: 'via_rhine'
Ethernet controller: VIA Technologies, Inc. VT6105M [Rhine-III] (rev 96)
Driver: 'via_rhine'
Ethernet controller: Atheros Communications, Inc. AR5413 802.11abg NIC (rev 01)
Driver: 'unknown'
ISA bridge: Advanced Micro Devices [AMD] CS5536 [Geode companion] ISA (rev 03)
Driver: 'unknown'
IDE interface: Advanced Micro Devices [AMD] CS5536 [Geode companion] IDE (rev 01)
Driver: 'amd74xx'
USB Controller: Advanced Micro Devices [AMD] CS5536 [Geode companion] OHC (rev 02)
Driver: 'ohci_hcd'
USB Controller: Advanced Micro Devices [AMD] CS5536 [Geode companion] EHC (rev 02)
Driver: 'ehci_hcd'
\end{alltt}
\end{example}

Vous pouvez voir que 3 cartes réseaux identiques sont installées, gérées par
le pilote 'via\_rhine' et une carte wifi Atheros, gérée par le pilote madwifi
(le nom n'est pas encore résolu).

\begin{description}

\config{OPT\_HW\_DETECT}{OPT\_HW\_DETECT}{OPTHWDETECT}

    Ce script s'occupe de vérifier les fichiers installés dans le routeur
    par rapport aux matériels identifiés. On peut alors voir le résultat sur
    la console après le boot, si vous mettez la variable \var{HW\_DETECT\_AT\_BOOTTIME}
    sur 'yes' vous pouvez voir les informations sur l'interface Web, bien entendu
    vous devez placer la variable \jump{OPTHTTPD}{\var{OPT\_HTTPD}} sur 'yes'.
    Sur l'interface Web, vous pourrez naturellement voir le contenu du fichier
    '/bootmsg.txt', si vous avez un accès réseau qui fonctionne.

\config{HW\_DETECT\_AT\_BOOTTIME}{HW\_DETECT\_AT\_BOOTTIME}{HWDETECTATBOOTTIME}

    Cette variable lance la détection du matériel lors du boot. La détection
    fonctionne en tâche de fond (cela prend un peu de temps), le résultat sera
    visible sur la console, puis sera écrit dans le fichier '/bootmsg.txt'.

\config{OPT\_LSPCI}{OPT\_LSPCI}{OPTLSPCI} Pour lister tous les périphériques PCI

\config{OPT\_I2CTOOLS}{OPT\_I2CTOOLS}{OPTI2CTOOLS} Outils pour accéder
au bus I\textsuperscript{2}C

\config{OPT\_IWLEEPROM}{OPT\_IWLEEPROM}{OPTIWLEEPROM} Outil pour accéder à
l'EEPROM des cartes WLAN (ou cartes wifi) Intel et Atheros

	Nécessaire par exemple pour reprogrammer le domaine réglementaire de la carte ath9k
	(voir \altlink{http://blog.asiantuntijakaveri.fi/2014/08/one-of-my-atheros-ar9280-minipcie-cards.html}).

\config{OPT\_ATH\_INFO}{OPT\_ATH\_INFO}{OPTATHINFO} Outil pour accéder à
l'EEPROM Intel et des cartes WLAN Atheros

	Cet outil peut extraire les informations détaillées du matériel utilisé pour les cartes
	wifi Atheros, par exemple ath5k. Ceux-la comprent, le chipset utilisé ou les données
	d'étalonnage.

\config{OPT\_FLASHROM}{OPT\_FLASHROM}{OPTFLASHROM} Outil pour flasher le chipset

	Grâce à cet outil, vous pouvez installer un nouveau BIOS ou un nouveau firmware
	par exemple, sur votre carte PC Engines. Pour plus de détails, vous pouvez aller
	sur le site \altlink{http://www.flashrom.org}.

\end{description}

\subsection{Outils pour gérer les fichiers}

\begin{description}

\config{OPT\_E3}{OPT\_E3}{OPTE3} Éditeur de texte pour fli4l

    Il s'agit d'un éditeur de texte de très petite taille, écrit en assembleur.
    Vous avez à disposition différents modes d'éditeurs, comme d'autre éditeur
    plus ("grand"). Pour choisir l'un des mode, il suffit d'utiliser la bonne
    commande de E3 pour démarrer. On obtient un rapide aperçu des raccourcis
    clavier avec le paramètre man, si vous lancez E3 sans le paramètre man,
    vous pouvez appuyer sur Alt+H (sauf dans le mode VI, dans le mode CMD à la
    place de man il faut saisir ":h"). Notez également que le caractère (\^{ })
    est représenté par la touche "Ctrl".

    \begin{tabular}{ll}
      Commande & Mode \\
      \hline
      e3 / e3ws & WordStar, JOE \\
      e3vi      & VI, VIM \\
      e3em      & Emacs \\
      e3pi      & Pico \\
      e3ne      & NEdit
    \end{tabular}

\config{OPT\_MTOOLS}{OPT\_MTOOLS}{OPTMTOOLS}

    Avec mtools nous mettons à disposition une série de commandes (pour la
    copie, le formatage, etc.) similaire aux commandes DOS, ces commandes
    serviront à la gestion des données sur des supports DOS.

    Vous trouverais dans le lien ci-dessous la documentation de mtools et les
    syntaxes des paramètres de commandes de chaque programme~:\\
    \altlink{http://www.gnu.org/software/mtools/manual/mtools.html}

\config{OPT\_SHRED}{OPT\_SHRED}{OPTSHRED} Pour effacer un fichier

    Si vous installez \emph{shred} sur le routeur, ce programme effacera
    définitivement les blocs de données.

\config{OPT\_YTREE}{OPT\_YTREE}{OPTYTREE} Gestionnaire de fichier

    Si vous installez Ytree sur le routeur, vous aurez un gestionnaire de
    fichier sur votre routeur fli4l.

\end{description}

\subsection {Outils pour les développeurs}

\begin{description}

\config{OPT\_OPENSSL}{OPT\_OPENSSL}{OPTOPENSSL}

    Avec l'outil OpenSSL vous pouvez mesurer la vitesse de chiffrement
    d'encodage et l'algorithme cryptographique.

\begin{example}
\begin{alltt}
openssl speed -evp des -elapsed
openssl speed -evp des3 -elapsed
openssl speed -evp aes128 -elapsed
\end{alltt}
\end{example}

\config{OPT\_STRACE}{OPT\_STRACE}{OPTSTRACE} Pour le débogage

    Avec le programme strace, vous pouvez surveiller les appel systèmes, pour
    voir le déroulement d'un programme.

    strace $<$programme$>$

\config{OPT\_REAVER}{OPT\_REAVER}{OPTREAVER} Attaque du code PIN WPS par brute force sur le wifi

    Cette outil teste tous les codes PIN WPS pour déterminer la vulnérabilité
    du mot de passe WPA sur votre routeur. Si vous voulez plus de
    détail pour l'utilisation par ligne de commande de reaver, lire la
    documentation sur le site~: \\
    \altlink{http://code.google.com/p/reaver-wps/}

\config{OPT\_VALGRIND}{OPT\_VALGRIND}{OPTVALGRIND} Pour le débogage de programme

    Si vous installez Valgrind sur le routeur, vous pouvez débusquer les
    failles d'un programme et mettre en évidence les fuites mémoires.

\end{description}
