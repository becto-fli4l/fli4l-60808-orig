% Synchronized to r36653

\section{VPN - Réseau privé virtuel}

Ce paquetage permet d'établir une connexion sécurisée entre réseau privé à
travers le réseau public qui n'est pas sécurisé.

\subsection{PPTP-Tunnel}

Le PPTP \footnote{"Point-to-Point Tunneling Protocol" protocole de tunnel
point-à-point voir RFC 2637} permet, d'établir un canal privée sur un réseau
public. Un tunnel sera construit et exploité en utilisant le protocole TCP/IP
spécial pour les régles de contrôle. Les paquets PPP de l'utilisateur sont
encapsulés avec le protocole GRE \footnote{"Generic Routing Encapsulation"
cette trame sera ensuite transféré sur le réseau IP à travers un tunnel,
voir RFC 2784}.

En Autriche (et dans d'autres pays européens) le protocole PPTP est également
utilisé entre le routeur et le modem DSL. Contrairement aux protocoles PPPoE
et PPPoA, il travail en dessous de la couche IP (avec une couche de liaison
"Link Layer"), le PPTP décrit ci-dessus, utilise donc deux flux pour les données.
Ainsi, il faut pour une connexion DSL via le protocole PPTP réserver une adresse IP
pour le PPTP sur la carte Ethernet, contrairement à d'autres méthodes d'accès
DSL. Selon le fournisseur vous devez configurez soit une adresse fixe ou via le
DHCPv4. Vous trouverez plus d'informations dans la description de la variable
\var{CIRC\_x\_PPP\_PPTP\_PEER}.

Si vous avez besion d'une connexion DSL ou d'une connexion distante par VPN
avec le PPTP, il est déconseillé d'utiliser le PPTP pour une solution VPN.
Car le cryptage du protocole PPTP se crack facilement 
\footnote{voir \altlink{http://heise.de/-1701365}} et vous ne devez pas envoyer
de données sensibles via un tunnel PPTP. Il vaut mieux utiliser OpenVPN pour
envoyer vos données via un tunnel, c'est certainement le meilleur choix.

\subsubsection{Connexion PPTP sortante}

Généralement une connexion PPTP sortant se configure comme un circuit PPP
(voir \jump{sect:ppp-circuits}{Circuit de type "ppp"}), c'est à dire, vous devez
avoir~:

\begin{example}
\begin{verbatim}
    CIRC_x_TYPE='ppp'
\end{verbatim}
\end{example}

En outre, la variable \verb+OPT_PPP_PPTP+ doit être activé~:

\begin{description}
\config{OPT\_PPP\_PPTP}{OPT\_PPP\_PPTP}{OPTPPTP}

Cette variable vous permet de supporter le protocole PPTP. C'est en fait une connexion
PPTP que vous allez utiliser, avec un circuit PPP du type "pptp", donc vous devez
également configurer

\begin{example}
\begin{verbatim}
    CIRC_x_TYPE='ppp'
    CIRC_x_PPP_TYPE='pptp'
\end{verbatim}
\end{example}

(le "x" est l'indexation de la variable des circuits).

Paramètre par défaut~: \verb+OPT_PPP_PPTP='no'+

Exemple~: \verb+OPT_PPP_PPTP='yes'+

\config{CIRC\_x\_PPP\_PPTP\_TYPE}{CIRC\_x\_PPP\_PPTP\_TYPE}{CIRCxPPPPPTPTYPE}

cette variable sert à configurer l'encapsulage des paquets de données PPP dans
le protocole GRE, ensuite cette trame sera envoyer sur le canal. Pour cela le programme
(\texttt{pptp}) ou le kernel peut être utilisé. la variable \var{CIRC\_x\_PPP\_PPTP\_TYPE}
est utilisée pour générer les paquets GRE voir le tableau \ref{tab:pptp-type}.

\begin{table}[h!]
  \centering
  \begin{tabular}{|l|p{10cm}|}
    \hline
    Valeur & Description \\
    \hline
    kernel & Les paquets PPP sont transmis directement au noyau Linux,
    les paquets GRE seront créés. Cela élimine la communication avec
    un second processus, donc une quantité de copie et une gestion du temps
    améliorée et qui à son tour conduit à réduire la charge du processeur.\\
    daemon & Les paquets sont générés par le démon \texttt{pptp}, la communication
    entre \texttt{pppd} et \texttt{pptp} est asynchrone. Cela signifie que
    le flux de données ont des marqueurs de début et de fin, ainsi le démon
    \texttt{pptp} peut distinguer les paquets individuelment. En raison du
    second processus et du marquages supplémentaires, cette méthode est
    plus complexe que la méthode "kernel".\\
    \hline
  \end{tabular}
  \caption{(Manière de générer les paquets GRE}
  \label{tab:pptp-type}
\end{table}

Actuellement, "daemon" est la seule méthode supporté (et donc par défaut, si
le type n'est pas spécifié). Une extension du paquetage pour bénéficier du
\texttt{module kernel PPTP} est envisageable.

Paramètre par défaut~: \verb+CIRC_x_PPP_PPTP_TYPE='daemon'+

\config{CIRC\_x\_PPP\_PPTP\_PEER}{CIRC\_x\_PPP\_PPTP\_PEER}{CIRCxPPPPPTPPEER}

Dans cette variable vous indiquez l'adresse IP du serveur PPTP. 

Si vous utilisez PPTP pour un accès Internet à haute débit, l'adresse IP de
la carte Ethernet du routeur fli4l pour l'accés au PPTP doit être choisi en
conséquence. Dans le tableau \ref{tab:pptp-provider} vous avez une listes
connues des options de configuration.

\begin{table}[htb]
  \centering
  \begin{tabular}{p{6cm}|p{4cm}|p{4cm}}
    Fournisseur &
    Adresse IP local \newline(\var{IP\_NET\_2}) &
    Adresse IP distant \newline(\var{CIRC\_x\_PPP\_PPTP\_PEER}) \\
    \hline
    Telekom Austria (Autriche)     & 10.0.0.140/29 & 10.0.0.138 \\
    mxstream (Pays-Bas, Danemark)  & 10.0.0.140/29 & 10.0.0.138 \\
    Inode xDSL (Autriche)          & via DHCPv4    & 10.0.0.138 \\
  \end{tabular}
  \caption{Réglages des fournisseurs qui utilisent le PPTP}
  \label{tab:pptp-provider}
\end{table}

Si vous utilisez le DHCP pour configurer la carte de réseau local, vous devez
installer le paquetage dhcp\_client et configurer le circuit DHCPv4 correspondant à
la carte ethernet (généralement eth1). Vous pouvez voir à la fin de la documentation
un exemple pour une connexion avec Inode xDSL.

\config{CIRC\_x\_PPP\_PPTP\_REORDER\_TIMEOUT}{CIRC\_x\_PPP\_PPTP\_REORDER\_TIMEOUT}{CIRCxPPPPPTPREORDERTIMEOUT}

Si le client PPTP a besoin d'une mémoire tampon pour les paquets vous pouvez
l'indiquer dans cette variable. Normalement un paquet attend 0,3 secondes avant
le traitement de celui-ci. Vous pouvez varier le délai entre 0.00 (pas de tampon)
et 10.00 (max. attend 10 secondes) dans cette variable. Le temps doit toujours
être indiqué avec un point et deux décimales.

Paramètre par défaut~: \verb+CIRC_x_PPP_PPTP_REORDER_TIMEOUT='0.30'+

Exemple~: \verb+CIRC_1_PPP_PPTP_REORDER_TIMEOUT='1.00'+

\config{CIRC\_x\_PPP\_PPTP\_LOGLEVEL}{CIRC\_x\_PPP\_PPTP\_LOGLEVEL}{CIRCxPPPPPTPLOGLEVEL}

Vous pouvez indiquer dans cette variable un nombre pour les informations du fichier
journal. Les nombres possibles sont, 0 (faible), 1 (milieu) et 2 (beaucoup).

Paramètre par défaut~: \verb+CIRC_x_PPP_PPTP_LOGLEVEL='1'+

Exemple~: \verb+CIRC_1_PPP_PPTP_LOGLEVEL='2'+

\end{description}

\subsubsection{Connexion PPTP entrante}

fli4l peut également être configuré pour \emph{accepter}  les connexions PPTP
entrantes, il agira comme un serveur. Ces connexions PPTP sont également configurés
comme un circuit PPP. (Voir \jump{sect:ppp-circuits}{Circuits du type "ppp"}),
c'est à dire, vous devez avoir~:

\begin{example}
\begin{verbatim}
    CIRC_x_TYPE='ppp'
\end{verbatim}
\end{example}

En outre, la variable \verb+OPT_PPP_PPTP_SERVER+ doit être activé~:

\begin{description}
\config{OPT\_PPP\_PPTP\_SERVER}{OPT\_PPP\_PPTP\_SERVER}{OPTPPPPPTPSERVER}

Cette variable vous permet d'activer une connexion PPTP entrante. C'est en fait
une connexion PPTP que vous allez utiliser, avec un circuit PPP du type "pptp-server",
donc vous devez également configurer

\begin{example}
\begin{verbatim}
    CIRC_x_TYPE='ppp'
    CIRC_x_PPP_TYPE='pptp-server'
\end{verbatim}
\end{example}

(le "x" est l'indexation de la variable circuit).

Paramètre par défaut~: \verb+OPT_PPP_PPTP_SERVER='no'+

Exemple~: \verb+OPT_PPP_PPTP_SERVER='yes'+
\end{description}

Les variables du circuit géréral sont les suivantes pour le circuit PPP du type
"pptp-server", des variables spécifiques ont également été ajouter~:

\begin{description}
\config{CIRC\_x\_PPP\_PPTP\_SERVER\_LISTEN}{CIRC\_x\_PPP\_PPTP\_SERVER\_LISTEN}{CIRCxPPPPPTPSERVERLISTEN}

Dans cette variable, vous indiquez l'adresse IPv4 sur laquelle écoute le serveur
PPTP. Si vous n'indiquez rien dans cette variable le serveur PPTP écoutera sur
\emph{toutes} les interfaces du routeur. \footnote{Cela correspond à l'adresse
IPv4 \texttt{0.0.0.0}.}

Exemple~: \verb+CIRC_1_PPP_PPTP_SERVER_LISTEN='IP_NET_1_ADDR'+

\configlabel{CIRC\_x\_PPP\_PPTP\_SERVER\_ALLOW\_FROM\_N}{CIRCxPPPPPTPSERVERALLOWFROMN}
\config{CIRC\_x\_PPP\_PPTP\_SERVER\_ALLOW\_FROM\_y}{CIRC\_x\_PPP\_PPTP\_SERVER\_ALLOW\_FROM\_y}{CIRCxPPPPPTPSERVERALLOWFROMy}

Dans ces variables vous configurez la liste des adresses de réseau IPv4 qui
accèderons au serveur PPTP et qui serons autorisées par le pare-feu. Avec les
adresses configurées ici et avec les variables 
\jump{PFINPUTACCEPTDEF}{\var{PF\_INPUT\_ACCEPT\_DEF='yes'}} et
\jump{PFOUTPUTACCEPTDEF}{\var{PF\_OUTPUT\_ACCEPT\_DEF='yes'}} le pare-feu ouvre
les chaines INPUT et OUTPUT pour le contrôle du protocole PPTP sur le port TCP
1723 et pour les paquets GRE. Si cet ensemble de variables, le pare-feu ainsi
que le contrôle et les paquets de données sont configurés tous sera acceptés
\emph{de partout} sur le serveur.

Exemple~:

\begin{example}
\begin{verbatim}
    CIRC_1_PPP_PPTP_SERVER_ALLOW_FROM_N='3'
    CIRC_1_PPP_PPTP_SERVER_ALLOW_FROM_1='IP_NET_1'
    CIRC_1_PPP_PPTP_SERVER_ALLOW_FROM_2='10.1.2.0/24'
    CIRC_1_PPP_PPTP_SERVER_ALLOW_FROM_3='{Labor}'
\end{verbatim}
\end{example}

\config{CIRC\_x\_PPP\_PPTP\_SERVER\_SESSIONS}{CIRC\_x\_PPP\_PPTP\_SERVER\_SESSIONS}{CIRCxPPPPPTPSERVERSESSIONS}

Avec cette variable vous pouvez indiquer le nombre de connexions que peut gérer
le serveur PPTP simultanément. 255 tunnels peuvent être pris en charge.
\footnote{Cette limitation résulte que le démon PPTP utilise une plage d'adresses
autorisées en utilisant uniquement un seul composant de l'adresse IPv4, par exemple
"192.168.222.0-254". Puisqu'un composant peut accepter seulement les valeurs 0--255
et si la valeur 255 est réservé pour l'adresse Broadcast (ou de diffusion), cela
se traduit par la valeur indiqué ci-dessus.}

Paramètre par défaut~: \verb+CIRC_x_PPP_PPTP_SERVER_SESSIONS='100'+

Exemple~: \verb+CIRC_1_PPP_PPTP_SERVER_SESSIONS='200'+

\end{description}

\subsubsection{Exemples}

\noindent
Exemple 1 (Accès à Internet via le PPTP avec une adresse locale fixe)~:

\begin{example}
\begin{verbatim}
    IP_NET_N='2'                      # (Au moins) deux réseaux (LAN + PPTP)
    IP_NET_1='192.168.6.0/24'         # Réseau local, configuration nécessaire
    IP_NET_1_DEV='eth0'               # Réseau local qui dépend de la première carte
    IP_NET_2='10.0.0.140/29'          # Votre adresse pour le réseau PPTP
    IP_NET_2_DEV='eth1'               # Modem Internet accroché à la deuxième carte
    #
    OPT_PPP='yes'                     # Circuits PPP activé
    OPT_PPP_PPTP='yes'                # Client du circuits PPTP activé
    #
    CIRC_N='1'
    CIRC_1_NAME='DSL-mxstream'        # Arbitraire, mais c'est évident
    CIRC_1_TYPE='ppp'                 # C'est un circuit PPP
    CIRC_1_ENABLED='yes'
    CIRC_1_NETS_IPV4_N='1'
    CIRC_1_NETS_IPV4_1='0.0.0.0/0'    # Route par défaut pour Internet
    CIRC_1_CLASS_N='1'
    CIRC_1_CLASS_1='internet'         # Classe pour la connexion Internet
    CIRC_1_UP='yes'                   # Activer au moment du boot
    CIRC_1_TIMES='Mo-Su:00-24:0.0:Y'
    CIRC_1_USEPEERDNS='yes'           # Utiliser les serveurs DNS du fournisseur
    CIRC_1_PPP_TYPE='pptp'            # Client PPTP
    CIRC_1_PPP_USERID='anonymer'      # Authentification de l'utilisateur
    CIRC_1_PPP_PASSWORD='surfer'      # Authentification par mot de passe
    CIRC_1_PPP_PPTP_PEER='10.0.0.138' # Adresse du modem Internet dans le réseau PPTP
    #
    CIRC_CLASS_N='1'
    CIRC_CLASS_1='internet'           # Classe de tous les circuits Internet
\end{verbatim}
\end{example}

\noindent
Exemple 2 (Accès à Internet via le PPTP avec une adresse locale dynamique)~:

\begin{example}
\begin{verbatim}
    IP_NET_N='2'                      # (Au moins) deux réseaux (LAN + PPTP)
    IP_NET_1='192.168.6.0/24'         # Réseau local, configuration nécessaire
    IP_NET_1_DEV='eth0'               # Réseau local qui dépend de la première carte
    IP_NET_2='{DHCP-Inode}'           # Réseau PPTP configuré via le DHCP
    IP_NET_2_DEV='eth1'               # Modem Internet accroché à la deuxième carte
    #
    OPT_DHCP_CLIENT='yes'             # Circuits DHCP activé
    OPT_PPP='yes'                     # Client du circuits PPP activé
    OPT_PPP_PPTP='yes'                # Client du circuits PPTP activé
    #
    CIRC_N='2'                        # Deux circuits: DHCP et PPTP
    #
    CIRC_1_NAME='DHCP-Inode'          # Arbitraire, mais évident
    CIRC_1_TYPE='dhcp'                # C'est un circuit DHCP
    CIRC_1_ENABLED='yes'
    CIRC_1_NETS_IPV4_N='1'            # Ici nombre de station distante PPTP
    CIRC_1_NETS_IPV4_1='10.0.0.138/32'# (= Modem Internet) pour être accessible
    CIRC_1_DHCP_DEV='IP_NET_2_DEV'    # Carte Ethernet PPTP
    CIRC_1_UP='yes'                   # Activer au moment du boot
    #
    CIRC_2_NAME='PPTP-Inode'          # Arbitraire, mais évident
    CIRC_2_TYPE='ppp'                 # C'est un circuit PPP
    CIRC_2_ENABLED='yes'
    CIRC_2_PPP_TYPE='pptp'            # Client PPTP
    CIRC_2_PPP_USER='anonymer'        # Authentification de l'utilisateur
    CIRC_2_PPP_PASS='surfer'          # Authentification par mot de passe
    CIRC_2_PPP_FILTER='yes'           # Activer le filtrage du trafic
    CIRC_2_PPP_PPTP_PEER='10.0.0.138' # Adresse du modem Internet dans le réseau PPTP
    CIRC_2_NETS_IPV4_N='1'
    CIRC_2_NETS_IPV4_1='0.0.0.0/0'    # Route par défaut pour Internet
    CIRC_2_USEPEERDNS='yes'           # Utiliser les serveurs DNS du fournisseur
    CIRC_2_HUP_TIMEOUT='600'          # Couper après 10 minutes d'inactivité
    CIRC_2_UP='yes'                   # Activer au moment du démarrage
    CIRC_2_DEPS='DHCP-Inode'          # PPTP nécessite une configuration DHCP
\end{verbatim}
\end{example}

\noindent
Exemple 3 (Client VPN)~:

\begin{example}
\begin{verbatim}
    IP_NET_N='1'                      # (Au moins) un réseau (local)
    IP_NET_1='192.168.6.0/24'         # Réseau local, configuration nécessaire
    IP_NET_1_DEV='eth0'               # Réseau local qui dépend de la première carte
    #
    OPT_PPP='yes'                     # Circuit PPP activé
    OPT_PPP_ETHERNET='yes'            # Client du circuit PPPoE activé (DSL)
    OPT_PPP_PPTP='yes'                # Client du circuit PPTP activé (VPN)
    #
    CIRC_N='2'                        # Deux circuits: PPPoE (Internet) et PPTP
    #
    CIRC_1_NAME='DSL-Telekom'         # Arbitraire, mais évident
    CIRC_1_TYPE='ppp'                 # C'est un circuit PPP
    CIRC_1_ENABLED='yes'
    CIRC_1_NETS_IPV4_N='1'
    CIRC_1_NETS_IPV4_1='0.0.0.0/0'    # Route par défaut pour Internet
    CIRC_1_CLASS_N='1'
    CIRC_1_CLASS_1='internet'         # Classe pour la connexion Internet
    CIRC_1_UP='yes'                   # Activer au moment du boot
    CIRC_1_TIMES='Mo-Su:00-24:0.0:Y'
    CIRC_1_USEPEERDNS='yes'           # Utiliser les serveurs DNS du fournisseur
    CIRC_1_PPP_TYPE='ethernet'        # Client PPPoE
    CIRC_1_PPP_USERID='anonymer'      # Authentification de l'utilisateur
    CIRC_1_PPP_PASSWORD='surfer'      # Authentification par mot de passe
    CIRC_1_PPP_ETHERNET_TYPE='kernel' # Kernel pour compacter les paquets PPPoE
    CIRC_1_PPP_ETHERNET_DEV='eth1'    # Modem DSL accroché à la deuxième carte
    #
    CIRC_2_NAME='VPN-Firma'           # Arbitraire, mais évident
    CIRC_2_TYPE='ppp'                 # C'est un circuit PPP
    CIRC_2_ENABLED='yes'
    CIRC_2_NETS_IPV4_N='1'
    CIRC_2_NETS_IPV4_1='10.11.12.0/24'# Réseau de la société
    CIRC_2_DEPS='internet/ipv4'       # Connexion Internet IPv4 si besoin
                                      # pour la société
    CIRC_2_UP='yes'                   # Activer au moment du boot
    CIRC_2_TIMES='Mo-Su:00-24:0.0:Y'
    CIRC_2_PPP_TYPE='pptp'            # Client PPTP
    CIRC_2_PPP_USERID='mustermann'    # Authentification de l'utilisateur
    CIRC_2_PPP_PASSWORD='geheim'      # Authentification par mot de passe
    CIRC_2_PPP_PPTP_PEER='192.0.2.1'  # Adresse du serveur PPTP de la société
    #
    CIRC_CLASS_N='1'
    CIRC_CLASS_1='internet'           # Classe pour la connexion Internet
\end{verbatim}
\end{example}

\noindent
Exemple 4 (Serveur VPN)~:

\begin{example}
\begin{verbatim}
    OPT_PPP='yes'                        # Circuit PPP activé
    OPT_PPP_PPTP_SERVER='yes'            # Circuit du serveur PPTP activé
    OPT_PPP_PEERS='yes'                  # Pour stocker les informations d'identification
    PPP_PEER_N='1'                       # 1x nombre de donnée d'enregistrement
    PPP_PEER_1_USERID='user'             # Nom d'utilisateur du client
    PPP_PEER_1_PASSWORD='pass'           # Mot de passe du client
    PPP_PEER_1_CIRCUITS='pptp-eth1'      # Enregistrement du circuit PPTP valide
    #
    CIRC_N='1'
    CIRC_1_NAME='pptp-eth1'              # Arbitraire, mais évident
    CIRC_1_TYPE='ppp'                    # C'est un circuit PPP
    CIRC_1_ENABLED='yes'
    CIRC_1_UP='yes'                      # Activer au moment du boot
    CIRC_1_TIMES='Mo-Su:00-24:0.0:Y'
    CIRC_1_PROTOCOLS='ipv4'              # Exécuter IPv4 sur la connexion
    CIRC_1_PPP_TYPE='pptp-server'        # Serveur PPTP
    CIRC_1_PPP_PEER_AUTH='yes'           # Authentification du client obligatoire
    CIRC_1_PPP_COMP_MPPE='yes'           # Utiliser le cryptage
    CIRC_1_PPP_LOCALIP='192.168.222.1'   # Adresse IP du serveur
    CIRC_1_PPP_REMOTEIP='192.168.222.2'  # Adresse IP de départ des clients
    CIRC_1_PPP_PPTP_SERVER_SESSIONS='10' # max. 10 tunnels
\end{verbatim}
\end{example}
