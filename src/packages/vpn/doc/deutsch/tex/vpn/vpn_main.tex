% Last Update: $Id$
\section{VPN - Unterstützung virtueller privater Netzwerke}

Dieses Paket erlaubt es, gesicherte Verbindungen zwischen privaten Netzwerken
über öffentliche, aber unsichere Netzwerke aufzubauen.

\subsection{PPTP-Tunnel}

PPTP\footnote{``Point-to-Point Tunneling Protocol'', siehe RFC 2637} bietet
eine Möglichkeit, einen privaten Kanal über ein öffentliches Netzwerk
aufzubauen. Dabei wird der Tunnelaufbau und -abbau über ein spezielles
TCP/IP-Kontrollprotokoll gesteuert. Die eigentlichen Nutzdaten werden in
PPP-Paketen verpackt, die über einen GRE-Tunnel\footnote{``Generic Routing
Encapsulation'', ein Protokoll, welches beliebige andere Protokolle kapseln und
über IP-Netzwerke transportieren kann, siehe RFC 2784} geschickt werden.

In Österreich (und anderen europäischen Ländern) wird PPTP zusätzlich als
Protokoll zwischen Router und DSL-Modem verwendet. Im Gegensatz zu PPPoE und
PPPoA, die beide noch unterhalb der IP-Ebene arbeiten (Sicherungsschicht,
``Link Layer''), gibt es bei PPTP wie oben beschrieben zwei Datenströme. Somit
benötigt man für DSL über PPTP im Gegensatz zu anderen DSL-Zugangsmethoden auch
für die für PPTP reservierte Ethernet-Karte eine IP-Adresse. Die ist je nach
Provider entweder fest vorgegeben oder muss via DHCPv4 konfiguriert werden.
Mehr dazu steht in der Beschreibung der Variable \var{CIRC\_x\_PPP\_PPTP\_PEER}.

Ist man auf Grund eines DSL-Anschlusses oder einer veralteten VPN-Gegenstelle
nicht gezwungen, PPTP zu nutzen, so wird davon abgeraten, PPTP als VPN-Lösung
zu verwenden. Die Verschlüsselung in PPTP gilt als geknackt\footnote{siehe
\altlink{http://heise.de/-1701365}}, so dass man über PPTP-Tunnel nicht
sicherheitskritische Daten verschicken sollte. Für einen solchen Einsatzweck
bilden Tunnel auf der Basis von OpenVPN sicherlich die bessere Wahl.

\subsubsection{Ausgehende PPTP-Verbindungen}

Generell werden ausgehende PPTP-Verbindungen als PPP-Circuits konfiguriert
(siehe \jump{sect:ppp-circuits}{Circuits vom Typ ``ppp''}), d.\,h. es gilt:

\begin{example}
\begin{verbatim}
    CIRC_x_TYPE='ppp'
\end{verbatim}
\end{example}

Zusätzlich muss das \verb+OPT_PPP_PPTP+ aktiviert werden:

\begin{description}
\config{OPT\_PPP\_PPTP}{OPT\_PPP\_PPTP}{OPTPPTP}

Diese Variable aktiviert die Unterstützung für PPTP. Damit auch tatsächlich
eine PPTP-Verbindung genutzt werden kann, muss mindestens ein PPP-Circuit den
Typ ``pptp'' besitzen, d.\,h. es muss zusätzlich gelten

\begin{example}
\begin{verbatim}
    CIRC_x_TYPE='ppp'
    CIRC_x_PPP_TYPE='pptp'
\end{verbatim}
\end{example}

(wobei ``x'' einen gültigen Circuit-Index darstellt).

Standard-Einstellung: \verb+OPT_PPP_PPTP='no'+

Beispiel: \verb+OPT_PPP_PPTP='yes'+

\config{CIRC\_x\_PPP\_PPTP\_TYPE}{CIRC\_x\_PPP\_PPTP\_TYPE}{CIRCxPPPPPTPTYPE}

Diese Variable definiert, wer die PPP-Datenpakete in GRE-Paketen kapselt und
über die Leitung schickt. Dies kann durch ein Benutzerprogramm (\texttt{pptp})
oder durch den Kern erfolgen. Mittels \var{CIRC\_x\_PPP\_PPTP\_TYPE} wird die
Art und Weise der GRE-Paketerzeugung definiert, siehe hierzu Tabelle
\ref{tab:pptp-type}.

\begin{table}[h!]
  \centering
  \begin{tabular}{|l|p{10cm}|}
    \hline
    Wert & Beschreibung \\
    \hline
    kernel & Die PPP-Pakete werden direkt an den Linux-Kern
    gereicht, der daraus GRE-Pakete macht. Dadurch entfällt die
    Kommunikation mit einem zweiten Prozess und damit eine Menge
    Kopier- und Scheduling-Aufwand, was wiederum zu geringerer Prozessorlast
    führt.\\
    daemon & Die Pakete werden durch den \texttt{pptp}-Dämon erzeugt; die
    Kommunikation zwischen \texttt{pppd} und \texttt{pptp} erfolgt asynchron.
    Das bedeutet, dass der Datenstrom mit Anfang- und Ende-Markern versehen
    wird, damit der \texttt{pptp}-Dämon die einzelnen Pakete auseinanderhalten
    kann. Aufgrund des zweiten Prozesses und der zusätzlichen Markierungen ist
    diese Methode aufwändiger als die Methode ``kernel''.\\
    \hline
  \end{tabular}
  \caption{Arten der GRE-Paketerzeugung}
  \label{tab:pptp-type}
\end{table}

Momentan ist ``daemon'' die einzige unterstützte Methode (und somit auch
Standard, falls der Typ nicht angegeben wird). Eine Erweiterung des Pakets um
das Nutzen des entsprechenden \texttt{pptp}-Kernel-Moduls steht noch aus.

Standard-Einstellung: \verb+CIRC_x_PPP_PPTP_TYPE='daemon'+

\config{CIRC\_x\_PPP\_PPTP\_PEER}{CIRC\_x\_PPP\_PPTP\_PEER}{CIRCxPPPPPTPPEER}

Hier wird die IP-Adresse der PPTP-Gegenstelle eingetragen.

Nutzt man PPTP für einen DSL-Internet-Zugang, muss dazu passend die IP-Adresse
der fli4l-PPTP-Ethernet-Karte gewählt werden. In Tabelle \ref{tab:pptp-provider}
sind bekannte Konfigurationsvarianten aufgelistet.

\begin{table}[htb]
  \centering
  \begin{tabular}{p{6cm}|p{4cm}|p{4cm}}
    Provider &
    lokale IP-Adresse\newline(\var{IP\_NET\_2}) &
    entfernte IP-Adresse\newline(\var{CIRC\_x\_PPP\_PPTP\_PEER}) \\
    \hline
    Telekom Austria (Östereich)      & 10.0.0.140/29 & 10.0.0.138 \\
    mxstream (Niederlande, Dänemark) & 10.0.0.140/29 & 10.0.0.138 \\
    Inode xDSL (Österreich)          & via DHCPv4    & 10.0.0.138 \\
  \end{tabular}
  \caption{Einstellungen für Provider, die PPTP nutzen}
  \label{tab:pptp-provider}
\end{table}

Für den Fall, dass DHCP zur Konfiguration der lokalen Netzwerk-Karte benötigt
wird, ist das dhcp\_client-Paket zu installieren und ein entsprechender
DHCPv4-Circuit für die jeweilige Ethernet-Karte (in der Regel eth1)
einzurichten. Ein Beispiel für Inode xDSL findet sich am Ende des Abschnitts.

\config{CIRC\_x\_PPP\_PPTP\_REORDER\_TIMEOUT}{CIRC\_x\_PPP\_PPTP\_REORDER\_TIMEOUT}{CIRCxPPPPPTPREORDERTIMEOUT}

Der PPTP-Client muss unter Umständen Pakete zwischenpuffern und umordnen.
Normalerweise wartet er 0,3 Sekunden auf ein ausstehendes Paket. Mit dieser
Variable kann man den Timeout zwischen 0.00 (gar nicht puffern) und 10.00
(max. 10 Sekunden warten) variieren. Die Zeiten müssen immer mit Punkt und zwei
Nachkommstellen angegeben werden.

Standard-Einstellung: \verb+CIRC_x_PPP_PPTP_REORDER_TIMEOUT='0.30'+

Beispiel: \verb+CIRC_1_PPP_PPTP_REORDER_TIMEOUT='1.00'+

\config{CIRC\_x\_PPP\_PPTP\_LOGLEVEL}{CIRC\_x\_PPP\_PPTP\_LOGLEVEL}{CIRCxPPPPPTPLOGLEVEL}

Mit dieser Variable kann konfiguriert werden, wieviel Ausgaben der PPTP-Client
produziert. Möglich sind 0 (wenig), 1 (mittel) und 2 (viel).

Standard-Einstellung: \verb+CIRC_x_PPP_PPTP_LOGLEVEL='1'+

Beispiel: \verb+CIRC_1_PPP_PPTP_LOGLEVEL='2'+

\end{description}

\subsubsection{Eingehende PPTP-Verbindungen}

Der fli4l kann auch konfiguriert werden, \emph{eingehende} PPTP-Verbindungen
anzunehmen, also als ein Server zu fungieren. Solche PPTP-Verbindungen werden
ebenfalls als PPP-Circuit konfiguriert (siehe
\jump{sect:ppp-circuits}{Circuits vom Typ ``ppp''}), d.\,h. es gilt:

\begin{example}
\begin{verbatim}
    CIRC_x_TYPE='ppp'
\end{verbatim}
\end{example}

Zusätzlich muss das \verb+OPT_PPP_PPTP_SERVER+ aktiviert werden:

\begin{description}
\config{OPT\_PPP\_PPTP\_SERVER}{OPT\_PPP\_PPTP\_SERVER}{OPTPPPPPTPSERVER}

Mit dieser Variable wird die Unterstützung für eingehende PPTP-Verbindungen
aktiviert.
Damit auch tatsächlich PPTP-Verbindungen angenommen werden können, muss
mindestens ein PPP-Circuit den Typ ``pptp-server'' besitzen, d.\,h. es muss
zusätzlich gelten

\begin{example}
\begin{verbatim}
    CIRC_x_TYPE='ppp'
    CIRC_x_PPP_TYPE='pptp-server'
\end{verbatim}
\end{example}

(wobei ``x'' einen gültigen Circuit-Index darstellt).

Standard-Einstellung: \verb+OPT_PPP_PPTP_SERVER='no'+

Beispiel: \verb+OPT_PPP_PPTP_SERVER='yes'+
\end{description}

Zu den allgemeinen Circuit-Variablen kommen die folgenden, für PPP-Circuits
des Typs ``pptp-server'' spezifischen Variablen hinzu:

\begin{description}
\config{CIRC\_x\_PPP\_PPTP\_SERVER\_LISTEN}{CIRC\_x\_PPP\_PPTP\_SERVER\_LISTEN}{CIRCxPPPPPTPSERVERLISTEN}

Mit dieser Variable kann die IPv4-Adresse festgelegt werden, an welcher der
PPTP-Server horcht. Wird diese Variable weggelassen, horcht der PPTP-Server
an \emph{allen} Schnittstellen des Routers.\footnote{Dies entspricht der
IPv4-Adresse \texttt{0.0.0.0}.}

Mit Hilfe der hier konfigurierten Adresse wird im Falle von
\jump{PFINPUTACCEPTDEF}{\var{PF\_INPUT\_ACCEPT\_DEF='yes'}} bzw.
\jump{PFOUTPUTACCEPTDEF}{\var{PF\_OUTPUT\_ACCEPT\_DEF='yes'}} die Firewall in den
INPUT- und OUTPUT-Ketten sowohl für das PPTP-Kontrollprotokoll auf TCP-Port
1723 als auch für die GRE-Pakete geöffnet. Fehlt diese Variable, wird die
Firewall so konfiguriert, dass die Kontroll- und Daten-Pakete an \emph{jeder}
Adresse des PPTP-Servers empfangen werden können.

Beispiel: \verb+CIRC_1_PPP_PPTP_SERVER_LISTEN='IP_NET_1_ADDR'+

\configlabel{CIRC\_x\_PPP\_PPTP\_SERVER\_ALLOW\_FROM\_N}{CIRCxPPPPPTPSERVERALLOWFROMN}
\config{CIRC\_x\_PPP\_PPTP\_SERVER\_ALLOW\_FROM\_y}{CIRC\_x\_PPP\_PPTP\_SERVER\_ALLOW\_FROM\_y}{CIRCxPPPPPTPSERVERALLOWFROMy}

Dieses Array enthält eine Liste von IPv4-Netzadressen, für die ein Zugriff auf
den PPTP-Server in der Firewall erlaubt wird. Mit Hilfe der hier konfigurierten
Adressen wird im Falle von
\jump{PFINPUTACCEPTDEF}{\var{PF\_INPUT\_ACCEPT\_DEF='yes'}} bzw.
\jump{PFOUTPUTACCEPTDEF}{\var{PF\_OUTPUT\_ACCEPT\_DEF='yes'}} die Firewall in den
INPUT- und OUTPUT-Ketten sowohl für das PPTP-Kontrollprotokoll auf TCP-Port
1723 als auch für die GRE-Pakete geöffnet. Fehlt dieses Array, wird die
Firewall so konfiguriert, dass die Kontroll- und Daten-Pakete von bzw. nach
\emph{überall} akzeptiert werden.

Beispiel:

\begin{example}
\begin{verbatim}
    CIRC_1_PPP_PPTP_SERVER_ALLOW_FROM_N='3'
    CIRC_1_PPP_PPTP_SERVER_ALLOW_FROM_1='IP_NET_1'
    CIRC_1_PPP_PPTP_SERVER_ALLOW_FROM_2='10.1.2.0/24'
    CIRC_1_PPP_PPTP_SERVER_ALLOW_FROM_3='{Labor}'
\end{verbatim}
\end{example}

\config{CIRC\_x\_PPP\_PPTP\_SERVER\_SESSIONS}{CIRC\_x\_PPP\_PPTP\_SERVER\_SESSIONS}{CIRCxPPPPPTPSERVERSESSIONS}

Diese Variable enthält die Anzahl der Verbindungen, die dieser PPTP-Server
maximal gleichzeitig verwalten kann. Maximal werden 255 Tunnel unterstützt.
\footnote{Diese Beschränkung resultiert daher, dass der verwendete PPTP-Dämon
einen Adressbereich nur in einer Komponente einer IPv4-Adresse erlaubt, also
z.B. ``192.168.222.0-254''. Da eine Komponente nur die Werte 0--255 annehmen
kann und der Wert 255 für die Broadcast-Adresse reserviert sind, resultiert
daraus die genannte Beschränkung.}

Standard-Einstellung: \verb+CIRC_x_PPP_PPTP_SERVER_SESSIONS='100'+

Beispiel: \verb+CIRC_1_PPP_PPTP_SERVER_SESSIONS='200'+

\end{description}

\subsubsection{Beispiele}

\noindent
Beispiel 1 (Internet-Zugang über PPTP mit fester lokaler Adresse):

\begin{example}
\begin{verbatim}
    IP_NET_N='2'                      # (mindestens) zwei Netze (LAN + PPTP)
    IP_NET_1='192.168.6.0/24'         # lokales Netz, wie benötigt konfigurieren
    IP_NET_1_DEV='eth0'               # lokales Netz hängt an erster Karte
    IP_NET_2='10.0.0.140/29'          # unsere Adresse im PPTP-Netz
    IP_NET_2_DEV='eth1'               # Internet-Modem hängt an zweiter Karte
    #
    OPT_PPP='yes'                     # PPP-Circuits aktivieren
    OPT_PPP_PPTP='yes'                # PPTP-Client-Circuits aktivieren
    #
    CIRC_N='1'
    CIRC_1_NAME='DSL-mxstream'        # beliebig, aber eindeutig
    CIRC_1_TYPE='ppp'                 # das ist ein PPP-Circuit
    CIRC_1_ENABLED='yes'
    CIRC_1_NETS_IPV4_N='1'
    CIRC_1_NETS_IPV4_1='0.0.0.0/0'    # Default-Route ins Internet
    CIRC_1_CLASS_N='1'
    CIRC_1_CLASS_1='internet'         # Klasse für Internet-Anbindung
    CIRC_1_UP='yes'                   # beim Booten aktivieren
    CIRC_1_TIMES='Mo-Su:00-24:0.0:Y'
    CIRC_1_USEPEERDNS='yes'           # DNS-Server des Providers nutzen
    CIRC_1_PPP_TYPE='pptp'            # PPTP-Client
    CIRC_1_PPP_USERID='anonymer'      # Benutzername zur Authentifizierung
    CIRC_1_PPP_PASSWORD='surfer'      # Passwort zur Authentifizierung
    CIRC_1_PPP_PPTP_PEER='10.0.0.138' # Adresse des Internet-Modems im PPTP-Netz
    #
    CIRC_CLASS_N='1'
    CIRC_CLASS_1='internet'           # Klasse aller Internet-Circuits
\end{verbatim}
\end{example}

\noindent
Beispiel 2 (Internet-Zugang über PPTP mit dynamisch zugewiesener lokaler
Adresse):

\begin{example}
\begin{verbatim}
    IP_NET_N='2'                      # (mindestens) zwei Netze (LAN + PPTP)
    IP_NET_1='192.168.6.0/24'         # lokales Netz, wie benötigt konfigurieren
    IP_NET_1_DEV='eth0'               # lokales Netz hängt an erster Karte
    IP_NET_2='{DHCP-Inode}'           # PPTP-Netz, via DHCP konfiguriert
    IP_NET_2_DEV='eth1'               # Internet-Modem hängt an zweiter Karte
    #
    OPT_DHCP_CLIENT='yes'             # DHCP-Circuits aktivieren
    OPT_PPP='yes'                     # PPP-Client-Circuits aktivieren
    OPT_PPP_PPTP='yes'                # PPTP-Client-Circuits aktivieren
    #
    CIRC_N='2'                        # zwei Circuits: DHCP und PPTP
    #
    CIRC_1_NAME='DHCP-Inode'          # beliebig, aber eindeutig
    CIRC_1_TYPE='dhcp'                # das ist ein DHCP-Circuit
    CIRC_1_ENABLED='yes'
    CIRC_1_NETS_IPV4_N='1'            # hierüber soll die PPTP-Gegenstelle
    CIRC_1_NETS_IPV4_1='10.0.0.138/32'# (= Internet-Modem) erreichbar sein
    CIRC_1_DHCP_DEV='IP_NET_2_DEV'    # die PPTP-Ethernet-Karte
    CIRC_1_UP='yes'                   # beim Booten aktivieren
    #
    CIRC_2_NAME='PPTP-Inode'          # beliebig, aber eindeutig
    CIRC_2_TYPE='ppp'                 # das ist ein PPP-Circuit
    CIRC_2_ENABLED='yes'
    CIRC_2_PPP_TYPE='pptp'            # PPTP-Client
    CIRC_2_PPP_USER='anonymer'        # Benutzername zur Authentifizierung
    CIRC_2_PPP_PASS='surfer'          # Passwort zur Authentifizierung
    CIRC_2_PPP_FILTER='yes'           # Datenverkehr-Filter aktivieren
    CIRC_2_PPP_PPTP_PEER='10.0.0.138' # Adresse des Internet-Modems im PPTP-Netz
    CIRC_2_NETS_IPV4_N='1'
    CIRC_2_NETS_IPV4_1='0.0.0.0/0'    # Default-Route ins Internet
    CIRC_2_USEPEERDNS='yes'           # DNS-Server des Providers nutzen
    CIRC_2_HUP_TIMEOUT='600'          # nach 10 Minuten Inaktivität auflegen
    CIRC_2_UP='yes'                   # beim Booten aktivieren
    CIRC_2_DEPS='DHCP-Inode'          # PPTP benötigt DHCP-Konfiguration
\end{verbatim}
\end{example}

\noindent
Beispiel 3 (VPN-Client):

\begin{example}
\begin{verbatim}
    IP_NET_N='1'                      # (mindestens) ein (lokales) Netz
    IP_NET_1='192.168.6.0/24'         # lokales Netz, wie benötigt konfigurieren
    IP_NET_1_DEV='eth0'               # lokales Netz hängt an erster Karte
    #
    OPT_PPP='yes'                     # PPP-Circuits aktivieren
    OPT_PPP_ETHERNET='yes'            # PPPoE-Client-Circuits aktivieren (DSL)
    OPT_PPP_PPTP='yes'                # PPTP-Client-Circuits aktivieren (VPN)
    #
    CIRC_N='2'                        # zwei Circuits: PPPoE (Internet) und PPTP
    #
    CIRC_1_NAME='DSL-Telekom'         # beliebig, aber eindeutig
    CIRC_1_TYPE='ppp'                 # das ist ein PPP-Circuit
    CIRC_1_ENABLED='yes'
    CIRC_1_NETS_IPV4_N='1'
    CIRC_1_NETS_IPV4_1='0.0.0.0/0'    # Default-Route ins Internet
    CIRC_1_CLASS_N='1'
    CIRC_1_CLASS_1='internet'         # Klasse für Internet-Anbindung
    CIRC_1_UP='yes'                   # beim Booten aktivieren
    CIRC_1_TIMES='Mo-Su:00-24:0.0:Y'
    CIRC_1_USEPEERDNS='yes'           # DNS-Server des Providers nutzen
    CIRC_1_PPP_TYPE='ethernet'        # PPPoE-Client
    CIRC_1_PPP_USERID='anonymer'      # Benutzername zur Authentifizierung
    CIRC_1_PPP_PASSWORD='surfer'      # Passwort zur Authentifizierung
    CIRC_1_PPP_ETHERNET_TYPE='kernel' # Kernel soll PPPoE-Pakete packen
    CIRC_1_PPP_ETHERNET_DEV='eth1'    # DSL-Modem hängt an zweiter Karte
    #
    CIRC_2_NAME='VPN-Firma'           # beliebig, aber eindeutig
    CIRC_2_TYPE='ppp'                 # das ist ein PPP-Circuit
    CIRC_2_ENABLED='yes'
    CIRC_2_NETS_IPV4_N='1'
    CIRC_2_NETS_IPV4_1='10.11.12.0/24'# Firmennetz
    CIRC_2_DEPS='internet/ipv4'       # Verbindung zur Firma benötigt
                                      # IPv4-Internet
    CIRC_2_UP='yes'                   # beim Booten aktivieren
    CIRC_2_TIMES='Mo-Su:00-24:0.0:Y'
    CIRC_2_PPP_TYPE='pptp'            # PPTP-Client
    CIRC_2_PPP_USERID='mustermann'    # Benutzername zur Authentifizierung
    CIRC_2_PPP_PASSWORD='geheim'      # Passwort zur Authentifizierung
    CIRC_2_PPP_PPTP_PEER='192.0.2.1'  # Adresse des PPTP-Servers der Firma
    #
    CIRC_CLASS_N='1'
    CIRC_CLASS_1='internet'           # Klasse aller Internet-Circuits
\end{verbatim}
\end{example}

\noindent
Beispiel 4 (VPN-Server):

\begin{example}
\begin{verbatim}
    OPT_PPP='yes'                        # PPP-Circuits aktivieren
    OPT_PPP_PPTP_SERVER='yes'            # PPTP-Server-Circuits aktivieren
    OPT_PPP_PEERS='yes'                  # zum Speichern der Anmeldedaten
    PPP_PEER_N='1'                       # 1x Anmeldedaten hinterlegen
    PPP_PEER_1_USERID='user'             # Benutzername vom Client
    PPP_PEER_1_PASSWORD='pass'           # Passwort vom Client
    PPP_PEER_1_CIRCUITS='pptp-eth1'      # Anmeldedaten gelten für PPTP-Circuit
    #
    CIRC_N='1'
    CIRC_1_NAME='pptp-eth1'              # beliebig, aber eindeutig
    CIRC_1_TYPE='ppp'                    # das ist ein PPP-Circuit
    CIRC_1_ENABLED='yes'
    CIRC_1_UP='yes'                      # beim Booten aktivieren
    CIRC_1_TIMES='Mo-Su:00-24:0.0:Y'
    CIRC_1_PROTOCOLS='ipv4'              # IPv4 soll über die Verbindung laufen
    CIRC_1_PPP_TYPE='pptp-server'        # PPTP-Server
    CIRC_1_PPP_PEER_AUTH='yes'           # Client-Authentifizierung ist Pflicht
    CIRC_1_PPP_COMP_MPPE='yes'           # benutze Verschlüsselung
    CIRC_1_PPP_LOCALIP='192.168.222.1'   # IP-Adresse des Servers
    CIRC_1_PPP_REMOTEIP='192.168.222.2'  # Start-IP-Adresse der Clients
    CIRC_1_PPP_PPTP_SERVER_SESSIONS='10' # max. 10 Tunnel
\end{verbatim}
\end{example}

\subsection{fastd-Tunnel}

Mit fastd\footnote{Fast and Secure Tunnelling Daemon} steht eine
schnelle und kleine Alternative zu bekannten VPN-Lösungen bereit. Ziele
bei der Entwicklung waren folgende Punkte:

\begin{itemize}
	\item sehr kleine Binärdatei (um unter OpenWRT auf Geräten mit sehr
		wenig Speicher eingesetzt werden zu können)
	\item austauschbare Krypto-Methoden
	\item Nutzung von UDP, um auch hinter NAT eingesetzt werden zu können
	\item Unterstützung von 1:1 und 1:n Szenarios
\end{itemize}

In vielen Freifunk-Projekten ist fastd Bestandteil der auf Gluon
basierenden Firmware. Während dort in der Regel \glqq klar\grqq{} ist,
was Client und Server ist, wird dies nicht durch das verwendete
Protokoll bestimmt, sondern nur durch die Nutzung. Für fastd werden
daher neben der Serverinstanz bei jedem beteiligten
Kommunikationspartner ein oder mehrere sogenannte Peers definiert. Für
die Konfiguration der jeweiligen Gegenstelle kehren sich damit die
Rollen um: was auf der einen Seite Serverdaemon ist, muss auf der
anderen Seite als Peer konfiguriert werden und umgekehrt.

\subsubsection{Generelle Einstellungen}

\begin{description}
\config{OPT\_FASTD}{OPT\_FASTD}{OPTFASTD}

Diese Variable aktiviert den fastd. Um tatsächlich Verbindungen
aufzubauen, müssen diese natürlich noch entsprechend konfiguriert
werden.

Standard-Einstellung: \verb+OPT_FASTD='no'+

Beispiel: \verb+OPT_FASTD='yes'+

\config{FASTD\_SECRET}{FASTD\_SECRET}{FASTSECRET}

Setzt den geheimen Schlüssel dieser fastd-Instanz. Ein Schlüsselpaar
kann beispielsweise mit dem Programm fastd erzeugt werden, welches auf
dem Router oder auf einem beliebigen anderen Linux-Rechner laufen kann:

\begin{example}
\begin{verbatim}
# fastd --generate-key
2016-01-06 18:54:13 +0100 --- Info: Reading 32 bytes from /dev/random...
Secret: d8dec7f65c89a39561ecde12623e8051d1d7c4286253b119f155e3fe169cd161
Public: e89e34e53c35bd6d750558a5e747fa72f25fbc922c86625b705ef1aa1865e32a
\end{verbatim}
\end{example}

Wird diese Variable leer gelassen, erzeugt das Startscript ein
Schlüsselpaar und legt den Inhalt in der Datei
\texttt{/tmp/fastd.secret} ab, die man sich dann vom Router runter
kopieren kann. Der fastd Service wird dann nicht gestartet!

Standard-Einstellung: \verb+FASTD_SECRET=''+

Beispiel: \verb+FASTD_SECRET='d8dec7f65c89a39561ecde12623e8051d1d7c4286253b119f155e3fe169cd161'+

\end{description}

