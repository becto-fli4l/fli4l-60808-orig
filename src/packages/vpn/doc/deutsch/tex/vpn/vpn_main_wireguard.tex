% Last Update: $Id: vpn_main.tex 44984 2016-03-12 10:17:48Z alex $

\subsection{WireGuard}

WireGuard\footnote{\altlink{https://www.wireguard.com/}} ist eine recht
neue Alternative zu bekannten VPN-Lösungen wie OpenVPN oder IPSec. Ziel
ist ein modernes, schnelles, einfaches VPN, das state-of-the-Art
Verschlüsselung einsetzt. IPv4, IPv6 und auch Dual-Stack sind mit
WireGuard kein Problem.

Ziele bei der Entwicklung waren unter anderem

\begin{itemize}
    \item Einfachheit
    \item Geschwindigkeit
    \item state-of-the-art Kryptographie
    \item schneller, einfacher, schlanker als IPsec
    \item performanter als OpenVPN
    \item genauso einfach zu konfigurieren wie SSH
\end{itemize}

Diese Einfachheit zeigt sich auch in der Konfiguration unter fli4l.


\subsubsection{Generelle Einstellungen}

\begin{description}

\config{OPT\_WIREGUARD}{OPT\_WIREGUARD}{OPTWIREGUARD}

Standard-Einstellung: \verb+OPT_WIREGUARD='no'+

Diese Variable aktiviert das Paket WireGuard. Zusätzlich müssen die einzelnen
Verbindungen oder Clients (peers) konfiguriert werden.

Beispiel: \verb+OPT_WIREGUARD='yes'+

\end{description}


\subsubsection{WireGuard Server}

WireGuard VPNs sind kein Client-Server-Modell im eigentlichen Sinne. Vielmehr werden
VPN-Verbindungen immer zwischen zwei Peers aufgebaut. Dies bringt den Vorteil mit
sich, dass die IP-Adressen beider Peers floating sein können. Beide Peers werden
Änderungen der IP-Adresse automatisch übernehmen ohne weitere Konfiguration.
Entscheidend ist nur, dass die ankommenden UDP-Pakete richtig signiert und
verschlüsselt sind (siehe Private/Public Key unten). Weder im fli4l noch bei den
Peers ist hierfür spezielle Konfiguration erforderlich.

Der Einfachheit halber sprechen wir im Weiteren teils dennoch von Server (damit
ist der fli4l gemeint) und Client (damit ist dann die Gegenstelle wie z.\,B. das Handy,
der PC oder auch ein anderer fli4l gemeint)

\achtung{Achtung:} Zusätzlich zur Konfiguration von WireGuard in
\texttt{vpn.txt} muss der Paketfilter entsprechend konfiguriert
werden. Dies geschieht über die normalen Wege in \texttt{base.txt}.
Wichtig sind v.\,a.:

\begin{description}
    \item[Port öffnen] In der Standardeinstellung werden die WireGuard Ports automatisch
    im Paketfilter geöffnet. Sollte dies nicht gewünscht sein, kann
    \var{WIREGUARD\_DEFAULT\_OPEN\_PORT='no'} gesetzt werden.
    In diesem Fall müssen die Ports dann manuell geöffnet werden z.\,B.
        \begin{example}
        \begin{verbatim}
            PF_INPUT[]='prot:udp 50002 ACCEPT'  # in base.txt
        \end{verbatim}
        \end{example}
    \item[Paketfilter-Regeln] entsprechende FORWARD- und MASQ-Regeln im Paketfilter
\end{description}


\paragraph{Erforderliche Parameter Server}

\begin{description}

\config{WIREGUARD\_N}{WIREGUARD\_N}{WIREGUARDN}

Legt fest wie viele WireGuard Konfigurationen ('Server') wir haben wollen.

Beispiel: \verb+WIREGUARD_N='1'+


\config{WIREGUARD\_x\_NAME}{WIREGUARD\_x\_NAME}{WIREGUARDxNAME}

Der Name dieser WireGuard-Instanz


\config{WIREGUARD\_x\_LOCAL\_IP4}{WIREGUARD\_x\_LOCAL\_IP4}{WIREGUARDxLOCALIP4}

Lokale IP-Adresse (bei \textbackslash32) oder Netzwerk (z.\,B. \textbackslash24)
dieser WireGuard-Instanz


\config{WIREGUARD\_x\_LISTEN\_PORT}{WIREGUARD\_x\_LISTEN\_PORT}{WIREGUARDxLISTENPORT}

Port, auf dem WireGuard lauschen soll. Der Port muss zusätzlich in \texttt{base.txt} geöffnet
werden, z.\,B.:

\begin{example}
\begin{verbatim}
    WIREGUARD_1_LISTEN_PORT='50002'     # in vpn.txt
    PF_INPUT[]='prot:udp 50002 ACCEPT'  # in base.txt
\end{verbatim}
\end{example}


\config{WIREGUARD\_x\_PRIVATE\_KEY}{WIREGUARD\_x\_PRIVATE\_KEY}{WIREGUARDxPRIVATEKEY}

Standardwert: \verb+WIREGUARD_x_PRIVATE_KEY='auto'+

Der private Schlüssel dieser WireGuard-Instanz.

Es gibt zwei Möglichkeiten:

\begin{description}
    \item[auto] Private und zugehöriger Public Key werden automatisch erstellt
    \item[\textit{some key}] ein gültiger WireGuard Private Key
\end{description}

Der Key kann manuell auf dem fli4l per

\begin{example}
\begin{verbatim}
    wg genkey > privatekey
\end{verbatim}
\end{example}

erstellt werden. Einfacher ist es aber sicherlich, beim ersten Start hier 'auto' zu
konfigurieren und den erstellten Private Key dann nach Start aus dem Webinterface
des fli4l zu übernehmen. Anders als die Private Keys der Peers muss der Private Key 
des Servers auf dem fli4l verbleiben.

\end{description}


\paragraph{Optionale Parameter Server}

\begin{description}


\config{WIREGUARD\_DEFAULT\_OPEN\_PORT}{WIREGUARD\_DEFAULT\_OPEN\_PORT}{WIREGUARDDEFAULTOPENPORT}

Standardwert: \verb+WIREGUARD_DEFAULT_OPEN_PORT='yes'+

Optional: Legt fest ob im Paketfilter automatisch Regeln angelegt werden sollen, um eingehende
WireGuard-Pakete zu akzeptieren. Damit werden alle in \var{WIREGUARD\_x\_LISTEN\_PORT} definierten 
UDP-Ports automatisch geöffnet.


\config{WIREGUARD\_x\_DEFAULT\_ALLOWED\_IPS\_N}{WIREGUARD\_x\_DEFAULT\_ALLOWED\_IPS\_N}{WIREGUARDxDEFAULTALLOWEDIPSN}

Optional: Anzahl der IPs bzw. Netzwerke, die den Peers vorgegeben werden sollen, diese durch den VPN-Tunnel zu routen.

Standardwert: \verb+WIREGUARD_x_DEFAULT_ALLOWED_IPS_N='0'+


\config{WIREGUARD\_x\_DEFAULT\_ALLOWED\_IPS}{WIREGUARD\_x\_DEFAULT\_ALLOWED\_IPS}{WIREGUARDxDEFAULTALLOWEDIPS}

Optional: IPs bzw. Netzwerke, die für alle Peers durch den VPN-Tunnel
geroutet werden sollen. Hier angegebene Netze werden in der
Peer-Config, die im Webinterface als QR-Code oder Datei einsehbar ist,
entsprechend übernommen. Bei manueller Konfiguration der Peers wären
diese Werte in \var{AllowedIPs} anzugeben.

Es können sowohl IPv4- als auch IPv6-Netzwerke angegeben werden. Interne Identifier wie \var{IP\_NET\_x} und
\var{IP6\_NET\_x} sind ebenfalls möglich.

Beispiel:

\begin{example}
\begin{verbatim}
    WIREGUARD_1_DEFAULT_ALLOWED_IPS_N='2'
    WIREGUARD_1_DEFAULT_ALLOWED_IPS_1='IP_NET_1'
    WIREGUARD_1_DEFAULT_ALLOWED_IPS_2='10.0.0.2/24'
\end{verbatim}
\end{example}


\config{WIREGUARD\_x\_KEEP\_ALIVE}{WIREGUARD\_x\_KEEP\_ALIVE}{WIREGUARDxKEEPALIVE}

Standard-Einstellung: \var{WIREGUARD\_x\_KEEP\_ALIVE}='0'

Optional: Intervall in Sekunden, in dem WireGuard UDP-Pakete
versendet, um die Verbindung aufrecht zu erhalten. Da UDP an sich
zustandslos ist, ist dies im Normalfall nicht notwendig.

%  \begin{example}
%  \begin{verbatim}
%      WIREGUARD_x_KEEP_ALIVE='25'
%  \end{verbatim}
%  \end{example}


\config{WIREGUARD\_x\_LOCAL\_HOST}{WIREGUARD\_x\_LOCAL\_HOST}{WIREGUARDxLOCALHOST}

Standard-Einstellung: \var{WIREGUARD\_x\_LOCAL\_HOST}=''

Optional: Der DNS-Name, unter dem fli4l im Internet erreichbar ist.
Für die Funktion von WireGuard nicht erforderlich. Allerdings bietet
es sich an, die Variable auf einen Hostname zu setzen, der z.\,B. von
OPT\_DYNDNS aktualisiert wird. Die Peer-Konfiguration für die
Gegenstellen im WebInterface (QR-Code bzw. Downlaod der Config-Datei)
enthält dann diesen Wert als Gegenstelle.

\begin{example}
\begin{verbatim}
    WIREGUARD_x_LOCAL_HOST='some.domain.de'
\end{verbatim}
\end{example}


\config{WIREGUARD\_x\_LOCAL\_IP6}{WIREGUARD\_x\_LOCAL\_IP6}{WIREGUARDxLOCALIP6}

Optional: IPv6-Adresse des Servers.


\config{WIREGUARD\_x\_PUBLIC\_KEY}{WIREGUARD\_x\_PUBLIC\_KEY}{WIREGUARDxPUBLICKEY}

Standard-Einstellung: \var{WIREGUARD\_x\_PUBLIC\_KEY}=''

Optional: Public Key der lokalen WireGuard-Instanz (Server).

Da der Public Key aus dem Private Key errechnet werden kann, ist der Public Key
optional.

Die allermeisten Nutzer werden diese Variable nicht benötigen und stattdessen den
Private Key beim ersten Start automatisch erstellen lassen und danach den Private
Key aus dem Webinterface in die Konfiguration übernehmen. Der Public Key wird dabei
automatisch aus dem Private Key abgeleitet.

\begin{example}
\begin{verbatim}
    WIREGUARD_x_PRIVATE_KEY='auto'
\end{verbatim}
\end{example}


\config{WIREGUARD\_x\_PUSH\_DNS\_N}{WIREGUARD\_x\_PUSH\_DNS\_N}{WIREGUARDxPUSHDNSN}

Optional: Anzahl der DNS-Server, die dem Peer in der Konfiguration
mitgegeben werden. \achtung{Achtung:} WireGuard unterstützt kein Push
von Netzwerkparametern beim Verbindungsaufbau. Die DNS-Server werden
in der Peer-Config, die im Webinterface als QR-Code und Download
verfügbar ist, hinterlegt.

Standardwert: \verb+WIREGUARD_x_PUSH_DNS_N='0'+


\config{WIREGUARD\_x\_PUSH\_DNS\_x}{WIREGUARD\_x\_PUSH\_DNS\_x}{WIREGUARDxPUSHDNSx}

Optional: IPs von DNS-Servern, die in der Peer-Kon\-fi\-gu\-ra\-tion
hinterlegt werden. Achtung: WireGuard unterstützt kein Push von
Netzwerkparametern beim Verbindungsaufbau. Die DNS-Server werden in
der Peer-Config, die im Webinterface als QR-Code und Download
verfügbar ist, hinterlegt und somit erst angewendet, wenn der Peer
diese Konfiguration neu bekommt.

Es können sowohl IPv4- als auch IPv6-Adressen angegeben werden.

Beispiel:

\begin{example}
\begin{verbatim}
    WIREGUARD_1_PUSH_DNS_N='2'
    WIREGUARD_1_PUSH_DNS_1='10.0.0.1'
    WIREGUARD_1_PUSH_DNS_2='2001:4860:4860::8888'
\end{verbatim}
\end{example}

\end{description}


\paragraph{Erforderliche Parameter Client (Peer)}

\begin{description}

\config{WIREGUARD\_x\_PEER\_N}{WIREGUARD\_x\_PEER\_N}{WIREGUARDxPEERN}

Standardwert: \verb+WIREGUARD_x_PEER_N='0'+

Legt fest wie viele Peers wir konfigurieren wollen.

Beispiel: \verb+WIREGUARD_1_PEER_N='3'+


\config{WIREGUARD\_x\_PEER\_x\_NAME}{WIREGUARD\_x\_PEER\_x\_NAME}{WIREGUARDxPEERxNAME}

Der Name des WireGuard Peers. Dient nur zur Identifikation z.\,B. im Webinterface.


\config{WIREGUARD\_x\_PEER\_x\_LOCAL\_IP4}{WIREGUARD\_x\_PEER\_x\_LOCAL\_IP4}{WIREGUARDxPEERxLOCALIP4}

IP-Adresse (bei \textbackslash32) des WireGuard Peers bzw. Netzwerk (z.\,B. \textbackslash24), das
sich hinter dem Peer verbirgt (z.\,B. bei Site-2-Site VPN). Die Adresse muss im selben Netzwerk liegen,
wie die IP-Adresse des zugehörigen WireGuard Servers WIREGUARD\_x\_LOCAL\_IP4

Beispiel:

\begin{example}
\begin{verbatim}
    WIREGUARD_1_LOCAL_IP4='10.0.0.1/24'
    WIREGUARD_1_PEER_1_LOCAL_IP4='10.0.0.2/32'
    WIREGUARD_1_PEER_2_LOCAL_IP4='10.0.0.3/32'
\end{verbatim}
\end{example}


\config{WIREGUARD\_x\_PEER\_x\_PUBLIC\_KEY}{WIREGUARD\_x\_PEER\_x\_PUBLIC\_KEY}{WIREGUARDxPEERxPUBLICKEY}

Der Public Key des WireGuard Peers.

Zum Verbindungsaufbau benötigt WireGuard zwar den Private Key der lokalen Instanz, vom Peer ist jedoch
natürlich nur der Public Key erforderlich.

Es gibt zwei Möglichkeiten:

\begin{description}
    \item[auto] es werden Private und zugehöriger Public Key des Peers erstellt. Dies bietet sich an
    wenn man die Konfiguration später einfach per QR-Code oder Download der Config-Datei zum Peer
    übertragen möchte
    \item[\textit{some key}] ein gültiger WireGuard Public Key des Peers
\end{description}

\end{description}


\paragraph{Optionale Parameter Client (Peer)}

\begin{description}

\config{WIREGUARD\_x\_PEER\_x\_ALLOWED\_IPS\_N}{WIREGUARD\_x\_PEER\_x\_ALLOWED\_IPS\_N}{WIREGUARDxPEERxALLOWEDIPSN}

Optional: Anzahl der IPs bzw. Netzwerke, die diesem einzelnen Peer vorgegeben werden sollen, diese durch den
VPN-Tunnel zu routen.

Standardwert: \verb+WIREGUARD_x_PEER_x_ALLOWED_IPS_N='0'+


\config{WIREGUARD\_x\_PEER\_x\_ALLOWED\_IPS\_x}{WIREGUARD\_x\_PEER\_x\_ALLOWED\_IPS\_x}{WIREGUARDxPEERxALLOWEDIPSx}

Optional: IPs bzw. Netzwerke, die der jeweilige Peer durch den
VPN-Tunnel routen soll. Hier angegebene Netze werden in der
Peer-Config, die im Webinterface als QR-Code oder Datei einsehbar ist,
entsprechend übernommen. Bei manueller Konfiguration der Peers wären
diese Werte in \var{AllowedIPs} anzugeben.  Die hier für jeden
einzelnen Peer konfigurierten Netzwerke werden mit den Netzen aus
\var{WIREGUARD\_x\_DEFAULT\_ALLOWED\_IPS} zusammengeführt und als
\var{AllowedIPs} in der jeweiligen Peer-Config hinterlegt.

Es können sowohl IPv4- als auch IPv6-Netzwerke angegeben werden. Interne Identifier wie \var{IP\_NET\_x} und
\var{IP6\_NET\_x} sind ebenfalls möglich.

Beispiel:

\begin{example}
\begin{verbatim}
    WIREGUARD_1_PEER_1_ALLOWED_IPS_N='2'
    WIREGUARD_1_PEER_1_ALLOWED_IPS_1='IP_NET_2'
    WIREGUARD_1_PEER_2_ALLOWED_IPS_2='192.168.1.0/24'
\end{verbatim}
\end{example}


\config{WIREGUARD\_x\_PEER\_x\_LOCAL\_IP6}{WIREGUARD\_x\_PEER\_x\_LOCAL\_IP6}{WIREGUARDxPEERxLOCALIP6}

Optional: IPv6-Adresse des Servers. Die Adresse muss im selben
Netzwerk liegen wie die Adresse des zugehörigen WireGuard Servers
WIREGUARD\_x\_LOCAL\_IP6.


\config{WIREGUARD\_x\_PEER\_x\_PRIVATE\_KEY}{WIREGUARD\_x\_PEER\_x\_PRIVATE\_KEY}{WIREGUARDxPEERxPRIVATEKEY}

Optional: Der Private Key des Peers. Im Gegensatz zum Public Key ist dieser nicht erforderlich. Falls man
die Konfiguration allerdings per QR-Code oder Download der Config-Datei im Webinterface auf den übernehmen möchte,
ist es sinnvoll, den Private Key des Peers hier anzugeben.

\begin{example}
\begin{verbatim}
    WIREGUARD_x_PEER_x_PUBLIC_KEY='auto'
\end{verbatim}
\end{example}

So erzeugte Private (und evtl. Public Keys des Peers sollten statisch in die Konfigurationsdatei übernommen werden, da
sonst bei jedem Neustart neue Keys erzeugt werden und der Client sich erst wieder verbinden kann, wenn diese
Keys entsprechend übertragen wurden.

Bei Private Keys der Peers sollte man sich überlegen, ob diese wirklich dauerhaft auf dem fli4l verbleiben sollen.
Für die VPN-Verbindung sind nur die Public Keys des Peers notwendig (sowie Private Key des Servers). Man kann 
also durchaus die Keys der Peers automatisch erstellen lassen, dann aber nur die jeweiligen Public Keys in die 
statische Konfiguration übernehmen. Selbst wenn Angreifer dann Zugriff auf auf den fli4l hätten, könnten sie 
keine VPN-Verbindung erstellen, da der Private Key des Peers fehlt.

Allerdings leidet darunter etwas die Einfachheit: Wenn der Private Key des Peers nicht auf dem fli4l bekannt ist,
ist die Peer-Konfiguration, die per QR-Code und Download angeboten wird, unvollständig und kann je nach 
Betriebssystem nicht eingelesen werden.

Man muss sich also entscheiden:

\begin{description}
    \item[Sicherheit] Private Keys der Peers nach automatischer Erstellung nicht in vpn.txt hinterlegen und höhere
    Sicherheit gegen eventuelle Eindringlinge.
    \item[Einfachheit] Private Keys der Peers in vpn.txt hinterlegen und vollständige Peer-Kon\-fi\-gu\-ra\-tio\-nen per 
    QR-Code und Download im Webinterface verfügbar haben.
\end{description}


\config{WIREGUARD\_x\_PEER\_x\_PRESHARED\_KEY}{WIREGUARD\_x\_PEER\_x\_PRESHARED\_KEY}{WIREGUARDxPEERxPRESHAREDKEY}

Optional: mittels eines zusätzlichem Preshared Keys zwischen Server
und Client kann die Sicherheit noch weiter erhöht werden.

Es gibt zwei Möglichkeiten:

\begin{description}
    \item[auto] Preshared Key wird bei (jedem!) Start automatisch erstellt
    \item[\textit{some key}] ein gültiger WireGuard Key
\end{description}

Wie auch die übrigen automatisch erstellten Keys, sollte auch der
Preshared Key nach Erstellung beim ersten Start statisch in die
Konfiguration übernommen werden, da ansonsten nach dem nächsten
Reboot (und damit verbundener automatischer Neuerstellung der Keys)
keine Verbindung mehr zum fli4l möglich ist.


\config{WIREGUARD\_x\_PEER\_x\_REMOTE\_HOST}{WIREGUARD\_x\_PEER\_x\_REMOTE\_HOST}{WIREGUARDxPEERxREMOTEHOST}

Optional: DNS-Name oder IP-Adresse des Peers. Der WireGuard Server
nutzt dies, um die Verbindung zu initiieren.  Falls nicht gesetzt,
wartet der WireGuard Server auf eingehende gültige (verschlüsselte
und signierte) Pakete von Peers und antwortet an die IP-Adresse, von
der das Paket kam.


\config{WIREGUARD\_x\_PEER\_x\_REMOTE\_PORT}{WIREGUARD\_x\_PEER\_x\_REMOTE\_PORT}{WIREGUARDxPEERxREMOTEPORT}

Optional: Der Port, auf dem der jeweilige Peer lauscht. Der WireGuard Server nutzt dies zum initialen
Verbindungsaufbau.

Standard-Einstellung: \var{WIREGUARD\_x\_PEER\_x\_REMOTE\_PORT}='51820'


\config{WIREGUARD\_x\_PEER\_x\_ROUTE\_TO\_N}{WIREGUARD\_x\_PEER\_x\_ROUTE\_TO\_N}{WIREGUARDxPEERxROUTETON}

Optional: Die Anzahl der Netzwerke, die über diese VPN-Verbindung
zum Peer geroutet werden sollen.

Standardwert: \verb+WIREGUARD_x_PEER_x_ROUTE_TO_N='0'+


\config{WIREGUARD\_x\_PEER\_x\_ROUTE\_TO\_x}{WIREGUARD\_x\_PEER\_x\_ROUTE\_TO\_x}{WIREGUARDxPEERxROUTETOx}

Optional: Die einzelnen Netzwerke, die über diesen Peer geroutet werden sollen.

Beispiel: Hinter Peer 1 sind die beiden Netzwerke 192.168.1.1/24 und 192.168.188.1/24 erreichbar:

\begin{example}
\begin{verbatim}
    WIREGUARD_1_PEER_1_ROUTE_TO_N='2'
    WIREGUARD_1_PEER_1_ROUTE_TO_1='192.168.1.1/24'
    WIREGUARD_1_PEER_1_ROUTE_TO_2='192.168.188.1/24'
\end{verbatim}
\end{example}

Bitte die entsprechenden Forward- und Post-Routing-Regeln in
\texttt{base.txt} nicht vergessen (ggf. auch auf der Gegenseite).

\end{description}

\subsubsection{Werkzeug wg-tool}

Zur Verwaltung einiger Grundfunktionen bringt Wireguard auf fli4l das 
Kommandozeilenwerkzeug wg-tool mit:

\begin{example}
\begin{verbatim}
    fli4l-rgb 4.0.0-r59812M # wg-tool
    This is a helper script to start and stop WireGuard connections
    Usage:
        wg-tool <wg interface> <up|down>
        wg-tool <wg interface> reresolve <peerName>
                where <peerName> is the name defined in WIREGUARD_x_PEER_x_NAME

    example: wg-tool wg1 down
    example: wg-tool wg1 reresolve peer5Name
\end{verbatim}
\end{example}

Im Normalbetrieb ist die Nutzung des Tools nicht erforderlich. Es erlaubt bei 
Bedarf jedoch auf einfache Art das Starten und Stoppen konfigurierter 
WireGuard-Instanzen.

Beispiel:
\begin{example}
\begin{verbatim}
    wg-tool wg1 down
\end{verbatim}
\end{example}

\var{wg interface} ergibt sich aus der Konfiguration in vpn.txt wobei die Zählung 
der Schnittstellen mit 0 beginnt. \var{WIREGUARD\_1} resultiert also in wg0.

Zusätzlich erlaubt wg-tool die erneute Auflösung des DNS-Namens eines Peers ohne 
den Server neu starten zu müssen, also ohne Unterbrechung der Verbindungen 
eventueller weiterer Peers, die mit der selben Server-Instanz verbunden sind. Im 
Normalbetrieb sollte auch dies nicht erforderlich sein. 

Eine bekannte Ausnahme ist, wenn der DNS-Name des Peers zum Boot-Zeitpunkt des fli4l 
nicht aufgelöst werden kann. In diesem Fall wird zwar der WireGuard Server gestartet, 
die Gegenstelle dieses einen Peers jedoch nicht gesetzt. Falls der Peer von sich 
aus die Verbindung initiiert, ist auch dies kein Problem, da bei WireGuard ein 
Verbindungsaufbau grundsätzlich von beiden Seiten (Peers) gleichermaßen initiiert 
werden kann.

Ansonsten bietet es sich an, entweder wg-tool manuell aufzurufen oder - meist 
einfacher - über Paket easycron zu automatisieren.

Die Parameter ergeben sich wieder aus der Konfiguration in vpn.txt:
\begin{description}
    \item[wg interface] ergibt sich aus  \var{WIREGUARD\_x}. Die Zählung der 
    Schnittstellen beginnt dabei mit 0. \var{WIREGUARD\_x} resultiert also in wg0.
    \item[peerName] ist der Name des Peers in \var{WIREGUARD\_x\_PEER\_x\_NAME}.
\end{description}

Beispiel:
\begin{example}
\begin{verbatim}
    WIREGUARD_2_PEER_3_NAME='peerTestName'
    WIREGUARD_2_PEER_3_REMOTE_HOST='somehost.dyndns.de'

    wg-tool wg1 reresolve peerTestName
\end{verbatim}
\end{example}

Oder alternativ per cron-Job über Paket easycron alle 6 Stunden:
\begin{example}
\begin{verbatim}
    EASYCRON_x_COMMAND='wg-tool wg1 reresolve peerTestName'
    EASYCRON_x_TIME='* */6 * * *'
\end{verbatim}
\end{example}


\subsubsection{WireGuard Beispiele}

\paragraph{Beispiel: minimal notwendige Konfiguration}

\noindent

\begin{example}
\begin{verbatim}
    OPT_WIREGUARD='yes'

    WIREGUARD_N='1'
    WIREGUARD_1_NAME='RoadWarriors'
    WIREGUARD_1_LOCAL_IP4='10.0.0.1/24'
    WIREGUARD_1_LISTEN_PORT='50002'

    WIREGUARD_1_PEER_N='1'
    WIREGUARD_1_PEER_1_NAME='peer1'
    WIREGUARD_1_PEER_1_LOCAL_IP4='10.0.0.2/32'
\end{verbatim}
\end{example}

Diese Konfiguration erzeugt eine lokalen WireGuard-Endpunkt auf dem fli4l mit
Namen \emph{RoadWarriors} sowie einen Peer mit Namen \emph{peer1}.

Alle notwendigen Krypto-Schlüssel (jeweils private und public key für Server und
Peer) werden automatisch erstellt und können später im Webinterface eingesehen und
in die Konfiguration übernommen werden.

Zusätzlich wird die Peer-Konfiguration im Webinterface

\begin{itemize}
    \item als QR-Code für Android- und iPhone-App
    \item als Download für Windows-/Linux-Clients
\end{itemize}

bereitgestellt.

Die automatisch erstellten private Keys sollten nach dem ersten Start
statisch in die Konfigurationsdatei \texttt{vpn.txt} übernommen werden.
Ansonsten werden sie bei jedem Neustart des Routers neu erzeugt und die
WireGuard-Verbindung kann erst wieder aufgebaut werden, wenn die
entsprechenden Parameter an den Peer neu übertragen wurden.

Eine vollständige, minimale Konfiguration, die auch Neustarts problemlos übersteht
sieht wie folgt aus:

\begin{example}
\begin{verbatim}
    OPT_WIREGUARD='yes'

    WIREGUARD_N='1'
    WIREGUARD_1_NAME='RoadWarriors'
    WIREGUARD_1_LOCAL_IP4='10.0.0.1/24'
    WIREGUARD_1_LISTEN_PORT='50002'
    WIREGUARD_1_PRIVATE_KEY='+PxRP6JmwDyM1R+KdeN+vIL2UY2WB+eG/AHLJ/OdsHo='

    WIREGUARD_1_PEER_N='1'
    WIREGUARD_1_PEER_1_NAME='peer1'
    WIREGUARD_1_PEER_1_LOCAL_IP4='10.0.0.2/32'
    WIREGUARD_1_PEER_1_PUBLIC_KEY='gP+moyiQtfeO07UH3ASGAy1tX0sLhivLIXGNP0IKbG8='
\end{verbatim}
\end{example}


\paragraph{Beispiel: RoadWarriors - von unterwegs ins Heimnetzwerk}

\noindent

% TODO  Erneut auf Trade-Off bzgl. private Key des Peers hinweisen. in vpn.txt angeben, damit
%       QR-Code funktioniert oder zur Sicherheit nur Public Key der Peers auf fli4l hinterlegen

Die folgende Konfiguration zeigt ein typisches Roadwarrior-Beispiel um von unterwegs jederzeit 
Zugriff aufs heimische Netz haben zu können. Sicherer Zugriff auf die Heimautomatisierung, den 
Fileserver zu Hause den digitalen Videorekorder, etc. können so umgesetzt werden.

Die notwendigen Private und Public Keys werden automatisch beim Booten erstellt und sollten nach 
dem ersten Start aus dem WireGuard Webinterface in die Konfigurationsdatei übernommen werden. 
Sonst werden beim nächsten Start neue Keys erzeugt und man kommt mit den vorherigen Keys von 
extern nicht mehr ins VPN.

Folgende Ziele wollen wir mit der Beispiel-Konfiguration erreichen

\begin{description}
    \item[Dual-Stack] Wir wollen sowohl IPv4- als auch IPv6-Adressen nutzen können.
    \item[DNS] DNS-Namen aus dem internen Netz sollen auch unterwegs aufgelöst werden, wir 
    wollen uns nicht die IP-Adressen der internen Hosts merken müssen.
    \item[kein Default-Gateway] Es soll nur Traffic ins interne Netz zu Hause durch den Tunnel 
    gerouten werden. Traffic des mobilen Geräts ins Internet soll weiterhin direkt über die 
    LTE- oder fremde WLAN-Verbindung gehen.
    \item[MASQ] Wir wollen kein Masquerading der Peer-Adressen sondern direktes Routing in das 
    interne Netz.
    \item[Network Prefix] Der Einfachheit halber wollen wir direkt Netzwerk-Präfixe nutzen.
\end{description}


Folgende Annahmen zu internen Netzen, IPs, etc. wurden getroffen

\begin{description}
    \item[internes Netz] IP\_NET\_1 ist das interne Netz, das per WireGuard angebunden erreicht 
    werden soll
    \item[DynDNS] Der fli4l ist unter router.dyndns.name erreichbar
    \item[IPv4] der WireGuard-Tunnel soll das IPv4-Netz 10.0.0.1/24 nutzen.
    \item[IPv6] der WireGuard-Tunnel soll das IPv6-Netz 'fd00:cfcf:abcd::/48' nutzen.
\end{description}


Um diese Konfiguration umzusetzen, müssen folgende Dateien angepasst werden

\begin{description}
    \item[vpn.txt] Für die Konfiguration von WireGuard-Server und Peers
    \item[base.txt] zur Konfiguration des Network-Prefixes sowie der notwendigen Regeln für 
    den Paketfilter
\end{description}


Konfiguration Wireguard Server und Peer in \verb+vpn.txt+:

\begin{example}
\begin{verbatim}
    OPT_WIREGUARD='yes'

    WIREGUARD[] {
        NAME='RoadWarriors'
        LOCAL_IP4='{wg-RoadWarriors}+0.0.0.1/24'
        LOCAL_IP6='{wg-RoadWarriors}+::1/64'
        PRIVATE_KEY='auto'
        LISTEN_PORT = '50002'
        LOCAL_HOST='router.dyndns.name'
        DEFAULT_ALLOWED_IPS[]='IP_NET_1'
        PUSH_DNS[]='10.0.0.1'
        PUSH_DNS[]='fd00:cfcf:abcd::1'

        PEER[] {
            NAME='Peer1'
            LOCAL_IP4='{wg-RoadWarriors}+0.0.0.2/32'
            LOCAL_IP6='{wg-RoadWarriors}+::2/128'
            PRIVATE_KEY='auto'
        }
    }
\end{verbatim}
\end{example}


Konfiguration der IPv4- und IPv6-Netze mittels Network Prefixes in \verb+base.txt+:

% TODO: better use 2001:DB8::/32 addresses? Explicitely designed for documentation
%       purposes. However then the example could not be directly copy & pasted...

\begin{example}
\begin{verbatim}
    OPT_NET_PREFIX='yes'
    NET_PREFIX
    {
        []
        {
            NAME='wg-RoadWarriors'  # Netzwerk-Präfix für den WireGuard-Tunnel
            TYPE='static'
            STATIC_IPV4='10.0.0.0/24'
            STATIC_IPV6='fd00:cfcf:abcd::/48'
        }
    }
\end{verbatim}
\end{example}


Paketfilter-Regeln für IPv4 in \verb+base.txt+:

\begin{example}
\begin{verbatim}
    # IPv4 Input-Regel, um Zugriff auf fli4l-Router zu erlauben
    PF_INPUT[]='{wg-RoadWarriors.prefix} ACCEPT'
    {
      COMMENT='Zugriff WireGuard RoadWarriors auf fli4l'
    }
    
    # IPv4 Forward-Regel, um Zugriff auf das Netz hinter dem fli4l zu erlauben
    PF_FORWARD[]='{wg-RoadWarriors.prefix} IP_NET_1 ACCEPT'
    {
      COMMENT='Weiterleitung WireGuard RoadWarriors in IP_NET_1 zulassen'
    }
    
    # IPv4 Postrouting-Regel, um Netz hinter fli4l ohne Masquerading zu erreichen
    PF_POSTROUTING[]='{wg-RoadWarriors.prefix} IP_NET_1 ACCEPT'
    {
      COMMENT='Zugriff WireGuard RoadWarriors auf IP_NET_1 ohne Masquerading'
    }

\end{verbatim}
\end{example}


Paketfilter-Regeln für IPv6 in \verb+base.txt+:

\begin{example}
\begin{verbatim}
    # IPv6 Input-Regel, um Zugriff auf fli4l-Router zu erlauben
    PF6_INPUT[]='{rgb-RoadWarriors.prefix} ACCEPT'
    {
      COMMENT='Zugriff WireGuard RoadWarriors auf fli4l'
    }
    
    # IPv6 Input-Regel, um Zugriff auf fli4l-Router zu erlauben
    PF6_FORWARD[]='{rgb-RoadWarriors.prefix} IPV6_NET_1 ACCEPT'
    {
      COMMENT='Zugriff WireGuard RoadWarriors auf IPV6_NET_1'
    }
    
    # IPv6 Postrouting-Regel, um Netz hinter fli4l ohne Masquerading zu erreichen
    PF6_POSTROUTING[]='{rgb-RoadWarriors.prefix} IPV6_NET_1 ACCEPT'
    {
      COMMENT='Zugriff WireGuard RoadWarriors auf IPV6_NET_1 ohne Masquerading'
    }
\end{verbatim}
\end{example}


\paragraph{Beispiel: RoadWarriors - kompletter Traffic durch den Tunnel (default route)}

\noindent

Falls der komplette Traffic des mobilen Geräts immer durch den WireGuard-Tunnel 
gehen soll (Default-Route durch den VPN-Tunnel), lässt sich dies  mit wenigen 
Anpassungen entsprechend ändern.

Konfiguration der Default-Routen des Peers. Die so erstellte Konfiguration muss 
per Download oder erneutem Einlesen des QR-Codes erneut auf das mobile Gerät 
übertragen werden

Neuer Peer in \verb+vpn.txt+

\begin{example}
\begin{verbatim}
    PEER[] {
        NAME='Peer2'
        LOCAL_IP4='{wg-RoadWarriors}+0.0.0.3/32'
        LOCAL_IP6='{wg-RoadWarriors}+::3/128'
        PRIVATE_KEY='auto'
        ALLOWED_IPS[] = '0.0.0.0/0'
        ALLOWED_IPS[] = '::/0'
    }
\end{verbatim}
\end{example}


Ergänzung der Paketfilter-Regeln für IPv4 und IPv6 in \verb+base.txt+:

\begin{example}
\begin{verbatim}
    # IPv4-Regel, um Zugriff auf Netze und Internet hinter dem fli4l zu erlauben
    PF_FORWARD[]='{wg-RoadWarriors.prefix} ACCEPT'
    {
      COMMENT='Weiterleitung WireGuard RoadWarriors in alle Netze des fli4l'
    }

    # IPv4-Regel, um Pakete ins Internet zu maskieren
    PF_POSTROUTING[]='{wg-RoadWarriors.prefix} {DHCPv4-cable} MASQUERADE'
    {
      COMMENT='MASQ WireGuard RoadWarriors to Internet'
    }

    # IPv6-Regel, um Zugriff auf Netze und Internet hinter dem fli4l zu erlauben
    PF6_FORWARD[]='{wg-RoadWarriors.prefix} ACCEPT'
    {
      COMMENT='Weiterleitung WireGuard RoadWarriors in alle Netze des fli4l'
    }

    # IPv6-Regel, um Pakete ins Internet zu maskieren
    PF6_POSTROUTING[]='{wg-RoadWarriors.prefix} {DHCPv6-cable} MASQUERADE'
    {
      COMMENT='MASQ WireGuard RoadWarriors to Internet'
    }
\end{verbatim}
\end{example}


Wir gehen dabei davon aus, dass in \verb+circuits.txt+:

\begin{description}
    \item[DHCPv4-cable] der Name für den IPv4-Circuit ist 
    \item[DHCPv6-cable] der Name für den IPv6-Circuit ist 
\end{description}


Die Beispiel-Konfiguration geht von einem fli4l hinter einer FritzBox aus wie im 
Wiki\footnote{siehe \altlink{https://web.nettworks.org/wiki/pages/viewpage.action?pageId=35651587}} beschrieben 


\paragraph{Beispiel: Site2Site-VPN}

\noindent

Die folgende Konfiguration zeigt ein typisches Beispiel, um zwei Standorte per WireGuard VPN-Tunnel 
zu verbinden. An beiden Standorten kümmert sich ein fli4l um Routing und verwaltet jeweils weitere 
interne Netze. 

Folgende Ziele wollen wir mit der Beispiel-Konfiguration erreichen

% TODO: 
\begin{description}
    \item[Dual-Stack] Wir wollen sowohl IPv4- als auch IPv6-Adressen nutzen können.
    \item[DNS] DNS-Namen aus dem jeweils anderen Netz sollen aufgelöst werden.
    \item[MASQ] Wir wollen kein Masquerading zwischen den Netzen sondern direktes Routing in die 
    jeweiligen interne Netz hinter den beiden Standorte.
    \item[VPN-Netz] Das VPN-Netz und die jeweiligen IPs dienen jeweils nur als Transportnetz zwischen 
                    den beiden fli4l. Sie tauchen nicht in den lokalen Netzen auf.
    \item[Network Prefix] Der Einfachheit halber wollen wir direkt Netzwerk-Präfixe nutzen.
\end{description}

Folgende Annahmen zu internen Netzen, IPs, etc. wurden getroffen

Standort A:
\begin{itemize}
    \item Ein internes IPv4-Netz 192.168.1.0/24
    \item Ein internes IPv6-Netz fd5e:aaaa:aaaa:aaaa::/64
    \item Der fli4l ist unter Dyndns fli4lA.dyndns.name erreichbar
    \item Lokale Domain ist standortA.home.somedomain.net (siehe \verb+DOMAIN_NAME+ in \verb+base.txt+)
\end{itemize}


Standort B:
\begin{itemize}
    \item Ein internes IPv4-Netz 192.168.2.0/24
    \item Ein internes IPv6-Netz fd5e:bbbb:bbbb:bbbb::/64
    \item Der fli4l ist unter Dyndns fli4lb.somedomain.org erreichbar
    \item Lokale Domain ist standortb.domain.de
\end{itemize}


Um diese Konfiguration umzusetzen, müssen auf beiden Seiten folgende Dateien angepasst werden

\begin{description}
    \item[vpn.txt] Für die Konfiguration von WireGuard-Server und Peers
    \item[base.txt] zur Konfiguration des Network-Prefixes sowie der notwendigen Regeln für 
    den Paketfilter
    \item[dns\_dhcp.txt] um den jeweils anderen fli4l als DNS-Server für lokale Namen 
    zu konfigurieren
\end{description}


\subsubsection{auf beiden fli4ls an beiden Standorten identisch}

\noindent

Netze in \verb+base.txt+:

\begin{example}
\begin{verbatim}
    OPT_NET_PREFIX='yes'
    NET_PREFIX
    {
        []
        {
            NAME='fli4a-intern'
            TYPE='static'
            STATIC_IPV4='192.168.1.0/24'
            STATIC_IPV6='fd5e:aaaa:aaaa::/48'
        }
        []
        {
            NAME='fli4b-intern'
            TYPE='static'
            STATIC_IPV4='192.168.2.0/24'
            STATIC_IPV6='fd5e:bbbb:bbbb::/48'
        }
        []
        {
            NAME='wg-Site2Site'
            TYPE='static'
            STATIC_IPV4='10.0.0.0/24'
            STATIC_IPV6='fd5e:cccc:cccc::/48'
        }
    }
\end{verbatim}
\end{example}


\subsubsection{Konfiguration fli4l an Standort A (neues Config-Format)}

\noindent

WireGuard VPN-Tunnel Standort A in \verb+vpn.txt+::

\begin{example}
\begin{verbatim}
    OPT_WIREGUARD='yes'

    WIREGUARD[] {
        NAME='Site2Site-VPN'
        LOCAL_IP4='{wg-Site2Site}+0.0.0.1/24'
        LOCAL_IP6='{wg-Site2Site}+::1/64'
        PRIVATE_KEY='iFwKnsJo/NO0WWiuMPxugdPOq9jqYhIP46uQi1AZj3E='  # unbedingt ändern
        LISTEN_PORT='50002'
        DEFAULT_ALLOWED_IPS[]='{fli4a-intern.prefix}'  # von allen Peers Pakete 
                                                       # für dieses Netz erlauben

        PEER[] {
            NAME='fli4l-B'
            LOCAL_IP4='{wg-Site2Site}+0.0.0.2/32'
            LOCAL_IP6='{wg-Site2Site}+::2/128'
            REMOTE_HOST='fli4lb.somedomain.org'
            REMOTE_PORT='50000'
            PUBLIC_KEY='eYyUIGxafeJwLC+BbFTJZ45CtKvCU3TlZ0DFCx9MlD0='
            ROUTE_TO[]='{fli4b-intern.prefix}'   # Pakete in dieses Netz durch den 
                                                 # Tunnel routen
        }
    }
\end{verbatim}
\end{example}


Paketfilter-Regeln Standort A für IPv4 und IPv6 in \verb+base.txt+:

\begin{example}
\begin{verbatim}
    # IPv4 Input-Regel, um Zugriff auf fli4l-Router zu erlauben
    PF_INPUT[]='{fli4b-intern.prefix} ACCEPT'
    {
      COMMENT='IPv4 Zugriff von Standort B auf fli4l'
    }
    
    # IPv4 Forward-Regel, um Zugriff auf das Netz hinter dem fli4l zu erlauben
    PF_FORWARD[]='{fli4b-intern.prefix} IP_NET_1 ACCEPT'
    {
      COMMENT='IPv4 Weiterleitung von Standort B auf IP_NET_1 zulassen'
    }
    
    # IPv4 Postrouting-Regel, um Netz hinter fli4l ohne Masquerading zu erreichen
    PF_POSTROUTING[]='{fli4b-intern.prefix} IP_NET_1 ACCEPT'
    {
      COMMENT='IPv4 Zugriff von Standort B auf IP_NET_1 ohne Masquerading'
    }

    # IPv6 Input-Regel, um Zugriff auf fli4l-Router zu erlauben
    PF6_INPUT[]='{fli4b-intern.prefix} ACCEPT'
    {
      COMMENT='IPv6 Zugriff von Standort B auf fli4l'
    }
    
    # IPv6 Input-Regel, um Zugriff auf fli4l-Router zu erlauben
    PF6_FORWARD[]='{fli4b-intern.prefix} IPV6_NET_1 ACCEPT'
    {
      COMMENT='IPvv Weiterleitung von Standort B auf IPV6_NET_1 zulassen'
    }
    
    # IPv6 Postrouting-Regel, um Netz hinter fli4l ohne Masquerading zu erreichen
    PF6_POSTROUTING[]='{fli4b-intern.prefix} IPV6_NET_1 ACCEPT'
    {
      COMMENT='IPv6 Zugriff  Zugriff Standort B auf IPV6_NET_1 ohne Masquerading'
    }
\end{verbatim}
\end{example}


DNS Zone Delegation an Standort A in \verb+dns_dhcp.txt+, um auch DNS-Namen des Netzes hinter fli4l-B aufzulösen:

\begin{example}
\begin{verbatim}
    OPT_DNS='yes'
    DNS_ZONE_DELEGATION_N='1'
    DNS_ZONE_DELEGATION_1_UPSTREAM_SERVER_N='1'
    DNS_ZONE_DELEGATION_1_UPSTREAM_SERVER_1_IP='192.168.2.1'   # an fli4l-B delegieren
    DNS_ZONE_DELEGATION_1_UPSTREAM_SERVER_1_QUERYSOURCEIP='IP_NET_1_IPADDR'

    DNS_ZONE_DELEGATION_1_DOMAIN_N='1'
    DNS_ZONE_DELEGATION_1_DOMAIN_1='standortb.domain.de'
    DNS_ZONE_DELEGATION_1_NETWORK_N='1'
    DNS_ZONE_DELEGATION_1_NETWORK_1='192.168.2.0/24'
\end{verbatim}
\end{example}

% =====================================================================

\subsubsection{Konfiguration fli4l an Standort B (altes Config-Format)}

\noindent

WireGuard VPN-Tunnel Standort B in \verb+vpn.txt+::

\begin{example}
\begin{verbatim}
    OPT_WIREGUARD='yes'

    WIREGUARD_N='1'
    WIREGUARD_NAME='Site2Site-VPN'
    WIREGUARD_LOCAL_IP4='{wg-Site2Site}+0.0.0.2/24'
    WIREGUARD_LOCAL_IP6='{wg-Site2Site}+::2/64'
    WIREGUARD_PRIVATE_KEY='eJ8s8+oyItWPUSs/5TDBroGWleEV7W/4p98qUY2xD2I='
    WIREGUARD_LISTEN_PORT='50002'

    WIREGUARD_1_PEER_N='1'
    WIREGUARD_1_PEER_1_NAME='fli4l-A'
    WIREGUARD_1_PEER_1_LOCAL_IP4='{wg-Site2Site}+0.0.0.1/32'
    WIREGUARD_1_PEER_1_LOCAL_IP6='{wg-Site2Site}+::1/128'
    WIREGUARD_1_PEER_1_REMOTE_HOST='fli4lA.dyndns.name'
    WIREGUARD_1_PEER_1_REMOTE_PORT='50002'
    WIREGUARD_1_PEER_1_PUBLIC_KEY='S9GsipIvFl36HkcXaET8v0G+UuAw6onDiC+22jJJjVs='
    WIREGUARD_1_PEER_1_ALLOWED_IPS__N='1'
    WIREGUARD_1_PEER_1_ALLOWED_IPS_1='{fli4a-intern.prefix}'
    WIREGUARD_1_PEER_1_ROUTE_TO_N='1'
    WIREGUARD_1_PEER_1_ROUTE_TO_1='{fli4a-intern.prefix}'
\end{verbatim}
\end{example}


Paketfilter-Regeln Standort B für IPv4 und IPv6 in \verb+base.txt+:

\begin{example}
\begin{verbatim}
    # IPv4 Input-Regel, um Zugriff auf fli4l-Router zu erlauben
    PF_INPUT[]='{fli4a-intern.prefix} ACCEPT'
    {
      COMMENT='IPv4 Zugriff von Standort A auf fli4l'
    }
    
    # IPv4 Forward-Regel, um Zugriff auf das Netz hinter dem fli4l zu erlauben
    PF_FORWARD[]='{fli4a-intern.prefix} IP_NET_1 ACCEPT'
    {
      COMMENT='IPv4 Weiterleitung von Standort A auf IP_NET_1 zulassen'
    }
    
    # IPv4 Postrouting-Regel, um Netz hinter fli4l ohne Masquerading zu erreichen
    PF_POSTROUTING[]='{fli4a-intern.prefix} IP_NET_1 ACCEPT'
    {
      COMMENT='IPv4 Zugriff von Standort A auf IP_NET_1 ohne Masquerading'
    }

    # IPv6 Input-Regel, um Zugriff auf fli4l-Router zu erlauben
    PF6_INPUT[]='{fli4a-intern.prefix} ACCEPT'
    {
      COMMENT='IPv6 Zugriff von Standort A auf fli4l'
    }
    
    # IPv6 Input-Regel, um Zugriff auf fli4l-Router zu erlauben
    PF6_FORWARD[]='{fli4a-intern.prefix} IPV6_NET_1 ACCEPT'
    {
      COMMENT='IPvv Weiterleitung von Standort A auf IPV6_NET_1 zulassen'
    }
    
    # IPv6 Postrouting-Regel, um Netz hinter fli4l ohne Masquerading zu erreichen
    PF6_POSTROUTING[]='{fli4a-intern.prefix} IPV6_NET_1 ACCEPT'
    {
      COMMENT='IPv6 Zugriff  Zugriff von Standort A auf IPV6_NET_1 ohne Masquerading'
    }
\end{verbatim}
\end{example}

DNS Zone Delegation an Standort B in \verb+dns_dhcp.txt+, um auch DNS-Namen des Netzes hinter fli4l-A aufzulösen:

\begin{example}
\begin{verbatim}
    OPT_DNS='yes'
    DNS_ZONE_DELEGATION_N='1'
    DNS_ZONE_DELEGATION_1_UPSTREAM_SERVER_N='1'
    DNS_ZONE_DELEGATION_1_UPSTREAM_SERVER_1_IP='192.168.1.1'   # an fli4l-A delegieren
    DNS_ZONE_DELEGATION_1_UPSTREAM_SERVER_1_QUERYSOURCEIP='IP_NET_1_IPADDR'

    DNS_ZONE_DELEGATION_1_DOMAIN_N='1'
    DNS_ZONE_DELEGATION_1_DOMAIN_1='standortA.home.somedomain.net'
    DNS_ZONE_DELEGATION_1_NETWORK_N='1'
    DNS_ZONE_DELEGATION_1_NETWORK_1='{fli4a-intern.prefix}'
\end{verbatim}
\end{example}

