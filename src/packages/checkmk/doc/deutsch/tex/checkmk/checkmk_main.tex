% Last Update: $Id$
\section {CHECKMK - monitoring agent}

B Beschreibung zum CheckMK Agent

\subsection {Installation des CheckMK Agents}

\begin{description}

\config {OPT\_CHECKMK}{OPT\_CHECKMK}{OPTCHECKMK}

  Standard-Einstellung: \var{OPT\_\-CHECKMK='no'}

  Um den CheckMK Agenten auf dem fli4l-Router zu installieren reicht
  es aus \var{OPT\_\-CHECKMK} auf \var{'yes'} zu ändern. Alle weiteren
  Einstellungen sind optional.

\config {CHECKMK\_LISTEN}{CHECKMK\_LISTEN}{CHECKMKLISTEN}

  Auf welche IP-Adresse der CheckMK gebunden (horchen) soll. Es sind
  wie fli4l üblich auch die Angabe von IP\_NET\_1\_IPADDR möglich. Der
  CheckMK Agent wird über xinetd gestartet und xinetd erlaubt nur eine
  Angabe auf der der Dienst sich bindet oder alternativ reagiert
  xinetd auf allen Adressen.

\config {CHECKMK\_ONLY\_FROM}{CHECKMK\_ONLY\_FROM}{CHECKMKONLYFROM}

  Der Zugriff auf den CheckMK Agenten kann über die Angabe von Hosts
  (mit der fli4l üblichen @checkmk Schreibweise beispielsweise) oder
  ganzen Netzen (IP\_NET\_1) oder durch Angabe von Netzen
  eingeschränkt werden.

  Die fli4l Firewall wird automatisch so konfiguriert, dass der
  Zugriff auf den CheckMK Agenten nur durch die angegebenen Hosts
  bzw. Netze möglich ist.

\end{description}

\subsection{Literatur}
CheckMK Homepage: \altlink{https://mathias-kettner.de/check_mk.html}

Erste Version der Dokumentation von
Claas Hilbrecht $<$babel@fli4l.de$>$, im August 2017
