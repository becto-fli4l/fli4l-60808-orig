% Do not remove the next line
% Synchronized to r48707

\section {CHECKMK - Agent de surveillance}

Description de l'agent CheckMK

\subsection {Installation de l'agent CheckMK}

\begin{description}

\config {OPT\_CHECKMK}{OPT\_CHECKMK}{OPTCHECKMK}

  Paramètre par défaut~: \var{OPT\_\-CHECKMK='no'}

  Pour installer l'agent CheckMK sur le routeur fli4l, il suffit de mettre
  dans la variable \var{OPT\_\-CHECKMK} le paramètre \var{'yes'}.
  Tous les autres paramètres sont facultatifs.

\config {CHECKMK\_LISTEN}{CHECKMK\_LISTEN}{CHECKMKLISTEN}

  l'agent CheckMK (écouter) l'adresse IP sur lequelle il est attaché.
  En général ont spécifient IP\_NET\_1\_IPADDR sur fli4l. L'agent CheckMK
  démarre via le démon xinetd, soit xinetd permet d'accéder aux services
  sur lequelle il est attaché, soit xinetd accéde aux services sur toutes
  les adresses.

\config {CHECKMK\_ONLY\_FROM}{CHECKMK\_ONLY\_FROM}{CHECKMKONLYFROM}

  L'accès de l'agent CheckMK peut être restreint en spécifiant un hôte
  (par exemple en utlisant l'orthographe fli4l @checkmk) ou en spécifiant
  un réseau (IP\_NET\_1) ou en spécifiant plusieurs réseaux.

  Le pare-feu fli4l est automatiquement configuré pour que l'accès de l'agent
  CheckMK soit uniquement possible via les hôtes ou les réseaux spécifiés.

\end{description}

\subsection{Chronique}

Page d'accueil de Check\_MK~: \altlink{https://mathias-kettner.de/check_mk.html}

Première version de la documentation par
Claas Hilbrecht $<$babel@fli4l.de$>$, août 2017
