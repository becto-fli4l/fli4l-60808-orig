% Do not remove the next line
% Synchronized to r32675

\section{QoS - Qualité de service}
\marklabel{sec:qos}{}

    Grâce à Qos, vous pouvez régler et partager la bande passante disponible,
    sur les différents ports, adresses IP et d'autres support.

    Le modem gère le Packet-Queue (ou la file d'attente) (Queue = file, en serie
    (être dans la file)) pour le stockage des paquets, quand la bande passante
    du modem est inférieur à la quantité de paquet disponible. Pour les modems
    DSL, la bande passante est plus grande. Ils ont l'avantage d'être
    relativement homogène et exploite au maximum de bande passante. Le routeur
    envoie au modem les paquets dans un court laps de temps (très court), le modem
    doit stockés ces paquets dans une file d'attente pour les traités. Ainsi,
    cette file d'attente a une organisation très simple, tous les paquets sont
    rangés et envoyés dans l'ordre d'arrivés, c'est juste un modem~:-D

    C'est ici que QoS entre en jeux. QoS gère également une file d'attente pour
    les paquets, dans le routeur lui-même, et là, on a la possibilité d'être
    le roi, on décide si les paquets doivent être envoyés en premier ou en second.
    Si QoS est configuré correctement, les paquets seront envoyés judicieusement
    et rapidement au modem sans qu'il n'attérissent dans la file d'attente
    du modem, s'est comme si on avait déplacé la file d'attente du modem sur le
    routeur.

    \marklabel{hint:speedunits}{}Encore une chose au sujet des unités de vitesse
    commune~: QoS prend en charge les Mibit/s (megabit/s) et les Kibit/s (kilobit/s)
    et surtout 1Mibit = 1024Kibit.

\subsection{Configuration}

\begin{description}
\config{OPT\_QOS}{OPT\_QOS}{OPTQOS}

    Vous pouvez placer 'yes' dans la variable \var{OPT\_\-QOS} pour activer
    QoS ou 'no' pour ne pas l'utiliser

\config{QOS\_INTERNET\_DEV\_N}{QOS\_INTERNET\_DEV\_N}{QOSINTERNETDEVN}

    Vous indiquez ici Le nombre d'interface du routeur, qui envoie les données
    sur Internet.

\config{QOS\_INTERNET\_DEV\_x}{QOS\_INTERNET\_DEV\_x}{QOSINTERNETDEVx}

    Vous indiquez ici, la liste des interfaces sur lesquels les données seront
    transmis vers Internet. Exemples~:

    \begin{tabular}[h!]{ll}
      \var{QOS\_\-INTERNET\_\-DEV\_N}='3'    & Nombre de périphérique \\
      \var{QOS\_\-INTERNET\_\-DEV\_1}='ethX' & Pour cable et autres liaison Ethernet \\
      \var{QOS\_\-INTERNET\_\-DEV\_2}='ppp0' & Pour DSL sur protocole PPPoE \\

      \var{QOS\_\-INTERNET\_\-DEV\_3}='ipppX' & Pour ISDN \\
    \end{tabular}

    Le premier circuit du périphérique ISDN devrait s'appeler ippp0, et le
    deuxième circuit ippp1 etc ... Mais, si pour le premier circuit l'agrégation
    des canaux a été activée (2 canaux par circuit), alors du premier circuit
    avec deux canaux s'appellera ippp1 et le deuxième Circuit s'appellera ippp2.
    Il convient pour bénéficier du QoS avec ISDN, de désactiver l'agrégation
    des canaux des circuits.

\config{QOS\_INTERNET\_BAND\_DOWN}{QOS\_INTERNET\_BAND\_DOWN}{QOSINTERNETBANDDOWN}

    On indique ici la bande passante maximum montante pour un accès
    Internet. Voir ci-dessus pour \jump{hint:speedunits}{les unités de vitesse}.

    Remarque~: pour l'actualisation des tâches, comme pour l'affectation des
    paquets-ACK, il ne faut pas indiquer une quantité de bande passante (ou B-P)
    supérieure à la B-P vraiment disponible, les paquets de la file d'attente du
    routeur seront triés comme il se doit, mais cela ne sera pas tout à fait
    exacte, et finalement les paquets seront stockés dans la file d'attente du
    modem et sera ralenti. Il est possible que le fournisseur d'accès n'indique
    pas réellement la bande passante correspondante, cela peut être un peu plus
    ou un peu moins. Il faut donc faire des tests. Pour trouver la bonne quantité.

\config{QOS\_INTERNET\_BAND\_UP}{QOS\_INTERNET\_BAND\_UP}{QOSINTERNETBANDUP}

    On indique ici la bande passante maximum descendante pour l'accès
    Internet. Voir \var{OPT\_\-QOS}, pour les unités de vitesse.

    Remarque~: voir les informations ci-dessus \var{QOS\_\-INTERNET\_\-BAND\_\-DOWN}.

\config{QOS\_INTERNET\_DEFAULT\_DOWN}{QOS\_INTERNET\_DEFAULT\_DOWN}{QOSINTERNETDEFAULTDOWN}

    On indique ici les classes par défaut des paquets qui proviennent d'Internet.
    Tous les paquets qui n'ont pas été paramétrés dans les classes filtrées,
    atterrissent ensuite dans la classe supplémentaire indiqué ici.

    Si aucune classe n'est paramètre et si la variable est sur
\begin{example}
\begin{verbatim}
        QOS_CLASS_x_DIRECTION='down'
\end{verbatim}
\end{example}
    alors, on indiquera~:
\begin{example}
\begin{verbatim}
        QOS_INTERNET_DEFAULT_DOWN='0'
\end{verbatim}
\end{example}

    Exemple~:

    On met en place 2 classes et un seul filtre pour les paquets. On paramètre
    par ex. une adresse-IP spécifique à la 1ère des deux classes. Tous les autres
    paquets passeront par la 2ème classe, c'est une classe par défaut qui ne sera
    pas filtrée. On indique par conséquent dans la variable~:

\begin{example}
\begin{verbatim}
        QOS_INTERNET_DEFAULT_DOWN='2'
\end{verbatim}
\end{example}

    Il faut faire attention à se que la variable \var{QOS\_\-INTERNET\_\-DEFAULT\_\-DOWN}
    a une classe de paramétrée et que dans la variable \var{QOS\_\-CLASS\_\-x\_\-DIRECTION}
    le paramètre 'down' soit indiqué.

\config{QOS\_INTERNET\_DEFAULT\_UP}{QOS\_INTERNET\_DEFAULT\_UP}{QOSINTERNETDEFAULTUP}

    On indique ici les classes par défaut, pour des paquets qui vont sur
    Internet. Tous les paquets qui n'ont pas été paramétrés dans les classes
    filtrées, atterrissent ensuite dans la classe supplémentaire indiqué ici.

    Si aucune classe n'est paramètre et si la variable est sur
\begin{example}
\begin{verbatim}
        QOS_CLASS_x_DIRECTION='up'
\end{verbatim}
\end{example}
    alors, on indiquera~:
\begin{example}
\begin{verbatim}
        QOS_INTERNET_DEFAULT_UP='0'
\end{verbatim}
\end{example}

    L'ensemble des fonctions sont similaire à la variable \var{QOS\_\-INTERNET\_\-DEFAULT\_\-DOWN}.

    Il faut faire attention à se que la variable \var{QOS\_\-INTERNET\_\-DEFAULT\_\-UP}
    a une classe de paramétrée et que dans la variable \var{QOS\_\-CLASS\_\-x\_\-DIRECTION}
    le paramètre 'up' soit indiqué.

\config{QOS\_CLASS\_N}{QOS\_CLASS\_N}{QOSCLASSN}

    On indique ici le nombre souhaité de classe.

\config{QOS\_CLASS\_x\_PARENT}{QOS\_CLASS\_x\_PARENT}{QOSCLASSxPARENT}

    Dans cette variable, on peut superposer les classes. On indique toujours
    un numéro pour la classe parent et on attribue la bande passante sur la classe
    parent, elle peut aussi être répartie en sous-classe. Le niveau maximal de
    d'imbrication s'élève à 8 niveaux, l'interface 0 représente déjà 1 niveau,
    il reste donc un maximum de 7 niveaux configurable

    Si dans la classe il n'y a pas de sous-classe on indique~:

\begin{example}
\begin{verbatim}
        QOS_CLASS_x_PARENT='0'
\end{verbatim}
\end{example}

    suivant la direction dont elle fait partie (avec
   \var{QOS\_\-CLASS\_\-x\_\-PORT}) ou avec
   \var{QOS\_\-CLASS\_\-x\_\-PORT\_\-TYPE} et aussi la bande passante attribuée
    avec \var{QOS\_\-INTERNET\_\-BAND\_\-DOWN}

    Important~: Si vous n'avez pas indiqué '0', il est important de s'assurer,
    que la classe se trouve bien entendu dans le niveau le plus haut (par rapport
    à la numérotation des niveaux).

\config{QOS\_CLASS\_x\_MINBANDWIDTH}{QOS\_CLASS\_x\_MINBANDWIDTH}{QOSCLASSxMINBANDWIDTH}

    Vous indiquez ici bande passante minimum, que l'on veut attribuer aux 
    classes. On peut parler aussi de rapport. Voir les indication sur
    les unités de vitesse dans \var{OPT\_\-QOS}.

    Exemple de classe, donc vous voulez limiter la bande passante à 128Kibit/s~:

\begin{example}
\begin{verbatim}
        QOS_CLASS_1_PARENT='0'
        QOS_CLASS_1_MAXBANDWIDTH='128Kibit/s'
        QOS_CLASS_1_MINBANDWIDTH='128Kibit/s'
\end{verbatim}
\end{example}

    Maintenant on dispose de 3 classes supplèmentaires, ont paramètre les variables
    \var{QOS\_\-CLASS\_\-x\_\-MINBANDWIDTH} et \var{QOS\_\-CLASS\_\-x\_\-MAXBANDWIDTH}
    des sous-classes de notre première classe~:

\begin{example}
\begin{verbatim}
        QOS_CLASS_2_PARENT='1'
        QOS_CLASS_2_MINBANDWIDTH='60Kibit/s'
        QOS_CLASS_2_MAXBANDWIDTH='128Kibit/s'

        QOS_CLASS_3_PARENT='1'
        QOS_CLASS_3_MINBANDWIDTH='40Kibit/s'
        QOS_CLASS_3_MAXBANDWIDTH='128Kibit/s'

        QOS_CLASS_4_PARENT='1'
        QOS_CLASS_4_MINBANDWIDTH='28Kibit/s'
        QOS_CLASS_4_MAXBANDWIDTH='128Kibit/s'
\end{verbatim}
\end{example}

    Toutes les sous-classes peuvent avoir la même (ou pas) priorité
    (voir \var{QOS\_\-CLASS\_\-x\_\-PRIO}). Maintenant un débit respectif est
    produit dans chacune des 3 classes avec la variable
    \var{QOS\_\-CLASS\_\-x\_\-MINBANDWIDTH}. Ainsi on attribut dans chaque classe
    la bande passante correspondante dans \var{QOS\_\-CLASS\_\-x\_\-MINBANDWIDTH}.
    Exemple si vous modifier la classe 2 et que vous lui attribuée 20Kibit/s, il
    restera donc 40Kibit/s à attribuer aux "autres" classes. Cette excédent peut
    être réparti avec la relation 40/28 dans les classes 3 et 4. Avec la
    variable \var{QOS\_\-CLASS\_\-x\_\-MAXBANDWIDTH} chaque classe sera limité
    à 128Kibit/s, étant donné que toutes les sous-classes sont limitées
    à 128Kibit/s, ils ne pourront pas dépasser le débit autorisé de 128Kibit/s.

\config{QOS\_CLASS\_x\_MAXBANDWIDTH}{QOS\_CLASS\_x\_MAXBANDWIDTH}{QOSCLASSxMAXBANDWIDTH}

    Vous indiquez ici bande passante maximum, que l'on veut attribuer aux
    classes. Il n'y a pas sens, d'indiquer une valeur plus basse que la
    variable \var{QOS\_\-CLASS\_\-x\_\-MINBANDWIDTH}. Si l'on indique aucune
    valeur, la variable prend automatiquement la valeur qui est indiqué dans
    \var{QOS\_\-CLASS\_\-x\_\-MINBANDWIDTH}. bien entendu pour cette classe
    il ne faut pas indiquer un surplus de bande passante.

    Voir \var{OPT\_\-QOS} pour les unités de vitesse.

\config{QOS\_CLASS\_x\_DIRECTION}{QOS\_CLASS\_x\_DIRECTION}{QOSCLASSxDIRECTION}

    On indique dans cette variable la direction de la bande passante, donc
    la classe fait partie. Si elle fait parti du débit montant, on indique~:

\begin{example}
\begin{verbatim}
        QOS_CLASS_x_DIRECTION='up'
\end{verbatim}
\end{example}

    Si elle fait parti du débit descendant, on indique~:

\begin{example}
\begin{verbatim}
        QOS_CLASS_x_DIRECTION='down'
\end{verbatim}
\end{example}

    Attention~: Up-stream (débit montant) c'est la transmission de votre ordinateur
    au serveur-Internet et Down-stream (débit descendant) c'est la transmission du
    serveur-Internet à votre ordinateur.

\config{QOS\_CLASS\_x\_PRIO}{QOS\_CLASS\_x\_PRIO}{QOSCLASSxPRIO}

    Dans cette variable on règle le niveau de priorité de la classe.
    Plus le chiffre est bas plus le niveau de priorité est élevé. Les
    chiffres autorisés sont de 0 à 7. Si la variable est laissée vide
    le niveau de priorité sera à 0 donc le plus haut.

    Lorsqu'il y a un excédent de bande passante, le système détermine
    la classe et ça priorité pour augmenter le débit. Pour expliquer cela, nous
    allons modifier légèrement l'exemple \var{QOS\_\-CLASS\_\-x\_\-MINIMUMBANDWIDTH},
    la première classe rien n'a été changée. On attribue à la classes 4 une priorité 2~:

\begin{example}
\begin{verbatim}
        QOS_CLASS_2_PARENT='1'
        QOS_CLASS_2_MINBANDWIDTH='60Kibit/s'
        QOS_CLASS_2_MAXBANDWIDTH='128Kibit/s'
        QOS_CLASS_2_PRIO='1'

        QOS_CLASS_3_PARENT='1'
        QOS_CLASS_3_MINBANDWIDTH='40Kibit/s'
        QOS_CLASS_3_MAXBANDWIDTH='128Kibit/s'
        QOS_CLASS_3_PRIO='1'

        QOS_CLASS_4_PARENT='1'
        QOS_CLASS_4_MINBANDWIDTH='28Kibit/s'
        QOS_CLASS_4_MAXBANDWIDTH='128Kibit/s'
        QOS_CLASS_4_PRIO='2'
\end{verbatim}
\end{example}

    Dans cette exemple si la classe 2 ne consomme que 20Kibit/s, il y a donc
    un excédent de 40Kibit/s. les Classes 3 et 4 peuvent recevoir encore
    plus de bande passante. Cependant, la classe 3 a une priorité plus élevée
    que la classe 4, donc la classe 3 peut récupérer l'excédent de la bande
    passante les 40Kibit/s.

    Cependant si la classe 3 a besoin seulement de 20Kibit/s de l'excédent
    des 40Kibit/s, alors la classe 4 recevra les 20Kibit/s restant.

    Prenons un autre exemple, si la classe 4 ne consomme pas la bande passante
    et si les classes 2 et 3 on besoin de plus de bande passante. Alors, chaque
    classes utilise la bande passante spécifiée dans la variable
    \var{QOS\_\-CLASS\_\-x\_\-MINBANDWIDTH} le reste sera divisé entre les deux
    avec le rapport 60/40, puisque les deux classes ont la même prioritée.

    La variable \var{QOS\_\-CLASS\_\-x\_\-PRIO} influe seulement sur
    l'excédent de bande passante, pour éventuellement là répartir.

\config{QOS\_CLASS\_x\_LABEL}{QOS\_CLASS\_x\_LABEL}{QOSCLASSxLABEL}

    Avec cette variable optionnelle, vous pouvez placer un intitulé pour une classe.
    Cet intitulé sera affiché pour le graphique de QOS dans OPT\_RRDTOOL s'il est activé.

\config{QOS\_FILTER\_N}{QOS\_FILTER\_N}{QOSFILTERN}

    Vous indiquez ici le nombre de filtre souhaité.

    Au sujet du filtrage, en général on peut dire ceci~: les paramètres
    des différentes variables sont reliées et/ou les différentes paramètres
    d'une même variable sont aussi reliées. on peut dire par exemple~: que dans
    un même filtre nous pouvons filtrer une adresse IP et un port, ainsi seul
    des paquets filtrés serons envoyé vers la cible de la (n) classe, et
    seront appliqués sur l'un et l'autre paramètre simultanément.

    Un autre exemple~: Dans le même filtre il y a deux ports (21 et 80) et
    une adresse IP. Bien sûr, un paquet de données ne peut pas être envoyé
    sur les deux ports en même temps. Le filtre se comporte alors comme
    ceci~: les paquets filtrés utilisent le port 21 et en même temps l'adresse
    IP, ou le port 80 et en même temps l'adresse IP.

    Important~: Cela dépend de la séquence de filtrage~!

    Un exemple~: on veut transporter les paquets, par l'intermédiaire du
    port 456 qui est ouvert, pour \textbf{tous} les ordinateurs de la classe A.
    De plus on voudrait faire passer tous les paquets de l'ordinateur de
    l'adresse IP 192.168.6.5 - par le port 456 ouvert - dans la classe B.
    Je règle maintenant le filtre de la première adresse IP, alors tous
    les paquets arrivent - sur le port 456 ouvert - dans la classe B, je
    règle un autre filtre pour le port 456 qui n'est pas altéré. Le filtre
    pour le port 456 doit donc être avant le filtre de l'adresse IP 192.168.6.5

\config{QOS\_FILTER\_x\_CLASS}{QOS\_FILTER\_x\_CLASS}{QOSFILTERxCLASS}

    On indique dans cette variable, les classes qui doivent s'appliquer
    aux paquets filtrés. Par exemple, on veut filtrer les paquets de la
    variable spécifié ici \var{QOS\_\-CLASS\_\-25\_\-MINBANDWIDTH} avec
    le numéro de classe, vous devez indiquer~:

\begin{example}
\begin{verbatim}
        QOS_FILTER_x_CLASS='25'
\end{verbatim}
\end{example}

    Avec la variable \var{QOS\_\-CLASS\_\-x\_\-DIRECTION} on peut indiquer
    ne classe appartenant maintenant au débit montant ou au débit descendant.
    Si maintenant on met en place un filtre pour filtrer les paquets par
    exemple dans la classe du débit montant, seul des paquets du débit montant
    seront ainsi filtrés par ce filtre et seront installé pour la classe
    donnée. Avec cette variable \var{QOS\_\-CLASS\_\-x\_\-DIRECTION} on
    détermine la "direction" du débit à filtrer.

    Depuis la version 2.1, il est désormais possible de l'indiquer plus
    d'une classe. Par exemple si vous souhaitez, envoyer le trafic sur le
    port 456 à la fois dans les classes des débit montant et descendant,
    vous pouvez indiquer~:

\begin{example}
\begin{verbatim}
        QOS_FILTER_x_CLASS='4 25'
\end{verbatim}
\end{example}

    Avec les numéros de classe 4 pour le débit montant et de classe 25 pour le
    débit descendant. Il n'y a aucun sens a indiquer ici un débit montant et
    descendant, cela est purement objectif à ne pas recopier.

\config{QOS\_FILTER\_x\_IP\_INTERN}{QOS\_FILTER\_x\_IP\_INTERN}{QOSFILTERxIPINTERN}

    On indique dans cette variable les adresses-IP et les adresses
    de domaine du réseau interne, qui doivent être filtrés. Elles sont
    séparées par un espace et peuvent être combinés librement.

    Cela pourrait être par exemple~:

\begin{example}
\begin{verbatim}
        QOS_FILTER_x_IP_INTERN='192.168.6.0/24 192.168.5.7 192.168.5.12'
\end{verbatim}
\end{example}

    Ici, toutes les adresses sous la forme 192.168.6.X sont filtrés
    en plus des adresses IP 192.168.5.7 et 192.168.5.12.

    Cette variable peut aussi rester vide.

    Si vous utilisez cette variable \var{QOS\_\-FILTER\_\-x\_\-IP\_\-EXTERN}
    avec une adresses-IP ou une plage d'adresses-IP, il y aura pas
    de trafic filtré entre \var{QOS\_\-FILTER\_\-x\_\-IP\_\-INTERN} et
    \var{QOS\_\-FILTER\_\-x\_\-IP\_\-EXTERN}.

    \sloppypar\wichtig{Si vous ajoutez la variable \var{QOS\_\-FILTER\_\-x\_\-OPTION}
    avec les options de filtrage ACK, TOSMD, TOSMT, TOSMR ou TOSMC,
    la variable \var{QOS\_\-CLASS\_\-x\_\-DIRECTION} avec le paramètre
    'down' sera alors ignorée.}

\config{QOS\_FILTER\_x\_IP\_EXTERN}{QOS\_FILTER\_x\_IP\_EXTERN}{QOSFILTERxIPEXTERN}

    On indique dans cette variable les adresses-IP et les adresses de
    domaine du réseau externe qui doivent être filtrés (elle se rapporte
    à la variable \var{QOS\_\-INTERNET\_\-DEV}). Les adresses sont séparées
    par un espace et peuvent être combinés librement. L'ensemble fonctionne
    de la même manière que \var{QOS\_\-FILTER\_\-x\_\-IP\_\-INTERN}.

    Cette variable peut aussi rester vide.

    \sloppypar\wichtig{Si vous ajoutez la variable \var{QOS\_\-FILTER\_\-x\_\-OPTION}
    avec les options de filtrages ACK, TOSMD, TOSMT, TOSMR ou TOSMC,
    la variable \var{QOS\_\-CLASS\_\-x\_\-DIRECTION} avec le paramètre
    'down' sera alors ignorée.}

\config{QOS\_FILTER\_x\_PORT}{QOS\_FILTER\_x\_PORT}{QOSFILTERxPORT}

    On peut paramétrer dans cette variable un port ou une plage de ports,
    ils seront séparées par un espace et peuvent être combinés librement. Si
    la variable est vide, le trafic se fait sur tous les ports.

    Au sujet de la plage de port~: si l'on souhaite filtre l'ensemble
    les ports de 5000 à 5099, on indiquera~:

\var{QOS\_\-FILTER\_\-x\_\-PORT}='5000-5099'

    Un autre exemple~: On voudrait filtrer le trafic des ports 20 à 21,
    137 à 139 et le port 80, dans la même classe. Alors, on indiquera~:

\begin{example}
\begin{verbatim}
        QOS_FILTER_x_PORT='20-21 137-139 80'
\end{verbatim}
\end{example}

    Cette variable peut aussi être laissée vide.

   Important~:
   \begin{itemize}
   \item Avec le filtrage de port, la variable \var{QOS\_\-FILTER\_\-x\_\-PORT\_\-TYPE}
    doit aussi être activée.

   \item Si la variable \var{QOS\_\-FILTER\_\-x\_\-OPTION} et activé
    avec les options de filtrages ACK, TOSMD, TOSMT, TOSMR ou TOSMC,
    la plage de ports sera ignorés.
    \end{itemize}

\config{QOS\_FILTER\_x\_PORT\_TYPE}{QOS\_FILTER\_x\_PORT\_TYPE}{QOSFILTERxPORTTYPE}

    Cette variable est seulement important, que si la variable \var{QOS\_\-FILTER\_\-x\_\-PORT}
    est utilisé, dans ce cas on peut là paramétrer (autrement elle sera
    simplement ignoré).

    Puisque les ports se différencient entre les ports du service-client
    et les ports du service-serveur, vous devez indiquer ici si les ports du
    serveur ou du client sont visé. Les ordinateurs du réseau seront considérés
    comme points de référence.

    Les réglages suivant sont possible~:

\begin{example}
\begin{verbatim}
        QOS_FILTER_x_PORT_TYPE='client'
        QOS_FILTER_x_PORT_TYPE='server'
\end{verbatim}
\end{example}

    Depuis la version 2.1 la combinaison des deux arguments dans une seul
    variable est possible, pour le trafic sur les ports dans n'autre
    propre réseau et pour le trafic sur les ports externe pour Internet, dans
    la même classe, par exemple~:

\begin{example}
\begin{verbatim}
        QOS_FILTER_x_PORT_TYPE='client server'
\end{verbatim}
\end{example}

    Cela correspond à l'élaboration de deux filtres semblables, dans lequel on
    a placé le client et le serveur sur la même variable \var{QOS\_\-FILTER\_\-x\_\-PORT\_\-TYPE}.

\config{QOS\_FILTER\_x\_OPTION}{QOS\_FILTER\_x\_OPTION}{QOSFILTERxOPTION}

    Avec cette variable on indique d'autres options pour le filtre actif.
    Il ne faut pas spécifier plus d'un paramètre, car une combinaison de
    plusieurs paramètres dans le même filtre n'a pas de sens. En revanche,
    il est parfaitement possible et même parfois utile, de filtrer les
    paquets-ACK, et un 2ème paquet. Par exemple filtrer les paquets-TOSMD,
    dans la même destination de la classe (voir \var{QOS\_\-FILTER\_\-x\_\-CLASS}).

    \begin{description}
  \item[ACK] Paquet de confirmation.

    Si on applique ce paquet dans la variable option, lorsqu'un paquet de données
    arrive il sera renvoyé un paquet de confirmation comme un "accusé de
    réception". Si vous téléchargez de grands fichiers beaucoup de paquets
    sont transmis, pour chaque paquet transmis vous devez envoyer une
    confirmation ou un accusé de réception pour indiquer que le paquet
    de donnés est bien arrivé. Si la confirmation se fait attendre,
    l'expéditeur attendra celui-ci avant d'envoyer un nouveau paquet,
    le prochain paquet ne sera pas pour vous si la confirmation n'est pas reçue.

    Toutes les connexions asymétriques sont particulièrement importantes,
    actuellement (le débit montant/descendant est inégal) dans la plus par
    des offres DSL privées. Généralement, le débit montant est beaucoup plus
    faible, poussé à ses limites les paquets sont empilés avant d'être envoyés
    et indiscutablement quelque part dans l'immense tas, il y a les petits
    paquets de confirmation. Dans des circonstances normales, cela fonctionne
    séquentiellement. Jusqu'à ce que le paquet de confirmation à son tour
    est envoyé, il serait bien que notre expéditeur de données puisse faire une
    petite pause, mais ce n'est pas bon pour le débit descendant.

    Nous devons veiller à ce que les paquets de confirmations soient bien à la
    suite, pour que les paquets de confirmation soient envoyés "normalement" au
    fur et à mesure, afin que l'expéditeur reçoive bien la confirmation. Cette
    option est logique d'être combinée à une classe, pour cet exemple d'application.

  \item[ICMP] Paquet-Ping (Protocole ICMP)

    On utilise un paquet Ping pour mesurer le temps en seconde que met ce paquet
    pour aller du point A au point B. Si vous voulez donner au paquet Ping une
    priorité plus élevée, vous pouvez indiquer cette valeur dans la variable
    option. Cette valeur n'a rien à voir pour jouer sur Internet. Ce n'est pas
    parce que vous avez activé le paquet Ping que vous serez prioritaire et que
    vous aurez un super Ping pour jouer sur le Net...

  \item[IGMP] Paquet-IGMP (Protocole IGMP)

    Si vous utilisez la TV par IP, le protocole IGMP sera utile pour le filtrage
    et la hiérarchisation.

  \item[TCPSMALL] Petit paquet TCP

    Vous pouvez utiliser ce filtrage pour les requêtes HTTP(s) sortantes, ce
    filtre sera prioritaire. Une combinaison avec un port de destination est
    possible et judicieux. Taille des paquets TCP~: max. 800 octets.

  \item[TCP] Paquet-TCP (Protocole TCP)

    Avec ce paquet on filtre uniquement les paquets qui utilisent le protocole TCP.

  \item[UDP] Paquet-UDP (Protocole UDP)

    Avec ce paquet on filtre uniquement les paquets qui utilisent le protocole UDP.

  \item[TOS*] Type of Service

    TOS "Type of Service" est une application qui place 4 Bit-TOS dans
    l'entête IP pour chaque paquet envoyé. Ceux-ci ont pour effet de
    modifier la manière dont les paquets sont traités. Par exemple, on
    peut placer TOS-Minimum-Delay pour Améliore la réactivité SSH et
    TOS-Maximum-Troughput pour expédition de fichiers. Généralement ce
    sont des programmes Linux/Unix qui utilisent plus fréquemment ces
    Bit-TOS, que les programme Windows. En outre, on peut placer ces
    Bit-TOS pour certains paquets à destination des IT Firewall. Cela
    dépend bien sûr que les Routeurs acceptent ou pas les Bit-TOS. En
    réalité pour fli4l Minimum-Delay et Maximum-Throughput présentent
    un vrai intérêt.

    \begin{description}
  \item [TOSMD - TOS Minimum-Delay]

    Ce service est utilisé pour améliorer la réactivité des connexions en
    réduisant le délai de transmission des paquets, Bit-TOS est recommandé
    pour le SSH, Telnet, FTP (contrôle), TFTP.

  \item[TOSMT - TOS Maximum-Troughput]

    Ce service est utilisé pour améliorer le débit des gros de fichier, au prix
    d'une possible détérioration de l'interactivité de la session. Les temps de
    latence ne sont pas importants, Bit-TOS est recommandé pour le FTP-data et WWW.

  \item[TOSMR - TOS Maximum-Reliability]

    Ce service est utilisé pour avoir la certitude que les données arrivent
    sans perte, améliore la fiabilité, on évite un revoie de paquet inutile,
    Bit-TOS est recommandé pour SNMP et DNS

  \item[TOSMC - TOS Minimum-Cost]

    Ce service est utilisé pour minimiser le délai, une meilleure rentabilité,
    Bit-TOS est recommandé pour NNTP, SMTP et ICMP.

  \item[DSCP*] Differentiated Services Code Point

    Le DSCP est appelé marquage conformément défini dans la RFC\ 2474.
    Ce processus de marquage a largement remplacé le TOS depuis 1998.

    Le filtrage des classes DSCP peut être configuré comme ceci~:

\begin{example}
\begin{verbatim}
        QOS_FILTER_x_OPTION='DSCPef'
        QOS_FILTER_x_OPTION='DSCPcs3'
\end{verbatim}
\end{example}

    S'il vous plaît, faite attention que le DSCP soit écrit en lettre capital
    et la classe en minuscule.

    Les classes suivantes peuvent être utilisées~:

    af11-af13, af21-af23, af31-af33, af41-af43, cs1-cs7, ef et be (par défaut)

\end{description}
    \end{description}
    \end{description}


\marklabel{sec:qosanwendung}{
  \subsection{Applications et exemples}
}

    Comment doit on configurer \var{OPT\_\-QoS} précisément? voici
    quelques exemples concret~:

    \begin{itemize}
  \item Exemple 1~: exemple simple pour le partager de la bande passante
     sur trois ordinateurs.

  \item Exemple 2~: exemple de configuration pour la distribution de la
    bande passante sur 2 ordinateurs et en plus une seconde répartition
    de la bande passante sur les ports des ordinateurs respectif, en plus
    il restera de la B-P.

  \item Exemple 3~: exemple de fonctionnement général de QoS avec un peut
    plus de détail.

  \item Exemple 4~: exemple de configuration des paquets-ACK avant d'être
    transmis, sur le débit descendant de sorte que le débit montant ne chute pas.
    \end{itemize}



\subsubsection{Exemple 1}



    L'objectif de cet exemple simple, est de distribuer la bande
    passante sur 3 ordinateurs.

    Pour ce faire, nous allons créer 4 classes (voir \var{QOS\_\-CLASS\_\-N} et
    se qui suit) pour différentes vitesses (voir
    \var{QOS\_\-CLASS\_\-x\_\-MINBANDWIDTH} / \var{QOS\_\-CLASS\_\-x\_\-MIN\-BAND\-WIDTH})
    cela dépend aussi de la classe 0 (voir \var{QOS\_\-CLASS\_\-x\_\-PARENT})
    Donc directement de l'interface pour "up" ou "down" (voir \var{QOS\_\-CLASS\_\-x\_\-DIRECTION}).

    La classe 4 est éventuellement pour un hôte en plus, avec moins de band
    passante. nous indiquons dans  \var{QOS\_\-INTERNET\_\-DEFAULT\_\-DOWN}='4'
    pour tous le transport non filtré et une quatrième Classe pour les
    "invités". Étant donné que nous avons rarement d'hôte, la bande passante
    sera la même pour les 3 autres classes, chaque ordinateur recevra 1/3 de
    l'ensemble de la bande passante, c'est-à-dire approximativement 256Kibit/s.

    Nous avons tout d'abord configurer la structure fondamentale. Maintenant,
    nous devons encore choisir pour le réglage quelle transport pour quelle classe.

    Pour ce faire, nous allons utiliser des filtres, pour associer le trafic à
    chacune des classes. Nous allons créer ainsi des filtres pour les 3 ordinateurs (voir
    \var{QOS\_\-FILTER\_\-N} et se qui suit) et la classification des filtres
    pour chaque classe (voir \var{QOS\_\-FILTER\_\-x\_\-CLASS}). Maintenant, nous
    pouvons avec \var{QOS\_\-FILTER\_\-x\_\-IP\_\-INTERN}, \var{QOS\_\-FILTER\_\-x\_\-IP\_\-EXTERN},
    \var{QOS\_\-FILTER\_\-x\_\-PORT}, \var{QOS\_\-FILTER\_\-x\_\-PORT} et
    \var{QOS\_\-FILTER\_\-x\_\-OPTION} déterminer le réglage de chacune des
    classes, pour lequelle appartient le filtre.

    Vous pouvez voir le principe dans la figure \ref{fig:qosbsp1}
    l'Interface 0 les 3 classe 1, 2 et 3, les 3 filtres F1, F2 et F3.

    \begin{figure}[htbp]
      \centering
      \includegraphics{qos_bsp_1}
      \caption{exemple 1}
      \marklabel{fig:qosbsp1}{}
    \end{figure}

    La configuration ressemble alors à~:

    Trois ordinateurs filtrés par IP reçoivent chacune 1/3 du débit,
    si il n'y a pas d'hôte en plus

\begin{small}
\begin{example}
\begin{verbatim}
OPT_QOS='yes'
QOS_INTERNET_DEV_N='1'
QOS_INTERNET_DEV_1='ppp0'
QOS_INTERNET_BAND_DOWN='768Kibit/s'
QOS_INTERNET_BAND_UP='128Kibit/s'
QOS_INTERNET_DEFAULT_DOWN='4'
QOS_INTERNET_DEFAULT_UP='0'

QOS_CLASS_N='4'

QOS_CLASS_1_PARENT='0'
QOS_CLASS_1_MINBANDWIDTH='232Kibit/s'
QOS_CLASS_1_MAXBANDWIDTH='768Kibit/s'
QOS_CLASS_1_DIRECTION='down'
QOS_CLASS_1_PRIO=''

QOS_CLASS_2_PARENT='0'
QOS_CLASS_2_MINBANDWIDTH='232Kibit/s'
QOS_CLASS_2_MAXBANDWIDTH='768Kibit/s'
QOS_CLASS_2_DIRECTION='down'
QOS_CLASS_2_PRIO=''

QOS_CLASS_3_PARENT='0'
QOS_CLASS_3_MINBANDWIDTH='232Kibit/s'
QOS_CLASS_3_MAXBANDWIDTH='768Kibit/s'
QOS_CLASS_3_DIRECTION='down'
QOS_CLASS_3_PRIO=''

QOS_CLASS_4_PARENT='0'
QOS_CLASS_4_MINBANDWIDTH='72Kibit/s'
QOS_CLASS_4_MAXBANDWIDTH='768Kibit/s'
QOS_CLASS_4_DIRECTION='down'
QOS_CLASS_4_PRIO=''

QOS_FILTER_N='3'

QOS_FILTER_1_CLASS='1'
QOS_FILTER_1_IP_INTERN='192.168.0.2'
QOS_FILTER_1_IP_EXTERN=''
QOS_FILTER_1_PORT=''
QOS_FILTER_1_PORT_TYPE=''
QOS_FILTER_1_OPTION=''

QOS_FILTER_2_CLASS='2'
QOS_FILTER_2_IP_INTERN='192.168.0.3'
QOS_FILTER_2_IP_EXTERN=''
QOS_FILTER_2_PORT=''
QOS_FILTER_2_PORT_TYPE=''
QOS_FILTER_2_OPTION=''

QOS_FILTER_3_CLASS='3'
QOS_FILTER_3_IP_INTERN='192.168.0.4'
QOS_FILTER_3_IP_EXTERN=''
QOS_FILTER_3_PORT=''
QOS_FILTER_3_PORT_TYPE=''
QOS_FILTER_3_OPTION=''
\end{verbatim}
\end{example}
\end{small}

    L'option \var{QOS\_\-INTERNET\_\-DEFAULT\_\-UP} a été mis sur 0 parce
    le débit montant ne devrait pas être limité.



\subsubsection{Exemple 2}



    L'objectif de cet exemple est de distribuer la bande passante sur 2
    ordinateurs, une seconde répartition de la bande passante se fera
    sur les ports des deux ordinateurs respectif, il restera de la Bande
    passante pour le transfert.

    Pour cela, nous avons de 2 classes avec leur vitesse respective,
    ils dépendent directement de l'interface pour "up" et/ou "down" (voir
    le premier exemple). Ensuite nous allons créer pour le première
    ordinateur de la première classe deux autre sous classes. Les deux
    sous classes sont créées comme la première classe directement sur
    l'interface, avec toutefois une particularité~: la variable
    \var{QOS\_\-CLASS\_\-x\_\-PARENT} n'est pas sur 0, mais sur le nombre
    de classe à laquelle les sous-classes sont attachées. Par exemple pour
    \var{QOS\_\-CLASS\_\-1}, nous devons maintenant ajouté 1 dans la classes
    \var{QOS\_\-CLASS\_\-1}, suivante. Il en sera de même pour le second
    ordinateur. On attache à nouveau deux sous classes à la classe
    du deuxième ordinateur, ils doivent être sur 2. Ceci peut être fait,
    non seulement pour deux ordinateurs, mais pour autant d'ordinateur que
    vous voulez. De même on peut créer autant de sous-classes par rapport
    à la classe supérieur.

    nous avons tout d'abord configurer la structure fondamentale. Ensuite
    nous devons assigner les filtres de chaque classe pour le trafic.
    (Voir le premier exemple)

    Nous allons créer 2 filtres pour le premier ordinateur et 2 filtres
    pour le deuxième ordinateur. Il y aura respectivement un filtre pour
    le port, et d'un filtre pour le transfert des donnés. Il faut
    absolument tenir compte des séquences. le premier filtre uniquement
    pour le port, puis le reste. Il est impératif de respecter l'ordre.
    Autrement le filtre sera déjà attribué pour le reste les classes.

    Vous pouvez voir le principe dans la figure \ref{fig:qosbsp2}
    l'Interface 0, les 6 classes 1, 2, 3, 4, 5, et 6, les 4 filtres F1,
    F2, F3 et F4.

    \begin{figure}[htbp]
      \centering
      \includegraphics{qos_bsp_2}%
      \caption{ exemple 2}
      \marklabel{fig:qosbsp2}{}
    \end{figure}

    La configuration ressemble alors à~:

    Deux classes pour 2 ordinateurs qui obtiennent 1/2 de BP, deux
    classes pour les ports qui obtiennent 2/3 de la B-P, il reste donc
    1/3 de B-D pour chaque classe parent~:

\begin{small}
\begin{example}
\begin{verbatim}
OPT_QOS='yes'
QOS_INTERNET_DEV_N='1'
QOS_INTERNET_DEV_1='ppp0'
QOS_INTERNET_BAND_DOWN='768Kibit/s'
QOS_INTERNET_BAND_UP='128Kibit/s'
QOS_INTERNET_DEFAULT_DOWN='7'
QOS_INTERNET_DEFAULT_UP='0'

QOS_CLASS_N='6'

QOS_CLASS_1_PARENT='0'
QOS_CLASS_1_MINBANDWIDTH='384Kibit/s'
QOS_CLASS_1_MAXBANDWIDTH='768Kibit/s'
QOS_CLASS_1_DIRECTION='down'
QOS_CLASS_1_PRIO=''

QOS_CLASS_2_PARENT='0'
QOS_CLASS_2_MINBANDWIDTH='384Kibit/s'
QOS_CLASS_2_MAXBANDWIDTH='768Kibit/s'
QOS_CLASS_2_DIRECTION='down'
QOS_CLASS_2_PRIO=''

QOS_CLASS_3_PARENT='1'
QOS_CLASS_3_MINBANDWIDTH='256Kibit/s'
QOS_CLASS_3_MAXBANDWIDTH='768Kibit/s'
QOS_CLASS_3_DIRECTION='down'
QOS_CLASS_3_PRIO=''

QOS_CLASS_4_PARENT='1'
QOS_CLASS_4_MINBANDWIDTH='128Kibit/s'
QOS_CLASS_4_MAXBANDWIDTH='768Kibit/s'
QOS_CLASS_4_DIRECTION='down'
QOS_CLASS_4_PRIO=''

QOS_CLASS_5_PARENT='2'
QOS_CLASS_5_MINBANDWIDTH='256Kibit/s'
QOS_CLASS_5_MAXBANDWIDTH='768Kibit/s'
QOS_CLASS_5_DIRECTION='down'
QOS_CLASS_5_PRIO=''

QOS_CLASS_6_PARENT='2'
QOS_CLASS_6_MINBANDWIDTH='128Kibit/s'
QOS_CLASS_6_MAXBANDWIDTH='768Kibit/s'
QOS_CLASS_6_DIRECTION='down'
QOS_CLASS_6_PRIO=''

QOS_FILTER_N='4'

QOS_FILTER_1_CLASS='3'
QOS_FILTER_1_IP_INTERN='192.168.0.2'
QOS_FILTER_1_IP_EXTERN=''
QOS_FILTER_1_PORT='80'
QOS_FILTER_1_PORT_TYPE='client'
QOS_FILTER_1_OPTION=''

QOS_FILTER_2_CLASS='4'
QOS_FILTER_2_IP_INTERN='192.168.0.2'
QOS_FILTER_2_IP_EXTERN=''
QOS_FILTER_2_PORT=''
QOS_FILTER_2_PORT_TYPE=''
QOS_FILTER_2_OPTION=''

QOS_FILTER_3_CLASS='5'
QOS_FILTER_3_IP_INTERN='192.168.0.3'
QOS_FILTER_3_IP_EXTERN=''
QOS_FILTER_3_PORT='80'
QOS_FILTER_3_PORT_TYPE='client'
QOS_FILTER_3_OPTION=''

QOS_FILTER_4_CLASS='6'
QOS_FILTER_4_IP_INTERN='192.168.0.3'
QOS_FILTER_4_IP_EXTERN=''
QOS_FILTER_4_PORT=''
QOS_FILTER_4_PORT_TYPE=''
QOS_FILTER_4_OPTION=''

\end{verbatim}
\end{example}
\end{small}

    Avec cet exemple l'option \var{QOS\_\-INTERNET\_\-DEFAULT\_\-DOWN}
    a été choisie de telle sorte que le transfert qui n'est pas assigné
    par une classe et un filtre, soit mis dans une classe non existante
    (non paramétrée). La raison, dans cette exemple, c'est qu'il reste 1/3 de
    bande passante non affectée. c'est pour cela que l'on conduit cette B-P sur
    une classe non existante, en plus elle est transmise très lentement.
    S'il reste de la B-P dans la configuration vous devez absolument,
    l'indiquer dans une classe existante.

    L'option \var{QOS\_\-INTERNET\_\-DEFAULT\_\-UP} a été réglé sur 0 pour
    que Upstream (ou débit montant) ne soit pas limité.



\subsubsection{Exemple 3}



    Exemple du mode de fonctionnement de QoS en général ou presque.

    \begin{figure}[htbp]
      \centering
      \includegraphics{qos_bsp_3}
      \caption{exemple 3}
      \marklabel{fig:qosbsp3}{}
    \end{figure}


    Dans la figure \ref{fig:qosbsp3} nous allons revoir la répartition
    du deuxième exemple, mais avec une extension supplémentaire. On ajoutées deux
    sous classe à la sous classe du deuxième niveau. Il est possible d'imbriquer
    des classes encore plus profondément dans cette représentation, la limite
    actuelle se situe ici à 8 branches, on peut produire au maximum de 7 branches
    après l'interface 0, ensuite c'est terminé. Cependant, dans la "largeur"
    aucune limite n'est fixée. Donc vous pouvez rajouter à l'une des sous-classes
    autant de classes que vous voulez.

    On peut voir sur cette figure, qu'il est aussi possible d'intégrer plus
    d'un filtre à une classe comme avec la classe 10. Il est à noter au sujet
    du filtrage, qu'il n'est pas possible de placer un filtre au milieu de
    "l'arborescence" comme indiqué en F8.

    Voyons maintenant, encore une fois le sens des classes et des sous-classes.
    Les classes sont réglementées et sont utilisées pour régler la vitesse
    de connexion. La répartition de la vitesse est effectué avec la variable
    \var{QOS\_\-CLASS\_\-x\_\-MINBANDWIDTH}. Cependant, cela peut avoir un
    inconvénient si par exemple, toutes les sous classes dépendent d'une
    classe. Si on donnait, par exemple, à un ordinateur la moitié de la
    bande passante et l'autre moitié repartie en 2/3 pour le http et 1/3 pour le
    reste, c'est-à-dire 2/6 et 1/6 de l'ensemble. Cela se présente maintenant~: à
    pleine charge chacune des branches reçoit une moitié. Quant à la deuxième,
    oui il ne reste pour le http que 2/6. Cependant, le 2e ordinateur ne reçoit
    pas les 2/6, mais cela est réparti selon la méthode décrite. Pour éviter ceci,
    on crée des sous-classes. Le trafic des classes est distribué d'abord aux
    sous-classes, si celle-ci ne demande pas le transfert complet de
    la bande passante, le restant sera réparti sur d'autres classes. dans la
    figure, les zones qui sont ensembles sont entourées, en Rouge = 1, bleu = 2,
    vert = 5 et orange = 6



\subsubsection{Exemple 4}



    Exemple de configuration, pour la priorité des paquets-ACK et pour que
    le Downstream (ou débit descendant) ne chute pas, il nous faut simultanément
    un Upstream (ou débit  montant) fort~:

\begin{small}
\begin{example}
\begin{verbatim}
OPT_QOS='yes'
QOS_INTERNET_DEV_N='1'
QOS_INTERNET_DEV_1='ppp0'
QOS_INTERNET_BAND_DOWN='768Kibit/s'
QOS_INTERNET_BAND_UP='128Kibit/s'
QOS_INTERNET_DEFAULT_DOWN='0'
QOS_INTERNET_DEFAULT_UP='2'

\end{verbatim}
\end{example}
\end{small}

    Nous configurons ici l'interface pour l'accés Internet (DSL) avec le
    protocole ppp0 et nous indiquons le débit de la bande passante Up/Down
    (ou montant/descendant). Qui est donné par (quelques fournisseur d'accés).
    Il est nécessaire de réduire au minimum la quantité du débit montant de
    la bande passante en Kibibit, pour cela vous devez faire des tests.

    Étant donné que nous ne voulons pas de classe avec Downstream (ou débit
    descendant) nous indiquons

\begin{example}
\begin{verbatim}
        QOS_INTERNET_DEFAULT_DOWN='0'
\end{verbatim}
\end{example}

    Nous indiquons 2 pour le nombre de classe standard avec Upstream (ou
    débit montant). L'interface réseau affecté eth0 a un débit de 10Mibit/s

\begin{small}
\begin{example}
\begin{verbatim}
    QOS_CLASS_N='2'

QOS_CLASS_1_PARENT='0'
QOS_CLASS_1_MINBANDWIDTH='127Kibit/s'
QOS_CLASS_1_MAXBANDWIDTH='128Kibit/s'
QOS_CLASS_1_DIRECTION='up'
QOS_CLASS_1_PRIO=''

\end{verbatim}
\end{example}
\end{small}

    Il s'agit ci-dessous de construire la classe dans laquelle nous
    allons insérer les paquets-ACK pour les (accusés de réception). Les
    paquets-ACK sont assez petites, c'est pour cette raison qu'il on besoin
    de peu de bande passante. Néanmoins, nous voulons en aucune façon partager
    ses 127Kibit. Nous laisserons donc 1Kibit/s pour le reste.

\begin{small}
\begin{example}
\begin{verbatim}
QOS_CLASS_2_PARENT='0'
QOS_CLASS_2_MINBANDWIDTH='1Kibit/s'
QOS_CLASS_2_MAXBANDWIDTH='128Kibit/s'
QOS_CLASS_2_DIRECTION='up'
QOS_CLASS_2_PRIO=''
\end{verbatim}
\end{example}
\end{small}

    Dans la classe ci-dessus, on insère le reste (tout sauf les paquets-ACK).
    La bande passante que nous allons indiquer dans cette classe est
    1Kibit/s restants, donc (128-127=1). La somme de 1Kibit/s que nous avons
    enregistré, n'est pas limité.

\begin{example}
\begin{verbatim}
        QOS_CLASS_2_MAXBANDWIDTH='128Kibit/s'
\end{verbatim}
\end{example}

    Notre première classe utilisera peut-être, à peine toute la bande
    passante attribuée et si il en reste un peu, alors, le reste de la
    B-P, passera sur la deuxième classe. Si l'on veut diviser encore
    davantage le Upstream (ou débit montant) (ce qui est généralement
    le cas), toutes les autres sous-classes "dépendent" de cette classe.
    Il faut, bien sûr paramétrer la variable \var{QOS\_\-INTERNET\_\-DEFAULT\_\-UP}
    en conséquence.

\begin{small}
\begin{example}
\begin{verbatim}
QOS_FILTER_N='1'

QOS_FILTER_1_CLASS='1'
QOS_FILTER_1_IP_INTERN=''
QOS_FILTER_1_IP_EXTERN=''
QOS_FILTER_1_PORT=''
QOS_FILTER_1_PORT_TYPE=''
QOS_FILTER_1_OPTION='ACK'
\end{verbatim}
\end{example}
\end{small}

    Ce filtre, filtre tous les paquets qui s'appliquent au option de filtrage,
    donc les paquets-ACK. Grâce à l'enregistrement de la variable \var{QOS\_\-FILTER\_\-1\_\-CLASS}='1'
    nous somme sur de filtrer les paquets de la 1ére classe.

    Pour tester, on doit rechercher au mieux une ou plusieurs bonnes
    sources en "envoyant/recevant des données", en sachant que l'on
    doit utiliser tout le débit montant/descendant et de faire "chauffer
    les câbles". Il faut jeter un coup d'oeil sur l'état du trafic avec
    Imonc (s'il est installé). Le meilleur moyen est de le faire sans activer QoS.

    Le Downstream (ou débit descendant) ne devrait pas du tout chuter
    ou beaucoup moins fort que sans cette configuration. Comme je l'ai dit,
    on peut encore améliorer la situation du débit montant de la bande passante
    par incrémentation en Kibibit pour réduire au minimum et de vérifier les
    effets. Chez moi, par exemple, le meilleur débit atteint est de 121Kibit/s
    (en plus pas de chute avec le débit descendant). Il faut bien sûr pour chaque
    classe prendre en considération les valeurs de MAXBANDWIDTH et MINBANDWIDTH.
