% Last Update: $Id$

\section{QoS - Quality of Service}
\marklabel{sec:qos}{}

   Mit QoS kann man die verfügbare Bandbreite regulieren und zum Beispiel
   auf verschiedene Ports, IP's und noch einiges mehr zu verteilen.

   Ein Modem verwaltet eine Packet-Queue (Queue = Schlange, Reihe (in
   einer Schlange stehen)) in der Pakete gespeichert werden, die die
   verfügbare Bandbreite überschreiten. Bei DSL-Modems zum Beispiel, sind
   diese sehr groß. Das hat den Vorteil, dass recht gleichmäßig die
   maximale Bandbreite ausgenutzt werden kann. Denn schickt der Router an
   das Modem für kurze Zeit (sehr kurz) weniger Pakete, dann hat das
   Modem noch immer Pakete in der Queue, die es abzuarbeiten gilt. So
   eine Queue ist sehr simpel gehalten, denn dort geht alles der Reihe
   nach, ist eben ein faires Modem :-D

   Und hier kommt dann QoS ins Spiel. QoS verwaltet auch eine
   Packet-Queue, nur eben im Router selber und dort hat man die
   Möglichkeit König zu sein, eben zu entscheiden welches Paket zu erst
   darf und welche Pakete sich noch ein bischen zurückhalten müssen. Wenn
   alles richtig konfiguriert wurde, dann sendet QoS die Pakete gerade
   eben so schnell, dass sie nicht in die Packet-Queue des Modems landen.
   Das wäre so, als hätte man die Queue vom Modem in den Router geholt.

   \marklabel{hint:speedunits}{}Noch etwas allgemeines zu den Geschwindigkeitseinheiten:
   QoS unterstützt Mibit/s (mebibit/s) und Kibit/s (kibibit/s), wobei gilt 1Mibit
   = 1024Kibit.

\subsection{Konfiguration}

\begin{description}

\config{OPT\_QOS}{OPT\_QOS}{OPTQOS}
   Hier ist yes zusetzen wenn man das \var{OPT\_\-QOS} einsetzen will und no wenn
   man das Gegenteil beabsichtigt

\config{QOS\_INTERNET\_DEV\_N}{QOS\_INTERNET\_DEV\_N}{QOSINTERNETDEVN}

Die Anzahl der Devices, die Daten ins Internet routen.

\config{QOS\_INTERNET\_DEV\_x}{QOS\_INTERNET\_DEV\_x}{QOSINTERNETDEVx}

   Hier sollte die Liste der Devices eingetragen werden, über die
   Daten ins Internet übertragen werden. Beispiele:

   \begin{tabular}[h!]{ll}
     \var{QOS\_\-INTERNET\_\-DEV\_N}='3'    & Anzahl der Geräte \\
     \var{QOS\_\-INTERNET\_\-DEV\_1}='ethX' & für Kabel und sonstige
     Ethernet-Verbindungen \\
     \var{QOS\_\-INTERNET\_\-DEV\_2}='ppp0' &   für DSL über PPPoE \\

     \var{QOS\_\-INTERNET\_\-DEV\_3}='ipppX' &    für ISDN\\
   \end{tabular}

   Das ISDN-Device für den ersten Circuit dürfte das ippp0 lauten, für
   den 2.  ippp1.  Wenn aber für den ersten Circuit Kanalbündelung
   aktiviert wurde, dann heißt der 2. Kanal des 1. Circuit ippp1 und
   der 2. Circuit ippp2.  Man sollte QOS mit ISDN nur dann nutzen,
   wenn Kanalbündelung für den benutzten Circuit deaktiviert ist.


\config{QOS\_INTERNET\_BAND\_DOWN}{QOS\_INTERNET\_BAND\_DOWN}{QOSINTERNETBANDDOWN}

   Maximale Downstreambandbreite des Internetzugangs. Siehe weiter oben: \jump{hint:speedunits}{Hinweis zu
   den Geschwindigkeitseinheiten}.

   Hinweis: Für zeitkritische Aufgaben, wie das bevorzugen von
   ACK-Paketen, ist es nötig die Bandbreite nicht höher zu setzen als
   wirklich vorhanden, da man sonst zwar innerhalb der Packet-Queue auf
   dem Router die Pakete sortiert, dies dann aber nicht ganz korrekt
   gemacht wird und letztendlich doch wieder in der Packet-Queue des
   Modems aufgehalten werden. Möglich ist es außerdem, das die vom
   Provider angegebene Bandbreite nicht hundertprozentig mit der wirklich
   verfügbaren übereinstimmt, es könnte ein bischen mehr oder auch
   weniger sein. Da ist also ausprobieren angesagt.


\config{QOS\_INTERNET\_BAND\_UP}{QOS\_INTERNET\_BAND\_UP}{QOSINTERNETBANDUP}

   Maximale Upstreambandbreite des Internet-Zugangs. Siehe Hinweis zu den
   Geschwindigkeitseinheiten unter \var{OPT\_\-QOS}.

   Hinweis: Siehe Hinweis bei \var{QOS\_\-INTERNET\_\-BAND\_\-DOWN}.


\config{QOS\_INTERNET\_DEFAULT\_DOWN}{QOS\_INTERNET\_DEFAULT\_DOWN}{QOSINTERNETDEFAULTDOWN}

   Hier ist die Standardklasse für Pakete anzugeben, die aus dem Internet
   kommen. Alle Pakete, die nicht durch einen Filter in eine Klasse
   gesteckt wurden, landen dann in der angegebenen Klasse.

   Wurde keine Klasse eingerichtet, für die die Variable
\begin{example}
\begin{verbatim}
        QOS_CLASS_x_DIRECTION='down'
\end{verbatim}
\end{example}
   gesetzt wurde, so setzt man:
\begin{example}
\begin{verbatim}
        QOS_INTERNET_DEFAULT_DOWN='0'
\end{verbatim}
\end{example}

   Beispiel:

   Es wurden 2 Klassen eingerichtet und ein Filter steckt alle Pakete,
   die z.B. an eine bestimmte IP-Adresse geschickt wurden in die 1. von
   den beiden. Alle anderen Pakete sollen in die 2. Klasse gesteckt
   werden. Folglich müßte hier

\begin{example}
\begin{verbatim}
        QOS_INTERNET_DEFAULT_DOWN='2'
\end{verbatim}
\end{example}

   eingetragen werden.

   Es ist darauf zu achten, dass für \var{QOS\_\-INTERNET\_\-DEFAULT\_\-DOWN} eine Klasse
   angegeben wird, deren \var{QOS\_\-CLASS\_\-x\_\-DIRECTION} Variable das Argument
   down enthält.


\config{QOS\_INTERNET\_DEFAULT\_UP}{QOS\_INTERNET\_DEFAULT\_UP}{QOSINTERNETDEFAULTUP}

   Hier ist die Standardklasse für Pakete anzugeben, die in das Internet
   gehen. Alle Pakete, die nicht durch einen Filter in eine Klasse
   gesteckt wurden, landen dann in der angegebenen Klasse.

   Wurde keine Klasse eingerichtet, für die die Variable
\begin{example}
\begin{verbatim}
        QOS_CLASS_x_DIRECTION='up'
\end{verbatim}
\end{example}

gesetzt wurde, so setzt man:
\begin{example}
\begin{verbatim}
        QOS_INTERNET_DEFAULT_UP='0'
\end{verbatim}
\end{example}

   Das ganze funktioniert analog zu \var{QOS\_\-INTERNET\_\-DEFAULT\_\-DOWN}.

   Es ist darauf zu achten, dass für \var{QOS\_\-INTERNET\_\-DEFAULT\_\-UP} eine Klasse
   angegeben wird, deren \var{QOS\_\-CLASS\_\-x\_\-DIRECTION} Variable das Argument
   up enthält.

\config{QOS\_CLASS\_N}{QOS\_CLASS\_N}{QOSCLASSN}

   Hier ist die gewünschte Anzahl der Klassen (engl. Class) anzugeben.


\config{QOS\_CLASS\_x\_PARENT}{QOS\_CLASS\_x\_PARENT}{QOSCLASSxPARENT}

   Mit dieser Variable kann man Klassen verschachteln. Man gibt hier
   immer die Nummer der Vaterklasse an. Die Bandbreite die der
   Vaterklasse zugeteilt wurde, kann dann unter den Unterklassen weiter
   aufgeteilt werden. Die maximale Verschachtelungstiefe beträgt hier 8
   Ebenen, wobei das Interface selber schon eine Ebene darstellt, es
   bleiben also maximal 7 konfigurierbar.

   Soll die Klasse keine Unterklasse sein, so gibt man hier folgendes an:

\begin{example}
\begin{verbatim}
        QOS_CLASS_x_PARENT='0'
\end{verbatim}
\end{example}

   Ihr wird dann je nachdem zu welcher Richtung sie gehört (siehe 
   \var{QOS\_\-CLASS\_\-x\_\-PORT\_\-TYPE}), maximal die in \var{QOS\_\-CLASS\_\-x\_\-PORT\_\-TYPE}
   oder \var{QOS\_\-INTERNET\_\-BAND\_\-DOWN} angegebene Bandbreite zugeteilt.

   Wichtig: Falls hier nicht '0' angegeben wird, so ist darauf zu achten,
   dass die Vaterklasse vorher definiert wird (auf die Nummerierung bezogen).


\config{QOS\_CLASS\_x\_MINBANDWIDTH}{QOS\_CLASS\_x\_MINBANDWIDTH}{QOSCLASSxMINBANDWIDTH}

   Bandbreite, die man der Klasse zusprechen will. Man könnte hier auch
   von einem Verhältnis sprechen. Siehe Hinweis zu den
   Geschwindigkeitseinheiten unter \var{OPT\_\-QOS}.

   Beispiel:
   Man hat eine Klasse, dessen Bandbreite auf 128Kibit/s beschränkt ist.:

\begin{example}
\begin{verbatim}
        QOS_CLASS_1_MINBANDWIDTH='128Kibit/s'
        QOS_CLASS_1_MAXBANDWIDTH='128Kibit/s'
        QOS_CLASS_1_PARENT='0'
\end{verbatim}
\end{example}


   Weiterhin hat man 3 Klassen dessen \var{QOS\_\-CLASS\_\-x\_\-MINBANDWIDTH}- und
   \var{QOS\_\-CLASS\_\-x\_\-MAXBANDWIDTH}-Einstellungen wie folgt aussehen und alle
   Unterklassen unserer ersten Klasse sind:

\begin{example}
\begin{verbatim}
        QOS_CLASS_2_PARENT='1'
        QOS_CLASS_2_MINBANDWIDTH='60Kibit/s'
        QOS_CLASS_2_MAXBANDWIDTH='128Kibit/s'

        QOS_CLASS_3_PARENT='1'
        QOS_CLASS_3_MINBANDWIDTH='40Kibit/s'
        QOS_CLASS_3_MAXBANDWIDTH='128Kibit/s'

        QOS_CLASS_4_PARENT='1'
        QOS_CLASS_4_MINBANDWIDTH='28Kibit/s'
        QOS_CLASS_4_MAXBANDWIDTH='128Kibit/s'
\end{verbatim}
\end{example}


   Alle Unterklassen besitzen die selbe (oder auch keine) Priorität
   (siehe \var{QOS\_\-CLASS\_\-x\_\-PRIO}).
   Wird nun auf jede dieser 3 Klassen mehr Verkehr produziert als in
   ihrer jeweiligen \var{QOS\_\-CLASS\_\-x\_\-MINBANDWIDTH} angegeben, so bekommt jede
   Klasse entsprechend ihrer \var{QOS\_\-CLASS\_\-x\_\-MINBANDWIDTH}-Einstellung
   Bandbreite zugewiesen.
   Wenn aber z.B. Klasse 2 nur 20Kibit/s an Verkehr produziert, dann läßt
   dies Klasse ja 40Kibit/s ``übrig''. Dieser Überschuß wird im Verhältnis
   40/28 unter Klasse 3 und 4 aufgeteilt.
   Jede Klasse selber ist durch \var{QOS\_\-CLASS\_\-x\_\-MAXBANDWIDTH} auf 128Kibit/s
   beschränkt und da sie alle Unterklassen einer auf 128Kibit/s beschränkten
   Klasse sind, können sie auch alle zusammen nicht mehr als 128Kibit/s
   konsumieren.


\config{QOS\_CLASS\_x\_MAXBANDWIDTH}{QOS\_CLASS\_x\_MAXBANDWIDTH}{QOSCLASSxMAXBANDWIDTH}

   Bandbreite, die man der Klasse maximal zuteilen will. Es macht keinen
   Sinn einen niedrigeren Wert als der in \var{QOS\_\-CLASS\_\-x\_\-MINBANDWIDTH}
   einzutragen. Gibt man hier nichts an, so nimmt diese Variable
   automatisch den Wert von \var{QOS\_\-CLASS\_\-x\_\-MINBANDWIDTH} an. Eine solche
   Klasse kann dann natürlich keine überschüssige Bandbreite
   beanspruchen.

   Siehe Hinweis zu den Geschwindigkeitseinheiten unter \var{OPT\_\-QOS}.


\config{QOS\_CLASS\_x\_DIRECTION}{QOS\_CLASS\_x\_DIRECTION}{QOSCLASSxDIRECTION}

   Mit dieser Variable wird angegeben, zu welcher Richtung die Klasse
   gehört. Soll sie zur Regulierung des Upstreams benutzt werden, so ist
   hier

\begin{example}
\begin{verbatim}
        QOS_CLASS_x_DIRECTION='up'
\end{verbatim}
\end{example}

   anzugeben, für den Downstream analog:

\begin{example}
\begin{verbatim}
        QOS_CLASS_x_DIRECTION='down'
\end{verbatim}
\end{example}


\config{QOS\_CLASS\_x\_PRIO}{QOS\_CLASS\_x\_PRIO}{QOSCLASSxPRIO}

   Hier wird geregelt, welche Priorität ein Klasse hat. Je niedriger die
   Nummer, desto höher die Priorität. Erlaubt sind werte zwischen 0 und
   7. Wenn die Variable leer gelassen wird, so kommt das dem setzen einer
   0 gleich.

   Wenn eine Priorität gesetzt wird, dann wird darüber bestimmt, welcher
   Klasse zuerst Überschüssige Bandbreite angeboten wird. Um das klar zu
   machen, ändern wir das Beispiel aus \var{QOS\_\-CLASS\_\-x\_\-MINIMUMBANDWIDTH}
   leicht ab:
   An der ersten Klasse wird nichts verändert. Die Klassen 2-4 bekommen
   eine Priorität zugewiesen:

\begin{example}
\begin{verbatim}
        QOS_CLASS_2_PARENT='1'
        QOS_CLASS_2_MINBANDWIDTH='60Kibit/s'
        QOS_CLASS_2_MAXBANDWIDTH='128Kibit/s'
        QOS_CLASS_2_PRIO='1'

        QOS_CLASS_3_MINBANDWIDTH='40Kibit/s'
        QOS_CLASS_3_PARENT='1'
        QOS_CLASS_3_MAXBANDWIDTH='128Kibit/s'
        QOS_CLASS_3_PRIO='1'

        QOS_CLASS_4_PARENT='1'
        QOS_CLASS_4_MINBANDWIDTH='28Kibit/s'
        QOS_CLASS_4_MAXBANDWIDTH='128Kibit/s'
        QOS_CLASS_4_PRIO='2'
\end{verbatim}
\end{example}


   Wie in dem Ursprungsbeispiel konsumiert Klasse 2 nur 20Kibit/s und läßt
   somit einen Überschuß von 40Kibit/s übrig. Klasse 3 und 4 wollen noch
   immer mehr Bandbreite als überhaupt verfügbar. Da nun aber Klasse 3
   eine höhere Priorität als Klasse 4 hat, darf sie den Überschuß von
   40Kibit/s vertilgen.

   Angenommen Klasse 3 braucht aber nur 20Kibit/s des ursprünglichen
   Überschusses von 40Kibit/s, dann bekommt Klasse 4 die restlichen 20Kibit/s.

   Nehmen wir nochmals etwas anderes an: Klasse 4 verbraucht gar keine
   Bandbreite und Klasse 2 und 3 wollen mehr als es überhaupt gibt. Dann
   bekommt jede erstmal ihre in \var{QOS\_\-CLASS\_\-x\_\-MINBANDWIDTH} angegebene
   Bandbreite und der Rest wird unter ihnen im 60/40 Verhältnis
   aufgeteilt, da beide Klassen die selbe Priorität haben.

   Wie man also sieht beeinflußt \var{QOS\_\-CLASS\_\-x\_\-PRIO} nur, wie ein
   eventueller Bandbreitenüberschuß aufgeteilt wird.
   
\config{QOS\_CLASS\_x\_LABEL}{QOS\_CLASS\_x\_LABEL}{QOSCLASSxLABEL}

   Mit dieser optionalen Variable kann ein Label für die Klasse gesetzt werden.
   Dieses wird bei aktivem OPT\_RRDTOOL zur Beschriftung der Graphen von QOS
   genutzt.


\config{QOS\_FILTER\_N}{QOS\_FILTER\_N}{QOSFILTERN}

   Gewünschte Anzahl der der Filter angeben.

   Zu den Filtern allgemein läßt sich noch folgendes sagen:
   Die Argumente von verschiedenen Variablen sind UND-verknüpft, mehrere
   Argumente der selben Variable sind ODER-verknüpft. Soll heißen: Wird
   zum Beispiel in einem und dem selben Filter nach einer IP-Adresse und
   einem Port gefiltert, so werden nur Pakete herausgefiltert und in die
   Zielklasse(n) gesteckt, die auf beides gleichzeitig zutreffen.

   Ein weiteres Beispiel:
   In einem und dem selben Filter sind zwei Ports (21 und 80) und eine
   IP-Adresse angegeben. Ein Datenpaket kann natürlich nicht von zwei
   Ports gleichzeitig kommen. Es verhält sich dann so: Der Filter filtert
   Pakete heraus, die entweder Port 21 benutzen und gleichzeitig die
   IP-Adresse, oder von Port 80 kommen und gleichzeitig von der
   IP-Adresse.

   Wichtig: Es kommt auf die Reihenfolge der Filter an!

   Ein Beispiel: Man möchte den Verkehr, der über den Port 456 läuft,
   für \textbf{alle} Rechner in Klasse A stopfen. Zusätzlich möchte man
   alle Pakete an den Rechner mit der IP 192.168.6.5 - bis auf die
   Pakete über Port 456 - in Klasse B laufen lassen.  Richte ich nun
   den Filter mit der IP als erstes ein, dann landen alle Pakete -
   auch die über Port 456 laufen - in Klasse B und ein nachfolgender
   Filter für den Port 456 ändert auch nichts daran. Der Filter für
   den Port 456 muß also noch vor dem Filter mit der IP 192.168.6.5
   stehen.


\config{QOS\_FILTER\_x\_CLASS}{QOS\_FILTER\_x\_CLASS}{QOSFILTERxCLASS}

   Mit dieser Variable stellt ihr ein, in welche Klasse das Paket, auf
   den dieser Filter zutrifft, gesteckt werden soll. Möchte man zum
   Beispiel die gefilterten Pakete in die Klasse, die mit den Variabeln
   \var{QOS\_\-CLASS\_\-25\_\-MINBANDWIDTH} spezifiziert wurde, stecken, so müßte man
   hier

\begin{example}
\begin{verbatim}
        QOS_FILTER_x_CLASS='25'
\end{verbatim}
\end{example}


   setzen.

   Mit \var{QOS\_\-CLASS\_\-x\_\-DIRECTION} gibt man ja an, ob eine Klasse nun zum Up-
   oder Downstream gehört. Wenn nun ein Filter gesetzt wird, der
   gefilterte Pakete zum Beispiel in eine Upstream-Klasse wandern läßt,
   dann werden auch nur Pakete aus dem Upstream von diesem Filter
   gefiltert und in die angegebene Klasse gesteckt. \var{QOS\_\-CLASS\_\-x\_\-DIRECTION}
   bestimmt also in welcher ``Richtung'' gefiltert wird.

   Seit Version 2.1 ist es nun auch möglich mehr als eine Zielklasse
   anzugeben. Möchte man zum Beispiel den Verkehr über Port 456 sowohl
   für den Upstream als auch für den Downstream klassifizieren, so würde
   man hier

\begin{example}
\begin{verbatim}
        QOS_FILTER_x_CLASS='4 25'
\end{verbatim}
\end{example}

   angeben, wobei zum Beispiel Klasse Nummer 4 die Upstreamklasse ist und
   25 die Downstreamklasse. Es macht keinen Sinn hier jeweils mehr als
   ein Up- und Downstreamklassen anzugeben, somit wird man auch
   nie mehr als zwei Zielklassen eintragen.


\config{QOS\_FILTER\_x\_IP\_INTERN}{QOS\_FILTER\_x\_IP\_INTERN}{QOSFILTERxIPINTERN}

   Hier können IP-Adressen und IP-Bereiche aus den internen Netzwerkwerken,
   nach denen gefiltert werden soll, angegeben werden. Sie sind durch Leerzeichen zu
   trennen und können frei kombiniert werden.

   Das könnte zum Beispiel so aussehen:

\begin{example}
\begin{verbatim}
        QOS_FILTER_x_IP_INTERN='192.168.6.0/24 192.168.5.7 192.168.5.12'
\end{verbatim}
\end{example}

   Hier werden alle Adressen in der Form 192.168.6.X gefiltert und
   zusätzlich noch die IPs 192.168.5.7 und 192.168.5.12.

   Diese Variable darf auch leer gelassen werden.

   Wird diese Variable gleichzeitig mit \var{QOS\_\-FILTER\_\-x\_\-IP\_\-EXTERN}
   genutzt, so wird nur Verkehr gefiltert, der zwischen den in
   \var{QOS\_\-FILTER\_\-x\_\-IP\_\-INTERN} und \var{QOS\_\-FILTER\_\-x\_\-IP\_\-EXTERN}
   angegebenen IPs oder IP-Bereichen stattfindet.

   \sloppypar\wichtig{Falls zusätzlich durch \var{QOS\_\-FILTER\_\-x\_\-OPTION} nach ACK,
   TOSMD, TOSMT, TOSMR oder TOSMC gefiltert wird und die Variable
   \var{QOS\_\-CLASS\_\-x\_\-DIRECTION} der Zielklasse 'down' entspricht, dann wird
   diese Variable ignoriert.}


\config{QOS\_FILTER\_x\_IP\_EXTERN}{QOS\_FILTER\_x\_IP\_EXTERN}{QOSFILTERxIPEXTERN}

   Hier können IP-Adressen und IP-Bereiche aus dem externen Netzwerke
   (welches über \var{QOS\_\-INTERNET\_\-DEV} angebunden ist), nach denen
   gefiltert werden soll, angegeben werden. Sie sind durch Leerzeichen zu
   trennen und können frei kombiniert werden. Das ganze funktioniert
   analog zu \var{QOS\_\-FILTER\_\-x\_\-IP\_\-INTERN}.

   Diese Variable darf auch leer gelassen werden.

   \sloppypar\wichtig{Falls zusätzlich durch \var{QOS\_\-FILTER\_\-x\_\-OPTION} nach ACK,
   TOSMD, TOSMT, TOSMR oder TOSMC gefiltert wird und die Variable
   \var{QOS\_\-CLASS\_\-x\_\-DIRECTION} der Zielklasse 'down' entspricht, dann wird
   diese Variable ignoriert.}


\config{QOS\_FILTER\_x\_PORT}{QOS\_FILTER\_x\_PORT}{QOSFILTERxPORT}

   Hier können Ports und Portranges angegeben werden, getrennt durch
   Leerzeichen und dürfen frei kombiniert werden. Falls die Variable leer
   ist, werden Übertragungen über sämtliche Ports limitiert.

   Zu dem Format von Portranges: Möchte man nach Ports von 5000 bis 5099
   filtern, so würde das folgender Maßen aussehen:

\var{QOS\_\-FILTER\_\-x\_\-PORT}='5000-5099'

   Ein weiteres Beispiel:
   Man möchte den Verkehr über die Ports 20 bis 21, 137 bis 139 und Port
   80 filtern und in die selbe Klasse stecken lassen. Das sähe dann so
   aus:

\begin{example}
\begin{verbatim}
        QOS_FILTER_x_PORT='20-21 137-139 80'
\end{verbatim}
\end{example}

   Diese Variable darf auch leer gelassen werden.

   Wichtig:
   \begin{itemize}
   \item  Wenn nach Ports gefilter wird, muß auch \var{QOS\_\-FILTER\_\-x\_\-PORT\_\-TYPE}
   etsprechend gesetzt werden.


 \item Wenn zusätzlich durch \var{QOS\_\-FILTER\_\-x\_\-OPTION} nach ACK, TOSMD,
   TOSMT, TOSMR oder TOSMC gefiltert wird, dann werden Portranges
   ignoriert.
   \end{itemize}


\config{QOS\_FILTER\_x\_PORT\_TYPE}{QOS\_FILTER\_x\_PORT\_TYPE}{QOSFILTERxPORTTYPE}

   Das setzen dieser Variable ist nur wichtig im Zusammenhang mit
   \var{QOS\_\-FILTER\_\-x\_\-PORT} und muß auch nur dann gesetzt werden (wird
   ansonsten ignoriert).

   Da sich die Ports beim Clientbetrieb von den Ports beim Serverbetrieb
   unterscheiden, muss angegeben werden ob der Port des Servers oder
   Clients gemeint ist. Als Bezugspunkte gelten hier die Rechner aus dem
   eigenen Netz.

   Folgende Einstellungen sind Möglich:

\begin{example}
\begin{verbatim}
        QOS_FILTER_x_PORT_TYPE='client'
        QOS_FILTER_x_PORT_TYPE='server'
\end{verbatim}
\end{example}
   Seit Version 2.1 ist auch die Kombination der beiden Argumenten
   möglich, um sowohl den Verkehr über den angegebenen Port aus dem
   eigenen Netz, als auch den Verkehr über selbigen Port aus dem Internet
   in die selbe Klasse zu stecken:

\begin{example}
\begin{verbatim}
        QOS_FILTER_x_PORT_TYPE='client server'
\end{verbatim}
\end{example}

   Dies entspricht der Erstellung von zwei ähnlichen Filtern, bei denen
   \var{QOS\_\-FILTER\_\-x\_\-PORT\_\-TYPE} einmal auf Client und einmal auf Server gesetzt
   wurde.


\config{QOS\_FILTER\_x\_OPTION}{QOS\_FILTER\_x\_OPTION}{QOSFILTERxOPTION}

   Mit dieser Variable kann man weitere Eigenschaften für den Filter
   aktiveren. Es darf hier höchstens eines der folgenden Argumente
   angegeben werden, denn eine Kombination dieser in ein und dem selben
   Filter macht keinen Sinn. Hingegen ist es sehr wohl möglich und auch
   teilweise sinnvoll, dass zum Beispiel ein Filter für ACK-Pakete und ein
   2. Filter für TOSMD-Pakete ihre Pakete in die selbe Zielklasse leiten
   (siehe \var{QOS\_\-FILTER\_\-x\_\-CLASS}).

   \begin{description}
   \item[ACK] Acknowledgement-Pakete.

   Ein Paket das auf diese Option zutrifft, wird als Bestätigung für ein 
   Datenpaket zurückgesendet. Wenn ihr z.B. einen großen Download am
   laufen habt, dann kommen bei euch viele Datenpakete an und für jedes
   muß eine Antwort gesendet werden, dass das Datenpaket angekommen ist.
   Lassen diese Bestätigungspakete auf sich warten, so so wartet der
   Datenversender ab, bis diese eingetroffen sind, bevor er neue
   Datenpakete sendet, was euch nicht so richtig schmeckt.

   Das ganze ist insbesondere wichtig bei asymetrischen Verbindungen
   (ungleiche Up/Downstream-Bandbreite), wie sie bei den meisten privaten
   DSL-Angeboten üblich sind. Wird der meist relativ kleine Upstream an
   seine Grenzen gefahren, so stapeln sich die Pakete vor dem Ausgang
   förmlich auf und irgendwo in diesem riesigen Haufen sitzen hier und da
   die kleinen Bestätigungspakete. Im Normalfall geht das dann hübsch der
   Reihe nach. Bis das Bestätigungspaket dann an der Reihe ist, kann es
   gut sein, dass es so lange gedauert hat, dass unser Datenversender eine
   kleine Pause einlegt und was wie gesagt nicht gut für den Downstream
   ist.

   Wir müssen also dafür sorgen, dass die Bestätigungspakete auf die
   Überholspur kommen, so dass sie in windeseile an allen ``normalen''
   Paketen vorbeihuschen, damit sie auch noch rechtzeitig beim
   Datenversender ankommen. Wie sich dies Option sinnvoll mit einer
   Klasse kombinieren läßt, wird bei den Anwendungsbeispielen
   erläutert.


 \item[ICMP] Ping-Pakete (Protokoll ICMP)

   Ping-Pakete werden dazu benutzt, die Zeit zu messen, die ein Paket von
   A nach B braucht. Wenn ihr also ordentlich angeben wollt, dann gebt
   den Ping-Paketen z.B. eine höhere Priorität. Das hat jetzt nichts mit
   dem Spielen im Internet selber zu tun, also nicht denken nur weil ihr
   den Ping-Pakete den Vorrang gibt, dass ihr super niedrige Pingzeiten im
   Spiel bekommt...

 \item[IGMP] IGMP-Pakete (Protokoll IGMP)

   Wenn IP-TV benutzt wird, ist es sinnvoll, das IGMP Protokoll zu filtern
   und zu priorisieren.

 \item[TCPSMALL] Kleine TCP Pakete

   Durch diesen Filter können ausgehende HTTP(s)-Requests gefiltert priorisiert
   werden. Eine Kombination mit einem Zielport ist möglich und sinnvoll.
   Größe der TCP Pakete: max. 800 Byte.

 \item[TCP]    TCP-Pakete (Protokoll TCP)

   Es wird nur nach Paketen gefiltert, die das Protokoll TCP benutzen.

 \item[UDP]  UDP-Pakete (Protokoll UDP)

   Es wird nur nach Paketen gefiltert, die das Protokoll UDP benutzen.


 \item[TOS*] Type of Service

   TOS steht für ``Type of Service''. Eine Applikation kann für jedes
   Paket was es verschickt eines der 4 TOS-Bits setzen. Damit wird
   angegeben welche Behandlung für die Pakete vorgesehen sind. So kann
   z.B. SSH TOS-Minimum-Delay für das Versenden der ein und Ausgabe
   setzen und TOS-Maximum-Troughput für das Versenden von Dateien.
   Generell benutzen Linux/Unix Programme diese Bits öfter als
   Windowsprogramme. Außerdem kann man z.B. auch it der Firewall die
   TOS-Bits für bestimmte Pakete setzen. Letztendlich kommt es dann
   aber auf die Router auf der Strecke an, ob die TOS-Bits beachtet
   werden, oder nicht.  Wirklich von Interesse für einen fli4l sind
   aber eigentlich nur die TOS-Bits Minimum-Delay und
   Maximum-Throughput.

   \begin{description}
   \item [TOSMD - TOS Minimum-Delay]

   Wird für Dienste benutzt, bei denen es wichtig ist, dass Pakete
   möglichst ohne Zeitverzögerung weitergeleitet werden. Empfohlen wird
   dieses TOS-Bit für FTP (Kontrolldaten), Telnet und SSH.


 \item[TOSMT - TOS Maximum-Troughput]

   Wird für Dienste benutzt, bei denen es wichtig ist, dass große
   Datenmengen mit hoher Geschwindigkeit weitergeleitet werden. Empfohlen
   wird dieses TOS-Bit für FTP-Data und WWW.


 \item[TOSMR - TOS Maximum-Reliability]

   Wird benutzt, wenn es wichtig ist, das man eine gewisse Sicherheit
   hat, dass die Daten an ihr Ziel gelangen, ohne das ein erneutes senden
   nötig ist. Empfohlen wird dieses TOS-Bit für SNMP und DNS.


 \item[TOSMC - TOS Minimum-Cost]

   Wird benutzt, wenn es wichtig ist die Kosten der Datenübertragung zu
   Minimieren. Empfohlen wird dieses TOS-Bit für NNTP und SMTP.

 \item[DSCP*] Differentiated Services Code Point

   Mit DSCP bezeichnet man die Markierung nach RFC\ 2474.
   Dieses Verfahren hat 1998 die TOS Markierung weitestgehend abgelöst.

   Die Filter auf DSCP-Klassen können wie folgt konfiguriert werden:

\begin{example}
\begin{verbatim}
        QOS_FILTER_x_OPTION='DSCPef'
        QOS_FILTER_x_OPTION='DSCPcs3'
\end{verbatim}
\end{example}

   Bitte beachten, dass DSCP groß und die Klasse kleingeschrieben wird.

   Es können folgende Klassen verwendet werden:

   af11-af13, af21-af23, af31-af33, af41-af43, cs1-cs7, ef und be (Standard)

 \end{description}
\end{description}
\end{description}





\marklabel{sec:qosanwendung}{
  \subsection{Anwendungsbeispiele}
}



   Wie konfiguriert man \var{OPT\_\-QoS} nun genau? Dies wird nun an einigen
   Beispielen gezeigt:
   \begin{itemize}

   \item Beispiel 1: Ein einfaches Beispiel mit dem Ziel die
     Bandbreite auf 3 Rechner zu verteilen.

   \item Beispiel 2: Ein Beispiel mit dem Ziel die Bandbreite auf 2
     Rechner zu verteilen und die jeweiligen Bandbreiten pro Rechner
     wiederum noch ein zweites mal aufzuteilen auf einen Port und den
     restlichen Verkehr des jeweiligen Rechners.

   \item Beispiel 3: Ein Beispiel, welches die allgemeine
     Funktionsweise von QoS versucht nahezubringen.

   \item Beispiel 4: Beispielkonfiguration für das Bevorteilen von
     ACK-Paketen, damit der Downstream bei gleichzeitig starkem
     Upstream nicht einbricht.
   \end{itemize}



\subsubsection{Beispiel 1}



   Ein einfaches Beispiel mit dem Ziel die Bandbreite auf 3 Rechner zu
   verteilen.

   Dazu erstellen wir 4 Klassen (siehe \var{QOS\_\-CLASS\_\-N} und folgende)
   mit den jeweiligen Geschwindigkeiten (siehe
   \var{QOS\_\-CLASS\_\-x\_\-MINBANDWIDTH} / \var{QOS\_\-CLASS\_\-x\_\-MIN\-BAND\-WIDTH}) und
   hängen sie an die Klasse 0 (siehe \var{QOS\_\-CLASS\_\-x\_\-PARENT}) also
   direkt an das Interface für ``up'' bzw. ``down'' (siehe
   \var{QOS\_\-CLASS\_\-x\_\-DIRECTION}).

   Die vierte Klasse ist nur für eventuelle Besucher und bekommt weniger
   Bandbreite zugeteilt. Mit \var{QOS\_\-INTERNET\_\-DEFAULT\_\-DOWN}='4' lassen wir in
   diesem Fall allen nicht gefilterten Verkehr in die vierte
   ``Gast''-Klasse wandern. Da wir aber selten Gäste haben und die
   Bandbreite für die anderen 3 Klassen jeweils die selbe beträgt,
   bekommt jeder Rechner 1/3 der gesamten Bandbreite, effektiv also
   256Kibit/s.

   Mit dieser Konfiguration haben wir allerdings erst das Grundgerüst
   erstellt. Jetzt müssen wir noch sagen welcher Verkehr durch welche
   Klasse geregelt werden soll.

   Dazu benutzen wir Filter, welche den Verkehr den einzelnen Klassen
   zuordnen.
   Wir erstellen also 3 Filter für die 3 Rechner (siehe
   \var{QOS\_\-FILTER\_\-N} und folgende) und ordnen jeden Filter einer Klasse zu
   (siehe \var{QOS\_\-FILTER\_\-x\_\-CLASS}). Jetzt können wir mit
   \var{QOS\_\-FILTER\_\-x\_\-IP\_\-INTERN}, \var{QOS\_\-FILTER\_\-x\_\-IP\_\-INTERN},
   \var{QOS\_\-FILTER\_\-x\_\-PORT}, \var{QOS\_\-FILTER\_\-x\_\-PORT} und
   \var{QOS\_\-FILTER\_\-x\_\-OPTION} bestimmen was durch die jeweilige Klasse zu
   der der Filter gehört geregelt werden soll.

   Nennen wir das Interface 0 und die 3 Klassen 1, 2 und 3 und die 3
   Filter F1, F2 und F3 ergibt sich das in Abbildung \ref{fig:qosbsp1}
   dargestellte Szenario.

   \begin{figure}[htbp]
     \centering
     \includegraphics{qos_bsp_1}
     \caption{Beispiel 1}
     \marklabel{fig:qosbsp1}{}
   \end{figure}


   Die Konfiguration sieht dann so aus:

   Drei Rechner nach IP gefiltert die je 1/3 bekommen falls kein Gast
   anwesend ist:

\begin{small}
\begin{example}
\begin{verbatim}
OPT_QOS='yes'
QOS_INTERNET_DEV_N='1'
QOS_INTERNET_DEV_1='ppp0'
QOS_INTERNET_BAND_DOWN='768Kibit/s'
QOS_INTERNET_BAND_UP='128Kibit/s'
QOS_INTERNET_DEFAULT_DOWN='4'
QOS_INTERNET_DEFAULT_UP='0'

QOS_CLASS_N='4'

QOS_CLASS_1_PARENT='0'
QOS_CLASS_1_MINBANDWIDTH='232Kibit/s'
QOS_CLASS_1_MAXBANDWIDTH='768Kibit/s'
QOS_CLASS_1_DIRECTION='down'
QOS_CLASS_1_PRIO=''

QOS_CLASS_2_PARENT='0'
QOS_CLASS_2_MINBANDWIDTH='232Kibit/s'
QOS_CLASS_2_MAXBANDWIDTH='768Kibit/s'
QOS_CLASS_2_DIRECTION='down'
QOS_CLASS_2_PRIO=''

QOS_CLASS_3_PARENT='0'
QOS_CLASS_3_MINBANDWIDTH='232Kibit/s'
QOS_CLASS_3_MAXBANDWIDTH='768Kibit/s'
QOS_CLASS_3_DIRECTION='down'
QOS_CLASS_3_PRIO=''

QOS_CLASS_4_PARENT='0'
QOS_CLASS_4_MINBANDWIDTH='72Kibit/s'
QOS_CLASS_4_MAXBANDWIDTH='768Kibit/s'
QOS_CLASS_4_DIRECTION='down'
QOS_CLASS_4_PRIO=''


QOS_FILTER_N='3'

QOS_FILTER_1_CLASS='1'
QOS_FILTER_1_IP_INTERN='192.168.0.2'
QOS_FILTER_1_IP_EXTERN=''
QOS_FILTER_1_PORT=''
QOS_FILTER_1_PORT_TYPE=''
QOS_FILTER_1_OPTION=''

QOS_FILTER_2_CLASS='2'
QOS_FILTER_2_IP_INTERN='192.168.0.3'
QOS_FILTER_2_IP_EXTERN=''
QOS_FILTER_2_PORT=''
QOS_FILTER_2_PORT_TYPE=''
QOS_FILTER_2_OPTION=''

QOS_FILTER_3_CLASS='3'
QOS_FILTER_3_IP_INTERN='192.168.0.4'
QOS_FILTER_3_IP_EXTERN=''
QOS_FILTER_3_PORT=''
QOS_FILTER_3_PORT_TYPE=''
QOS_FILTER_3_OPTION=''
\end{verbatim}
\end{example}
\end{small}

   Die Option \var{QOS\_\-INTERNET\_\-DEFAULT\_\-UP} wurde auf 0 gesetzt da der Upstream
   nicht beschränkt werden soll.




\subsubsection{Beispiel 2}



   Ein Beispiel mit dem Ziel die Bandbreite auf 2 Rechner zu verteilen
   und die jeweiligen Bandbreiten pro Rechner wiederum noch ein zweites
   mal aufzuteilen auf einen Port und den restlichen Verkehr des
   jeweiligen Rechners.

   Dazu erstellen wir erst einmal wieder 2 Klassen mit den jeweiligen
   Gesamtgeschwindigkeit und hängen sie direkt an das Interface für
   ``up'' bzw. ``down'' (siehe erstes Beispiel). Jetzt erstellen wir für
   den ersten Rechner an der ersten Klasse zwei weitere Klassen. Die
   Klassen werden genau so erstellt wie die beiden ersten Klassen direkt
   am Interface, allerdings mit einer Besonderheit:
   \var{QOS\_\-CLASS\_\-x\_\-PARENT} ist jetzt nicht 0, sondern die Nummer der
   Klasse an die die Klassen angehängt werden sollen. Ist dies z. B.
   \var{QOS\_\-CLASS\_\-1}, so muss man jetzt \var{QOS\_\-CLASS\_\-1} von der Klasse
   die angehängt werden soll auf 1 setzen. Das gleiche wird für den
   zweiten Rechner auch gemacht. Man hängt wieder zwei Klassen an die
   Klasse für den zweiten Rechner. Dies kann man nun nicht nur für zwei
   Rechner machen, sondern für so viele wie man möchte. Auch kann man
   so viele Unterklassen an einer Klasse erstellen wie man möchte.

   Hiermit haben wir wieder das Grundgerüst erstellt und müssen nun mit
   den Filtern den Verkehr den einzelnen Klassen zuordnen. (siehe
   erstes Beispiel)

   Wir erstellen also 2 Filter für den ersten Rechner und 2 Filter für
   den zweiten Rechner. Jeweils einen Filter für den Port und einen
   Filter für den restlichen Verkehr vom Rechner. Hierbei ist unbedingt
   auf die Reihenfolge zu achten. Als erstes jeweils nur den Port und
   danach den Rest. Anders herum würde ja schon der Filter für den Rest
   alles einer Klasse zuordnen.

   Nennen wir das Interface 0 und die 6 Klassen 1, 2, 3, 4, 5 und 6
   und die 4 Filter F1, F2, F3 und F4 ergibt sich das in Abbildung
   \ref{fig:qosbsp1} dargestellte Szenario.

   \begin{figure}[htbp]
     \centering
     \includegraphics{qos_bsp_2}%
     \caption{Beispiel 2}
     \marklabel{fig:qosbsp2}{}
   \end{figure}

   Die Konfiguration sieht dann so aus:

   Zwei Klassen für 2 Rechner die je 1/2 bekommen, und zwar vom
   Interface, mit jeweils 2 Klassen für einen Port der 2/3 bekommt und
   den Rest der 1/3 bekommt, und zwar jeweils von der Vaterklasse:

\begin{small}
\begin{example}
\begin{verbatim}
OPT_QOS='yes'
QOS_INTERNET_DEV_N='1'
QOS_INTERNET_DEV_1='ppp0'
QOS_INTERNET_BAND_DOWN='768Kibit/s'
QOS_INTERNET_BAND_UP='128Kibit/s'
QOS_INTERNET_DEFAULT_DOWN='7'
QOS_INTERNET_DEFAULT_UP='0'

QOS_CLASS_N='6'

QOS_CLASS_1_PARENT='0'
QOS_CLASS_1_MINBANDWIDTH='384Kibit/s'
QOS_CLASS_1_MAXBANDWIDTH='768Kibit/s'
QOS_CLASS_1_DIRECTION='down'
QOS_CLASS_1_PRIO=''

QOS_CLASS_2_PARENT='0'
QOS_CLASS_2_MINBANDWIDTH='384Kibit/s'
QOS_CLASS_2_MAXBANDWIDTH='768Kibit/s'
QOS_CLASS_2_DIRECTION='down'
QOS_CLASS_2_PRIO=''

QOS_CLASS_3_PARENT='1'
QOS_CLASS_3_MINBANDWIDTH='256Kibit/s'
QOS_CLASS_3_MAXBANDWIDTH='768Kibit/s'
QOS_CLASS_3_DIRECTION='down'
QOS_CLASS_3_PRIO=''

QOS_CLASS_4_PARENT='1'
QOS_CLASS_4_MINBANDWIDTH='128Kibit/s'
QOS_CLASS_4_MAXBANDWIDTH='768Kibit/s'
QOS_CLASS_4_DIRECTION='down'
QOS_CLASS_4_PRIO=''

QOS_CLASS_5_PARENT='2'
QOS_CLASS_5_MINBANDWIDTH='256Kibit/s'
QOS_CLASS_5_MAXBANDWIDTH='768Kibit/s'
QOS_CLASS_5_DIRECTION='down'
QOS_CLASS_5_PRIO=''

QOS_CLASS_6_PARENT='2'
QOS_CLASS_6_MINBANDWIDTH='128Kibit/s'
QOS_CLASS_6_MAXBANDWIDTH='768Kibit/s'
QOS_CLASS_6_DIRECTION='down'
QOS_CLASS_6_PRIO=''

QOS_FILTER_N='4'

QOS_FILTER_1_CLASS='3'
QOS_FILTER_1_IP_INTERN='192.168.0.2'
QOS_FILTER_1_IP_EXTERN=''
QOS_FILTER_1_PORT='80'
QOS_FILTER_1_PORT_TYPE='client'
QOS_FILTER_1_OPTION=''

QOS_FILTER_2_CLASS='4'
QOS_FILTER_2_IP_INTERN='192.168.0.2'
QOS_FILTER_2_IP_EXTERN=''
QOS_FILTER_2_PORT=''
QOS_FILTER_2_PORT_TYPE=''
QOS_FILTER_2_OPTION=''

QOS_FILTER_3_CLASS='5'
QOS_FILTER_3_IP_INTERN='192.168.0.3'
QOS_FILTER_3_IP_EXTERN=''
QOS_FILTER_3_PORT='80'
QOS_FILTER_3_PORT_TYPE='client'
QOS_FILTER_3_OPTION=''

QOS_FILTER_4_CLASS='6'
QOS_FILTER_4_IP_INTERN='192.168.0.3'
QOS_FILTER_4_IP_EXTERN=''
QOS_FILTER_4_PORT=''
QOS_FILTER_4_PORT_TYPE=''
QOS_FILTER_4_OPTION=''

\end{verbatim}
\end{example}
\end{small}

   Bei diesem Beispiel wurde die Option \var{QOS\_\-INTERNET\_\-DEFAULT\_\-DOWN} so
   gewählt, dass der Verkehr, welcher nicht durch einen Filter einer
   Klasse zugeordnet wird, in eine nicht existierende Klasse gesteckt
   wird. Einfach aus dem Grund um das Beispiel zu vereinfachen und weil
   in dem Beispiel davon ausgegangen wird dass es keinen Rest gibt.
   Verkehr der in eine nicht existierende Klasse geleitet wird, wird nur
   sehr langsam weiter geleitet. Wenn es einen Rest gibt, ist also
   unbedingt darauf zu achten, dass dieser in eine eigene Klasse gesteckt
   wird, die auch existiert.

   Die Option \var{QOS\_\-INTERNET\_\-DEFAULT\_\-UP} wurde auf 0 gesetzt da der Upstream
   nicht beschränkt werden soll.



\subsubsection{Beispiel 3}



   Ein Beispiel, welches die allgemeine Funktionsweise von QoS versucht
   nahezubringen.

   \begin{figure}[htbp]
     \centering
     \includegraphics{qos_bsp_3}
     \caption{Beispiel 3}
     \marklabel{fig:qosbsp3}{}
   \end{figure}


   In Abbildung \ref{fig:qosbsp3} ist noch einmal die Aufteilung aus
   dem zweiten Beispiel zu sehen, allerdings mit einer Erweiterung. An
   die Beiden Unterklassen der zweiten Klasse sind jeweils noch zwei
   weitere Unterklassen Angehängt. Es ist also möglich noch tiefer
   zu verschachteln. Es ist möglich, tiefer zu verschachteln als auf
   diesem Bild, die momentane Grenze liegt hier bei 8 Stufen, also man
   darf maximal nur 7 Weitere Stufen nach dem Interface erstellen,
   danach ist schluss. In die ``Breite'' ist allerdings keine Grenze
   gesetzt. Man kann also an eine Unterklasse innerhalb einer Stufe so
   viele Klassen anhängen wie man möchte.

   Auf diesem Bild ist ausserdem noch zu erkennen, dass es auch möglich
   ist mehr als einen Filter an eine Klasse zu hängen, so wie es an der
   Klasse 10 geschehen ist. Aber auch bei den Filtern bleibt zu beachten
   dass es Momentan nicht möglich ist einen Filter mitten in den ``Baum''
   zu heangen, so wie es mit F8 geschehen sollte.

   Schauen wir uns nun als letztes noch den Sinn von Klassen und
   Unterklassen an. Klassen dienen dazu die Geschwindigkeit einer
   Verbindung einzustellen und zu regeln. Die Verteilung der
   Geschwindigkeit erfolgt wie bei \var{QOS\_\-CLASS\_\-x\_\-MINBANDWIDTH}
   beschrieben. Allerdings kann dies einen Nachteil haben wenn man z.b.
   alle Klassen an eine Klasse hängt. Möchte man z.b. einem Rechner die
   hälfte der bandbreite geben und dem zweiten ebenfalls die hälfte
   allerdings aufgeteilt auf 2/3 http und 1/3 Rest also jeweils 2/6 und
   1/6 vom ganzen. So geschieht nun folgendes: bei Vollast bekommt jeder
   seine hälfte. Ueberträgt der zweite jedoch nichts über http so
   bleibt ja 2/6 über. Diese 2/6 bekommt jedoch nun nicht der 2. Rechner
   alleine, sondern es wird nach dem beschriebenen Verfahren aufgeteilt.
   Um dieses zu verhindern erstellt man Unterklassen. Der Verkehr einer
   Klasse wird somit erst an die Unklassen verteilt, erst wenn diese
   nicht den Kompletten Verkehr beanspruchen wird der Rest an andere
   Klassen Verteilt. In dem Bild sind jeweils die Bereiche eingekreist
   welche zusammengehoren. Rot = 1, Blau = 2, Grün = 5 und Orange = 6.



\subsubsection{Beispiel 4}


   Beispielkonfiguration für das Priorisieren von ACK-Paketen, damit der
   Downstream bei gleichzeitig starkem Upstream nicht einbricht:

\begin{small}
\begin{example}
\begin{verbatim}
OPT_QOS='yes'
QOS_INTERNET_DEV_N='1'
QOS_INTERNET_DEV_1='ppp0'
QOS_INTERNET_BAND_DOWN='768Kibit/s'
QOS_INTERNET_BAND_UP='128Kibit/s'
QOS_INTERNET_DEFAULT_DOWN='0'
QOS_INTERNET_DEFAULT_UP='2'

\end{verbatim}
\end{example}
\end{small}

   Hier konfigurieren wir ppp0 als Internetdevice (DSL) und geben die für
   TDSL (und einiger anderer Provider) übliche Up/Downstreambandbreiten
   an. Eventuell ist es nötig, dass wir die Upstreambandbreite noch um das
   eine oder andere Kibibit herabsetzen, das muß man ausprobieren.

   Da wir keine Klassen für den Downstream einrichten wollen, setzen wir
\begin{example}
\begin{verbatim}
        QOS_INTERNET_DEFAULT_DOWN='0'
\end{verbatim}
\end{example}

   Für den Upstream soll die Klasse mit der Nummer 2 die
   Standardklasse sein. Das Netzwerkdevice ist eth0 und auf 10Mibit/s
   eingestellt.

\begin{small}
\begin{example}
\begin{verbatim}
   QOS_CLASS_N='2'

QOS_CLASS_1_PARENT='0'
QOS_CLASS_1_MINBANDWIDTH='127Kibit/s'
QOS_CLASS_1_MAXBANDWIDTH='128Kibit/s'
QOS_CLASS_1_DIRECTION='up'
QOS_CLASS_1_PRIO=''

\end{verbatim}
\end{example}
\end{small}
   Dies ist die Klasse in die wir unsere ACK-(Bestätigungs-)Pakete
   hineinstecken wollen. Die ACK-Pakete sind recht klein und benötigen
   deswegen nur recht wenig Bandbreite. Trotzdem wollen wir sie
   eigentlich in keinster Weise einschränken und teilen ihr 127Kibit/s zu.
   1Kibit/s lassen wir übrig für den Rest.

\begin{small}
\begin{example}
\begin{verbatim}
QOS_CLASS_2_PARENT='0'
QOS_CLASS_2_MINBANDWIDTH='1Kibit/s'
QOS_CLASS_2_MAXBANDWIDTH='128Kibit/s'
QOS_CLASS_2_DIRECTION='up'
QOS_CLASS_2_PRIO=''
\end{verbatim}
\end{example}
\end{small}

   In diese Klasse soll dann der Rest (alles außer ACK-Pakete)
   hineingesteckt werden. Die Bandbreite, die wir dieser Klasse zuteilen
   sind die verbleibenden 1Kibit/s (128-127=1). Wir begrenzen sie aber auch
   nicht auf 1Kibit/s, begrenzt wird die Klasse durch den Eintrag
\begin{example}
\begin{verbatim}
        QOS_CLASS_2_MAXBANDWIDTH='128Kibit/s'
\end{verbatim}
\end{example}

   Da unsere erste Klasse die zugeteilte Bandbreite wohl kaum
   ausnutzen wird, bleibt also immer etwas übrig und das was übrig
   bleibt schnappt sich dann die zweite.  Wenn man den Upstream noch
   weiter aufteilen möchte (was meist der Fall ist), so sind alle
   weiteren Klassen unter diese Klasse zu ``hängen''.  Dabei muß
   natürlich auch \var{QOS\_\-INTERNET\_\-DEFAULT\_\-UP} entsprechend angepaßt
   werden.

\begin{small}
\begin{example}
\begin{verbatim}
QOS_FILTER_N='1'

QOS_FILTER_1_CLASS='1'
QOS_FILTER_1_IP_INTERN=''
QOS_FILTER_1_IP_EXTERN=''
QOS_FILTER_1_PORT=''
QOS_FILTER_1_PORT_TYPE=''
QOS_FILTER_1_OPTION='ACK'
\end{verbatim}
\end{example}
\end{small}

   Dieser Filter filtert alle Pakete, die auf die Option ACK zutreffen,
   also ACK-Pakete. Durch den Eintrag \var{QOS\_\-FILTER\_\-1\_\-CLASS}='1' erreichen
   wir, dass diese gefilterten Pakete in die 1. Klasse gesteckt werden.

   Zum Testen muß sucht man sich am besten eine gute oder mehrere Up- und
   Downloadquellen aus, von denen man weiß, dass sie sowohl den Up- als
   auch den Downstream voll auslasten können und läßt die Kabel glühen.
   Dabei sollte man einen Blick auf die Trafficanzeige des ImonC werfen.
   Am besten führt man das ganze auch mal ohne QoS durch.

   Der Downstream sollte gar nicht oder wesentlich weniger stark
   einbrechen als ohne diese Konfiguration. Wie schon gesagt kann man die
   Lage noch verbessern, in dem man die Upstreambandbreite in
   Kibibit-Schritten herabsetzt und dann die Auswirkungen beobachtet.
   Bei mir wurde zum Beispiel das Optimum bei 121Kibit/s erreicht (kein
   Einbruch des Downstreams mehr). Dabei sind natürlich auch die
   MAXBANDWIDTH- und MINBANDWIDTH-Werte der Klassen entsprechend
   anzupassen.

