% Synchronized to r32675
\section{QoS - Quality of Service}
\marklabel{sec:qos}{}

   By QoS the available bandwidth can be regulated and for
   example be distributed to several ports, IP addresses and more.

   A modem manages a packet queue where packets are stored that exceed the
   available bandwidth. With DSL modems for example these queues are
   rather big. The advantage is a constant usage of maximum bandwidth.
   If the router sends lesser packets for a short period of time the
   modem has packets in the queue that it can send. Such a queue is a
   simple thing sending packets first in first out - rather fair, isn't it?

   This is where QoS comes into play. QoS also manages a packet queue
   in the router itself which makes it possible to decide which packets
   are to be sent at first and which to hold back. If everything is
   configured the right way QoS sends packets at the speed the modem sends
   them out not filling the modem queue at any time. This is like moving
   the modem queue into the router.

   \marklabel{hint:speedunits}{}Something general on speed units:
   QoS supports Mibit/s (mebibit/s) and Kibit/s (kibibit/s), where applies 1Mibit
   = 1024Kibit.

\subsection{Configuration}

\begin{description}

\config{OPT\_QOS}{OPT\_QOS}{OPTQOS}
   Set this to 'yes' to enable and 'no' to disable \var{OPT\_\-QOS}.

\config{QOS\_INTERNET\_DEV\_N}{QOS\_INTERNET\_DEV\_N}{QOSINTERNETDEVN}

Number of devices routing data to the Internet.

\config{QOS\_INTERNET\_DEV\_x}{QOS\_INTERNET\_DEV\_x}{QOSINTERNETDEVx}

   List of devices that route data to the Internet. Examples:

   \begin{tabular}[h!]{ll}
     \var{QOS\_\-INTERNET\_\-DEV\_N}='3'    & device number \\
     \var{QOS\_\-INTERNET\_\-DEV\_1}='ethX' & for cable and other ethernet connections \\
     \var{QOS\_\-INTERNET\_\-DEV\_2}='ppp0' & for DSL over PPPoE \\

     \var{QOS\_\-INTERNET\_\-DEV\_3}='ipppX' & for ISDN \\
   \end{tabular}

   The ISDN device for the first circuit should be named ippp0 the second
   ippp1.  If channel bundeling is activated for the first circuit
   the second channel of the first circuit is named ippp1 and the
   second circuit ippp2.  QOS should only be used with ISDN if channel
   bundeling is deactivated for the circuit used.


\config{QOS\_INTERNET\_BAND\_DOWN}{QOS\_INTERNET\_BAND\_DOWN}{QOSINTERNETBANDDOWN}

   Maximum downstream bandwidth of the Internet connection. See above: \jump{hint:speedunits}
   {Something general on speed units}.

   Hint: For time-critical jobs like preferring ACK packets it is necessary
   to limit the bandwidth to the actual value. Else packets will be sorted
   correctly in the packet queue but are withheld in the modem's packet queue.
   It may be possible that bandwidth declared by your provider is not in accordance
   with real conditions. It could be a little less or a little more. At the
   end only trying helps here.


\config{QOS\_INTERNET\_BAND\_UP}{QOS\_INTERNET\_BAND\_UP}{QOSINTERNETBANDUP}

   Maximum upstream bandwidth of the Internet connection.
   See above: \jump{hint:speedunits}{Something general on speed units}.

   Also see hint at \var{QOS\_\-INTERNET\_\-BAND\_\-DOWN}.


\config{QOS\_INTERNET\_DEFAULT\_DOWN}{QOS\_INTERNET\_DEFAULT\_DOWN}{QOSINTERNETDEFAULTDOWN}

   Set the default class for packets coming from the Internet here. All packets
   which are not classified by a filter will end here.

   If no class is specified for which variable
\begin{example}
\begin{verbatim}
        QOS_CLASS_x_DIRECTION='down'
\end{verbatim}
\end{example}
   is set specify:
\begin{example}
\begin{verbatim}
        QOS_INTERNET_DEFAULT_DOWN='0'
\end{verbatim}
\end{example}

   Example:

   Two classes have been created and a filter puts all packets for a certain
   IP address into the first one. All other packets should go to the second one.
   This must be set like this:

\begin{example}
\begin{verbatim}
        QOS_INTERNET_DEFAULT_DOWN='2'
\end{verbatim}
\end{example}

   Pay attention to set a class for \var{QOS\_\-INTERNET\_\-DEFAULT\_\-DOWN} where
   its \var{QOS\_\-CLASS\_\-x\_\-DIRECTION} variable contains the argument 'down'.


\config{QOS\_INTERNET\_DEFAULT\_UP}{QOS\_INTERNET\_DEFAULT\_UP}{QOSINTERNETDEFAULTUP}

   Set the default class for packets going out to the Internet here. All packets
   which are not classified by a filter will end here.

   If no class is specified for which variable
\begin{example}
\begin{verbatim}
        QOS_CLASS_x_DIRECTION='up'
\end{verbatim}
\end{example}

is set specify:
\begin{example}
\begin{verbatim}
        QOS_INTERNET_DEFAULT_UP='0'
\end{verbatim}
\end{example}

   This works in analog to \var{QOS\_\-INTERNET\_\-DEFAULT\_\-DOWN}.

   Pay attention to set a class for \var{QOS\_\-INTERNET\_\-DEFAULT\_\-UP} where
   its \var{QOS\_\-CLASS\_\-x\_\-DIRECTION} variable contains the argument 'up'.

\config{QOS\_CLASS\_N}{QOS\_CLASS\_N}{QOSCLASSN}

   Set the number of classes to be created.


\config{QOS\_CLASS\_x\_PARENT}{QOS\_CLASS\_x\_PARENT}{QOSCLASSxPARENT}

   By this variable classes can be stacked. Set the number of the parent
   class here. Bandwidth allocated to the parent class can be spread
   between the subclasses. Maximum subclass layer depth is 8 whereas the
   interface itself is the first layer which leaves a maximum of 7 layers
   to be configured.

   If the class is no subclass write the following:

\begin{example}
\begin{verbatim}
        QOS_CLASS_x_PARENT='0'
\end{verbatim}
\end{example}

   It will get the maximum bandwidth set in \var{QOS\_\-CLASS\_\-x\_\-PORT\_\-TYPE}
   depending on its direction (in- or outbound, see \var{QOS\_\-CLASS\_\-x\_\-PORT\_\-TYPE})

   Important: If this is not '0' pay attention to define the parent class before
   (in numbering)

\config{QOS\_CLASS\_x\_MINBANDWIDTH}{QOS\_CLASS\_x\_MINBANDWIDTH}{QOSCLASSxMINBANDWIDTH}

   Bandwidth to allocate to the classes.
   See above: \jump{hint:speedunits}{Something general on speed units}.

   Example:
   A class with a bandwidth limited to 128Kibit/s:

\begin{example}
\begin{verbatim}
        QOS_CLASS_1_MINBANDWIDTH='128Kibit/s'
        QOS_CLASS_1_MAXBANDWIDTH='128Kibit/s'
        QOS_CLASS_1_PARENT='0'
\end{verbatim}
\end{example}


   Three subclasses of our parent class above where \var{QOS\_\-CLASS\_\-x\_\-MINBANDWIDTH}-
   and \var{QOS\_\-CLASS\_\-x\_\-MAXBANDWIDTH} settings look like this:

\begin{example}
\begin{verbatim}
        QOS_CLASS_2_PARENT='1'
        QOS_CLASS_2_MINBANDWIDTH='60Kibit/s'
        QOS_CLASS_2_MAXBANDWIDTH='128Kibit/s'

        QOS_CLASS_3_PARENT='1'
        QOS_CLASS_3_MINBANDWIDTH='40Kibit/s'
        QOS_CLASS_3_MAXBANDWIDTH='128Kibit/s'

        QOS_CLASS_4_PARENT='1'
        QOS_CLASS_4_MINBANDWIDTH='28Kibit/s'
        QOS_CLASS_4_MAXBANDWIDTH='128Kibit/s'
\end{verbatim}
\end{example}


   All subclasses have the same (or no) priority (see \var{QOS\_\-CLASS\_\-x\_\-PRIO}).
   If traffic on all subclasses exceeds their \var{QOS\_\-CLASS\_\-x\_\-MINBANDWIDTH}
   they all get \var{QOS\_\-CLASS\_\-x\_\-MINBANDWIDTH} allocated.
   If class 2 has only 20Kibit/s traffic there are 40Kibit/s ``left''. This overrun
   will be split in 40/28 ratio between class 3 and 4.
   Each class is limited to 128Kibit/s by \var{QOS\_\-CLASS\_\-x\_\-MAXBANDWIDTH}
   and because all classes are subclasses of a parent class limited to 128Kibit/s
   the maximum bandwidth consumed is 128Kibit/s.


\config{QOS\_CLASS\_x\_MAXBANDWIDTH}{QOS\_CLASS\_x\_MAXBANDWIDTH}{QOSCLASSxMAXBANDWIDTH}

   Maximum bandwidth to be allocated to the class. There is no sense in entering
   a lower value than in \var{QOS\_\-CLASS\_\-x\_\-MINBANDWIDTH}. If nothing
   is set here this variable automatically will be set to the value of
   \var{QOS\_\-CLASS\_\-x\_\-MINBANDWIDTH}. Such a class can't use any free
   bandwidth resources.

   See above: \jump{hint:speedunits}{Something general on speed units}.


\config{QOS\_CLASS\_x\_DIRECTION}{QOS\_CLASS\_x\_DIRECTION}{QOSCLASSxDIRECTION}

   This variable sets the direction the class belongs to. If upstream regulation
   is intended set:

\begin{example}
\begin{verbatim}
        QOS_CLASS_x_DIRECTION='up'
\end{verbatim}
\end{example}

   for downstream in analog:

\begin{example}
\begin{verbatim}
        QOS_CLASS_x_DIRECTION='down'
\end{verbatim}
\end{example}


\config{QOS\_CLASS\_x\_PRIO}{QOS\_CLASS\_x\_PRIO}{QOSCLASSxPRIO}

   Set the priority of the class here. The lower the number the higher the priority.
   Values between 0 and 7 are allowed. Leaving this empty equals to setting '0'.

   Priorities are used to determine which class can use free bandwidth resources.
   Lets change our example in \var{QOS\_\-CLASS\_\-x\_\-MINIMUMBANDWIDTH}
   to reflect this:
   Nothing was changed for the first class. Classes 2 to 4 get a priority:

\begin{example}
\begin{verbatim}
        QOS_CLASS_2_PARENT='1'
        QOS_CLASS_2_MINBANDWIDTH='60Kibit/s'
        QOS_CLASS_2_MAXBANDWIDTH='128Kibit/s'
        QOS_CLASS_2_PRIO='1'

        QOS_CLASS_3_MINBANDWIDTH='40Kibit/s'
        QOS_CLASS_3_PARENT='1'
        QOS_CLASS_3_MAXBANDWIDTH='128Kibit/s'
        QOS_CLASS_3_PRIO='1'

        QOS_CLASS_4_PARENT='1'
        QOS_CLASS_4_MINBANDWIDTH='28Kibit/s'
        QOS_CLASS_4_MAXBANDWIDTH='128Kibit/s'
        QOS_CLASS_4_PRIO='2'
\end{verbatim}
\end{example}


   Like in the original example class 2 consumes only 20Kibit/s and leaves
   free bandwidth of 40Kibit/s. Classes 3 and 4 both need more bandwidth
   than available. Class 3 now has a higher priority than class 4 and
   may use the free bandwidth of 40Kibit/s.

   If class 3 needs only 20Kibit/s of the free bandwidth of 40Kibit/s
   class 4 will get the remaining 20Kibit/s.

   Lets assume something different: Class 4 needs no bandwidth at all and class
   2 and 3 both need more than exists. So they both get the bandwidth set in
   \var{QOS\_\-CLASS\_\-x\_\-MINBANDWIDTH} and the remaining will be divided in
   60/40 ratio between them because both classes have the same priority.

   As you see \var{QOS\_\-CLASS\_\-x\_\-PRIO} only has influence on how an
   enventual overrun in bandwidth is spread.

\config{QOS\_CLASS\_x\_LABEL}{QOS\_CLASS\_x\_LABEL}{QOSCLASSxLABEL}

   By using this optional variable a label for the class may be specified
   which will be used as the caption for the QOS-graphs created by an activated
   OPT\_RRDTOOL.

\config{QOS\_FILTER\_N}{QOS\_FILTER\_N}{QOSFILTERN}

   Set the number of filters desired here.

   A quick note on filters:
   Arguments of the different variables are AND-linked, arguments of the
   same variable are OR-linked. This means: If the same filter filters by
   an IP-address and a port only packets will get selected and put in
   the queue which match both conditions at a time.

   Another example:
   Two ports (21 and 80) and an IP address are set in the same filter.
   Since a packet can't use two ports at a time the filter will match
   those packets that either use port 21 and the named IP address or
   port 80 and the named IP address.

   Important: The ordering of filters is essential!

   An example: All traffic on port 456 for \textbf{all} clients should be
   queued to class A. In addition all packets to a client with IP address
   192.168.6.5 should be queued to class B, except those on port 456.
   If the filter on the IP is created first all packets (those on port
   456 too) will be queued to class B and the filter on port 456 doesn't
   change this behavior. The filter on port 456 has to be created before
   the one on the IP address 192.168.6.5 to achieve our goals.

\config{QOS\_FILTER\_x\_CLASS}{QOS\_FILTER\_x\_CLASS}{QOSFILTERxCLASS}

   This variable specifies the class a packet is queued in that matches
   the given filter. If for example packets should be put in the queue
   specified by \var{QOS\_\-CLASS\_\-25\_\-MINBANDWIDTH} this should be
   done like this:

\begin{example}
\begin{verbatim}
        QOS_FILTER_x_CLASS='25'
\end{verbatim}
\end{example}

   By \var{QOS\_\-CLASS\_\-x\_\-DIRECTION} it is set if a class belongs to
   up- or downstream. If a filter is set then queueing packets to an
   upstream class only upstream packets will be filtered and queued
   to the class mentioned. \var{QOS\_\-CLASS\_\-x\_\-DIRECTION} defines
   the ``direction '' of filtering.

   As of version 2.1 more than one target class can be set. If for example
   traffic on port 456 upstream and downstream is our target

\begin{example}
\begin{verbatim}
        QOS_FILTER_x_CLASS='4 25'
\end{verbatim}
\end{example}

   would be specified, if class 4 is the upstream class and 25 is the downstream
   one. It does not make sense to specifiy more than one up- or downstream class
   so there never will be more than two target classes.

\config{QOS\_FILTER\_x\_IP\_INTERN}{QOS\_FILTER\_x\_IP\_INTERN}{QOSFILTERxIPINTERN}

   Set IP addresses and IP ranges from internal networks here which should be
   filtered. They have to be separated by spaces and can be combined in any
   manner.

   An example:

\begin{example}
\begin{verbatim}
        QOS_FILTER_x_IP_INTERN='192.168.6.0/24 192.168.5.7 192.168.5.12'
\end{verbatim}
\end{example}

   This filters all IP addresses that match 192.168.6.X and in addition
   IP 192.168.5.7 and 192.168.5.12.

   This variable can be empty.

   If this variable is used in conjunction with \var{QOS\_\-FILTER\_\-x\_\-IP\_\-EXTERN}
   only traffic between IPs or IP ranges defined by \var{QOS\_\-FILTER\_\-x\_\-IP\_\-INTERN}
   and \var{QOS\_\-FILTER\_\-x\_\-IP\_\-EXTERN} will be filtered.

   \sloppypar\wichtig{If filtering by \var{QOS\_\-FILTER\_\-x\_\-OPTION} for ACK,
   TOSMD, TOSMT, TOSMR or TOSMC takes place and variable
   \var{QOS\_\-CLASS\_\-x\_\-DIRECTION} is of target class 'down'
   this variable will be ignored.}


\config{QOS\_FILTER\_x\_IP\_EXTERN}{QOS\_FILTER\_x\_IP\_EXTERN}{QOSFILTERxIPEXTERN}

   Specify IP addresses and IP ranges from external networks (being connected on
   \var{QOS\_\-INTERNET\_\-DEV}) here which should be filtered. They have to be
   separated by spaces and can be combined in any manner. This works the same
   way as \var{QOS\_\-FILTER\_\-x\_\-IP\_\-INTERN}.

   This variable can be empty.

   \sloppypar\wichtig{If filtering by \var{QOS\_\-FILTER\_\-x\_\-OPTION} for ACK,
   TOSMD, TOSMT, TOSMR or TOSMC takes place and variable
   \var{QOS\_\-CLASS\_\-x\_\-DIRECTION} is of target class 'down'
   this variable will be ignored.}

\config{QOS\_FILTER\_x\_PORT}{QOS\_FILTER\_x\_PORT}{QOSFILTERxPORT}

   Ports and port ranges can be set here, separated by spaces and combined in
   any manner. If this variable is empty traffic on all ports will be limited.

   Filtering for a port range from 5000 up to 5099 would look like this:

\var{QOS\_\-FILTER\_\-x\_\-PORT}='5000-5099'

   Another example:
   If traffic on ports 20 to 21, 137 to 139 and port 80 should be filtered
   to the same class this would look like this:

\begin{example}
\begin{verbatim}
        QOS_FILTER_x_PORT='20-21 137-139 80'
\end{verbatim}
\end{example}

   This variable can be empty.

   Important:
   \begin{itemize}
   \item  If filtering for ports \var{QOS\_\-FILTER\_\-x\_\-PORT\_\-TYPE}
   has to be set as well.


   \item Port ranges will be ignored if filtering for ACK, TOSMD, TOSMT, TOSMR
   or TOSMC takes place by the use of \var{QOS\_\-FILTER\_\-x\_\-OPTION}.
   \end{itemize}


\config{QOS\_FILTER\_x\_PORT\_TYPE}{QOS\_FILTER\_x\_PORT\_TYPE}{QOSFILTERxPORTTYPE}

   This variable is only valid and important in conjunction with
   \var{QOS\_\-FILTER\_\-x\_\-PORT} (it is ignored in other cases).

   Ports in client mode are different from those in server mode. Specify
   here if a port is of type server or client. Use PCs from your own net
   as a point of reference to decide what to use.
   Possible settings:

\begin{example}
\begin{verbatim}
        QOS_FILTER_x_PORT_TYPE='client'
        QOS_FILTER_x_PORT_TYPE='server'
\end{verbatim}
\end{example}
   As of version 2.1 a combination of those two arguments is also valid to
   put traffic from the own net as well as traffic from the Internet into
   the same class on this port.

\begin{example}
\begin{verbatim}
        QOS_FILTER_x_PORT_TYPE='client server'
\end{verbatim}
\end{example}

   This equals to two similar filters once with \var{QOS\_\-FILTER\_\-x\_\-PORT\_\-TYPE}
   set to client and once set to server.


\config{QOS\_FILTER\_x\_OPTION}{QOS\_FILTER\_x\_OPTION}{QOSFILTERxOPTION}

   This variable activates additional properties for the filter. Only one of
   the following arguments can be passed (a combination wouldn't make sense
   in the same filter). It is perfectly right and makes sense sometimes to
   set a filter for ACK packets and a second filter for TOSMD packets to
   move their packets to the same target class (see \var{QOS\_\-FILTER\_\-x\_\-CLASS}).

   \begin{description}
   \item[ACK] Acknowledgement packets.

   A packet matching this option is sent as an acknowledgement to
   a data packet. If i.e. a huge download is running a lot of data packets
   will come in and for each of them an acknowledgement has to be sent to
   confirm it has reached you. If no acknowledgement packets reach the
   download source it will wait for them before sending the next chunk
   of data.

   This is extremely important with asymetric connections (up- and downstream
   bandwidths differ) like used in most DSL lines. Those most likely have a
   small upstream that tends to reach its maximum rather fast. If ACK
   packets are normally queued this may end in the data server delaying
   its transaction to wait for ACK packets to come in. This results in
   download rates lower than they could be.

   So ACK packets have to bypass the ``normal'' packets in order
   to get to the data server as fast as possible. How to combine
   this option in a meaningful way with a class is explained in
   the examples.

 \item[ICMP] Ping packets (Protocol ICMP)

   Ping packets are used to measure time a packet takes to get from A to B.
   To take influence on this value give ping packets a higher priority.
   This does not influence ping times for online games.

 \item[IGMP] IGMP-Pakete (Protokoll IGMP)

   If using IP-TV it makes sense to filter and priorize IGMP packets.

 \item[TCPSMALL] Small TCP Packets

   By using this filter outgoing HTTP(s)-Requests can be filtered and priorized.
   A combination with a destination port is possible and makes sense.
   Approx. size of this TCP packets is 800 Byte max.

 \item[TCP]    TCP packets (Protocol TCP)

   Only packets using protocol TCP are filtered.

 \item[UDP]  UDP packets (Protocol UDP)

   Only packets using protocol UDP are filtered.


 \item[TOS*] Type of Service

   An application can set one of four TOS bits for each packet transmitted.
   This specifies the intended handling for those packets. For example SSH
   can set TOS-Minimum-Delay for sending in- and output and
   TOS-Maximum-Troughput for sending files. Linux/Unix programs use these
   bits more often than Windows programs. A firewall can set TOS bits for
   certain packets as well. In the end it all depends on routers in the
   transport chain to honour TOS bits or not. Only TOS bits Minimum-Delay
   and Maximum-Throughput are really important for fli4l.

   \begin{description}
   \item [TOSMD - TOS Minimum-Delay]

   Used for services that need packets to be transferred without time delays.
   It is recommended to use this TOS bit for FTP (control data), Telnet and SSH.


 \item[TOSMT - TOS Maximum-Troughput]

   Used for services that need big amounts of data to be transferred with
   high speed. It is recommended to use this TOS bit for FTP data and WWW.


 \item[TOSMR - TOS Maximum-Reliability]

   Used to ensure that data reaches its target without being resent.
   Recommended use for this TOS bit is SNMP and DNS.


 \item[TOSMC - TOS Minimum-Cost]

   Used to minimize costs for transferring data.
   Recommended use for this TOS bit is NNTP and SMTP.

 \item[DSCP*] Differentiated Services Code Point

   DSCP is a marking according to RFC\ 2474.
   This process has replaced TOS marking mostly since 1998.

   Filters on DSCP-classes can be configured as follows:

\begin{example}
\begin{verbatim}
        QOS_FILTER_x_OPTION='DSCPef'
        QOS_FILTER_x_OPTION='DSCPcs3'
\end{verbatim}
\end{example}

   Please note that DSCP is in capital letters while the class is lower case.

   The following classes can be used:

   af11-af13, af21-af23, af31-af33, af41-af43, cs1-cs7, ef und be (Standard)

 \end{description}
\end{description}
\end{description}





\marklabel{sec:qosanwendung}{
  \subsection{Examples}
}



   How do we configure \var{OPT\_\-QoS} in detail? This will be
   shown below for some use cases:
   \begin{itemize}

   \item Example 1: targets at spreading bandwidth between 3 clients.

   \item Example 2: targets at spreading bandwidth between 2
     clients and on the clients between one port and the rest of the traffic
     on this client.

   \item Example 3: targets at learning the general structures of working with QoS.

   \item Example 4: a configuration for preferring ACK packets in order to keep
     downstream high even if upstream is overloaded.

   \end{itemize}



\subsubsection{Example 1}



   Spreading bandwidth between 3 clients.

   Four classes are created (see \var{QOS\_\-CLASS\_\-N}) with the following speeds
   (see \var{QOS\_\-CLASS\_\-x\_\-MINBANDWIDTH} / \var{QOS\_\-CLASS\_\-x\_\-MIN\-BAND\-WIDTH}).
   They are subclasses of class 0 (see \var{QOS\_\-CLASS\_\-x\_\-PARENT}) and therefore
   are bound directly to the interface for ``up'' res. ``down''
   (see \var{QOS\_\-CLASS\_\-x\_\-DIRECTION}).

   The fourth class is only for visitors and gets lesser bandwidth.
   By \var{QOS\_\-INTERNET\_\-DEFAULT\_\-DOWN}='4' all traffic not
   filtered is queued to the forth ``guest'' class. Because we seldom
   have visitors and the bandwidth is the same for the other 3 classes
   clients get 1/3 of the overall bandwidth (256Kibit/s each).

   This configuration is only a beginning. It has to be specified how
   traffic is regulated by the classes.

   We use two filters to assign traffic to the classes. We create 3 filters,
   one for each of the 3 clients (see \var{QOS\_\-FILTER\_\-N} a.s.o.) and
   attach a filter to each class (see \var{QOS\_\-FILTER\_\-x\_\-CLASS}).
   By specifying \var{QOS\_\-FILTER\_\-x\_\-IP\_\-INTERN},
   \var{QOS\_\-FILTER\_\-x\_\-IP\_\-INTERN}, \var{QOS\_\-FILTER\_\-x\_\-PORT},
   \var{QOS\_\-FILTER\_\-x\_\-PORT} and \var{QOS\_\-FILTER\_\-x\_\-OPTION}
   it can be defined what rules apply to each class.

   Let's call the interface 0, the 3 classes 1, 2 and 3 and the 3 filters F1,
   F2 and F3. The scenario looks like this image \ref{fig:qosbsp1}.

   \begin{figure}[htbp]
     \centering
     \includegraphics{qos_bsp_1}
     \caption{Example 1}
     \marklabel{fig:qosbsp1}{}
   \end{figure}


   Configuration looks like this:

   Three client PCs filtered by IP with 1/3 bandwidth each if
   visitors are absent:

\begin{small}
\begin{example}
\begin{verbatim}
OPT_QOS='yes'
QOS_INTERNET_DEV_N='1'
QOS_INTERNET_DEV_1='ppp0'
QOS_INTERNET_BAND_DOWN='768Kibit/s'
QOS_INTERNET_BAND_UP='128Kibit/s'
QOS_INTERNET_DEFAULT_DOWN='4'
QOS_INTERNET_DEFAULT_UP='0'

QOS_CLASS_N='4'

QOS_CLASS_1_PARENT='0'
QOS_CLASS_1_MINBANDWIDTH='232Kibit/s'
QOS_CLASS_1_MAXBANDWIDTH='768Kibit/s'
QOS_CLASS_1_DIRECTION='down'
QOS_CLASS_1_PRIO=''

QOS_CLASS_2_PARENT='0'
QOS_CLASS_2_MINBANDWIDTH='232Kibit/s'
QOS_CLASS_2_MAXBANDWIDTH='768Kibit/s'
QOS_CLASS_2_DIRECTION='down'
QOS_CLASS_2_PRIO=''

QOS_CLASS_3_PARENT='0'
QOS_CLASS_3_MINBANDWIDTH='232Kibit/s'
QOS_CLASS_3_MAXBANDWIDTH='768Kibit/s'
QOS_CLASS_3_DIRECTION='down'
QOS_CLASS_3_PRIO=''

QOS_CLASS_4_PARENT='0'
QOS_CLASS_4_MINBANDWIDTH='72Kibit/s'
QOS_CLASS_4_MAXBANDWIDTH='768Kibit/s'
QOS_CLASS_4_DIRECTION='down'
QOS_CLASS_4_PRIO=''


QOS_FILTER_N='3'

QOS_FILTER_1_CLASS='1'
QOS_FILTER_1_IP_INTERN='192.168.0.2'
QOS_FILTER_1_IP_EXTERN=''
QOS_FILTER_1_PORT=''
QOS_FILTER_1_PORT_TYPE=''
QOS_FILTER_1_OPTION=''

QOS_FILTER_2_CLASS='2'
QOS_FILTER_2_IP_INTERN='192.168.0.3'
QOS_FILTER_2_IP_EXTERN=''
QOS_FILTER_2_PORT=''
QOS_FILTER_2_PORT_TYPE=''
QOS_FILTER_2_OPTION=''

QOS_FILTER_3_CLASS='3'
QOS_FILTER_3_IP_INTERN='192.168.0.4'
QOS_FILTER_3_IP_EXTERN=''
QOS_FILTER_3_PORT=''
QOS_FILTER_3_PORT_TYPE=''
QOS_FILTER_3_OPTION=''
\end{verbatim}
\end{example}
\end{small}

   Option \var{QOS\_\-INTERNET\_\-DEFAULT\_\-UP} is set to 0 because upstream
   should not be regulated.




\subsubsection{Example 2}



   This example targets at spreading bandwidth between 2 client PCs and
   then those client bandwidths between one port and the remaining traffic
   on the client.

   We create 2 classes at first with the complete speed for each client
   and attach them directly to the interface for ``up'' res. ``down''
   (see example 1). No we create two additional classes for the first
   client attached to the first class. These classes are created in the
   same way like the first ones directly attached to the interface except
   for one difference: \var{QOS\_\-CLASS\_\-x\_\-PARENT} is not 0 but
   the number of the parent class it is attached to. If this for example
   is \var{QOS\_\-CLASS\_\-1} the classes' \var{QOS\_\-CLASS\_\-1} has to
   be set to 1. The same is applied for the second client PC. Attach two
   subclasses to the class for the second PC. This could be done for an
   infinite number of PCs if needed. Subclasses of a class can also be
   created in the amount needed.

   This is our skeleton. Now filters have to be defined to assign traffic
   to the classes (see example 1).

   2 filters have to be created for each client. One filter on a port and
   one for the remaining traffic of the client. Filter sequence is of
   essential importance here. First filter the port and then the rest.
   The other way round the filter for the remaining traffic would assign
   all traffic to its class (including the port traffic).

   Let's call the interface 0, the 6 classes 1, 2, 3, 4, 5, and 6 and the
   4 filters F1, F2, F3 and F4. The scenario looks like this image \ref{fig:qosbsp1}.

   \begin{figure}[htbp]
     \centering
     \includegraphics{qos_bsp_2}%
     \caption{Example 2}
     \marklabel{fig:qosbsp2}{}
   \end{figure}

   Configuration looks like this:

   2 classes for 2 PCs getting 1/2 interface bandwidth each with 2 classes for a port
   getting 2/3 and the rest getting 1/3 of its parent class:

\begin{small}
\begin{example}
\begin{verbatim}
OPT_QOS='yes'
QOS_INTERNET_DEV_N='1'
QOS_INTERNET_DEV_1='ppp0'
QOS_INTERNET_BAND_DOWN='768Kibit/s'
QOS_INTERNET_BAND_UP='128Kibit/s'
QOS_INTERNET_DEFAULT_DOWN='7'
QOS_INTERNET_DEFAULT_UP='0'

QOS_CLASS_N='6'

QOS_CLASS_1_PARENT='0'
QOS_CLASS_1_MINBANDWIDTH='384Kibit/s'
QOS_CLASS_1_MAXBANDWIDTH='768Kibit/s'
QOS_CLASS_1_DIRECTION='down'
QOS_CLASS_1_PRIO=''

QOS_CLASS_2_PARENT='0'
QOS_CLASS_2_MINBANDWIDTH='384Kibit/s'
QOS_CLASS_2_MAXBANDWIDTH='768Kibit/s'
QOS_CLASS_2_DIRECTION='down'
QOS_CLASS_2_PRIO=''

QOS_CLASS_3_PARENT='1'
QOS_CLASS_3_MINBANDWIDTH='256Kibit/s'
QOS_CLASS_3_MAXBANDWIDTH='768Kibit/s'
QOS_CLASS_3_DIRECTION='down'
QOS_CLASS_3_PRIO=''

QOS_CLASS_4_PARENT='1'
QOS_CLASS_4_MINBANDWIDTH='128Kibit/s'
QOS_CLASS_4_MAXBANDWIDTH='768Kibit/s'
QOS_CLASS_4_DIRECTION='down'
QOS_CLASS_4_PRIO=''

QOS_CLASS_5_PARENT='2'
QOS_CLASS_5_MINBANDWIDTH='256Kibit/s'
QOS_CLASS_5_MAXBANDWIDTH='768Kibit/s'
QOS_CLASS_5_DIRECTION='down'
QOS_CLASS_5_PRIO=''

QOS_CLASS_6_PARENT='2'
QOS_CLASS_6_MINBANDWIDTH='128Kibit/s'
QOS_CLASS_6_MAXBANDWIDTH='768Kibit/s'
QOS_CLASS_6_DIRECTION='down'
QOS_CLASS_6_PRIO=''

QOS_FILTER_N='4'

QOS_FILTER_1_CLASS='3'
QOS_FILTER_1_IP_INTERN='192.168.0.2'
QOS_FILTER_1_IP_EXTERN=''
QOS_FILTER_1_PORT='80'
QOS_FILTER_1_PORT_TYPE='client'
QOS_FILTER_1_OPTION=''

QOS_FILTER_2_CLASS='4'
QOS_FILTER_2_IP_INTERN='192.168.0.2'
QOS_FILTER_2_IP_EXTERN=''
QOS_FILTER_2_PORT=''
QOS_FILTER_2_PORT_TYPE=''
QOS_FILTER_2_OPTION=''

QOS_FILTER_3_CLASS='5'
QOS_FILTER_3_IP_INTERN='192.168.0.3'
QOS_FILTER_3_IP_EXTERN=''
QOS_FILTER_3_PORT='80'
QOS_FILTER_3_PORT_TYPE='client'
QOS_FILTER_3_OPTION=''

QOS_FILTER_4_CLASS='6'
QOS_FILTER_4_IP_INTERN='192.168.0.3'
QOS_FILTER_4_IP_EXTERN=''
QOS_FILTER_4_PORT=''
QOS_FILTER_4_PORT_TYPE=''
QOS_FILTER_4_OPTION=''

\end{verbatim}
\end{example}
\end{small}

   Option \var{QOS\_\-INTERNET\_\-DEFAULT\_\-DOWN} was set in a way that
   traffic not being assigned to a class by a filter is put in a non-existent
   class. This is to simplify the example and because it is assumed that
   there is no unassigned traffic left. Traffic being sent to a non-existent
   class is forwarded very slow. If a rest of traffic exists always ensure
   that it is assigned to its own (existing) class.

   Option \var{QOS\_\-INTERNET\_\-DEFAULT\_\-UP} was set to 0 because upstream
   should not be regulated.



\subsubsection{Example 3}



   An example targeting at learning the general structures of working with QoS.

   \begin{figure}[htbp]
     \centering
     \includegraphics{qos_bsp_3}
     \caption{Example 3}
     \marklabel{fig:qosbsp3}{}
   \end{figure}


   Picture \ref{fig:qosbsp3} once again shows the layout of example two
   but this time with an extension. Both subclasses of the second class
   have two more subclasses attached. You see that it is possible to have
   nested subclasses. The maximum for nested subclasses is a depth of 8
   where 0 is the interface itself leaving 7 more possible subclass levels.
   ``Width'' is unlimited though. A subclass can have an unlimited number
   of classes attached.

   The picture shows as well that it is possible to have more than one filter
   attached to a class as it is done in class 10. Pay attention that at the
   moment it is not possible to attach a filter in the middle of the ``tree''
   as F8 intended to.

   Lets have a closer look at the sense of classes and subclasses. Classes
   set and control speed of a connection. Spreading of speed is handled
   described as in \var{QOS\_\-CLASS\_\-x\_\-MINBANDWIDTH}. This can have
   disadvantages if i.e. you attach all classes to one parent class. If it
   is for example intended that one client PC should have half of the available
   bandwidth and the other half is for a second client PC divided in 2/3 http
   and 1/3 for the rest (2/6 and 1/6 of the whole bandwidth) this would happen:
   If both clients run at full load both get their half of the bandwidth. If
   the second one is not transferring http 2/6 of unused bandwidth are
   distributed not only to the second but to both PCs as decribed above.
   To avoid this subclasses are created. Traffic of a class is at first
   distributed to its subclasses. Only if they don't use the complete
   traffic the rest is spread to the other classes. In the picture the
   areas that belong together are encircled (red = 1, blue = 2, green = 5
   and orange = 6.



\subsubsection{Example 4}


   Configuration for ACK packet priorization in order to keep
   downstream high if upstream has heavy load:

\begin{small}
\begin{example}
\begin{verbatim}
OPT_QOS='yes'
QOS_INTERNET_DEV_N='1'
QOS_INTERNET_DEV_1='ppp0'
QOS_INTERNET_BAND_DOWN='768Kibit/s'
QOS_INTERNET_BAND_UP='128Kibit/s'
QOS_INTERNET_DEFAULT_DOWN='0'
QOS_INTERNET_DEFAULT_UP='2'

\end{verbatim}
\end{example}
\end{small}

   Configure ppp0 as the Internet device (DSL) and give it the usual
   up/downstream bandwidth for TDSL (and some other providers). It may be
   necessary to lower upstream bandwidth for some Kibibit for the trying.

   No classes for downstream should be defined:
\begin{example}
\begin{verbatim}
        QOS_INTERNET_DEFAULT_DOWN='0'
\end{verbatim}
\end{example}

   For upstream class number two should be the default class. The network device
   eth0 is set to 10Mibit/s.

\begin{small}
\begin{example}
\begin{verbatim}
   QOS_CLASS_N='2'

QOS_CLASS_1_PARENT='0'
QOS_CLASS_1_MINBANDWIDTH='127Kibit/s'
QOS_CLASS_1_MAXBANDWIDTH='128Kibit/s'
QOS_CLASS_1_DIRECTION='up'
QOS_CLASS_1_PRIO=''

\end{verbatim}
\end{example}
\end{small}
   This is the class for ACK (acknowledgement) packets. ACK packets are rather
   small and thus need only a minimum bandwidth. Because they should not be
   affected in any way they get 127Kibit/s. 1Kibit/s is left for the rest.

\begin{small}
\begin{example}
\begin{verbatim}
QOS_CLASS_2_PARENT='0'
QOS_CLASS_2_MINBANDWIDTH='1Kibit/s'
QOS_CLASS_2_MAXBANDWIDTH='128Kibit/s'
QOS_CLASS_2_DIRECTION='up'
QOS_CLASS_2_PRIO=''
\end{verbatim}
\end{example}
\end{small}

   This class is for everything else (except ACK packets). The bandwidth
   that is left is 1Kibit/s (128-127=1). We don't limit it to 1Kibit/s though,
   the class is limited by the entry
\begin{example}
\begin{verbatim}
        QOS_CLASS_2_MAXBANDWIDTH='128Kibit/s'
\end{verbatim}
\end{example}

   Because our first class never can use all of its bandwidth there will be
   something left over which then gets allocated to the second class. If upstream
   should be divided some more (prominent use case) all other classes have to
   be subclasses ``under'' this class.  Of course \var{QOS\_\-INTERNET\_\-DEFAULT\_\-UP}
   has to be adapted then.

\begin{small}
\begin{example}
\begin{verbatim}
QOS_FILTER_N='1'

QOS_FILTER_1_CLASS='1'
QOS_FILTER_1_IP_INTERN=''
QOS_FILTER_1_IP_EXTERN=''
QOS_FILTER_1_PORT=''
QOS_FILTER_1_PORT_TYPE=''
QOS_FILTER_1_OPTION='ACK'
\end{verbatim}
\end{example}
\end{small}

   This filter filters all packets matching option ACK (i.e. ACK packets).
   By specifying \var{QOS\_\-FILTER\_\-1\_\-CLASS}='1' we achieve that
   all these packets filtered are sent to class 1.

   For testing purposes look for one or more good up- and download sources
   that can produce full load for up- as well as for downstream. Have a
   look at the traffic display in ImonC. Try this once with and once without
   QoS.

   Downstream should not or not as much decline as without this configuration.
   It could get even better by declining upstream bandwidth in steps of Kibibits
   and analyze the effect. I reached my optimum at 121Kibit/s (no declining
   downstreams anymore). Of course MAXBANDWIDTH- and MINBANDWIDTH- values
   of all classes have to be adapted accordingly.

