% Synchronized to r29817
\paragraph*{Name}
DS18S20 - High-Precision 1-Wire Digital Thermometer 

DS1920 - iButton version of the thermometer 
\paragraph*{Synopsis}
Thermometer. 

\textbf{10} [.]XXXXXXXXXXXX[XX][/[
\textbf{die} $|$ \textbf{power} $|$ \textbf{temperature} $|$ \textbf{temphigh} $|$ \textbf{templow} $|$ \textbf{trim} $|$ \textbf{trimblanket} $|$ \textbf{trimvalid}
$|$           \textbf{address} $|$ \textbf{crc8} $|$ \textbf{id} $|$ \textbf{locator} $|$ \textbf{r\_address} $|$ \textbf{r\_id} $|$ \textbf{r\_locator} $|$ \textbf{type}
 ]] 
\paragraph*{Family Code}


\textit{10} 
\paragraph*{Special Properties}

\subparagraph*{power}\textit{read-only,yes-no} 

Is the chip powered externally (=1) or from the parasitically from the
data bus (=0)? 
\subparagraph*{temperature}\textit{read-only,} floating point 

\textit{Temperature} read by the chip at high resolution (~12 bits). Units are selected
from the invoking command line. See \textsf{\textbf{owfs(1)}} or \textsf{\textbf{owhttpd(1)}} for choices. Default
is Celsius. Conversion takes 1000 msec. 
\paragraph*{Temperature Alarm Limits}
When the
device exceeds either \textit{temphigh} or \textit{templow} temperature threshold the device
is in the alarm state, and will appear in the alarm directory. This provides
an easy way to poll for temperatures that are unsafe, especially if  \textit{simultaneous}
temperature conversion is done. 

Units for the temperature alarms are in
the same \textit{temperature} scale that was set for \textit{temperature} measurements. 

Temperature
thresholds are stored in non-volatile memory and persist until changed,
even if power is lost. 
\subparagraph*{temphigh}\textit{read-write,} integer 

Shows or sets the lower limit for the high temperature alarm state.  
\subparagraph*{templow}\textit{read-write,}
integer 

Shows or sets the upper limit for the low temperature alarm state.   
\paragraph*{Temperature
Errata}
There are a group of obscure internal properties exposed to protect
against an hardware defect in certain batches of the B7 die of some DS18x20
chips. See \altlink{http://www.1wire.org/en-us/pg_18.html} or request AN247.pdf from Dallas
directly. 
\subparagraph*{errata/die}\textit{read-only,ascii} 

Two character manufacturing die lot. "B6" "B7" or "C2" 
\subparagraph*{errata/trim}\textit{read-write,unsigned}
integer 

32 bit trim value in the EEPROM of the chip. When written, it does not seem
to read back. Used for a production problem in the B7 \textit{die.} 

Read allowed for
all chips. Only the B7 chips can be written. 
\subparagraph*{errata/trimblanket}\textit{read-write,yes-no}


Writing non-zero (=1) puts a default trim value in the chip. Only applied
to the B7 \textit{die.} Reading will be true (non-zero) if trim value is the blanket
value. Again, only B7 chips will register true, and since the written trim
values cannot be read, this value may have little utility. 
\subparagraph*{errata/trimvalid}\textit{read-only,yes-no}


Is the  \textit{trim}  value in the valid range? Non-zero if true, which includes
all non-B7 chips.  
\paragraph*{Standard Properties}
          
\subparagraph*{address}
\subparagraph*{r\_address}\textit{read-only,}
ascii 

The entire 64-bit unique ID. Given as upper case hexidecimal digits (0-9A-F).


\textit{address} starts with the \textit{family} code 

\textit{r} address is the \textit{address} in reverse order, which is often used in other
applications and labeling. 
\subparagraph*{crc8}\textit{read-only,} ascii 

The 8-bit error correction portion. Uses cyclic redundancy check. Computed
from the preceding 56 bits of the unique ID number. Given as upper case
hexidecimal digits (0-9A-F). 
\subparagraph*{family}\textit{read-only,} ascii 

The 8-bit family code. Unique to each \textit{type} of device. Given as upper case
hexidecimal digits (0-9A-F). 
\subparagraph*{id}
\subparagraph*{r\_id}\textit{read-only,} ascii 

The 48-bit middle portion of the unique ID number. Does not include the family
code or CRC. Given as upper case hexidecimal digits (0-9A-F). 

\textit{r} id is the \textit{id} in reverse order, which is often used in other applications
and labeling. 
\subparagraph*{locator}
\subparagraph*{r\_locator}\textit{read-only,} ascii 

Uses an extension of the 1-wire design from iButtonLink company that associated
1-wire physical connections with a unique 1-wire code. If the connection is
behind a \textbf{Link Locator} the \textit{locator} will show a unique 8-byte number (16 character
hexidecimal) starting with family code FE. 

If no \textbf{Link Locator} is between the device and the master, the \textit{locator} field
will be all FF. 

\textit{r} locator is the \textit{locator} in reverse order. 
\subparagraph*{present (DEPRECATED)}\textit{read-only,}
yes-no 

Is the device currently \textit{present} on the 1-wire bus? 
\subparagraph*{type}\textit{read-only,} ascii 

Part name assigned by Dallas Semi. E.g. \textit{DS2401} Alternative packaging (iButton
vs chip) will not be distiguished.  
\paragraph*{Description}
          
\subparagraph*{1-Wire}\textit{1-wire}  is
a wiring protocol and series of devices designed and manufactured by Dallas
Semiconductor, Inc. The bus is a low-power low-speed low-connector scheme where
the data line can also provide power. 

Each device is uniquely and unalterably
numbered during manufacture. There are a wide variety of devices, including
memory, sensors (humidity, temperature, voltage, contact, current), switches,
timers and data loggers. More complex devices (like thermocouple sensors)
can be built with these basic devices. There are also 1-wire devices that
have encryption included. 

The 1-wire scheme uses a single  \textit{bus} master and
multiple \textit{slaves} on the same wire. The bus master initiates all communication.
The slaves can be  individually discovered and addressed using their unique
ID. 

Bus masters come in a variety of configurations including serial, parallel,
i2c, network or USB adapters. 
\subparagraph*{OWFS design}\textit{OWFS} is a suite of programs that
designed to make the 1-wire bus and its devices easily accessible. The underlying
priciple is to create a virtual filesystem, with the unique ID being the
directory, and the individual properties of the device are represented
as simple files that can be read and written. 

Details of the individual
slave or master design are hidden behind a consistent interface. The goal
is to  provide an easy set of tools for a software designer to create monitoring
or control applications. There  are some performance enhancements in the
implementation, including data caching, parallel access to bus  masters,
and aggregation of device communication. Still the fundemental goal has
been ease of use, flexibility  and correctness rather than speed.  
\subparagraph*{Ds18s20
Ds1920}The \textsf{\textbf{DS18S20 (3)}} is one of several available 1-wire temperature sensors.
It has been largely replaced by the \textsf{\textbf{DS18B20 (3)}} and \textsf{\textbf{DS1822 (3)}} as well
as temperature/vlotage measurements in the \textsf{\textbf{DS2436 (3)}} and \textsf{\textbf{DS2438 (3)}.} For
truly versatile temperature measurements, see the protean \textsf{\textbf{DS1921 (3)} \textsf{Thermachron
(3)}.} 
\paragraph*{Addressing}
          All 1-wire devices are factory assigned a unique
64-bit address. This address is of the form: \begin{description}
\item [\textbf{Family Code} ] 8 bits 
\item [\textbf{Address} ] 48
bits 
\item [\textbf{CRC} ] 8 bits 
\end{description}


\begin{description}
\item [Addressing under OWFS is in hexidecimal, of form: ] 
\item [\textbf{01.123456789ABC}
] 
\end{description}


where \textbf{01} is an example 8-bit family code, and \textbf{12345678ABC} is an example
48 bit address. 

The dot is optional, and the CRC code can included. If included,
it must be correct.  
\paragraph*{Datasheet}


\altlink{http://pdfserv.maxim-ic.com/en/ds/DS18S20.pdf} 
\paragraph*{See Also}

\subparagraph*{Programs}\textsf{\textbf{owfs (1)} \textsf{owhttpd
(1)} \textsf{owftpd (1)} \textsf{owserver (1)}} \textsf{\textbf{owdir (1)} \textsf{owread (1)} \textsf{owwrite (1)} \textsf{owpresent
(1)}} \textsf{\textbf{owtap (1)}} 
\subparagraph*{Configuration and testing}\textsf{\textbf{owfs (5)} \textsf{owtap (1)} \textsf{owmon (1)}} 
\subparagraph*{Language
bindings}\textsf{\textbf{owtcl (3)} \textsf{owperl (3)} \textsf{owcapi (3)}} 
\subparagraph*{Clocks}\textsf{\textbf{DS1427 (3)} \textsf{DS1904(3)} \textsf{DS1994
(3)} \textsf{DS2404 (3)} \textsf{DS2404S (3)} \textsf{DS2415 (3)} \textsf{DS2417 (3)}} 
\subparagraph*{ID}\textsf{\textbf{DS2401 (3)} \textsf{DS2411 (3)}
\textsf{DS1990A (3)}} 
\subparagraph*{Memory}\textsf{\textbf{DS1982 (3)} \textsf{DS1985 (3)} \textsf{DS1986 (3)} \textsf{DS1991 (3)} \textsf{DS1992 (3)}
\textsf{DS1993 (3)} \textsf{DS1995 (3)} \textsf{DS1996 (3)} \textsf{DS2430A (3)} \textsf{DS2431 (3)} \textsf{DS2433 (3)} \textsf{DS2502
(3)} \textsf{DS2506 (3)} \textsf{DS28E04 (3)} \textsf{DS28EC20 (3)}} 
\subparagraph*{Switches}\textsf{\textbf{DS2405 (3)} \textsf{DS2406 (3)} \textsf{DS2408
(3)} \textsf{DS2409 (3)} \textsf{DS2413 (3)} \textsf{DS28EA00 (3)}} 
\subparagraph*{Temperature}\textsf{\textbf{DS1822 (3)} \textsf{DS1825 (3)}
\textsf{DS1820 (3)} \textsf{DS18B20 (3)} \textsf{DS18S20 (3)} \textsf{DS1920 (3)} \textsf{DS1921 (3)} \textsf{DS1821 (3)} \textsf{DS28EA00
(3)} \textsf{DS28E04 (3)} \textsf{EDS0064 (3)} \textsf{EDS0065 (3)} \textsf{EDS0066 (3)} \textsf{EDS0067 (3)} \textsf{EDS0068
(3)} \textsf{EDS0071 (3)} \textsf{EDS0072 (3)}} 
\subparagraph*{Humidity}\textsf{\textbf{DS1922 (3)} \textsf{DS2438 (3)} \textsf{EDS0065 (3)} \textsf{EDS0068
(3)}} 
\subparagraph*{Voltage}\textsf{\textbf{DS2450 (3)}} 
\subparagraph*{Resistance}\textsf{\textbf{DS2890 (3)}} 
\subparagraph*{Multifunction (current, voltage,
temperature)}\textsf{\textbf{DS2436 (3)} \textsf{DS2437 (3)} \textsf{DS2438 (3)} \textsf{DS2751 (3)} \textsf{DS2755 (3)} \textsf{DS2756
(3)} \textsf{DS2760 (3)} \textsf{DS2770 (3)} \textsf{DS2780 (3)} \textsf{DS2781 (3)} \textsf{DS2788 (3)} \textsf{DS2784 (3)}} 
\subparagraph*{Counter}\textsf{\textbf{DS2423
(3)}} 
\subparagraph*{LCD Screen}\textsf{\textbf{LCD (3)} \textsf{DS2408 (3)}} 
\subparagraph*{Crypto}\textsf{\textbf{DS1977 (3)}} 
\subparagraph*{Pressure}\textsf{\textbf{DS2406 (3)} -- TAI8570
\textsf{EDS0066 (3)} \textsf{EDS0068 (3)}}  
\paragraph*{Availability}
\altlink{http://www.owfs.org/} 
\paragraph*{Author}
Paul Alfille
(\email{paul.alfille@gmail.com}) 
