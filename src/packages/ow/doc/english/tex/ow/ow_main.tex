% Synchronized to r37690
\marklabel{sec:ow}

\setcounter{secnumdepth}{6}

\newcommand{\IsqC}{$\textrm{I}^{\textrm{2}}\textrm{C}$}
\newcommand{\isqc}{$\textrm{i}^{\textrm{2}}\textrm{c}$}

\section{Introduction}

This package installs OWFS (see chapter \ref{cap:OW_OWFS}) and provides read/write
access to a 1--wire bus plugged to fli4l. A 1--wire busmaster is connected to a serial
interface\footnote{DS9097U COM Port adadpter.} or to a USB port\footnote{DS9490R USB Bridge
also in conjunction with DS1402D-DR8 (Blue Dot™) for iButton. All DS9490 adapters based on
DS2490 USB--1--wire-chips.} of the PC.
In addition, the OPT also supports \IsqC{} adapters and binding to
an OWServer. Find more details in the man pages following (chapter:
\ref{cap:OW_MANPAGES}). The 1--wire-side of the adapters is connected to the 1--wire bus.

\section{Hardware}
\subsection{The 1--Wire Standard}
The 1--Wire ® resp. One-Wire bus from Maxim (Maxim/Dallas) describes a serial interface
that uses just one data wire and is used both as a power supply and as a transmit and
receive line. However, a \glqq{}return\grqq{} (GND) is required. Every 1--Wire chip has
a unique ID number by which it can be addressed. So several 1--Wire devices can be
connected on a single bus.

\subsection{The 1--Wire-Components}
Maxim offers a variety of 1--Wire-Components: Serial-, USB-,
\IsqC-adapters, thermometers, switches(up to 8 channels), EEPROMs, Clocks, A/D-converters,
digital potentiometer. You really get everything you need for home automation.
An overview of the major components can be found in the appendix
under \ref{cap:OW_FAMILYCODE}.
You may also connect iButton ® parts (NV-RAM, EPROM, EEPROM, Temperature,
Humidity, RTC, SHA, Logger).

\subsection{The 1--Wire-Bus}
The 1--wire-bus in principle consists of two twisted lines, in accordance with
corresponding topologies longer distances up to 150 m should not be a problem.
Cat5 twisted pair Ethernet cable is often used for the wiring.
For the assignment of the individual cores, different approaches exist. Maxim
uses 6-pin modular jacks and plugs (RJ-11) and has created its own standard,
but this does not fit the 8-pin RJ-45 plug stuff. Further standards are
described in the appendix (chapter \ref{cap:OW_PINADERBELEGUNG}).
You will also find informations on the topology of the bus at Maxim, their
website offers everything you need to deal with 1--wire.

\marklabel{cap:OW_OWFS}
{
\section{OWFS}
}
OWFS  stands for \glqq{}One Wire File System\grqq{}. This is a software developed by Paul H.
Alfille licensed under the GPL. Based on a 1--Wire-protocol system libraries (OWLib) depict
the 1--Wire-Bus OWFS as a file system. In addition, the program offers additional implementations,
as owserver, owshell, owhttpd, owftpd, owtap and language modules for capi,
perl, tcl, php, which were not included in the present adaption for fli4l.
Details on OWFS and a lot of interesting things for 1--Wire may be found
at: \altlink{ http://owfs.org/} and \altlink{http://sourceforge.net/projects/owfs/}.

\section{Fuse}
Fuse stands for \glqq{}filesystem in userspace\grqq{}. Fuse enables the implementation of a
fully functional filesystem in userspace.
With the installation of \var{OPT\_OW} the Fuse kernel module is automatically loaded at startup.
Everything else about Fuse can be found at:
\altlink{http://fuse.sourceforge.net/} and \altlink{http://sourceforge.net/projects/fuse/.}

\section{libusb}
libusb is a free, GPL-licensed USB library which is required to access
the 1--wire bus with a USB adapter.
Everything else concerning libusb can be found at: \altlink{http://libusb.sourceforge.net/}

\section{License}
This program is licensed under the GNU General Public License, Version 2, June 1991
and can be freely used, reproduced and altered under the conditions indicated.
The text of the GNU General Public License can be found at:
\altlink{http://www.gnu.org/licenses/gpl.txt}

\section{Warranty And Liability Disclaimer}
This program was devloped with the will and in the hope
that it will be useful. Nevertheless, there is no warranty of any kind --
Warranty of merchantability or fitness for a particular purpose are rejected.
For details, refer to the GNU General Public License (GPL).
There is no liability for loss of data, damage to hardware or software or any other damage.

\section{System Requirements}
 based on the size of \var{OPT\_OW} a harddisk resp. a flash card is needed.
For details see \var{OPT\_HD}.

For display in a browser the webserver from fli4l package \glqq{}httpd\grqq{} is
needed. For further details see chapter \ref{cap:OW_BROWSER}.

\textbf{Please note:}

The USB control via the W1 kernel modules does not yet work (according to Paul Alfille,
the maintainer of OWFS) and has not been tested in Opt.
(The W1 modules for version V2.8 p16 and p19 were tested once, but since
connection and evaluation are completely different from the standard version,
were not examined further for correct funtion)

To use the USB adapter the details of the system must be present in "udev" in "rules.d".
Only if these settings are correct, the connection to OWSERVER and OWFS will work.

The use of the two programs \emph{owshell} and \emph{owhttpd} did not work
properly on some hardware environments. The authors try to find a solution to
the problem in collaboration with Paul Alfille. If errors occur, you may try posting
on the fli4l newsgroup with a detailed description of your problem.

\section{Installation}
After unpacking the tar.gz archive into the fli4l directory adapt the file
config/ow.txt to your needs. To use the web interface activate the httpd
webserver via \var{OPT\_HTTPD}='yes' (see chapter \ref{cap:OW_SONSTIGEVARIABLEN}).
If RRDTool is used for recording system readings, the configuration of the
text fileconfig/rrdtool.txt is needed too (see chapter \ref{cap:OW_RRDTOOL}).

\section{Configuration}
Example configuration without comments, further explanations below:

\begin{example}
\begin{verbatim}
    OPT_OW='yes'                      # install OPT_OW (yes/no)
    OW_USER_SCRIPT=''                 # e.g. 'usr/local/bin/ow-user-script.sh'

    OW_OWFS='yes'                     # start owfs (yes/no)
    OW_OWFS_DEV='usb'                 # usb*, ttyS*, ip:port, etc.
    OW_OWFS_GROUP_N='4'                           # number of groups
    OW_OWFS_GROUP_1_NAME='1--Wire at USB'         # name of first group
    OW_OWFS_GROUP_1_PORT_N='2'                    # number of ports of device
    OW_OWFS_GROUP_1_PORT_1_ID='81.70D42A000000/ID'      # ID of device
    OW_OWFS_GROUP_1_PORT_1_ALIAS='ID'                   # alias of ID
    OW_OWFS_GROUP_1_PORT_2_ID='81.70D42A000000/Admin/*' # admin-access
    OW_OWFS_GROUP_1_PORT_2_ALIAS='Admin/'               # alias of admin

    OW_OWFS_GROUP_2_NAME='Heating'
    OW_OWFS_GROUP_2_PORT_N='7'
    OW_OWFS_GROUP_2_PORT_1_ID='3A.F6E401000000/PA'
    OW_OWFS_GROUP_2_PORT_1_ALIAS='1. circulation pump'
    OW_OWFS_GROUP_2_PORT_2_ID='3A.F6E401000000/PB'
    OW_OWFS_GROUP_2_PORT_2_ALIAS='2. charging pump'
    OW_OWFS_GROUP_2_PORT_3_ID='10.651BA9010800/temp'
    OW_OWFS_GROUP_2_PORT_3_ALIAS='4. Return temperature'
    OW_OWFS_GROUP_2_PORT_4_ID='10.DEF0A8010800/temp'
    OW_OWFS_GROUP_2_PORT_4_ALIAS='3. flow temperature'
    OW_OWFS_GROUP_2_PORT_5_ID='3A.F6E401000000/Admin/*'
    OW_OWFS_GROUP_2_PORT_5_ALIAS='Admin/Switch-'
    OW_OWFS_GROUP_2_PORT_6_ID='10.DEF0A8010800/Admin/*'
    OW_OWFS_GROUP_2_PORT_6_ALIAS='Admin/VLT-'
    OW_OWFS_GROUP_2_PORT_7_ID='10.651BA9010800/Admin/*'
    OW_OWFS_GROUP_2_PORT_7_ALIAS='Admin/RLT-'

    OW_OWFS_GROUP_3_NAME='Solar devices'
    OW_OWFS_GROUP_3_PORT_N='3'
    OW_OWFS_GROUP_3_PORT_1_ID='1C.7F6CF7040000/P0'
    OW_OWFS_GROUP_3_PORT_1_ALIAS='1. charging pump'
    OW_OWFS_GROUP_3_PORT_2_ID='1C.7F6CF7040000/P1'
    OW_OWFS_GROUP_3_PORT_2_ALIAS='2. valve'
    OW_OWFS_GROUP_3_PORT_3_ID='1C.7F6CF7040000/Admin/*'
    OW_OWFS_GROUP_3_PORT_3_ALIAS='Admin/Switch-'

    OW_OWSHELL='yes'
    OW_OWSHELL_RUN='yes'
    OW_OWSHELL_DEV='usb'
    OW_OWSHELL_PORT='127.0.0.1:4304'

    OW_OWHTTPD='yes'
    OW_OWHTTPD_RUN='yes'
    OW_OWHTTPD_DEV='127.0.0.1:4304'
    OW_OWHTTPD_PORT='8080'
\end{verbatim}
\end{example}

The following variables have to be to configured in the file config/ow.txt:

\begin{description}
\config{OPT\_OW}{OPT\_OW}{OPTOW}
With the default setting OPT\_OW='no', the package is not installed.
Using \var{OPT\_OW}='yes' activates the package.

\config{OW\_USER\_SCRIPT}{OW\_USER\_SCRIPT}{OWUSERSCRIPT}
This variable defines the path and file name of an optional background control,
with which, for example, the heating system can be controlled.
Further details can be found in chapter \ref{cap:OW_OWUSERSCRIPT}.

\config{OW\_OWFS}{OW\_OWFS}{OWOWFS}
OWFS provides easy access to the 1--wire bus via the fli4l web interface. By
specifying \var{OW\_OWFS}='yes' a file system in the default path '/var/run/ow'
is generated using fuse. The 1--wire bus is depicted there. The directories created
in the file system are sorted by Identnumbers (see Appendix \ref{cap:OW_FAMILYCODE})
of the chips. With the family code of the components a corresponding sortorder
can easily be created.

\config{OW\_OWFS\_DEV}{OW\_OWFS\_DEV}{OWOWFSDEV}
The variable \var{OW\_OWFS\_DEV} defines the PC interface the
1--wire adapter is connected to.

\begin{tabular}{|l|l|p{0.5\textwidth}|}
\hline
\textbf{PC interface} & \textbf{Variable set to} & \textbf{Example} \\
\hline
serial           & ttyS*          & ttyS0 = COM1, ttyS1 = COM2 \\
\hline
\multirow{3}{*}{}{USB}
                 & ttyUSB*        & ttyUSB1 = first USB adapter \\
\cline{2-3}
          \latex{&} usb           & usb = first USB adapter \\
\cline{2-3}
          \latex{&} usb[2-9]      & usb3 = third USB adapter \\
\hline
\IsqC{}          & \isqc{}-[0-9]  & \isqc{}-0 = first \IsqC{} port \\
\hline
\multirow{2}{*}{}{Simulation}
                 & fake           & \multirow{2}{*}{}{For using '\var{FAKE}'
                                    and '\var{TESTER}' modes set the
                                    variables \var{OW\_OWFS\_FAKE} or
                                    \var{OW\_OWFS\_TESTER} to valid family
                                    codes, see chapter
                                    \ref{cap:OW_SONSTIGEVARIABLEN}} \\
\cline{2-2}
     \latex{&} tester \latex{&} \\
\hline
\end{tabular}

\config{OW\_OWFS\_GROUP\_N}{OW\_OWFS\_GROUP\_N}{OWOWFSGROUPN}
The variable \var{OW\_OWFS\_GROUP\_N} specifies the number of groups
displayed in the browser where In- and Outputs belonging together i.e.
for driving a solar device and the corresponding names for
OW\_OWFS\_GROUP\_NAME are defined.

\configlabel{OW\_OWFS\_GROUP\_x\_PORT\_N}{OWOWFSGROUPxPORTN}
\configlabel{OW\_OWFS\_GROUP\_x\_PORT\_x\_ALIAS}{OWOFSGROUPxPORTxALIAS}
\config{OW\_OWFS\_GROUP\_x\_PORT\_N
	OW\_OWFS\_GROUP\_x\_PORT\_x\_ID
	OW\_OWFS\_\_GROUP\_x\_PORT\_x\_ALIAS}
	{OW\_OWFS\_GROUP\_x\_PORT\_x\_ID}{OWOWFSGROUPxPORTxID}
The variable \var{OW\_OWFS\_GROUP\_x\_PORT\_N} defines the number of ports
for a group. With the two subsquent variables \var{OW\_OWFS\_GROUP\_x\_PORT\_x\_ID} and \\
\var{OW\_OWFS\_GROUP\_x\_PORT\_x\_ALIAS} you assign a name to the In-
resp. Output of 1--wire components.

If you want to suppress display of certain data in the web interface, i.e. because
the port of a component has not been established or after completion of the
configuration the admin branch is no longer needed, you can prefix the name with
an exclamation mark (!).

\config{OW\_OW\_SHELL}{OW\_OW\_SHELL}{OWOWSHELL}
Activation of the OWFS "server", for providing the OWFS-BUS simultaneously
for multiple applications (OWFS and OWHTTPD). For using this no other application
may be set to the direct interface of the adapter, but instead must be linked to
the server.

\config{OW\_OW\_SHELL\_RUN}{OW\_OW\_SHELL\_RUN}{OWOWSHELLRUN}
Should  the Server service be started at boot time?
\config{OW\_OW\_SHELL\_DEV}{OW\_OW\_SHELL\_DEV}{OWOWSHELLDEV}
The device the server accesses (hardware)
\config{OW\_OW\_SHELL\_PORT}{OW\_OW\_SHELL\_PORT}{OWOWSHELLPORT}
IP-address and port the server uses.
Only the localhost address 127.0.0.1 makes sense here.
As a default port 4304 (OWFS port) should be used for the Server.
This address is hardcoded into package RRDTool. If you want to collect
values via RRDTool you may not change this setting.

\config{OW\_OWHTTPD}{OW\_OWHTTPD}{OWOWHTTPD}
Activation of OWFS's own web server.
\config{OW\_OWHTTPD\_RUN}{OW\_OWHTTPD\_RUN}{OWOWHTTPDRUN}
Should the web server be started during system boot?
\config{OW\_OWHTTPD\_READONLY}{OW\_OWHTTPD\_READONLY}{OWOWHTTPDREADONLY}
Should write access to the components be allowed in OWFS via the web server?
\config{OW\_OWHTTPD\_DEV}{OW\_OWHTTPD\_DEV}{OWOWHTTPDDEV}
Device, the web server accesses. In conclusion with OW\_OWSHELL (Server) also
a single device may be accessed here.
\config{OW\_OWHTTPD\_PORT}{OW\_OWHTTPD\_PORT}{OWOWHTTPDPORT}
HTTP port for the web server.

Configuration example:
\begin{example}
\begin{verbatim}
    OW_OWFS_GROUP_x_PORT_x_ID='29.57D305000000/P6'
    OW_OWFS_GROUP_x_PORT_x_ALIAS='EA-Modul/!P6'        # Signal suppressed
    OW_OWFS_GROUP_x_PORT_x_ID='29.57D305000000/Admin/*'
    OW_OWFS_GROUP_x_PORT_x_ALIAS='EA-Modul/Admin/!'    # Admin path deactivated
                                                       # completely
\end{verbatim}
\end{example}

A detailed description for the configuration of OWFS can be found in the appendix
\glqq{}\ref{cap:OW_MANPAGES}\grqq{} and here: \altlink{http://owfs.org/index.php?page=owfs}.

\end{description}

\marklabel{cap:OW_SONSTIGEVARIABLEN}
{
\section{Miscellaneous Variables}
}
The following variables can be customized via the file config/ow.txt if needed:

\begin{description}
\config{OW\_LOG\_DESTINATION}{OW\_LOG\_DESTINATION}{OWLOGDESTINATION}
Target for status and error outputs.

\begin{verbatim}
    0 = mixed (1 and 2)
    1 = syslog
    2 = stderr
    3 = off
\end{verbatim}

Default setting is '1'.

\config{OW\_LOG\_LEVEL}{OW\_LOG\_LEVEL}{OWLOGLEVEL}
 The Log level (1-9) determines the amount of status and error outputs, with:

\begin{verbatim}
    1 = silent and 9 = maximum verbosity
\end{verbatim}

Default setting is '1'.

\config{OW\_TEMP\_SCALE}{OW\_TEMP\_SCALE}{OWTEMPSCALE}
 The available temperature scales.
\begin{verbatim}
    C = "Celsius"
    F = "Fahrenheit"
    K = "Kelvin"
    R = "Rankine"
\end{verbatim}

Default setting is 'C'.

\config{OW\_REFRESH\_INTERVAL}{OW\_REFRESH\_INTERVAL}{OWREFRESHINTERVAL}
 Refresh for fli4l-HTTP in seconds. '0' = no refresh.

 Default setting is '10'.

\config{OW\_OWFS\_FAKE}{OW\_OWFS\_FAKE}{OWOWFSFAKE}
Enables the random simulation of 1--wire components.
There may be specified multiple components with family
code separated by spaces. The simulated states are purely
coincidental. The option can not be activated simultaneously
with the 'TESTER' mode.

\config{OW\_OWFS\_TESTER}{OW\_OWFS\_TESTER}{OWOWFSTESTER}
Enables the systematic simulation of 1--wire components.
There may be specified multiple components with family
code separated by spaces. The simulated conditions follow
realistic values. This option can not be activated
simultaneously with the 'FAKE' mode.

\config{OW\_OWFS\_RUN}{OW\_OWFS\_RUN}{OWOWFSRUN}
Specifies whether owfs should be started automatically during
the boot of the router. Default value is 'yes', whereas
with 'no' the application must be started manually.

\config{OW\_OWFS\_READONLY}{OW\_OWFS\_READONLY}{OWOWFSREADONLY}
Sets with "yes" stated that component states can only be
read but not written via owfs.

 Default setting is 'no'.

\config{OW\_OWFS\_PATH}{OW\_OWFS\_PATH}{OWOWFSPATH}
Specifies the root directory for the fuse directory structure.
The default value is '/var/run/ow'. The selected directory
should be situated on the RAMdisk for reasons of system performance!

\config{OW\_CACHE\_SIZE}{OW\_CACHE\_SIZE}{OWCACHESIZE}
Is used to set the maximum size of the cache in
[bytes] on systems with very little RAM disk.

  The default value of '0' removes any limitation.

\config{OW\_USER\_SCRIPT\_INTERVAL}{OW\_USER\_SCRIPT\_INTERVAL}{OWUSERSCRIPTINTERVAL}
Specifies the waiting time between two runs
of the user-script. The value '0 'should only be used, if
'sleep' is executed in the script.

\config{OW\_DEVICE\_LIB}{OW\_DEVICE\_LIB}{OWDEVICELIB}
Specifies the absolute path and file name of the component library
on the router. By using a value other than the default value
'/srv/www/include/ow-device.lib' it may be ensured that
the component library will not be overwritten when updating
the opt library and personal changes are preserved.

\config{OW\_INVERT\_PORT\_LEDS}{OW\_INVERT\_PORT\_LEDS}{OWINVERTPORTLEDS}
 Inverts the state of the port Leds of i/o ports (latch*,
sensed*, PIO*).

Default setting is 'no'.
\end{description}

\section{Variables Not Documented}

The following variables are not (yet) documented:
\begin{description}
\config{OW\_MODULE\_CONF\_FILE}{OW\_MODULES\_CONF\_FILE}{OWMODULECONFFILE}
\config{OW\_USER\_SCRIPT\_STOP}{|OW\_USER\_SCRIPT\_STOP}{OWUSERSCRIPTSTOP}
\config{OW\_SCRIPT\_WRAPPER}{OW\_SCRIPT\_WRAPPER}{OWSCRIPTWRAPPER}
\config{OW\_MENU\_ITEM}{OW\_MENU\_ITEM}{OWMENUITEM}
\config{OW\_RIGHTS\_SECTION}{OW\_RIGHTS\_SECTION}{OWRIGHTSSECTION}

\config{OW\_OWFS\_PID\_FILE}{OW\_OWFS\_PID\_FILE}{OWOWFSPIDFILE}
\config{OW\_OWFS\_GROUP\_x\_NAME}{OW\_OWFS\_GROUP\_x\_NAME}{OWOWFSGROUPxNAME}
\config{OW\_REFRESH\_FILE}{OW\_REFRESH\_FILE}{OWREFRESHFILE}
\config{OW\_REFRESH\_TEMP}{OW\_REFRESH\_TEMP}{OWREFRESHTEMP}
\config{OW\_ALIAS\_FILE}{OW\_ALIAS\_FILE}{OWALIASFILE}
\config{OW\_CSS\_FILE}{OW\_CSS\_FILE}{OWCSSFILE}

\config{OW\_OWHTTPD\_FAKE}{OW\_OWHTTPD\_FAKE}{OWOWHTTPDFAKE}
\config{OW\_OWHTTPD\_TESTER}{OW\_OWHTTPD\_TESTER}{OWOWHTTPDTESTER}
\config{OW\_OWHTTPD\_PID\_FILE}{OW\_OWHTTPD\_PID\_FILE}{OWOWHTTPDPIDFILE}
\end{description}

\section{Operation In Browser And Console}

\marklabel{cap:OW_BROWSER}
{
\subsection{Browser}
}
\subsubsection{Web Server}
fli4l's optionally installable web server (opt\_httpd) offers the possibility
to execute own shell/CGI script applications from any browser on the network.
This was used here. To use the web server config/httpd.txt must be configured
accordingly.

In \var{OPT\_OW} a browser application is provided. It only will be installed, if
\var{OW\_OWFS}='yes' is set in /config/ow.txt. The script is located at
/srv/www/admin/ow.cgi and under\\ \verb!fli4l-version\opt\files\srv\www\admin\ow.cgi!
in the fli4l installation directory. The menu entry appears at \glqq{}Opt /
1--Wire-Bus\grqq{}.

\subsubsection{Display}
Under the tab \glqq{}Status\grqq{} the componetens connected to the 1--wire bus
are displayed grouped in a tree structure according to the settings in config/ow.txt.
The group is opened by \glq{}clicking\grq{} on it. The values configured will be
shown. In the admin structure all parameters defined for the component in the
component library (see 8.4) are shown. Concerning the meaning of the parameters
see the data sheets of Maxim and the referenced manpages.

Under the tab \glqq{}Admin\grqq{} (only shown in admin mode) the chosen applications
may be switched on or off.

The LEDs shown signal the following states by colors:
\begin{itemize}
\item LED green = not active (Unoperational)
\item LED red = active (Operational)
\item LED yellow = not active (Warning)
\end{itemize}

The control buttons are used to switch the assigned ports. The icon indicates
the current switching status in addition. Regarding the permissions, see 8.1.

\subsection{Console}
The query and control of sensors and actuators is also possible on the console
of the fli4l or via remote access (i.e. WinSCP, Putty).

For example with:
\begin{itemize}
\item cat /var/run/ow/10.DEF0A8010800/temperature \\
      the temperature of a DS19S20 is queried.
\item echo ``1'' $>$ /var/run/ow/1C.7F6CF7040000/PIO.O \\
      output 1 of a DS28E04-100 (dual switch) is activated.
\item echo ``0'' $>$ /var/run/ow/1C.7F6CF7040000/PIO.O \\
      output 1 is deactivated again.
\end{itemize}

Further description can be found in the appendix \glqq{}\ref{cap:OW_MANPAGES}\grqq{} and here: \\
\altlink{http://owfs.org/index.php?page=owfs}

\section{Advanced Features}
\subsection{Assignment Of Rights}
The assignment of user rights is implemented in fli4l's web interface\\
(see the notes in doc/english/pdf/httpd.pdf).\\
\var{OPT\_OW} akes use of this too.
To use the OW rights the following modes can be specified in the file config/httpd.txt
for the area \glqq{}ow\grqq{}:
\begin{itemize}
\item admin = all rights\\
\item exec = execute commands, switch In- and Outputs, View data\\
\item view = view data
\end{itemize}

The Admin tab by which owfs and the user-script may be switched on and off
will not be displayed in the mode \glqq{}exec\grqq{} and \glqq{}view\grqq{}.
All entries containing an \glqq{}Admin\grqq{} will be disabled too.

\subsection{Components Library}
Due to the variety of 1--wire components offered by MAXIM an own component
library was created. The corresponding library script is on the fli4l
in /srv/www/include/ow-device.lib and in the fli4l installation directory under
\verb!fli4l-version\opt\files\srv\www\include\ow-device.lib!. The library already
contains some important components. Own devices can be added according to the
nomenclature used and shared with other fli4l users on the fli4l newsgroups
at 'spline.fli4l.opt'. Only components in the library will be shown in the
browser. The library scripts may be edited as desired either for testing
purposes or as a permanent change on the router itself using programs such
as \glqq{}WinSCP\grqq{} or in the fli4l installation directory.

\marklabel{cap:OW_OWUSERSCRIPT}
{
\subsection{OW\_USER\_SCRIPT}
}

The script can be found on the router under /usr/local/bin/ow-user-script.sh and
in the fli4l installation directory at \verb!fli4l-version\opt\files\usr\local\bin\ow-userscript.sh!.
It can be adapted to reflect your current needs for applications to be monitored
and/or controlled. The advantage of the script is the fact that even large and
complex controls are possible on existing hardware.

\marklabel{cap:OW_RRDTOOL}
{
\section{RRDTool}
}
\subsection{Interface}
The data collected via the 1--Wire-Bus may be recorded and graphically presented by
the fli4l-Opt \glqq{}RRDTool\grqq{} This opt already contains the necessary interfaces.
Owfs (see /config/ow.txt) has to be installed. When installing RRDTool configure the
entries in /config/rrdtool.txt to your needs. For package OW the \var{OW\_SHELL} has
to be set to port 127.0.0.1:4304 because RRDTool's Collectd-Plugin listens on this
port will display the data from all sensors in separate graphics.
Sensors will be sorted by Sensor ID (Unique sensor number of the sensor itself).
In addition, a second group is displayed according to group definitions made in the
package's config file, ordered in the way defined there.
In addition, all temperature sensors in each group will be presented together in a
single graphic ordered by the port number given in the configuration.

\section{Feedback}
We welcome any, even short feedback, even if the package is running without
any problems.

Have a lot of fun with 1--wire!

Klaus der Tiger \email{der.tiger.opt-ow@arcor.de}\\
Karl M. Weckler \email{news4kmw@web.de}\\
Roland Franke \email{fli4l@franke-prem.de}
