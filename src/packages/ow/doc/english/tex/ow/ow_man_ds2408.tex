% Synchronized to r29817
\paragraph*{Name}
DS2408 - 1-Wire 8 Channel Addressable Switch 
\paragraph*{Synopsis}
8 port
switch 

\textbf{29} [.]XXXXXXXXXXXX[XX][/[ \textbf{latch.[0-7$|$ALL$|$BYTE]} $|$ \textbf{LCD\_M/[clear$|$home$|$screen$|$message]}
$|$ \textbf{LCD\_H/[clear$|$home$|$yxscreen$|$screen$|$message$|$onoff]} $|$ \textbf{PIO.[0-7$|$ALL$|$BYTE]} $|$ \textbf{power} $|$ \textbf{sensed.[0-7$|$ALL$|$BYTE]}
$|$ \textbf{strobe} $|$ \textbf{por} $|$ \textbf{set\_alarm} $|$           \textbf{address} $|$ \textbf{crc8} $|$ \textbf{id} $|$ \textbf{locator} $|$ \textbf{r\_address}
$|$ \textbf{r\_id} $|$ \textbf{r\_locator} $|$ \textbf{type}  ]] 
\paragraph*{Family Code}


\textit{29} 
\paragraph*{Special Properties}

\subparagraph*{latch.0 ... latch.7
latch.ALL latch.BYTE}\textit{read-write,} binary 

The 8 pins (PIO) latch a bit when their state changes, either externally,
or through a write to the pin.  

Reading the \textit{latch} property indicates that the latch has been set. 

Writing "true" (non-zero) to ANY  \textit{latch}  will reset them all. (This is the
hardware design). 

\textit{ALL} is all  \textit{latch}  states, accessed simultaneously, comma separated. 

\textit{BYTE} references all channels simultaneously as a single byte. Channel 0
is bit 0. 
\subparagraph*{Pio.0 ...  Pio.7 Pio.all Pio.byte}\textit{read-write,} yes-no 

State of the open-drain output ( \textit{PIO} ) pin. 0 = non-conducting (off), 1 =
\textsf{conducting (on)}. 

Writing zero will turn off the switch, non-zero will turn on the switch.
Reading the \textit{PIO} state will return the switch setting. To determine the actual
logic level at the switch, refer to the \textit{sensed.0} ... sensed.7 sensed.ALL sensed.BYTE
property. 

\textit{ALL} references all channels simultaneously, comma separated. 

\textit{BYTE} references all channels simultaneously as a single byte. Channel 0
is bit 0. 
\subparagraph*{power}\textit{read-only,} yes-no 

Is the \textit{DS2408} powered parasitically (0) or separately on the Vcc \textsf{pin (1)}?

\subparagraph*{sensed.0 ... sensed.7 sensed.ALL}\textit{read-only,} yes-no 

Logic level at the \textit{PIO} pin. 0 = ground. 1 = high (~2.4V - 5V ). Really makes
sense only if the \textit{PIO} state is set to zero (off), else will read zero. 

\textit{ALL} references all channels simultaneously, comma separated. 

\textit{BYTE} references all channels simultaneously as a single byte. Channel 0
is bit 0. 
\subparagraph*{strobe}\textit{read-write,} yes-no 

RSTZ Pin Mode Control. Configures RSTZ as either RST input or STRB output:
\begin{description}
\item [\textit{0} ] configured as RST input (default) 
\item [\textit{1} ] configured as STRB output 
\end{description}



\subparagraph*{por}\textit{read-write,}
yes-no 

Specifies whether the device has performed power-on reset. This bit can only
be cleared to 0 under software control. As long as this bit is 1 the device
will allways respond to a conditional search. 
\subparagraph*{set\_alarm}\textit{read-write,} integer
unsigned (0-333333333) 

A number consisting of 9 digits XYYYYYYYY, where: \begin{description}
\item [X ] select source and logical
term  

\textit{0} PIO   OR  

\textit{1} latch OR  

\textit{2} PIO   AND  

\textit{3} latch AND 
\item [Y ] select channel and polarity 

\textit{0} Unselected (LOW)  

\textit{1} Unselected (HIGH)  

\textit{2} Selected    LOW  

\textit{3} Selected    HIGH 
\end{description}


All digits will be truncated to the 0-3 range. Leading
zeroes are optional. Low-order digit is channel 0. 

Example: \begin{description}
\item [100000033 ] Responds
on Conditional Search when latch.1 or latch.0 are set to 1. 
\item [222000000 ] Responds
on Conditional Search when sensed.7 and sensed.6 are set to 0. 
\item [000000000 (0)
] Never responds to Conditional Search. 
\end{description}

\paragraph*{Lcd\_h Lcd Screen Properites}
This mode
uses the  \textit{DS2408} attached to a Hitachi HD44780 LCD controller in 4-bit mode.
See  \textit{DATASHEET} for published details. Based on a commercial product from
\textit{HobbyBoards} by Erik Vickery. 
\subparagraph*{LCD\_H/clear}\textit{write-only,} yes-no 

This will \textit{clear} the screen and place the cursor at the start. 
\subparagraph*{LCD\_H/home}\textit{write-only,}
yes-no 

Positions the cursor in the \textit{home} (upper left) position, but leaves the
current text intact. 
\subparagraph*{LCD\_H/screen}\textit{write-only,} ascii text 

Writes to the LCD  \textit{screen} at the current position. 
\subparagraph*{LCD\_H/screenyc}\textit{write-only,}
ascii text 

Writes to an LCD screen at a specified location. The controller doesn't know
the true LCD dimensions, but typical selections are: 2x16 2x20 4x16 and
4x20. \begin{description}
\item [Y (row) ] range 1 to 2 (or 4) 
\item [X (column) ] range 1 to 16 (or 20) 
\end{description}


There
are two formats allowed for the \textit{screenyx} text, either ascii (readable text)
or a binary form. \begin{description}
\item [2 binary bytes ] The two first characters of the passed
string have the line and row:  e.g. "$\backslash$x02$\backslash$x04string" perl string writes "string"
at line 2 column 4. 
\item [ascii 2,12: ] Two numbers giving line and row:  Separate
with a comma and end with a colon e.g. "2,4:string" writes "string" at line
2 column 4. 
\item [ascii 12: ] Single column number on the (default) first line:
 End with a colon e.g. "12:string" writes "string" at line 1 column 12. 
\end{description}


The
positions are 1-based (i.e. the first position is 1,1). 
\subparagraph*{LCD\_H/onoff}\textit{write-only,}
unsigned 

Sets several screen display functions. The selected choices should be added
together. \begin{description}
\item [4 ] Display on 
\item [2 ] Cursor on 
\item [1 ] Cursor blinking 
\end{description}

\subparagraph*{LCD\_H/message}\textit{write-only,}
ascii text 

Writes a \textit{message} to the LCD screen after clearing the screen first. This
is the easiest way to display a message. 
\paragraph*{Lcd\_m Lcd Screen Properites}
This
mode uses the  \textit{DS2408} attached to a Hitachi HD44780 LCD controller in 8-bit
mode. See  \textit{DATASHEET} for published details. Based on a design from Maxim
and a commercial product from \textit{AAG.} 
\subparagraph*{LCD\_M/clear}\textit{write-only,} yes-no 

This will \textit{clear} the screen and place the cursor at the start. 
\subparagraph*{LCD\_M/home}\textit{write-only,}
yes-no 

Positions the cursor in the \textit{home} (upper left) position, but leaves the
current text intact. 
\subparagraph*{LCD\_M/screen}\textit{write-only,} ascii text 

Writes to the LCD  \textit{screen} at the current position. 
\subparagraph*{LCD\_M/screenyc}\textit{write-only,}
ascii text 

Writes to an LCD screen at a specified location. The controller doesn't know
the true LCD dimensions, but typical selections are: 2x16 2x20 4x16 and
4x20. \begin{description}
\item [Y (row) ] range 1 to 2 (or 4) 
\item [X (column) ] range 1 to 16 (or 20) 
\end{description}


There
are two formats allowed for the \textit{screenyx} text, either ascii (readable text)
or a binary form. \begin{description}
\item [2 binary bytes ] The two first characters of the passed
string have the line and row:  e.g. "$\backslash$x02$\backslash$x04string" perl string writes "string"
at line 2 column 4. 
\item [ascii 2,12: ] Two numbers giving line and row:  Separate
with a comma and end with a colon e.g. "2,4:string" writes "string" at line
2 column 4. 
\item [ascii 12: ] Single column number on the (default) first line:
 End with a colon e.g. "12:string" writes "string" at line 1 column 12. 
\end{description}


The
positions are 1-based (i.e. the first position is 1,1). 
\subparagraph*{LCD\_M/onoff}\textit{write-only,}
unsigned 

Sets several screen display functions. The selected choices should be added
together. \begin{description}
\item [4 ] Display on 
\item [2 ] Cursor on 
\item [1 ] Cursor blinking 
\end{description}

\subparagraph*{LCD\_M/message}\textit{write-only,}
ascii text 

Writes a \textit{message} to the LCD screen after clearing the screen first. This
is the easiest way to display a message. 
\paragraph*{Standard Properties}
          
\subparagraph*{address}
\subparagraph*{r\_address}\textit{read-only,}
ascii 

The entire 64-bit unique ID. Given as upper case hexidecimal digits (0-9A-F).


\textit{address} starts with the \textit{family} code 

\textit{r} address is the \textit{address} in reverse order, which is often used in other
applications and labeling. 
\subparagraph*{crc8}\textit{read-only,} ascii 

The 8-bit error correction portion. Uses cyclic redundancy check. Computed
from the preceding 56 bits of the unique ID number. Given as upper case
hexidecimal digits (0-9A-F). 
\subparagraph*{family}\textit{read-only,} ascii 

The 8-bit family code. Unique to each \textit{type} of device. Given as upper case
hexidecimal digits (0-9A-F). 
\subparagraph*{id}
\subparagraph*{r\_id}\textit{read-only,} ascii 

The 48-bit middle portion of the unique ID number. Does not include the family
code or CRC. Given as upper case hexidecimal digits (0-9A-F). 

\textit{r} id is the \textit{id} in reverse order, which is often used in other applications
and labeling. 
\subparagraph*{locator}
\subparagraph*{r\_locator}\textit{read-only,} ascii 

Uses an extension of the 1-wire design from iButtonLink company that associated
1-wire physical connections with a unique 1-wire code. If the connection is
behind a \textbf{Link Locator} the \textit{locator} will show a unique 8-byte number (16 character
hexidecimal) starting with family code FE. 

If no \textbf{Link Locator} is between the device and the master, the \textit{locator} field
will be all FF. 

\textit{r} locator is the \textit{locator} in reverse order. 
\subparagraph*{present (DEPRECATED)}\textit{read-only,}
yes-no 

Is the device currently \textit{present} on the 1-wire bus? 
\subparagraph*{type}\textit{read-only,} ascii 

Part name assigned by Dallas Semi. E.g. \textit{DS2401} Alternative packaging (iButton
vs chip) will not be distiguished.  
\paragraph*{Alarms}
Use the \textit{set\_alarm} property to
set the alarm triggering criteria. 
\paragraph*{Description}
          
\subparagraph*{1-Wire}\textit{1-wire}  is a
wiring protocol and series of devices designed and manufactured by Dallas
Semiconductor, Inc. The bus is a low-power low-speed low-connector scheme where
the data line can also provide power. 

Each device is uniquely and unalterably
numbered during manufacture. There are a wide variety of devices, including
memory, sensors (humidity, temperature, voltage, contact, current), switches,
timers and data loggers. More complex devices (like thermocouple sensors)
can be built with these basic devices. There are also 1-wire devices that
have encryption included. 

The 1-wire scheme uses a single  \textit{bus} master and
multiple \textit{slaves} on the same wire. The bus master initiates all communication.
The slaves can be  individually discovered and addressed using their unique
ID. 

Bus masters come in a variety of configurations including serial, parallel,
i2c, network or USB adapters. 
\subparagraph*{OWFS design}\textit{OWFS} is a suite of programs that
designed to make the 1-wire bus and its devices easily accessible. The underlying
priciple is to create a virtual filesystem, with the unique ID being the
directory, and the individual properties of the device are represented
as simple files that can be read and written. 

Details of the individual
slave or master design are hidden behind a consistent interface. The goal
is to  provide an easy set of tools for a software designer to create monitoring
or control applications. There  are some performance enhancements in the
implementation, including data caching, parallel access to bus  masters,
and aggregation of device communication. Still the fundemental goal has
been ease of use, flexibility  and correctness rather than speed.  
\subparagraph*{Ds2408}The
\textsf{\textbf{DS2408 (3)}} allows control of other devices, like LEDs and relays. It extends
the \textbf{DS2406} to 8 channels and includes memory. 

Alternative switches include the \textbf{DS2406, DS2407} and even \textbf{DS2450} 
\paragraph*{Addressing}

         All 1-wire devices are factory assigned a unique 64-bit address.
This address is of the form: \begin{description}
\item [\textbf{Family Code} ] 8 bits 
\item [\textbf{Address} ] 48 bits 
\item [\textbf{CRC} ] 8 bits

\end{description}


\begin{description}
\item [Addressing under OWFS is in hexidecimal, of form: ] 
\item [\textbf{01.123456789ABC} ] 
\end{description}


where
\textbf{01} is an example 8-bit family code, and \textbf{12345678ABC} is an example 48 bit
address. 

The dot is optional, and the CRC code can included. If included,
it must be correct.  
\paragraph*{Datasheet}


\altlink{http://pdfserv.maxim-ic.com/en/ds/DS2408.pdf} 

\altlink{http://www.hobby-boards.com/catalog/howto\_lcd\_driver.php} 

\altlink{http://www.maxim-ic.com/appnotes.cfm/appnote\_number/3286} 
\paragraph*{See Also}

\subparagraph*{Programs}\textsf{\textbf{owfs
(1)} \textsf{owhttpd (1)} \textsf{owftpd (1)} \textsf{owserver (1)}} \textsf{\textbf{owdir (1)} \textsf{owread (1)} \textsf{owwrite (1)}
\textsf{owpresent (1)}} \textsf{\textbf{owtap (1)}} 
\subparagraph*{Configuration and testing}\textsf{\textbf{owfs (5)} \textsf{owtap (1)} \textsf{owmon
(1)}} 
\subparagraph*{Language bindings}\textsf{\textbf{owtcl (3)} \textsf{owperl (3)} \textsf{owcapi (3)}} 
\subparagraph*{Clocks}\textsf{\textbf{DS1427 (3)} \textsf{DS1904(3)}
\textsf{DS1994 (3)} \textsf{DS2404 (3)} \textsf{DS2404S (3)} \textsf{DS2415 (3)} \textsf{DS2417 (3)}} 
\subparagraph*{ID}\textsf{\textbf{DS2401 (3)} \textsf{DS2411
(3)} \textsf{DS1990A (3)}} 
\subparagraph*{Memory}\textsf{\textbf{DS1982 (3)} \textsf{DS1985 (3)} \textsf{DS1986 (3)} \textsf{DS1991 (3)} \textsf{DS1992
(3)} \textsf{DS1993 (3)} \textsf{DS1995 (3)} \textsf{DS1996 (3)} \textsf{DS2430A (3)} \textsf{DS2431 (3)} \textsf{DS2433 (3)}
\textsf{DS2502 (3)} \textsf{DS2506 (3)} \textsf{DS28E04 (3)} \textsf{DS28EC20 (3)}} 
\subparagraph*{Switches}\textsf{\textbf{DS2405 (3)} \textsf{DS2406
(3)} \textsf{DS2408 (3)} \textsf{DS2409 (3)} \textsf{DS2413 (3)} \textsf{DS28EA00 (3)}} 
\subparagraph*{Temperature}\textsf{\textbf{DS1822 (3)}
\textsf{DS1825 (3)} \textsf{DS1820 (3)} \textsf{DS18B20 (3)} \textsf{DS18S20 (3)} \textsf{DS1920 (3)} \textsf{DS1921 (3)} \textsf{DS1821
(3)} \textsf{DS28EA00 (3)} \textsf{DS28E04 (3)} \textsf{EDS0064 (3)} \textsf{EDS0065 (3)} \textsf{EDS0066 (3)} \textsf{EDS0067
(3)} \textsf{EDS0068 (3)} \textsf{EDS0071 (3)} \textsf{EDS0072 (3)}} 
\subparagraph*{Humidity}\textsf{\textbf{DS1922 (3)} \textsf{DS2438 (3)} \textsf{EDS0065
(3)} \textsf{EDS0068 (3)}} 
\subparagraph*{Voltage}\textsf{\textbf{DS2450 (3)}} 
\subparagraph*{Resistance}\textsf{\textbf{DS2890 (3)}} 
\subparagraph*{Multifunction (current,
voltage, temperature)}\textsf{\textbf{DS2436 (3)} \textsf{DS2437 (3)} \textsf{DS2438 (3)} \textsf{DS2751 (3)} \textsf{DS2755
(3)} \textsf{DS2756 (3)} \textsf{DS2760 (3)} \textsf{DS2770 (3)} \textsf{DS2780 (3)} \textsf{DS2781 (3)} \textsf{DS2788 (3)} \textsf{DS2784
(3)}} 
\subparagraph*{Counter}\textsf{\textbf{DS2423 (3)}} 
\subparagraph*{LCD Screen}\textsf{\textbf{LCD (3)} \textsf{DS2408 (3)}} 
\subparagraph*{Crypto}\textsf{\textbf{DS1977 (3)}} 
\subparagraph*{Pressure}\textsf{\textbf{DS2406
(3)} -- TAI8570 \textsf{EDS0066 (3)} \textsf{EDS0068 (3)}}  
\paragraph*{Availability}
\altlink{http://www.owfs.org/} 
\paragraph*{Author}
Paul
Alfille (\email{paul.alfille@gmail.com}) 

