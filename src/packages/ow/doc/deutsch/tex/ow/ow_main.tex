% Last Update: $Id$
\marklabel{sec:ow}
{
\section {OW -- 1--Wire Bus}
}
\setcounter{secnumdepth}{6}

\newcommand{\IsqC}{$\textrm{I}^{\textrm{2}}\textrm{C}$}
\newcommand{\isqc}{$\textrm{i}^{\textrm{2}}\textrm{c}$}

\subsection{Einleitung}

Dieses Paket installiert das OWFS (siehe Kapitel \ref{cap:OW_OWFS}) und bietet so lesenden und
schreibenden Zugriff auf einen an den fli4l kontaktierten 1--Wire Bus. Hierzu wird ein
1--Wire Busmaster an eine serielle Schnittstelle\footnote{DS9097U COM Port
Adapter.} oder einen USB Port\footnote{DS9490R USB Bridge, auch zusammen mit
DS1402D-DR8 (Blue Dot™) für iButton. Alle DS9490 Adapter basieren auf dem
DS2490 USB--1--Wire-Baustein.} des PC angeschlossen.
Darüber hinaus unterstützt das OPT auch \IsqC{} Adapter und die Anbindung
an OWServer. Näheres hierzu in den man-pages im Anhang (Kapitel:
\ref{cap:OW_MANPAGES}). An die 1--Wire-seitige
Buchse des entsprechenden Adapters wird dann der eigentliche 1--Wire Bus angelegt.

\subsubsection{Hardware}
\paragraph{Der 1--Wire Standard}
Der 1--Wire ® bzw. One-Wire oder Eindraht-Bus von Maxim (Maxim/Dallas) beschreibt
eine serielle Schnittstelle, die mit einer Datenader auskommt und sowohl als
Stromversorgung als auch als Sende-und Empfangsleitung genutzt wird. Gleichwohl
ist jedoch eine \glqq{}Rückleitung\grqq{} (GND) erforderlich. Jeder 1--Wire Chip hat eine unikale
Identnummer über die er angesprochen wird. So können mehrere 1--Wire Geräte an
einem einzelnen Bus betrieben werden.

\paragraph{Die 1--Wire-Bauteile}
Maxim bietet eine Vielzahl von 1--Wire-Bauteilen an, als da sind: Serielle-, USB-,
\IsqC-Adapter,
Thermometer, Schalter (bis 8 Kanäle), EEPROMs, Uhren, A/D-Wandler,
digitale Potentiometer. Man bekommt eigentlich alles, was man so für die Hausautomation
braucht. Eine Übersicht über die wichtigsten Bauteile findet sich im Anhang
unter \ref{cap:OW_FAMILYCODE}.
An den Bus können auch iButton ® Bauteile (NV-RAM, EPROM, EEPROM, Temperatur,
Feuchtigkeit, RTC, SHA, Logger) angeschlossen werden.

\paragraph{Der 1--Wire-Bus}
Der 1--Wire-Bus besteht im Prinzip aus zwei verdrillten Leitungen, wobei unter Beachtung
entsprechender Topologien auch längere Strecken bis 150 m kein Problem sein
sollten. Häufig wird für die Verkabelung normales Cat.5 Twisted Pair Ethernet-Kabel
genommen.
Zur Belegung der einzelnen Adern gibt es unterschiedliche Ansätze. Maxim benutzt
6-polige Modular-Buchsen und -Stecker (RJ-11) und hat einen eigenen Standard
kreiert, der jedoch nicht zum 8-poligen RJ-45 Steckzeug passt. Weitere Standards
sind im Anhang (Kapitel \ref{cap:OW_PINADERBELEGUNG}) beschrieben.
Auch zum Thema Topologie des Busses wird man bei Maxim fündig, wie überhaupt
deren Webseite alles Nötige bietet, um gut mit 1--Wire zu recht zu kommen.

\marklabel{cap:OW_OWFS}
{
\subsubsection{OWFS}
}
OWFS steht für \glqq{}One Wire File System\grqq{}. Dabei handelt es sich um eine von Paul H.
Alfille entwickelte Software die unter der GPL lizenziert ist. Auf Grundlage einer 1Wire-
Protokoll verständigen Systembibliothek (OWLib) bildet OWFS den 1--Wire-Bus
als Dateisystem ab. Darüber hinaus bietet das Programm noch weitere Implementierungen,
wie owserver, owshell, owhttpd, owftpd, owtap und Sprachmodule für capi,
perl, tcl, php, die jedoch in der hier vorliegenden Anpassung für fli4l nicht berücksichtigt
wurden. Alles Weitere zu OWFS und viel Interessantes zu 1--Wire findet man
auf:\altlink{ http://owfs.org/} und \altlink{http://sourceforge.net/projects/owfs/}.

\subsubsection{Fuse}
Fuse steht für \glqq{}Filesystem in userspace\grqq{}. Fuse ermöglicht die Implementierung eines
voll funktionsfähigen Dateisystems im Userspace.
Mit der Installation von \var{OPT\_OW} wird das mit fli4l ausgelieferte Fuse automatisch
als Kernelmodul beim Systemstart geladen. Alles weitere zu Fuse findet man auf:
\altlink{http://fuse.sourceforge.net/} und \altlink{http://sourceforge.net/projects/fuse/.}

\subsubsection{libusb}
Bei libusb handelt es sich um eine freie, GPL-lizensierte USB-Bibliothek. Diese wird
benötigt, um über einen USB-Adapter auf den 1--Wire-Bus zuzugreifen.
Alles weitere zu libusb findet man auf: \altlink{http://libusb.sourceforge.net/}

\subsection{Lizenz}
Dieses Programm ist durch die GNU General Public License, Version 2, Juni 1991,
lizenziert und kann unter den angegebenen Bedingungen frei verwendet, vervielfältigt
und verändert werden. Der Text der GNU General Public License ist im Internet
veröffentlicht unter: \altlink{http://www.gnu.org/licenses/gpl.txt}
Eine inoffizielle deutsche Übersetzung findet sich unter:
\altlink{http://www.gnu.de/gpl-ger.html}

Diese Übersetzung soll nur zu einem besseren Verständnis der GPL verhelfen,
rechtsverbindlich ist alleine die englischsprachige Version.

\subsection{Gewährleistungs-und Haftungsausschluss}
Die Veröffentlichung dieses Programms erfolgt mit dem Willen und in der Hoffnung,
dass es von Nutzen sein wird. Dennoch wird jegliche Gewährleistung -auch die implizite
Gewährleistung der Marktreife oder der Eignung für einen bestimmten Zweck abgelehnt.
Details hierzu finden Sie in der GNU General Public License (GPL).
Für Datenverlust, Schäden an Hard-oder Software oder sonstige Schäden wird keine
Haftung übernommen.

\subsection{Systemvoraussetzungen}
Auf Grund der Größe des \var{OPT\_OW} benötigt man eine Festplatte bzw. Flashkarte.
Näheres dazu unter \var{OPT\_HD}. Eine Installation auf FD ist nicht möglich.

Für die Anzeige im Browser ist der im fli4l-Paket \glqq{}httpd\grqq{} enthaltene Mini-Webserver
erforderlich. Weitere Hinweise hierzu unter Kapitel \ref{cap:OW_BROWSER}.

\textbf{Zur Beachtung:}

Die USB Ansteuerung über die W1-Kernelmodule klappt nach Auskunft von Paul Alfille,
dem Maintainer von OWFS, noch nicht und wurde im Opt bisher nicht getestet.
(Testhalber hatte ich (Roland Franke) die W1 Module mit der Version vom OWFS
V2.8 p16 bzw. p19 bereits einmal am laufen, da jedoch die Anbindung und Auswertung 
völlig anders ist, als dies bei der Standardversion der Fall ist, werde ich dies
vermutlich nicht weiter verfolgen)

Für die Verwendung der USB-Adapter muss im System die Angaben des "udev" im "rules.d"
vorhanden sein. Nur wenn diese Einstellungen stimmen, funktioniert auch die Anbindung
im Zusammenhang OWSERVER und OWFS.

Die Verwendung der beiden Programme \emph{owshell} und \emph{owhttpd} funktionierte im Testbetrieb
auf einigen Hardwareumgebungen bisher nicht zufrieden stellend. Im Falle von Fehlern
kann man versucehn mit setaillierter Beschreibung in den Newsgroups für fli4l zu posten.

\subsection{Installation}
Nach dem Entpacken des tar.gz-Archivs in das fli4l-Verzeichnis ist die Textdatei
config/ow.txt zu bearbeiten. Für die Verwendung des Webinterface muss in config/
httpd.txt die Variable \var{OPT\_HTTPD}='yes' gesetzt werden (siehe Kapitel
\ref{cap:OW_SONSTIGEVARIABLEN}). Wird
RRDTool für die Aufzeichnung von Messwerten benötigt, so ist die Konfiguration der
Textdatei config/rrdtool.txt erforderlich (siehe Kapitel \ref{cap:OW_RRDTOOL}).

\subsection{Konfiguration}
Beispielkonfiguration ohne Kommentare, nähere Erläuterungen weiter unten:

\begin{example}
\begin{verbatim}
    OPT_OW='yes'                      # install OPT_OW (yes/no)
    OW_USER_SCRIPT=''                 # e.g. 'usr/local/bin/ow-user-script.sh'

    OW_OWFS='yes'                     # start owfs (yes/no)
    OW_OWFS_DEV='usb'                 # usb*, ttyS*, ip:port, etc.
    OW_OWFS_GROUP_N='4'                           # number of groups
    OW_OWFS_GROUP_1_NAME='1--Wire an USB'         # name of first group
    OW_OWFS_GROUP_1_PORT_N='2'                    # number of ports of device
    OW_OWFS_GROUP_1_PORT_1_ID='81.70D42A000000/ID'      # ID of device
    OW_OWFS_GROUP_1_PORT_1_ALIAS='ID'                   # alias of ID
    OW_OWFS_GROUP_1_PORT_2_ID='81.70D42A000000/Admin/*' # admin-access
    OW_OWFS_GROUP_1_PORT_2_ALIAS='Admin/'               # alias of admin

    OW_OWFS_GROUP_2_NAME='Heizung'
    OW_OWFS_GROUP_2_PORT_N='7'
    OW_OWFS_GROUP_2_PORT_1_ID='3A.F6E401000000/PA'
    OW_OWFS_GROUP_2_PORT_1_ALIAS='1. Umwälzpumpe'
    OW_OWFS_GROUP_2_PORT_2_ID='3A.F6E401000000/PB'
    OW_OWFS_GROUP_2_PORT_2_ALIAS='2. Ladepumpe'
    OW_OWFS_GROUP_2_PORT_3_ID='10.651BA9010800/temp'
    OW_OWFS_GROUP_2_PORT_3_ALIAS='4. Rücklauftemperatur'
    OW_OWFS_GROUP_2_PORT_4_ID='10.DEF0A8010800/temp'
    OW_OWFS_GROUP_2_PORT_4_ALIAS='3. Vorlauftemperatur'
    OW_OWFS_GROUP_2_PORT_5_ID='3A.F6E401000000/Admin/*'
    OW_OWFS_GROUP_2_PORT_5_ALIAS='Admin/Switch-'
    OW_OWFS_GROUP_2_PORT_6_ID='10.DEF0A8010800/Admin/*'
    OW_OWFS_GROUP_2_PORT_6_ALIAS='Admin/VLT-'
    OW_OWFS_GROUP_2_PORT_7_ID='10.651BA9010800/Admin/*'
    OW_OWFS_GROUP_2_PORT_7_ALIAS='Admin/RLT-'

    OW_OWFS_GROUP_3_NAME='Solaranlage'
    OW_OWFS_GROUP_3_PORT_N='3'
    OW_OWFS_GROUP_3_PORT_1_ID='1C.7F6CF7040000/P0'
    OW_OWFS_GROUP_3_PORT_1_ALIAS='1. Ladepumpe'
    OW_OWFS_GROUP_3_PORT_2_ID='1C.7F6CF7040000/P1'
    OW_OWFS_GROUP_3_PORT_2_ALIAS='2. Ventil'
    OW_OWFS_GROUP_3_PORT_3_ID='1C.7F6CF7040000/Admin/*'
    OW_OWFS_GROUP_3_PORT_3_ALIAS='Admin/Switch-'
	
	OW_OWSHELL='yes'
	OW_OWSHELL_RUN='yes'
	OW_OWSHELL_DEV='usb'
	OW_OWSHELL_PORT='127.0.0.1:4304'

	OW_OWHTTPD='yes'
	OW_OWHTTPD_RUN='yes'
	OW_OWHTTPD_DEV='127.0.0.1:4304'
	OW_OWHTTPD_PORT='8080'
\end{verbatim}
\end{example}

Die folgenden Variablen in der Datei /config/ow.txt sind zu konfigurieren:

\begin{description}
\config{OPT\_OW}{OPT\_OW}{OPTOW}
Die Standardeinstellung von OPT\_OW='no', das Paket wird nicht installiert.
Mit \var{OPT\_OW}='yes' wird das Paket aktiviert.

\config{OW\_USER\_SCRIPT}{OW\_USER\_SCRIPT}{OWUSERSCRIPT}
Hinter dieser Variablen verbirgt sich Pfad und Dateiname einer optionalen Hintergrundsteuerung,
mit der zum Beispiel die Heizungsanlage geregelt werden kann.
Nähere Erläuterungen finden sich in Kapitel \ref{cap:OW_OWUSERSCRIPT}.

\config{OW\_OWFS}{OW\_OWFS}{OWOWFS}
OWFS bietet einfachen Zugriff auf den 1--Wire Bus über die fli4l-Weboberfläche. Mit
der Auswahl \var{OW\_OWFS}='yes' wird mittels Fuse ein Dateisystem unter dem Standardpfad
'/var/run/ow' erzeugt. Dort wird der 1--Wire-Bus abgebildet. Die im Dateisystem
angelegten Verzeichnisse sind nach den Identnummern (siehe Anhang \ref{cap:OW_FAMILYCODE}) der
Chips geordnet. Über den family code der Bauteile ist eine entsprechende Systematik
leicht herzustellen.

\config{OW\_OWFS\_DEV}{OW\_OWFS\_DEV}{OWOWFSDEV}
Mit der Variablen \var{OW\_OWFS\_DEV} legt man die PC-Schnittstelle fest, an der der
1--Wire-Adapter angeschlossen wird.

\begin{tabular}{|l|l|p{0.5\textwidth}|}
\hline
\textbf{PC-Schnittstelle} & \textbf{Variablenbelegung} & \textbf{Beispiel} \\
\hline
seriell          & ttyS*          & ttyS0 = COM1, ttyS1 = COM2 \\
\hline
\multirow{3}{*}{}{USB}
                 & ttyUSB*        & ttyUSB1 = erster USB-Adapter \\
\cline{2-3}
          \latex{&} usb           & usb = erster USB-Adapter \\
\cline{2-3}
          \latex{&} usb[2-9]      & usb3 = dritter USB-Adapter \\
\hline
\IsqC{}          & \isqc{}-[0-9]  & \isqc{}-0 = erster \IsqC{}-Port \\
\hline
\multirow{2}{*}{}{Simulation}
                 & fake           & \multirow{2}{*}{}{Für die Verwendung
                                    der Modi '\var{FAKE}'
                                    und '\var{TESTER}' müssen die
                                    Variablen \var{OW\_OWFS\_FAKE} oder
                                    \var{OW\_OWFS\_TESTER} auf gültige family
                                    codes gesetzt werden, siehe Kapitel
                                    \ref{cap:OW_SONSTIGEVARIABLEN}} \\
\cline{2-2}
     \latex{&} tester \latex{&} \\
\hline
\end{tabular}

\config{OW\_OWFS\_GROUP\_N}{OW\_OWFS\_GROUP\_N}{OWOWFSGROUPN}
Die Variable \var{OW\_OWFS\_GROUP\_N} bestimmt die Anzahl der im Browser angezeigten
Gruppen in die zusammen gehörige Ein-und Ausgänge für zum Beispiel die
Steuerung einer Solaranlage zusammen gefasst werden und deren Namen mit
OW\_OWFS\_GROUP\_NAME festgelegt wird.

\configlabel{OW\_OWFS\_GROUP\_x\_PORT\_N}{OWOWFSGROUPxPORTN}
\configlabel{OW\_OWFS\_GROUP\_x\_PORT\_x\_ALIAS}{OWOFSGROUPxPORTxALIAS}
\config{OW\_OWFS\_GROUP\_x\_PORT\_N OW\_OWFS\_GROUP\_x\_PORT\_x\_ID OW\_OWFS\_\_GROUP\_x\_PORT\_x\_ALIAS}{OW\_OWFS\_GROUP\_x\_PORT\_x\_ID}{OWOWFSGROUPxPORTxID}
Die Variable \var{OW\_OWFS\_GROUP\_x\_PORT\_N} bestimmt die Anzahl Ports
einer Gruppe. Mit den beiden untergeordneten Variablen \var{OW\_OWFS\_GROUP\_x\_PORT\_x\_ID} und \\
\var{OW\_OWFS\_GROUP\_x\_PORT\_x\_ALIAS} weist man den jeweiligen Aus-,
bzw. Eingängen des 1--Wire Bauteils einen Decknamen zu.

Möchte man bestimmte Anzeigen im Webinterface unterdrücken, weil zum Beispiel
der Port eines Bauteils nicht belegt wurde oder der Admin-Zweig nach Abschluss der
Konfiguration nicht mehr benötigt wird, dann kann man dem Namen ein Rufzeichen
(!) voranstellen.

\config{OW\_OW\_SHELL}{OW\_OW\_SHELL}{OWOWSHELL}
Aktivierung des "Serverdienstes" vom OWFS, zur Bereitstellung des OWFS-BUS gleichzeitig
für mehrere Anwendungen (OWFS und OWHTTPD). Dabei darf dann keine der anderen
Anwendungen auf die direkte Schnittstelle des Adapters eingestellt sein, sondern
muss hier auf den Server verknüpft werden.
\config{OW\_OW\_SHELL\_RUN}{OW\_OW\_SHELL\_RUN}{OWOWSHELLRUN}
Soll der Serverdienst beim Booten sofort gestartet werden?
\config{OW\_OW\_SHELL\_DEV}{OW\_OW\_SHELL\_DEV}{OWOWSHELLDEV}
Device, auf welches der Server zugreift (Hardware)
\config{OW\_OW\_SHELL\_PORT}{OW\_OW\_SHELL\_PORT}{OWOWSHELLPORT}
IP-Adresse und Port, auf dem der Serverdienst arbeitet.
Sinnvoll ist hier nur die Localhost-Adresse 127.0.0.1
Standardmäßig sollte der Port 4304 (Standard bei OWFS) für den
Serverdienst eingestellt bleiben. Diese Addresse ist fest im Paket
RRDTool hinterlegt. Wenn also Werte per RRDTool erfasst werden sollen,
muss dies so bleiben.

\config{OW\_OWHTTPD}{OW\_OWHTTPD}{OWOWHTTPD}
Aktivierung des vom OWFS selbst bereitgestellten Webservers.
\config{OW\_OWHTTPD\_RUN}{OW\_OWHTTPD\_RUN}{OWOWHTTPDRUN}
Soll der Webserver beim Booten gestartet werden?
\config{OW\_OWHTTPD\_READONLY}{OW\_OWHTTPD\_READONLY}{OWOWHTTPDREADONLY}
Darf ein schreibender Zugriff auf die Bauteile im OWFS über den Webserver erfolgen.
\config{OW\_OWHTTPD\_DEV}{OW\_OWHTTPD\_DEV}{OWOWHTTPDDEV}
Device, auf das der Webserver zugreift. Im Zusammenhang mit OW\_OWSHELL (Server) kann
hier auch gleichzeitig auf ein einzelnes Device zugegriffen werden.
\config{OW\_OWHTTPD\_PORT}{OW\_OWHTTPD\_PORT}{OWOWHTTPDPORT}
HTTP-Port, auf dem der Webserver seinen Dienst bereit stellt.

Konfigurationsbeispiel:
\begin{example}
\begin{verbatim}
    OW_OWFS_GROUP_x_PORT_x_ID='29.57D305000000/P6'
    OW_OWFS_GROUP_x_PORT_x_ALIAS='EA-Modul/!P6'        # Signal unterdrückt
    OW_OWFS_GROUP_x_PORT_x_ID='29.57D305000000/Admin/*'
    OW_OWFS_GROUP_x_PORT_x_ALIAS='EA-Modul/Admin/!'    # Admin-Pfad komplett
                                                       # abgeschaltet
\end{verbatim}
\end{example}

Eine weitergehende Beschreibung zur Konfiguration von OWFS gibt es im Anhang
\glqq{}\ref{cap:OW_MANPAGES}\grqq{} und hier: \altlink{http://owfs.org/index.php?page=owfs}.

\end{description}

\marklabel{cap:OW_SONSTIGEVARIABLEN}
{
\subsubsection{Sonstige Variablen}
}
Die folgenden Variablen können bei Bedarf über die Datei config/ow.txt angepasst
werden:

\begin{description}
\config{OW\_LOG\_DESTINATION}{OW\_LOG\_DESTINATION}{OWLOGDESTINATION}
 Ziel für Status-und Fehlerausgaben.

\begin{verbatim}
    0 = mixed (1 und 2)
    1 = syslog
    2 = stderr
    3 = off
\end{verbatim}

Standardwert ist '1'.

\config{OW\_LOG\_LEVEL}{OW\_LOG\_LEVEL}{OWLOGLEVEL}
 Der Log-Level (1-9) bestimmt die Anzahl der Status-und Fehlerausgaben, wobei:

\begin{verbatim}
    1 = leise und 9 = geschwätzig
\end{verbatim}

Standardwert ist '1'.

\config{OW\_TEMP\_SCALE}{OW\_TEMP\_SCALE}{OWTEMPSCALE}
 Die zur Verfügung stehenden Temperaturskalen.
\begin{verbatim}
    C = "Celsius"
    F = "Fahrenheit"
    K = "Kelvin"
    R = "Rankine"
\end{verbatim}

Standardwert ist 'C'.

\config{OW\_REFRESH\_INTERVAL}{OW\_REFRESH\_INTERVAL}{OWREFRESHINTERVAL}
 Refresh der fli4l-HTTP-Anwendung in Sekunden. '0' = kein Refresh.

Standardwert ist '10'.

\config{OW\_OWFS\_FAKE}{OW\_OWFS\_FAKE}{OWOWFSFAKE}
 Aktiviert die zufällige Simulation von 1--Wire-Bauteilen.
Es können mehrere Bauteile mit dem family code und
durch Leerzeichen getrennt angegeben werden. Die
simulierten Zustände sind rein zufällig. Die Option kann
nicht gleichzeitig mit dem 'TESTER'-Modus aktiviert werden.

\config{OW\_OWFS\_TESTER}{OW\_OWFS\_TESTER}{OWOWFSTESTER}
 Aktiviert die systematische Simulation von 1--Wire-
Bauteilen. Es können mehrere Bauteile mit dem family
code und durch Leerzeichen getrennt angegeben werden.
Die simulierten Zustände folgen natürlichen Werten.
Die Option kann nicht gleichzeitig mit dem 'FAKE'-Modus
aktiviert werden.

\config{OW\_OWFS\_RUN}{OW\_OWFS\_RUN}{OWOWFSRUN}
 Gibt an, ob owfs beim Boot des Routers automatisch
gestartet werden soll. Standardwert ist 'yes', wohingegen
nach dem Setzen von 'no' die Anwendung manuell gestartet
werden muss.

\config{OW\_OWFS\_READONLY}{OW\_OWFS\_READONLY}{OWOWFSREADONLY}
 Legt mit 'yes' fest, dass Bauteilzustände über owfs nur
gelesen, aber nicht geschrieben werden können.

Standardwert ist 'no'.

\config{OW\_OWFS\_PATH}{OW\_OWFS\_PATH}{OWOWFSPATH}
 Gibt das Wurzelverzeichnis der Fuse-Verzeichnisstruktur
an. Standardwert ist '/var/run/ow'. Das gewählte Verzeichnis
sollte aus Gründen der Systemleistung unbedingt
auf der RAMdisk liegen!

\config{OW\_CACHE\_SIZE}{OW\_CACHE\_SIZE}{OWCACHESIZE}
 Dient zur Anpassung der maximalen Größe des cache in
[bytes] auf Systemen mit sehr kleiner RAMdisk.

 Der Standardwert '0' hebt jegliche Limitierung auf.

\config{OW\_USER\_SCRIPT\_INTERVAL}{OW\_USER\_SCRIPT\_INTERVAL}{OWUSERSCRIPTINTERVAL}
 Gibt in Sekunden an, wie lange zwischen zwei Durchläufen
des user-script gewartet wird. Der Wert '0' sollte nur
verwendet werden, wenn innerhalb des user-script ein
'sleep' ausgeführt wird.

\config{OW\_DEVICE\_LIB}{OW\_DEVICE\_LIB}{OWDEVICELIB}
 Legt den absoluten Pfad und Dateinamen der Bauteilbibliothek
auf dem Router fest. Durch die Verwendung eines anderen Werts als des Standardwerts
'/srv/www/include/ow-device.lib' kann sichergestellt werden,
dass bei einer Aktualisierung des Opt die Bibliothek
nicht überschrieben wird und so persönliche Änderungen
an der Bauteilbibliothek erhalten bleiben.

\config{OW\_INVERT\_PORT\_LEDS}{OW\_INVERT\_PORT\_LEDS}{OWINVERTPORTLEDS}
 Invertiert den Status der Port-Leds von i/o Ports (latch*,
sensed*, PIO*).

Standardwert ist 'no'.
\end{description}

\subsubsection{Nicht dokumentierte Variablen}

Die folgende Variablen sind nicht dokumentiert:
\begin{description}
\config{OW\_MODULE\_CONF\_FILE}{OW\_MODULES\_CONF\_FILE}{OWMODULECONFFILE}
\config{OW\_USER\_SCRIPT\_STOP}{|OW\_USER\_SCRIPT\_STOP}{OWUSERSCRIPTSTOP}
\config{OW\_SCRIPT\_WRAPPER}{OW\_SCRIPT\_WRAPPER}{OWSCRIPTWRAPPER}
\config{OW\_MENU\_ITEM}{OW\_MENU\_ITEM}{OWMENUITEM}
\config{OW\_RIGHTS\_SECTION}{OW\_RIGHTS\_SECTION}{OWRIGHTSSECTION}

\config{OW\_OWFS\_PID\_FILE}{OW\_OWFS\_PID\_FILE}{OWOWFSPIDFILE}
\config{OW\_OWFS\_GROUP\_x\_NAME}{OW\_OWFS\_GROUP\_x\_NAME}{OWOWFSGROUPxNAME}
\config{OW\_REFRESH\_FILE}{OW\_REFRESH\_FILE}{OWREFRESHFILE}
\config{OW\_REFRESH\_TEMP}{OW\_REFRESH\_TEMP}{OWREFRESHTEMP}
\config{OW\_ALIAS\_FILE}{OW\_ALIAS\_FILE}{OWALIASFILE}
\config{OW\_CSS\_FILE}{OW\_CSS\_FILE}{OWCSSFILE}

\config{OW\_OWHTTPD\_FAKE}{OW\_OWHTTPD\_FAKE}{OWOWHTTPDFAKE}
\config{OW\_OWHTTPD\_TESTER}{OW\_OWHTTPD\_TESTER}{OWOWHTTPDTESTER}
\config{OW\_OWHTTPD\_PID\_FILE}{OW\_OWHTTPD\_PID\_FILE}{OWOWHTTPDPIDFILE}
\end{description}

\subsection{Bedienung im Browser und auf der Konsole}

\marklabel{cap:OW_BROWSER}
{
\subsubsection{Browser}
}
\paragraph{Webserver}
Der in fli4l optional zu installierende Webserver (opt\_httpd) bietet die Möglichkeit,
eigene Shell/CGI Skript Anwendungen über jeden Browser im Netz zu bedienen. Davon
wurde im vorliegenden Falle Gebrauch gemacht. Um den Webserver zu nutzen
muss config/httpd.txt entsprechend konfiguriert werden.

Im \var{OPT\_OW} wird eine Browser Applikation mitgeliefert. Sie wird nur installiert, wenn
in /config/ow.txt \var{OW\_OWFS}='yes' gesetzt wird.
Das zugehörige Skript befindet sich gemäß Vorgaben auf dem fli4l unter
/srv/www/admin/ow.cgi und im fli4l-Installationsverzeichnis unter
\verb!fli4l-version\opt\files\srv\www\admin\ow.cgi!. Der zugehörige Menüpunkt erscheint unter \glqq{}Opt /
1--Wire-Bus\grqq{}.

\paragraph{Darstellung}
Im Reiter \glqq{}Status\grqq{} werden die am 1--Wire-Bus angeschlossenen Bauteile gemäß der
in config/ow.txt vorgenommenen Konfiguration gruppenweise in einer Baumstruktur
angezeigt. Mittels \glq{}Klick\grq{} wird die jeweilige Gruppe geöffnet. Die konfigurierten Werte
werden angezeigt. In der Admin-Struktur werden alle in der Bauteilbibliothek (siehe
8.4) für das jeweilige Bauteil definierten Parameter angezeigt. Hinsichtlich der Bedeutung
dieser Parameter wird auf die Datenblätter von Maxim und die beiliegenden
Man-Pages verwiesen.

Im Reiter \glqq{}Admin\grqq{}, der nur im Admin-Modus angezeigt wird, können die gewählten
Applikationen ein-und ausgeschaltet werden.

Die angezeigten LEDs signalisieren mit ihren Farben folgende Zustände:
LED grün = inaktiv (Ruhe)
LED rot = aktiv (Betrieb)
LED gelb = inaktiv (Warnung)


Die Schaltknöpfe dienen zum Umschalten der zugeordneten Ports. Das Symbol zeigt
zusätzlich den aktuellen Schaltzustand. Hinsichtlich der Berechtigungen siehe 8.1.


\subsubsection{Konsole}
Die Abfrage und Steuerung der Sensoren und Aktoren ist auch auf der Konsole des
fli4l oder über einen Remotezugriff (WinSCP, Putty o.ä.) möglich.

Mit z.B.:
\begin{itemize}
\item cat /var/run/ow/10.DEF0A8010800/temperature \\
      ruft man die Temperaturmessung eines DS19S20 ab.
\item echo \dq{}1\dq{} $>$ /var/run/ow/1C.7F6CF7040000/PIO.O \\
      schaltet man den Ausgang 1 eines DS28E04-100 (dual switch) ein.
\item echo \dq{}0\dq{} $>$ /var/run/ow/1C.7F6CF7040000/PIO.O \\
      schaltet man den Ausgang 1 wieder aus.
\end{itemize}

Eine weitere Beschreibung gibt es im Anhang \glqq{}\ref{cap:OW_MANPAGES}\grqq{} und hier: \\
\altlink{http://owfs.org/index.php?page=owfs}


\subsection{Erweiterte Funktionen}
\subsubsection{Rechtevergabe}
In der fli4l-Webanwendung ist die Vergabe von Benutzerrechten implementiert, siehe
hierzu die Erläuterungen in doc/deutsch/pdf/httpd.pdf. Dieser Rechtevergabe bedient
sich auch \var{OPT\_OW}. Zur Nutzung der OW-Rechte können in der Datei config/
httpd.txt für den Bereich \glqq{}ow\grqq{} die folgenden Stufen vorgegeben werden:

admin = alle Rechte\\
exec = Befehle ausführen, Ein-und Ausgänge schalten, Ansicht der Daten\\
view = Ansicht der Daten

Die Admin-Tabelle, über die owfs und das user-script ein-und ausgeschaltet werden
können, wird in den Stufen \glqq{}exec\grqq{} und \glqq{}view\grqq{} nicht angezeigt. Weiterhin werden in
diesen Berechtigungsstufen alle Anzeigen, die ein \glqq{}Admin\grqq{} enthalten, ausgeblendet.

\subsubsection{Bauteilebibliothek}
Auf Grund der Vielzahl der von MAXIM angebotenen 1--Wire Bauteile wurde eine eigene
Bauteilebibliothek angelegt. Das entsprechende Library-script liegt auf dem
fli4l unter /srv/www/include/ow-device.lib und im fli4l-Installationsverzeichnis unter
\verb!fli4l-version\opt\files\srv\www\include\ow-device.lib!. Die Bibliothek enthält bereits einige
wichtige Bauteile. Eigene Devices können entsprechend der verwendeten Systematik
nachgetragen und über die fli4l-Newsgroup 'spline.fli4l.opt' den übrigen
fli4l-Nutzern zur Verfügung gestellt werden. Es werden nur in der Bibliothek angelegte
Bauteile im Browser angezeigt. Das Libraryscript kann wunschgemäß entweder
mittels Programmen wie \glqq{}WinSCP\grqq{} auf dem fli4l selbst zu Testzwecken oder als
dauerhafte Änderung im fli4l-Installationsverzeichnis editiert werden.

\marklabel{cap:OW_OWUSERSCRIPT}
{
\subsubsection{OW\_USER\_SCRIPT}
}
Das Skript findet sich auf dem fli4l unter /usr/local/bin/ow-user-script.sh und im
fli4l-Installationsverzeichnis unter \verb!fli4l-version\opt\files\usr\local\bin\ow-userscript.sh!.
Es kann nach eigenen Wünschen und Bedürfnissen an die zu überwachenden
und zu steuernden Anwendungen angepasst werden. Der Vorteil des Skript
ist darin zu sehen, dass auch umfangreiche und komplexe Steuerungen auf bestehender
Hardware möglich sind.

\marklabel{cap:OW_RRDTOOL}
{
\subsubsection{RRDTool}
}
\paragraph{Schnittstelle}
Die über den 1--Wire-Bus erfassten Daten können mittels fli4l-Opt \glqq{}RRDTool\grqq{} aufgezeichnet
und graphisch aufbereitet werden. Das vorliegende Opt bringt die erforderlichen
Schnittstellen bereits mit. Hierzu muss owfs (siehe /config/ow.txt) installiert
sein. Bei der Installation von RRDTool sind in der /config/rrdtool.txt die gewünschten
Einträge vorzunehmen. Dabei ist es zwingend erforderlich, dass beim Paket OW das
\var{OW\_SHELL} auf den Port 127.0.0.1:4304 aktiviert wird, das dass Collectd-Plugin von
RRDTool auf diesen Port lauscht und die Daten aller Sensoren in getrennte Grafiken
abbildet.
Im RRDTool werden zum einen alle Sensoren die gefunden werden direkt erfasst und in 
separaten Grafiken dargestellt. Hierbei erfolgt die Sortierung der Sensoren gemäß der
Sensor-ID (Einmalige Sensornummer des Sensors selbst).
Zusätzlich erfolgt nun die Darstellung in einer zweiten Gruppe gemäß der im Paket
vergebenen Gruppen. Die Reihenfolge entspricht der Reihenfolge in der Konfiguration.
Zusätzlich werden alle Temperatursensoren pro Gruppe in einer einzigen Grafik zusammen-
gefasst und in der Reihenfolge dargestellt, wie diese in der Konfiguration gemäß
der Portnummer vorgegeben wurden.

\subsection{Feedback}
Wir freuen uns über jede, auch kurze Rückmeldung, selbst wenn das Paket ohne
jegliche Probleme laufen sollte.

Viel Spaß mit 1--Wire wünschen:

Klaus der Tiger \email{der.tiger.opt-ow@arcor.de}\\
Karl M. Weckler \email{news4kmw@web.de}\\
Roland Franke \email{fli4l@franke-prem.de}
