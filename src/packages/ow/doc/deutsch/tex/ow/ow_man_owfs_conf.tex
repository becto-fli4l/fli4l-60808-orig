% Last Update: $Id$
\paragraph*{Name}
\textbf{owfs.conf} - owfs programs configuration file 
\paragraph*{Synopsis}
An OWFS
configuration file is specified on the command line: \begin{description}
\item [\textbf{owfs -c config\_file
[other options]} ] The file name is arbitrary, there is no default configuration
file used. 
\end{description}

\paragraph*{Usage}
A configuration file can be invoked for any of the OWFS
programs ( \textsf{\textbf{owfs (1)} \textsf{owhttpd (1)} \textsf{owserver (1)} \textsf{owftpd (1)}} ) or any of the
language bindings ( \textsf{\textbf{owperl (1)} \textsf{owcapi (1)} \textsf{owtcl (1)} owphp owpython} ) to
set command line parameters. 
\paragraph*{Syntax}


Similar to Unix shell script or perl
syntax \begin{description}
\item [Comments ] \# Any  \textit{\#} marks the start of a comment 

\# blank lines are ignored  
\item [Options ] \textbf{option } \# some options (like 'foreground')
take no values 

\textbf{option = value } \# other options need a value 

\textbf{option value   } \# '=' can be omitted if whitespace separates 

\textbf{Option} \# Case is ignored (for options, not values) 

\textbf{opt            } \# non-ambiguous abbreviation allowed 

\textbf{-opt --opt     } \# hyphens ignored 
\item [\textit{owserver} ] \textbf{server: } opt = value \# only \textit{owserver}
effected by this line 

\textbf{! server: } opt = value \#  \textit{owserver} NOT effected by this line 
\item [\textit{owhttpd} ] \textbf{http:
} opt = value \# only \textit{owhttpd} effected by this line 

\textbf{! http: } opt = value \#  \textit{owhttpd} NOT effected by this line 
\item [\textit{owftpd} ] \textbf{ftp: }
opt = value \# only \textit{owftpd} effected by this line 

\textbf{! ftp: } opt = value \#  \textit{owftpd} NOT effected by this line 
\item [\textit{owfs} ] \textbf{owfs: } opt
= value \# only \textit{owfs} effected by this line 

\textbf{! owfs: } opt = value \#  \textit{owfs} NOT effected by this line 
\item [Limits ] \# maximum
line length of 250 characters 

\# no limit on number of lines 
\end{description}

\paragraph*{Description}
          
\subparagraph*{1-Wire}\textit{1-wire}  is a wiring
protocol and series of devices designed and manufactured by Dallas Semiconductor,
Inc. The bus is a low-power low-speed low-connector scheme where the data line
can also provide power. 

Each device is uniquely and unalterably numbered
during manufacture. There are a wide variety of devices, including memory,
sensors (humidity, temperature, voltage, contact, current), switches, timers
and data loggers. More complex devices (like thermocouple sensors) can be
built with these basic devices. There are also 1-wire devices that have encryption
included. 

The 1-wire scheme uses a single  \textit{bus} master and multiple \textit{slaves}
on the same wire. The bus master initiates all communication. The slaves
can be  individually discovered and addressed using their unique ID. 

Bus
masters come in a variety of configurations including serial, parallel,
i2c, network or USB adapters. 
\subparagraph*{OWFS design}\textit{OWFS} is a suite of programs that
designed to make the 1-wire bus and its devices easily accessible. The underlying
priciple is to create a virtual filesystem, with the unique ID being the
directory, and the individual properties of the device are represented
as simple files that can be read and written. 

Details of the individual
slave or master design are hidden behind a consistent interface. The goal
is to  provide an easy set of tools for a software designer to create monitoring
or control applications. There  are some performance enhancements in the
implementation, including data caching, parallel access to bus  masters,
and aggregation of device communication. Still the fundemental goal has
been ease of use, flexibility  and correctness rather than speed.  
\subparagraph*{Configuration}\textsf{\textbf{owfs.conf
(5)}} allows a uniform set of command line parameters to be set. 

Not all OWFS
programs use the same command line options, but the non-relevant ones will
be ignored. 

Command line and configuration options can mixed. They will be
invoked in the order presented. Left to right for the command line. Top to
bottom for the configuration file. 

Configuration files can call other configuration
files. There is an arbitrary depth of 5 levels to prevent infinite loops.
More than one configuration file can be specified. 
\paragraph*{Sample}
\begin{description}
\item [Here is a sample
configuration file with all the possible parameters included. ] \# \textbf{Sources}


\textit{device} = /dev/ttyS0 \# serial port: DS9097U DS9097 ECLO or LINK 

\textit{device} = /dev/i2c-0 \# i2c port: DS2482-100 or DS2482-800 

\textit{usb} \#       USB device: DS9490 PuceBaboon 

\textit{usb} = 2 \#   Second DS9490 

\textit{usb} = all \# All DS9490s 

\textit{altUSB} \# Willy Robison's tweaks 

\textit{LINK} = /dev/ttyS0 \#     serial LINK in ascii mode 

\textit{LINK} = [address:]port \# LINK-HUB-E (tcp access) 

\textit{HA7} \# HA7Net autodiscovery mode 

\textit{HA7} = address[:port] \# HA7Net at tcp address (port 80) 

\textit{etherweather} = address[:port] \# Etherweather device 

\textit{server} = [address:]port \# \textbf{owserver} tcp address 

\textit{FAKE} = 10,1B \# Random simulated device with family codes (hex) 

\textit{TESTER} = 28,3E \# Predictable simulated device with family codes 

\# 

\# \textbf{Sinks} 

\# \# \textbf{owfs} specific 

\textit{mountpoint} = filelocation \# \textit{FUSE} mount point 

\textit{allow\_other} \# Short hand for \textit{FUSE} mount option " 

\# \# \textbf{owhttpd owserver owftpd} specific 

\textit{port} = [address:]port \# tcp out port 

\# 

\# \textbf{Temperature scales} 

\textit{Celsius} \# default 

\textit{Fahrenheit} 

\textit{Kelvin} 

\textit{Rankine} 

\# 

\# \textbf{Timeouts (all in seconds)} 

\#                    cache for values that change on their own 

\textit{timeout\_volatile} = value \# seconds "volatile" values remain in cache 

\#                    cache for values that change on command 

\textit{timeout\_stable} = value \# seconds "stable" values remain in cache 

\#                    cache for directory lists (non-alarm) 

\textit{timeout\_directory} = value \# seconds "directory" values remain in cache


\#                    cache for 1-wire device location 

\textit{timeout\_presence} = value \# seconds "device presence" (which bus) 

\textit{timeout\_serial} = value \# seconds to wait for serial response 

\textit{timeout\_usb} = value \# seconds to wait for USB response 

\textit{timeout\_network} = value \# seconds to wait for tcp/ip response 

\textit{timeout\_ftp} = value \# seconds inactivity before closing ftp session 

\# 

\# \textbf{Process control} 

\textit{configuration} = filename \# file (like this) of program options 

\textit{pid\_file} = filename \# file to store PID number 

\textit{foreground} 

\textit{background} \# default 

\textit{readonly} \# prevent changing 1-wire device contents 

\textit{write} \# default 

\textit{error\_print} = 0-3 \# 0-mixed 1-syslog 2-stderr 3-suppressed 

\textit{error\_level} = 0-9 \# increasing noise 

\# 

\# \textbf{zeroconf / Bonjour} 

\textit{zero} \#   turn on zeroconf announcement (default) 

\textit{nozero} \#   turn off zeroconf announcement 

\textit{annouce} = name  \# name of announced service (optional) 

\textit{autoserver} \#   Add owservers descovered by zeroconf/Bonjour 

\textit{noautoserver} \#   Don't use zeroconf/Bonjour owservers (default) 

\# 

\# \textbf{tcp persistence} 

\textit{timeout\_persistent\_low} = 600 \# minimum time a persistent socket will stay
open 

\textit{timeout\_persistent\_high} = 3600 \# max time an idle client socket will stay
around 

\textit{} 

\# 

\# \textbf{Display} 

\textit{format} = f[.]i[[.]c] \# 1-wire address \textit{f} amily \textit{i} d code \textit{c} rc 

\# 

\# \textbf{Cache} 

\textit{cache\_size} = 1000000 \# maximum cache size (in bytes) or 0 for no limit
(default 0) \# 

\# \textbf{Information} 

\# (silly in a configuration file) 

\textit{version} 

\textit{help} 

\textit{morehelp} 
\end{description}

\paragraph*{See Also}

\subparagraph*{Programs}\textsf{\textbf{owfs (1)} \textsf{owhttpd (1)} \textsf{owftpd (1)} \textsf{owserver (1)}} \textsf{\textbf{owdir
(1)} \textsf{owread (1)} \textsf{owwrite (1)} \textsf{owpresent (1)}} \textsf{\textbf{owtap (1)}} 
\subparagraph*{Configuration and testing}\textsf{\textbf{owfs
(5)} \textsf{owtap (1)} \textsf{owmon (1)}} 
\subparagraph*{Language bindings}\textsf{\textbf{owtcl (3)} \textsf{owperl (3)} \textsf{owcapi (3)}}

\subparagraph*{Clocks}\textsf{\textbf{DS1427 (3)} \textsf{DS1904(3)} \textsf{DS1994 (3)} \textsf{DS2404 (3)} \textsf{DS2404S (3)} \textsf{DS2415 (3)}
\textsf{DS2417 (3)}} 
\subparagraph*{ID}\textsf{\textbf{DS2401 (3)} \textsf{DS2411 (3)} \textsf{DS1990A (3)}} 
\subparagraph*{Memory}\textsf{\textbf{DS1982 (3)} \textsf{DS1985
(3)} \textsf{DS1986 (3)} \textsf{DS1991 (3)} \textsf{DS1992 (3)} \textsf{DS1993 (3)} \textsf{DS1995 (3)} \textsf{DS1996 (3)} \textsf{DS2430A
(3)} \textsf{DS2431 (3)} \textsf{DS2433 (3)} \textsf{DS2502 (3)} \textsf{DS2506 (3)} \textsf{DS28E04 (3)} \textsf{DS28EC20 (3)}}

\subparagraph*{Switches}\textsf{\textbf{DS2405 (3)} \textsf{DS2406 (3)} \textsf{DS2408 (3)} \textsf{DS2409 (3)} \textsf{DS2413 (3)} \textsf{DS28EA00
(3)}} 
\subparagraph*{Temperature}\textsf{\textbf{DS1822 (3)} \textsf{DS1825 (3)} \textsf{DS1820 (3)} \textsf{DS18B20 (3)} \textsf{DS18S20 (3)}
\textsf{DS1920 (3)} \textsf{DS1921 (3)} \textsf{DS1821 (3)} \textsf{DS28EA00 (3)} \textsf{DS28E04 (3)}} 
\subparagraph*{Humidity}\textsf{\textbf{DS1922
(3)}} 
\subparagraph*{Voltage}\textsf{\textbf{DS2450 (3)}} 
\subparagraph*{Resistance}\textsf{\textbf{DS2890 (3)}} 
\subparagraph*{Multifunction (current, voltage,
temperature)}\textsf{\textbf{DS2436 (3)} \textsf{DS2437 (3)} \textsf{DS2438 (3)} \textsf{DS2751 (3)} \textsf{DS2755 (3)} \textsf{DS2756
(3)} \textsf{DS2760 (3)} \textsf{DS2770 (3)} \textsf{DS2780 (3)} \textsf{DS2781 (3)} \textsf{DS2788 (3)} \textsf{DS2784 (3)}} 
\subparagraph*{Counter}\textsf{\textbf{DS2423
(3)}} 
\subparagraph*{LCD Screen}\textsf{\textbf{LCD (3)} \textsf{DS2408 (3)}} 
\subparagraph*{Crypto}\textsf{\textbf{DS1977 (3)}} 
\subparagraph*{Pressure}\textsf{\textbf{DS2406 (3)} -- TAI8570}
 
\paragraph*{Availability}
\altlink{http://www.owfs.org/} 
\paragraph*{Author}
Paul Alfille (\email{paul.alfille@gmail.com})
