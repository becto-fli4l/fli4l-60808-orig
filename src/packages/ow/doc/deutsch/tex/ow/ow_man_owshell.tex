% Last Update: $Id$
\paragraph*{Name}
owdir, owread, owwrite, owget, owpresent - lightweight owserver
access 
\paragraph*{Synopsis}
\textbf{owdir} \textit{-s}  [host:]port [directory] 

\textbf{owread} \textit{-s}  [host:]port filepath 

\textbf{owwrite} \textit{-s}  [host:]port filepath value 

\textbf{owget} \textit{-s}  [host:]port [directory] $|$ filepath 



\textbf{owdir} \textit{--autoserver} [directory] 

\textbf{owread} \textit{--autoserver} filepath 

\textbf{owwrite} \textit{--autoserver} filepath value 

\textbf{owget} \textit{--autoserver} [directory] $|$ filepath 

\textbf{owdir} \textit{-f} --format f[.]i[[.]c] ] [ \textit{--dir}
] \textit{-s}  [host:]port [directory] [directory2 ...] 

\textbf{owread} \textit{-C} --celsius \textit{-K} --kelvin \textit{-F}
--fahrenheit \textit{-R} --rankine [ \textit{--hex} ] [ \textit{--start=} offset ] [ \textit{--size=} bytes ] \textit{-s}  [host:]port
filepath [filepath2 ...] 

\textbf{owwrite} \textit{-C} --celsius \textit{-K} --kelvin \textit{-F} --fahrenheit \textit{-R} --rankine [ \textit{--hex} ] [ \textit{--start=} offset
] \textit{-s}  [host:]port filepath value [filepath2 value2 ...] 

\textbf{owget} \textit{-f} --format f[.]i[[.]c]
\textit{-C} --celsius \textit{-K} --kelvin \textit{-F} --fahrenheit \textit{-R} --rankine [ \textit{--hex} ] [ \textit{--start=} offset ] [ \textit{--size=}
bytes ] [ \textit{--dir} ] \textit{-s}  [host:]port [directory] $|$ filepath 

\textbf{owdir} \textit{-V} --version 

\textbf{owread} \textit{-V} --version 

\textbf{owwrite} \textit{-V} --version 

\textbf{owget} \textit{-V} --version 



\textbf{owdir} \textit{-h} $|$ --help 

\textbf{owread} \textit{-h} $|$ --help 

\textbf{owwrite} \textit{-h} $|$ --help 

\textbf{owget} \textit{-h} $|$ --help 


\paragraph*{Description}
          
\subparagraph*{1-Wire}\textit{1-wire}  is a wiring protocol and series of devices
designed and manufactured by Dallas Semiconductor, Inc. The bus is a low-power
low-speed low-connector scheme where the data line can also provide power.


Each device is uniquely and unalterably numbered during manufacture. There
are a wide variety of devices, including memory, sensors (humidity, temperature,
voltage, contact, current), switches, timers and data loggers. More complex
devices (like thermocouple sensors) can be built with these basic devices.
There are also 1-wire devices that have encryption included. 

The 1-wire scheme
uses a single  \textit{bus} master and multiple \textit{slaves} on the same wire. The bus
master initiates all communication. The slaves can be  individually discovered
and addressed using their unique ID. 

Bus masters come in a variety of configurations
including serial, parallel, i2c, network or USB adapters. 
\subparagraph*{OWFS design}\textit{OWFS}
is a suite of programs that designed to make the 1-wire bus and its devices
easily accessible. The underlying priciple is to create a virtual filesystem,
with the unique ID being the directory, and the individual properties of
the device are represented as simple files that can be read and written.


Details of the individual slave or master design are hidden behind a consistent
interface. The goal is to  provide an easy set of tools for a software designer
to create monitoring or control applications. There  are some performance
enhancements in the implementation, including data caching, parallel access
to bus  masters, and aggregation of device communication. Still the fundemental
goal has been ease of use, flexibility  and correctness rather than speed.
 
\subparagraph*{OWSHELL programs}\textbf{owdir owread owwrite } and  \textbf{owget} are collectively called
the \textbf{owshell} programs. They allow lightweight access to an \textsf{\textbf{owserver (1)}} for
use in command line scripts. 

Unlike \textsf{\textbf{owserver (1)} \textsf{owhttpd (1)} \textsf{owftpd (1)}
\textsf{owhttpd (1)}} there is not persistent connection with the 1-wire bus, no caching
and no multithreading. Instead, each program connects to a running \textsf{\textbf{owserver
(1)}} and performs a quick set of queries. 

\textsf{\textbf{owserver (1)}} performs the actual
1-wire connection (to physical 1-wire busses or other \textbf{owserver} programs),
performs concurrency locking, caching, and error collection. 

\textbf{owshell} programs
are intended for use in command line scripts. An alternative approach is
to mount an \textsf{\textbf{owfs (1)}} filesystem and perform direct file lists, reads and
writes. 
\subparagraph*{owdir}\textbf{owdir} performs a \textit{directory} listing. With no argument, all the
devices on the main 1-wire bus will be listed. Given the name of a 1-wire
device, the available properties will be listed. It is the equivalent of
\begin{description}
\item [\textit{ls} directory ] 
\end{description}


in the \textsf{\textbf{owfs (1)}} filesystem. 
\subparagraph*{owread}\textbf{owread} obtains for value
of a 1-wire device property. e.g. 28.0080BE21AA00/temperature gives the DS18B20
temperature. It is the equivalent of \begin{description}
\item [\textit{cat} filepath ] 
\end{description}


in the \textsf{\textbf{owfs (1)}} filesystem.

\subparagraph*{owwrite}\textbf{owwrite} performs a change of a property, changing a 1-wire device
setting or writing to memory. It is the equivalent of \begin{description}
\item [\textit{echo} "value" $>$ filepath
] 
\end{description}


in the \textsf{\textbf{owfs (1)}} filesystem. 
\subparagraph*{owget}\textsf{\textbf{owget (1)}} is a convenience program, combining
the function of \textsf{\textbf{owdir (1)}} and \textsf{\textbf{owread (1)}} by first trying to read the argument
as a directory, and if that fails as a 1-wire property. 
\paragraph*{Standard Options}

\subparagraph*{--autoserver}Find
an  \textit{owserver} using the Service Discovery protocol. Essentially Apple's Bonjour
(aka zeroconf). Only the first  \textit{owserver}  will be used, and that choice
is probably arbitrary. 
\subparagraph*{-s [host:]port}Connect via tcp (network) to an \textit{owserver}
process that is connected to a physical 1-wire bus. This allows multiple
processes to share the same bus. The \textit{owserver} process can be local or remote.

\paragraph*{Data Options}

\paragraph*{--hex}
Hexidecimal mode. For reading data, each byte of character
will be displayed as two characrters 0-9ABCDEF. Most useful for reading memory
locations. No spaces between data. 

Writing data in hexidecimal mode just
means that the data should be given as one long hexidecimal string. 
\paragraph*{--start=offset}
Read
or write memory locations starting at the offset byte rather than the beginning.
An offset of 0 means the beginning (and is the default). 


\paragraph*{--size=bytes}
Read
up to the specified number of bytes of a memory location. 
\paragraph*{Help Options}

\subparagraph*{-h
--help}Shows basic summary of options. 
\subparagraph*{-V --version}\textit{Version} of this program and
related libraries. 
\paragraph*{Display Options}

\subparagraph*{--dir}Modify the display of directories to
indicate which entries are also directories. A directory member will have
a trailing '/' if it is a directory itself. This aids recursive searches. 
\subparagraph*{-f
--format \dq{}f[.]i[[.]c]\dq{}}Display format for the 1-wire devices. Each device has
a 8 byte address, consisting of: \begin{description}
\item [\textit{f} ] family code, 1 byte 
\item [\textit{i} ] ID number, 6 bytes

\item [\textit{c} ] CRC checksum, 1 byte 
\end{description}


Possible formats are \textit{f.i} (default, 01.A1B2C3D4E5F6),
\textit{fi} fic f.ic f.i.c and \textit{fi.c} 

All formats are accepted as input, but the output
will be in the specified format. 
\paragraph*{Example}
\begin{description}
\item [owdir -s 3000 --format fic ] Get the
device listing (full 16 hex digits, no dots) from the local \textit{owserver} at
port 3000 
\item [owread -F --autoserver 51.125499A32000/typeK/temperature ] Read temperature
from the DS2751-based thermocouple on an auto-discovered \textit{owserver} Temperature
in fahrenheit. 
\item [owwrite -s 10.0.1.2:3001 32.000800AD23110/pages/page.1 "Passed"
] Connect to a OWFS server process ( \textit{owserver} ) that was started on another
machine at tcp port 3001 and write to the memory of a DS2780 
\end{description}

\paragraph*{See Also}

\subparagraph*{Programs}\textsf{\textbf{owfs
(1)} \textsf{owhttpd (1)} \textsf{owftpd (1)} \textsf{owserver (1)}} \textsf{\textbf{owdir (1)} \textsf{owread (1)} \textsf{owwrite (1)}
\textsf{owpresent (1)}} \textsf{\textbf{owtap (1)}} 
\subparagraph*{Configuration and testing}\textsf{\textbf{owfs (5)} \textsf{owtap (1)} \textsf{owmon
(1)}} 
\subparagraph*{Language bindings}\textsf{\textbf{owtcl (3)} \textsf{owperl (3)} \textsf{owcapi (3)}} 
\subparagraph*{Clocks}\textsf{\textbf{DS1427 (3)} \textsf{DS1904(3)}
\textsf{DS1994 (3)} \textsf{DS2404 (3)} \textsf{DS2404S (3)} \textsf{DS2415 (3)} \textsf{DS2417 (3)}} 
\subparagraph*{ID}\textsf{\textbf{DS2401 (3)} \textsf{DS2411
(3)} \textsf{DS1990A (3)}} 
\subparagraph*{Memory}\textsf{\textbf{DS1982 (3)} \textsf{DS1985 (3)} \textsf{DS1986 (3)} \textsf{DS1991 (3)} \textsf{DS1992
(3)} \textsf{DS1993 (3)} \textsf{DS1995 (3)} \textsf{DS1996 (3)} \textsf{DS2430A (3)} \textsf{DS2431 (3)} \textsf{DS2433 (3)}
\textsf{DS2502 (3)} \textsf{DS2506 (3)} \textsf{DS28E04 (3)} \textsf{DS28EC20 (3)}} 
\subparagraph*{Switches}\textsf{\textbf{DS2405 (3)} \textsf{DS2406
(3)} \textsf{DS2408 (3)} \textsf{DS2409 (3)} \textsf{DS2413 (3)} \textsf{DS28EA00 (3)}} 
\subparagraph*{Temperature}\textsf{\textbf{DS1822 (3)}
\textsf{DS1825 (3)} \textsf{DS1820 (3)} \textsf{DS18B20 (3)} \textsf{DS18S20 (3)} \textsf{DS1920 (3)} \textsf{DS1921 (3)} \textsf{DS1821
(3)} \textsf{DS28EA00 (3)} \textsf{DS28E04 (3)}} 
\subparagraph*{Humidity}\textsf{\textbf{DS1922 (3)}} 
\subparagraph*{Voltage}\textsf{\textbf{DS2450 (3)}} 
\subparagraph*{Resistance}\textsf{\textbf{DS2890
(3)}} 
\subparagraph*{Multifunction (current, voltage, temperature)}\textsf{\textbf{DS2436 (3)} \textsf{DS2437 (3)}
\textsf{DS2438 (3)} \textsf{DS2751 (3)} \textsf{DS2755 (3)} \textsf{DS2756 (3)} \textsf{DS2760 (3)} \textsf{DS2770 (3)} \textsf{DS2780
(3)} \textsf{DS2781 (3)} \textsf{DS2788 (3)} \textsf{DS2784 (3)}} 
\subparagraph*{Counter}\textsf{\textbf{DS2423 (3)}} 
\subparagraph*{LCD Screen}\textsf{\textbf{LCD (3)}
\textsf{DS2408 (3)}} 
\subparagraph*{Crypto}\textsf{\textbf{DS1977 (3)}} 
\subparagraph*{Pressure}\textsf{\textbf{DS2406 (3)} -- TAI8570}  
\paragraph*{Availability}
\altlink{http://www.owfs.org/}

\paragraph*{Author}
Paul Alfille (\email{paul.alfille@gmail.com}) 

