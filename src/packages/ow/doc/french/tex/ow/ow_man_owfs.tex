% Synchronized to r29817
\paragraph*{Name}
\textbf{owfs} - 1-wire filesystem 
\paragraph*{Synopsis}
\textbf{owfs}           [ \textit{-c} config
] \textit{-d} serialport $|$ \textit{-u} $|$ \textit{-s} [host:]port  \textit{-m} mountdir 
\paragraph*{Description}
          
\subparagraph*{1-Wire}\textit{1-wire}
 is a wiring protocol and series of devices designed and manufactured by
Dallas Semiconductor, Inc. The bus is a low-power low-speed low-connector scheme
where the data line can also provide power. 

Each device is uniquely and
unalterably numbered during manufacture. There are a wide variety of devices,
including memory, sensors (humidity, temperature, voltage, contact, current),
switches, timers and data loggers. More complex devices (like thermocouple
sensors) can be built with these basic devices. There are also 1-wire devices
that have encryption included. 

The 1-wire scheme uses a single  \textit{bus} master
and multiple \textit{slaves} on the same wire. The bus master initiates all communication.
The slaves can be  individually discovered and addressed using their unique
ID. 

Bus masters come in a variety of configurations including serial, parallel,
i2c, network or USB adapters. 
\subparagraph*{OWFS design}\textit{OWFS} is a suite of programs that
designed to make the 1-wire bus and its devices easily accessible. The underlying
priciple is to create a virtual filesystem, with the unique ID being the
directory, and the individual properties of the device are represented
as simple files that can be read and written. 

Details of the individual
slave or master design are hidden behind a consistent interface. The goal
is to  provide an easy set of tools for a software designer to create monitoring
or control applications. There  are some performance enhancements in the
implementation, including data caching, parallel access to bus  masters,
and aggregation of device communication. Still the fundemental goal has
been ease of use, flexibility  and correctness rather than speed.  
\subparagraph*{owfs}\textsf{\textbf{owfs
(1)}} is the filesystem client of the  \textit{OWFS} family of programs. It runs on
linux, freebsd and Mac OS X, and requires the  \textit{fuse} kernel module and library.
(\altlink{http://fuse.sourceforge.net/}) which is a user-mode filesystem driver. 

Essentially,
the entire 1-wire bus is mounted to a place in your filesystem. All the 1-wire
devices are accessible using standard file operations (read, write, directory
listing). The system is safe, no actual files are exposed, these files are
virtual. Not all operations are supported. Specifically, file creation, deletion,
linking and renaming are not allowed. (You can link from outside to a owfs
file, but not the other way around).           
\paragraph*{Device Options (1-wire Bus
Master)}
These options specify the device (bus master) connecting the computer
to the 1-wire bus. The 1-wire slaves are connected to the 1-wire bus, and the
bus master connects to a port on the computer and controls the 1-wire bus.
The bus master is either an actual physical device, the kernel w1 module,
or an  \textsf{\textbf{owserver (1)}. } 

At least one device option is required. There is no
default. More than one device can be listed, and all will be used. (A logical
union unless you explore the \textit{/bus.n/} directories.) 

Linux and BSD enforce
a security policy restricting access to hardware ports. You must have sufficient
rights to access the given port or access will silently fail. 
\paragraph*{ Serial devices}
\textit{port}
 specifies a serial port, e.g.  \textit{/dev/ttyS0} \begin{description}
\item [\textit{-d port\textit{ }}$|$ \textit{--device=port\textit{ \textbf{(DS2480B)\textbf{
}}}}] DS2480B-based bus master (like the DS9097U or the LINK in emulation mode).
If the adapter doesn't respond, a passive type (DS9907E or diode/resistor)
circuit will be assumed. 
\item [\textit{--serial\_flextime} $|$ --serial\_regulartime \textbf{(DS2480B)\textbf{ }}] 

Changes details of bus timing (see DS2480B datasheet). Some devices, like
the \textit{Swart} LCD cannot work with \textit{flextime.} 
\item [\textit{--baud=\textit{}}1200$|$9600$|$19200$|$38400$|$57600$|$115200
\textbf{(DS2480B,LINK,HA5)\textbf{ }}] Sets the initial serial port communication speed for
all bus masters. Not all serial devices support all speeds. You can change
the individual bus master speed for the \textbf{LINK} and \textbf{DS2880B} in the interface/settings
directory. The \textbf{HA5} speed is set in hardware, so the command line buad rate
should match that rate. 

Usually the default settings (9600 for \textbf{LINK} and \textbf{DS2480B} ) and 115200 for
the \textbf{HA5} are sane and shouldn't be changed. 
\item [\textit{--straight\_polarity\textit{  }}$|$ \textit{--reverse\_polarity\textit{
\textbf{(DS2480B)\textbf{ }}}}] Reverse polarity of the DS2480B output transistors? Not needed
for the DS9097U, but required for some other designs. 
\item [\textit{--link=port\textit{ \textbf{(LINK)\textbf{ }}}}] \textbf{iButtonLink}
\textit{LINK} adapter (all versions) in non-emulation mode. Uses an ascii protocol
over serial. 
\item [\textit{--ha7e=port\textit{ \textbf{(HA7E)\textbf{ }}}}] \textbf{Embedded Data Systems} \textit{HA7E} adapter ( and \textit{HA7S}
) in native ascii mode. 
\item [\textit{--ha5=port $|$ --ha5=port:a $|$ --ha5=port:acg\textit{ \textbf{(HA5)\textbf{ }}}}] \textbf{Embedded
Data Systems} \textit{HA5} mutidrop adapter in native ascii mode. Up to 26 adapters
can share the same port, each with an assigned letter. If no letter specified,
the program will scan for the first response (which may be slow). 
\item [\textit{--checksum}
$|$ --no\_checksum \textbf{(HA5)\textbf{ }}] 

Turn on (default) or off the checksum feature of the HA5 communication.
 
\item [\textit{--passive=port} $|$ \textit{--ha2=port} $|$ \textit{--ha3=port} $|$ \textit{--ha4b=port \textbf{(Passive)\textbf{ }}}] Passive 1-wire adapters.
Powered off the serial port and using passive electrical components (resitors
and diodes). 
\item [\textit{--8bit} $|$ --6bit \textbf{(Passive)\textbf{ }}] 

Synthesize the 1-wire waveforme using a 6-bit (default) serial word, or 8-bit
word. Not all UART devices support 6 bit operation. 
\item [\textit{--timeout\_serial=5\textit{ }}] Timeout
(in seconds) for all serial communications. 5 second default. Can be altered
dynamically under  \textit{/settings/timeout/serial} 
\end{description}

\paragraph*{ USB devices}
The only supported
true USB bus masters are based on the DS2490 chip. The most common is the
DS9490R which has an included 1-wire ID slave with family code 81. 

There
are also bus masters based on the serial chip with a USB to serial conversion
built in. These are supported by the serial bus master protocol.  \begin{description}
\item [\textit{-u}  $|$ \textit{--usb
}] DS2490 based bus master (like the DS9490R). 
\item [\textit{-u2}  $|$ \textit{--usb=2 }] Use the second USB
bus master. (The order isn't predicatble, however, since the operating system
does not conssitently order USB devices). 
\item [\textit{-uall}  $|$ \textit{--usb=ALL }] Use all the USB
devices. 
\item [\textit{--usb\_flextime} $|$ --usb\_regulartime ] Changes the details of 1-wire waveform
timing for certain network configurations. 
\item [\textit{--altusb} ] Willy Robion's alternative
USB timing.  
\item [\textit{--timeout\_usb=5} ] Timeout for USB communications. This has a 5 second
default and can be changed dynamically under \textit{/settings/timeout/usb} 
\end{description}

\paragraph*{ I2C
devices}
I2C is  2 wire protocol used for chip-to-chip communication. The bus
masters: \textit{DS2482-100,} DS2482-101 and \textit{DS2482-800} can specify (via pin voltages)
a subset of addresses on the i2c bus. Those choices are 

\textit{i2c\_address} \begin{description}
\item [0,1,2,3
] 0x18,0x19,0x1A,0x1B 
\item [4,5,6,7 ] 0x1C,0x1D,0x1E,0x1F (DS2482-800 only) 
\end{description}


\textit{port} for
i2c masters have the form  \textit{/dev/i2c-0,} /dev/i2c-1, ... \begin{description}
\item [\textit{-d port} $|$ \textit{--device=port }] This
simple form only permits a specific  \textit{port}  and the first available \textit{i2c\_address}

\item [\textit{--i2c=port} $|$ \textit{--i2c=port:i2c\_address} $|$ \textit{--i2c=port:ALL }] Specific i2c \textit{port} and the
\textit{i2c\_address} is either the first, specific, or all or them. The  \textit{i2c\_address}
is 0,1,2,... 
\item [\textit{--i2c} $|$ \textit{--i2c=:} $|$ \textit{--i2c=ALL:ALL }] Search the available i2c buses for either
the first, the first, or every i2c adapter. 
\end{description}


The \textit{DS2482-800} masters 8 1-wire
buses and so will generate 8 \textit{/bus.n} entries. 
\paragraph*{ Network devices}
These bus masters
communicate via the tcp/ip network protocol and so can be located anywhere
on the network. The \textit{network\_address} is of the form tcp\_address:port 

E.g. 192.168.0.1:3000
or localhost:3000 \begin{description}
\item [\textit{--link=network\_address} ] LinkHubE network LINK adapter by
 \textbf{iButtonLink} 
\item [\textit{--ha7net=network\_address} $|$ --ha7net ] HA7Net network 1-wire adapter
with specified tcp address or discovered by udp multicast. By \textbf{Embedded Data
Systems} 

\textit{--timeout\_ha7=60} specific timeout for HA7Net communications (60 second default).

\item [\textit{--etherweather=network\_address} ] Etherweather adapter 
\item [\textit{-s network\_address} $|$ \textit{--server=network\_address
}] Location of an \textsf{\textbf{owserver (1)}} program that talks to the 1-wire bus. The default
port is 4304. 
\item [\textit{--timeout\_network=5} ] Timeout for network bus master communications.
This has a 1 second default and can be changed dynamically under \textit{/settings/timeout/network}

\end{description}

\paragraph*{ Simulated devices}
Used for testing and development. No actual hardware
is needed. Useful for separating the hardware development from the rest
of the software design. \begin{description}
\item [\textit{devices} ] is a list of comma-separated 1-wire devices
in the following formats. Note that a valid CRC8 code is created automatically.

\item [10,05,21 ] Hexidecimal \textit{family} codes (the DS18S20, DS2405 and DS1921 in this
example). 
\item [10.12AB23431211 ] A more complete hexidecimal unique address. Useful
when an actual hardware device should be simulated. 
\item [DS2408,DS2489 ] The 1-wire
device name. (Full ID cannot be speciifed in this format). 
\item [\textit{--fake=devices} ] Random
address and random values for each read. The device ID is also random (unless
specified). 
\item [\textit{--temperature\_low=12} --temperature\_high=44 ] Specify the temperature
limits for the \textit{fake} adapter simulation. These should be in the same temperature
scale that is specified in the command line. It is possible to change the
limits dynamically for each adapter under \textit{/bus.x/interface/settings/simulated/[temperature\_low$|$temperature\_high]}

\item [\textit{--tester=devices} ] Predictable address and predictable values for each read.
(See the website for the algorhythm). 
\end{description}

\paragraph*{ w1 kernel module}
This a linux-specific
option for using the operating system's access to bus masters. Root access
is required and the implementation was still in progress as of owfs v2.7p12
and linux 2.6.30. 

Bus masters are recognized and added dynamically. Details
of the physical bus master are not accessible, bu they include USB, i2c
and a number of GPIO designs on embedded boards. 

Access is restrict to superuser
due to the netlink broadcast protocol employed by w1. Multitasking must
be configured (threads) on the compilation. \begin{description}
\item [\textit{--w1} ] Use the linux kernel w1 virtual
bus master. 
\item [\textit{--timeout\_w1=10} ] Timeout for w1 netlink communications. This has
a 10 second default and can be changed dynamically under \textit{/settings/timeout/w1}
 
\end{description}

\paragraph*{Specific Options}

\subparagraph*{-m --mountpoint=directory\_path}Path of a directory to mount
the 1-wire file system 

The mountpoint is required. There is no default. 
\subparagraph*{--allow\_other}Shorthand
for fuse mount option "-o allow\_other"  Allows other users to see the fuse
(owfs) mount point and file system. Requires a setting in /etc/fuse.conf
as well. 
\subparagraph*{--fuse-opt "options"}Sends options to the fuse-mount process. Options
should be quoted, e.g. "-o allow\_other".
\paragraph*{Temperature Scale Options}

\subparagraph*{-C --Celsius}
\subparagraph*{-F --Fahrenheit}
\subparagraph*{-K
--Kelvin}
\subparagraph*{-R --Rankine}Temperature scale used for data output. Celsius is the default.


Can also be changed within the program at \textit{/settings/units/temperature\_scale}
           
\paragraph*{Pressure Scale Options}

\subparagraph*{--mbar (default)}
\subparagraph*{--atm}
\subparagraph*{--mmHg}
\subparagraph*{--inHg}
\subparagraph*{--psi}
\subparagraph*{--Pa}Pressure
scale used for data output. Millibar is the default. 

Can also be changed
within the program at \textit{/settings/units/pressure\_scale}             
\paragraph*{Format
Options}
Choose the representation of the 1-wire unique identifiers. OWFS uses
these identifiers as unique directory names. 

Although several display formats
are selectable, all must be in  \textit{family-id-crc8} form, unlike some other programs
and the labelling on iButtons, which are \textit{crc8-id-family} form. 
\subparagraph*{-f --format="f[.]i[[.]c]"}Display
format for the 1-wire devices. Each device has a 8byte address, consisting
of: \begin{description}
\item [\textit{f} ] family code, 1 byte 
\item [\textit{i} ] ID number, 6 bytes 
\item [\textit{c} ] CRC checksum, 1 byte 
\end{description}


Possible
formats are \textit{f.i} (default, 01.A1B2C3D4E5F6), \textit{fi} fic f.ic f.i.c and \textit{fi.c} 

All formats
are accepted as input, but the output will be in the specified format. 

The
address elements can be retrieved from a device entry in owfs by the  \textit{family,}
id and crc8 properties, and as a whole with \textit{address.} The reversed id and
address can be retrieved as \textit{r\_id} and  \textit{r\_address.}            
\paragraph*{Job Control
Options}

\subparagraph*{-r --readonly}
\subparagraph*{-w --write}Do we allow writing to the 1-wire bus (writing memory,
setting switches, limits, PIOs)? The \textit{write} option is available for symmetry,
it's the default. 
\subparagraph*{-P --pid-file "filename"}Places the PID -- process ID of owfs
into the specified filename. Useful for startup scripts control. 
\subparagraph*{--background
$|$ --foreground}Whether the program releases the console and runs in the \textit{background}
after evaluating command line options. \textit{background} is the default. 
\subparagraph*{--error\_print=0$|$1$|$2$|$3}\begin{description}
\item [\textit{=0}
] default mixed destination: stderr foreground / syslog background 
\item [\textit{=1} ] syslog
only 
\item [\textit{=2} ] stderr only 
\item [\textit{=3} ] /dev/null (quiet mode). 
\end{description}

\subparagraph*{--error\_level=0..9}\begin{description}
\item [\textit{=0} ] default
errors only 
\item [\textit{=1} ] connections/disconnections 
\item [\textit{=2} ] all high level calls 
\item [\textit{=3} ] data
summary for each call 
\item [\textit{=4} ] details level 
\item [\textit{$>$4} ] debugging chaff 
\end{description}


\textit{--error\_level=9}
produces a lot of output            
\paragraph*{Configuration File}

\subparagraph*{-c file $|$ --configuration
file}Name of an \textsf{\textbf{owfs (5)}} configuration file with more command line parameters
            
\paragraph*{Help Options}
See also this man page and the web site \altlink{http://www.owfs.org/}

\subparagraph*{-h --help=[device$|$cache$|$program$|$job$|$temperature]}Shows basic summary of options.
\begin{description}
\item [\textit{device} ] 1-wire bus master options 
\item [\textit{cache} ] cache and communication size and
timing 
\item [\textit{program} ] mountpoint or TCP server settings 
\item [\textit{job} ] control and debugging
options 
\item [\textit{temperature} ] Unique ID display format and temperature scale 
\end{description}

\subparagraph*{-V --version}\textit{Version}
of this program and related libraries.            
\paragraph*{Time Options}
Timeouts for
the bus masters were previously listed in \textit{Device} options. Timeouts for the
cache affect the time that data stays in memory. Default values are shown.

\subparagraph*{--timeout\_volatile=15}Seconds until a  \textit{volatile}  property expires in the cache.
Volatile properties are those (like temperature) that change on their own.


Can be changed dynamically at  \textit{/settings/timeout/volatile} 
\subparagraph*{--timeout\_stable=300}Seconds
until a  \textit{stable}  property expires in the cache. Stable properties are those
that shouldn't change unless explicitly changed. Memory contents for example.


Can be changed dynamically at  \textit{/settings/timeout/stable} 
\subparagraph*{--timeout\_directory=60}Seconds
until a  \textit{directory}  listing expires in the cache. Directory lists are the
1-wire devices found on the bus. 

Can be changed dynamically at  \textit{/settings/timeout/directory}

\subparagraph*{--timeout\_presence=120}Seconds until the \textit{presence} and bus location of a 1-wire
device expires in the cache. 

Can be changed dynamically at  \textit{/settings/timeout/presence}


\textbf{There are also timeouts for specific program responses:} 
\subparagraph*{--timeout\_server=5}Seconds
until the expected response from the \textsf{\textbf{owserver (1)}} is deemed tardy. 

Can be
changed dynamically at  \textit{/settings/timeout/server} 
\subparagraph*{--timeout\_ftp=900}Seconds
that an ftp session is kept alive. 

Can be changed dynamically at  \textit{/settings/timeout/ftp}
 
\paragraph*{Example}
\begin{description}
\item [owfs -d /dev/ttyS0 -m /mnt/1wire ] Bus master on serial port 
\item [owfs -F
-u -m /mnt/1wire ] USB adapter, temperatures reported in Fahrenheit 
\item [owfs -s
10.0.1.2:4304 -m /mnt/1wire ] Connect to an  \textsf{\textbf{owserver (1)}} process that was started
on another machine at tcp port 4304 
\end{description}

\paragraph*{See Also}

\subparagraph*{Programs}\textsf{\textbf{owfs (1)} \textsf{owhttpd (1)}
\textsf{owftpd (1)} \textsf{owserver (1)}} \textsf{\textbf{owdir (1)} \textsf{owread (1)} \textsf{owwrite (1)} \textsf{owpresent (1)}}
\textsf{\textbf{owtap (1)}} 
\subparagraph*{Configuration and testing}\textsf{\textbf{owfs (5)} \textsf{owtap (1)} \textsf{owmon (1)}} 
\subparagraph*{Language
bindings}\textsf{\textbf{owtcl (3)} \textsf{owperl (3)} \textsf{owcapi (3)}} 
\subparagraph*{Clocks}\textsf{\textbf{DS1427 (3)} \textsf{DS1904(3)} \textsf{DS1994
(3)} \textsf{DS2404 (3)} \textsf{DS2404S (3)} \textsf{DS2415 (3)} \textsf{DS2417 (3)}} 
\subparagraph*{ID}\textsf{\textbf{DS2401 (3)} \textsf{DS2411 (3)}
\textsf{DS1990A (3)}} 
\subparagraph*{Memory}\textsf{\textbf{DS1982 (3)} \textsf{DS1985 (3)} \textsf{DS1986 (3)} \textsf{DS1991 (3)} \textsf{DS1992 (3)}
\textsf{DS1993 (3)} \textsf{DS1995 (3)} \textsf{DS1996 (3)} \textsf{DS2430A (3)} \textsf{DS2431 (3)} \textsf{DS2433 (3)} \textsf{DS2502
(3)} \textsf{DS2506 (3)} \textsf{DS28E04 (3)} \textsf{DS28EC20 (3)}} 
\subparagraph*{Switches}\textsf{\textbf{DS2405 (3)} \textsf{DS2406 (3)} \textsf{DS2408
(3)} \textsf{DS2409 (3)} \textsf{DS2413 (3)} \textsf{DS28EA00 (3)}} 
\subparagraph*{Temperature}\textsf{\textbf{DS1822 (3)} \textsf{DS1825 (3)}
\textsf{DS1820 (3)} \textsf{DS18B20 (3)} \textsf{DS18S20 (3)} \textsf{DS1920 (3)} \textsf{DS1921 (3)} \textsf{DS1821 (3)} \textsf{DS28EA00
(3)} \textsf{DS28E04 (3)}} 
\subparagraph*{Humidity}\textsf{\textbf{DS1922 (3)}} 
\subparagraph*{Voltage}\textsf{\textbf{DS2450 (3)}} 
\subparagraph*{Resistance}\textsf{\textbf{DS2890 (3)}}

\subparagraph*{Multifunction (current, voltage, temperature)}\textsf{\textbf{DS2436 (3)} \textsf{DS2437 (3)} \textsf{DS2438
(3)} \textsf{DS2751 (3)} \textsf{DS2755 (3)} \textsf{DS2756 (3)} \textsf{DS2760 (3)} \textsf{DS2770 (3)} \textsf{DS2780 (3)} \textsf{DS2781
(3)} \textsf{DS2788 (3)} \textsf{DS2784 (3)}} 
\subparagraph*{Counter}\textsf{\textbf{DS2423 (3)}} 
\subparagraph*{LCD Screen}\textsf{\textbf{LCD (3)} \textsf{DS2408 (3)}}

\subparagraph*{Crypto}\textsf{\textbf{DS1977 (3)}} 
\subparagraph*{Pressure}\textsf{\textbf{DS2406 (3)} -- TAI8570}  
\paragraph*{Availability}
\altlink{http://www.owfs.org/}

\paragraph*{Author}
Paul Alfille (\email{paul.alfille@gmail.com}) 
