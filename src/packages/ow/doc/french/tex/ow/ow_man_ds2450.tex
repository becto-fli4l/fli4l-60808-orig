% Synchronized to r29817
\paragraph*{Name}
DS2450 - Quad A/D Converter 
\paragraph*{Synopsis}

\subparagraph*{Voltage * 4  and Memory.}

\textbf{20}
[.]XXXXXXXXXXXX[XX][/[ \textbf{PIO.[A-D$|$ALL]} $|$ \textbf{volt.[A-D$|$ALL]} $|$ \textbf{volt2.[A-D$|$ALL]} ]] 

\textbf{20} [.]XXXXXXXXXXXX[XX][/[
\textbf{8bit/volt.[A-D$|$ALL]} $|$ \textbf{8bit/volt2.[A-D$|$ALL]} ]] 

\textbf{20} [.]XXXXXXXXXXXX[XX][/[ \textbf{memory}
$|$ \textbf{pages/page.[0-3$|$ALL]} $|$ \textbf{power} ] 

\textbf{20} [.]XXXXXXXXXXXX[XX][/[ \textbf{alarm/high.[A-D$|$ALL]}
$|$ \textbf{alarm/low.[A-D$|$ALL]} $|$ \textbf{set\_alarm/high.[A-D$|$ALL]} $|$ \textbf{set\_alarm/low.[A-D$|$ALL]} $|$ \textbf{set\_alarm/unset}
$|$ \textbf{set\_alarm/volthigh.[A-D$|$ALL]} $|$ \textbf{set\_alarm/volt2high.[A-D$|$ALL]} $|$ \textbf{set\_alarm/voltlow.[A-D$|$ALL]}
$|$ \textbf{set\_alarm/volt2low.[A-D$|$ALL]} ] 

\textbf{20} [.]XXXXXXXXXXXX[XX][/[           \textbf{address}
$|$ \textbf{crc8} $|$ \textbf{id} $|$ \textbf{locator} $|$ \textbf{r\_address} $|$ \textbf{r\_id} $|$ \textbf{r\_locator} $|$ \textbf{type}  ]] 
\subparagraph*{CO2 sensor}

\textbf{20} [.]XXXXXXXXXXXX[XX][/[
\textbf{CO2/ppm} $|$ \textbf{CO2/power} $|$ \textbf{CO2/status} ] 
\paragraph*{Family Code}


\textit{20} 
\paragraph*{Special Properties}

\subparagraph*{alarm/high.A
... alarm/high.D alarm.high.ALL}
\subparagraph*{alarm/high.A ... alarm/high.D alarm.high.ALL}\textit{read-write,}
binary 

The alarm state of the voltage channel. The alarm state is set one of two
ways: \begin{description}
\item [voltage conversion ] Whenever the \textit{DS2450} measures a voltage on a channel,
that voltage is compared to the high and low limits \textit{set\_alarm/volthigh}
and/or \textit{set\_alarm/voltlow} and if the alarm is enabled \textit{set\_alarm/high} and/or
\textit{set\_alarm/low} the corresponding flag is set in \textit{alarm/high} and/or \textit{alarm/low}

\item [manual set ] The flag can be set by a direct write to \textit{alarm/high} or \textit{alarm/low}

\end{description}

\subparagraph*{memory}\textit{read-write,} binary 

32 bytes of data. Much has special implications. See the datasheet. 
\subparagraph*{pages/page.0
... pages/page.3 pages/page.ALL}\textit{read-write,} binary 

Memory is split into 4 pages of 8 bytes each. Mostly for reading and setting
device properties. See the datasheet for details. 

\textit{ALL} is an aggregate of the pages. Each page is accessed sequentially. 
\subparagraph*{Pio.a
... Pio.d Pio.all}\textit{read-write,} yes-no 

Pins used for digital control. 1 turns the switch on (conducting). 0 turns
the switch off (non-conducting). 

Control is specifically enabled. Reading \textit{volt} will turn off this control.


\textit{ALL} is an aggregate of the voltages. Readings are made separately. 
\subparagraph*{power}\textit{read-write,}
yes-no 

Is the \textit{DS2450} externally powered? (As opposed to parasitically powered
from the data line). 

The analog A/D  will be kept on continuously. And the bus will be released
during a conversion allowing other devices to communicate. 

Set \textsf{true (1)} only if Vcc powered (not parasitically powered). Unfortunately,
the \textit{DS2450} cannot sense it's powered state. This flag must be explicitly
written, and thus is a potential source of error if incorrectly set. 

It is always safe to leave \textit{power} set to the default 0 (off) state. 
\subparagraph*{set\_alarm/high.A
... set\_alarm/high.D set\_alarm/high.ALL}
\subparagraph*{set\_alarm/low.A ... set\_alarm/low.D set\_alarm/low.ALL}\textit{read-write,}
yes-no 

Enabled status of the voltage threshold. 1 is on. 0 is off. 
\subparagraph*{set\_alarm/volthigh.A
... set\_alarm/volthigh.D set\_alarm/volthigh.ALL}
\subparagraph*{set\_alarm/volt2high.A ... set\_alarm/volt2high.D
set\_alarm/volt2high.ALL}
\subparagraph*{set\_alarm/voltlow.A ... set\_alarm/voltlow.D set\_alarm/voltlow.ALL}
\subparagraph*{set\_alarm/volt2low.A
... set\_alarm/volt2low.D set\_alarm/volt2low.ALL}\textit{read-write,} floating point 

The upper or lower limit for the voltage measured before triggering an
alarm. 

Note that the alarm must be enabled \textit{alarm/high} or \textit{alarm.low} and an actual
reading must be requested \textit{volt} for the alarm state to actually be set. The
alarm state can be sensed at \textit{alarm/high} and \textit{alarm/low} 
\subparagraph*{set\_alarm/unset}\textit{read-write,}
yes-no 

Status of the power-on-reset (POR) flag. 

The POR is set when the \textit{DS2450} is first powered up, and will match the
alarm state until explicitly cleared. (By writing 0 to it). 

The purpose of the POR is to alert the user that the chip is not yet fully
configured, especially alarm thresholds and enabling. 
\subparagraph*{volt.A ... volt.D volt.ALL}
\subparagraph*{8bit/volt.A
... 8bit/volt.D 8bit/volt.ALL}\textit{read-only,} floating point 

Voltage read, 16 bit resolution (or 8 bit for the  \textit{8bit} directory). Range
0 - 5.10V. 

Output ( \textit{PIO} ) is specifically disabled. 

\textit{ALL} is an aggregate of the voltages. Readings are made separately. 
\subparagraph*{volt2.A
... volt2.D volt2.ALL}
\subparagraph*{8bit/volt2.A ... 8bit/volt2.D 8bit/volt2.ALL}\textit{read-only,} floating
point 

Voltage read, 16 bit resolution (or 8 bit for the  \textit{8bit} directory). Range
0 - 2.55V. 

Output ( \textit{PIO} ) is specifically disabled. 

\textit{ALL} is an aggregate of the voltages. Readings are made separately. 
\paragraph*{CO2 (Carbon
Dioxide) SENSOR PROPERTIES}
The CO2 sensor is a device constructed from a
SenseAir K30 and a \textit{DS2450} 
\subparagraph*{CO2/power}\textit{read-only,} floating point 

Supply voltage to the CO2 sensor (should be around 5V) 
\subparagraph*{CO2/ppm}\textit{read-only,}
unsigned 

CO2 level in ppm (parts per million). Range 0-5000. 
\subparagraph*{CO2/status}\textit{read-only,} yes-no


Is the internal voltage correct (around 3.2V)? 
\paragraph*{Standard Properties}
     
    
\subparagraph*{address}
\subparagraph*{r\_address}\textit{read-only,} ascii 

The entire 64-bit unique ID. Given as upper case hexidecimal digits (0-9A-F).


\textit{address} starts with the \textit{family} code 

\textit{r} address is the \textit{address} in reverse order, which is often used in other
applications and labeling. 
\subparagraph*{crc8}\textit{read-only,} ascii 

The 8-bit error correction portion. Uses cyclic redundancy check. Computed
from the preceding 56 bits of the unique ID number. Given as upper case
hexidecimal digits (0-9A-F). 
\subparagraph*{family}\textit{read-only,} ascii 

The 8-bit family code. Unique to each \textit{type} of device. Given as upper case
hexidecimal digits (0-9A-F). 
\subparagraph*{id}
\subparagraph*{r\_id}\textit{read-only,} ascii 

The 48-bit middle portion of the unique ID number. Does not include the family
code or CRC. Given as upper case hexidecimal digits (0-9A-F). 

\textit{r} id is the \textit{id} in reverse order, which is often used in other applications
and labeling. 
\subparagraph*{locator}
\subparagraph*{r\_locator}\textit{read-only,} ascii 

Uses an extension of the 1-wire design from iButtonLink company that associated
1-wire physical connections with a unique 1-wire code. If the connection is
behind a \textbf{Link Locator} the \textit{locator} will show a unique 8-byte number (16 character
hexidecimal) starting with family code FE. 

If no \textbf{Link Locator} is between the device and the master, the \textit{locator} field
will be all FF. 

\textit{r} locator is the \textit{locator} in reverse order. 
\subparagraph*{present (DEPRECATED)}\textit{read-only,}
yes-no 

Is the device currently \textit{present} on the 1-wire bus? 
\subparagraph*{type}\textit{read-only,} ascii 

Part name assigned by Dallas Semi. E.g. \textit{DS2401} Alternative packaging (iButton
vs chip) will not be distiguished.  
\paragraph*{Alarms}
None. 
\paragraph*{Description}
          
\subparagraph*{1-Wire}\textit{1-wire}
 is a wiring protocol and series of devices designed and manufactured by
Dallas Semiconductor, Inc. The bus is a low-power low-speed low-connector scheme
where the data line can also provide power. 

Each device is uniquely and
unalterably numbered during manufacture. There are a wide variety of devices,
including memory, sensors (humidity, temperature, voltage, contact, current),
switches, timers and data loggers. More complex devices (like thermocouple
sensors) can be built with these basic devices. There are also 1-wire devices
that have encryption included. 

The 1-wire scheme uses a single  \textit{bus} master
and multiple \textit{slaves} on the same wire. The bus master initiates all communication.
The slaves can be  individually discovered and addressed using their unique
ID. 

Bus masters come in a variety of configurations including serial, parallel,
i2c, network or USB adapters. 
\subparagraph*{OWFS design}\textit{OWFS} is a suite of programs that
designed to make the 1-wire bus and its devices easily accessible. The underlying
priciple is to create a virtual filesystem, with the unique ID being the
directory, and the individual properties of the device are represented
as simple files that can be read and written. 

Details of the individual
slave or master design are hidden behind a consistent interface. The goal
is to  provide an easy set of tools for a software designer to create monitoring
or control applications. There  are some performance enhancements in the
implementation, including data caching, parallel access to bus  masters,
and aggregation of device communication. Still the fundemental goal has
been ease of use, flexibility  and correctness rather than speed.  
\subparagraph*{Ds2450}The
\textsf{\textbf{DS2450 (3)}} is a (supposedly) high resolution A/D converter with 4 channels.
Actual resolutin is reporterd to be 8 bits. The channels can also function
as switches. Voltage sensing (with temperature and current, but sometimes
restricted voltrage ranges) can also be obtained with the \textbf{DS2436} , \textbf{DS2438}
and \textbf{DS276x} 
\paragraph*{Addressing}
          All 1-wire devices are factory assigned a
unique 64-bit address. This address is of the form: \begin{description}
\item [\textbf{Family Code} ] 8 bits 
\item [\textbf{Address}
] 48 bits 
\item [\textbf{CRC} ] 8 bits 
\end{description}


\begin{description}
\item [Addressing under OWFS is in hexidecimal, of form: ] 
\item [\textbf{01.123456789ABC}
] 
\end{description}


where \textbf{01} is an example 8-bit family code, and \textbf{12345678ABC} is an example
48 bit address. 

The dot is optional, and the CRC code can included. If included,
it must be correct.  
\paragraph*{Datasheet}
\begin{description}
\item [DS2450 ] \altlink{http://pdfserv.maxim-ic.com/en/ds/DS2450.pdf}

\item [CO2 sensor ] \altlink{http://www.senseair.se/Datablad/k30\%20.pdf} 
\item [CO2 device ] \altlink{https://www.m.nu/co2meter-version-2-p-259.html?language=en}

\end{description}

\paragraph*{See Also}

\subparagraph*{Programs}\textsf{\textbf{owfs (1)} \textsf{owhttpd (1)} \textsf{owftpd (1)} \textsf{owserver (1)}} \textsf{\textbf{owdir (1)}
\textsf{owread (1)} \textsf{owwrite (1)} \textsf{owpresent (1)}} \textsf{\textbf{owtap (1)}} 
\subparagraph*{Configuration and testing}\textsf{\textbf{owfs
(5)} \textsf{owtap (1)} \textsf{owmon (1)}} 
\subparagraph*{Language bindings}\textsf{\textbf{owtcl (3)} \textsf{owperl (3)} \textsf{owcapi (3)}}

\subparagraph*{Clocks}\textsf{\textbf{DS1427 (3)} \textsf{DS1904(3)} \textsf{DS1994 (3)} \textsf{DS2404 (3)} \textsf{DS2404S (3)} \textsf{DS2415 (3)}
\textsf{DS2417 (3)}} 
\subparagraph*{ID}\textsf{\textbf{DS2401 (3)} \textsf{DS2411 (3)} \textsf{DS1990A (3)}} 
\subparagraph*{Memory}\textsf{\textbf{DS1982 (3)} \textsf{DS1985
(3)} \textsf{DS1986 (3)} \textsf{DS1991 (3)} \textsf{DS1992 (3)} \textsf{DS1993 (3)} \textsf{DS1995 (3)} \textsf{DS1996 (3)} \textsf{DS2430A
(3)} \textsf{DS2431 (3)} \textsf{DS2433 (3)} \textsf{DS2502 (3)} \textsf{DS2506 (3)} \textsf{DS28E04 (3)} \textsf{DS28EC20 (3)}}

\subparagraph*{Switches}\textsf{\textbf{DS2405 (3)} \textsf{DS2406 (3)} \textsf{DS2408 (3)} \textsf{DS2409 (3)} \textsf{DS2413 (3)} \textsf{DS28EA00
(3)}} 
\subparagraph*{Temperature}\textsf{\textbf{DS1822 (3)} \textsf{DS1825 (3)} \textsf{DS1820 (3)} \textsf{DS18B20 (3)} \textsf{DS18S20 (3)}
\textsf{DS1920 (3)} \textsf{DS1921 (3)} \textsf{DS1821 (3)} \textsf{DS28EA00 (3)} \textsf{DS28E04 (3)} \textsf{EDS0064 (3)} \textsf{EDS0065
(3)} \textsf{EDS0066 (3)} \textsf{EDS0067 (3)} \textsf{EDS0068 (3)} \textsf{EDS0071 (3)} \textsf{EDS0072 (3)}} 
\subparagraph*{Humidity}\textsf{\textbf{DS1922
(3)} \textsf{DS2438 (3)} \textsf{EDS0065 (3)} \textsf{EDS0068 (3)}} 
\subparagraph*{Voltage}\textsf{\textbf{DS2450 (3)}} 
\subparagraph*{Resistance}\textsf{\textbf{DS2890
(3)}} 
\subparagraph*{Multifunction (current, voltage, temperature)}\textsf{\textbf{DS2436 (3)} \textsf{DS2437 (3)}
\textsf{DS2438 (3)} \textsf{DS2751 (3)} \textsf{DS2755 (3)} \textsf{DS2756 (3)} \textsf{DS2760 (3)} \textsf{DS2770 (3)} \textsf{DS2780
(3)} \textsf{DS2781 (3)} \textsf{DS2788 (3)} \textsf{DS2784 (3)}} 
\subparagraph*{Counter}\textsf{\textbf{DS2423 (3)}} 
\subparagraph*{LCD Screen}\textsf{\textbf{LCD (3)}
\textsf{DS2408 (3)}} 
\subparagraph*{Crypto}\textsf{\textbf{DS1977 (3)}} 
\subparagraph*{Pressure}\textsf{\textbf{DS2406 (3)} -- TAI8570 \textsf{EDS0066 (3)} \textsf{EDS0068
(3)}}  
\paragraph*{Availability}
\altlink{http://www.owfs.org/} 
\paragraph*{Author}
Paul Alfille (\email{paul.alfille@gmail.com})
