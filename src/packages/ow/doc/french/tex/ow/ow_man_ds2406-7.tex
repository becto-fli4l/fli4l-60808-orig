% Synchronized to r29817
\paragraph*{Name}
DS2406, DS2407 - Dual Addressable Switch with 1kbit Memory

\paragraph*{Synopsis}
Dual Switch, Write-once Memory 

\textbf{12} [.]XXXXXXXXXXXX[XX][/[ \textbf{channels}
$|$ \textbf{latch.[A$|$B$|$ALL$|$BYTE]} $|$ \textbf{memory} $|$ \textbf{pages/page.[0-3$|$ALL]} $|$ \textbf{PIO.[A$|$B$|$ALL$|$BYTE]} $|$ \textbf{power} $|$ \textbf{sensed.[A$|$B$|$ALL$|$BYTE]}
$|$ \textbf{set\_alarm} $|$ \textbf{TAI8570/[sibling$|$temperature$|$pressure]} $|$ \textbf{T8A/volt.[0-7,ALL]}    
      \textbf{address} $|$ \textbf{crc8} $|$ \textbf{id} $|$ \textbf{locator} $|$ \textbf{r\_address} $|$ \textbf{r\_id} $|$ \textbf{r\_locator} $|$ \textbf{type}  ]]

\paragraph*{Family Code}


\textit{12} 
\paragraph*{Special Properties}

\subparagraph*{channels}\textit{read-only,} unsigned integer 

Is this a 1 or 2 channel switch? The \textit{DS2406} comes in two forms, one has
only one \textit{PIO} pin (PIO.A). Returns 1 or 2. 
\subparagraph*{latch.A latch.B latch.ALL latch.BYTE}\textit{read-write,}
yes-no 

The activity latch is set to 1 with the first negative or positive edge
detected on the associated PIO channel. 

Writing any data will clear latch for all (both)) channels. This is a hardware
"feature" of the chip. 

\textit{ALL} references both channels simultaneously, comma separated 

\textit{BYTE} references both channels simultaneously as a single byte, with channel
A in bit 0. 
\subparagraph*{memory}\textit{read-write,} binary 

128 bytes of non-volatile, write-once data. 
\subparagraph*{pages/page.0 ... pages/page.3 pages/page.ALL}\textit{read-write,}
binary 

Memory organized as 4 pages or 32 bytes. Memory is write-once. 

\textit{ALL} is the aggregate of all 4 pages, sequentially accessed. 
\subparagraph*{Pio.a Pio.b Pio.all
Pio.byte}\textit{read-write,} yes-no 

State of the open-drain output ( \textit{PIO} ) pin. 0 = non-conducting (off), 1 =
\textsf{conducting (on)}. 

Writing zero will turn off the switch, non-zero will turn on the switch.
Reading the \textit{PIO} state will return the switch setting (flipflop in the data
sheet). To determine the actual logic level at the switch, refer to the
\textit{sensed} property. 

Note that the actual pin setting for the chip uses the opposite polarity
-- 0 for conducting, 1 for non-conducting. However, to turn a connected device
on (i.e. to deliver power) we use the software concept of 1 as conducting
or "on". 

\textit{ALL} references both channels simultaneously, comma separated. 

\textit{BYTE} references both channels simultaneously as a single byte, with channel
A in bit 0. 
\subparagraph*{power}\textit{read-only,} yes-no 

Is the \textit{DS2406} powered parasitically =0 or separately on the Vcc pin =1

\subparagraph*{sensed.A sensed.B sensed.ALL sensed.BYTE}\textit{read-only,} yes-no 

Logic level at the \textit{PIO} pin. 0 = ground. 1 = high (~2.4V - 5V ). Really makes
sense only if the \textit{PIO} state is set to zero (off), else will read zero. 

\textit{ALL} references both channels simultaneously, comma separated. 

\textit{BYTE} references both channels simultaneously as a single byte, with channel
A in bit 0. 
\subparagraph*{set\_alarm}\textit{read-write,} unsigned integer (0-331) 

A number consisting of three digits XYZ, where: \begin{description}
\item [X ] channel selection 

\textit{0} neither 

\textit{1} A only 

\textit{2} B only 

\textit{3} A or B 
\item [Y ] source selection 

\textit{0} undefined 

\textit{1} latch 

\textit{2} PIO 

\textit{3} sensed 
\item [Z ] polarity selection 

\textit{0} low 

\textit{1} high 
\end{description}


All digits will be truncated to the 0-3 (or 0-1) range. Leading zeroes
are optional (and may be problematic for some shells). 

Example: \begin{description}
\item [311 ] Responds
on Conditional Search when either latch.A or latch.B (or both) are set to
1. 
\item [$<$100 ] Never responds to Conditional Search. 
\end{description}

\subparagraph*{Tai8570/}\textit{subdirectory} 

Properties when the \textsf{\textit{DS2406} (3)} is built into a \textit{TAI8570.} 

If the \textsf{\textit{DS2406} (3)} is not part of a \textit{TAI8570} or is not the controlling switch,
attempts to read will result in an error. 
\subparagraph*{TAI8570/pressure}\textit{read-only,} floating
point 

Barometric \textit{pressure} in millibar. 
\subparagraph*{TAI8570/sibling}\textit{read-only,} ascii 

Hex address of the \textsf{\textit{DS2406} (3)} paired with this chip in a \textit{TAI8570.} 
\subparagraph*{TAI8570/temperature}\textit{read-only,}
floating-point 

Ambient \textit{temperature} measured by the \textit{TAI8570} in prevailing temperature units
(Centigrade is the default). 
\subparagraph*{T8A/volt.[0-7$|$ALL]}\textit{read-only,} floating-point 

Uses the T8A (by \textit{Embedded} Data Systems ) 8 channel voltage converter. Units
in volts, 0 to 5V range. 12 bit resolution. 
\paragraph*{Standard Properties}
         

\subparagraph*{address}
\subparagraph*{r\_address}\textit{read-only,} ascii 

The entire 64-bit unique ID. Given as upper case hexidecimal digits (0-9A-F).


\textit{address} starts with the \textit{family} code 

\textit{r} address is the \textit{address} in reverse order, which is often used in other
applications and labeling. 
\subparagraph*{crc8}\textit{read-only,} ascii 

The 8-bit error correction portion. Uses cyclic redundancy check. Computed
from the preceding 56 bits of the unique ID number. Given as upper case
hexidecimal digits (0-9A-F). 
\subparagraph*{family}\textit{read-only,} ascii 

The 8-bit family code. Unique to each \textit{type} of device. Given as upper case
hexidecimal digits (0-9A-F). 
\subparagraph*{id}
\subparagraph*{r\_id}\textit{read-only,} ascii 

The 48-bit middle portion of the unique ID number. Does not include the family
code or CRC. Given as upper case hexidecimal digits (0-9A-F). 

\textit{r} id is the \textit{id} in reverse order, which is often used in other applications
and labeling. 
\subparagraph*{locator}
\subparagraph*{r\_locator}\textit{read-only,} ascii 

Uses an extension of the 1-wire design from iButtonLink company that associated
1-wire physical connections with a unique 1-wire code. If the connection is
behind a \textbf{Link Locator} the \textit{locator} will show a unique 8-byte number (16 character
hexidecimal) starting with family code FE. 

If no \textbf{Link Locator} is between the device and the master, the \textit{locator} field
will be all FF. 

\textit{r} locator is the \textit{locator} in reverse order. 
\subparagraph*{present (DEPRECATED)}\textit{read-only,}
yes-no 

Is the device currently \textit{present} on the 1-wire bus? 
\subparagraph*{type}\textit{read-only,} ascii 

Part name assigned by Dallas Semi. E.g. \textit{DS2401} Alternative packaging (iButton
vs chip) will not be distiguished.  
\paragraph*{Alarms}
Use the \textit{set\_alarm} property to
set the alarm triggering criteria. 
\paragraph*{Description}
          
\subparagraph*{1-Wire}\textit{1-wire}  is a
wiring protocol and series of devices designed and manufactured by Dallas
Semiconductor, Inc. The bus is a low-power low-speed low-connector scheme where
the data line can also provide power. 

Each device is uniquely and unalterably
numbered during manufacture. There are a wide variety of devices, including
memory, sensors (humidity, temperature, voltage, contact, current), switches,
timers and data loggers. More complex devices (like thermocouple sensors)
can be built with these basic devices. There are also 1-wire devices that
have encryption included. 

The 1-wire scheme uses a single  \textit{bus} master and
multiple \textit{slaves} on the same wire. The bus master initiates all communication.
The slaves can be  individually discovered and addressed using their unique
ID. 

Bus masters come in a variety of configurations including serial, parallel,
i2c, network or USB adapters. 
\subparagraph*{OWFS design}\textit{OWFS} is a suite of programs that
designed to make the 1-wire bus and its devices easily accessible. The underlying
priciple is to create a virtual filesystem, with the unique ID being the
directory, and the individual properties of the device are represented
as simple files that can be read and written. 

Details of the individual
slave or master design are hidden behind a consistent interface. The goal
is to  provide an easy set of tools for a software designer to create monitoring
or control applications. There  are some performance enhancements in the
implementation, including data caching, parallel access to bus  masters,
and aggregation of device communication. Still the fundemental goal has
been ease of use, flexibility  and correctness rather than speed.  
\subparagraph*{Ds2406}The
\textsf{\textbf{DS2406 (3)}} allows control of other devices, like LEDs and relays. It superceeds
the \textbf{DS2405} and \textbf{DS2407} Alternative switches include the \textbf{DS2408} or even \textbf{DS2450}


The \textbf{DS2407} is practically identical to the \textit{DS2406} except for a strange
\textit{hidden} mode. It is supported just like the \textbf{DS2406} 
\subparagraph*{Tai8570}The \textit{TAI-8570} Pressure
Sensor is based on a 1-wire composite device by \textit{AAG} Electronica. The \textit{TAI8570}
uses 2 \textsf{\textit{DS2406} (3)} chips, paired as a reader and writer to synthesize 3-wire
communication. Only 1 of the \textsf{\textit{DS2406} (3)} will allow \textit{temperature} or \textit{pressure}
readings. It's mate's address can be shown as \textit{sibling.} 

The \textit{TAI8570} uses the
\textit{Intersema} MS5534a pressure sensor, and stores calibration and temperature
compensation values internally. 

Design and code examples are available from
AAG Electronica \altlink{http://aag.com.mx/} , specific permission to use code in a
GPL product was given by Mr. Aitor Arrieta of AAG and Dr. Simon Melhuish
of OWW. 
\paragraph*{Addressing}
          All 1-wire devices are factory assigned a unique
64-bit address. This address is of the form: \begin{description}
\item [\textbf{Family Code} ] 8 bits 
\item [\textbf{Address} ] 48
bits 
\item [\textbf{CRC} ] 8 bits 
\end{description}


\begin{description}
\item [Addressing under OWFS is in hexidecimal, of form: ] 
\item [\textbf{01.123456789ABC}
] 
\end{description}


where \textbf{01} is an example 8-bit family code, and \textbf{12345678ABC} is an example
48 bit address. 

The dot is optional, and the CRC code can included. If included,
it must be correct.  
\paragraph*{Datasheet}


\altlink{http://pdfserv.maxim-ic.com/en/ds/DS2406.pdf} 

\altlink{http://pdfserv.maxim-ic.com/en/ds/DS2407.pdf} 

\altlink{http://www.embeddeddatasystems.com/page/EDS/PROD/IO/T8A}

\altlink{http://oww.sourceforge.net/hardware.html\#bp} 
\paragraph*{See Also}

\subparagraph*{Programs}\textsf{\textbf{owfs (1)} \textsf{owhttpd
(1)} \textsf{owftpd (1)} \textsf{owserver (1)}} \textsf{\textbf{owdir (1)} \textsf{owread (1)} \textsf{owwrite (1)} \textsf{owpresent
(1)}} \textsf{\textbf{owtap (1)}} 
\subparagraph*{Configuration and testing}\textsf{\textbf{owfs (5)} \textsf{owtap (1)} \textsf{owmon (1)}} 
\subparagraph*{Language
bindings}\textsf{\textbf{owtcl (3)} \textsf{owperl (3)} \textsf{owcapi (3)}} 
\subparagraph*{Clocks}\textsf{\textbf{DS1427 (3)} \textsf{DS1904(3)} \textsf{DS1994
(3)} \textsf{DS2404 (3)} \textsf{DS2404S (3)} \textsf{DS2415 (3)} \textsf{DS2417 (3)}} 
\subparagraph*{ID}\textsf{\textbf{DS2401 (3)} \textsf{DS2411 (3)}
\textsf{DS1990A (3)}} 
\subparagraph*{Memory}\textsf{\textbf{DS1982 (3)} \textsf{DS1985 (3)} \textsf{DS1986 (3)} \textsf{DS1991 (3)} \textsf{DS1992 (3)}
\textsf{DS1993 (3)} \textsf{DS1995 (3)} \textsf{DS1996 (3)} \textsf{DS2430A (3)} \textsf{DS2431 (3)} \textsf{DS2433 (3)} \textsf{DS2502
(3)} \textsf{DS2506 (3)} \textsf{DS28E04 (3)} \textsf{DS28EC20 (3)}} 
\subparagraph*{Switches}\textsf{\textbf{DS2405 (3)} \textsf{DS2406 (3)} \textsf{DS2408
(3)} \textsf{DS2409 (3)} \textsf{DS2413 (3)} \textsf{DS28EA00 (3)}} 
\subparagraph*{Temperature}\textsf{\textbf{DS1822 (3)} \textsf{DS1825 (3)}
\textsf{DS1820 (3)} \textsf{DS18B20 (3)} \textsf{DS18S20 (3)} \textsf{DS1920 (3)} \textsf{DS1921 (3)} \textsf{DS1821 (3)} \textsf{DS28EA00
(3)} \textsf{DS28E04 (3)} \textsf{EDS0064 (3)} \textsf{EDS0065 (3)} \textsf{EDS0066 (3)} \textsf{EDS0067 (3)} \textsf{EDS0068
(3)} \textsf{EDS0071 (3)} \textsf{EDS0072 (3)}} 
\subparagraph*{Humidity}\textsf{\textbf{DS1922 (3)} \textsf{DS2438 (3)} \textsf{EDS0065 (3)} \textsf{EDS0068
(3)}} 
\subparagraph*{Voltage}\textsf{\textbf{DS2450 (3)}} 
\subparagraph*{Resistance}\textsf{\textbf{DS2890 (3)}} 
\subparagraph*{Multifunction (current, voltage,
temperature)}\textsf{\textbf{DS2436 (3)} \textsf{DS2437 (3)} \textsf{DS2438 (3)} \textsf{DS2751 (3)} \textsf{DS2755 (3)} \textsf{DS2756
(3)} \textsf{DS2760 (3)} \textsf{DS2770 (3)} \textsf{DS2780 (3)} \textsf{DS2781 (3)} \textsf{DS2788 (3)} \textsf{DS2784 (3)}} 
\subparagraph*{Counter}\textsf{\textbf{DS2423
(3)}} 
\subparagraph*{LCD Screen}\textsf{\textbf{LCD (3)} \textsf{DS2408 (3)}} 
\subparagraph*{Crypto}\textsf{\textbf{DS1977 (3)}} 
\subparagraph*{Pressure}\textsf{\textbf{DS2406 (3)} -- TAI8570
\textsf{EDS0066 (3)} \textsf{EDS0068 (3)}}  
\paragraph*{Availability}
\altlink{http://www.owfs.org/} 
\paragraph*{Author}
Paul Alfille
(\email{paul.alfille@gmail.com})  
