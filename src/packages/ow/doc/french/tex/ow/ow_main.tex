% Synchronized to r37690
\marklabel{sec:ow}
{
\section {OW -- Bus 1--Wire }
}
\setcounter{secnumdepth}{6}

\newcommand{\IsqC}{$\textrm{I}^{\textrm{2}}\textrm{C}$}
\newcommand{\isqc}{$\textrm{i}^{\textrm{2}}\textrm{c}$}

\subsection{Introduction}

Ce paquetage installe OWFS (voir le chapitre \ref{cap:OW_OWFS}), il offre un accés en écriture
et en lecture par le bus 1--Wire branché à fli4l. Le bus maître 1--Wire est connecté à une
interface série avec un \footnote{adaptateur DS9097U COM Port} ou sur le port USB avec le
\footnote{Bridge DS9490R USB, ou encore le DS1402D-DR8 de (Blue Dot™) de iButton, tous
les adaptateurs basé sur le DS9490 et le module DS2490 USB--1--Wire} de votre ordinateur.
En outre, l'OPT prend également en charge l'adaptateurs \IsqC{} qui sera relié à un serveur-OW.
Vous trouverez plus de détails sur la page du manuel suivante (chapitre: \ref{cap:OW_MANPAGES}).
Le raccordement de l'adaptateur 1--Wire est connecté à l'autre bus 1--Wire côte serveur.

\subsubsection{Matériel}
\paragraph{Le 1--Wire standard}
Le 1--Wire ® ou One-Wire de Maxim (Maxim/Dallas) est une interface série qui utilise un seul fil,
il est utilisé à la fois comme source d'alimentation et comme source d'émission et de réception
pour les données. Cependant, un autre fil est nécessaire pour le \flqq{}retour\frqq{} (la masse).
Chaque composant 1--Wire a un numéro d'identification unique par lequelle il peut être adressée.
Donc, plusieurs composants 1--Wire peuvent être connectés sur le même bus.

\paragraph{Les composants 1--Wire}
Maxim propose différent composant 1--Wire pour les adaptateurs~: serie, USB, \IsqC, des thermomètres,
des commutateurs (jusqu'à 8 canaux), des EEPROM, des Horloges, des convertisseurs A/N, des
potentiomètres digital. On a vraiment tout ce qu'il faut pour la domotique. Une vue d'ensemble
des principaux composants peuvent être trouvées dans l'annexe au chapitre \ref{cap:OW_FAMILYCODE}.
Vous pouvez également connecter des composants iButton ® (NV-RAM, EPROM, EEPROM, sonde de température,
d'humidité, RTC, SHA, Logger).

\paragraph{Le bus 1--Wire}
Le bus 1--Wire est principalement composé de deux lignes torsadées, en observant une topologies
habituelle une distance de 150 m ne devrait pas être un problème. Un câble Ethernet à paire
torsadée de catégorie 5 est souvent utilisé pour le câblage. Différentes approches existent pour
l'affectation des conducteurs. Maxim utilise 6 broches de la prise modulaire et une prise (RJ-11),
il a créé sa propre norme, mais cela ne correspond pas au 8 broches de la prise RJ-45. D'autres
normes sont décrites dans l'annexe du (chapitre \ref{cap:OW_PINADERBELEGUNG}). Vous trouverez
également des informations sur la topologie du bus Maxim sur leur site et tout ce dont vous
avez besoin pour bien utiliser le 1--Wire.

\marklabel{cap:OW_OWFS}
{
\subsubsection{OWFS}
}
OWFS est un \flqq{}fichier système pour One wire\frqq{}. Il s'agit d'un logiciel sous
licence GPL, développé par Paul H. Alfille. Il est basé sur le protocole de communication
1--wire, avec un système de bibliothèque (OWLib), cela forme OWFS avec 1--Wire-Bus comme fichier
système. En outre, le programme propose d'autres implémentations, comme owserver, owshell,
owhttpd, owftpd, owtap et des modules linguistiques pour capi. perl, tcl, php, n'ont pas été
inclus dans la présente adaptation de fli4l. Vous pouvez trouver des détails sur OWFS et
beaucoup de choses intéressantes pour 1--Wire sur le site~: \altlink{ http://owfs.org/} et
\altlink{http://sourceforge.net/projects/owfs/}.

\subsubsection{Fuse}
Fuse signifie \flqq{}système de fichiers pour un espace utilisateur\frqq{}. Fuse permet la mise
en \oe{}uvre d'un système de fichiers entièrement fonctionnel dans l'espace utilisateur.
Avec l'installation de \var{OPT\_OW} pour fli4l, Fuse sera automatiquement chargé en tant
que module kernel au démarrage. Vous pouvez trouver sur le site de Fuse plus d'information~:
\altlink{http://fuse.sourceforge.net/} et \altlink{http://sourceforge.net/projects/fuse/.}

\subsubsection{libusb}
libusb est une bibliothèque USB libre sous licence GPL, qui est nécessaire pour accéder au bus
1--Wire avec un adaptateur USB. Tous ce qui concernent libusb peut être trouvé sur le site~:
\altlink{http://libusb.sourceforge.net/}

\subsection{Licence}
Ce programme est sous licence GNU General Public License, Version 2, Juin 1991 et
peut être librement utilisé, reproduit et modifié dans les conditions indiquées.
Le texte de la licence GNU General Public License peut être trouvé sur le site~:
\altlink{http://www.gnu.org/licenses/gpl.txt}

Une traduction Française non officielle peut être trouvé sur le site~:
\altlink{http://www.linux-france.org/article/these/gpl.html}

Cette traduction est destiné seulement à une meilleure compréhension de
la licence GPL, seul la version anglaise est juridiquement légalement.

\subsection{Garantie et responsabilité}
Ce programme a été réalisé avec la volonté et l'espoir qu'il sera utile.
Néanmoins, il n'y a aucune sorte de garantie - également la garantie de qualité
marchande ou d'adéquation à un usage particulier est rejetée. Pour plus de détails,
reportez-vous à la Licence Publique Générale GNU (GPL). Nous déclinons toute
responsabilité en cas de perte de données, détériorations de matériel ou de logiciel
ou de tout autre dommage.

\subsection{Configuration requise}
En raison de la taille du paquetage \var{OPT\_OW} vous aurez besoin pour l'installation
d'un disque dur ou l'une carte mémoire. Pour plus de détails voir \var{OPT\_HD}.
Pour l'affichage dans le navigateur web du serveur fli4l le paquetage \flqq{}httpd\frqq{}
est nécessaire. Pour plus de détails voir le chapitre \ref{cap:OW_BROWSER}.

\textbf{Remarque~:}

Le contrôleur USB via le module W1 et le Kernel ne fonctionne pas encore (selon
Paul Alfille, responsable du OWFS), en plus il n'a pas été testé dans l'Opt. (Le module
W1 de la version V2.8 pour p16 et p19 ont été testés une seule fois, ensuite,
la connexion et l'évaluation sont complètement différente avec la version
standard, les tests ne seront probablement pas poursuivis).

Pour utiliser l'adaptateur USB des paramètres système doivent être présents dans
"udev" et dans "rules.d". La connexion OWSERVER et OWFS fonctionnera que si les
paramètres des fichiers sont corrects.

L'utilisation des programmes \emph{owshell} et \emph{owhttpd} ne fonctionnent pas
correctement sur certains environnements matériels. Les auteurs du programme tentent
de trouver une solution au problème en collaboration avec Paul Alfille. Si des erreurs
se produisent, vous pouvez essayer de poster sur le forum fli4l avec une description
détaillée de votre problème.

\subsection{Installation}
Après avoir décompacté l'archive tar.gz dans le répertoire fli4l vous devez paramétrer
le fichier /config/ow.txt selon vos besoins. En plus, pour utiliser l'interface Web
vous devez activer le serveur Web httpd via \var{OPT\_HTTPD}='yes' (voir le chapitre
\ref{cap:OW_SONSTIGEVARIABLEN}). Si vous utilisez RRDTool pour enregistrer les
valeurs du système, vous devez paramétrer le fichier /config/rrdtool.txt
(voir le chapitre \ref{cap:OW_RRDTOOL}).

\subsection{Configuration}
Exemple de configuration sans les commentaires, vous trouverez d'autres explications
ci-dessous~:

\begin{example}
\begin{verbatim}
    OPT_OW='yes'                      # install OPT_OW (yes/no)
    OW_USER_SCRIPT=''                 # e.g. 'usr/local/bin/ow-user-script.sh'

    OW_OWFS='yes'                     # start owfs (yes/no)
    OW_OWFS_DEV='usb'                 # usb*, ttyS*, ip:port, etc.
    OW_OWFS_GROUP_N='4'                           # number of groups
    OW_OWFS_GROUP_1_NAME='1--Wire an USB'         # name of first group
    OW_OWFS_GROUP_1_PORT_N='2'                    # number of ports of device
    OW_OWFS_GROUP_1_PORT_1_ID='81.70D42A000000/ID'      # ID of device
    OW_OWFS_GROUP_1_PORT_1_ALIAS='ID'                   # alias of ID
    OW_OWFS_GROUP_1_PORT_2_ID='81.70D42A000000/Admin/*' # admin-access
    OW_OWFS_GROUP_1_PORT_2_ALIAS='Admin/'               # alias of admin

    OW_OWFS_GROUP_2_NAME='Heizung'
    OW_OWFS_GROUP_2_PORT_N='7'
    OW_OWFS_GROUP_2_PORT_1_ID='3A.F6E401000000/PA'
    OW_OWFS_GROUP_2_PORT_1_ALIAS='1. Umwälzpumpe'
    OW_OWFS_GROUP_2_PORT_2_ID='3A.F6E401000000/PB'
    OW_OWFS_GROUP_2_PORT_2_ALIAS='2. Ladepumpe'
    OW_OWFS_GROUP_2_PORT_3_ID='10.651BA9010800/temp'
    OW_OWFS_GROUP_2_PORT_3_ALIAS='4. Rücklauftemperatur'
    OW_OWFS_GROUP_2_PORT_4_ID='10.DEF0A8010800/temp'
    OW_OWFS_GROUP_2_PORT_4_ALIAS='3. Vorlauftemperatur'
    OW_OWFS_GROUP_2_PORT_5_ID='3A.F6E401000000/Admin/*'
    OW_OWFS_GROUP_2_PORT_5_ALIAS='Admin/Switch-'
    OW_OWFS_GROUP_2_PORT_6_ID='10.DEF0A8010800/Admin/*'
    OW_OWFS_GROUP_2_PORT_6_ALIAS='Admin/VLT-'
    OW_OWFS_GROUP_2_PORT_7_ID='10.651BA9010800/Admin/*'
    OW_OWFS_GROUP_2_PORT_7_ALIAS='Admin/RLT-'

    OW_OWFS_GROUP_3_NAME='Solaranlage'
    OW_OWFS_GROUP_3_PORT_N='3'
    OW_OWFS_GROUP_3_PORT_1_ID='1C.7F6CF7040000/P0'
    OW_OWFS_GROUP_3_PORT_1_ALIAS='1. Ladepumpe'
    OW_OWFS_GROUP_3_PORT_2_ID='1C.7F6CF7040000/P1'
    OW_OWFS_GROUP_3_PORT_2_ALIAS='2. Ventil'
    OW_OWFS_GROUP_3_PORT_3_ID='1C.7F6CF7040000/Admin/*'
    OW_OWFS_GROUP_3_PORT_3_ALIAS='Admin/Switch-'
	
	OW_OWSHELL='yes'
	OW_OWSHELL_RUN='yes'
	OW_OWSHELL_DEV='usb'
	OW_OWSHELL_PORT='127.0.0.1:4304'

	OW_OWHTTPD='yes'
	OW_OWHTTPD_RUN='yes'
	OW_OWHTTPD_DEV='127.0.0.1:4304'
	OW_OWHTTPD_PORT='8080'
\end{verbatim}
\end{example}

Les variables suivantes sont à configurer dans le fichier /config/ow.txt~:

\begin{description}
\config{OPT\_OW}{OPT\_OW}{OPTOW}
Le réglage par défaut est OPT\_OW='no', le paquetage ne sera pas installé.
Avec \var{OPT\_OW}='yes', le paquetage est activé.

\config{OW\_USER\_SCRIPT}{OW\_USER\_SCRIPT}{OWUSERSCRIPT}
Avec cette variable optionnelle vous définissez le chemin et le nom du fichier pour
le contrôle en arrière-plan. De plus amples détails peuvent être trouvés dans le
chapitre \ref{cap:OW_OWUSERSCRIPT}.

\config{OW\_OWFS}{OW\_OWFS}{OWOWFS}
WFS offre un accès facile au bus 1--wire via l'interface web fli4l. Si vous spécifiez
\var{OW\_OWFS}='yes' un système de fichiers dans le chemin par défaut '/var/run/ow' est
généré à partir de fuse. Le bus 1--wire est alors mappé. Les répertoires créés
dans le système de fichiers sont triés par numéro d'identification (voir l'annexe
\ref{cap:OW_FAMILYCODE}) des chips (ou puces). Au sujet des codes les familles des composants
une correspondante systématique peut facilement être créée.

\config{OW\_OWFS\_DEV}{OW\_OWFS\_DEV}{OWOWFSDEV}
Avec la variable \var{OW\_OWFS\_DEV} vous définissez l'interface du PC sur lequel l'adaptateur
1--Wire sera connecté.
\begin{tabular}{|l|l|p{0.5\textwidth}|}
\hline
\textbf{Interface du PC} & \textbf{Paramètre de la variable} & \textbf{Exemple} \\
\hline
serie          & ttyS*          & ttyS0 = COM1, ttyS1 = COM2 \\
\hline
\multirow{3}{*}{}{USB}
                 & ttyUSB*        & ttyUSB1 = premier adaptateur USB \\
\cline{2-3}
          \latex{&} usb           & usb = premier adaptateur USB \\
\cline{2-3}
          \latex{&} usb[2-9]      & usb3 = troisième adaptateur USB \\
\hline
\IsqC{}          & \isqc{}-[0-9]  & \isqc{}-0 = premier \IsqC{} Port \\
\hline
\multirow{2}{*}{}{Simulation}
                 & fake           & \multirow{2}{*}{}{Pour l'utilisation les modes 
                                    '\var{FAKE}' et '\var{TESTER}' vous devez
									paramétrer la variable \var{OW\_OWFS\_FAKE}
									ou \var{OW\_OWFS\_TESTER} avec le code de
									la famille du composant valide, voir le
									chapitre \ref{cap:OW_SONSTIGEVARIABLEN}} \\
\cline{2-2}
     \latex{&} tester \latex{&} \\
\hline
\end{tabular}

\config{OW\_OWFS\_GROUP\_N}{OW\_OWFS\_GROUP\_N}{OWOWFSGROUPN}
Avec la variable \var{OW\_OWFS\_GROUP\_N} vous indiquez le nombre de groupe qui sera
affiché dans le navigateur ainsi que des entrées et sorties ratachées à ce groupe,
par exemple pour contrôler d'un système solaire, dans la variable OW\_OWFS\_GROUP\_NAME
vous pouvez indiquer un nom pour le système.

\configlabel{OW\_OWFS\_GROUP\_x\_PORT\_N}{OWOWFSGROUPxPORTN}
\configlabel{OW\_OWFS\_GROUP\_x\_PORT\_x\_ALIAS}{OWOFSGROUPxPORTxALIAS}
\config{OW\_OWFS\_GROUP\_x\_PORT\_N OW\_OWFS\_GROUP\_x\_PORT\_x\_ID OW\_OWFS\_\_GROUP\_x\_PORT\_x\_ALIAS}{OW\_OWFS\_GROUP\_x\_PORT\_x\_ID}{OWOWFSGROUPxPORTxID}
Avec la variable \var{OW\_OWFS\_GROUP\_x\_PORT\_N} vous indiquez le nombre de port
pour le groupe. Avec les deux variables suivantes \var{OW\_OWFS\_GROUP\_x\_PORT\_x\_ID} et \\
\var{OW\_OWFS\_GROUP\_x\_PORT\_x\_ALIAS} vous indiquez le nom d'identification et le nom
d'alias pour le composant 1--Wire.

Si vous voulez supprimer l'affichage de certaines données dans l'interface web, 
soit parce que le port d'un composant n'a pas été connecté ou que l'administration
du groupe n'est plus nécessaire une fois la configuration terminée, vous pouvez faire
précéder le nom d'un point d'exclamation (!).

\config{OW\_OW\_SHELL}{OW\_OW\_SHELL}{OWOWSHELL}
Avec cette variable vous activez le "serveur" OWFS pour fournir au Bus-OWFS de multiples
applications (OWFS et OWHTTPD). Aucune autre application ne doit être définie directement
sur l'interface de l'adaptateur, mais elle sera associée au serveur.
\config{OW\_OW\_SHELL\_RUN}{OW\_OW\_SHELL\_RUN}{OWOWSHELLRUN}
Avec cette variable vous lancez le service serveur au démarrage.
\config{OW\_OW\_SHELL\_DEV}{OW\_OW\_SHELL\_DEV}{OWOWSHELLDEV}
Dans cette variable vous indiquez le périphérique sur lequel le serveur peut accéder
(matériel).
\config{OW\_OW\_SHELL\_PORT}{OW\_OW\_SHELL\_PORT}{OWOWSHELLPORT}
Dans cette variable vous indiquez l'adresse IP et le port du serveur que vous utilisez.
Il est logique d'indiquer ici l'adresse 127.0.0.1 localhost. Le port 4304 par défaut
(port OWFS) est utilisé pour le serveur. Cette adresse sera stockée de manière permanente
dans le paquetage RRDTool. Si RRDTool veut utiliser ces valeurs, elles doivent être enregistrées.
\config{OW\_OWHTTPD}{OW\_OWHTTPD}{OWOWHTTPD}
Avec cette variable vous activez le serveur Web pour OWFS.
\config{OW\_OWHTTPD\_RUN}{OW\_OWHTTPD\_RUN}{OWOWHTTPDRUN}
Avec cette variable vous lancez le serveur Web au démarrage.
\config{OW\_OWHTTPD\_READONLY}{OW\_OWHTTPD\_READONLY}{OWOWHTTPDREADONLY}
Dans cette variable vous autorisez l'accès en écriture pour les composants dans OWFS.
\config{OW\_OWHTTPD\_DEV}{OW\_OWHTTPD\_DEV}{OWOWHTTPDDEV}
Dans cette variable vous indiquez le périphérique sur lequel le serveur Web peut
accéder. Si vous utilisez le OW\_OWSHELL (serveur) un seul périphérique peut être
consulté ici.
\config{OW\_OWHTTPD\_PORT}{OW\_OWHTTPD\_PORT}{OWOWHTTPDPORT}
Dans cette variable vous indiquez le port HTTP.

Exemple de configuration~:
\begin{example}
\begin{verbatim}
    OW_OWFS_GROUP_x_PORT_x_ID='29.57D305000000/P6'
    OW_OWFS_GROUP_x_PORT_x_ALIAS='EA-Modul/!P6'        # Signal sera supprimé
    OW_OWFS_GROUP_x_PORT_x_ID='29.57D305000000/Admin/*'
    OW_OWFS_GROUP_x_PORT_x_ALIAS='EA-Modul/Admin/!'    # Chemin Admin totalement
                                                       # désactivé
\end{verbatim}
\end{example}

Une description plus détaillée de la configuration peut être trouvée dans l'annexe
\flqq{}\ref{cap:OW_MANPAGES}\frqq{} et dans~: \altlink{http://owfs.org/index.php?page=owfs}.

\end{description}

\marklabel{cap:OW_SONSTIGEVARIABLEN}
{
\subsubsection{Variables divers}
}
Les variables suivantes peuvent être configuré dans le fichier config/ow.txt si cela
est nécessaire~:

\begin{description}
\config{OW\_LOG\_DESTINATION}{OW\_LOG\_DESTINATION}{OWLOGDESTINATION}
 Permet d'indiquer l'indice des erreurs et de l'état des sorties.

\begin{verbatim}
    0 = mixed (1 et 2)
    1 = syslog
    2 = stderr
    3 = off
\end{verbatim}

La valeur par défaut est '1'.

\config{OW\_LOG\_LEVEL}{OW\_LOG\_LEVEL}{OWLOGLEVEL}
 Permet d'indiquer le niveau de journalisation (1-9) détermine la quantité d'erreur et
 de l'état des sorties, avec~:

\begin{verbatim}
    1 = silencieux et 9 = bavard
\end{verbatim}

La valeur par défaut est '1'.

\config{OW\_TEMP\_SCALE}{OW\_TEMP\_SCALE}{OWTEMPSCALE}
 Permet d'indiquer l'échelle de température disponibles.
\begin{verbatim}
    C = "Celsius"
    F = "Fahrenheit"
    K = "Kelvin"
    R = "Rankine"
\end{verbatim}

La valeur par défaut est 'C'.

\config{OW\_REFRESH\_INTERVAL}{OW\_REFRESH\_INTERVAL}{OWREFRESHINTERVAL}
 Permet d'indiquer le taux de rafraichissement du HTTP de fli4l en seconde '0' = aucun
 rafraichissement.

La valeur par défaut est '10'.

\config{OW\_OWFS\_FAKE}{OW\_OWFS\_FAKE}{OWOWFSFAKE}
 Permet de faire une simulation aléatoire de l'adaptateur 1--Wire. Si vous avez
 plusieurs codes de famille de composants, vous devez les séparés par un espace.
 Les conditions de simulation sont purement fortuite. Cette option ne peut pas
 être activée simultanément avec le mode 'TESTER'.

\config{OW\_OWFS\_TESTER}{OW\_OWFS\_TESTER}{OWOWFSTESTER}
 Permet de faire une simulation systématique de l'adaptateur 1--Wire. Si vous avez
 plusieurs codes de famille de composants, vous devez les séparés par un espace.
 Les conditions de simulation doivent avoir des valeurs réalistes. Cette option
 ne peut pas être activée simultanément avec le mode 'FAKE'.

\config{OW\_OWFS\_RUN}{OW\_OWFS\_RUN}{OWOWFSRUN}
 Permet à owfs de démarré automatiquement lors du démarrage du routeur. La valeur
 par défaut est 'yes', si vous indiquez 'no', l'application doit être lancée
 manuellement.

\config{OW\_OWFS\_READONLY}{OW\_OWFS\_READONLY}{OWOWFSREADONLY}
 Si vous indiquez 'yes', l'état de l'adaptateur peut être lu par owfs
 mais pas être en écriture.

La valeur par défaut est 'no'.

\config{OW\_OWFS\_PATH}{OW\_OWFS\_PATH}{OWOWFSPATH}
 Ici vous indiquez le répertoire racine de l'arborescence des répertoires de fuse.
 La valeur par défaut est '/var/run/ow'. Le répertoire sélectionné devrait pour
 des raisons de performances du système, se trouver nécessairement sur le RAMdisk!

\config{OW\_CACHE\_SIZE}{OW\_CACHE\_SIZE}{OWCACHESIZE}
 Permet de régler la taille maximale de la mémoire cache en [octets] sur le système,
 si vous avez très peu de RAMdisk.

 La valeur par défaut '0' supprime toute limitation.

\config{OW\_USER\_SCRIPT\_INTERVAL}{OW\_USER\_SCRIPT\_INTERVAL}{OWUSERSCRIPTINTERVAL}
 Permet de spécifier, en seconde, la durée d'attente entre deux passages de script créé
 par l'utilisateur. La valeur '0' doit être utilisée seulement si dans le script
 d'utilisateur a indiqué la commande 'sleep'.

\config{OW\_DEVICE\_LIB}{OW\_DEVICE\_LIB}{OWDEVICELIB}
 Permet de spécifier, le chemin absolu et le nom du fichier de la bibliothèque des composants
 sur le routeur. Si vous utilisez une valeur autre que la valeur par défaut
 '/srv/www/include/ow-device.lib', la bibliothèque de composants ne sera pas écrasée,
 les changements personnalisés de la bibliothèque de composants seront conservés.

\config{OW\_INVERT\_PORT\_LEDS}{OW\_INVERT\_PORT\_LEDS}{OWINVERTPORTLEDS}
 Permet d'inverser l'état des LEDs du port I/O (latch*, sensed*, PIO*).

La valeur par défaut est 'no'.
\end{description}

\subsubsection{Variable sans documentation}

Les variables suivantes ne sont pas documentées~:
\begin{description}
\config{OW\_MODULE\_CONF\_FILE}{OW\_MODULES\_CONF\_FILE}{OWMODULECONFFILE}
\config{OW\_USER\_SCRIPT\_STOP}{|OW\_USER\_SCRIPT\_STOP}{OWUSERSCRIPTSTOP}
\config{OW\_SCRIPT\_WRAPPER}{OW\_SCRIPT\_WRAPPER}{OWSCRIPTWRAPPER}
\config{OW\_MENU\_ITEM}{OW\_MENU\_ITEM}{OWMENUITEM}
\config{OW\_RIGHTS\_SECTION}{OW\_RIGHTS\_SECTION}{OWRIGHTSSECTION}

\config{OW\_OWFS\_PID\_FILE}{OW\_OWFS\_PID\_FILE}{OWOWFSPIDFILE}
\config{OW\_OWFS\_GROUP\_x\_NAME}{OW\_OWFS\_GROUP\_x\_NAME}{OWOWFSGROUPxNAME}
\config{OW\_REFRESH\_FILE}{OW\_REFRESH\_FILE}{OWREFRESHFILE}
\config{OW\_REFRESH\_TEMP}{OW\_REFRESH\_TEMP}{OWREFRESHTEMP}
\config{OW\_ALIAS\_FILE}{OW\_ALIAS\_FILE}{OWALIASFILE}
\config{OW\_CSS\_FILE}{OW\_CSS\_FILE}{OWCSSFILE}

\config{OW\_OWHTTPD\_FAKE}{OW\_OWHTTPD\_FAKE}{OWOWHTTPDFAKE}
\config{OW\_OWHTTPD\_TESTER}{OW\_OWHTTPD\_TESTER}{OWOWHTTPDTESTER}
\config{OW\_OWHTTPD\_PID\_FILE}{OW\_OWHTTPD\_PID\_FILE}{OWOWHTTPDPIDFILE}
\end{description}

\subsection{Fonctionnement dans le navigateur et sur la console}

\marklabel{cap:OW_BROWSER}
{
\subsubsection{Navigateur}
}
\paragraph{Serveur Web}
Vous pouvez configurer en option dans fli4l un serveur Web avec (opt\_httpd), avec cette
option vous avez la possibilité exécuter vos propres script Shell/CGI depuis n'importe
quel navigateur sur le réseau. C'est se que nous avons activé ici. Pour utiliser le serveur web
vous devez configurer le fichier config/httpd.txt en conséquence.

Dans le paquetage \var{OPT\_OW} une application pour le navigateur est inclus.
Elle est installée que lorsque dans le fichier /config/ow.txt la variable \var{OW\_OWFS}='yes'
est activée. Le script se trouve dans le répertoire /srv/www/admin/ow.cgi ou dans le répertoire
d'installation de fli4l sous \verb!fli4l-version\opt\files\srv\www\admin\ow.cgi~! L'élément
associé dans le menu du navigateur apparaît sous le nom \flqq{}Opt/1--Wire-Bus\frqq{}.

\paragraph{Présentation}
dans l'onglet \flqq{}Statut\frqq{} vous pouvez voir l'adaptateur connecté au Bus 1--Wire, avec
les groupes affichées et structurés en arborescente selon la configuration du fichier config/ow.txt.
Vous pouvez ouvrir les groupes respectif par un \flqq{}clic\frqq{} droit. Les valeurs configurées
seront affichées. Dans la structure Admin vous avez toute la bibliothèque de composants (voir 8.4)
pour définir les paramètres des composants. En ce qui concerne l'importance de ce paramètre, s'il
vous plaît, vous devez vous référer aux fiches de données Maxim et aux man-page accompagné (en Anglais).

Dans l'onglet \flqq{}Admin\frqq{} qui apparaît uniquement en mode administrateur, les applications
sélectionnées peuvent être activés ou désactivés

Les LED affichées indiquent par leurs couleurs les conditions suivantes~:
LED vert = inactif (au repos)
LED rouge = actif (en fonctionnement)
LED jaune = inactif (alerte)

Les boutons de contrôle sont utilisés pour commuter les ports affectés. Une icône affiche également
l'état de commutation  En ce qui concerne les autorisations, (voir 8.1).

\subsubsection{Console}
Il est possible d'utiliser les requêtes pour le contrôle des capteurs et des actionneurs sur la console
de fli4l ou via l'accès à distance (c.-à-dire WinSCP, Putty).

Par exemple~:
\begin{itemize}
\item cat /var/run/ow/10.DEF0A8010800/temperature \\
      on demande à partir d'un DS19S20 de mesurer la température.
\item echo "1" $>$ /var/run/ow/1C.7F6CF7040000/PIO.O \\
      commute la sortie  avec d'un DS28E04-100 (double switch).
\item echo "0" $>$ /var/run/ow/1C.7F6CF7040000/PIO.O \\
      commute une seule fois la sortie.
\end{itemize}

Vous trouverez une description plus détaillée dans l'annexe \flqq{}\ref{cap:OW_MANPAGES}\frqq{}
et ici~: \altlink{http://owfs.org/index.php?page=owfs}

\subsection{Fonctionnalités avancées}
\subsubsection{Attribution des droits}
L'attribution des droits de l'utilisateur est à régler dans l'interface Web de fli4l,
voir la documentation doc/french/pdf/httpd.pdf. \\
Cette attribution des droits est utilisé également dans \var{OPT\_OW}. Pour utiliser
les droits de OW, les paramètres suivants peuvent être spécifiées dans le fichier config/httpd.txt
pour la partie \flqq{}ow\frqq{}~:

admin = Tous les droits \\
exec = Exécuter des commandes commutées entrées et sorties, visualise les données \\
view = Voir les données

Vous pouvez activée ou désactivée le script utilisateur dans l'onglet Admin de owfs
il ne sera pas affichée dans le mode \flqq{}exec\frqq{} et \flqq{}view\frqq{}.
Toutes les paramètres d'autorisations dans \flqq{}Admin\frqq{} seront cachés.

\subsubsection{Bibliothèque de composants}
En raison de la variété des composants fourni par MAXIM - une bibliothèque de composants
pour 1--Wire a été créée. Le script pour cette bibliothèque se trouve dans le répertoire
/srv/www/include/ow-device.lib ou dans le répertoire d'installation de fli4l sous
\verb!fli4l-version\opt\files\srv\www\include\ow-device.lib~! La bibliothèque contient déjà
des éléments les plus importants. Vos propres dispositifs peuvent aussi être spécifiées
selon la nomenclature utilisée, vous pouvez ensuite envoyer dans le newsgroup de fli4l
'spline.fli4l.opt' pour les autres utilisateurs de fli4l. Ainsi les composants de la bibliothèque
seront affichés dans le navigateur. Le script pour la bibliothèque peut être au choix, soit
être utilisé par le programme \flqq{}WinSCP\frqq{} afin de le tester sur fli4l, ou soit vous
pouvez l'édité pour une modification permanent dans le répertoire d'installation de fli4l.

\marklabel{cap:OW_OWUSERSCRIPT}
{
\subsubsection{OW\_USER\_SCRIPT}
}

Le script se trouve dans le répertoire /usr/local/bin/ow-user-script.sh ou 
dans le répertoire d'installation de fli4l sous
\verb!fli4l-version\opt\files\usr\local\bin\ow-userscript.sh~! Il peut être
adapté en fonction de vos besoins pour les applications il est à surveiller
et/ou contrôler. L'avantage du script est le fait que même les contrôles 
mportants et complexes sont possibles sur le matériel existant.

\marklabel{cap:OW_RRDTOOL}
{
\subsubsection{RRDTool}
}
\paragraph{Interface}
Les données recueillies par le bus 1--Wire peuvent être enregistrées pour l'opt
\flqq{}RRDTool\frqq{} de fli4l, elles seront ensuite présentées graphiquement.
Cette opt doit déjà contenir les interfaces nécessaires. Owfs (voir /config/ow.txt)
doit être installé. Lors de l'installation de RRDTool vous devez configurer les
variables dans le fichier /config/rrdtool.txt selon vos besoins. Il est impératif que
la variable \var{OW\_SHELL} soit activé avec les valeurs 127.0.0.1:4304 pour le port
dans le paquetage OW, cela permettra de récupérer les données de tous les capteurs pour
afficher les graphiques.
RRDTool détecte tous les capteurs du système et enregistre les données pour les afficher
en plusieurs graphiques. La gestion des capteurs se fait selon l'identification du
capteur (par le numéro unique du capteur). De plus, un second groupe sera représenté
selon la définition des groupes dans le paquetage, L'ordre des groupes affichés correspond
à l'ordre dans la configuration. Tous les capteurs de températures de chaque groupe seront
combinés pour faire un seul graphique, ils seront indiqués dans l'ordre des numéros de port
dans la configuration.

\subsection{Information}
Nous serons heureux d'avoir des retours sur le fonctionnement de ce paquetage, même s'il
fonctionne sans aucun problème.

Nous vous souhaitons beaucoup de plaisir avec 1--Wire~:

Klaus le Tigre \email{der.tiger.opt-ow@arcor.de}\\
Karl M. Weckler \email{news4kmw@web.de}\\
Roland Franke \email{fli4l@franke-prem.de}
