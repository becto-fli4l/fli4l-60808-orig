% Last Update: $Id$
\marklabel{sec:opt-pcmcia }
{
\section {PCMCIA - PC-Card Unterstützung}
}

           
\subsection{PCMCIA-Treiber}\index{OPT\_PCMCIA}

    fli4l kann auch mit PCMCIA-Karten zusammenarbeiten. Bei
        \var{OPT\_\-PCMCIA}='yes' werden die entsprechenden Basis-Treiber installiert.
    Welche konkreten Kartentreiber verwendet werden sollen, wird z.B.
    über \jump{NETDRVx}{\var{NET\_\-DRV\_\-x}} eingestellt.

\begin{description}
\config{PCMCIA\_PCIC}{PCMCIA\_PCIC}{PCMCIAPCIC} - PCMCIA Socket-Driver

Es kann dabei gewählt werden: 'i82365' oder 'tcic' für PCMCIA Bridges, sowie 'yenta\_socket'
und 'i82092' für Cardbus Bridges.

        Standard-Einstellung: \var{PCMCIA\_\-PCIC}='i82365'


\config{PCMCIA\_PCIC\_OPTS}{PCMCIA\_PCIC\_OPTS}{PCMCIAPCICOPTS} - Optionen für den PCMCIA Socket-Driver

        Standard-Einstellung: \var{PCMCIA\_\-PCIC\_\-OPTS}=''

        Mögliche Einstellungen:
                poll\_interval=n        n in je 10 Millisekunden - Sinnvoller Wert: 1000
                                        Stellt das Abfrageintervall für Kartenwechsel ein
                irq\_list=x,y,z,...     Eine Liste der zu verwendenden Interrupts


\configlabel{PCMCIA\_MISC\_x}{PCMCIAMISCx}
\config{PCMCIA\_MISC\_N PCMCIA\_MISC\_x}{PCMCIA\_MISC\_N}{PCMCIAMISCN}
        Anzahl der zusätzlich zu ladenden PCMCIA-Module:
                serial\_cs		für Modems und Combo-Karten
                parport\_cs             Druckerschnittstellen
                
\end{description}
