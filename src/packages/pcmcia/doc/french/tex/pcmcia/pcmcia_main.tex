% Do not remove the next line
% Synchronized to r29817

\marklabel{sec:opt-pcmcia}
{
\section {PCMCIA - Supporte les cartes PC}
}

           
\subsection{Pilote PCMCIA}\index{OPT\_PCMCIA}

    fli4l peut également travailler avec les cartes PCMCIA. Si vous activez la variable
    \var{OPT\_\-PCMCIA}='yes' vous devez installer les pilotes correspondants. Quels
    pilotes de cartes concrets doit on utiliser, on les paramètres dans la variable
    \jump{NETDRVx}{\var{NET\_\-DRV\_\-x}} du fichier config/base.txt.

\begin{description}
\config{PCMCIA\_PCIC}{PCMCIA\_PCIC}{PCMCIAPCIC} - PCMCIA Socket-Driver

    Voici les pilotes pour les contrôleurs de bus PCMCIA, vous pouvez choisir entre~: 'i82365'
    ou 'tcic' pour les PCMCIA Bridges, ainsi que 'yenta\_socket' et 'i82092' pour les
    cardbus Bridges.

        Installation par défaut~: \var{PCMCIA\_\-PCIC}='i82365'


\config{PCMCIA\_PCIC\_OPTS}{PCMCIA\_PCIC\_OPTS}{PCMCIAPCICOPTS} - Options pour Socket-Driver PCMCIA

        Installation par défaut~: \var{PCMCIA\_\-PCIC\_\-OPTS}=''

        Réglage possible~:
                poll\_interval=n        n par 10 Millisecondes - La valeur logique 1000
                                        durée logique pour un changement de carte pcmcia \\
                irq\_list=x,y,z,...     Pour indiquer une liste interruptions (IRQ) à utiliser


\configlabel{PCMCIA\_MISC\_x}{PCMCIAMISCx}
\config{PCMCIA\_MISC\_N PCMCIA\_MISC\_x}{PCMCIA\_MISC\_N}{PCMCIAMISCN}

    Dans la première variable on indique le nombre de module-PCMCIA, dans 
    la seconde les modules-PCMCIA~:

    serial\_cs		Pour modem et cartes Combo. \\
    parport\_cs		Pour interface parallèle (imprimante). \\

\end{description}
