% Synchronized to r29817
\marklabel{sec:opt-pcmcia }
{
\section {PCMCIA - PC-Card Support}
}

           
\subsection{PCMCIA Drivers}\index{OPT\_PCMCIA}

    fli4l can work with PCMCIA cards as well. Specify
        \var{OPT\_\-PCMCIA}='yes' to install according base drivers.
    The card driver to be used concretely is set by
    \jump{NETDRVx}{\var{NET\_\-DRV\_\-x}}.

\begin{description}
\config{PCMCIA\_PCIC}{PCMCIA\_PCIC}{PCMCIAPCIC} - PCMCIA socket driver

It can be chosen between: 'i82365' or 'tcic' for PCMCIA bridges and 'yenta\_socket'
res. 'i82092' for Cardbus bridges.

        Default setting: \var{PCMCIA\_\-PCIC}='i82365'


\config{PCMCIA\_PCIC\_OPTS}{PCMCIA\_PCIC\_OPTS}{PCMCIAPCICOPTS} - Options for the PCMCIA socket driver

        Default setting: \var{PCMCIA\_\-PCIC\_\-OPTS}=''

        Possible Settings: \\
                poll\_interval=n        n in intervals of 10 milliseconds - reasonable value: 1000
                                        Sets polling interval for card changes. \\
                irq\_list=x,y,z,...     A list of interrupts to be used \\


\configlabel{PCMCIA\_MISC\_x}{PCMCIAMISCx}
\config{PCMCIA\_MISC\_N PCMCIA\_MISC\_x}{PCMCIA\_MISC\_N}{PCMCIAMISCN}
        Number of PCMCIA modules to be loaded additionally:
                serial\_cs		for modems and combo cards
                parport\_cs             printer interface cards
                
\end{description}
