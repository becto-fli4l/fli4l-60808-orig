% Last Update: $Id$
\marklabel{sec:opt-lcd4linux }
{
  \section {LCD4Linux - Anzeige von Statusinformationen über LC-Display}
}

\subsection{Einleitung}

    Mit diesem Paket ist es möglich, ein LCD-Modul über verschiedene 
    Schnittstellen an den fli4l Rechner anzuschließen 
    (parallel und USB sind direkt über die fli4l-Konfiguration möglich, 
    seriell muss über den 'Experten'-Modus konfiguriert werden, siehe 
    entsprechenden Abschnitt weiter unten).

    Auf diesem Display werden Informationen wie Datum, externe IP-Adresse, 
    die aktuellen Loadwerte und natürlich auch der ISDN- oder DSL-Durchsatz 
    für Up- und Download in kb/s und ein Balken angegeben.

    Es können durch den Benutzer Icon's frei definiert werden, die animiert sein 
    können, oder auch erst bei erreichen von definierbaren Schwellenwerten 
    erscheinen (z.B. ein Blitz bei hoher CPU Auslastung).

    Es ist möglich Balken darzustellen (auch zwei Balken in einer Zeile), die 
    sich in jegliche Richtung bewegen können (links, rechts, von oben oder unten).

    Den Gestaltungsmöglichkeiten sind kaum Grenzen gesetzt - die meisten Grenzen 
    setzt die Größe des Display's ;-)

    Zudem können die Anzeigemöglichkeiten durch Plugins nahezu grenzenlos 
    erweitert werden.

    Es ist auch ein IMON-Plugin vorhanden, mit dem Statusinformationen abgefragt 
    werden können - d.h. Dauer Onlineverbindung, Übertragungsrate und Menge, usw.

    Zum Einsatz kommt dabei eine Software namens lcd4linux in der 
    Version 0.10.1-CVS (vom 21.02.2007).

    Auf der Homepage des Projektes
    	(\altlink{https://ssl.bulix.org/projects/lcd4linux})
    sind einige Bilder der unterstützten Displays und detailierte Informationen
    zu den jeweiligen Konfigurationsmöglichkeiten im 'Experten'-Modus.


\subsection{Konfiguration des Displaytreibers}
\configlabel{OPT\_LCD4LINUX}{OPTLCD4LINUX}

    Möchte man das LCD4Linux-Paket einsetzen, sind noch folgende Variablen zu
    setzen:

\begin{example}
\begin{verbatim}
        OPT_LCD4LINUX='yes'  (Standard-Einstellung: OPT_LCD4LINUX='no')
\end{verbatim}
\end{example}

    Zuerst wird das gewünschte Display mit seinen jeweiligen möglichen Zusatzoptionen
    und, sofern möglich, die Anschlussart konfiguriert. Danach dann, was wo auf
    dem Display erscheinen soll.

    Sollten Sie ein Display haben welches nicht vordefiniert ist und es wird von
    lcd4linux unterstützt (siehe dazu obige WebSeite des Projektes), so können
    Sie den 'Experten'-Modus aktivieren und direkt selbst an der eigentlichen
    lcd4linux-Konfiguration die gewünschten Änderungen vornehmen.



\begin{description}
\config{LCD4LINUX\_DRV}{LCD4LINUX\_DRV}{LCD4LINUXDRV}

  Wählt den gewünschten Treiber aus.

\begin{example}
\begin{verbatim}
  Mögliche Treiber sind:
    HD44780      - Hitachi HD44780 basierte Displays und kompatible
    LCD2USB      - LCD2USB - http://www.harbaum.org/till/lcd2usb/
    GLCD2USB     - GLCD2USB - http://www.harbaum.org/till/glcd2usb/
    HP12542R     - Hyunday HP12542R-DYO
    CTINCLUD     - http://www.ct-maeusekino.de/
    FW8888       - Allnet FW8888 internal Display
    DPF          - Patched Digital Photo Frame http://geekparadise.de/tag/ax206/
    EA232Graphic - Some RS232 Graphic Displays made by Electronic Assembly
    M50530       - Mitsubishi M50530 mit z.B. 24x8 Zeichen
\end{verbatim}
\end{example}


\config{LCD4LINUX\_DRV\_MODEL}{LCD4LINUX\_DRV\_MODEL}{LCD4LINUXDRVMODEL}

  Welches Display-Modell liegt vor.

\begin{example}
\begin{verbatim}
  Für den Treiber HD44780 gibt es folgende Modelle zur Auswahl:

  generic	- Standard HD44780 Display (standard)
  Noritake	- Wie generic aber mit software-einstellbarem backlight
  Soekris	- Spezielles Interface für Soekris-Rechner mit Busy-Check im 4-Bit Modus
  HD66712	- ähnlich HD44780 aber mit einem leicht anderen Ram-Layout. Auch für KS0073
  LCM-162	- Spezielles Interface in Nexcom Blade Servern
\end{verbatim}
\end{example}

\begin{example}
\begin{verbatim}
  Für den Treiber EA232Graphic gibt es folgende Modelle:

  GE120-5NV24	120x32
  GE128-6N9V24	128x64
  GE128-6N3V24	128x64
  GE128-7KV24	128x128
  GE240-6KV24	240x64
  GE240-6KCV24	240x64
  GE240-7KV24	240x128
  GE240-7KLWV24	240x128
  GE240-6KLWV24	240x64
  KIT120-5	120x32
  KIT129-6	128x64
  KIT160-6	160x80
  KIT160-7	160x128
  KIT240-6	240x64
  KIT240-7	240x128
  KIT320-8	320x240
\end{verbatim}
\end{example}

\config{LCD4LINUX\_DRV\_PORT}{LCD4LINUX\_DRV\_PORT}{LCD4LINUXDRVPORT}

  Wählt den zu verwendenden Port aus.

\begin{example}
\begin{verbatim}
  Für den Treiber HD44780 gibt es folgende Ports:

  parport0 bis parport2 um den Parallel-Port über die Linux-Parport-API anzusprechen.
  0x278, 0x378, 0x3BC um den Parallel-Port direkt anzusprechen
  i2c-0 bis i2x-9 für I2C Bus gesteuerte Displays.

  Um i2c Nutzen zu können, muß LCD4LINUX\_DRV\_BUS='i2c' gesetzt werden.
\end{verbatim}
\end{example}

\begin{example}
\begin{verbatim}
  Die Treiber EA232Graphic, FW8888, HP12542R kennen folgende Port Einstellungen:

  ttyS0 usw. für lokale serielle Ports (Legacy, PCI)
  ttyUSB0 usw. für USB-angebundene serielle Ports.
\end{verbatim}
\end{example}

\config{LCD4LINUX\_DRV\_SPEED}{LCD4LINUX\_DRV\_SPEED}{LCD4LINUXDRVSPEED}

  Gibt die Geschwindigkeit der seriellen Schnittstelle an.

\begin{example}
\begin{verbatim}
  Mögliche Werte:

  1200, 2400, 4800, 9600, 19200, 38400, 57600, 115200

  EA232Graphic hat als Default 115200 Baud.
\end{verbatim}
\end{example}

\config{LCD4LINUX\_DRV\_WIRE\_TYPE}{LCD4LINUX\_DRV\_WIRE\_TYPE}{LCD4LINUXDRVWIRETYPE}

  Legt die Verdrahtung von HD44780 und M50530 Displays fest.

\begin{example}
\begin{verbatim}
  Mögliche Werte für HD44780:

  generic	# auch LCD4Linux Wiring genannt (sehr häufig)
  winamp	# ebenfalls eine sehr häufige Beschaltung
  soekris	# spezielle Beschaltung für die Soekris-Boards
\end{verbatim}
\end{example}

\begin{example}
\begin{verbatim}
  Mögliche Werte für M50530:

  simple	# Wie im Beispiel auf: http://ssl.bulix.org/projects/lcd4linux/wiki/M50530
  rw		# das selbe aber mit RW auf INIT (default)
  usebusy	# wie 'rw' aber mit Busy-Flag Check. Sehr langsam
\end{verbatim}
\end{example}

\config{LCD4LINUX\_DRV\_BUS}{LCD4LINUX\_DRV\_BUS}{LCD4LINUXDRVBUS}

  Bus-Typ für HD44780 Displays
  
\begin{example}
\begin{verbatim}
  Mögliche Werte:

  parport	# Parallel-Port (default)
  i2c		# I2C
\end{verbatim}
\end{example}

  Bei Auswahl von I2C muß der I2C Bus vor dem Start von lcd4linux bereits von einem anderen opt initialisiert worden sein.

\config{LCD4LINUX\_DRV\_DEVICE}{LCD4LINUX\_DRV\_DEVICE}{LCD4LINUXDRVDEVICE}

  Device auf dem I2C Bus für HD44780 Displays

  Dies ist eine Bus-ID, die auf der Hardware eingestellt werden muß.


\config{LCD4LINUX\_DRV\_ROTATE}{LCD4LINUX\_DRV\_ROTATE}{LCD4LINUXDRVROTATE}
  - die 'serdisplib' hat noch ein interessantes Feature, nämlich die Möglichkeit
  die Anzeige auf dem Display zu drehen.
  Diese wird zur Zeit von den Displays CTINCLUD und HP12542R genutzt.

\config{LCD4LINUX\_DRV\_CONTRAST}{LCD4LINUX\_DRV\_CONTAST}{LCD4LINUXDRVCONTRAST}

  Gibt den Kontrastpegel an. Mögliche Werte gehen von 0 bis 255. Manche Displays auch weniger.

  Aktuell unterstützt von LCD2USB, HP12542R und einigen EA232Graphic sowie HD44780.

\config{LCD4LINUX\_DRV\_BRIGHTNESS}{LCD4LINUX\_DRV\_BRIGHTNESS}{LCD4LINUXDRVBRIGHTNESS}

  Gibt den Helligkeitspegel an. Mögliche Werte gehen von 0 bis 255. Manche Displays auch weniger.

  Aktuell verwendet von LCD2USB, GLCD2USB und manchen HD44780.

\config{LCD4LINUX\_DRV\_BACKLIGHT}{LCD4LINUX\_DRV\_BACKLIGHT}{LCD4LINUXDRVBACKLIGHT}

  Aktiviert das Backlight. Mögliche Werte: 'yes' oder 'no'.

  Aktuell verwendet von HP12542R sowie einigen HD44780.

\config{LCD4LINUX\_DRV\_ASC255BUG}{LCD4LINUX\_DRV\_ASC255BUG}{LCD4LINUXDRVASC255BUG}

  Manche HD44780 kompatiblen Displays haben einen Bug und zeigen anstelle eines massiven
  'Blocks' ein invertiertes 'P' oder ein '\{' an.
  In diesem Fall ist hier 'yes' anzugeben.

\config{LCD4LINUX\_DRV\_CONTROLLERS}{LCD4LINUX\_DRV\_CONTROLLERS}{LCD4LINUXDRVCONTROLLERS}

  HD44780 Displays können nur maximal 80 Zeichen darstellen. Um mehr zu erreichen (z.B. 40x4)
  werden 2 Controller an ein Display angeschlossen. Aber auch kleinere Display können mit 2 Controllern ausgestattet sein.
  Sollte ein solches Display vorliegen, dann hier '2' angeben.

\config{LCD4LINUX\_DRV\_BITS}{LCD4LINUX\_DRV\_BITS}{LCD4LINUXDRVBITS}

  HD44780 Displays können im 4- oder 8-Bit Modus betrieben werden. Je nach Verkabelung hier also '4' oder '8' angeben.
  Default ist '8' wenn nichts angegeben ist.

\config{LCD4LINUX\_DRV\_USEBUSY}{LCD4LINUX\_DRV\_USEBUSY}{LCD4LINUXDRVUSEBUSY}

  Um das Busy-Flag eines HD44780 Displays zu nutzen, hier 'yes' angeben.

\config{LCD4LINUX\_DRV\_OPTION\_N}{LCD4LINUX\_DRV\_OPTION\_N}{LCD4LINUXDRVOPTIONN}

  Anzahl der manuell vergebenen Treiber-Optionen

\config{LCD4LINUX\_DRV\_OPTION\_N}{LCD4LINUX\_DRV\_OPTION\_N}{LCD4LINUXDRVOPTIONN}

  Hier können eigene Optionen angegeben werden, die noch nicht direkt vom Opt umgesetzt werden.

\begin{example}
\begin{verbatim}
  Zum Beispiel einen größeren Font bei Grafik-Displays:

  LCD4LINUX\_DRV\_OPTION\_N='1'
  LCD4LINUX\_DRV\_OPTION\_1='Font'
  LCD4LINUX\_DRV\_OPTION\_1\_VALUE='12x16'

\end{verbatim}
\end{example}

\end{description}

\subsection{Konfiguration der Displayanzeige}

\begin{description}
  \config{LCD4LINUX\_ICONS}{LCD4LINUX\_ICONS}{LCD4LINUXICONS} 
  - LCD4Linux bietet die Möglichkeit von animierten Icons, z.B. eines schlagenden
  Herzens oder auch einen Blitz. Diese Icons können über das Layout konfiguriert
  werden, allerdings benötigt das Programm dazu etwas Speicherplatz, den es sich
  beim Programmstart reservieren muss, um die Icons zu laden. Geben Sie hier die
  Anzahl der gewünschten Icons an (max. 8). Sollten Sie im Layout mehr Icons
  konfigurieren, als Sie hier Speicherplatz reservieren, werden die 'zusätzlichen'
  nicht angezeigt.

\config{LCD4LINUX\_DISPLAY\_SIZE}{LCD4LINUX\_DISPLAY\_SIZE}{LCD4LINUXDISPLAYSIZE} 
  - die Größe des Displays. Anzugeben in Länge (Zeichen) x Höhe (Zeilen). Wenn Ihr
  Display 20 Zeichen in 4 Zeilen darstellen kann, so tragen Sie hier bitte '20x4'
  ein.

\config{LCD4LINUX\_DSL\_SPEED\_IN}{LCD4LINUX\_DSL\_SPEED\_IN}{LCD4LINUXDSLSPEEDIN} 
  - Die Download-Geschwindigkeit des DSL Anschlusses. Wird benutzt um die Anzeige
  des Download-Balkens zu kalibrieren.
  
\config{LCD4LINUX\_DSL\_SPEED\_OUT}{LCD4LINUX\_DSL\_SPEED\_OUT}{LCD4LINUXDSLSPEEDOUT} 
  - Die Upload-Geschwindigkeit des DSL Anschlusses. Wird benutzt um die Anzeige
  des Upload-Balkens zu kalibrieren.

\config{LCD4LINUX\_LAYOUT\_N}{LCD4LINUX\_LAYOUT\_N}{LCD4LINUXLAYOUTN} 
  - Der interessanteste Teil des Paketes, denn hier geht es um die eigentliche
  Darstellung - also was auf dem Display wo erscheinen soll. Geben Sie hier die
  Anzahl der Layout-Konfigurations-Zeilen an.

  LCD4Linux benutzt ein Konzept namens 'Widgets'. Das sind im Grunde genommen
  kleine eigenständige Funktionen, die dann irgendetwas machen.

  Über 'LCD4LINUX\_LAYOUT\_x' werden dann diese Funktionen an die ihnen zugedachte
  Position verwiesen. 
  'Row1.Col1  :Info:' bedeutet das der Info-Lauftext (die Erklärung welches 
  Widget was macht und wie lang es ist erfolgt weiter unten im Text) an der ersten
  Position der ersten Zeile startet, und da der Info-Text die ganze Zeile belegt
  passt da auch sonst nichts mehr hin. 
  Möchten Sie jetzt aber trotzdem, dass vor dem Info-Text noch ein kleines Icon 
  dargestellt werden soll, so müssen Sie die Konfiguration ein wenig ändern.

  z.B. in:
\begin{example}
\begin{verbatim}
        LCD4LINUX_LAYOUT_1='Row1.Col1  :Lightning:'
        LCD4LINUX_LAYOUT_2='Row1.Col2  :Info:'
\end{verbatim}
\end{example}

  Beachten Sie bitte, dass dann in diesem Beispiel das letzte Zeichen des Lauftextes
  'abgeschnitten' wird, d.h. es kann nicht dargestellt werden, weil das Display
  nicht groß genug ist. Das ist in dem Fall des Lauftextes nicht weiter schlimm,
  da ja ohnehin die Zeichen 'weiterwandern' und dann der Text auch erscheint, nur
  bei anderen Widgets die statisch sind und nicht weiterlaufen, könnte dann das
  ein oder andere Zeichen fehlen.


  Die zur Zeit möglichen Widgets sind in Table \ref{tab:lcd4linux-widgets-x} aufgeführt.
  (die Info\_Tel\_x Widgets haben eine wechselnde Anzeige, in Abhängigkeit ob eine 
  'letzte Rufnummer' vorliegt, darum steht dort ein '-ODER-')

      \begin{small}
       \begin{center}
        \begin{longtable}{rp{7cm}r}

               Typ &     Information   &             Zeichenbreite\\
               \\
                  :Info: &       Lauftext mit Name und IP    &  20 \\
                 :Info2: &       Lauftext mit Name, Version, Ram und CPU   &  20 \\

    :Date\_dd\_mm\_yyyy: &       Datum mit vier Stellen Jahr &  10 \\
      :Date\_dd\_mm\_yy: &       Datum mit zwei Stellen Jahr &   8 \\
      :Time\_hh\_mm\_ss: &       Uhrzeit                     &   8 \\                              

   :ImonDSLQuantity\_In: &       DSL Eingangs Volumen        &  10 \\
  :ImonDSLQuantity\_Out: &       DSL Ausgangs Volumen        &   9 \\
       :ImonDSLRate\_In: &       DSL Eingangs Rate           &   9 \\
      :ImonDSLRate\_Out: &       DSL Ausgangs Rate           &   9 \\
      :ImonDSLRate\_Bar: &       DSL Rate Balken             &  20 \\
   :ImonDSLOnline\_Time: &       DSL Onlinezeit              &  20 \\
           :ImonDSL\_IP: &       DSL IP oder Offline         &  19 \\
        :ImonDSL\_IP\_2: &       DSL IP oder Datum, Uhrzeit  &  20 \\
           :ImonDSLName: &       DSL Circuit Name            &  20 \\
         :ImonDSLCharge: &       DSL Online-Kosten           &  17 \\
         
       :ImonISDN1Status: &       ISDN Circuit Name            &  20 \\
       :ImonISDN2Status: &       ISDN Circuit Name            &  20 \\
         :ImonISDN1Name: &       ISDN Circuit Name            &  20 \\
         :ImonISDN2Name: &       ISDN Circuit Name            &  20 \\
    :ImonISDN1Rate\_Bar: &       ISDN Circuit Name            &  20 \\
    :ImonISDN2Rate\_Bar: &       ISDN Circuit Name            &  20 \\
     :ImonISDN1Rate\_In: &       ISDN Circuit Name            &  20 \\
     :ImonISDN2Rate\_In: &       ISDN Circuit Name            &  20 \\
    :ImonISDN1Rate\_Out: &       ISDN Circuit Name            &  20 \\
    :ImonISDN2Rate\_Out: &       ISDN Circuit Name            &  20 \\
       :ImonISDN1Charge: &       ISDN Circuit Name            &  20 \\
       :ImonISDN2Charge: &       ISDN Circuit Name            &  20 \\
         :ImonISDN1\_IP: &       ISDN Circuit Name            &  19 \\
         :ImonISDN2\_IP: &       ISDN Circuit Name            &  19 \\
 :ImonISDN1Online\_Time: &       ISDN Circuit Name            &  20 \\
 :ImonISDN2Online\_Time: &       ISDN Circuit Name            &  20 \\
 :ImonISDN1Quantity\_In: &       ISDN Circuit Name            &  15 \\
 :ImonISDN2Quantity\_In: &       ISDN Circuit Name            &  15 \\
:ImonISDN1Quantity\_Out: &       ISDN Circuit Name            &  15 \\
:ImonISDN2Quantity\_Out: &       ISDN Circuit Name            &  15 \\

          :Info\_Tel\_1: &       Kein Anruf! -ODER- Nummer, Datum, Uhrzeit      &  20 \\
          :Info\_Tel\_2: &       Kein Anruf, DSL Onlinezeit -ODER- Nummer, Datum, Uhrzeit, DSL Onlinezeit    &  20 \\
          :Info\_Tel\_3: &       DSL Onl.Zeit -ODER- Rufnummer &  20 \\
                 :TelNr: &       TelNummer letzter Anruf     &  20 \\
               :TelDate: &       Datum letzter Anruf         &   8 \\
               :TelTime: &       Uhrzeit letzter Anruf       &   8 \\
       
                    :OS: &       Anzeige Betriebssystem      &  20 \\
                   :CPU: &       Anzeige CPU Version         &   9 \\
                   :RAM: &       Anzeige gesamt RAM Speicher &  11 \\
             :RAM\_FREE: &       Anzeige freien RAM Speicher &  16 \\                   
                  :Busy: &       CPU Auslastung (cpu usage)  &   9 \\
               :BusyBar: &       CPU Auslastungs Balken      &  10 \\
                  :Load: &       Rechner Auslastung          &  10 \\
               :LoadBar: &       Rechner Auslastungs Balken  &  10 \\
                  :Eth0: &       Volumen von Eth0            &  10 \\
               :Eth0Bar: &       Rate von Eth0               &  14 \\
                   :PPP: &       Volumen über PPP-Verbindung &   9 \\
                :Uptime: &       Zeit seit letztem Neustart  &  20 \\

              :VarText1: &       text -$>$ /etc/lcd\_text1.txt  & max 20 \\
              :VarText2: &       text -$>$ /etc/lcd\_text2.txt  & max 20 \\
              :VarText3: &       text -$>$ /etc/lcd\_text3.txt  & max 20 \\
              :VarText4: &       text -$>$ /etc/lcd\_text4.txt  & max 20 \\

             :Lightning: &       Icon: Blitz             &  1 \\
             :Heartbeat: &       Icon: Herzschlag        &  1 \\
                 :Heart: &       Icon: Herz              &  1 \\
                  :Blob: &       Icon: (Luft)Blase       &  1 \\
                  :Wave: &       Icon: Welle             &  1 \\
              :Squirrel: &       Icon: Wirbel            &  1 \\
                  :Rain: &       Icon: Regen(tropfen)    &  1 \\

            \caption{Übersicht über mögliche Widgets}
            \marklabel{tab:lcd4linux-widgets-x}{}

        \end{longtable}
       \end{center}
      \end{small}

\config{LCD4LINUX\_START\_STOP\_MSG}{LCD4LINUX\_START\_STOP\_MSG}{LCD4LINUXSTARTSTOPMSG} 
  - wenn diese Option auf 'yes' gesetzt wird, so werden beim Systemstart und
  beim Herunterfahren Textmeldungen ausgegeben.
  
  Wenn das LCD im Expertenmodus (\var{LCD4LINUX\_EXPERT\_MODE='yes'}) konfiguriert wird
  müssen zusätzlich vier Layouts mit den Namen \verb*?Startup?, \verb*?Halt?,
  \verb*?Poweroff? und \verb*?Reboot? angelegt werden.
  Siehe dazu auch die Beispiel-Konfig unter 'opt$\backslash$etc$\backslash$lcd4linux')

\config{LCD4LINUX\_EXPERT\_MODE}{LCD4LINUX\_EXPERT\_MODE}{LCD4LINUXEXPERTMODE} 
  - wenn diese Option auf 'yes' gesetzt wird, so wird die komplette Konfiguration 
  ignoriert und stattdessen eine 'lcd4linux.conf' aus dem Ordner 'config$\backslash$etc$\backslash$lcd4linux$\backslash$' 
  (Ordner muss selbst erstellt werden, eine Beispiel-Konfig findet sich unter 
  'opt$\backslash$etc$\backslash$lcd4linux') mit auf den Router kopiert. Diese Datei muss entsprechend 
  den eigenen Bedürfnissen angepasst werden.

\config{LCD4LINUX\_TEST}{LCD4LINUX\_TEST}{LCD4LINUXTEST} 
  - zum Testen der Konfiguration der 'lcd4linux.conf' kann man diese Option 
  einschalten.

  Der LCD4LINUX-Daemon wird dann nicht automatisch gestartet, sondern man muss 
  sich auf die Console verbinden (per ssh oder direkt) und den Daemon im 
  DebugModus starten - man sieht dann direkt evtl. Fehlermeldungen und der 
  Daemon läuft dann nicht als Hintergrunddienst, sondern kann über 'strg+c' 
  abgebrochen werden.

  Wenn man nun einen Editor auf dem fli4l mitinstalliert hat (z.B. den e3) kann
  man nun unter /etc/lcd4linux/lcd4linux.conf die Konfiguration korrigieren.

  Der Aufruf für den LCD4Linux-DebugModus lautet:
\begin{example}
\begin{verbatim}
  'lcd4linux -f /etc/lcd4linux/lcd4linux.conf -Fvv'
\end{verbatim}
\end{example}

\end{description}


\subsection{Anschlussbelegung eines LCD-Moduls am Parallelport}
 
\begin{example}
\begin{verbatim}
   13 _____________________________ 1 Draufsicht auf den
      \ o o o o o o o o o o o o o /   Parallelport, Rück-
       \ o o o o o o o o o o o o /    seite PC
     25 ------------------------- 14
\end{verbatim}
\end{example}
   

 Der Anschluss eines LCD-Moduls an den Router wird folgendermaßen aufgetrennt:

 
\begin{example}
\begin{verbatim}
 
 Parallelport-Pin   Beschreibung   LCD-Modul    LCD-Pin
         18-25      GND                             --|
                    GND                          1  --|- Brücke
                    R/W                          5  --|
                    +5V                          2
             1      STROBE         EN(1)         6
             2      D0             D0            7
             3      D1             D1            8
             4      D2             D2            9
             5      D3             D3           10
             6      D4             D4           11
             7      D5             D5           12
             8      D6             D6           13
             9      D7             D7           14
            14      Autofeed       RS            4
            17      Select In      EN(2)         ? (für LCDs mit 2 Controller)

  Bei Display mit Hintergrundbeleuchtung:
                                   HG+          15 (mit Vorwiderstand ca. 20Ohm)
                                   GND          16
\end{verbatim}
\end{example}

  An Pin 3 kann der Abgriff eines $>$= 20kOhm Potis zwischen +5V und GND 
  geschaltet werden. Damit kann der Kontrast des Displays reguliert werden.
  Bei meinem Display (Conrad) liegt Pin 3 direkt an Masse und man kann
  alles einwandfrei erkennen.

\begin{example}
\begin{verbatim}
  
  +5V ---+
         /
         \ <--+
         /    |
         \    |
  GND ---+    +--- VL (Pin 3 - driver input)
\end{verbatim}
\end{example}



\subsection{Anschluss eines 4x40 Displays}

  Da sich der Anschluss eines 4x40 Displays stark von anderen Displays
  unterscheidet, hier ein Beispiel (Conrad - NLC-40x4x05):

\begin{example}
\begin{verbatim}

Parallelport-Pin   Beschreibung   LCD-Modul           LCD-Pin
        18-25                                             --|
                   GND                                13  --|- Brücke
                   R/W                                10  --|
                   +5V                                14
            1      STROBE         EU (Enable-Upper)    9
            2      D0             D0                   8
            3      D1             D1                   7
            4      D2             D2                   6
            5      D3             D3                   5
            6      D4             D4                   4
            7      D5             D5                   3
            8      D6             D6                   2
            9      D7             D7                   1
           14      Autofeed       RS                  11
           17      Select In      ED (Enable-Down)    15
\end{verbatim}
\end{example}


  An Pin 12 kann der Abgriff eines $>$= 20kOhm Potis zwischen +5V und GND 
  geschaltet werden. Damit kann der Kontrast des Displays reguliert werden.
  Es kann aber auch reichen, Pin 12 direkt an Masse zu legen um alles
  einwandfrei erkennen zu können.

\begin{example}
\begin{verbatim}

  +5V ---+
         /
         \ <--+
         /    |
         \    |
  GND ---+    +--- VL (Pin 12 - driver input)
\end{verbatim}
\end{example}


  \begin{itemize}
  \item Die Leitung ED wird an Pin 17 des parallelen Ports angeschlossen.
  
  \item Das Display wird in der lcd.txt als 2x40 Display definiert.
  
  \item Bei den Typendefinitionen für isdn\_rate wird aber die 4x40 als
    Zeilen-/Spaltengröße angesehen.
  \end{itemize}


  Leider gibt es keinen Standard, was die Pinbelegung des Parallelports
  auf dem Motherboard betrifft. Für die interne Verwendung von LCD-Modulen
  muß man also die Anschlüsse anhand des zum Motherboard mitgelieferten
  Slotblechadapters durchmessen.

  Die erforderliche Spannungsversorgung kann man leider nicht dem Parallelport
  entnehmen, da die Stromaufnahme eines LCD-Modules zu hoch ist. Geeignet dafür
  sind die Anschlüsse für Maus (PS/2), Tastatur (DIN, PS/2), Gameport, USB oder ein
  freier Anschluss vom PC-Netzteil. Da einige Soundkartenhersteller am Gameport
  spezielle Signale generieren, kann keine Garantie übernommen, dass es in jeder
  Kombination funktioniert. Daher gilt hier: Immer vorher messen!


\subsection{Danksagung}

  Dank geht an:

  \begin{itemize}
  \item Frank Meyer für das imond Interface und fli4l :)
   
  \item Gernot Miksch für das Paket LCD
   
  \item Michael Reinelt (\altlink{https://ssl.bulix.org/projects/lcd4linux}) für 
        das LCD4Linux Programm
  \end{itemize}
