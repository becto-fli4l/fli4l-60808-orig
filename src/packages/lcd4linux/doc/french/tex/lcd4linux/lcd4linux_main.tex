% Synchronized to r38650
\marklabel{sec:opt-lcd4linux }
{
  \section {LCD4Linux - Affichage des informations de l'état du routeur sur écran LCD}
}

\subsection{Introduction}

    Avec ce paquetage, il est possible de connecter un module LCD sur les
	différentes interfaces de l'ordinateur fli4l (le port parallèle et USB
	sont disponibles directement à partir de la configuration de fli4l,
	le port série doit être configuré via le mode "expert", voir la section
	correspondante ci-dessous).

    Les informations suivant peuvent être configurées pour l'affichage telles
	que la date, l'adresse IP externe, les valeurs de charge CPU en cours et
	bien sûr le débit montant et dessendant en ko/s du ISDN ou DSL et une barre
	d'indication.

    Des icônes animées peuvent être définis librement par l'utilisateur, en
	permanence ou seulement en atteignant un seuil qui sera définissable 
	(par exemple, un flash qui apparait lors de la charge élevée du processeur).

    Il est possible de représenter des barres (deux barres par rangée), qui
	peuvent se déplacer dans n'importe quelle direction (gauche, droite, haut ou bas).

    Les possibilités de création n'ont pratiquement pas de limites - les limites
	sont fixe par la taille de l'écran ;-)

    En outre, l'affichage peut être étendu avec des plugins, presque sans limite.

    Il y a également un plugin IMON disponible, qui pourra être interrogé pour avoir
	des informations sur l'état du routeur - c'est à dire la durée de connexion en ligne,
	le taux de transfert et de quantité, etc.

    Pour réaliser tout cela,  le logiciel a utiliser est lcd4linux. Avec la
    Version 0.10.1-CVS (du 21.02.2007).

    Vous pouvez trouver sur la page d'accueil du projet
    (\altlink{https://ssl.bulix.org/projects/lcd4linux})
    quelques images sur l'affichages en charge, des informations détaillées sur
	les options de configuration en mode "expert".

\subsection{Configuration du pilote d'affichage}
\configlabel{OPT\_LCD4LINUX}{OPTLCD4LINUX}

    Si vous voulez utiliser le paquetage lcd4linux, la variables suivante doit
	être activée

\begin{example}
\begin{verbatim}
        OPT_LCD4LINUX='yes'  (paramètre par défaut~: OPT_LCD4LINUX='no')
\end{verbatim}
\end{example}

    Ensuite régler l'affichage souhaité avec ses éventuels moyens supplémentaires et
	si possible, la configuation de la connexion. Après cela, il devrait apparaître
	sur l'écran - quelque chose.

	Si vous avez un écran LCD qui n'est pas pré-défini mais qui est supporté par
	lcd4linux (voir ci-dessus le site Web du projet), vous pouvez activer le mode
	"expert" et faire les modifications directement dans le fichier de configuration
	de lcd4linux.

\begin{description}
\config{LCD4LINUX\_DRV}{LCD4LINUX\_DRV}{LCD4LINUXDRV}

  Vous sélectionnez ici le pilote du contrôleur que vous avez besion.

\begin{example}
\begin{verbatim}
  Les pilotes possibles sont~:
    HD44780      - Hitachi HD44780 écrans de base et compatible
    LCD2USB      - LCD2USB - http://www.harbaum.org/till/lcd2usb/
    GLCD2USB     - GLCD2USB - http://www.harbaum.org/till/glcd2usb/
    HP12542R     - Hyunday HP12542R-DYO
    CTINCLUD     - http://www.ct-maeusekino.de/
    FW8888       - Allnet FW8888 internal Display
    DPF          - Patched Digital Photo Frame http://geekparadise.de/tag/ax206/
    EA232Graphic - Some RS232 Graphic Displays made by Electronic Assembly
    M50530       - Mitsubishi M50530 mit z.B. 24x8 Zeichen
\end{verbatim}
\end{example}


\config{LCD4LINUX\_DRV\_MODEL}{LCD4LINUX\_DRV\_MODEL}{LCD4LINUXDRVMODEL}

  Vous sélectionnez ici le modèle d'affichage que vous avez besion.

\begin{example}
\begin{verbatim}
  Pour le contrôleur HD44780, nous avons à disposition les modèles suivants~:

  generic	- Affichage standard HD44780 (par défaut)
  Noritake	- Comme affichage standard, mais avec un réglage pour le rétro-éclairage par logiciel
  Soekris	- Pour PC-Soekris avec une interface spéciale Busy-Check en mode 4 bits
  HD66712	- Comme HD44780 et KS0073 mais avec une disposition de la RAM légèrement différente
  LCM-162	- Interface spéciale pour les serveurs Nexcom Blade
\end{verbatim}
\end{example}

\begin{example}
\begin{verbatim}
  Pour le contrôleur EA232Graphic, voici les modèles suivants~:

  GE120-5NV24	120x32
  GE128-6N9V24	128x64
  GE128-6N3V24	128x64
  GE128-7KV24	128x128
  GE240-6KV24	240x64
  GE240-6KCV24	240x64
  GE240-7KV24	240x128
  GE240-7KLWV24	240x128
  GE240-6KLWV24	240x64
  KIT120-5	120x32
  KIT129-6	128x64
  KIT160-6	160x80
  KIT160-7	160x128
  KIT240-6	240x64
  KIT240-7	240x128
  KIT320-8	320x240
\end{verbatim}
\end{example}

\config{LCD4LINUX\_DRV\_PORT}{LCD4LINUX\_DRV\_PORT}{LCD4LINUXDRVPORT}

  Vous sélectionnez ici le port à utiliser.

\begin{example}
\begin{verbatim}
  Pour le contrôleur HD44780 vous pouvez utiliser les ports suivants~:

  parport0 à parport2 pour trouver l'adresse du port parallèle sur Linux-Parport-API
  0x278, 0x378, 0x3BC adresse des ports parallèles
  i2c-0 à i2x-9 pour affichage commandé par bus I2C.

  Pour utiliser le bus I2C, vous devez paramétrer dans la variable LCD4LINUX\_DRV\_BUS='i2c'.

\end{verbatim}
\end{example}

\begin{example}
\begin{verbatim}
  Pour les contrôleurs EA232Graphic, FW8888, HP12542R vous pouvez utiliser les ports suivants~:

  ttyS0 etc. pour les ports série locaux (Legacy, PCI)
  ttyUSB0 etc. pour se connecter aux ports USB .
\end{verbatim}
\end{example}

\config{LCD4LINUX\_DRV\_SPEED}{LCD4LINUX\_DRV\_SPEED}{LCD4LINUXDRVSPEED}

  Dans cette variable vous indiquez la vitesse de transmition de l'interface série.

\begin{example}
\begin{verbatim}
  Les valeurs possibles sont~:

  1200, 2400, 4800, 9600, 19200, 38400, 57600, 115200

  Avec le contrôleur EA232Graphic la vitesse par défaut est de 115200 bauds.
\end{verbatim}
\end{example}

\config{LCD4LINUX\_DRV\_WIRE\_TYPE}{LCD4LINUX\_DRV\_WIRE\_TYPE}{LCD4LINUXDRVWIRETYPE}

  Dans cette variable vous indiquez le système de câblage pour les contrôleurs
  HD44780 et M50530.

\begin{example}
\begin{verbatim}
Les valeurs possibles pour le contrôleur HD44780 sont~:

  generic	# également appelé câblage lcd4linux (très fréquent)
  winamp	# c'est aussi un câblage très commun
  soekris	# câblage spécial pour les cartes Soekris
\end{verbatim}
\end{example}

\begin{example}
\begin{verbatim}
  Les valeurs possibles pour le contrôleur M50530 sont~:

  simple	# comme dans l'exemple~: http://ssl.bulix.org/projects/lcd4linux/wiki/M50530
  rw		# le même mais avec RW sur INIT (par défaut)
  usebusy	# comme avec 'rw' mais avec un contrôle Busy-Flag. Très lent
\end{verbatim}
\end{example}

\config{LCD4LINUX\_DRV\_BUS}{LCD4LINUX\_DRV\_BUS}{LCD4LINUXDRVBUS}

  Dans cette variable vous indiquez le type de bus pour l'affichage avec le contrôleur
  HD44780.
  
\begin{example}
\begin{verbatim}
  Les valeurs possibles sont~:

  parport	# Port-Parallèle (par défaut)
  i2c		# I2C
\end{verbatim}
\end{example}

  Si vous sélectionnez I2C, le bus I2C doit être initialisé par un autre opt,
  avant le démarrage de lcd4linux.
  

\config{LCD4LINUX\_DRV\_DEVICE}{LCD4LINUX\_DRV\_DEVICE}{LCD4LINUXDRVDEVICE}

  Dans cette variable vous indiquez le périphérique sur le bus I2C pour l'affichage
  avec le contrôleur HD44780.

  Il s'agit d'un numéro ID du bus qui doit être configuré pour le matériel.


\config{LCD4LINUX\_DRV\_ROTATE}{LCD4LINUX\_DRV\_ROTATE}{LCD4LINUXDRVROTATE}

  - La valeur 'serdisplib' a une caractéristique intéressante, à savoir la possibilité
  de faire pivoter l'affichage de l'écran. Ceci est actuellement utilisé par les contrôleurs
  CTINCLUD et HP12542R.

\config{LCD4LINUX\_DRV\_CONTRAST}{LCD4LINUX\_DRV\_CONTAST}{LCD4LINUXDRVCONTRAST}

  Dans cette variable vous paramétrez le niveau de contraste de l'écran. Les valeurs possibles
  sont de 0 à 255. Sur certains écrans encore moins.

  Actuellement le niveau de contraste est supporté par les LCD2USB, HP12542R et
  par certain contrôleur EA232Graphic, HD44780.

\config{LCD4LINUX\_DRV\_BRIGHTNESS}{LCD4LINUX\_DRV\_BRIGHTNESS}{LCD4LINUXDRVBRIGHTNESS}

  Dans cette variable vous paramétrez le niveau de luminosité. Les valeurs possibles sont
  de 0 à 255. Sur certains écrans encore moins.

  Actuellement le niveau de luminosité est supporté par les LCD2USB, GLCD2USB et
  certain contrôleur HD44780.

\config{LCD4LINUX\_DRV\_BACKLIGHT}{LCD4LINUX\_DRV\_BACKLIGHT}{LCD4LINUXDRVBACKLIGHT}

  Dans cette variable vous activez le rétro-éclairage. Valeur possible~: 'yes' ou 'no'.

  Actuellement le rétro-éclairage est supporté par le HP12542R et plusieurs contrôleurs HD44780.

\config{LCD4LINUX\_DRV\_ASC255BUG}{LCD4LINUX\_DRV\_ASC255BUG}{LCD4LINUXDRVASC255BUG}

  Certain contrôleur HD44780 ont un bug, la place d'un "bloc" un 'P' inversé ou le signe '\{'
  est affichage. Si vous avez ce problème, vous devez indiquer 'yes' dans cette variable.

\config{LCD4LINUX\_DRV\_CONTROLLERS}{LCD4LINUX\_DRV\_CONTROLLERS}{LCD4LINUXDRVCONTROLLERS}

  Le contrôleur HD44780 peut afficher un maximum de 80 caractères. Afin d'augmenter l'affichage
  par exemple, 40x4), deux contrôleurs doivent être connectés à un seul écran. Les plus petits
  écrans peuvent être équipés de deux contrôleurs. Dans ce cas, vous pouvez spécifier '2' dans
  cette variable.

\config{LCD4LINUX\_DRV\_BITS}{LCD4LINUX\_DRV\_BITS}{LCD4LINUXDRVBITS}

  Le contrôleur HD44780 peut être utilisé dans le mode 4- ou  8-bits, selon le câblage. vous
  pouvez spécifier '4' ou '8' dans cette variable. si rien n'est spécifié, la valeur
  par défaut est '8'.

\config{LCD4LINUX\_DRV\_USEBUSY}{LCD4LINUX\_DRV\_USEBUSY}{LCD4LINUXDRVUSEBUSY}

  Avec le contrôleur HD44780 il est possible d'utiiser un Flag d'occupation, si vous indiquez
  'yes' dans cette variable.

\config{LCD4LINUX\_DRV\_OPTION\_N}{LCD4LINUX\_DRV\_OPTION\_N}{LCD4LINUXDRVOPTIONN}

  Dans cette variable vous indiquez le nombre d'options qui sera attribuées manuellement au pilote.

\config{LCD4LINUX\_DRV\_OPTION\_N}{LCD4LINUX\_DRV\_OPTION\_N}{LCD4LINUXDRVOPTIONN}

  Dans cette variable vous pouvez spécifier vos propres options, qui ne sont pas encore
  installées directement par l'OPT.

\begin{example}
\begin{verbatim}
  Par exemple, une police plus grande pour l'affichage~:

  LCD4LINUX\_DRV\_OPTION\_N='1'
  LCD4LINUX\_DRV\_OPTION\_1='Font'
  LCD4LINUX\_DRV\_OPTION\_1\_VALUE='12x16'

\end{verbatim}
\end{example}

\end{description}

\subsection{Configuration de l'écran d'affichage}

\begin{description}
  \config{LCD4LINUX\_ICONS}{LCD4LINUX\_ICONS}{LCD4LINUXICONS} 
  - Lcd4linux offre la possibilité d'afficher des icônes animées, par exemple,
  un battement de c\oe{}ur ou même un flash. Ces icônes peuvent être configurés via
  la mise en page, mais le programme a besoin d'espace pour qu'il réserve au
  démarrage pour charger les icônes. Dans cette varible vous pouvez indiquer un
  nombre d'icônes désirées (max. 8). Si vous configurez une mise en page avec
  plus d'icônes indiquées ici avec une réservation d'espace, les icônes
  "supplémentaires" ne s'afficheront pas.

\config{LCD4LINUX\_DISPLAY\_SIZE}{LCD4LINUX\_DISPLAY\_SIZE}{LCD4LINUXDISPLAYSIZE} 
  - Dans cette variable vous indiquez la taille de l'écran, longueur (en caractères)
  x hauteur (en lignes). Si votre écran peut afficher 20 caractères sur 4 lignes, vous
  écrivez ici '20x4'.

\config{LCD4LINUX\_DSL\_SPEED\_IN}{LCD4LINUX\_DSL\_SPEED\_IN}{LCD4LINUXDSLSPEEDIN} 
  - Dans cette variable vous indiquez la vitesse descendante de votre connexion
  ADSL. Cette valeur est utilisée pour étalonner l'affichage de la barre de réception.

\config{LCD4LINUX\_DSL\_SPEED\_OUT}{LCD4LINUX\_DSL\_SPEED\_OUT}{LCD4LINUXDSLSPEEDOUT} 
  - Dans cette variable vous indiquez la vitesse montante de votre connexion ADSL.
  Cette valeur est utilisé pour étalonner l'affichage de la barre d'émission.

\config{LCD4LINUX\_LAYOUT\_N}{LCD4LINUX\_LAYOUT\_N}{LCD4LINUXLAYOUTN} 
  - C'est la partie la plus intéressante, car il s'agit de la présentation réelle
  du paquetage. C'est se qui doit ou devrait apparaître sur l'écran. Dans cette
  variable vous indiquez le nombre de rangées disposition pour la configuration.

  Lcd4linux utilise un concept appelé "widgets". Ce sont essentiellement des petites
  fonctions indépendantes, qui font quelque chose.

  Dans la variable 'LCD4LINUX\_LAYOUT\_x' vous indiquez une fonction, elle sera assignée
  à une position sur l'écran. 'Row1.Col1  :Info:' signifie qu'un texte d'information défilera
  (l'explication du widget sera fait plus tard dans le document), il démarre sur la première
  ligne et au premier bloc, le texte info occupera aussi toute la ligne. Si vous aimez une
  petite icône en face du texte-info, vous devez changer légèrement la configuration.

  Par exemple~:
\begin{example}
\begin{verbatim}
        LCD4LINUX_LAYOUT_1='Row1.Col1  :Lightning:'
        LCD4LINUX_LAYOUT_2='Row1.Col2  :Info:'
\end{verbatim}
\end{example}

  Noter que dans cet exemple, le dernier caractère du texte de défilement sera "coupée",
  c'est à dire qu'il ne peut pas être visible car l'écran n'est pas assez grand.
  Ce n'est pas trop gave pour un texte en cours d'exécution, parce que les caractères
  défilent de toute façon. Avec d'autres widgets qui sont statiques, texte qui ne
  défile pas, certains caractères pourrait être manquant.

  Vous pouvez voir dans le tableau suivant \ref{tab:lcd4linux-widgets-x}, les widgets
  actuellement utilisés. (le widget Info\_Tel\_x a un affichage alternatif, selon s'il
  y a un "dernier appel" présent ou pas, c'est pourquoi il y a un '-OU-')

      \begin{small}
       \begin{center}
        \begin{longtable}{rp{7cm}r}

               Type &     Information   &             Largeur des caractères\\
               \\
                  :Info: &       Défilement du texte avec le nom et l'IP &  20 \\
                 :Info2: &       Défilement du texte avec le nom, la version, la Ram et le CPU &  20 \\

    :Date\_dd\_mm\_yyyy: &       Date et année à quatre chiffres &  10 \\
      :Date\_dd\_mm\_yy: &       Date et année à deux chiffres &   8 \\
      :Time\_hh\_mm\_ss: &       Heure de la journée        &   8 \\

   :ImonDSLQuantity\_In: &       Volume de réception du DSL &  10 \\
  :ImonDSLQuantity\_Out: &       Volume d'émission du DSL   &   9 \\
       :ImonDSLRate\_In: &       Débit de réception du DSL  &   9 \\
      :ImonDSLRate\_Out: &       Débit d'émission du DSL    &   9 \\
      :ImonDSLRate\_Bar: &       Barre de débit du DSL      &  20 \\
   :ImonDSLOnline\_Time: &       Temps en ligne du DSL      &  20 \\
           :ImonDSL\_IP: &       IP DSL ou hors ligne       &  19 \\
        :ImonDSL\_IP\_2: &       IP DSL ou date, heure      &  20 \\
           :ImonDSLName: &       Nom du circuit DSL         &  20 \\
         :ImonDSLCharge: &       Coûts de la connexion DSL  &  17 \\
         
       :ImonISDN1Status: &       ISDN nom du circuit        &  20 \\
       :ImonISDN2Status: &       ISDN nom du circuit        &  20 \\
         :ImonISDN1Name: &       ISDN nom du circuit        &  20 \\
         :ImonISDN2Name: &       ISDN nom du circuit        &  20 \\
    :ImonISDN1Rate\_Bar: &       ISDN nom du circuit        &  20 \\
    :ImonISDN2Rate\_Bar: &       ISDN nom du circuit        &  20 \\
     :ImonISDN1Rate\_In: &       ISDN nom du circuit        &  20 \\
     :ImonISDN2Rate\_In: &       ISDN nom du circuit        &  20 \\
    :ImonISDN1Rate\_Out: &       ISDN nom du circuit        &  20 \\
    :ImonISDN2Rate\_Out: &       ISDN nom du circuit        &  20 \\
       :ImonISDN1Charge: &       ISDN nom du circuit        &  20 \\
       :ImonISDN2Charge: &       ISDN nom du circuit        &  20 \\
         :ImonISDN1\_IP: &       ISDN nom du circuit        &  19 \\
         :ImonISDN2\_IP: &       ISDN nom du circuit        &  19 \\
 :ImonISDN1Online\_Time: &       ISDN nom du circuit        &  20 \\
 :ImonISDN2Online\_Time: &       ISDN nom du circuit        &  20 \\
 :ImonISDN1Quantity\_In: &       ISDN nom du circuit        &  15 \\
 :ImonISDN2Quantity\_In: &       ISDN nom du circuit        &  15 \\
:ImonISDN1Quantity\_Out: &       ISDN nom du circuit        &  15 \\
:ImonISDN2Quantity\_Out: &       ISDN nom du circuit        &  15 \\

          :Info\_Tel\_1: &       Pas d'appel~! -OU- numéro, date, heure &  20 \\
          :Info\_Tel\_2: &       Pas d'appel DSL en ligne -OU- numéro, date, heure, DSL en ligne &  20 \\
          :Info\_Tel\_3: &       DSL temp en ligne -OU- numéro de téléphone &  20 \\
                 :TelNr: &       Numéro de téléphone du dernier appel &  20 \\
               :TelDate: &       Date du dernier appel      &   8 \\
               :TelTime: &       Heure du dernier appel     &   8 \\
       
                    :OS: &       Affichage du système d'exploitation &  20 \\
                   :CPU: &       Affichage de la version CPU         &   9 \\
                   :RAM: &       Affichage de la mémoire RAM totale  &  11 \\
             :RAM\_FREE: &       Affichage de la mémoire RAM libre   &  16 \\                   
                  :Busy: &       Utilisation du CPU (cpu usage)      &   9 \\
               :BusyBar: &       Barre d'utilisation CPU             &  10 \\
                  :Load: &       Utilisation de l'ordinateur         &  10 \\
               :LoadBar: &       Barre d'utilisation de l'ordinateur &  10 \\
                  :Eth0: &       Volume du Eth0                      &  10 \\
               :Eth0Bar: &       Vitesse du Eth0                     &  14 \\
                   :PPP: &       Volume de la connexion PPP          &   9 \\
                :Uptime: &       Temps écoulé depuis la dernière réinitialisation &  20 \\

              :VarText1: &       text -$>$ /etc/lcd\_text1.txt & max 20 \\
              :VarText2: &       text -$>$ /etc/lcd\_text2.txt & max 20 \\
              :VarText3: &       text -$>$ /etc/lcd\_text3.txt & max 20 \\
              :VarText4: &       text -$>$ /etc/lcd\_text4.txt & max 20 \\

             :Lightning: &       Icon: Flash          &  1 \\
             :Heartbeat: &       Icon: Pulsation      &  1 \\
                 :Heart: &       Icon: C\oe{}ur       &  1 \\
                  :Blob: &       Icon: Bulle          &  1 \\
                  :Wave: &       Icon: Vague          &  1 \\
              :Squirrel: &       Icon: Vortex         &  1 \\
                  :Rain: &       Icon: Pluie(gouttes) &  1 \\

            \caption{Vue d'ensemble des Widgets}
            \marklabel{tab:lcd4linux-widgets-x}{}

        \end{longtable}
       \end{center}
      \end{small}

\config{LCD4LINUX\_START\_STOP\_MSG}{LCD4LINUX\_START\_STOP\_MSG}{LCD4LINUXSTARTSTOPMSG} 
  - Si cette variable est paramétré sur 'yes', un message apparaîtra au démarrage du sytème
  et à l'arrêt.

  Si vous avez configuré l'écran LCD en mode expert (\var{LCD4LINUX\_EXPERT\_MODE='yes'}),
  vous devez indiquer quatre paramètres supplémentaires \verb*?Startup?, \verb*?Halt?,
  \verb*?Poweroff? et \verb*?Reboot?. (Vous pouvez voir aussi l'exemple de configuration
  à cette endroit 'opt$\backslash$etc$\backslash$lcd4linux')

\config{LCD4LINUX\_EXPERT\_MODE}{LCD4LINUX\_EXPERT\_MODE}{LCD4LINUXEXPERTMODE} 
  - Si cette variable est paramétré sur 'yes', toute la configuration sera ignoré
  et a la place le fichier 'lcd4linux.conf' dans le répertoire 'config$\backslash$etc$\backslash$lcd4linux$\backslash$'
  sera utilisé. (Vous devez créer ce dossier, un exemple de configuration se trouve
  à cette endroit 'opt$\backslash$etc$\backslash$lcd4linux') il peut être copié dans
  le routeur. Ce fichier doit être réglé selon vos besoins.

\config{LCD4LINUX\_TEST}{LCD4LINUX\_TEST}{LCD4LINUXTEST} 
  - Pour tester la configuration du fichier 'lcd4linux.conf' vous pouvez activer
  cette option.

  Le démon lcd4linux est lancé automatiquement, mais vous pouvez vous connecter
  à la console (via le ssh ou directement) pour démarrer le démon en mode debug -
  Vous pouvez vérifier les messages d'erreur, le démon ne fonctionne pas comme le
  service de base, il peut être arrêté par les touches 'strg+c'.

  Si vous avez installé un éditeur de texte sur fli4l (par exemple, e3), vous
  pourrez corriger la configuration du fichier '/etc/lcd4linux/lcd4linux.conf'.

  La commande pour le mode de débogage lcd4linux est~:
\begin{example}
\begin{verbatim}
  'lcd4linux -f /etc/lcd4linux/lcd4linux.conf -Fvv'
\end{verbatim}
\end{example}

\end{description}


\subsection{Branchement du module LCD sur le port parallèle}
 
\begin{example}
\begin{verbatim}
   13 _____________________________ 1  Vue du port parallèle
      \ o o o o o o o o o o o o o /    à l'arrière du PC
       \ o o o o o o o o o o o o /
     25 ------------------------- 14
\end{verbatim}
\end{example}
   

 La connexion du module LCD pour le routeur est détaillée de la façon suivante~:

 
\begin{example}
\begin{verbatim}
 
Pin port parallèle  Description    Module LCD    Pin LCD
         18-25      GND                             --|
                    GND                          1  --|- pont
                    R/W                          5  --|
                    +5V                          2
             1      STROBE         EN(1)         6
             2      D0             D0            7
             3      D1             D1            8
             4      D2             D2            9
             5      D3             D3           10
             6      D4             D4           11
             7      D5             D5           12
             8      D6             D6           13
             9      D7             D7           14
            14      Autofeed       RS            4
            17      Select In      EN(2)         ? (pour les écrans LCD avec 2 contrôleurs)

  Module LCD avec rétro-éclairage~:
                                   HG+          15 (avec une résistance en série de 20O Ohm)
                                   GND          16
\end{verbatim}
\end{example}

    Vous pouvez aussi brancher un potentiomètre de $>$= 20 kOhm entre +5 V et GND
	(ou la masse) et la sortie sur le pin 3 du driver input. Ainsi, vous pourrez
	régler le contraste de l'affichage. Sur mon écran (Conrad) le pin 3 se trouve
	directement relié à la masse, vous pouvez tout voir correctement.

\begin{example}
\begin{verbatim}
  
  +5V ---+
         /
         \ <--+
         /    |
         \    |
  GND ---+    +--- VL (Pin 3 - driver input)
\end{verbatim}
\end{example}



\subsection{Connexion d'un écran 4x40}

   La connexion d'un écran de 4x40 est très différent des autres écrans, 
  voici un exemple avec un (Conrad - NLC-40x4x05)~:

\begin{example}
\begin{verbatim}

Pin port parallèle  Description    Module LCD         Pin LCD
        18-25                                             --|
                   GND                                13  --|- pont
                   R/W                                10  --|
                   +5V                                14
            1      STROBE         EU (Enable-Upper)    9
            2      D0             D0                   8
            3      D1             D1                   7
            4      D2             D2                   6
            5      D3             D3                   5
            6      D4             D4                   4
            7      D5             D5                   3
            8      D6             D6                   2
            9      D7             D7                   1
           14      Autofeed       RS                  11
           17      Select In      ED (Enable-Down)    15
\end{verbatim}
\end{example}
 
     Vous pouvez aussi brancher un potentiomètre de $>$= 20 kOhm entre +5 V et GND
	(ou la masse) et la sortie sur le pin 12 du driver input. Ainsi, vous pourrez
	régler le contraste de l'affichage. Sur mon écran (Conrad) le pin 12 se trouve
	directement relié à la masse, vous pouvez tout voir correctement.

\begin{example}
\begin{verbatim}

  +5V ---+
         /
         \ <--+
         /    |
         \    |
  GND ---+    +--- VL (Pin 12 - driver input)
\end{verbatim}
\end{example}


  \begin{itemize}
  \item Le fil ED du module doit être connecté à la broche 17 du port parallèle.
  
  \item L'affichage est défini comme étant un affichage de 2x40 dans le fichier lcd.txt.
  
  \item Pour le type de définition avec isdn\_rate et 4x40, vous devez faire attention
	aux lignes et à la taille de la colonne.
  \end{itemize}

  Malheureusement, il n'existe pas de norme pour le brochage du port parallèle sur
  la carte mère. Pour l'utilisation interne des modules LCD, consultez la documentation
  de la carte mère concernant le câblage des broches. Il convient d'utiliser les
  connecteurs suivant, la souris (PS/2), le clavier (DIN, PS/2), le port jeu, le port
  USB ou une connexion libre de l'alimentation du PC. Certains fabricants de carte son
  génèrent des signaux spéciaux sur le port jeu, on ne peut pas garantir que cela
  fonctionnera dans n'importe quelle combinaison. Prenez soin de toujours mesurer avant~!

\subsection{Remerciment}

  Je remercie~:

  \begin{itemize}
  \item Frank Meyer pour l'interface imond et fli4l :)
   
  \item Gernot Miksch pour le paquetage LCD
   
  \item Michael Reinelt (\altlink{https://ssl.bulix.org/projects/lcd4linux}) pour
        le programme LCD4Linux
  \end{itemize}
