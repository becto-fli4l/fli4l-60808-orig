% Last Update: $Id$
\marklabel{sec:opt-easycron}
{
\section {EASYCRON - Befehle zeitgesteuert ausführen}
}

Dieses Paket wurde von Stefan Manske
\email{fli4l@stephan.manske-net.de} zusammengestellt und vom
fli4l-Team an 2.1 angepaßt.


\subsection{Konfiguration}


       Mit \var{OPT\_\-EASYCRON} kann man über das entsprechende config-file
       gesteuert zu bestimmten Zeiten Befehle ausführen lassen.

       Dabei werden folgende Einträge benutzt:


\begin{description}
\config{OPT\_EASYCRON}{OPT\_EASYCRON}{OPTEASYCRON} mit \var{OPT\_EASYCRON}='yes' wird das Paket aktiviert

         Standard-Einstellung: \var{OPT\_\-EASYCRON}='no'


\config{EASYCRON\_MAIL}{EASYCRON\_MAIL}{EASYCRONMAIL}
         Da immer wieder Probleme auftraten, dass der crond
         unerwünschte Mails verschickt, kann man dies generell mit
         diesem Flag verhindern. 

         Standard-Einstellung: \var{EASYCRON\_MAIL}='no'


\config{EASYCRON\_N}{EASYCRON\_N}{EASYCRONN}
         Die Anzahl der verschiedenen Befehle, die von cron gestartet
         werden sollen.


\config{EASYCRON\_x\_CUSTOM}{EASYCRON\_x\_CUSTOM}{EASYCRONxCUSTOM}
         Wer sich mit den Einstellungen in der crontab auskennt, kann
         hier für jeden Eintrag eigene Einstellungen wie MAILTO, PATH,
         ... einstellen. Mehrere Einträge müssen durch 
	 \var{$\backslash\backslash$} getrennt werden. Hier sollte man sich
	 aber wirklich mit cron auskennen.

         Standard-Einstellung: \var{EASYCRON\_\-CUSTOM}=''


\config{EASYCRON\_x\_COMMAND}{EASYCRON\_x\_COMMAND}{EASYCRONxCOMMAND}
         In \var{EASYCRON\_\-x\_\-COMMAND} wird der gewünschte Befehl
         eigetragen, wie z.B.
\begin{example}
\begin{verbatim}
        EASYCRON_1_COMMAND='echo 1 '>' /dev/console'
\end{verbatim}
\end{example}

\config{EASYCRON\_x\_TIME}{EASYCRON\_x\_TIME}{EASYCRONxTIME}
         In \var{EASYCRON\_\-x\_\-TIME} wird die Ausführungszeit gemäß der üblichen cron-Syntax eingetragen.




\subsection{Beispiele}

\begin{itemize}
\item Der Computer wünscht uns ``Ein gutes neues Jahr''
\begin{example}
\begin{verbatim}
        EASYCRON_1_COMMAND = 'echo Ein gutes neues Jahr! > /dev/console'
        EASYCRON_1_TIME    = '0 0 31 12 *'
\end{verbatim}
\end{example}



\item   xxx wird von Montag bis Freitag jeweils von 7-20 Uhr zu jeder
       vollen Stunde ausgeführt.
\begin{example}
\begin{verbatim}
        EASYCRON_1_COMMAND = 'xxx'
        EASYCRON_1_TIME    = '0 7-20 0 * 1-5'
\end{verbatim}
\end{example}



\item   Der Router beendet jede Nacht um 03:40 die Internet-Verbindung
       die per DSL aufgebaut ist baut sie nach 5sec Wartezeit wieder auf.
       Die folgenden Devicenamen sind möglich: pppoe, ippp[1-9], ppp[1-9].
\begin{example}
\begin{verbatim}
        EASYCRON_1_COMMAND = 'fli4lctrl hangup pppoe; sleep 5; fli4lctrl dial pppoe'
        EASYCRON_1_TIME    = '40 3 * * *'
\end{verbatim}
\end{example}


\end{itemize}



       Weitere Informationen zur cron-Syntax finden Sie unter
       \begin{itemize}
       \item \altlink{http://www.pro-linux.de/artikel/2/146/der-batchdaemon-cron.html} 
       \item \altlink{http://de.linwiki.org/wiki/Linuxfibel_-_System-Administration_-_Zeit_und_Steuerung\#Die_Datei_crontab} 
       \item \altlink{http://web.archive.org/web/20021229004331/http://www.linux-magazin.de/Artikel/ausgabe/1998/08/Cron/cron.html}
       \item \altlink{http://web.archive.org/web/20070810063838/http://www.newbie-net.de/anleitung_cron.html}
       \end{itemize}





\subsection{Voraussetzungen}

\begin{itemize}
\item fli4l in einer Version $>$ 2.1.0    
\item für ältere Versionen bitte die entspechenden
  opt\_easycron-Versionen aus der OPT\-Datenbank verwenden
\end{itemize}



\subsection{Installation}

\var{OPT\_\-EASYCRON} wird einfach wie jedes andere OPT im aktuelle
fli4l-Verzeichnis entpackt.


\end{description}
