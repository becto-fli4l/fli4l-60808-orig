% Do not remove the next line
% Synchronized to r29817

\marklabel{sec:opt-easycron}
{
\section {EASYCRON~- Exécuter une commande planifiée}
}

Ce paquetage a été élaboré par Stefan Manske
\email{fli4l@stephan.manske-net.de} et a été adapté pour la version 2.1 par
l'équipe pour fli4l.


\subsection{Configuration}

      Dans \var{OPT\_\-EASYCRON} on peut exporter avec le fichier config des
      commandes et les exécuter à un moment donné.

      Pour cela il faut enregistrer les paramètres suivants~:

\begin{description}
\config{OPT\_EASYCRON}{OPT\_EASYCRON}{OPTEASYCRON}

         Avec la variable sur \var{OPT\_EASYCRON}='yes' le paquetage est activé

         Installation par défaut~: \var{OPT\_\-EASYCRON}='no'

\config{EASYCRON\_MAIL}{EASYCRON\_MAIL}{EASYCRONMAIL}

         Depuis toujours, des problèmes se produisaient par l'envoi de Mails
         indésirables par crond, on peut généralement empêcher cela avec le
         paramètre suivant.

         Installation par défaut~: \var{EASYCRON\_MAIL}='no'

\config{EASYCRON\_N}{EASYCRON\_N}{EASYCRONN}

         Avec cette variable on indique le nombre de commande à exécuter
         par cron.

\config{EASYCRON\_x\_CUSTOM}{EASYCRON\_x\_CUSTOM}{EASYCRONxCUSTOM}

         Ceux qui sont familiers avec les paramètres de crontab, peut définir
         les paramètres supplémentaires comme MAILTO, PATH, ... Si vous définissez
         plusieurs paramètres, vous devez les séparées par \var{$\backslash\backslash$}.
         Vous devez être très familier avec cron pour utiliser ces options

         Installation par défaut~: \var{EASYCRON\_\-CUSTOM}=''

\config{EASYCRON\_x\_COMMAND}{EASYCRON\_x\_COMMAND}{EASYCRONxCOMMAND}

         Dans cette variable \var{EASYCRON\_\-x\_\-COMMAND} vous Indiquez la
         commande à exécuter, par ex.
\begin{example}
\begin{verbatim}
        EASYCRON_1_COMMAND='echo 1 '>' /dev/console'
\end{verbatim}
\end{example}

\config{EASYCRON\_x\_TIME}{EASYCRON\_x\_TIME}{EASYCRONxTIME}

         Avec cette variable \var{EASYCRON\_\-x\_\-TIME} vous indiquez le temps
         d'exécution de la syntaxe-cron.


\subsection{Exemples}

\begin{itemize}
\item   L'ordinateur nous souhaite une "Bonne nouvelle année" chaque année.
\begin{example}
\begin{verbatim}
        EASYCRON_1_COMMAND = 'echo Bonne nouvelle année~! > /dev/console'
        EASYCRON_1_TIME    = '0 0 31 12 *'
\end{verbatim}
\end{example}

\item   La commande xxx est exécutée du lundi au vendredi de 7-20 heures
       à chaque heure pleine.
\begin{example}
\begin{verbatim}
        EASYCRON_1_COMMAND = 'xxx'
        EASYCRON_1_TIME    = '0 7-20 0 * 1-5'
\end{verbatim}
\end{example}


\item   Le routeur arrête la connexion DSL toutes les nuits au alentour 03:40,
       il y a un temps d'attente 5 secondes avant la déconnexion. Il est possibles
       d'indiquer les noms de périphériques~: pppoe, ippp[1-9], ppp[1-9].
\begin{example}
\begin{verbatim}
        EASYCRON_1_COMMAND = 'fli4lctrl hangup pppoe; sleep 5; fli4lctrl dial pppoe'
        EASYCRON_1_TIME    = '40 3 * * *'
\end{verbatim}
\end{example}
\end{itemize}


       Pour plus d'informations sur les syntaxes de cron, consultez les sites
       \begin{itemize}
       \item \altlink{http://www.pro-linux.de/artikel/2/146/der-batchdaemon-cron.html}
       \item \altlink{http://de.linwiki.org/wiki/Linuxfibel_-_System-Administration_-_Zeit_und_Steuerung\#Die_Datei_crontab}
       \item \altlink{http://web.archive.org/web/20021229004331/http://www.linux-magazin.de/Artikel/ausgabe/1998/08/Cron/cron.html}
       \item \altlink{http://web.archive.org/web/20070810063838/http://www.newbie-net.de/anleitung_cron.html}
       \end{itemize}


\subsection{Conditions}

\begin{itemize}
\item  S'utilise avec la version fli4l $>$ 2.1.0
\item  Pour les anciennes versions de fli4l voir dans OPT\-Base-de-donnée
      opt\_easycron-version
\end{itemize}


\subsection{Installation}

  Décompresser simplement la paquetage \var{OPT\_\-EASYCRON} dans le répertoire
  actuelle de fli4l.
\end{description}
