% Synchronized to r29817

\section{Localized keyboard layouts}

\begin{description}

  \config{KEYBOARD\_LOCALE}{KEYBOARD\_LOCALE}{KEYBOARDLOCALE}
  
  Default Setting: \var{KEYBOARD\_LOCALE='auto'}

  If you sometimes work directly at the router's console you will appreciate a
  localized keyboard layout. With \var{KEYBOARD\_LOCALE='auto'}, fli4l tries
  to find a keyboard layout that is compatible with the \var{LOCALE} setting.
  With \var{KEYBOARD\_LOCALE=''}, no keyboard layout will be installed on the
  fli4l router, causing the kernel's default layout to be used. Alternatively,
  you may set the variable to the name of a local keyboard layout map. If you
  e.g. use \var{KEYBOARD\_LOCALE='de-latin1'}, the build process checks whether
  there is a file named de-latin1.map in the directory opt/etc. If this is the
  case, this file will be used when configuring the keyboard layout.

  \config{OPT\_MAKEKBL}{OPT\_MAKEKBL}{OPTMAKEKBL}
  
  Default Setting: \var{OPT\_MAKEKBL='no'}

  If you want to create a map file for your keyboard, you have to proceed as
  follows:

\begin{itemize}

  \item Set \var{OPT\_MAKEKBL} to `yes'.

  \item Invoke 'makekbl.sh' on the router. Preferably, you use a SSH connection
  as the keyboard layout changes and this can be quite annoying.

  \item Follow the instructions.

  \item You will find your new $<$locale$>$.map file in /tmp.

  The tasks to be done directly on the router are now completed.

  \item Copy the keyboard layout map you have just created to your fli4l
  directory under opt/etc/$<$locale$>$.map. If you now set
  \var{KEYBOARD\_LOCALE}='$<$locale$>$', your freshly created keyboard layout
  will be used when building the fli4l images the next time.

  \item Don't forget to set \var{OPT\_MAKEKBL} to `no' again.

\end{itemize}

\end{description}
