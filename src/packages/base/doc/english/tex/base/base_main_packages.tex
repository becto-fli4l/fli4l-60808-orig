% Synchronized to r38958

  \chapter{Packages}

  Besides the BASE installation there are also packages.
  Each package contains one or more ``OPTs''
  \footnote{abbreviation for ``OPTional module''} which can be installed in
  addition to the base installation. Some of the OPTs are part of the BASE
  package, other have to be downloaded separately. The download site
  (\altlink{http://www.fli4l.de/en/download/stable-version/}) gives an overview over
  the packages provided by the fli4l team, the OPT database
  (\altlink{http://extern.fli4l.de/fli4l_opt-db3/}) contains packages offered
  by other authors. In the following, we describe the packages supplied by
  the fli4l team.

  \section{Tools In The Package 'Base'}

  The following OPTs are contained in the BASE package:
  \begin{center}
  \begin{tabular}{@{}lp{12cm}@{}}\hline
    Name               & Description \\\hline
    \var{OPT\_SYSLOGD} & \jump{OPTSYSLOGD}{Tool for logging system messages}\\
    \var{OPT\_KLOGD}   & \jump{OPTKLOGD}{Tool for logging kernel messages}\\
    \var{OPT\_LOGIP}   & \jump{OPTLOGIP}{Tool for logging WAN IP addresses}\\
    \var{OPT\_Y2K}     & \jump{OPTY2K}{Date correction utility for systems that are not Y2K-safe}\\
    \var{OPT\_PNP}     & \jump{OPTPNP}{Installation of ISAPnP tools}\\
    \var{OPT\_HOTPLUG\_PCI} & \jump{OPTHOTPLUGPCI}{\mtr{Aktivating PCI hotplugging}}\\\hline
  \end{tabular}
  \end{center}

\marklabel{OPTSYSLOGD}{\subsection{OPT\_SYSLOGD~-- Logging system messages}}\index{OPT\_SYSLOGD}

  Many programs use the Syslog interface to log messages. If you want to see
  these messages on your fli4l console you have to start the syslogd daemon.

  Setting \var{OPT\_\-SYSLOGD} to `yes' enables debugging messages, `no'
  disables them.

  See also \jump{ISDNCIRCxDEBUG}{\var{ISDN\_CIRC\_x\_DEBUG}} and
  \jump{PPPOEDEBUG}{\var{PPPOE\_DEBUG}}.


  Default Setting: \var{OPT\_\-SYSLOGD}='no'

  \begin{description}
    \config{SYSLOGD\_RECEIVER}{SYSLOGD\_RECEIVER}{SYSLOGDRECEIVER}

  \var{SYSLOGD\_RECEIVER} controls whether fli4l can receive Syslog messages
  from other hosts in the network.

    \configlabel{SYSLOGD\_DEST\_x}{SYSLOGDDESTx}
    \config{SYSLOGD\_DEST\_N SYSLOGD\_DEST\_x}{SYSLOGD\_DEST\_N}{SYSLOGDDESTN}

  \var{SYSLOGD\_\-DEST\_\-x} describes where the system messages
  being received by syslogd should be displayed. Normally, this is fli4l's console,
  hence:

\begin{example}
\begin{verbatim}
        SYSLOGD_DEST_1='*.* /dev/console'
\end{verbatim}
\end{example}

  If you want to log the messages into a file, you can e.g. use:

\begin{example}
\begin{verbatim}
        SYSLOGD_DEST_1='*.* /var/log/messages'
\end{verbatim}
\end{example}

  If you have a so-called ``log host'' in your network you can redirect the
  Syslog messages to that host if you supply its IP address.

  Beispiel:
\begin{example}
\begin{verbatim}
        SYSLOGD_DEST_1='*.* @192.168.4.1'
\end{verbatim}
\end{example}

  The ``@'' sign has to be prepended to the IP address.

  If you want the Syslog messages to be delivered to multiple destinations
  it is necessary to increase the variable \var{SYSLOGD\_DEST\_N} (number of
  destinations used) accordingly and to fill the variables
  \var{SYSLOG\_DEST\_1}, \var{SYSLOG\_DEST\_2} etc. with appropriate content.

  The syntax `*.*' directs syslogd to log all messages. However, you are also
  able to constrain the messages to be logged for certain destinations by the
  use of so-called ``priorities''. In this case you need to replace the
  asterisk (*) after the dot (.) by one of the following keywords:
  \begin{itemize}
  \item debug
  \item info
  \item notice
  \item warning (deprecated: warn)
  \item err (deprecated: error)
  \item crit
  \item alert
  \item emerg (deprecated: panic)
  \end{itemize}

  The items in the list are descending sorted according to severity. The
  keywords ``error'', ``warn'', and ``panic'' are deprecated---you should not
  use them anymore.

  You can replace the asterisk (*) in front of the dot by a so-called ``facility''.
  However, a detailed explanation is outside this scope. You can find an
  overview over the available facilities at the man page of syslog.conf:

  \enlargethispage{\baselineskip}
  \noindent \altlink{http://linux.die.net/man/5/syslog.conf}

  In most cases an asterisk is completely sufficient. Example:
\begin{example}
\begin{verbatim}
          SYSLOGD_DEST_1='*.warning @192.168.4.1'
\end{verbatim}
\end{example}
  Windows hosts can serve as log hosts as well as Unix/Linux hosts. You can
  find links to adequate software at \altlink{http://www.fli4l.de/en/other/links/}.
  Using a log host is strongly recommended if you want a detailed
  logging protocol. The protocol is also useful for debugging purposes. The Windows
  client imonc also ``understands'' the Syslog protocol and is able to display
  the messages in a window.

  Unfortunately, messages generated during the boot process cannot be
  directed to syslogd. However, you can configure fli4l to use a serial port
  as a terminal. You can find more information on this topic in the section
  \jump{CONSOLESETTINGS}{Console settings}.

  \config{SYSLOGD\_ROTATE}{SYSLOGD\_ROTATE}{SYSLOGDROTATE}

  You can use \var{SYSLOGD\_ROTATE} in order to control whether Syslog message
  files are rotated once a day, thereby archiving the messages of the last x
  days.

  \config{SYSLOGD\_ROTATE\_DIR}{SYSLOGD\_ROTATE\_DIR}{SYSLOGDROTATEDIR}

  The optional variable \var{SYSLOGD\_ROTATE\_DIR} lets you specify the
  directory where the archived Syslog files should be stored. Leave it empty
  to use the default directory /var/log.

  \config{SYSLOGD\_ROTATE\_MAX}{SYSLOGD\_ROTATE\_MAX}{SYSLOGDROTATEMAX}

  The optional variable \var{SYSLOGD\_ROTATE\_MAX} lets you specify the number of
  archived/rotated Syslog files.

  \config{SYSLOGD\_ROTATE\_AT\_SHUTDOWN}{SYSLOGD\_ROTATE\_AT\_SHUTDOWN}{SYSLOGDROTATEATSHUTDOWN}

  With the optional variable \var{SYSLOGD\_ROTATE\_AT\_SHUTDOWN} you can disable the rotate
  of syslog files at shutdown. Please only do this, if your syslogfiles are written directly to
  a destination on a permanent disk.

\end{description}


\marklabel{OPTKLOGD}{\subsection{OPT\_KLOGD~-- Logging kernel messages}}\index{OPT\_KLOGD}

  Many errors, e.g. a dial-in that failed, are written directly to the console
  by the Linux kernel. If you set \var{OPT\_\-KLOGD}='yes', these messages are
  redirected to the Syslog daemon which can log them to a file or send them to
  a log host (see above). This keeps your fli4l console (almost) clear.

  \noindent Recommendation: If you use \var{OPT\_\-SYSLOGD}='yes' you should
  also set \var{OPT\_\-KLOGD} to `yes'.

  Default setting: \var{OPT\_\-KLOGD}='no'


\marklabel{OPTLOGIP}{\subsection{OPT\_LOGIP~-- Logging WAN IP addresses}}\index{OPT\_LOGIP}

  LOGIP logs your WAN IP address to a log file. You activate this logging by
  setting \var{OPT\_LOGIP} to `yes'.

  Default setting: \var{OPT\_LOGIP}='no'

\begin{description}
  \config{LOGIP\_LOGDIR}{LOGIP\_LOGDIR}{LOGIPLOGDIR}{~-- Configure directory of log file}

  The variable \var{LOGIP\_LOGDIR} contains the directory where the log
  file should be created or 'auto' for autodetect.

  Default setting: \var{LOGIP\_LOGDIR}='auto'
\end{description}

\marklabel{OPTY2K}{\subsection{OPT\_Y2K~-- Date correction for systems that are not Y2K-safe}}\index{OPT\_Y2K}

  fli4l routers are often assembled from old hardware parts. Older
  mainboards may have a BIOS that is not Y2K-safe. This can lead to the
  situation that setting the system date to the 27th May 2000 causes the
  BIOS date to become the 27th May 2094 after a reboot. By the way, Linux 
  will then show the 27th May 1994 as system date.

  Normally the system date reflected by fli4l is not important and should
  not matter at all. If you use the LCR (Least Cost Routing) functionality
  of your fli4l router this may very well play a role.

  The reason: The 27th May 1994 was a Friday, the 27th May 2000 in contrast was
  a Saturday. And for the weekend there are lower-priced rates or providers,
  respectively \ldots

  A first solution to that problem is as follows: The BIOS date is changed
  from the 27th May 2000 to the 28th May 1994 which was a Saturday, too.
  However, the problem is not solved completely yet: Not only does fli4l
  use the day of week and the current time for least-cost routing but it also
  respects bank holidays.

\begin{description}
  \config{Y2K\_DAYS}{Y2K\_DAYS}{Y2KDAYS}{~-- add N days to the system date}

  Because the BIOS date differs from the actual one by exactly 2191 days, the
  setting

\begin{example}
\begin{verbatim}
        Y2K_DAYS='2191'
\end{verbatim}
\end{example}

  causes the fli4l router to add 2191 days to the BIOS date before using it
  as the Linux system date The BIOS date is left untouched because
  otherwise the year would be wrong (2094 or 1994, resp.) again after the next
  boot.
\end{description}
  There is an additional alternative:

  Using a time server, fli4l is able to fetch the current date and time from
  the Internet. The package \jump{sec:opt-chrony}{\var{CHRONY}} is designed
  for this purpose. Both settings can be combined. This is useful as it allows
  to correct the date via \var{Y2K\_DAYS} before setting the exact time using
  the information from the time server.

  If you do not have any problems with Y2K, set \var{OPT\_\-Y2K}='no' and
  forget it \ldots


\marklabel{OPTPNP}{\subsection{OPT\_PNP~-- Installation of ISAPnP tools}}\index{OPT\_PNP}

  Some ISAPnP adapters have to be configured by the ``isapnp'' tool. This
  especially affects ISDN adapters with a \var{ISDN\_\-TYPE} of 7, 12, 19, 24,
  27, 28, 30, and 106~-- but only if the adapter is really an ISAPnP adapter.

  For proper configuration you have to create the file
  ``etc/isapnp.conf''.

  Brief instructions to create this file follow:

  \begin{itemize}
  \item In $<$config$>$/base.txt, set \var{OPT\_\-PNP}='yes' and
    \var{MOUNT\_\-BOOT}='rw'
  \item boot your fli4l~-- the ISAPnP adapter will most likely not be detected
  \item Logon to the fli4l's console and type:
\begin{example}
\begin{verbatim}
        pnpdump -c >/boot/isapnp.conf
        umount /boot
\end{verbatim}
\end{example}
    This saves the ISAPnP configuration to your boot medium.
  \end{itemize}
  Continue on your PC (Unix/Linux/Windows):
  \begin{itemize}
  \item Copy the file isapnp.conf from your boot medium to $<$config$>$/etc/isapnp.conf
  \item Edit isapnp.conf and save your changes\\
        The default values can be left unchanged or be replaced by the values
        proposed. The relevant lines are shown in the following example:

\begin{example}
\begin{verbatim}
            #     Start dependent functions: priority acceptable
            #       Logical device decodes 16 bit IO address lines
            #             Minimum IO base address 0x0160
            #             Maximum IO base address 0x0360
            #             IO base alignment 8 bytes
            #             Number of IO addresses required: 8
       1)      (IO 0 (SIZE 8) (BASE 0x0160))
            #       IRQ 3, 4, 5, 7, 10, 11, 12 or 15.
            #             High true, edge sensitive interrupt (by default)
       2)      (INT 0 (IRQ 10 (MODE +E
\end{verbatim}
\end{example}

    \begin{description}
      \item 1)~-- Here, you can choose the I/O \glqq{}BASE\grqq{} address. This
                 address must lie between the minimum and maximum address and
                 conform to the \glqq{}base alignment\grqq{}.\\
                 If your system uses more than one ISA adapter, you will have
                 to ensure that there are no overlaps between address ranges.
                 The address range starts at \glqq{}BASE\grqq{} and ends at
                 \glqq{}BASE + Number of IO addresses required\grqq{}.

      \item 2)~-- Here you can pick an IRQ from the list shown.
                 The IRQs 2(9), 3, 4, 5, and 7 should be avoided
                 as these IRQs may clash with your serial and parallel
                 interfaces or the interrupt cascading.\\
                 ISA adapters are not able to share IRQs, thus
                 the IRQ you choose here may not be used elsewhere.
    \end{description}
  \item Put the chosen configuration (IRQ/IO) into $<$config$>$/isdn.txt
  \item In order to let the necessary files be copied to the boot medium, you
    must set \var{OPT\_\-PNP} to 'yes' in $<$config$>$/base.txt. The variable
    \var{MOUNT\_BOOT} can be chosen freely, however.
  \item Create new boot medium
  \end{itemize}
  \achtung {The automatically generated file contains Unix line endings (LF
  without CR). Thus, if you use Notepad under Windows, all content is shown in
  a single line. In contrast, the DOS editor ``edit'' is able to cope with the
  Unix line endings. When saved, however, they are changed to DOS line endings
  (CR+LF).}

  Workaround:
  \begin{itemize}
  \item start DOS box
  \item change to the directory $<$config$>$/etc
  \item type:
    edit isapnp.conf
  \item edit file and save your changes
  \end{itemize}
  After that you can also use Notepad to edit the file.

  Under Windows you may also use the Wordpad editor.

  The CRs generated by the ``edit'' tool are filtered when fli4l boots and thus
  do not disturb.

  Please try first to get along without using \var{OPT\_\-PNP}. If the
  adapter is not recognized you may follow the procedure described
  above.

  If you update to a more recent fli4l version, you may reuse the previously
  created \hbox{isapnp.conf}.

  Default setting: \var{OPT\_\-PNP}='no'

\marklabel{OPTHOTPLUGPCI}{\subsection{OPT\_HOTPLUG\_PCI~-- Aktivating PCI hotplugging}}\index{OPT\_HOTPLUG\_PCI}

  Specifying \var{OPT\_HOTPLUG\_PCI='yes'} copies some modules to fli4l activating
  PCI hotplugging and loads them during the boot process. This enables adding
  and removing of PCI adapters during runtime. A suitable PCI hotplug controller
  has to be used for this functionality.

  This option does \emph{not} have to be activated for adding and removing
  of virtual devices in \emph{virtualization environments} like KVM, those
  use ACPI mechanisms and ACPI drivers are activated in the kernel anyway.
