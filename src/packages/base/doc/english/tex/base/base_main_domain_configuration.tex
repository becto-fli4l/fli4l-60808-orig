% Synchronized to r40999

\marklabel{sec:domainkonfiguration}{\section{Domain configuration}}

  Windows PCs exhibit a somewhat annoying behaviour: If a DNS server is needed
  and configured at the Windows system, the server is queried regularly (every
  five minutes)~-- even if you don't work at the PC!

  If you configured an Internet DNS server at your Windows PC, your next
  bill might become quite expensive :-(

  If you don't already run a DNS server in your LAN, this problem can be
  solved by enabling the DNS server of your fli4l router. The DNS server
  software used is DNSMASQ.

  Before you start configuring your DNS, however, you should give careful
  consideration to the domain name and the names of the PCs in your network.
  The domain name you use will not be visible in the Internet. Therefore, you
  are free to choose any domain name you like.

  Additionally, each of your PCs in the LAN has to have a name assigned. These
  names have to be known by the fli4l router.

  \begin{description}

    \config{DOMAIN\_NAME}{DOMAIN\_NAME}{DOMAINNAME}
    
    Default Setting: \var{DOMAIN\_NAME='lan.fli4l'}

    {You can freely choose any domain name as this local domain is not visible 
      in the Internet. However, you should avoid choosing a name that may exist in
      the Internet (e.g. somewhat.com) because you won't be able to access
      that Internet domain.}


    \config{DNS\_FORWARDERS}{DNS\_FORWARDERS}{DNSFORWARDERS}

    Default Setting: \var{DNS\_FORWARDERS=''}
    
    {This variable contains the address of your Internet provider's DNS server
      if you want your fli4l router to route Internet traffic. The fli4l router
      will forward all DNS queries which it is not able to answer on its own to the
      address in this variable.

      You can specify more than one DNS forwarder by separating the addresses
      by blanks.

      If more DNS server are specified, they will be queried in the order given
      by the configuration file, meaning the second will only be used in case
      that the first server does not return a valid answer and so on.

      It is also possible to specify a port number for each DNS forwarder
      address which is then to be separated from the address by a colon.
      However, in this case it is required to set \jump{OPTDNS}{\var{OPT\_DNS='yes'}}
      (Package \jump{sec:dnsdhcp}{dns\_dhcp}), and you are not allowed to use
      any of the various \var{*\_USEPEERDNS} options.

      Beware: Even if
        \begin{itemize}
        \item \jump{PPPOEUSEPEERDNS}{\var{PPPOE\_USEPEERDNS}},
        \item \jump{ISDNCIRCxUSEPEERDNS}{\var{ISDN\_CIRC\_x\_USEPEERDNS}}
          or
        \item \jump{DHCPCLIENTxUSEPEERDNS}{\var{DHCPCLIENT\_x\_USEPEERDNS}}
        \end{itemize}

        are set to `yes', you need to fill this variable with a valid DNS
        server address as otherwise no DNS resolution will be possible directly
        after the router has booted.

      Exception: If you use fli4l as a local router \emph{without} a connection
      to the Internet or other (company) networks with DNS servers, you should
      set this variable to `127.0.0.1' in order to disable DNS forwarding
      completely.}

    \config{HOSTNAME\_IP}{HOSTNAME\_IP}{HOSTNAMEIP} (optional)

    {This variable can optionally specify to which network 'IP\_NET\_x' the
     hostname set by \var{HOSTNAME} is bound.}

    \config{HOSTNAME\_ALIAS\_N}{HOSTNAME\_ALIAS\_N}{HOSTNAMEALIASN} (optional)

    {Number of additional alias host names for the router.}

    \config{HOSTNAME\_ALIAS\_x}{HOSTNAME\_ALIAS\_x}{HOSTNAMEALIASx} (optional)

    {Additional alias host name for the router.}

  \end{description}
