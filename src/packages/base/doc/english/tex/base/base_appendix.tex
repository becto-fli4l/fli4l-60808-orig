% Synchronized to r54273

\marklabel{NULLMODEMKABEL}{\section{Null Modem Cable}}

    For using the otional package \jump{sec:optppp}{PPP}
    a null modem cable is needed.

    It needs at least three wires. This is the pin layout:

\begin{example}
\begin{verbatim}

male-  female-                        female-  male-
     pins                                pins
25pin     9pin                        9pin     25pin

    8  +- 1                             1  -+   8
       |                                    |
    3  |  2 ------------\ /------------ 2   |   3
       |                 X                  |
    2  |  3 ------------/ \------------ 3   |   2
       |                                    |
   20  +- 4                             4  -+  20
       |                                    |
    7  |  5 --------------------------- 5   |   7
       |                                    |
    6  +- 6                             6  -+   6

    4  +- 7                             7  -+   4
       |                                    |
    5  +- 8                             8  -+   5


\end{verbatim}
\end{example}


    The plugs have to be soldered with the bridges shown above.

\marklabel{SERIALCONSOLE}{\section{Serial Console}}

    fli4l can be used without monitor and keyboard. A drawback of this 
    setup is that eventual error messages will not get noticed because 
    not all messages can be piped to the syslog-port.

    A possible solution is redirecting of all console messages to a PC 
    or a classic terminal using the serial port of the router.
    Configuration is done by the variables
    \jump{SERCONSOLE}{\var{SER\_CONSOLE}},
    \jump{SERCONSOLEIF}{\var{SER\_CONSOLE\_IF}} and
    \jump{SERCONSOLERATE}{\var{SER\_CONSOLE\_RATE}}.

    Machines with older mainboards/cards don not support higher serial 
    speeds than 38400 Bd. This is why you should try with a maximum of 38400 Bd 
    at first before testing higher port speeds. Since only text messages are
    displayed on the console higher speeds are not evne necessary.

    All messages that usually would go to the console are now redirected 
    to the serial port~-- also messages of the boot process!

    As a cable to the terminal or PC with terminal emulation a 
    \jump{NULLMODEMKABEL}{Null Modem Cable} is used. Using a standard null modem
    cable is discouraged because these have bridges on the handshake wires.
    If Terminal or PC are powered off (or no terminal emulation is loaded) the use
    of a standard null modem cable can thus lead to a hangup.

    This is why a special wiring is needed here for using fli4l also when the 
    terminal is deactivated. You need a 3-wire cable, with some bridges on the plug.
    See \jump{NULLMODEMKABEL}{Nullmodemkabel}.

%\begin{verbatim}
%                      female                   female
%                    9pol  25pol             25pol   9pol
%                     3      2 -------------- 3        2
%                     2      3 -------------- 2        3
%                     7      4 -+          +- 5        8
%                               |          |
%                     8      5 -+          +- 4        7
%                     6      6 -+          +- 6        6
%                               |          |
%                     1      8 -+          +- 8        1
%                               |          |
%                     4     20 -+          +- 20       4
%                     5      7 -------------- 7        5
%\end{verbatim}



    \section{Programs}

    To save space on the boot media the program ``BusyBox'' is used. Ti is a single executable
    containing the standard Unix programs

\begin{example}
\begin{verbatim}
        [, [[, arping, ash, base64, basename, bbconfig, blkid, bunzip2, bzcat, bzip2,
        cat, chgrp, chmod, chown, chroot, cmp, cp, cttyhack, cut, date, dd, df,
        dirname, dmesg, dnsdomainname, echo, egrep, expr, false, fdflush, fdisk, find,
        findfs, grep, gunzip, gzip, halt, hdparm, head, hostname, inetd, init, insmod,
        ip, ipaddr, iplink, iproute, iprule, iptunnel, kill, killall, klogd, less, ln,
        loadkmap, logger, ls, lsmod, lzcat, makedevs, md5sum, mdev, mkdir, mknod,
        mkswap, modprobe, mount, mv, nameif, nice, nslookup, ping, ping6, poweroff,
        ps, pscan, pwd, reboot, reset, rm, rmmod, sed, seq, sh, sleep, sort, swapoff,
        swapon, sync, sysctl, syslogd, tail, tar, test, top, tr, true, tty, umount,
        uname, unlzma, unxz, unzip, uptime, usleep, vi, watch, xargs, xzcat, zcat
\end{verbatim}
\end{example}

    \noindent . These are mostly ``minimalistic'' implementations which 
    do not cover the full functional range but fully reflect the modest
    requirements of fli4l.

    BusyBox is GPL'ed and its source can be obtained at

    \altlink{http://www.busybox.net/}


    \section{Other i4l-Tools}

    There are other tools for isdn4linux that could be used for fli4l.
    It could be that isdnlog is more adequate as a tool to compute online-fees 
    but it's size is 10 times higher than imond's which additionally does monitoring,
    controlling and Least-Cost-Routing.

    \section{Debugging}

    Console-Outputs are most helpful for hunting bugs. But these go by so fast
    on the screen, don't they? Hint: SHIFT-PAGE-UP scrolls back,
    SHIFT-PAGE-DOWN scroll forwards.

    If the error message ``try-to-free-pages'' occurs during router use
    there is not enough RAM left for the programs. Try the following
    options to recover:
    \begin{itemize}
    \item add more RAM
    \item use less Opt-Packages
    \item try a harddisk-installation according to \jump{INSTALLTYPB}{Typ B}
    \end{itemize}

    proc-files can help debugging, for example executing

\begin{example}
\begin{verbatim}
                cat /proc/interrupts
\end{verbatim}
\end{example}
    
    shows the interrupts used by the drivers~-- not those
    used by the hardware!
    
    More interesting files under /proc are devices, dma,
    ioports, kmsg, meminfo, modules, uptime, version and pci (if
    the router has a PCI-Bus).
    
    Often a connection problem with ipppd is caused by 
    failing authentification. The variables

\begin{example}
\begin{verbatim}
        OPT_SYSLOGD='yes'
\end{verbatim}
\end{example}

\begin{example}
\begin{verbatim}
        OPT_KLOGD='yes'
\end{verbatim}
\end{example}

    in config/base.txt and

\begin{example}
\begin{verbatim}
        ISDN_CIRC_x_DEBUG='yes'
\end{verbatim}
\end{example}

    in config/isdn.txt can help here.


    \section{Literature}

    \begin{itemize}
    \item Computer Networks, Andy Tanenbaum
    \item TCP/IP Netzanbindung von PCs, Craig Hunt
    \item TCP/IP, Kevin Washburn, Jim Evans, Verlag: Addison-Wesley, \\ISBN: 3-8273-1145-4
    \item TCP/IP Netzanbindung von PCs, ISBN 3-930673-28-2
    \item TCP/IP Netzwerk Administration, ISBN 3-89721-110-6
    \item Linux-Anwenderhandbuch, ISBN 3-929764-06-7
    \item TCP/IP im Detail:\\
      \altlink{http://www.nickles.de/c/s/ip-adressen-112-1.htm}
    \item Generell das online Linuxanwenderhandbuch von Lunetix unter:\\
      \altlink{http://www.linux-ag.com/LHB/}
    \item Einführung in die Linux-Firewall:
      \altlink{http://www.little-idiot.de/firewall/}
    \end{itemize}

    \section{Prefixes}

    For units, prefixes addressed in this document are according to \verb+IEC 60027-2+.\\
    See: \altlink{http://physics.nist.gov/cuu/Units/binary.html}.

    \section{Warranty and Liability}

    There is no warranty and liability whatsoever for the whole fli4l
    distribution or parts of it. Also there is no guarantee for
    function or correct documentation whereever you may find it.
    
    There is no Liability at all for eventual damages or costs that
    may arise! In other words: Don't complain if it eats your hamster.

    \section{Credits}
    \newcommand{\membermail}[3]{\multicolumn{2}{l}{#1 (\emph{#2})}\\\nopagebreak & \email{#3}\\}
    \newcommand{\member}[2]{#1 (\emph{#2})\\}
    \newcommand{\personmail}[2]{#1 & \email{#2}\\}
    \newcommand{\person}[1]{#1\\}
    
    In this part of the documentation all people are honored who
    contribute or have contributed to the development of fli4l.

    \subsection{Foundation Of The Project}
    
    \begin{tabular}{ll}
      \person{Meyer, Frank}
    \end{tabular}\latex{\\}

    \noindent\latex{\parbox{\textwidth}}{
    Frank started the Projekt fli4l on May, 4th 2000!\\
    See: \altlink{http://www.fli4l.de/home/eigenschaften/historie/}}

    \subsection {Developer- and Testteam}

    \noindent \textbf{The fli4l-Team consists of (in alphabetical order):}

    \begin{tabular}{l}
      \member{Charrier, Bernard}     {french translation}
      \member{Eckhofer, Felix}       {documentation, howtos}
      \member{Franke, Roland}        {OW, FBR}
      \member{Hilbrecht, Claas}      {VPN, kernel}
      \member{Klein, Sebastian}      {kernel, wlan}
      \member{Knipping, Michael}     {accounting}
      \member{Krister, Stefan}       {opt-Cop, lcd4linux}
      \member{Miksch, Gernot}        {LCD}
      \member{Schiefer, Peter}       {fli4l-CD, opt-cop, website, releasemanagement}
      \member{Schliesing, Manfred}   {testing}
      \member{Schulz, Christoph}     {FBR, IPv6, kernel}
      \member{Siebmanns, Harvey}     {documentation, english translation}
      \member{Spieß, Carsten}        {dsltool, hwsupp, rrdtool, webgui}
      \member{Vosselman, Arwin}      {LZS-compression, documentation}
      \member{Weiler, Manuela}       {CD-shipping, treasurer}
      \member{Weiler, Marcel}        {quality management}
      \member{Wolters, Florian}      {firmware, kernel}
    \end{tabular}

    \subsection {Developer- and Testteam (inactive)}

    \begin{tabular}{l}
      \member{Arndt, Kai-Christian}  {USB}
      \member{Bauer, Jürgen}         {LCD-Package, fliwiz}
      \member{Behrends, Arno}        {Support}
      \member{Blokland, Kees}        {english translation}
      \member{Bork, Thomas}          {lpdsrv}
      \member{Bußmann, Lars}         {testing}
      \member{Cerny, Carsten}        {Website, fliwiz}
      \member{Dawid, Oliver}         {dhcp, uClibc}
      \member{Ebner, Hannes}         {QoS}
      \member{Fischer, Joerg}        {testing}
      \member{Frauenhoff, Peter}     {testing}
      \member{Grabner, Hans-Joerg}   {imonc}
      \member{Grammel, Matthias}     {english translation}
      \member{Gruetzmacher, Tobias}  {Mini-httpd, imond, proxy}
      \member{Hahn, Joerg}           {IPSEC}
      \member{Hanselmann, Michael}   {Mac OS X/Darwin}
      \member{Hoh, Jörg}             {Newsletter, NIC-DB, events}
      \member{Hornung, Nicole}       {Verein}
      \member{Horsmann, Karsten}     {Mini-httpd, WLAN}
      \member{Janus, Frank}          {LCD}
      \member{Kaiser, Gerrit}        {Logo}
      \member{Karner, Christian}     {PPTP-Package}
      \member{Klein, Marcus}         {Problemfeedback}
      \member{Lammert, Gerrit}       {HTML-documentation}
      \member{Lanz, Ulf}             {LCD}
      \member{Lichtenfeld, Nils}     {QoS}
      \member{Neis, Georg}           {fli4l-CD, documentation}
      \member{Peiser, Steffen}       {FAQ}
      \member{Peus, Christoph}       {uClibc}
      \member{Pohlmann, Thorsten}    {Mini-httpd}
      \member{Raschel, Tom}          {IPX}
      \member{Reinard, Louis}        {CompactFlash}
      \member{Resch, Robert}         {PCMCIA, WLAN}
      \member{Schäfer, Harald}       {HDD-Support}
      \member{Schmitts, Jupp}        {testing}
      \member{Strigler, Stefan}      {GTK-Imonc, Opt-DB, NG}
      \member{Wallmeier, Nico}       {windows-imonc}
      \member{Walter, Gerd}          {UMTS}
      \member{Walter, Oliver}        {QoS}
      \member{Wolter, Jean}          {Paketfilter, uClibc}
      \member{Zierer, Florian}       {wishlist}
    \end{tabular}

    \subsection {Sponsors}

    \noindent Meanwhile fli4l is a registered als Word-/trademark.
    The following fli4l-Users (there are more that did not want to be mentioned) have
    helped to raise the money needed for this:\\

    \begin{tabular}{l}
      \person{Bebensee, Norbert}
      \person{Becker, Heiko}
      \person{Behrends, Arno}
      \person{Böhm, Stefan}
      \person{Brederlow, Ralf}
      \person{Groot, Vincent de}
      \person{Hahn, Olaf}
      \person{Hogrefe, Paul}
      \person{Holpert, Christian}
      \person{Hornung, Nicole}
      \person{Kuhn, Robert}
      \person{Lehnen, Jens}
      \person{Ludwig, Klaus-Ruediger}
      \person{Mac Nelly, Christa}
      \person{Mahnke, Hans-Jürgen}
      \person{Menck, Owen}
      \person{Mende, Stefan}
      \person{Mücke, Michael}
      \person{Roessler, Ingo}
      \person{Schiele, Michael}
      \person{Schneider, Juergen}
      \person{Schönleber, Suitbert}
      \person{Sennewald, Matthias}
      \person{Sternberg, Christoph}
      \person{Vollmar, Thomas}
      \person{Walter, Oliver}
      \person{Wiebel, Christian}
      \person{Woelk, Fabian}
    \end{tabular}\latex{\\}

    \noindent For some time fli4l has its own sponsors, whose
    \mbox{(Hardware-)donations} have helped to support the fli4l development.
    In detail this are adapters, CompactFlash and Ethernet cards.\\

    \noindent Hardware-donours (in alphabetical order):\\

    \begin{tabular}{l}
      \person{Baglatzis, Stephanos}
      \person{Bauer, Jürgen}
      \person{Dross, Heiko}
      \person{Kappenhagen, Wenzel}
      \person{Kipka, Joachim}
      \person{Klopfer, Tom}
      \person{Peiser, Steffen}
      \person{Reichelt, Detlef}
      \person{Reinard, Louis}
      \person{Stärkel, Christopher}
    \end{tabular}\latex{\\\\}

    \noindent\latex{\parbox{\textwidth}}{
    More sponsors can be found on the fli4-homepage:\\
    \altlink{http://www.fli4l.de/sonstiges/sponsoren/}}

    \section{Feedback}

    Critics, feedback and cooperation are always welcome.

    The primary point of contact are the fli4l-Newsgroups. Those having problems
    in the setup of a fli4l-Routers, should at first read the FAQ, Howtos and the 
    NG-Archives, before posting in the newsgroups. Informations on the
    different groups and netiquette can be found on the fli4l-Webseite:

           \altlink{http://www.fli4l.de/hilfe/newsgruppen/}\\
    \indent\altlink{http://www.fli4l.de/hilfe/faq/}\\
    \indent\altlink{http://www.fli4l.de/hilfe/howtos/}\\

    Because mostly older hardware is used for fli4l problems may be
    inevitable. Informations can help other fli4l-users having hardware
    problems with PC-Cards (I/O-Addresses, Interrupts and so on).

    On fli4l's website is a link to a network card database, where
    appropriate drivers for certain cards are listed and can be entered.

        \altlink{http://www.fli4l.de/hilfe/nic-db/}

    \bigskip

    Have fun with fli4l!
