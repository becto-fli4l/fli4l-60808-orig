% Synchronized to r35857

\section{Module Concept}

As of version 2.0 fli4l is split into modules (packages), i.e.

\begin{itemize}
    \item fli4l-\version~~$<$--- The Base Package
    \item dns-dhcp
    \item dsl
    \item isdn
    \item sshd
\item and much more...
\end{itemize}


With the base package fli4l acts as a pure Ethernet router. For ISDN
and/or DSL the packages isdn and/or dsl have to be unpacked to the
fli4l directory. The same applies for the other packages.


\marklabel{mkfli4l}{\subsection{mkfli4l}}

Depending on the current configuration a file called \texttt{rc.cfg} and two
archives \texttt{rootfs.img} and \texttt{opt.img} will be generated
which contain all required configuration informations and files.
These files are generated using \var{mkfli4l} which reads the individual
package files and checks for configuration errors.

\var{mkfli4l} will accept the parameters listed in table \ref{tab:mkfli4l}.
If omitted the default values noted in brackets are used. A complete list
of all options (Table \ref{tab:mkfli4l}) is displayed when executing
\begin{verbatim}
    mkfli4l -h
\end{verbatim}
.

\begin{table}[htbp]
  \centering
  \caption{Parameters for \var{mkfli4l}}
  \begin{tabular}{|lp{2cm}|p{8cm}|}
    \hline
    \multicolumn{1}{|c|}{\textbf{Option}} & \multicolumn{2}{c|}{\textbf{Meaning}}\\
    \hline
    -c, -\,-config    & \multicolumn{2}{|p{11cm}|} { Declaration of the
      directory \var{mkfli4l} will scan for package config files (default: config)} \\
    -x, -\,-check     & \multicolumn{2}{|p{11cm}|} { Declaration of the
      directory \var{mkfli4l} will scan for files needed for package error checking
      (\texttt{<package>.txt}, \texttt{<package>.exp} and
      \texttt{<package>.ext}; default: check)} \\
    -l, -\,-log       & \multicolumn{2}{|p{11cm}|} { Declaration of the log
      file to which \var{mkfli4l} will log error messages and warnings
      (default: \texttt{img/mkfli4l.log})} \\
    -p, -\,-package   & \multicolumn{2}{|p{11cm}|} { Declaration of the packages
      to be checked, this option may be used more than once in case of
      a desired check for several packages in conjunction. If using -p,
      however, the file \texttt{<check\_dir>/base.exp} will always be read
      first to provide the common regular expressions provided by the base
      package. Hence, this file must exist.} \\
    -i, -\,-info       & \multicolumn{2}{|p{11cm}|} { Provides information on
      the check (which files are read, which tests are run, which uncommon things
      happened during the review process)} \\
    -v, -\,-verbose    & \multicolumn{2}{|p{11cm}|} { More verbose
      variant of option -i} \\
    -h, -\,-help       & \multicolumn{2}{|p{11cm}|} { Displays the help text} \\
    \html{\multirow{8}{*}{top}} \latex{\multirow{8}{*}{}}{-d, -\,-debug} &
      \multicolumn{2}{|p{11cm}|} { Allows for debugging the review process. This
      is meant to be a help for package developers wishing to know in detail how
      the process of package checking is working.} \\
    \cline{2-3}
    \latex{&} \multicolumn{1}{|c|}{\textbf{Debug Option}} & \multicolumn{1}{c|}{\textbf{Meaning}} \\
    \cline{2-3}
    \latex{&} \multicolumn{1}{|l|}{check} & show check process \\
    \latex{&} \multicolumn{1}{|l|}{zip-list} & show generation of zip list \\
    \latex{&} \multicolumn{1}{|l|}{zip-list-skipped} & show skipped files \\
    \latex{&} \multicolumn{1}{|l|}{zip-list-regexp} & show regular expressions for ziplist \\
    \latex{&} \multicolumn{1}{|l|}{opt-files} & check all files in \texttt{opt/<package>.txt} \\
    \latex{&} \multicolumn{1}{|l|}{ext-trace} & show trace of extended checks \\
    \hline
  \end{tabular}
  \label{tab:mkfli4l}
\end{table}


\subsection{Structure}

A package can contain multiple OPTs, if it contains only one, however, it is
appropriate to name the package like the OPT. Below \texttt{<PACKAGE>} is
to be replaced by the respective package name.
A package consists of the following parts:

\begin{itemize}
\item Administrative Files
\item Documentation
\item Developer Documentation
\item Client Programs
\item Source Code
\item More Files
\end{itemize}

The individual parts are described in more detail below.

\subsection{Configuration of Packages}

The user's changes to the package's configuration are made in the
file \texttt{config/<PACKAGE>.txt}. All the OPT's variables should begin with
the name of the OPT, for example:

\begin{example}
\begin{verbatim}
    #-------------------------------------------------------------------
    # Optional package: TELNETD
    #-------------------------------------------------------------------
    OPT_TELNETD='no'        # install telnetd: yes or no
    TELNETD_PORT='23'       # telnet port, see also FIREWALL_DENY_PORT_x
\end{verbatim}
\end{example}

An OPT should be prefixed by a header in the configuration file (see
above). This increases readability, especially as a package indeed can
contain multiple OPTs. Variables associated to the OPT should~--- again
in the interest of readability~--- not be indented further. Comments and
blank lines are allowed, with comments always starting in column 33.
If a variable including its content has more than 32 characters, the
comment should be inserted with a row offset, starting in column 33.
Longer comments are spread over multiple lines, each starting at column
33. All this increases easy review of the configuration file.

All values following the equal sign must be enclosed in quotes\footnote{Both
single and double quotes are valid. You can hence write
\texttt{FOO='bar'} as well as \texttt{FOO="bar"}. The use of double quotes should be an
exception and you should previously inform about how an *nix shell uses single and
double quotes.}
not doing so can lead to problems during system boot.

    Activated variables (see below), will be transferred to \texttt{rc.cfg},
    everything else will be ignored. The only exceptions are variables by
    the name of \var{<PACKAGE>\_DO\_DEBUG}. These are used for debugging and
    are transferred as is.

\marklabel{sec:opt_txt}{
  \subsection{List of Files to Copy}
}

    The file \texttt{opt/<PACKAGE>.txt} contains instructions that describe
\begin{itemize}
\item which files are owned by the OPT,
\item the preconditions for inclusion in the \texttt{opt.img} resp. \texttt{rootfs.img},
\item what User ID (uid), Group ID (gid) and rights will be applied to files,
\item which conversions have to be made before inclusion in the archive.
\end{itemize}

Based on this information \var{mkfli4l} will generate the archives needed.

Blank lines and lines beginning with ``\#'' are ignored. In one of the first lines
the version of the package file format should be noted as follows:

\begin{example}
\begin{verbatim}
    <first column>      <second column> <third column>
    opt_format_version  1                    -
\end{verbatim}
\end{example}

    The remaining lines have the following syntax:

\begin{example}
\begin{verbatim}
    <first column>  <second column> <third column> <columns following>
    Variable        Value            File           Options
\end{verbatim}
\end{example}

    \begin{enumerate}
    \item
        The first column contains the name of a variable which triggers
        inclusion of the file referenced in the third column depending on
        its value in the package's config file. The name of a variable may
        appear in the first column as often as needed if multiple files depend
        on it. Any variable that appears in the file \texttt{opt/<PACKAGE>.txt}
        is marked by \var{mkfli4l}.

        If multiple variables should be tested for the same value a list of
        variables (separated by commas) can be used instead. It is sufficient
        in this case if at least \emph{one} variable contains the value required
        in the second column. It is important \emph{not} to use spaces between
        the individual variables!

        In OPT variables (ie variables that begin with \var{OPT\_} and typically have
        the type \var{YESNO}), the prefix ``\var{OPT\_}'' can be omitted. It does not
        matter whether variables are noted in upper- or lowercase (or mixed).

      \item The second column contains a value. If the variable in the first column
	is identical with this value and is activated too (see below), the file referenced
	in the third column will be included. If the first column contains a \%-variable
        it will be iterated over all indices and checked whether the respective
        variable matches the value. If this is the case copying will be executed.
        In addition, the copy process based on the current value of the variable
        will be logged.

        It is possible to write a ``!'' in front of the value. In this case,
        the test is negated, meaning the file is only copied if the variable
        does \emph{not} contain the value.

      \item  In the third column a file name is referenced. The path must be given
	relative to the \texttt{opt} directory. The file must exist and be readable,
	otherwise an error is raised while generating the boot medium and the build process
	is aborted.

        If the file name is prefixed with a ``\texttt{rootfs:}'' the file is included in the
        list of files to be copied to the RootFS. The prefix will be stripped before.

        If the file is located below the current configuration directory it is added
        to the list of files to be copied from there, otherwise the file found below
        opt is taken. Those files are not allowed to have a \texttt{rootfs:} prefix.

        If the file to copy is a kernel module the actual kernel version may be substituted
        by \var{\$\{KERNEL\_VERSION\}}. \var{mkfli4l} will then pick the version from the configuration
        and place it there. Using this you may provide modules for several kernel versions
        for the package and the module matching the current kernel version will be copied to
        the router.
        For kernel modules the path may be omitted, \var{mkfli4l} will find the module
        using \texttt{modules.dep} and \texttt{modules.alias}, see the section
        \jump{subsec:automatic-dependencies}{``Automatically Resolving Dependencies for Kernel Modules''}.

        \begin{table}[ht!]
          \centering
          \small
          \caption{Options for Files}
          \label{table:options}
          \begin{tabular}{|p{2.5cm}|p{7.5cm}|p{3.5cm}|}
            \hline
            Option & Meaning & Default Value \\
            \hline
            type= & Type of the Entry:\newline\newline
            \begin{tabular}{ll}
            \emph{local} & Filesystem Object\\
            \emph{file} & File\\
            \emph{dir} & Directory\\
            \emph{node} & Device\\
            \emph{symlink} & (symbolic) Link
            \end{tabular}\newline\newline
            This option has to be placed in front when given. The type
            ``local'' represents the type of an object existing in the file system
            and hence matches ``file'', ``dir'', ``node'' or ``symlink'' (depends).
            All other types except for ``file'' can create entries in the archive
            that do not have to exist in the local file system. This can i.e. be used
            to create devices files in the RootFS archive. & local \\
            uid= & The file owner, either numeric or as a name from passwd & root \\
            gid= & File group, either numeric or as a name from group & root \\
            mode= & Access rights &
            Files and Devices:\newline\verb?rw-r--r--? (644)\newline
            Directories:\newline\verb?rwxr-xr-x? (755)\newline
            Links:\newline
            \verb?rwxrwxrwx? (777)\newline\\
            flags=\newline
            (type=file) & Conversions before inclusion in the archive:\newline\newline
            \begin{tabular}{lp{6cm}}
            \emph{utxt} & Conversion to *nix format\\
            \emph{dtxt} & Conversion to DOS format\\
            \emph{sh}   & Shell script: Conversion to *nix format, stripping of superfluous chars\\
            \emph{luac} & Lua script: Translation into byte code of the Lua VM
            \end{tabular}
            & \\
            name= & Alternative name for inclusion of the entry in the archive &  \\
            devtype=\newline
            (type=node) & Descibes the type of the device (``c'' for character and ``b'' für block
            oriented devices). Has to be placed in second position. & \\
            major=\newline
            (type=node) & Decribes the so-called
            ``Major'' number of the device file. Has to be placed in third position. & \\
            minor=\newline
            (type=node) & Decribes the so-called
            ``Minor'' number of the device file. Has to be placed in fourth position. & \\
            linktarget=\newline
            (type=symlink) & Describes the target of the
            symbolic link. Has to be placed in second position. & \\
            \hline

          \end{tabular}
        \end{table}

      \item the other columns may contain the options for owner, group, rights for files and conversion
	listed in table~\ref{table:options}.

    \end{enumerate}

    Some examples:
    \begin{itemize}
    \item copy file if \verb+OPT_TELNETD='yes'+, set its
uid/gid to root and the rights to 755 (\verb?rwxr-xr-x?)

\begin{example}
\begin{verbatim}
    telnetd     yes    usr/sbin/in.telnetd mode=755
\end{verbatim}
\end{example}

    \item copy file, set
uid/gid to root, the rights to 555 (\verb?r-xr-xr-x?) and convert the
file to *nix format while stripping all superfluous chars

\begin{example}
\begin{verbatim}
    base    yes     etc/rc0.d/rc500.killall mode=555 flags=sh
\end{verbatim}
\end{example}

         \item copy file if \verb+PCMCIA_PCIC='i82365'+, set
uid/gid to root and the rights to 644 (\verb?rw-r--r--?)

\begin{example}
\begin{verbatim}
    pcmcia_pcic i82365 lib/modules/${KERNEL_VERSION}/pcmcia/i82365.ko
\end{verbatim}
\end{example}

         \item copy file if one of the \var{NET\_DRV\_\%} variables matches the second field,
set uid/gid to root and the rights to 644 (\verb?rw-r--r--?)

\begin{example}
\begin{verbatim}
    net_drv_%   3c503  3c503.ko
\end{verbatim}
\end{example}

        \item copy file if the variable \var{POWERMANAGEMENT} does \emph{not}
        contain the value ``none'':

        \begin{example}
\begin{verbatim}
    powermanagement !none etc/rc.d/rc100.pm mode=555 flags=sh
\end{verbatim}
\end{example}

        \item copy file if any of the OPT variables \var{OPT\_MYOPTA}
        or \var{OPT\_MYOPTB} contains the value ``yes'':

\begin{example}
\begin{verbatim}
    myopta,myoptb yes usr/local/bin/myopt-common.sh mode=555 flags=sh
\end{verbatim}
\end{example}

        This example is only an abbreviation for:

\begin{example}
\begin{verbatim}
    myopta yes usr/local/bin/myopt-common.sh mode=555 flags=sh
    myoptb yes usr/local/bin/myopt-common.sh mode=555 flags=sh
\end{verbatim}
\end{example}

        And the latter is a shorthand notation for:

\begin{example}
\begin{verbatim}
    opt_myopta yes usr/local/bin/myopt-common.sh mode=555 flags=sh
    opt_myoptb yes usr/local/bin/myopt-common.sh mode=555 flags=sh
\end{verbatim}
\end{example}

        \item copy file \texttt{opt/usr/bin/beep.sh} to the RootFS archive,
        but rename it to \texttt{bin/beep} before:

\begin{example}
\begin{verbatim}
    base yes rootfs:usr/bin/beep.sh mode=555 flags=sh name=bin/beep
\end{verbatim}
\end{example}

    \end{itemize}

    The files will be copied only if the above conditions are met and
    \verb+OPT_PACKAGE='yes'+ of the corresponding package is set. What OPT variable is
    referenced is decribed in the file \texttt{check/<PACKAGE>.txt}.

    If a variable is referenced in a package that is not defined by the package
    itself, it may happen that the corresponding package is not installed. This
    would result in an error message from \var{mkfli4l}, as it expects that all of
    the variables referenced by \texttt{opt/<PACKAGE>.txt} are defined.

    To handle this situation correctly the ``weak'' declaration has been introduced.
    It has the following format:

\begin{example}
\begin{verbatim}
    weak        <Variable>    -
\end{verbatim}
\end{example}

    By this the variable it is defined (if not already existing) and its value is
    set to ``undefined''. Please note that the prefix ``\var{OPT\_}" \emph{must} be
    provided (if existing) because else a variable \emph{without} this prefix will
    be created.

    An example taken from \texttt{opt/rrdtool.txt}:
\begin{example}
\begin{verbatim}
    weak opt_openvpn -
    [...]
    openvpn    yes    usr/lib/collectd/openvpn.so
\end{verbatim}
\end{example}

Without the \texttt{weak} definition \var{mkfli4l} would display an error message
when using the package ``rrdtool'' while the ``openvpn'' package is not activated.
By using the \texttt{weak} definition no error message is raised even in the case that
the ``openvpn'' package does not exist.

\marklabel{subsec:konfigspezdatei}{
\subsubsection{Files adapted by Configuration}
}

In some situations it is desired to replace original files with configuration-specific
files for inclusion in the archive, i.e. host keys, own firewall scripts, \ldots{}
\var{mkfli4l} supports this scenario by checking whether a file can be found in the
configuration directory and, if so, including this one instead in the file list for
\texttt{opt.img} resp.\ \texttt{rootfs.img}.

Another option to add configuration-specific files to an archive is decribed
in the section \jump{subsec:addtoopt}{Extended Checks of the Configuration}.


\marklabel{subsec:automatic-dependencies}{
\subsubsection{Automatically Resolving Dependencies for Kernel Modules}}
Kernel modules may depend on other kernel modules. Those must be loaded before
and therefore also have to be added to the archive. \var{mkfli4l} resolves this
dependencies based on \texttt{modules.dep} and \texttt{modules.alias}
(two files generated during the kernel build), automatically including
all required modules in the archives. Thus, for example the following
entry

\begin{example}
\begin{verbatim}
    net_drv_%   ne2k-pci    ne2k-pci.ko
\end{verbatim}
\end{example}

triggers that both 8390.ko and crc32.ko are included in the archive
because ne2k\_pci depends on both of them.

The necessary entries from \texttt{modules.dep} and \texttt{modules.alias}
are included in the RootFS and can be used by \texttt{modprobe} for
loading the drivers.

\marklabel{subsec:dev:var-check}{
\subsection{Checking Configuration Variables}
}

By the help of \texttt{check/<PACKAGE>.txt} the content of variables can be
checked for validity. In earlier version of the program \var{mkfli4l} this
check was hard coded there but it was outsourced to the check files in
the course of modularizing fli4l. This file contains a line for each
variable in the config files. These lines consist of four to five columns
which have the following functions:

\begin{enumerate}

\item Variable: this column specifies the name of the configuration file
  variable to check. If this is an \emph{array variable}, it can appear
  multiple times with different indices, so instead of the index number a percent
  sign (\%) is added to the variable name. It is always used as ``\var{\_\%\_}''
  in the middle of a name resp.\ ``\var{\_\%}'' at the end of a name. The name
  may contain more than one percent sign allowing the use of multidimensional
  arrays. It is recommended (but not mandatory) to add some text between
  the percent signs to avoid weird names like ``\var{FOO\_\%\_\_\%}''.

  Often the problem occurs that certain variables describe options that are
  needed only in some situations. Therefore variables may be marked as optional.
  Optional variables are identified by the prefix ``+''. They may then exist,
  but do not have to. Arrays can also use a ``++'' prefix. Prefixed with a ``+''
  the array can exist or be entirely absent.  Prefixed with a``++'' in addition
  some elements of the array may be missing.

\item \var{OPT\_\-VARIABLE}: This column assigns the variable to a specific OPT.
  The variable is checked for validity only if the OPT variable is set to ``yes''.
  If there is no OPT variable a ``-'' indicates this. In this case,
  the variable must be defined in the configuration file, unless a default
  value is defined (see below). The name of the OPT variable may be arbitrary
  but should start with the prefix ``\var{OPT\_}''.

  If a variable does not depend on any OPT variables, it is considered
  \emph{active}. If it is depending on an OPT variable, it is precisely
  active if

  \begin{itemize}
  \item its OPT variable is active and
  \item its OPT variable contains the value ``yes''.
  \end{itemize}

  In all other cases the variable is inactive.

  \textbf{Hint:} Inactive OPT variables will be set back to ``no'' by
  \var{mkfli4l} if set to ``yes'' in the configuration file, an appropriate
  warning will be generated then (i.e. \verb+OPT_Y='yes'+ is ignored, because
  \verb+OPT_X='no'+). For transitive dependency chains (\var{OPT\_Z} depends
  on \var{OPT\_Y} which in turn depends on \var{OPT\_X}) this will only work
  reliable, if the names of all OPT-variables start with ``\var{OPT\_}''.

\item \var{VARIABLE\_\-N}: If the first column contains a variable with a
  ``\%'' in its name, it indicates the number of occurrences of the variable
  (the so-called \emph{N-variable}). In case of a multi-dimensional variable,
  the occurences of the last index are specified. If the variable depends on a
  certain OPT, the N-variable must be dependant on the same or no OPT. If the
  variable does not depend on any OPT, the N-variable also shouldn't.
  If no N-variable exists, specify ``-'' to indicate that.

  For compatibility with future versions of fli4l the variable specified here
  \emph {must} be identical with the variable in \var{OPT\_VARIABLE} where the last
  ``\%'' is replaced by an ``N'' and everything following is removed.
  An array \var{HOST\_\%\_IP4} must have the N-Variable \var{HOST\_N} assigned
  and an array  \var{PF\_USR\_CHAIN\_\%\_RULE\_\%} hence the N-variable
  \var{PF\_USR\_CHAIN\_\%\_RULE\_N}, and this N-variable itself is an
  array variable with the corresponding N-variable \var{PF\_USR\_CHAIN\_N}.
  \emph{All other namings of the N variables will be incompatible with future
  versions of fli4l!}

\item \var{VALUE}: This column provides the values a variable can hold.
  For example the following settings are possible:

  \begin{tabular}[ht!]{|l|l|}
    \hline
    Name & Meaning \\
    \hline
    \hline
    \var{NONE}     &  No error checking will be done\\
    \var{YESNO}    &  The variable must be  ``yes'' or ``no''\\
    \var{NOTEMPTY} &  The variable can't be empty\\
    \var{NOBLANK}  &  The variable can't contain spaces\\
    \var{NUMERIC}  &  The variable must be numeric\\
    \var{IPADDR}   &  The variable must be an IP address\\
    \var{DIALMODE} &  The variable must be ``on'', ``off'' or ``auto''\\
    \hline
  \end{tabular}
  \\

  I values are prefixed by ``\var{WARN\_}'' an illegal content will not raise
  an error message and abort the build by \var{mkfli4l}, but only display a warning.

  The possible checks are defined by regular expressions in \texttt{check/base.exp}.
  This file may be extended and now contains some new checking routines, for example:
  \var{HEX}, \var{NUMHEX}, \var{IP\_ROUTE}, \var{DISK} and \var{PARTITION}.

  The number of expressions may be extended at any time for the future needs of
  package developers. Provide feedback!

  In addition, regular expressions can also be directly defined in the check-files,
  even relations to existing expressions can be made. Instead of \var{YESNO} you
  could, for example also write
\begin{example}
\begin{verbatim}
    RE:yes|no.
\end{verbatim}
\end{example}
This is useful if a test is performed only once and is relatively easy. For
more details see the next chapter.

\item Default Setting: In this column, an optional default value for the variables
can be defined in the case that the variable is not specified in the configuration file.

\textbf{Hint:} At the moment this does not work for array variables. Additionally,
the variable can't be optional (no ``+'' in front of the variable name).

Example:
\begin{example}
\begin{verbatim}
    OPT_TELNETD     -      -      YESNO    "no"
\end{verbatim}
\end{example}

If \var{OPT\_TELNETD} is missing in the config file, ``no'' will be assumed
and written as a value to \texttt{rc.cfg}.

\end{enumerate}

    The percent sign thingie is best decribed with an example. Let's assume
    \texttt{check/base.txt} amongst others has the following content:
\begin{example}
\begin{verbatim}
    NET_DRV_N          -                  -                  NUMERIC
    NET_DRV_%          -                  NET_DRV_N          NONE
    NET_DRV_%_OPTION   -                  NET_DRV_N          NONE
\end{verbatim}
\end{example}

      This means that depending on the value of \var{NET\_\-DRV\_\-N} the variables \var{NET\_\-DRV\_\-N},
      \var{NET\_\-DRV\_\-1\_\-OPTION}, \var{NET\_\-DRV\_\-2\_\-OPTION}, \var{NET\_\-DRV\_\-3\_\-OPTION}, a.s.o.
      will be checked.

\subsection{Own Definitions for Checking the Configuration Variables}

\subsubsection{Introduction of Regular Expressions}

  In version 2.0 only the above mentioned value ranges for variable checks existed:
  \var{NONE}, \var{NOTEMPTY}, \var{NUMERIC}, \var{IPADDR}, \var{YESNO}, \var{NOBLANK},
  \var{DIALMODE}. Checking was hard-coded to \var{mkfli4l}, not expandable and
  restricted to essential ``data types'' which could be evaluated with reasonable
  efforts.

  As of version 2.1 this checking has been reimplemented. The aim of the
  new implementation is a more flexible testing of variables, that is also
  able to examine more complex expressions. Therefore, regular expressions
  are used that can be stored in one or more separate files. This on one hand
  makes it possible to examine variables that are not checked for the moment and
  on the other hand, developers of optional packages can now define own terms in
  order to check the configuration of their packages.

  A description of regular expressions can be found via ``man 7 regex''
  or i.e. here:\\ \altlink{http://unixhelp.ed.ac.uk/CGI/man-cgi?regex+7}.

\subsubsection{Specification of Regular Expressions}

  Specification of regular expressions can be accomplished in two ways:

  \begin{enumerate}
  \item Package specific exp files \texttt{check/<PACKAGE>.exp}

    This file can be found in the \texttt{check} directory and has the same name
    as the package containing it, i.e. \texttt{check/base.exp}. It contains
    definitions for expressions that can be referenced in the file \texttt{check/<PACKAGE>.txt}.
    \texttt{check/base.exp} for example at the moment contains definitions for the
    known tests and \texttt{check/isdn.exp} a definition for the variable
    \var{ISDN\_\-CIRC\_\-x\_ROUTE} (the absence of this check was the trigger
    for the changes).

The syntax is as follows (again, double quotes can be used if needed):
\begin{example}
\begin{verbatim}
    <Name> = '<Regular Expression>' : '<Error Message>'
\end{verbatim}
\end{example}
as an example \texttt{check/base.exp}:
\begin{example}
\begin{verbatim}
    NOTEMPTY = '.*[^ ]+.*'          : 'should not be empty'
    YESNO    = 'yes|no'             : 'only yes or no are allowed'
    NUMERIC  = '0|[1-9][0-9]*'      : 'should be numeric (decimal)'
    OCTET    = '1?[0-9]?[0-9]|2[0-4][0-9]|25[0-5]'
             : 'should be a value between 0 and 255'
    IPADDR   = '((RE:OCTET)\.){3}(RE:OCTET)' : 'invalid ipv4 address'
    EIPADDR  = '()|(RE:IPADDR)'
             : 'should be empty or contain a valid ipv4 address'
    NOBLANK  = '[^ ]+'              : 'should not contain spaces'
    DIALMODE = 'auto|manual|off'    : 'only auto, manual or off are allowed'
    NETWORKS = '(RE:NETWORK)([[:space:]]+(RE:NETWORK))*'
             : 'no valid network specification, should be one or more
                network address(es) followed by a CIDR netmask,
                for instance 192.168.6.0/24'
\end{verbatim}
\end{example}

The regular expressions can also include already existing definitions by a reference.
These are then pasted to substitute the reference. This makes it easier to construct regular
expressions. The references are inserted by '(RE: Reference)'. (See the definition of the term
\var{NETWORKS} above for an appropriate example.)

The error messages tend to be too long. Therefore, they may be displayed on multiple lines.
The lines afterwards always have to start with a space or tab then. When reading the file
\texttt{check/<PACKAGE>.exp} superfluous whitespaces are reduced to one and tabs are replaced
by spaces. An entry in \texttt{check/<PACKAGE>.exp} could look like this:

\begin{example}
\begin{verbatim}
    NUM_HEX         = '0x[[:xdigit:]]+'
                    : 'should be a hexadecimal number
                       (a number starting with "0x")'
\end{verbatim}
\end{example}

\item  Regular expressions directly in the check file \texttt{check/<PACKAGE>.txt}

Some expressions occur but once and are not worth defining a regular expression in a
\texttt{check/<PACKAGE>.exp} file. You can simply write this expression to the check
file for example:

\begin{example}
\begin{verbatim}
    # Variable      OPT_VARIABLE    VARIABLE_N     VALUE
    MOUNT_BOOT      -               -              RE:ro|rw|no
\end{verbatim}
\end{example}

\var{MOUNT\_\-BOOT} can only take the value ``ro'', ``rw'' or ``no'',
everything else will be denied.

If you want to refer to existing regular expressions, simply add
a reference via `(RE:...)''. Example:

\begin{example}
\begin{verbatim}
    # Variable      OPT_VARIABLE    VARIABLE_N     VALUE
    LOGIP_LOGDIR    OPT_LOGIP       -              RE:(RE:ABS_PATH)|auto
\end{verbatim}
\end{example}

\end{enumerate}


\subsubsection{Expansion of Existing Regular Expressions}

If an optional package adds an additional value for a variable
which will be examined by a regular expression, then the regular
expression has to be expanded. This is done simply by defining the
new possible values by a regular expression (as described above) and
complement the existing regular expression in a separate \texttt{check/<PACKAGE>.exp}
file. That an existing expression is modified is indicated by a leading ``+''.
The new expression complements the existing expression by appending the new value
to the existing value(s) as an alternative. If another expression makes use of the
complemented expression, the supplement is also there. The specified error message
is simply appended to the end of the existing one.

Using the Ethernet driver as an example this could look like here:

\begin{itemize}
\item The base packages provides a lot of Ethernet drivers and checks
  the variable \var{NET\_DRV\_x} using the regular expression \var{NET\_DRV},
  which is defined as follows:

\begin{example}
\begin{verbatim}
    NET_DRV         = '3c503|3c505|3c507|...'
                    : 'invalid ethernet driver, please choose one'
                      ' of the drivers in config/base.txt'
\end{verbatim}
\end{example}
\item The package ``pcmcia'' provides additional device drivers,
  and hence has to complement \var{NET\_DRV}. This is done as follows:

\begin{example}
\begin{verbatim}
    PCMCIA_NET_DRV = 'pcnet_cs|xirc2ps_cs|3c574_cs|...' : ''
    +NET_DRV       = '(RE:PCMCIA_NET_DRV)' : ''
\end{verbatim}
\end{example}
\end{itemize}

Now PCMCIA drivers can be chosen in addition.

\subsubsection{Extend Regular Expressions in Relation to \var{YESNO} Variables}

If you have extended \var{NET\_DRV} with the PCMCIA drivers as shown above, but
the package ``pcmcia'' has been deactivated, you still could select a PCMCIA driver
in \texttt{config/base.txt} without an error message generated when creating the archives.
To prevent this, you may let the regular expression depend on a \var{YESNO} variable in the
configuration. For this purpose, the name of the variable that determines whether the
expression is extended is added with brackets immediately after the name of the expression.
If the variable is active and has the value ``yes'', the term is extended,
otherwise not.

\begin{example}
\begin{verbatim}
    PCMCIA_NET_DRV       = 'pcnet_cs|xirc2ps_cs|3c574_cs|...' : ''
    +NET_DRV(OPT_PCMCIA) = '(RE:PCMCIA_NET_DRV)' : ''
\end{verbatim}
\end{example}

If specifying \verb+OPT_PCMCIA='no'+  and using i.e. the PCMCIA driver
\texttt{xirc2ps\_cs} in\\ \texttt{config/base.txt}, an error message will
be generated during archive build.\\

\textbf{Hint:} This does \emph{not} work if the variable is not set
explicitely in the configuration file but gets its value by a default
setting in \texttt{check/<PACKAGE>.txt}. In this case the variable hence has
to be set explicitely and the default setting has to be avoided if necessary.

\marklabel{sec:regexp-dependencies}{
  \subsubsection{Extending Regular Expressions Depending on other Variables}
}

Alternatively, you may also use arbitrary values of variables as conditions,
the syntax looks like this:

\begin{example}
\begin{verbatim}
    +NET_DRV(KERNEL_VERSION=~'^3\.18\..*$') = ...
\end{verbatim}
\end{example}

If \var{KERNEL\_VERSION} matches the given regular expression (if any of the
kernels of the 3.18 line is used) then the list of network driver allowed is
extended with the drivers mentioned.\\

\textbf{Hint:} This does \emph{not} work if the variable is not set
explicitely in the configuration file but gets its value by a default
setting in \texttt{check/<PACKAGE>.txt}. In this case the variable hence has
to be set explicitely and the default setting has to be avoided if necessary.

\subsubsection{Error Messages}

If the checking process detects an error, an error message of the following kind
is displayed:

\begin{example}
\begin{verbatim}
    Error: wrong value of variable HOSTNAME: '' (may not be empty)
    Error: wrong value of variable MOUNT_OPT: 'rx' (user supplied regular expression)
\end{verbatim}
\end{example}

For the first error, the term was defined in a \texttt{check/<PACKAGE>.exp} file and
an explanation of the error is displayed. In the second case the term was
specified directly in a \texttt{check/<PACKAGE>.txt} file, so there is no
additional explanation of the error cause.

\subsubsection{Definition of Regular Expressions}

Regular expressions are defined as follows:

Regular expression: One or more alternatives, separated by
'$|$', i.e. ``ro$|$rw$|$no''. If one option matches, the whole term
matches (in this case ``ro'', ``rw'' and ``no'' are valid expressions).

An alternative is a concatenation of several sections that are simply added.

A section is an ``atom'', followed by a single ``*'', ``+'',
``?'' or ``\{min, max\}''. The meaning is as follows:
\begin{itemize}
\item ``a*''~--- as many ``a''s as wished (allows also no ``a'' is existing at all)
\item   ``a+''~--- at least one``a''
\item   ``a?''~--- none or one ``a''
\item   ``a\{2,5\}''~--- two to five ``a''s
\item   ``a\{5\}''~--- exactly five ``a''s
\item   ``a\{2,\}''~--- at least two ``a''s
\item   ``a\{,5\}''~--- a maximum of five ``a''s
\end{itemize}

An ``atom'' is a
\begin{itemize}
\item  regular expression enclosed in brackets, for example ``(a$|$b)+''
          matches any string containing at least one ``a'' or ``b'', up to an arbitrary
          number and in any order
        \item   an empty pair of brackets stands for an ``empty'' expression
        \item   an expression in square brackets ``[\,]'' (see below)
        \item a dot ``.'', matching an arbitrary character, for example
          a ``.+'' matches any string containing at least one char
        \item a ``\^\,'' represents the beginning of a line, for example a ``\^\,a.*''
          matches a string beginning with an ``a'' followed by any char like in
          ``a'' or ``adkadhashdkash''
        \item a ``\$'' represents the end of a line
        \item a ``$\backslash$'' followed by one of the special characters
          \texttt{\^\,.\,[\,\$\,(\,)\,$|$\,*\,+\,?\,\{\,$\backslash$} stands for the second char
          without its special meaning
        \item  a normal char matches exactly this char, for example
          ``a'' matches exactly an ``a''.
\end{itemize}

An expression in square brackets indicates the following:
\begin{itemize}
\item ``x-y''~--- matches any char inbetween ``x'' and ``y'', for example ``[0-9]''
                  matches all chars between ``0'' and ``9''; ``[a-zA-Z]'' symbolizes all chars,
                  either upper- or lowercase.

                \item ``\^\,x-y''~--- matches any char \emph{not} contained in the given interval,
                  for example ``[\^\,0-9]'' matches all chars \emph{except} for digits.

                \item ``[:\emph{character-class}:]''~--- matches a char from \emph{character-class}.
                Relevant standard character classes are: \texttt{alnum}, \texttt{alpha}, \texttt{blank},
                \texttt{digit}, \texttt{lower}, \texttt{print}, \texttt{punct}, \texttt{space}, \texttt{upper}
                and \texttt{xdigit}. I.e. ``[\,[:alpha:]\,]'' stands for all upper- or lowercase chars and hence
                is identical with ``[\,[:lower:]\,[:upper:]\,]''.
\end{itemize}


\subsubsection{Examples for regular Expressions}

Let's have a look at some examples!

\var{NUMERIC}: A numeric value consists of at least one, but otherwise
any number of digits. ``At least one'' is expressed with a ``+'', one digit
was already in an example above. So this results in:

\begin{example}
\begin{verbatim}
    NUMERIC = '[0-9]+'
\end{verbatim}
\end{example}
or alternatively
\begin{example}
\begin{verbatim}
    NUMERIC = '[[:digit:]]+'
\end{verbatim}
\end{example}

\var{NOBLANK}: A value that does not contain spaces, is any
char (except for the char ``space'') and any number of them:

\begin{example}
\begin{verbatim}
    NOBLANK = '[^ ]*'
\end{verbatim}
\end{example}

or, if the value is not allowed to be empty:

\begin{example}
\begin{verbatim}
    NOBLANK = '[^ ]+'
\end{verbatim}
\end{example}

\var{IPADDR}: Let's have a look at an example with an IP4-address. An
ipv4 address consists of four ``Octets'', divided by dots (``.''). An
octet is a number between 0 and 255. Let's define an octet at first.
It may be\\

\begin{tabular}[ht!]{lr}
  a number between 0 and 9: &       [0-9]\\
  a number between 10 and 99: &     [1-9][0-9]\\
  a number between 100 and 199:&   1[0-9][0-9]\\
  a number between 200 and 249: &  2[0-4][0-9]\\
  a number between 250 and 255: & 25[0-5]\\
\end{tabular}\\

All are alternatives hence we concatenate them with ``$|$'' forming
one expression: ``[0-9]$|$[1-9][0-9]$|$1[0-9][0-9]$|$2[0-4][0-9]$|$25[0-5]'' and
get an octet. Now we compose an IP4 address, four octets divided by dots (the dot
must be masked with a \emph{backslash}, because else it would represent an arbitrary
char). Based on the syntax of an exp-file it would look like this:

\begin{example}
\begin{verbatim}
    OCTET  = '[0-9]|[1-9][0-9]|1[0-9][0-9]|2[0-4][0-9]|25[0-5]'
    IPADDR = '((RE:OCTET)\.){3}(RE:OCTET)'
\end{verbatim}
\end{example}


\subsubsection{Assistance for the Design of Regular Expressions}

If you want to design and test regular expressions, you can use the
``regexp'' program located in the \texttt{unix} or \texttt{windows}
directory of the package ``base''. It accepts the following
syntax:

\begin{example}
\begin{verbatim}
    usage: regexp [-c <check dir>] <regexp> <string>
\end{verbatim}
\end{example}

The parameters explained in short:
\begin{itemize}
\item \texttt{<check dir>} is the directory containing check and exp files.
These are read by ``regexp'' to use expressions already defined there.

\item \texttt{<regexp>} is the regular expression (enclosed in \verb+'...'+
or \verb+"..."+ if in doubt, with double quotes needed only if single
quotes are used in the expression itself)


\item \texttt{<string>} is the string to be checked
\end{itemize}

This may for example look like here:
\begin{example}
\begin{verbatim}
./i586-linux-regexp -c ../check '[0-9]' 0
adding user defined regular expression='[0-9]' ('^([0-9])$')
checking '0' against regexp '[0-9]' ('^([0-9])$')
'[0-9]' matches '0'

./i586-linux-regexp -c ../check '[0-9]' a
adding user defined regular expression='[0-9]' ('^([0-9])$')
checking 'a' against regexp '[0-9]' ('^([0-9])$')
regex error 1 (No match) for value 'a' and regexp '[0-9]' ('^([0-9])$')

./i586-linux-regexp -c ../check IPADDR 192.168.0.1
using predefined regular expression from base.exp
adding IPADDR='((RE:OCTET)\.){3}(RE:OCTET)'
 ('^(((1?[0-9]?[0-9]|2[0-4][0-9]|25[0-5])\.){3}(1?[0-9]?[0-9]|2[0-4][0-9]|25[0-5]))$')
'IPADDR' matches '192.168.0.1'

./i586-linux-regexp -c ../check IPADDR 192.168.0.256
using predefined regular expression from base.exp
adding IPADDR='((RE:OCTET)\.){3}(RE:OCTET)'
 ('^(((1?[0-9]?[0-9]|2[0-4][0-9]|25[0-5])\.){3}(1?[0-9]?[0-9]|2[0-4][0-9]|25[0-5]))$')
regex error 1 (No match) for value '192.168.0.256' and regexp
 '((RE:OCTET)\.){3}(RE:OCTET)'
(unknown:-1) wrong value of variable cmd_var: '192.168.0.256' (invalid ipv4 address)
\end{verbatim}
\end{example}


\subsection{Extended Checks of the Configuration}

    Sometimes it is necessary to perform more complex checks.
    Examples of such complex things would be i.e. dependencies
    between packages or conditions that must be satisfied only
    when variables take certain values. For example if a PCMCIA
    ISDN adapter is used the package ``pcmcia'' has to be installed, too.

    In order to perform these checks you may write small tests to
    \texttt{check/<PACKAGE>.ext} (also called ext-script). The language
    consists of the following elements:

    \begin{enumerate}
    \item Keywords:

      \begin{itemize}
      \item Control Flow:

        \begin{itemize}
        \item \texttt{if (\textit{expr}) then \textit{statement} else \textit{statement} fi}
        \item \texttt{foreach \textit{var} in \textit{set\_var} do \textit{statement} done}
        \item \texttt{foreach \textit{var} in \textit{set\_var\_1 ... set\_var\_n} do \textit{statement} done}
        \item \texttt{foreach \textit{var} in \textit{var\_n} do \textit{statement} done}
        \end{itemize}

      \item
        Dependencies:
        \begin{itemize}
        \item \texttt{provides \textit{package} version \textit{x.y.z}}
        \item \texttt{depends on \textit{package} version \textit{x1.y1 x2.y2.z2 x3.y3 \ldots}}
        \end{itemize}

      \item Actions:
        \begin{itemize}
        \item \texttt{warning "\textit{warning}"}
        \item \texttt{error   "\textit{error}"}
        \item \texttt{fatal\_error "\textit{fatal error}"}
        \item \texttt{set \textit{var} = \textit{value}}
        \item \texttt{crypt (\textit{variable})}
        \item \texttt{stat (\textit{filename}, \textit{res})}
        \item \texttt{fgrep (\textit{filename}, \textit{regex})}
        \item \texttt{split (\textit{string}, \textit{set\_variable}, \textit{character})}
        \end{itemize}
      \end{itemize}
    \item Data Types:      strings, positive integers, version numbers
    \item Logical Operations:    \texttt{<}, \texttt{==}, \texttt{>}, \texttt{!=}, \texttt{!}, \texttt{\&\&}, \texttt{||},
      \texttt{=}\verb+~+, \texttt{copy\_pending}, \texttt{samenet}, \texttt{subnet}
    \end{enumerate}

\marklabel{subsec:dev:data-types}{
\subsubsection{Data Types}
}

    Concerning data types please note that variables, based on the associated
    regular expression are permanently assigned to a data type:

\begin{itemize}
\item Variables, starting with type ``\var{NUM}'' are numeric and
      contain positive integers
\item Variables representing an N-variable for any kind of array are numeric as well
\item all other variables are treated as strings
\end{itemize}

    This means, among other things, that a variable of type \var{ENUMERIC}
    can \emph{not} be used as an index when accessing an array variable, even if
    you have checked at first that it is not empty.
    The following code thus does not work as expected:
\begin{example}
\begin{verbatim}
    # TEST should be a variable of type ENUMERIC
    if (test != "")
    then
        # Error: You can't use a non-numeric ID in a numeric
        #         context. Check type of operand.
        set i=my_array[test]
        # Error: You can't use a non-numeric ID in a numeric
        #         context. Check type of operand.
        set j=test+2
    fi
\end{verbatim}
\end{example}

    A solution for this problem is offered by \jump{subsec:split}{\texttt{split}}:
\begin{example}
\begin{verbatim}
    if (test != "")
    then
        # all elemente of test_% are numeric
        split(test, test_%, ' ', numeric)
        # OK
        set i=my_array[test_%[1]]
        # OK
        set j=test_%[1]+2
    fi
\end{verbatim}
\end{example}

\marklabel{subsec:dev:string-rewrite}{
\subsubsection{Substitution of Strings and Variables}
}

    At various points strings are needed, such as when a
    \jump{subsec:dev:print}{Warning} should be issued. In some cases
    described in this documentation, such a string is scanned for variables.
    If found, these are \emph{replaced} by their contents or other attributes.
    This replacement is called \emph{variable substitution}.

    This will be illustrated by an example. Assume this configuration:

\begin{example}
\begin{verbatim}
    # config/base.txt
    HOSTNAME='fli4l'
    # config/dns_dhcp.txt
    HOST_N='1' # Number of hosts
    HOST_1_NAME='client'
    HOST_1_IP4='192.168.1.1'
\end{verbatim}
\end{example}

    Then the character strings are rewritten as follows, if
    variable substitution is active in this context:

\begin{example}
\begin{verbatim}
    "My router is called $HOSTNAME"
    # --> "My router is called fli4l"
    "HOSTNAME is part of the package %{HOSTNAME}"
    # --> "HOSTNAME is part of the package base"
    "@HOST_N is $HOST_N"
    # --> " # Number of hosts is 1"
\end{verbatim}
\end{example}

    As you can see, there are basically three options for replacement:
    \begin{itemize}
    \item \texttt{\$<Name>} resp.\ \texttt{\$\{<Name>\}}: Replaces the
          variable name with the contents of the variable. This is the most common
          form of replacement. The name must be enclosed in \texttt{\{...\}} if in
          the string it is directly followed by a char that may be a valid part of
          a variable name (a letter, a digit, or an underscore). In all other cases,
          the use of curly brackets is possible, but not mandatory.

    \item \texttt{\%<Name>} resp.\ \texttt{\%\{<Name>\}}: Replaces the variable name
	  with the name of the package in which the variable is defined. This does
	  \emph{not} work with variables assigned in the script via
          \jump{subsec:dev:assignment}{\texttt{set}} or with counting
          variables of a \jump{subsec:dev:control}{\texttt{foreach}-loop}
          since such variables do not have a package and their syntax is different.

    \item \texttt{@<Name>} resp.\ \texttt{@\{<Name>\}}: Replaces the
          variable name with the comment noted in the configuration after the variable.
          Again, this does not make sense for variables defined by the script.
    \end{itemize}

    \textbf{Hint:} Elements of array variables can \emph{not} be integrated into
    strings this way, because there is no possibility to provide an index.

    In general, only \emph{constants} can be used for variable substitution,
    strings that come from a variable remain unchanged. An example will make
    this clear - assume the following configuration:

\begin{example}
\begin{verbatim}
    HOSTNAME='fli4l'
    TEST='${HOSTNAME}'
\end{verbatim}
\end{example}

    This code:

\begin{example}
\begin{verbatim}
    warning "${TEST}"
\end{verbatim}
\end{example}

    leads to the following output:

\begin{example}
\begin{verbatim}
    Warning: ${HOSTNAME}
\end{verbatim}
\end{example}

    It will \emph{not} display:

\begin{example}
\begin{verbatim}
    Warning: fli4l
\end{verbatim}
\end{example}

    In the following sections it will be explicitly noted under which conditions
    strings are subject of variable substitution.

\subsubsection{Definition of a Service with an associated
    Version Number: \texttt{provides}}

    For instance, an OPT may declare that it provides a Printer service or a
    Webserver service. Only one package can provide a certain service.
    This prevents i.e. that two web servers are installed in parallel, which
    is not possible for obvious reasons, since the two servers would both
    register port 80. In addition, the current version of the service is
    provided so that updates can be triggered. The version number consists
    of two or three numbers separated by dots, such as ``4.0'' or ``2.1.23''.

    Services typically originate from OPTs, not from packages. For example
    the package ``tools'' has a number of programs that each have their own
    \texttt{provides} statement defined if activated by \verb+OPT_...='yes'+.

    The syntax is as follows:

\begin{example}
\begin{verbatim}
    provides <Name> version <Version>
\end{verbatim}
\end{example}

    Example from package ``easycron'':

\begin{example}
\begin{verbatim}
    provides cron version 3.10.0
\end{verbatim}
\end{example}

    The version number should be incemented by the OPT-developer in the third
    component, if only functional enhancements have been made and the OPT's
    interface is still. The version number should be increased in the first or
    second component when the interface has changed in any incompatible way (eg.
    due to variable renaming, path changes, missing or renamed utilities, etc.).

\subsubsection{Definition of a Dependency to a Service with a
    specific Version: \texttt{depends}}

    If another service is needed to provide the own function (eg. a web server)
    this dependency to a specific version may be defined here. The version can
    be given with two (i.e. ``2.1'') or three numbers (i.e. ``2.1.11'')
    while the two-number version accepts all versions starting with this
    number and the three-number version only accepting just the specified one.
    A list of version numbers may also be specified if multiple versions of
    the service are compatible with the package.

    The syntax is as follows:

\begin{example}
\begin{verbatim}
    depends on <Name> version <Version>+
\end{verbatim}
\end{example}

    An example: Package ``server'' contains:
\begin{example}
\begin{verbatim}
    provides server version 1.0.1
\end{verbatim}
\end{example}

    A Package ``client'' with the following
    \texttt{depends}-instruction is given:\footnote{Of course only
    one at a time!}

\begin{example}
\begin{verbatim}
    depends on server version 1.0       # OK, '1.0' matches '1.0.1'
    depends on server version 1.0.1     # OK, '1.0.1' matches '1.0.1'
    depends on server version 1.0.2     # Error, '1.0.2' does not match with '1.0.1'
    depends on server version 1.1       # Error, '1.1' does not match with '1.0.1'
    depends on server version 1.0 1.1   # OK, '1.0' matches '1.0.1'
    depends on server version 1.0.2 1.1 # Error, neither '1.0.2' nor '1.1' are matching
                                        # '1.0.1'
\end{verbatim}
\end{example}

\marklabel{subsec:dev:print}{
\subsubsection{Communication with the User: \texttt{warning}, \texttt{error}, \texttt{fatal\_error}}
}

    Using these three functions users may be warned, signalized an
    errors or stop the test immediately. The syntax is as follows:

    \begin{itemize}
    \item \verb+warning "text"+
    \item \verb+error "text"+
    \item \verb+fatal_error "text"+
    \end{itemize}

    All strings passed to these funtions are subject of
    \jump{subsec:dev:string-rewrite}{variable substitution}.

\marklabel{subsec:dev:assignment}{
\subsubsection{Assignments}
}

    If for any reason a temporary variable is required it can be created by
    ``\texttt{set var [= value]}''. \emph{The variable can not be a configuration
    variable!} \footnote{This is a desired restriction: Check scripts are \emph{not}
    able to change the user configuration.} If you omit the ``= value'' part the
    variable is simply set to ``yes'' so it may be tested in an \texttt{if}-statement.
    If an assignment part is given, anything may be specified after the equal sign:
    normal variables, indexed variables, numbers, strings and version numbers.

    Please note that by the assignment also the \emph{type} of the temporary variable
    is defined. If a number is assigned \var{mkfli4l}  ``remembers'' that the variable
    contains a number and later on allows calculations with it. Trying to do
    calculations with variables of other types will fail.\\ Example:

\begin{example}
\begin{verbatim}
    set i=1   # OK, i is a numeric variable
    set j=i+1 # OK, j is a numeric variable and contains the value 2
    set i="1" # OK, i now is a string variable
    set j=i+1 # Error "You can't use a non-numeric ID in a numeric
              #         context. Check type of operand."
              # --> no calculations with strings!
\end{verbatim}
\end{example}

    You may also define temporary arrays (see below). Example:

\begin{example}
\begin{verbatim}
    set prim_%[1]=2
    set prim_%[2]=3
    set prim_%[3]=5
    warning "${prim_n}"
\end{verbatim}
\end{example}

    The number of array elements is kept by \var{mkfli4l} in the variable
    \var{prim\_n}. The code above hence leads to the following output:

\begin{example}
\begin{verbatim}
    Warning: 3
\end{verbatim}
\end{example}

    If the right side of an assignment is a string constant, it is subject of
    \jump{subsec:dev:string-rewrite}{variable substitution} at the time of
    assignment. The following example demonstrates this. The code:

\begin{example}
\begin{verbatim}
    set s="a"
    set v1="$s" # v1="a"
    set s="b"
    set v2="$s" # v2="b"
    if (v1 == v2)
    then
      warning "equal"
    else
      warning "not equal"
    fi
\end{verbatim}
\end{example}

    will output ``not equal'', because the variables \var{v1} and \var{v2}
    replace the content of the variable \var{s} already at the time of assignment.\\

    \textbf{Hint:} A variable set in a script is visible while processing further
    scripts ~-- currently there exists no such thing as local variables. Since the
    order of processing scripts of different packages is not defined, you should
    never rely on any variable having values defined by another package.

\subsubsection{Arrays}

    If you want to access elements of a \%-variable (of an array) you have to use
    the original name of the variable like mentioned in the file \texttt{check/<PACKAGE>.txt}
    and add an index for each ``\%'' sign by using ``[\emph{Index}]''.

    Example: If you want to access the elements of variable
    \var{PF\_USR\_CHAIN\_\%\_RULE\_\%} you need two indices because the
    variable has two ``\%'' signs. All elements may be printed for example
    using the following code (the \texttt{foreach}-loop
    is exlained in \jump{subsec:dev:control}{see below}):

\begin{example}
\begin{verbatim}
    foreach i in pf_usr_chain_n
    do
        # only one index needed, only one '%' in the variable's name
        set j_n=pf_usr_chain_%_rule_n[i]
        # Attention: a
        # foreach j in pf_usr_chain_%_rule_n[i]
        # is not possible, hence the use of j_n!
        foreach j in j_n
        do
            # two indices needed, two '%' in the variable's name
            set rule=pf_usr_chain_%_rule_%[i][j]
            warning "Rule $i/$j: ${rule}"
        done
    done
\end{verbatim}
\end{example}

    With this sample configuration

\begin{example}
\begin{verbatim}
    PF_USR_CHAIN_N='2'
    PF_USR_CHAIN_1_NAME='usr-chain_a'
    PF_USR_CHAIN_1_RULE_N='2'
    PF_USR_CHAIN_1_RULE_1='ACCEPT'
    PF_USR_CHAIN_1_RULE_2='REJECT'
    PF_USR_CHAIN_2_NAME='usr-chain_b'
    PF_USR_CHAIN_2_RULE_N='1'
    PF_USR_CHAIN_2_RULE_1='DROP'
\end{verbatim}
\end{example}

    the following output is printed:

\begin{example}
\begin{verbatim}
    Warning: Rule 1/1: ACCEPT
    Warning: Rule 1/2: REJECT
    Warning: Rule 2/1: DROP
\end{verbatim}
\end{example}

    Alternatively, you can iterate directly over all values of the array
    (but the exact indices of the entries are not always known, because this is not
    required):

\begin{example}
\begin{verbatim}
    foreach rule in pf_usr_chain_%_rule_%
    do
        warning "Rule %{rule}='${rule}'"
    done
\end{verbatim}
\end{example}

    That produces the following output with the sample configuration from above:

\begin{example}
\begin{verbatim}
    Warning: Rule PF_USR_CHAIN_1_RULE_1='ACCEPT'
    Warning: Rule PF_USR_CHAIN_1_RULE_2='REJECT'
    Warning: Rule PF_USR_CHAIN_2_RULE_1='DROP'
\end{verbatim}
\end{example}

    The second example nicely shows the meaning of the
    \texttt{\%{<Name>}}-syntax: Within the string
    \texttt{\%{rule}} is substitued by the \emph{name} of the variable in question
    (for example \var{PF\_USR\_CHAIN\_1\_RULE\_1}), while \texttt{\${rule}}
    is substituted by its \emph{content} (i.e. \var{ACCEPT}).

\subsubsection{Encryption of Passwords: \texttt{crypt}}

Some variables contain passswords that should not be noted in plain text in
\texttt{rc.cfg}. These variables can be encrypted by the use of \texttt{crypt}
and are transferred to a format also needed on the router. Use this like here:

\begin{example}
\begin{verbatim}
    crypt (<Variable>)
\end{verbatim}
\end{example}

The \texttt{crypt} function is the \emph{only} point at which a configuration
variable can be changed.

\marklabel{subsec:statdatei}{
\subsubsection{Querying File Properties: \texttt{stat}}
}

    \texttt{stat} is used to query file properties. At the moment only file
    size can be accessed. If checking for files under the current configuration
    directory you may use the internal variable \var{config\_dir}. The Syntax:

\begin{example}
\begin{verbatim}
    stat (<file name>, <key>)
\end{verbatim}
\end{example}

    The command looks like this (the
    parameters used are only examples):

\begin{example}
\begin{verbatim}
    foreach i in openvpn_%_secret
    do
       stat("${config_dir}/etc/openvpn/$i.secret", keyfile)
       if (keyfile_res != "OK")
       then
          error "OpenVPN: missing secretfile <config>/etc/openvpn/$i.secret"
       fi
    done
\end{verbatim}
\end{example}

    The example checks whether a file exists in the current configuration directory.\\
    If \verb+OPENVPN_1_SECRET='test'+ is set in the configuration file, the loop
    in the first run checks for the existence of the file \texttt{etc/openvpn/test.secret}
    in the current configuration directory.

    After the call two variables are defined:

    \begin{itemize}
    \item \texttt{<Key>\_res}: Result of the system call stat() (``OK'', if
      system call was successful, else the error message of the system call)
    \item \texttt{<Key>\_size}: File size
    \end{itemize}

    It may for example look like this:

\begin{example}
\begin{verbatim}
    stat ("unix/Makefile", test)
    if ("$test_res" == "OK")
    then
            warning "test_size = $test_size"
    else
            error "Error '$test_res' while trying to get size of 'unix/Makefile'"
    fi
\end{verbatim}
\end{example}

    A file name passed as a string constant is subject of
    \jump{subsec:dev:string-rewrite}{variable substitution}.

\marklabel{subsec:fgrepdatei}{
\subsubsection{Search files: \texttt{fgrep}}
}

    If you wish to search a file via ``grep''\footnote{``grep'' is a common
    command on *nix-like OSes for filtering text streams.}
    you may use the \texttt{fgrep} command. The syntax is:

\begin{example}
\begin{verbatim}
    fgrep (<File name>, <RegEx>)
\end{verbatim}
\end{example}

    If the file \texttt{<File name>} does not exist \var{mkfli4l}
    will abort with a fatal error! If it is not sure if the file exists,
    test this before with \texttt{stat}. After calling \texttt{fgrep} the
    search result is present in an array called \var{FGREP\_MATCH\_\%}, with
    its index \emph{x} as usual ranging from one to \var{FGREP\_MATCH\_N}.
    \var{FGREP\_MATCH\_1} points to the whole range of the line the regular
    expression has matched, while \var{FGREP\_MATCH\_2} to \var{FGREP\_MATCH\_N}
    contain the \emph{n-1} th part in brackets.

    A first example will illustrate the use. The file
    \texttt{opt/etc/shells} contains the line:

\begin{example}
\begin{verbatim}
/bin/sh
\end{verbatim}
\end{example}

    The following code

\begin{example}
\begin{verbatim}
    fgrep("opt/etc/shells", "^/(.)(.*)/")
    foreach v in FGREP_MATCH_%
    do
      warning "%v='$v'"
    done
\end{verbatim}
\end{example}

    produces this output:

\begin{example}
\begin{verbatim}
    Warning: FGREP_MATCH_1='/bin/'
    Warning: FGREP_MATCH_2='b'
    Warning: FGREP_MATCH_3='in'
\end{verbatim}
\end{example}

    The RegEx has (only) matched with ``/bin/'' (only this part of the
    line is contained in the variable \var{FGREP\_MATCH\_1}). The first bracketed
    part in the expression only matches the first char after the first ``/'',
    this is why only ``b'' is contained in \var{FGREP\_MATCH\_2}. The second
    bracketed part contains the rest after ``b'' up to the last ``/'',
    hence ``in'' is noted in variable \var{FGREP\_MATCH\_3}.

    The following second example demonstrates an usual use of \texttt{fgrep}
    taken from \texttt{check/base.ext}. It will be tested if all \texttt{tmpl:}-references
    given in \var{PF\_FORWARD\_x} are really present.

\begin{example}
\begin{verbatim}
    foreach n in pf_forward_n
    do
      set rule=pf_forward_%[n]
      if (rule =~ "tmpl:([^[:space:]]+)")
      then
        foreach m in match_%
        do
          stat("$config_dir/etc/fwrules.tmpl/$m", tmplfile)
          if(tmplfile_res == "OK")
          then
            add_to_opt "etc/fwrules.tmpl/$m"
          else
            stat("opt/etc/fwrules.tmpl/$m", tmplfile)
            if(tmplfile_res == "OK")
            then
              add_to_opt "etc/fwrules.tmpl/$m"
            else
              fgrep("opt/etc/fwrules.tmpl/templates", "^$m[[:space:]]+")
              if (fgrep_match_n == 0)
              then
                error "Can't find tmpl:$m for PF_FORWARD_${n}='$rule'!"
              fi
            fi
          fi
        done
      fi
    done
\end{verbatim}
\end{example}

    Both a filename value as well as a regular expression passed as a string constant are subject to
    \jump{subsec:dev:string-rewrite}{variable substitution}.

\marklabel{subsec:split}{
\subsubsection{Splitting Parameters: \texttt{split}}
}

    Often variables can be assigned with several parameters, which then
    have to be split apart again in the startup scripts. If it is desired
    to split these previously and perform tests on them \texttt{split}
    can be used. The syntax is like this:

\begin{example}
\begin{verbatim}
    split (<String>, <Array>, <Separator>)
\end{verbatim}
\end{example}

    The string can be specified by a variable or directly as a
    constant. \var{mkfli4l} splits it where a separator is found
    and generates an element of the array for each part. You may iterate
    over these elements later on and perform tests. If nothing is found between
    two separators an array element with an empty string as its value is created.
    The exception is `` '': Here all spaces are deleted and no empty variable is
    created.

    If the elements generated by such a split should be in a numeric context (e.g.
    as indices) this has to be specified when calling \texttt{split}. This is done by
    the additional attribute ``numeric''. Such a call looks as follows:

\begin{example}
\begin{verbatim}
    split (<String>, <Array>, <Separator>, numeric)
\end{verbatim}
\end{example}

   An example:

\begin{example}
\begin{verbatim}
    set bar="1.2.3.4"
    split (bar, tmp_%, '.', numeric)
    foreach i in tmp_%
    do
            warning "%i = $i"
    done
\end{verbatim}
\end{example}

    the output looks like this:

\begin{example}
\begin{verbatim}
    Warning: TMP_1 = 1
    Warning: TMP_2 = 2
    Warning: TMP_3 = 3
    Warning: TMP_4 = 4
\end{verbatim}
\end{example}

    \textbf{Hint:} If using the ``numeric'' variant \var{mkfli4l} will \emph{not}
    check the generated string parts for really being numeric! If you use such a
    non-numeric construct later in a numeric context (i.e. in an addtion) \var{mkfli4l}
    will raise a fatal error. Example:

\begin{example}
\begin{verbatim}
    set bar="a.b.c.d"
    split (bar, tmp_%, '.', numeric)
    # Error: invalid number 'a'
    set i=tmp_%[1]+1
\end{verbatim}
\end{example}

    A string constant passed to \texttt{split} in the first parameter is subject of
    \jump{subsec:dev:string-rewrite}{variable substitution}.

\marklabel{subsec:addtoopt}{
\subsubsection{Adding Files to the Archives: \texttt{add\_to\_opt}}
}

    The function \texttt{add\_to\_opt} can add additional files to the
    Opt- or RootFS-Archives. \emph{All} files under \texttt{opt/} or from
    the configuration directory may be chosen. There is no limitation to only
    files from a specific package. If a file is found under \texttt{opt/} as
    well as in the configuration directory, \texttt{add\_to\_opt} will prefer
    the latter. The function \texttt{add\_to\_opt} is typically used if complex
    logical rules decide if and what files have to be included in the archives.\\

    The syntax looks like this:
\begin{example}
\begin{verbatim}
    add_to_opt <File> [<Flags>]
\end{verbatim}
\end{example}

    Flags are optional. The defaults from table ~\ref{table:options}
    are used if no flags are given.

    See an example from the package ``sshd'':

\begin{example}
\begin{verbatim}
    if (opt_sshd)
    then
       foreach pkf in sshd_public_keyfile_%
       do
         stat("$config_dir/etc/ssh/$pkf", publickeyfile)
         if(publickeyfile_res == "OK")
         then
             add_to_opt "etc/ssh/$pkf" "mode=400 flags=utxt"
         else
             error "sshd: missing public keyfile %pkf=$pkf"
         fi
       done
    fi
\end{verbatim}
\end{example}

    \jump{subsec:statdatei}{\texttt{stat}} at first checks for the file existing
    in the configuration directory. If it is, it will be included in the archive,
    if not, \var{mkfli4l} will abort with an error message.\\

    \textbf{Hint:} Also for \texttt{add\_to\_opt} \var{mkfli4l} will first
    \jump{subsec:konfigspezdatei}{check} if the file to be copied can be found
    in the configuration directory.\\

    Filenames as well flags passed as string constants are subject of
    \jump{subsec:dev:string-rewrite}{variable substitution}.

\marklabel{subsec:dev:control}{
\subsubsection{Control Flow}
}

\begin{example}
\begin{verbatim}
    if (expr)
    then
            statement
    else
            statement
    fi
\end{verbatim}
\end{example}

    A classic case distinction, as we know it. If the condition is true,
    the \texttt{then} part is executed, if the condition is wrong the \texttt{else} part.

    If you want to run tests on array variables, you have to test every single
    variable. The \texttt{foreach} loop in two variants for this.

    \begin{enumerate}
    \item Iterate over array variables:

\begin{example}
\begin{verbatim}
    foreach <control variable> in <array variable>
    do
            <instruction>
    done

    foreach <control variable> in <array variable-1> <array variable-2> ...
    do
            <instruction>
    done
\end{verbatim}
\end{example}

    This loop iterates over all of the specified array variables, each
    starting with the first to the last element, the number of elements
    in this array is taken from the N-variable associated with this array.
    The control variable takes the values of the respective array variables.
    It should be noted that when processing optional array variables that are
    not present in the configuration, an empty element is generated. You may
    have to take this into account in the script, for example like this:

\begin{example}
\begin{verbatim}
    foreach i in template_var_opt_%
    do
        if (i != "")
        then
            warning "%i is present (%i='$i')"
        else
            warning "%i is undefined (empty)"
        fi
    done
\end{verbatim}
\end{example}

    As you also can see in the example, the \emph{name} of the respective
    array variables can be determined with the \texttt{\%<control variable>}
    construction.

    The instruction in the loop may be one of the above
    control elements or functions (\texttt{if}, \texttt{foreach},
    \texttt{provides}, \texttt{depends}, \ldots).

    If you want to access exactly one element of an array, you can address it
    using the syntax \texttt{<Array>[<Index>]}. The index can be a normal variable,
    a numeric constant or again an indexed array.

    \item Iteration over N-variables:

\begin{example}
\begin{verbatim}
    foreach <control variable> in <N-variable>
    do
            <instruction>
    done
\end{verbatim}
\end{example}

    This loop executes from 1 to the value that is given in the N-variable. You can
    use the control variable to index array variables. So if you want to iterate over
    not only one but more array variables at the same time all controlled by the
    \emph{same} N-variable you take this variant of the loop and use the
    control variable for indexing multiple array variables. Example:

\begin{example}
\begin{verbatim}
    foreach i in host_n
    do
        set name=host_%_name[i]
        set ip4=host_%_ip4[i]
        warning "$i: name=$name ip4=$ip4"
    done
\end{verbatim}
\end{example}

    The resulting content of the \var{HOST\_\%\_NAME}- and \var{HOST\_\%\_IP4}-arrays
    for this example:

\begin{example}
\begin{verbatim}
    Warning: 1: name=berry ip4=192.168.11.226
    Warning: 2: name=fence ip4=192.168.11.254
    Warning: 3: name=sandbox ip4=192.168.12.254
\end{verbatim}
\end{example}

    \end{enumerate}

\subsubsection{Expressions}

    Expressions link values and operators to a new value. Such a value
    can be an normal variable, an array element, or a constant (Number,
    string or version number). All string constants in expressions are
    subject to \jump{subsec:dev:string-rewrite}{variable substitution}.

    Operators allow just about everything you could want from a
    programming language. A test for the equality of two variables
    could look like this:

\begin{example}
\begin{verbatim}
    var1 == var2
    "$var1" == "$var2"
\end{verbatim}
\end{example}

    It should be noted that the comparison is done depending on the type
    that was defined for the variable in \texttt{check/<PACKAGE>.txt}.
    If one of the two variables is \jump{subsec:dev:data-types}{numeric}
    the comparison is made numeric-based, meaning that the strings are
    converted to numbers and then compared. Otherwise, the comparison is done
    string-based; comparing \texttt{"05"\ == "5"} gives the result ``false'',
    a comparison \texttt{"18"\ < "9"} ``true'' due to the lexicographical string
    order: the digit ``1'' precedes the digit  ``9'' in the ASCII character set.

    For the comparison of version numbers the construct \texttt{numeric(version)}
    is introduced, which generates the numeric value of a version number for
    comparison purposes. Here applies:


\begin{example}
\begin{verbatim}
    numeric(version) := major * 10000 + minor * 1000 + sub
\end{verbatim}
\end{example}

    whereas ``major'' is the first component of the version number, ``minor'' the
    second and ``sub'' the third. If ``sub'' is missing the term in the addition above
    is omitted (in other words ``sub'' will be equalled to zero).

    A complete list of all expressions can be found in table \ref{tab:expr}. ``val''
    stands for any value of any type, ``number'' for a numeric value and ``string''for
    a string.

    \begin{table}[htb]
      \centering
      \caption{Logical Epressions}
      \label{tab:expr}
      \begin{tabular}{ll}
        \hline
        Expression &                     true if\\
        \hline
        \hline
       id                    &    id == ``yes''\\
       val  == val           &    values of identical type are equal\\
       val  != val           &    values of identical type are unequal\\
       val  == number        &    numeric value of val == number\\
       val  != number        &    numeric value of val != number\\
       val  $<$  number      &    numeric value of val $<$ number\\
       val  $>$  number      &    numeric value of val $>$ number\\
       val  == version       &    numeric(val) == numeric(version) \\
       val  $<$  version     &    numeric(val) $<$  numeric(version) \\
       val  $>$  version     &    numeric(val) $>$  numeric(version) \\
       val  =\verb?~? string &    regular expression in string matches val\\
       ( expr )              &    Expression in brackets is true\\
       expr \&\& expr        &    both expressions are true\\
       expr || expr          &    at least one of both expressions is true\\
       copy\_pending(id)     &    see description\\
       samenet (string1, string2) & string1 describes the same net as string2\\
       subnet (string1, string2)  & string1 describes a subnet of string2\\
        \hline
      \end{tabular}
    \end{table}

\subsubsection{Match-Operator}

With the match operator \verb?=~? you can check whether a regular
expression matches the value of a variable. Furthermore, one can
also use the operator to extract subexpressions from a variable.
After successfully applying a regular expression on a variable
the array \var{MATCH\_\%} contains the parts found. May look
like this:

\begin{example}
\begin{verbatim}
    set foo="foobar12"
    if ( foo =~ "(foo)(bar)([0-9]*)" )
    then
            foreach i in match_%
            do
                    warning "match %i: $i"
            done
    fi
\end{verbatim}
\end{example}

Calling \var{mkfli4l} then would lead to this output:

\begin{example}
\begin{verbatim}
    Warning: match MATCH_1: foo
    Warning: match MATCH_2: bar
    Warning: match MATCH_3: 12
\end{verbatim}
\end{example}

When using \verb?=~? you may take all existing regular expressions
into account. If you i.e. want to check whether a PCMCIA Ethernet driver
is selected without \var{OPT\_PCMCIA} being set to ``yes'', it might
look like this:

\begin{example}
\begin{verbatim}
    if (!opt_pcmcia)
    then
        foreach i in net_drv_%
        do
           if (i =~ "^(RE:PCMCIA_NET_DRV)$")
           then
               error "If you want to use ..."
           fi
        done
    fi
\end{verbatim}
\end{example}

As demonstrated in the example, it is important to \emph{anchor} the regular
expression with \texttt{\^} and \texttt{\$} if intending to apply the
expression on the \emph{complete} variable. Otherwise, the match-expression
already returns ``true'' if only a \emph{part} of the variable is covered by
the regular expression, which is certainly not desired in this case.

\subsubsection{Check if a File has been copied depending on the Value of a Variable: \texttt{copy\_pending}}

        With the information gained during the checking process the function \texttt{copy\_pending}
        tests if a file has been copied depending on the value of a variable or not.
        This can be used i.e. in order to test whether the driver specified by the user
        really exists and has been copied. \texttt{copy\_pending} accepts the name to be
        tested in the form of a variable or a string. \footnote{As described before
        the string is subject of variable substitution, i.e via a
        \jump{subsec:dev:control}{\texttt{foreach}-loop} and a
        \jump{subsec:dev:string-rewrite}{\texttt{\%<Name>}-subsitution}
        all elements of an array may be examined.} In order to accomplish this \texttt{copy\_pending}
        checks whether

        \begin{itemize}
        \item the variable is active (if it depends on an OPT it has to be set to ``yes''),

         \item the variable was referenced in an \texttt{opt/<PACKAGE>.txt}-file and

         \item whether a file was copied dependant on the current value.

        \end{itemize}

        \texttt{copy\_pending} will return ``true'' if it detects that during
        the last step \emph{no} file was copied, the copy process hence still is ``pending''.

    A small example of the use of all these functions
    can be found in \texttt{check/base.ext}:

\begin{example}
\begin{verbatim}
    foreach i in net_drv_%
    do
        if (copy_pending("%i"))
        then
            error "No network driver found for %i='$i', check config/base.txt"
        fi
    done
\end{verbatim}
\end{example}

    Alle elements of the array \var{NET\_DRV\_\%} are detected for
    which no copy action has been done because there is no corresponding
    entry existing in \texttt{opt/base.txt}.

\subsubsection{Comparison of Network Addresses: \texttt{samenet} und \texttt{subnet}}

For testing routes from time to time a test is needed whether two
networks are identical or if one is a subnet of the other. The two
functions \texttt{samenet} and \texttt{subnet} are of help here.

\begin{example}
\begin{verbatim}
    samenet (netz1, netz2)
\end{verbatim}
\end{example}

returns ``true'' if both nets are identical and

\begin{example}
\begin{verbatim}
    subnet (net1, net2)
\end{verbatim}
\end{example}

returns ``true'' if ``net1'' is a subnet of ``net2''.

\subsubsection{Expanding the Kernel Command Line}

If an OPT must pass other boot parameters to the kernel, in former times
the variable \var{KERNEL\_BOOT\_OPTION} had to be checked whether the required
parameter was included, and if necessary, a warning or error message had to be
displayed. With the internal variable \var{KERNEL\_BOOT\_OPTION\_EXT}
you may add a necessary but missing option directly in an ext-script. An
Example taken from \texttt{check/base.ext}:

\begin{example}
\begin{verbatim}
    if (powermanagement =~ "apm.*|none")
    then
        if ( ! kernel_boot_option =~ "acpi=off")
        then
            set kernel_boot_option_ext="${kernel_boot_option_ext} acpi=off"
        fi
    fi
\end{verbatim}
\end{example}

This passes ``acpi=off'' to the kernel if no or ``APM''-type power
management is desired.

\subsection{Support for Different Kernel Version Lines}

Different kernel version lines often differ in some details:
\begin{itemize}
\item changed drivers are provided, some are deleted, others are added
\item module names simply differ
\item module dependencies are different
\item modules are stored in different locations
\end{itemize}

These differences are mostly handled automatically by \var{mkfli4l}.
To describe the available modules you can, on one hand expand tests
dependant on the version
(\jump{sec:regexp-dependencies}{conditional regular expressions}), or,
on the other hand \var{mkfli4l} allows \emph{version dependant}
\texttt{opt/<PACKAGE>.txt}-files. These are then named
\texttt{opt/<PACKAGE>\_<Kernel-Version>.txt}, where the components of the
kernel version are separated from each other by underscores. An example:
the package ``base'' contains these files in its \texttt{opt}-directory:

\begin{itemize}
\item \texttt{base.txt}
\item \texttt{base\_3\_18.txt}
\item \texttt{base\_3\_19.txt}
\end{itemize}

the first file (\texttt{base.txt}) is \emph{always} considered. Both
other files are only considered if the kernel version is called ``3.18(.*)''
resp.\ ``3.19(.*)''. As seen here, some parts of the version may be omitted
in file names, if a group of kernels should be addressed.
If \verb+KERNEL_VERSION='3.18.9'+ is given, the following files (if existing)
are considered for the package \texttt{<PACKAGE>}:

\begin{itemize}
\item \texttt{<PACKAGE>.txt}
\item \texttt{<PACKAGE>\_3.txt}
\item \texttt{<PACKAGE>\_3\_18.txt}
\item \texttt{<PACKAGE>\_3\_18\_9.txt}
\end{itemize}

\subsection{Documentation}

    Documentation should be placed in the files

    \begin{itemize}
    \item \texttt{doc/<LANGUAGE>/opt/<PACKAGE>.txt}
    \item \texttt{doc/<LANGUAGE>/opt/<PACKAGE>.html}.
    \end{itemize}

    HTML-files may be splitted, meaning one for each OPT contained.
    Nevertheless a file \texttt{<PACKAGE>.html} has to be created
    linking to the other files. Changes should be documented in:

    \begin{itemize}
    \item \texttt{changes/<PACKAGE>.txt}
    \end{itemize}

    The entire text documentation may not contain any tabs and has
    to have a line feed no later than after 79 characters. This ensures
    that the documentation can also be read correctly with an editor
    without automatic line feed.

    Also a documentation in \LaTeX-format is possible, with HTML and
    PDF versions generated from it. The documentation of fli4l may serve as an example
    here. A documentation framework for required \LaTeX-macros can be found in
    the package ``template''. A brief description is to be found in the
    following subsections.

    The fli4l documentation is currently available in the following languages:
    German (\texttt{<LANGUAGE>} = ``deutsch''), English (\texttt{<LANGUAGE>} = ``english'') and
    French (\texttt{<LANGUAGE>} = ``french''). It is the package developer's
    decision to document his package in any language. For the purposes of
    clarity it is recommended to create a documentation in German and/or
    English (ideally in both languages).

\subsubsection{Prerequisites for Creating a \LaTeX Documentation}

  To create a documentation from \LaTeX-sources the following
  requirements apply:

  \begin{itemize}
  \item Linux/OS~X-Environment: For ease of production, a makefile
    exists to automate all other calls (Cygwin should work too, but
    is not tested by the fli4l team)
  \item LaTeX2HTML for the HTML version
  \item of course \LaTeX\ (Recommended: ``TeX Live'' for Linux/OS~X and
  ``MiKTeX'' for Microsoft Windows)  the ``pdftex''program and these
    \TeX-packages:
    \begin{itemize}
    \item current KOMA-Skript (at least version 2)
    \item all packages necessary for pdftex
    \item unpacked documentation package for fli4l, it provides the
      necessary makefiles and \TeX-styles
  \end{itemize}
  \end{itemize}


\subsubsection{File Names}

The documentation files are named according to the following scheme:

\begin{description}
\item [\texttt{<PACKAGE>\_main.tex}:] This file contains the main
  part of the documentation. \texttt{<PACKAGE>} stands for the name of the
  package to be described (in lowercase letters).
\item[\texttt{<PACKAGE>\_appendix.tex}:] If further comments should be
  added to the package, they should be placed there.
\end{description}

The files should be stored in the directory
\texttt{fli4l/<PACKAGE>/doc/<SPRACHE>/tex/<PACKAGE>}.
For the package sshd this looks like here:

\begin{verbatim}
    $ ls fli4l/doc/deutsch/tex/sshd/
    Makefile sshd_appendix.tex  sshd_main.tex  sshd.tex
\end{verbatim}

The Makefile is responsible for generating the documentation,
the \texttt{sshd.tex}-file provides a framework for the actual
documentation and the appendix, which is located in the other
two files. See an example in the documentation of the package
``template''.

\subsubsection{\LaTeX-Basics}

\LaTeX\ is, just like HTML, ``Tag-based'' , only that the tags
are called ``commands'' and have this format: \verb*?\command?
resp.\ \verb*?\begin{environment}? \ldots \verb*?\end{environment}?

By the help of commands you should rather emphasize the \emph{importance}
of the text less the \emph{display}. It is therefore of advantage to use

\begin{example}
\verb*?\warning{Please do not...}?
\end{example}

\noindent instead of

\begin{example}
\verb*?\emph{Please do not...}?
\end{example}

\noindent.

Each command rsp.\ each environment may take some more parameters
noted like this: \verb*?\command{parameter1}{parameter2}{parameterN}?.

Some commands have optional parameters in square (instead of curly)
brackets:\\ \verb*?\command[optionalParameter]{parameter1}?
\ldots\ Usually only one optional parameter is used, in
rare cases there may be more.

Individual paragraphs in the document are separated by blank lines.
Within these paragraphs \LaTeX\ itself takes care of line breaks and
hyphenation.

The following characters have special meaning in \LaTeX\ and, if
occuring in normal text, must be masked prefixed by a \verb*?\?:
\# \$ \& \_ \% \{ \}. ``\verb?~?'' and ``\verb?^?'' have to be
written as follows: \verb!\verb?~?! \verb!\verb?^?!

The main \LaTeX-commands are explained in the documentation of the
package ``template''.

\subsection{File Formats}

    All text files (both documentation and scripts, which later reside on
    the router) should be added to the package in DOS file format, with
    CR/LF instead of just LF at the end of a line. This ensures that Windows
    users can read the documentation even with ``notepad'' and that after changing
    a script under Windows everything still is executable on the router.

    The scripts are converted to the required format during archive creation
    (see the description of the flags in table~\ref{table:options}).

\subsection{Developer Documentation}

    If a program from the package defines a new interface that other programs
    can use, please store the documentation for this interface in a separate
    documentation in \texttt{doc/dev/<PACKAGE>.txt}.

\subsection{Client Programs}

    If a package also provides additional client programs, please store them
    in the directory \texttt{windows/} for Windows clients and in the directory
    \texttt{unix/} for *nix and Linux clients.

\subsection{Source Code}

    Customized programs and source code may be enclosed in the directory
    \texttt{src/<PACKAGE>/}. If the programs should be built like the rest
    of the fi4l programs, please have a look at the documentation of the
    \jump{buildroot}{``src''-package} .

\marklabel{sec:script_names}{
  \subsection{More Files}
}

    All files, which will be copied to the router have to be stored under
    \texttt{opt/}. Be under
    \begin{itemize}
    \item \texttt{opt/etc/boot.d/} and \texttt{opt/etc/rc.d/}: scripts, that should be
      executed on system start
    \item \texttt{opt/etc/rc0.d/}: scripts, that should be
      executed on system shutdown
    \item \texttt{opt/etc/ppp/}: scripts, that should be
      executed on dialin or hangup
    \item \texttt{opt/}: executable programs and other files
       according to their positions in the file system (for example the file
      \texttt{opt/bin/busybox} will later be situated in the directory \texttt{/bin}
      on the router)
    \end{itemize}

    Scripts in \texttt{opt/etc/boot.d/}, \texttt{opt/etc/rc.d/} and
    \texttt{opt/etc/rc0.d/}
    have the following naming scheme:

    \begin{example}
    \begin{verbatim}
    rc<number>.<name>
    \end{verbatim}
    \end{example}

    The number defines the order of execution, the name gives a hint on
    what program/package is processed by this script.
