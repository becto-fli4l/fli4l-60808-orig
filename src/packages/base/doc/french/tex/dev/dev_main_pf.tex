% Synchronized to r49529

\providecommand{\fwaction}[1]{{\small\textsf{#1}}}
\providecommand{\fwchain}[1]{\texttt{#1}}
\providecommand{\fwtable}[1]{\textsc{#1}}
\providecommand{\fwmatch}[1]{\texttt{#1}}
\providecommand{\fwpktstate}[1]{\texttt{#1}}
\providecommand{\fwloglevel}[1]{\texttt{#1}}

\section{Utiliser le filtrage de paquets}

\subsection{Ajouter vos propres chaînes et règles}

Un ensemble de routines est fourni pour manipuler le filtrage de paquets,
elles permettent d'ajouter ou de supprimer des chaînes (en anglais "Chains")
et des règles. Une chaîne sert à nommée une liste de règles ordonnées.
Il y a déjà un ensemble de chaînes prédéfinies sur le routeur fli4l
(\fwchain{PREROUTING}, \fwchain{INPUT}, \fwchain{FORWARD}, \fwchain{OUTPUT},
\fwchain{POSTROUTING}). Vous pouvez créer d'autre chaînes pour certaines
fonctions et selon vos besoins.

\begin{description}
\item [\texttt{add\_chain/add\_nat\_chain <chain>}~:]
   ajoute une chaîne de \og{}filtrage\fg{} ou de table \og{}nat\fg{}
\item [\texttt{flush\_chain/flush\_nat\_chain <chain>}~:]
   supprime toutes les règles d'une chaîne de \og{}filtrage\fg{} ou de table \og{}nat\fg{}
\item [\texttt{del\_chain/del\_nat\_chain <chain>}~:]
   supprime une chaîne à partir du \og{}filtrage\fg{} ou de la table \og{}nat\fg{}.
   Les chaînes doivent être vides avant qu'elles puissent être supprimées, et ces chaînes
   ne doivent pas avoir de référence. Une telle référence est par exemple l'action \fwaction{JUMP}
   dont la cible est précisément cette chaîne.
\item[\texttt{add\_rule/ins\_rule/del\_rule}~:]
   ajoute une règle à la fin d'une chaîne (\texttt{add\_rule}) ou partout dans la chaîne
   \texttt{ins\_rule}) ou supprime une règle d'une chaîne (\texttt{del\_rule}). l'exécution
   sera la suivante~:

\begin{example}
\begin{verbatim}
    add_rule <table> <chain> <rule> <comment>
    ins_rule <table> <chain> <rule> <position> <comment>
    del_rule <table> <chain> <rule> <comment>
\end{verbatim}
\end{example}

  \noindent les paramètres ont les significations suivantes~:
  \begin{description}
  \item[table] La table dans laquelle la chaîne est insérée
  \item[chain] La chaîne dans laquelle la règle est insérée
  \item[rule] La règle qui doit être insérée, vous devez utiliser le même
    format que dans le fichier-config
  \item[position] La position dans laquelle la règle doit être insérée
    (seulement avec \texttt{ins\_rule})
  \item[comment] Le commentaire, il sera affiché avec la règle, il sera vu
    dans le filtrage de paquets.
  \end{description}
\end{description}


\subsection{Classer les règles dans une infrastructure}

fli4l configure les règles du filtrage de paquets avec certaine norme. Si vous
souhaitez insérer vos propres règles, vous pouvez ajouter ces règles, après
les régles par défaut. Pour cela, vous devez savoir quelle action l’utilisateur
a effectué pour rejeter les paquets. Ces informations peut être obtenue pour 
les chaînes \fwchain{FORWARD} et \fwchain{INPUT} en utilisant deux fonctions,
\texttt{get\_defaults} et \texttt{get\_count}. Après l'exécution de

\begin{example}
\begin{verbatim}
    get_defaults <chain>
\end{verbatim}
\end{example}

On obtient les résultats suivants~:

\begin{description}
\item[\var{drop}~:] cette variable contient la chaîne dans laquelle le paquet
  sera dévié et rejeté.
\item[\var{reject}~:] cette variable contient la chaîne dans laquelle le paquet
  sera dévié et refusé.
\end{description}

Après l'exécution de

\begin{example}
\begin{verbatim}
    get_count <chain>
\end{verbatim}
\end{example}

la variable \var{res} contient le nombre de règle de la chaîne \texttt{<chain>}.
Cette position est importante car vous ne pouvez \emph{pas} utiliser simplement
\texttt{add\_rule} pour ajouter une règle à la fin de les chaînes de \og{}filtage\fg{}
prédéfini \fwchain{INPUT}, \fwchain{FORWARD} et \fwchain{OUTPUT}. Car ces chaînes
sont déjà terminées par des règles par défaut, qui traite tous les paquets restants,
en fonction de la disponibilité de la variable \var{PF\_<Chaîne>\_POLICY}. Donc si
vous insérez cette dernière règle \emph{après} les autres il n'y aura aucun effet.
La fonction \texttt{get\_count} permet maintenant, de détecter l'emplace directement
\emph{avant} la dernier règle et de transmettre cette position à la fonction
\texttt{ins\_rule} avec le paramètre \texttt{<position>} ainsi, la règle souhaitée
sera ajouter à la fin de la chaîne appropriée, cependant devant la dernier règle saisis. 

Voici un exemple à partir du script \texttt{opt/etc/rc.d/rc390.dns\_dhcp} du paquetage
\og{}dns\_dhcp\fg{} pour une petit explication~:

\begin{example}
\begin{verbatim}
    case $OPT_DHCPRELAY in
        yes)
            begin_script DHCRELAY "starting dhcprelay ..."

            idx=1
            interfaces=""
            while [ $idx -le $DHCPRELAY_IF_N ]
            do
                eval iface='$DHCPRELAY_IF_'$idx

                get_count INPUT
                ins_rule filter INPUT "prot:udp  if:$iface:any 68 67 ACCEPT" \
                    $res "dhcprelay access"

                interfaces=$interfaces' -i '$iface
                idx=`expr $idx + 1`
            done
            dhcrelay $interfaces $DHCPRELAY_SERVER

            end_script
        ;;
  esac
\end{verbatim}
\end{example}

Ici vous pouvez voir dans le milieu de la boucle l'exécution de \texttt{get\_count}
suivie par l'exécution de la fonction \texttt{ins\_rule}, entre autres la variable
\var{res} est passé comme paramètre \texttt{position}.


\subsection{Extension pour le test de filtrage de paquets}

fli4l utilise la syntaxe \fwmatch{match:params} dans les règles de filtrage des
paquets, cela permet d'avoir des conditions supplémentaires pour les paquets
(voir \fwmatch{mac:}, \fwmatch{limit:}, \fwmatch{length:}, \fwmatch{prot:}, \ldots).
Si vous souhaitez ajouter des tests supplémentaires, vous devez faire comme ci-dessous~:

\begin{enumerate}
\item Vous devez définir un nom approprié. Ce nom doit commencer par une lettre minuscule
entre a-z et ensuite être composé de n'importe quelles lettres et de nombres.

\achtung{Si vous testez de filtrage de paquets pour les règles IPv6, assurez-vous alors
que le nom du test n'est pas un composant d'une adresse IPv6 valide !}

\item Créer d'un fichier \texttt{opt/etc/rc.d/fwrules-<name>.ext}. Le contenu de ce
fichier ressemble à ceci~:

\begin{example}
\begin{verbatim}
    # IPv4 extension is available
    foo_p=yes

    # the actual IPv4 extension, adding matches to match_opt
    do_foo()
    {
        param=$1
        get_negation $param
        match_opt="$match_opt -m foo $neg_opt --fooval $param"
    }

    # IPv6 extension is available
    foo6_p=yes

    # the actual IPv6 extension, adding matches to match_opt
    do6_foo()
    {
        param=$1
        get_negation6 $param
        match_opt="$match_opt -m foo $neg_opt --fooval $param"
    }
\end{verbatim}
\end{example}
Le test de filtrage de paquets ne doit pas nécessairement être implémenté pour
les protocoles IPv4 et IPv6 (bien que cela soit préférable et logique pour les
deux protocoles de couche 3).

\item Test de l'extension~:

\begin{example}
\begin{verbatim}
    $ cd opt/etc/rc.d
    $ sh test-rules.sh 'foo:bar ACCEPT'
    add_rule filter FORWARD 'foo:bar ACCEPT'
    iptables -t filter -A FORWARD -m foo --fooval bar -s 0.0.0.0/0 \
        -d 0.0.0.0/0 -m comment --comment foo:bar ACCEPT -j ACCEPT
\end{verbatim}
\end{example}

\item L’extension est incluse et tous les fichiers restants
  (les composants \texttt{iptables}) sont archivés avec
  le mécanisme connus.
\item Permet l'extension dans la configuration, il faut ajouter
  \var{FW\_GENERIC\_MATCH} et/ou \var{FW\_GENERIC\_MATCH6} dans le fichier-exp,
  par exemple~:

\begin{example}
\begin{verbatim}
    +FW_GENERIC_MATCH(OPT_FOO) = 'foo:bar' : ''
    +FW_GENERIC_MATCH6(OPT_FOO) = 'foo:bar' : ''
\end{verbatim}
\end{example}
\end{enumerate}
