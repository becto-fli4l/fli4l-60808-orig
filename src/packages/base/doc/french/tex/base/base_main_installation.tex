% Do not remove the next line
% Synchronized to r30469

\chapter{Installation et configuration}

\section{Décompacter les archives}

Sous Linux~:

\begin{verse}\texttt{tar xvfz fli4l-\version.tar.gz}\end{verse}

\noindent Si cela ne fonctionne pas, essayez la procédure suivante~:

\begin{verse}\texttt{gzip -d < fli4l-\version.tar.gz | tar xvf -}\end{verse}

Pour ceux qui veulent installer fli4l dans un répertoire existant, ils doivent
utiliser le script \texttt{mkfli4l.sh -c} avant l'installation~:

\begin{verse}
    \texttt{cd fli4l-\version}\\
    \texttt{sh mkfli4l.sh -c}
\end{verse}

Il est toutefois recommandé d'utiliser un nouveau répertoire pour chaque nouvelle
version~-- la configuration peut être prise en charge très simplement avec un
outil servant à la comparaison des fichiers.

Sous Windows, l'archive de compression .tar peut être extraite, par exemple,
avec WinZip. Il faut faire attention que les fichiers dans les sous-répertoires
soient {\emph bien} décompressés (voir les paramètres dans Winzip~!). Il faut
vérifier que l'option "Smart TAR CR conversion" est décochée dans
{\emph Options \pfeil Configuration}. Si cette option est cochée il peut y avoir
quelques erreurs (plus ou moin important) à l'extraction des fichiers.

7-Zip (\altlink{http://www.7-zip.org/}) est un programme alternatif, il est
aussi puissant que WinZip et en plus il est open source, je vous le recommande.

Les fichiers suivants sont installés dans le sous-répertoire \texttt{fli4l-\version/}~:

\begin{itemize}
\item Documentation~:
  \begin{itemize}
  \item doc/deutsch/* Documentation Allemande
  \item doc/english/* Documentation Anglaise
  \item doc/french/* Documentation Française
  \end{itemize}

\item Configuration~:
  \begin{itemize}
  \item config/*.txt Fichiers de configurations, ils doivent être adaptés
  \end{itemize}

\item Scripts/Procédures~:
  \begin{itemize}
  \item mkfli4l.sh création du média de boot pour les fichiers configurés~:
    Version-Linux/Unix
  \item mkfli4l.bat création du média de boot pour les fichiers configurés~:
    Version-Windows
  \end{itemize}

\item Kernel/Fichier de boot~:
  \begin{itemize}
  \item img/kernel Linux-Kernel
  \item img/boot*.msg texte avec écran de démarrage
  \end{itemize}

\item Paquetage supplémentaire~:
  \begin{itemize}
  \item opt/*.txt Ces fichiers décrivent, la direction des programmes source
    et de configuration pour l'archive OPT.img.
  \item opt/... Optionnel Module-Kernel, fichiers et programmes.
  \end{itemize}

\item Code source~:
  \begin{itemize}
  \item src/* Code source/outils pour Linux, voir src/README
  \end{itemize}

\item Programme~:
  \begin{itemize}
  \item unix/mkfli4l* Création du disque de Boot~: Version-Unix/Linux
  \item windows/* Création du disque de Boot~: Version-Windows
  \item unix/imonc* client-imond pour Unix/Linux
  \item windows/imonc/* client-imond pour Windows
  \end{itemize}
\end{itemize}


\section{Configuration}

\subsection{Éditer les fichiers de configurations}

Pour configurer fli4l, vous devez paramétrez dans config/*.txt les fichiers.
Il est recommandé pour pouvoir comparer par la suite sa configuration ou pour
pouvoir gérer plusieurs configurations, de créer une copie du répertoire config
et d'effectuer la configuration dans cette copie. La comparaison des fichiers
de configurations sera alors possible au moyen d'outil approprié (par ex. "diff"
sous *nix) est relativement facile. Supposons que votre copie de config se trouve
dans le répertoire "ma\_config", vous devez d'abord aller dans le répertoire
fli4l et utiliser la commande~:

\begin{example}
\begin{verbatim}
    ~/src/fli4l> diff -u {config,ma_config}/build/rc.cfg | grep '^[+-]'
    --- config/build/rc.cfg    2007-03-22 15:34:39.085103706 +0100
    +++ ma_config/build/rc.cfg        2007-03-22 15:34:31.094317441 +0100
    -PASSWORD='/P6h4iOIN5Bbc'
    +PASSWORD='3P8F3KbjYgzUc'
    -NET_DRV_1='ne2k-pci'
    +NET_DRV_1='pcnet32'
    -START_IMOND='no'
    +START_IMOND='yes'
    -OPT_PPPOE='no'
    +OPT_PPPOE='yes'
    -PPPOE_USER='anonymer'
    -PPPOE_PASS='surfer'
    +PPPOE_USER='moi'
    +PPPOE_PASS='mon mot de passe'
    -OPT_SSHD='no'
    +OPT_SSHD='yes'
\end{verbatim}
\end{example}

On voit très bien ici, les différents paramètres qui sont configurés pour un
simple routeur-DSL, même si à première vue le fichier de configuration effraye
avec ca profusion de réglages.

\subsection{Configuration via un fichier de configuration spéciale}

La configuration se répartit sur différents fichiers avec le concept de module,
le travaille devient parfois un peut laborieux, on peut placer les fichiers de
configuration dans un fichier unique \emph{$<$liste~config$>$/\_fli4l.txt}
Il est plus facile de lire ou comparer son contenu que d'ouvrir la liste des
fichiers *.txt un par un, mais l'on doit quand même configurer et garder les
fichiers originaux pour la construction de fli4l. Pour rester sur l'exemple
mentionné ci-dessus, on peut configurer un simple routeur-DSL dans ce fichier~:

\begin{example}
\begin{verbatim}
    PASSWORD='3P8F3KbjYgzUc'
    NET_DRV_N='1'
    NET_DRV_1='pcnet32'
    START_IMOND='yes'
    OPT_PPPOE='yes'
    PPPOE_USER='moi'
    PPPOE_PASS='mon mot de passe'
    OPT_SSHD='yes'
\end{verbatim}
\end{example}

Vous devez éviter de mélanger différente version de configuration.

\subsection{Variables}

  Vous remarquerez que certaines variables sont commentées. Si c'est le cas, ils
  sont réduit à une information raisonnable. Cette attribution par défaut est
  documentée pour chaque variable. Si vous souhaitez insérer un autre commentaire
  pour cette variable, vous devez supprimer le commentaire et définir le votre,
  vous devez garder la caractère ('\#') au début du commentaire.


\marklabel{VARIANTEN}{\section{Procédures d'installation}}

Dans les versions précédentes, la seule option pour fli4l était de démarrer
sur une disquette. Maintenant, ce n'est plus possible pour les raisons
mentionnée ci-dessus, en alternatif vous pouvez utiliser une clé USB.

Maintenant vous pouvez démarrer sur d'autre média comme par exemple (CD,
HD, Réseau, Compact-Flash, DoC, \ldots), fli4l peut être installé sur
divers médias (HD, Compact-Flash, DoC). En plus, fli4l peut être démarré de
trois manières différentes~:

\begin{description}
\item [Single Image] Le Bootloader (ou chargeur automatique) charge le noyau
  Linux ensuite, fli4l est dans une seule image, ainsi fli4l peut être lancé
  sans avoir accés à aucun média de boot. Exemples d'utilisation pour les
  différents types de boot \emph{integrated}, \emph{attached}, \emph{netboot} et \emph{cd}.
\item [Split Image] Le Bootloader charge le noyau Linux, dans une première étape
  l'image rudimentaire de fli4l configure et monte sur le média. Dans une
  deuxième étape les fichiers restants sont chargés dans ce média à partir
  du média de boot. Exemple d'utilisation pour les différents types
  de boot \emph{hd (Typ A)}, \emph{ls120}, \emph{attached}, \emph{cd-emul}.
\item [Installation Medium] Le Bootloader charge le noyau Linux ensuite l'image
  rudimentaire de fli4l installe les fichiers systèmes sur le média existant,
  il n'a pas besoin de décompresser d'autre archive. Exemple pour une installation
  de disque dur avec le type B
\end{description}

Vous devez d'abord installer fli4l une fois dans la version minimale et ainsi
acquérir de expériences. Ensuite vous pourrez utiliser fli4l comme un répondeur
téléphonique ou comme un Proxy-HTTP. Vous avez ainsi l'avance d'avoir l'expérience
d'avoir un routeur essentiel qui fonctionne.

Pour l'installation, nous distinguons au total quatre versions~:

\begin{description}
\item[Clé-USB] Le routeur sur une clé-USB
\item[Lecteur-CD] Le routeur sur un CD
\item[réseau] pour booter sur le réseau filaire
\item[Installation-HD Typ A] routeur sur un disque dur, CF, DoC~-- avec une
  seul partition FAT
\item[Installation-HD Typ B] routeur sur un disque dur, CF, DoC~-- avec une
  partition FAT et une partitions ext3
\end{description}


\marklabel{INSTALLTYP0}{\subsection{Routeur sur une clé USB}}

Linux traite les clés USB comme des disques durs, donc les explications sont
les mêmes que pour une installation sur un disque dur. Notez s'il vous plaît,
que les pilotes en fonction du port USB doivent être chargées avec \var{OPT\_\-USB}
pour accéder à la clé USB puis avec \var{OPT\_\-HDINSTALL} pour l'installation 


\marklabel{INSTALLTYP0}{\subsection{Routeur sur CD ou par le réseau}}

Tous les fichiers nécessaires se trouvent sur le média de boot et décompacté
dans un disque RAM dynamique. Dans une configuration minimale le routeur fli4l
a besion de seulement 64 Mio RAM. Pour une configuration maximale le routeur est
limitée que par la capacité du média de boot et la mémoire principale installé.


\marklabel{INSTALLTYPA}{\subsection{Type A~: Routeur sur disque dur~-- Une seule partition FAT}}

C'est la même installation qu'avec la version CD, sauf que les fichiers sont stockés
sur un disque dur. Quand nous employons le terme \flqq{}Disque dur\frqq{} cela signifie
également d'autres dispositifs comme, un compacts flash de 8 Mio et d'autres
dispositifs que Linux peut traiter comme un disque dur. Depuis la version 2.1.4
fli4l peut utiliser le DiskOnChip mémoire flash de M-System et le disque SCSI.

La taille de l'archive OPT.img est limitée à la capacité du disque, tous les
fichiers systèmes doivent être installer sur un disque RAM la taille de la RAM
doit être appropriée. La consommation de RAM augmente par rapport au nombre de
paquetage.

Pour une mise à jour des progiciels (c.-à-d.~: mettre à jour opt.img et rc.cfg par
le réseau) la partition FAT doit avoir assez de place pour le kernel, le fichier
RootFS doit être plus ou moins égal à DEUX FOIS l'archive OPT.img~! Si vous voulez
utiliser une option supplémentaire, encore une fois le besoin d'espace augmente par
rapport à l'archive OPT.img.


\marklabel{INSTALLTYPB}{\subsection{Type B~: Routeur sur disque dur~-- Partition FAT et ext3}}

Contrairement au type A on utilise pas de disque virtuel. Les fichiers de l'archive
OPT.img sont copiés lors du démarrage du routeur sur la partition ext3 et seront
chargés depuis cette partition lorsque cela est nécessaire. Cette version a besoin
de moins de mémoire RAM et le nombre de paquetage n'est seulement limité que par
la taille du disque dur.

Pour plus d'informations sur l'installation des disques durs consultez la
documentation du paquetage HD, qui sera téléchargé séparément~- Pour commencer
activer la variable \var{OPT\_\-HDINSTALL}.
