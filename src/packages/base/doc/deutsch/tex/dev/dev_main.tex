% Last Update: $Id$
\chapter{Entwickler-Dokumentation}

Dieses Kapitel führt in die Aspekte von fli4l ein, die hauptsächlich für jene
interessant sind, die mit dem Gedanken spielen, fli4l durch die Entwicklung
eigener Pakete und OPTs zu erweitern. Es wird erläutert, wie fli4l "`unter der
Haube"' arbeitet und wie man in gewisse Prozesse eingreifen kann, um die
Funktionalität von fli4l zu verändern oder zu verbessern.

\section{Allgemeine Regeln}

Damit ein neues Paket in die OPT-Datenbank auf der fli4l-Homepage
aufgenommen wird, müssen einige Regeln beachtet werden. Pakete, die
sich nicht an diese Regeln halten, können ohne Vorwarnung aus der
Datenbank entfernt werden.

  \begin{enumerate}
    \item \emph{Keine} Kopieraktionen von Seiten des Benutzers! fli4l bietet ein
    ausgefeiltes System, um die Daten der fli4l-Pakete in die
    Installationsarchive einzupacken. Alle Dateien, die auf den Router sollen,
    liegen im Verzeichnis \texttt{opt/}.

    \item Pakete richtig packen und komprimieren: Die Pakete müssen so
    aufgebaut sein, dass sie sich mühelos ins fli4l-Verzeichnis entpacken
    lassen.

    \item Die Pakete sollen sich \emph{vollständig} über die Konfigurationsdatei
    konfigurieren lassen. Ein weiteres Bearbeiten der Konfigurationsdateien
    darf nicht vom Benutzer verlangt werden. Schwierige Entscheidungen dem
    Benutzer abnehmen oder in einen erweiterten Bereich verlagern (ans Ende
    der Konfigurationsdatei mit einem dicken Hinweis, etwa: ONLY MODIFY IF YOU
    KNOW WHAT YOU DO).

    \item Noch ein Hinweis zur Konfigurationsdatei: Anhand des Namens einer
    Variablen muss sich eindeutig erkennen lassen, zu welchem OPT sie gehört.
    So gehören z.\,B. zum \var{OPT\_\-HTTPD} die Variablen \var{OPT\_\-HTTPD},
    \var{HTTPD\_\-USER\_\-N}, usw.

    \item Bitte, bitte, macht möglichst kleine Binaries (Programme)! Das
    passiert automatisch, wenn ihr das FBR aus dem \emph{src}-Paket verwendet.
    Denkt auch daran, unnötige Features zu deaktivieren. Bei
    vorkompilierten Programmen hilft u.\,U. ein
\begin{example}
\begin{verbatim}
      strip -R .comment -R .note <Dateiname>
\end{verbatim}
\end{example}
    weiter. Nichts ist frustrierender als ein Download von 2~MiB, wenn man
    nachher feststellt, dass 500~kiB gereicht hätten\ldots

    \wichtig{Bitte auf diese Art und Weise keine Kernel-Module behandeln! Sie
    werden hinterher nicht mehr funktionieren!}

    \item Prüft euer Copyright! Wenn ihr Dateien als Vorlagen benutzt, achtet
    bitte darauf, das Copyright entsprechend zu ändern. Dies gilt besonders
    für die Config-, Check- und Opt-Textdateien. Ersetzt hier das Copyright
    durch euren eigenen Namen. Bei wortwörtlich kopierter Dokumentation muss
    natürlich das Copyright des Orginal-Autors erhalten bleiben!

    \item Bitte als Archivtypen nur verbreitete, freie Formate benutzen. Dazu
    gehören:
    \begin{itemize}
      \item ZIP  (\texttt{.zip})
      \item GZIP (\texttt{.tgz} oder \texttt{.tar.gz})
    \end{itemize}
    Andere Formate wie RAR, ACE, Blackhole, LHA etc.\ bitte nicht verwenden.
    Auch Windows-Installer-Dateien (\texttt{.msi}) oder selbstextrahierende Archive und
    Installer (\texttt{.exe}) sind nicht zu benutzen.
  \end{enumerate}

\marklabel{sec:libc}{\section{Übersetzen von Programmen}}

Die für das Übersetzen von Programmen zum Einsatz auf dem fli4l-Router
erforderliche Umgebung wird im separat erhältlichen \emph{src}-Paket angeboten.
Dort wird auch dokumentiert, wie sich eigene Programme für den fli4l übersetzen
lassen.

% Synchronized to r52055

\section{Concept modulaire}

Depuis la version 2.0, fli4l est réparti en modules (ou paquetages),
par exemple
\begin{itemize}
    \item fli4l-\version~~$<$--- paquetage base

    \item dns-dhcp
    \item dsl
    \item isdn
    \item sshd
\item et bien d'autres ...
\end{itemize}

Avec le paquetage base, fli4l est un simple routeur Ethernet. Pour ISDN
et/ou DSL, vous devez décompacter le paquetage ISDN et/ou DSL dans le
répertoire fli4l. Il en va de même pour les autres paquetages.


\marklabel{mkfli4l}{\subsection{mkfli4l}}

À partir des paquetages et en fonction des paramètres spécifiques des fichiers
de configurations, les fichiers suivants seront créés, le fichier avec
les configurations, \texttt{rc.cfg} et deux archives \texttt{rootfs.img} et
\texttt{opt.img}, ces fichiers sont produits à l'aide du programme \var{mkfli4l},
celui-ci lie les différents paquetages et contrôle les fichiers de
configurations pour repérer d'éventuelles erreurs.

\var{mkfli4l} accepte les options indiqués dans le tableau \ref{tab:mkfli4l}. Si
aucune valeur n'est indiquée, les valeurs par défaut entre les parenthèses
seront prises en compte. Vous pouvez voir la liste complète des options dans
(le tableau \ref{tab:mkfli4l}) avec la commande~:

\begin{verbatim}
    mkfli4l -h
\end{verbatim}

\begin{table}[htbp]
  \centering
  \caption{Paramètres pour \var{mkfli4l}}
  \begin{tabular}{|lp{2cm}|p{8cm}|}
   \hline
    \multicolumn{1}{|c|}{\textbf{Option}} & \multicolumn{2}{c|}{\textbf{Signification}} \\
    \hline
    -c, -\,-config    & \multicolumn{2}{|p{11cm}|} { Avec cette option, on
      défini le répertoire dans lequel \var{mkfli4l} recherche les fichiers
      de config des paquetages; (par défaut: config)} \\
    -x, -\,-check     & \multicolumn{2}{|p{11cm}|} { On place ici la liste
      des fichiers des paquets, dans le quel \var{mkfli4l} doit contrôler et
      rechercher les erreurs (\texttt{<PAQUETAGE>.txt}, \texttt{<PAQUETAGE>.exp} et
      \texttt{<PAQUETAGE>.ext}; (par défaut: check)} \\
    -l, -\,-log       & \multicolumn{2}{|p{11cm}|} { On place ici le
      fichier jounal, dans le quel \var{mkfli4l} enregistre les messages
      d'erreurs et les avertissements; (par défaut: \texttt{img/mkfli4l.log})} \\
    -p, -\,-package   & \multicolumn{2}{|p{11cm}|} { On place ici le paquetage
      qui doit être examiné, on peut indiquer dans cette option plusieurs paquetages
      à la fois pour qu'ils soient examinés. Cependant si vous utilisez
      l'option -p, l'option \texttt{<check\_dir>/base.exp} est en principe d'abord
      vérifier, afin de gérer les paramètres général du paquetage de base.} \\
    -i, -\,-info       & \multicolumn{2}{|p{11cm}|} { Avec cette option,
      on affiche des informations sur le déroulement de la construction
      (lecture des fichiers, vérification des fichiers, les problèmes
      rencontrés seront affichés lors du processus de contrôle).} \\
    -v, -\,-verbose    & \multicolumn{2}{|p{11cm}|} { Même signification que
      la variante -i} \\
    -h, -\,-help       & \multicolumn{2}{|p{11cm}|} { Avec cette option, on
      affiche l'aide} \\
    \html{\multirow{8}{*}{top}} \latex{\multirow{8}{*}{}}{-d, -\,-debug}  & \multicolumn{2}{|p{11cm}|}
      { Avec cette option, vous
      pouvez déboguer le processus de construction. Cette option sert surtout
      pour les développeurs de paquetage, qui souhaitent en savoir un peu plus,
      sur le déroulement du contrôle des paquetages.} \\
    \cline{2-3}
    \latex{&} \multicolumn{1}{|c|}{\textbf{Debugoption}} & \multicolumn{1}{c|}{\textbf{Signification}} \\
    \cline{2-3}
    \latex{&} \multicolumn{1}{|l|}{check} & show check process \\
    \latex{&} \multicolumn{1}{|l|}{zip-list} & show generation of zip list \\
    \latex{&} \multicolumn{1}{|l|}{zip-list-skipped} & show skipped files \\
    \latex{&} \multicolumn{1}{|l|}{zip-list-regexp} & show regular expressions for ziplist \\
    \latex{&} \multicolumn{1}{|l|}{opt-files} & check all files in \texttt{opt/<PAQUETAGE>.txt} \\
    \latex{&} \multicolumn{1}{|l|}{ext-trace} & show trace of extended checks \\
    \hline
  \end{tabular}
  \label{tab:mkfli4l}
\end{table}


\subsection{Structure}

Un paquetage peut contenir plusieurs opts, mais si en général le paquetage contient
seulement un opt, il est opportun de nommer le paquetage du même nom que l'OPT.
Vous pourrez ensuite remplacer \texttt{<PAQUETAGE>} par le nom des paquetages respectif.
Un paquetage comprend les éléments suivants~:

\begin{itemize}
\item Gestionnaire de fichiers
\item Documentation
\item Documentation du développeur
\item Programme client
\item Code source
\item Autres fichiers
\end{itemize}

Les différentes parties sont détaillées ci-dessous.


\subsection{Configuration du paquetage}

Dans le fichier \texttt{config/<PAQUETAGE>.txt}, les modifications de la configuration
du paquetage sont réalisées par l'utilisateur. Toutes les variables du fichier
de configuration doivent commencer uniformèment par le nom de ou des OPTs,
par exemple~:

\begin{example}
\begin{verbatim}
    #-------------------------------------------------------------------
    # Optional package: TELNETD
    #-------------------------------------------------------------------
    OPT_TELNETD='no'        # install telnetd: yes or no
    TELNETD_PORT='23'       # telnet port, see also FIREWALL_DENY_PORT_x
\end{verbatim}
\end{example}

Le fichier de configuration de l'OPT doit être configuré en conséquence avec une
en-tête (voir ci-dessus). Cela augmente la clarté, en particulier si le paquetage
contient plusieurs OPTs. Les variables associées à l'OPT~- ne doivent pas être écrites
en retrait~- également pour plus de clarté. Les commentaires et lignes vides sont
autorisés, les commentaires doivent commencer de manière uniforme jusqu'à la 33ème colonne.
Si une variable ou un commentaire occupe plus de 32 caractères, la ligne sera décalés
à la 33ème colonne. Les commentaires plus longs doivent être séparés dès la 33ème colonne
pour commençer une nouvelle ligne. Cette mesure vise à améliorer la lisibilité du fichier
de configuration.

Toutes les valeurs derrière le signe égale doivent être écrite entre des guillemets
\footnote{Il est possible d'utiliser les guillemets simples ou les guillemets doubles,
on peut donc écrire \texttt{FOO='bar'} ou \texttt{FOO="bar"}. Cependant, l'utilisation des
guillemets doubles doivent être exceptionnelle, vous devez vérifier absolument comment
le shell-Unix traite des guillemets simples et doubles}, autrement, lors du boot vous
pouvez avoir des problèmes.

    Les variables incluses (voir plus bas) dans le fichier \texttt{rc.cfg} sont activées,
    toutes les autres sont ignorées. La seule exception est la variable qui porte le nom
	\var{<PAQUETAGE>\_DO\_DEBUG}. Celle-ci sert pour déboguer des paquetages et est
	enregistrée en globalitée.


\marklabel{sec:opt_txt}{
  \subsection{Liste des fichiers à copier}
}

    Le fichier \texttt{opt/<PAQUETAGE>.txt} doit contenir les instructions suivantes.

\begin{itemize}
\item Quels fichiers correspond, à quel OPT,
\item Quand le fichier \texttt{opt.img} ou \texttt{rootfs.img} doit être transféré
	 et généré,
\item Quel ID-utilisateur (uid), ID-groupe (gid) et les droits qu'ils doivent obtenir,
\item Quel conversion doit être faite à l'archive avant l'enregistrement.

\end{itemize}

    Quand \var{mkfli4l} est exécuté il se base sur les archives nécessaires à l'installation.

    Les lignes vides et commençant par \og{}\#\fg{} sont ignorées. Dans l'une des deux premières
	lignes, vous devez indiquer la version du format du paquetage, qui doit être~:

\begin{example}
\begin{verbatim}
    <première colonne>   <deuxième colonne>   <troisième colonne>
    opt_format_version        1                    -
\end{verbatim}
\end{example}

    Les autres lignes ont la syntaxe suivante~:

\begin{example}
\begin{verbatim}
    <première colonne> <deuxième colonne> <troisième colonne> <colonne suivante>
     Variable           valeur             Fichier             Option
\end{verbatim}
\end{example}

    \begin{enumerate}
      \item Dans la première colonne se trouve les noms des variables,
        elle dépend du fichier appliqué de la troisième colonne. Si le nom
        de la variable apparait plusieurs fois dans la première colonne,
        cela veut dire que plusieurs fichiers en dépendent. Chaque variable
        qui est placée dans le fichier \texttt{opt/<PAQUETAGE>.txt}, est marqué
        par \var{mkfli4l}.

        Si plusieurs variables doivent être testés avec la même valeur, une liste
		de variables peut être utilisée (elles seront séparés par une virgule). Dans ce
		cas, il sera suffisant d'enregistrer au moins \emph{une} valeur requise dans la
		deuxième colonne pour toutes les variables. Il est important de ne \emph{pas} mettre
		d'espace entre les variables~!

        Pour les variables OPT (se sont les variables qui commence par \var{OPT\_} et qui
		accepte les valeurs \var{YESNO}) le préfixe og{}\var{OPT\_}fg{} peut être omis.
		En outre, il n'y a pas d'importance si les variables sont indiquées en majuscule,
		en minuscule (ou mixte).

      \item Dans la deuxième colonne sont indiquées les valeurs. Si la variable
        de la première colonne correspond à la valeur indiqué dans la deuxième
        colonne et si elle est activé, (voir plus bas), de plus si vous avez
		dans la première colonne \%-Variable qui se répète pour chaque indice différent,
		alors ces variables vérifient la valeur de la deuxième colonne, si elle
		correspond, alors tous les fichiers de la troisième colonne seront copiés.
		Il est à noter qu’en raison des valeurs des même variables un seul fichier
        sera copié.

        Il est possible d'indiquer le caractère og{}!fg{} devant la valeur. Dans ce cas,
		le test est annulé, ce qui signifie que le fichier sera copié seulement si la variable
		ne contient \emph{pas} de valeur.

      \item Dans la troisième colonne est indiqué le nom du fichier. Le chemin
        est relatif au répertoire \texttt{opt}. Le fichier doit exister et être lisible,
        sinon, il y aura une erreur et le programme \var{mkfli4l} qui génère
        automatiquement la construction du média de boot s’arrêtera.

        Si le nom du fichier commence par
        \texttt{rootfs:}-préfixe, le fichier sera copié dans la
        liste et sera inclus dans le RootFS avec les autre fichiers. Si il y a
        un préfixe au fichier il sera supprimé avant la copie.

        Si le fichier se trouve dans le sous répertoire config, il sera ajouté
        dans la liste des fichiers du répertoire config, mais ce fichier ne
        pourra pas avoir de préfixe \texttt{rootfs:}, comme les fichiers qui
        proviennent du sous répertoire \texttt{opt}.

        Si le fichier à copier est un module-kernel, on peut remplacer le nom
        de la version du kernel par \var{\$\{KERNEL\_VERSION\}}. \var{mkfli4l} prendra alors
        la version configurée et l'intègrera. Cela permet d'avoir un paquetage de
        module-kernel pour chaque versions différente, en plus la bonne version
        du kernel sera toujours copiée sur le routeur. Au sujet des modules-kernel
        le chemin peut être complètement omis, car \var{mkfli4l} trouve le chemin
        des modules en utilisant les fichiers \texttt{modules.dep} et
        \texttt{modules.alias}, reportez-vous à la section
        \jump{subsec:automatic-dependencies}{\og{}Résolution automatique pour la traçabilité des modules-kernel\fg{}}

        \begin{table}[ht!]
          \centering
          \caption{Options pour les fichiers}
          \label{table:options}
          \begin{scriptsize}
          \begin{tabular}{|p{2.5cm}|p{7cm}|p{4cm}|}
            \hline
            Option & Signification & Valeur standard \\
            \hline
            type= & type d'entrée~:\newline\newline
            \begin{tabular}{ll}
            \emph{local} & Objet fichiers-système \\
            \emph{file} & Fichier \\
            \emph{dir} & Répertoire \\
            \emph{node} & Matériel \\
            \emph{symlink} & Lien (symbolique)
            \end{tabular}\newline\newline
            Si elle est présente, cette option doit venir en premier.
            Le type \og{}local\fg{} représente un type d'objet existant dans le
            fichier système et correspond donc (le cas échéant) à \og{}file\fg{},
            \og{}dir\fg{}, \og{}node\fg{} ou \og{}symlink\fg{}. Les autres types, à l'exception de
            \og{}file\fg{} peuvent être utilisés pour enregistrer des archives, il ne
            doivent pas être présent dans le fichier système local. Par ex.,
            vous pouvez les utiliser pour créer des fichiers de périphérique
            dans l'archive-RootFS. & local \\
            uid= & Propriété du fichier, soit numériquement, soit
            en tant que mot de passe & root \\
            gid= & Groupe du fichier, soit numériquement, soit en
            tant que nom de groupe & root \\
            mode= & Pour les droits d'accès &
            Les Fichiers et matériels~:\newline\verb?rw-r--r--? (644)\newline
            Les Répertoires~:\newline\verb?rwxr-xr-x? (755)\newline
            Les liens~:\newline\verb?rwxrwxrwx? (777)\newline \\
            Les flags=\newline
            (type=file) & Conversion avant l'enregistrement dans l'archive~:\newline\newline
            \begin{tabular}{lp{5.5cm}}
            \emph{utxt} & Conversion au format-Unix \\
            \emph{dtxt} & Conversion au format-DOS \\
            \emph{sh} & Script Shell~: Conversion dans le format-UNIX,
            suppression des signes inutiles\\
            \emph{luac} & Script Lua~: Traduction en bytecode de Lua-VM
            \end{tabular}
            & \\ 
            name= & Nom alternatif sous lequel l'entrée est enregistrée dans
            l'archive & \\
            devtype=\newline
            (type=node) & Décrit le type de matériel (\og{}c\fg{} pour s'orienter vers
            la marque et \og{}b\fg{} pour s'orienter vers le matériel de bloc. Doit
            être à la deuxième place. & \\
            major=\newline
            (type=node) & Décrit le nombre soi-disant \og{}Majeur\fg{} -Numéro
            de fichier du périphérique. Doit être à la troisième place. & \\
            minor=\newline
            (type=node) & Décrit le nombre soi-disant \og{}Mineur\fg{} -Numéro
            de fichier du périphérique. Doit être à la quatrième place. & \\
            linktarget=\newline
            (type=symlink) & Décrit la cible du lien symbolique.
            Doit être à la deuxième place. & \\
            \hline
          \end{tabular}
          \end{scriptsize}
        \end{table}

      \item Vous pouvez voir dans la colonne du tableau
      \ref{table:options} le détail des options sur les propriétés,
      les groupes, les droits d'accès des fichiers et les conversions.

    \end{enumerate}

    Quelques exemples~:

    \begin{itemize}
    \item Copie le fichier, si \verb+OPT_TELNETD='yes'+,
      placez uid/gid pour le root et placez les droits sur 755 (\verb?rwxr-xr-x?).

\begin{example}
\begin{verbatim}
    telnetd     yes    usr/sbin/in.telnetd mode=755
\end{verbatim}
\end{example}

    \item Copie le fichier, placez uid/gid pour le root, les droits sur 555
      (\verb?r-xr-xr-x?) et converti le fichier dans le format Unix en même
      temps il supprime tous les caractères superflus.

\begin{example}
\begin{verbatim}
    base    yes     etc/rc0.d/rc500.killall mode=555 flags=sh
\end{verbatim}
\end{example}

    \item Copie le fichier, si \verb+PCMCIA_PCIC='i82365'+, placez
      uid/gid pour le root et placez les droits sur 644 (\verb?rw--r--r?)

\begin{example}
\begin{verbatim}
    pcmcia_pcic i82365 lib/modules/${KERNEL_VERSION}/pcmcia/i82365.ko
\end{verbatim}
\end{example}

         \item Copie le fichier, si l'une des variables \var{NET\_DRV\_\%} correspond à la
           deuxième colonne, placez uid/gid pour le root et placez les droits
           sur 644 (\verb?rw--r--r?)

\begin{example}
\begin{verbatim}
    net_drv_%   3c503  3c503.ko
\end{verbatim}
\end{example}

        \item Copie le fichier, si la variable \var{POWERMANAGEMENT} ne contient
		\emph{pas} la valeur \og{}none\fg{}~:

\begin{example}
\begin{verbatim}
    powermanagement !none etc/rc.d/rc100.pm mode=555 flags=sh
\end{verbatim}
\end{example}

        \item Copie le fichier, si l'une des variables OPT \var{OPT\_MYOPTA}
        ou \var{OPT\_MYOPTB} contient la valeur \og{}yes\fg{}~:

\begin{example}
\begin{verbatim}
    myopta,myoptb yes usr/local/bin/myopt-common.sh mode=555 flags=sh
\end{verbatim}
\end{example}

        Cet exemple est juste un raccourci pour~:

\begin{example}
\begin{verbatim}
    myopta yes usr/local/bin/myopt-common.sh mode=555 flags=sh
    myoptb yes usr/local/bin/myopt-common.sh mode=555 flags=sh
\end{verbatim}
\end{example}

        Et celui-ci est un raccourci pour~:

\begin{example}
\begin{verbatim}
    opt_myopta yes usr/local/bin/myopt-common.sh mode=555 flags=sh
    opt_myoptb yes usr/local/bin/myopt-common.sh mode=555 flags=sh
\end{verbatim}
\end{example}

        \item Copie du fichier \texttt{opt/usr/bin/beep.sh} dans l'archive rootfs,
		mais sera  renommer \texttt{bin/beep} avant~:

\begin{example}
\begin{verbatim}
    base yes rootfs:usr/bin/beep.sh mode=555 flags=sh name=bin/beep
\end{verbatim}
\end{example}

    \end{itemize}

    Les fichiers sont uniquement copiés, si les conditions indiquées plus
    haut sont remplies, si la variable est placé sur \verb+OPT_PAQUETAGE='yes'+.
    Quelle variable-OPT est associée? Pour la vérification voir le fichier
    \texttt{check/<PAQUETAGE>.txt}

    Si une variable est référencée dans le paquetage et que celle-ci n'est pas
    définie, il peut arriver que le paquetage correspondant n’est pas installé.
    Cela mènerait à un message d'erreur du programme \var{mkfli4l}, car \var{mkfli4l} attend
    que toutes les variables référencées dans le fichier \texttt{opt/<PAQUETAGE>.txt}
    soient définis.

    Pour gérer correctement cette situation, on introduit la fonction-\og{}weak\fg{}
    dans le format suivant~:

\begin{example}
\begin{verbatim}
    weak        variable    -
\end{verbatim}
\end{example}

    Si la variable est définie, même si elle n’est pas disponible la valeur de
	celle-ci sera indiquée \og{}undefiniert\fg{}. Toutefois, il convient de noter ici
	que le préfixe \og{}\var{OPT\_}\fg{} ne doit \emph{pas} être omis (s'il existe),
	sinon la variable sera définie \emph{sans} ce préfixe.

    Un exemple du fichier \texttt{opt/rrdtool.txt}~:

\begin{example}
\begin{verbatim}
    weak opt_openvpn -
    [...]
    openvpn    yes    usr/lib/collectd/openvpn.so
\end{verbatim}
\end{example}

    Sans la définition \texttt{weak}, la commande \var{mkfli4l} affichera un message
	d'erreur si vous utilisez le paquetage \og{}rrdtool\fg{} et si le paquetage \og{}openvpn\fg{}
	n'est pas aussi présent. La définition \texttt{weak} est également utilisé dans le cas
	où le paquetage \og{}openvpn\fg{} n'existe pas, il n'y aura aucun message d'erreur.


\marklabel{subsec:konfigspezdatei}{
\subsubsection{Fichiers de configurations spécifiques}
}

Dans certaine situations, on aimerait remplacer des fichiers de configurations
originaux par des fichiers spécifiques, qui serait compacté dans l’archive \texttt{opt.img},
par exemple, ajouter une Key-hôte ou ajouter son propre Scripte-Firewall, \ldots{}
var{mkfli4l} supporte ce scénario, il vérifie si un fichier est disponible dans le
répertoire config, dans ce cas, ce fichier sera ajouté dans la liste des fichiers de
l’archive \texttt{opt.img} ou \texttt{rootfs.img}.

Une autre façon d’ajouter des fichiers de configurations spécifiques dans
l'archive, est décrit dans le chapitre
\jump{subsec:addtoopt}{Contrôle de la configuration avancée}.


\marklabel{subsec:automatic-dependencies}{
\subsubsection{Résolution automatique pour la traçabilité des modules-kernel}}

Dans certaine circonstance un module-kernel a parfois besoin d'autres modules-kernel.
Ces modules doivent être chargés avant et serons également inclus dans l'archive.
\var{mkfli4l} détermine la traçabilité des modules avec les fichiers
\texttt{modules.dep} et \texttt{modules.alias} (les deux fichiers sont générés à la
compilation du kernel) ainsi ils ajouteront automatiquement tous les modules
nécessaires dans l’archive. Par exemple l’enregistrement peut être le suivant

\begin{example}
\begin{verbatim}
    net_drv_%   ne2k-pci    ne2k-pci.ko
\end{verbatim}
\end{example}

Le pilote ne2k\_pci dépend des fichiers suivant 8390.ko, et crc32.ko ils seront
ajoutés en premiers dans l'archive.

L’enregistrement du \texttt{modules.dep} et du \texttt{modules.alias} dans le RootFS
est nécessaire, ensuite \texttt{modprobe} utilise ces fichiers pour charger les pilotes.

\marklabel{subsec:dev:var-check}{
\subsection{Analyse des variables de configuration}
}

Avec le fichier \texttt{check/<PAQUETAGE>.txt} les variables peuvent être contrôlés
pour leurs validités. Ce contrôle était intégré dans les versions précédentes
du programme \var{mkfli4l}, mais les paquetages fli4l sont modulaire et nous avons
du installé un second-contrôle dans ce fichier. Dans ce fichier, une ligne est
accessible pour chaque variable du fichier de configuration. Ces lignes se composent
de quatre à cinq colonnes, qui ont les fonctions suivantes~:

\begin{enumerate}

\item Variable: Cette colonne indique le nom des variables à vérifier, qui sont
  dans le fichier config. Cette colonne indique le nom des variables à vérifier,
  si c'est une \emph{liste de variables}, elle peut apparaître plusieurs fois avec
  différents indices, donc à la place de l'indice le signe pourcentage (\%) sera inséré
  dans la variable. Le signe sera toujours indiqué comme ceci \og{}\var{\_\%\_}\fg{} au
  milieu de la variable et comme ceci \og{}\var{\_\%}\fg{} à la fin de la variable. La
  variable peut contenir plusieurs signes pourcentages, de sorte à réaliser des matrices
  multidimensionnels. Normalement vous ne serez pas amené à voir ces variables un peut
  étranges avec deux signes pourcentage, tels que \og{}\var{foo\_\%\_\_\%}\fg{}.

  Dans certaine situation, nous pouvons avoir besoin de variable optionnelle
  supplémentaire dans le fichier config. Ces variables sont présentes dans le
  fichier de contrôle, elles sont marquées et précédées du signe \og{}+\fg{}. Dans le
  tableau, vous pouvez voir aussi des variables précédées du signe \og{}++\fg{}. Avec
  le signe \og{}+\fg{}, ces variables sont présentes ou absente du fichier config. Avec
  le signe \og{}++\fg{}, ces variables sont totalement absente du fichier config. Vous
  pouvez les rajouter manuellement, si vous en avez besoin dans le fichier config.

\item \var{OPT\_\-VARIABLE}: Cette colonne indique, les variables OPT. Ces variables
  sont seulement vérifiées pour leurs validitées, à savoir si la variable est sur \og{}yes\fg{}.
  S'il n'y a pas de variable OPT un \og{}-\fg{} sera indiqué. Dans ce cas, la variable
  doit être définie dans le fichier de configuration, sauf si une valeur par défaut est
  défini (voir ci-dessous). Le nom de la variable OPT peut être arbitraire mais doit commencer
  par le préfixe \og{}\var{OPT\_}\fg{}.

  Si une variable ne dépend d'aucune variables OPT, elle sera considérée comme
  \emph{active}. Si elle dépende d'une variable OPT, elle sera explicitement active, si

  \begin{itemize}
  \item la variable OPT est active et
  \item la variable OPT contient la valeur "yes".
  \end{itemize}

  Dans tous les autres cas, la variable est inactive.

  \textbf{Remarque~:} les variables OPT inactifs seront replacées sur \og{}no\fg{} par \var{mkfli4l}
  si elle sont réglées sur \og{}yes\fg{} dans le fichier de configuration, un message
  d'avertissement sera alors généré (c'est à dire \og{}\verb+OPT_Y='yes' ignoré, parce que OPT_X='no'+\fg{}).
  Pour les chaînes de dépendance transitive (\var{OPT\_Z} dépend de \var{OPT\_Y}
  qui à son tour dépend de \var{OPT\_X}) cela fonctionnera de manière fiable, si
  les noms de tous les variables OPT commencent par \og{}\var{OPT\_}\fg{}.

\item \var{VARIABLE\_\-N}: Si dans la première colonne, vous avez des noms de
  variables avec le signe \%, elles spécifient la fréquence d'apparition de la
  variable (c'est la variable qui s'appelle \emph{N-variable}). Si la variable est
  multidimensionnel, le dernier index est en l'occurence spécifié. Si la variable
  dépend d'un OPT, la N-variable doit dépendre ou pas du même OPT. Si la variable
  ne dépend pas d'un OPT, la N-variable ne doit également pas en dépendre. Si aucune
  N-variable existe, le signe \og{}-\fg{} sera spécifier.

  Pour la compatibilité avec les futures versions de fli4l la variable spécifiée ici
  \emph{doit} être identique à la variable qui est dans \var{OPT\_VARIABLE} et le dernier
  signe \og{}\%\fg{} sera remplacé par un \og{}N\fg{} tout ce qui suit sera retiré. Pour
  la liste \var{HOST\_\%\_IP4} vous devez assigner N-Variable \var{HOST\_N} et pour la liste
  \var{PF\_USR\_CHAIN\_\%\_RULE\_\%} vous devez aussi assigner N-Variable \var{PF\_USR\_CHAIN\_\%\_RULE\_N}.
  \emph{Toutes les autres définitions de N-variable ne seront pas compatibles avec les
  versions futures de fli4l~!}

\item \var{VALUE}: cette colonne donne les valeurs possibles que peut prendre la
  variable. Vous pouvez par ex. avoir les informations suivant~:

  \begin{tabular}[ht!]{|l|l|}
    \hline
    Nom & Signification \\
    \hline
    \hline
    \var{NONE}     &  ne déclenche aucun contrôle\\
    \var{YESNO}    &  La variable doit être sur \og{}yes\fg{} ou sur \og{}no\fg{}\\
    \var{NOTEMPTY} &  La variable ne peut pas être vide\\
    \var{NOBLANK}  &  La variable ne devrait pas contenir d'espaces\\
    \var{NUMERIC}  &  La variable doit être numérique\\
    \var{IPADDR}   &  La variable doit être une adresse IP\\
    \var{DIALMODE} &  La variable doit être sur \og{}on\fg{}, \og{}off\fg{} ou \og{}auto\fg{}\\
    \hline
  \end{tabular}
  \\

  Si vous indiquez le préfixe \og{}\var{WARN\_}\fg{}, la valeur sera irrégulière
  et sera indiqué par un message, \var{mkfli4l} ne s’arrêtera pas, mais affichera
  seulement un avertissement.

  Le contrôle est défini par des expressions régulières dans le fichier
  \texttt{check/base.exp}. Ce fichier à été récemment étendu, il contient maintenant des
  contrôles complémentaires suivants~: \var{HEX}, \var{NUMHEX},
  \var{IP\_ROUTE}, \var{DISK} et \var{PARTITION}.

  Si les développeurs-opt ont besoin de rajouter une entrée, le nombre de termes
  peut, à tout moment être étendu.

  En outre, les expressions régulières peuvent être ajoutées directement dans
  le fichier du répertoire check, on peut également se référer à des expressions
  existantes. Par exemple au lieu d’utiliser \var{YESNO} on pourrait également écrire
\begin{example}
\begin{verbatim}
    RE:yes|no
\end{verbatim}
\end{example}
cela est utile pour un test qui est effectué qu’une seule fois, il est
relativement simple. Pour de plus amples informations, voir le chapitre suivant.

\item Paramètre par défaut~: dans cette colonne, une valeur facultative par défaut
pour les variables peut être définie, dans le cas où la variable n'est pas spécifié dans
le fichier de configuration.

\textbf{Remarque~:} à présent si cela ne fonctionne pas pour les variables de la liste.
La variable ne doit pas être facultative, donc il ne doit pas avoir le signe \og{}+\fg{}
devant le nom de la variable.

Exemple~:
\begin{example}
\begin{verbatim}
    OPT_TELNETD     -      -      YESNO    "no"
\end{verbatim}
\end{example}

La variable \var{OPT\_TELNETD} est maintenant manquante dans le fichier de configuration,
mais dans le fichier \texttt{rc.cfg} elle sera affichée avec la valeur \og{}no\fg{}.

\end{enumerate}

    La variable avec le signe pourcentage peut être mieux expliquée avec
    un exemple. Voici une partie du fichier \texttt{check/base.txt}~:
\begin{example}
\begin{verbatim}
    NET_DRV_N          -                  -                  NUMERIC
    NET_DRV_%          -                  NET_DRV_N          NONE
    NET_DRV_%_OPTION   -                  NET_DRV_N          NONE
\end{verbatim}
\end{example}

      En d'autres termes, en fonction de la valeur indiquée dans \var{NET\_\-DRV\_\-N}
      les variables \var{NET\_\-DRV\_\-N}, \var{NET\_\-DRV\_\-1\_\-OPTION},
      \var{NET\_\-DRV\_\-2\_\-OPTION}, \var{NET\_\-DRV\_\-3\_\-OPTION},
      etc. seront vérifiées.


\subsection{Définitions pour contrôler les variables de configuration}

\subsubsection{Introduction sur les expressions régulières}

  Dans la version 2.0, il y a que 7 expressions, susceptibles d’examinées les
  variables: \var{NONE}, \var{NOTEMPTY}, \var{NUMERIC},
  \var{IPADDR}, \var{YESNO}, \var{NOBLANK}, \var{DIALMODE}. Ces
  expressions ont été fixé dans \var{mkfli4l} pour la vérification, ils ne sont pas
  extensibles et se limite à essentielle aux \og{}types de données\fg{}, avec juste se
  qu’il faut pour pouvoir faire le contrôle.

  Dans la version 2.1 un nouveau contrôle a été créé. L'objectif de cette
  nouvelle création est de faite un contrôle plus flexible des variables, qui
  sera en mesure d'examiner des expressions plus complexes. C'est la raison pour
  laquelle les expressions régulières seront utilisées dans un ou plusieurs
  fichiers séparés. Il sera possible, d'une part, de vérifier les variables
  avant le contrôle par \var{mkfli4l} et d'autre part, les développeurs pourront
  définir leurs propres expressions, pour la configuration et le contrôle de
  leur paquetage.

  Vous pouvez trouver une description des expressions régulières, dans
  \og{}man 7 regex\fg{}, ou par exemple ici~: \altlink{http://unixhelp.ed.ac.uk/CGI/man-cgi?regex+7}.


\subsubsection{Spécification des expressions régulières}

  On peut spécifier des expressions de deux manières différentes~:

  \begin{enumerate}
  \item Extension exp spécifique au paquetage, dans le fichier \texttt{check/<PAQUETAGE>.exp}

    Ce fichier se trouve dans le répertoire-\texttt{check} et porte le même nom que son
    paquetage, par ex. \texttt{base.exp}. Les expressions contiennent les définitions, qui
    sont référencé dans le fichier \texttt{check/<PAQUETAGE>.txt}. Ainsi, \texttt{check/base.exp}
	peut contrôler les définitions. Bien connu le fichier \texttt{check/isdn.exp} qui contrôle
	la définition de la variable \var{ISDN\_\-CIRC\_\-}?\_ROUTE (à l'origine le contrôle de cette
    variable était absente cela a été modifié).

Chaque définition sera écrite entre deux apostrophes, la syntaxe est la suivante~:
\begin{example}
\begin{verbatim}
    <Name> = '<expression régulière>' : '<le message d'erreur>'
\end{verbatim}
\end{example}
Autre exemple de la \texttt{check/base.exp}~:
\begin{example}
\begin{verbatim}
    NOTEMPTY = '.*[^ ]+.*'          : 'should not be empty'
    YESNO    = 'yes|no'             : 'only yes or no are allowed'
    NUMERIC  = '0|[1-9][0-9]*'      : 'should be numeric (decimal)'
    OCTET    = '1?[0-9]?[0-9]|2[0-4][0-9]|25[0-5]'
             : 'should be a value between 0 and 255'
    IPADDR   = '((RE:OCTET)\.){3}(RE:OCTET)' : 'invalid ipv4 address'
    EIPADDR  = '()|(RE:IPADDR)'
             : 'should be empty or contain a valid ipv4 address'
    NOBLANK  = '[^ ]+'              : 'should not contain spaces'
    DIALMODE = 'auto|manual|off'    : 'only auto, manual or off are allowed'
    NETWORKS = '(RE:NETWORK)([[:space:]]+(RE:NETWORK))*'
             : 'no valid network specification, should be one or more
                network address(es) followed by a netmask,
                for instance 192.168.6.0/24'
\end{verbatim}
\end{example}

Dans les expressions régulières, vous pouvez toujours rajouter une référence,
dans une définition existante. Cela est plus facile que de construire une
expression régulière. Il suffit simplement d’insérer la référence de cette manière
'(RE:référence)'. (Voir la définition de l'expression \var{NETWORKS} ci-dessus
pour un exemple approprié.)

Les messages d'erreur ont tendance à être trop longs. Il est donc possible, de
les afficher sur plusieurs lignes.
Vous devez toujours utiliser au début de la ligne un espace ou une tabulation.
Lors de la lecture du fichier \texttt{check/<PAQUETAGE>.exp} des espaces supplémentaires
sont réduits à un seul espace et une tabulation sera remplacé par des espaces.

La configuration du fichier \texttt{check/<PAQUETAGE>.exp} pourrait alors ressembler à ce qui suit~:

\begin{example}
\begin{verbatim}
    NUM_HEX         = '0x[[:xdigit:]]+'
                    : 'should be a hexadecimal number
                       (a number starting with "0x")'
\end{verbatim}
\end{example}

\item \sloppypar{Les expressions régulières sont directement installées dans le fichier
  de contrôle \texttt{check/<PAQUETAGE>.txt}}

Certaines expressions sont utilisé qu’une seule fois, ce n’est pas la peine de
définir ces expressions régulières dans le fichier \texttt{check/<PAQUETAGE>.exp}. Nous pouvons alors,
enregistrer simplement ces expressions dans le fichier-Check, par ex.~:

\begin{example}
\begin{verbatim}
    # Variable      OPT_VARIABLE    VARIABLE_N     VALUE
    MOUNT_BOOT      -               -              RE:ro|rw|no
\end{verbatim}
\end{example}

La variable \var{MOUNT\_\-BOOT} ne peut qu’accepter les valeurs \og{}ro\fg{}, \og{}rw\fg{} ou
\og{}no\fg{} toutes les autres sont rejetées

Si vous voulez référencer une expression régulière existants, vous devez
simplement ajouter la référence de cette manière \og{}(RE:...)\fg{}, par ex.:

\begin{example}
\begin{verbatim}
    # Variable      OPT_VARIABLE    VARIABLE_N     VALUE
    LOGIP_LOGDIR    OPT_LOGIP       -              RE:(RE:ABS_PATH)|auto
\end{verbatim}
\end{example}
\end{enumerate}


\subsubsection{Extension d’une expression régulière déjà existant}

Si vous ajoutez un paquetage optionnel supplémentaire, vous devez ajouter une
expression régulière pour que la valeur de la variable soit examinée,
l'expression régulière doit être agrandie, cela se passe simplement par l’ajout
une nouvelle définition des valeurs dans l’expression régulière (comme décrit
plus haut). Le complément de l'expression régulière existante est copié dans le
fichier \texttt{check/<PAQUETAGE>.exp}. L’expression existante sera modifiée, par
le caractère \og{}+\fg{} qui sera ajouté au premier plan. L’expression existante sera
complétée par une nouvelle valeur, ainsi la nouvelle valeur est ajoutée comme
l'alternative à la valeur existante. Si une autre expression utilise l'expression
qui a été complétée, le complément de cette expression sera aussi valable. Vous
pouvez indiquer un message d'erreur, il sera simplement ajouté après l’expression.

Voici un exemple pour les pilotes-Ethernet~:

\begin{itemize}
\item Le paquetage-base a un ensemble de pilotes Ethernet prédéfini, on sélectionne
la variable \var{NET\_DRV\_x} avec l'expression régulière \var{NET\_DRV} pour
contrôler cette variable, voici ce qui est écrit dans le fichier de contrôle~:

\begin{example}
\begin{verbatim}
    NET_DRV         = '3c503|3c505|3c507|...'
                    : 'invalid ethernet driver, please choose one'
                      ' of the drivers in config/base.txt'
\end{verbatim}
\end{example}
\item Nous mettons à disposition le paquetage-PCMCIA pour avoir des pilotes de
  périphérique supplémentaires, vous devez alors compléter la variable
  \var{NET\_DRV}, pour avoir quelque chose comme ceci~:

\begin{example}
\begin{verbatim}
    PCMCIA_NET_DRV = 'pcnet_cs|xirc2ps_cs|3c574_cs|...' : ''
    +NET_DRV       = '(RE:PCMCIA_NET_DRV)' : ''
\end{verbatim}
\end{example}
\end{itemize}

Maintenant, vous pouvez également sélectionner les pilotes PCMCIA supplémentaires.


\subsubsection{Expression régulière élargie en relation avec les variables \var{YESNO}}

Après avoir ajouté le pilote PCMCIA dans NET\_DRV comme décrit plus haut.
Si le paquetage \og{}pcmcia\fg{} est désactivé et si vous voulez quand même choisir
un pilote PCMCIA dans \texttt{config/base.txt}, sans avoir de message d'erreur à
la création du support de boot. Vous pouvez modifier l'expression régulière avec la
variable \var{YESNO} dans la configuration. Pour cela vous devez ajouter des parenthèses
immédiatement après le nom de la variable sur l'expression pour déterminer
si l'expression est étendu. Si la variable est active avec la valeur \og{}yes\fg{},
l'expression sera étendu, autrement elle ne l'est pas.

\begin{example}
\begin{verbatim}
    PCMCIA_NET_DRV       = 'pcnet_cs|xirc2ps_cs|3c574_cs|...' : ''
    +NET_DRV(OPT_PCMCIA) = '(RE:PCMCIA_NET_DRV)' : ''
\end{verbatim}
\end{example}

Maintenant si on veut utiliser par ex. le pilote \texttt{xirc2ps\_cs} dans \texttt{config/base.txt} et
si la variable est paramétré sur \verb+OPT_PCMCIA='no'+, il y aura un message
d’erreur à la création des archives.

\textbf{Remarque~:} si cela ne fonctionne \emph{pas}, la variable n'est peut être pas définie
explicitement dans le fichier de configuration mais elle obtient une valeur avec un paramètre
par défaut dans \texttt{check/<PAQUETAGE>.txt}. Dans ce cas, vous devez définir explicitement
la variable et enlever le paramètre par défaut si nécessaire.


\marklabel{sec:regexp-dependencies}{
  \subsubsection{Définir une expression régulière en fonction d'autres variables}
}

Vous pouvez alternativement utiliser toutes les valeurs de la variable, à
condition, que la syntaxe apparaît comme indiquer ci-dessous~:

\begin{example}
\begin{verbatim}
    +NET_DRV(KERNEL_VERSION=~'^3\.18\..*$') = ...
\end{verbatim}
\end{example}

Si l'expression \var{KERNEL\_VERSION} correspond au donné, c'est-à-dire ici à
utilisation du kernel version 3.18, alors la liste complète des pilotes réseaux
qui tournent avec celui-ci seront autorisés.

\textbf{Remarque~:} si cela ne fonctionne \emph{pas}, la variable n'est peut être pas définie
explicitement dans le fichier de configuration mais elle obtient une valeur avec un paramètre
par défaut dans \texttt{check/<PAQUETAGE>.txt}. Dans ce cas, vous devez définir explicitement
la variable et enlever le paramètre par défaut si nécessaire.


\subsubsection{Message d'erreur}

Si pendant le contrôle, \var{mkfli4l} trouve une erreur, un message de cette erreur
apparaîtra exemple~:

\begin{example}
\begin{verbatim}
    Error: wrong value of variable HOSTNAME: '' (may not be empty)
    Error: wrong value of variable MOUNT_OPT: 'rx' (user supplied regular expression)
\end{verbatim}
\end{example}

La première erreur, a été définie dans le fichier \texttt{check/<PAQUETAGE>.exp},
une indication sur l'erreur sera affichée. La deuxième erreur à été spécifié
directement dans le fichier \texttt{check/<PAQUETAGE>.txt}, il n'y aura aucune indication
supplémentaire sur la raison de l'erreur.


\subsubsection{Définition des expressions régulières}

Les expressions régulières sont définies comme indiqué ci-dessous~:

Expression régulière~: vous pouvez paramétrer une ou plusieurs options, vous
devez les séparer par le signe \og{}$|$\fg{}, par ex. \og{}ro$|$rw$|$no\fg{}. Si l’une des
options s'applique, alors l'expression sera appliquée (ici les expressions
valides sont \og{}ro\fg{}, \og{}rw\fg{} et \og{}no\fg{}).

Une option est une concaténation de tronçon, qui sont simplement reliés
entre eux.

Un tronçon est un \og{}Atome\fg{}, suivi pas un signe quantificateur \og{}*\fg{}, \og{}+\fg{},
\og{}?\fg{} ou \og{}\{min, max\}\fg{}. La signification est la suivante~:
\begin{itemize}
\item   \og{}a*\fg{} - Capture le caractère a, zéro ou plusieurs fois
\item   \og{}a+\fg{} - Capture le caractère a, une ou plusieurs fois
\item   \og{}a?\fg{} - Capture le caractère a, zéro ou une fois
\item   \og{}a\{2,5\}\fg{} - Capture le caractère a, entre 2 et 5 fois
\item   \og{}a\{5\}\fg{} - Capture le caractère a, 5 fois
\item   \og{}a\{2,\}\fg{} - Capture le caractère a, au moins 2 fois
\item   \og{}a\{,5\}\fg{} - un maximum de cinq \og{}a\fg{}s
\end{itemize}

Un \og{}atome\fg{} est
\begin{itemize}
        \item Une expression régulière entre parenthèses, par ex. \og{}(a$|$b)+\fg{} s'applique à
          toute chaîne qui contient au moins un \og{}a\fg{} ou \og{}b\fg{}, dans n'importe quel ordre
        \item Une paire de parenthèses vide, représente une expression \og{}vide\fg{}
        \item Une expression entre les crochets \og{}[\,]\fg{} (voir ci-dessous)
        \item Le point \og{}.\fg{} remplace n'importe quel caractère, par ex. \og{}.+\fg{}
          s'applique à toute chaîne qui contient au moins un caractère.
        \item \og{}\^\,\fg{} représentre le début d’une chaine de caractères, par ex.
          \og{}\^\,a.*\fg{} s'applique à une chaîne de caractères qui commence par un \og{}a\fg{} et
          suivi par un caractère quelconque, \og{}a\fg{} ou \og{}adkadhashdkash\fg{}.
        \item \og{}\$\fg{} fin d’une chaine de caractères, fin de ligne
        \item \og{}$\backslash$\fg{} suivis par l'un des caractères spéciaux
          \texttt{\^\,.\,[\,\$\,(\,)\,$|$\,*\,+\,?\,\{\,$\backslash$} correspond au caractère réel sans sa
          signification spéciale.
        \item Un caractère écrit, est exactement un caractère normale, par ex.
          \og{}a\fg{} est le caractère alphabétique \og{}a\fg{}.
\end{itemize}

Signification d'une expression entre des crochets, lire la suite.
\begin{itemize}
                \item \og{}x-y\fg{} - Trouvera tout les lettres qui se situe
                  entre \og{}x\fg{} et \og{}y\fg{}, par ex. \og{}[0-9]\fg{} correspond à tous les caractères
                  de 0-9, \og{}[a-zA-Z]\fg{} correspond à toutes les lettres, qu'elles
                  soient majuscule ou minuscule

                \item \og{}\^\,x-y\fg{} - Trouvera n'importe quel lettres qui est
                  en dehors des crochets par exemple \og{}[\^\,0-9]\fg{} s'applique à toutes les lettres
				  mais \emph{pas} au chiffres

                \item \og{}[:\emph{character-class}:]\fg{} - Trouvera une classe de caractères
				  \emph{character-class}. Voici les classes de caractères standard~: \texttt{alnum}, \texttt{alpha},
                  \texttt{blank}, \texttt{digit}, \texttt{lower}, \texttt{print}, \texttt{punct}, \texttt{space},
				  \texttt{upper} et \texttt{xdigit}. Donc \og{}[\,[:alpha:]\,]\fg{} est utilisé pour toutes
				  les lettres majuscules et minuscules et est identique à \og{}[\,[:lower:]\,[:upper:]\,]\fg{}.

\end{itemize}


\subsubsection{Exemples d'expressions régulières}

Jetons un coup d'oeil sur quelques exemples~:

\var{NUMERIC}~: Valeur numérique qui est constitué d’au moins de un, mais aussi d'un
nombre quelconque de chiffres. Avec le signe \og{}+\fg{} la valeur peut être répéter au
moins une ou plusieurs fois à partir d'un nombre, voici un exemple pour obtenir
un nombre composé~:

\begin{example}
\begin{verbatim}
    NUMERIC = '[0-9]+'
\end{verbatim}
\end{example}

    ou alors

\begin{example}
\begin{verbatim}
    NUMERIC = '[[:digit:]]+'
\end{verbatim}
\end{example}

\var{NOBLANK}~: valeur qui ne contient pas d'espace, les caractères peuvent être
quelconque (à l'exception de l'espace), voici un exemple avec divers caractères~:

\begin{example}
\begin{verbatim}
    NOBLANK = '[^ ]*'
\end{verbatim}
\end{example}

De plus cette valeur, ne doit pas être vide~:

\begin{example}
\begin{verbatim}
        NOBLANK = '[^ ]+'
\end{verbatim}
\end{example}

\var{IPADDR}: Une adresse IP se compose de 4 octet ils sont séparés par un \og{}.\fg{} point. Un octet
peut être un nombre compris entre 0 et 255. Si nous voulons définir le première
octet, il faut.\\

\begin{tabular}[ht!]{lr}
  Avoir un chiffre entre 0 et 9: &       [0-9]\\
  un nombre compris entre 10 et 99: &     [1-9][0-9]\\
  un nombre compris entre 100 et 199:&   1[0-9][0-9]\\
  un nombre compris entre 200 et 249: &  2[0-4][0-9]\\
  Avoir un nombre entre 250 et 255: & 25[0-5]\\
\end{tabular}\\

Suite de la solution, nous allons séparer tout simplement par le signe '$|$'
chaque partie de l'adresse-IPv4 avec l'expression~:
\og{}[0-9]$|$[1-9][0-9]$|$1[0-9][0-9]$|$2[0-4][0-9]$|$25[0-5]\fg{} ainsi nous avons
tous les octets. Cela nous permet désormais d'avoir notre adresse-IPv4 de 4 octets
séparé par un point (pour écrire un point vous devez placer un \emph{Backslashs}
(ou barre oblique inversée) sinon, il remplacera tout caractère). Vous pouvez voir
ci-dessous la syntaxe tiré du fichier exp~:

\begin{example}
\begin{verbatim}
    OCTET  = '[0-9]|[1-9][0-9]|1[0-9][0-9]|2[0-4][0-9]|25[0-5]'
    IPADDR = '((RE:OCTET)\.){3}(RE:OCTET)'
\end{verbatim}
\end{example}


\subsubsection{Aide à la conception des expressions régulières}

Si vous voulez créer et tester les expressions régulières, vous pouvez
utiliser l'outil regexp pour l'expression rationnelle,
qui est situé dans le répertoire \texttt{unix}
ou \texttt{windows} du paquetage \og{}base\fg{}.
Avec la syntaxe suivante~:

\begin{example}
\begin{verbatim}
    utilisation de~: regexp [-c <check dir>] <regexp> <string>
\end{verbatim}
\end{example}

Courte explication des paramètres~:

\begin{itemize}
\item \texttt{<check dir>} est le répertoire contenant les fichiers de contrôle et
de vérification avec les fichiers exp, ainsi \og{}regexp\fg{} pourra recourrir au
expression déja définie.

\item \texttt{<regexp>} est une expression régulière (dans le doute, toujours
utiliser les guillemets \verb+'...'+ ou \verb+"..."+ ils sont nécessaires
si les apostrophes veulent apparaître dans l'expression)

\item \texttt{<string>} est la chaîne à examiner
\end{itemize}

Voici quelques exemples~:
\begin{example}
\begin{verbatim}
./i586-linux-regexp -c ../check '[0-9]' 0
adding user defined regular expression='[0-9]' ('^([0-9])$')
checking '0' against regexp '[0-9]' ('^([0-9])$')
'[0-9]' matches '0'

./i586-linux-regexp -c ../check '[0-9]' a
adding user defined regular expression='[0-9]' ('^([0-9])$')
checking 'a' against regexp '[0-9]' ('^([0-9])$')
regex error 1 (No match) for value 'a' and regexp '[0-9]' ('^([0-9])$')

./i586-linux-regexp -c ../check IPADDR 192.168.0.1
using predefined regular expression from base.exp
adding IPADDR='((RE:OCTET)\.){3}(RE:OCTET)'
 ('^(((1?[0-9]?[0-9]|2[0-4][0-9]|25[0-5])\.){3}(1?[0-9]?[0-9]|2[0-4][0-9]|25[0-5]))$')
'IPADDR' matches '192.168.0.1'

./i586-linux-regexp -c ../check IPADDR 192.168.0.256
using predefined regular expression from base.exp
adding IPADDR='((RE:OCTET)\.){3}(RE:OCTET)'
 ('^(((1?[0-9]?[0-9]|2[0-4][0-9]|25[0-5])\.){3}(1?[0-9]?[0-9]|2[0-4][0-9]|25[0-5]))$')
regex error 1 (No match) for value '192.168.0.256' and regexp
 '((RE:OCTET)\.){3}(RE:OCTET)'
(unknown:-1) wrong value of variable cmd_var: '192.168.0.256' (invalid ipv4 address)
\end{verbatim}
\end{example}


\subsection{Contrôle détaillé de la configuration}

    Il est parfois nécessaire d'effectuer des contrôles plus ou moins complexes.
    Exemple de choses complexe. Avoir une relation entre les paquetages ou avoir
    une condition à remplir lorsque des variables prennent certaines valeurs.
    Par ex. lors du choix d’un adaptateur-PCMCIA-ISDN dans le paquetage \og{}pcmcia\fg{}.

    Pour effectuer ces contrôles, on peut écrire dans le fichier
    \texttt{check/<PAQUETAGE>.ext} (également appelé ext-script)
	différent petit test. Ce langage se compose des éléments suivants~:

    \begin{enumerate}
    \item Mot-clé~:

      \begin{itemize}
      \item Contrôle de flux~:

        \begin{itemize}
        \item \texttt{if (\textit{expr}) then \textit{statement} else \textit{statement} fi}
        \item \texttt{foreach \textit{var} in \textit{set\_var} do \textit{statement} done}
        \item \texttt{foreach \textit{var} in \textit{set\_var\_1 ... set\_var\_n} do \textit{statement} done}
        \item \texttt{foreach \textit{var} in \textit{var\_n} do \textit{statement} done}
        \end{itemize}

      \item
        Relation~:
        \begin{itemize}
        \item \texttt{provides \textit{package} version \textit{x.y.z}}
        \item \texttt{depends on \textit{package} version \textit{x1.y1 x2.y2.z2 x3.y3 \ldots}}
        \end{itemize}

      \item Action~:
        \begin{itemize}
        \item \texttt{warning "\textit{warning}"}
        \item \texttt{error   "\textit{error}"}
        \item \texttt{fatal\_error "\textit{fatal error}"}
        \item \texttt{set \textit{var} = \textit{value}}
        \item \texttt{crypt (\textit{variable})}
        \item \texttt{stat (\textit{filename}, \textit{res})}
        \item \texttt{fgrep (\textit{filename}, \textit{regex})}
        \item \texttt{split (\textit{string}, \textit{set\_variable}, \textit{character})}
        \end{itemize}
      \end{itemize}
    \item Type de donnée~:      chaîne de caractère, nombre entier positif, numéros de version
    \item Opération logique~:    \texttt{<}, \texttt{==}, \texttt{>}, \texttt{!=}, \texttt{!}, \texttt{\&\&}, \texttt{||},
       \texttt{=}\verb+~+, \texttt{copy\_pending}, \texttt{samenet}, \texttt{subnet}
    \end{enumerate}


\marklabel{subsec:dev:data-types}{
\subsubsection{Type de données}
}

    C'est-à-dire que les types de données de la variable sont associées à
	une expression régulière et sont affectés en permanence à un type de données~:

\begin{itemize}
\item Variable dont le type commence par \og{}\var{NUM}\fg{} numérique et qui
contient un nombre entier positif
\item Variable, représentant une N-variable pour tout type de liste, elle est
également numériquement
\item Toutes les autres variables sont traitées comme des chaînes
\end{itemize}

    Cela signifie, entre autres chose, qu'une variable de type \var{ENUMERIC} ne
	peut \emph{pas} être utilisé comme un indice, lors d'un accès à une liste de variables
	même si vous avez d'abord vérifié si elle n'était pas vide. Le code suivant ne
	fonctionnera pas comme prévu~:
\begin{example}
\begin{verbatim}
    # TEST should be a variable of type ENUMERIC
    if (test != "")
    then
        # Error: You can't use a non-numeric ID in a numeric
        #         context. Check type of operand.
        set i=my_array[test]
        # Error: You can't use a non-numeric ID in a numeric
        #         context. Check type of operand.
        set j=test+2
    fi
\end{verbatim}
\end{example}

    Voici une solution à ce problème \jump{subsec:split}{\texttt{split}}~:
\begin{example}
\begin{verbatim}
    if (test != "")
    then
        # all elemente of test_% are numeric
        split(test, test_%, ' ', numeric)
        # OK
        set i=my_array[test_%[1]]
        # OK
        set j=test_%[1]+2
    fi
\end{verbatim}
\end{example}

\marklabel{subsec:dev:string-rewrite}{
\subsubsection{Substitution de chaîne et de variable}
}

    À divers moments les chaînes sont nécessaires, par ex. lorsque vous devez émettre
	\jump{subsec:dev:print}{Attention} un avertissement. Certains cas, sont décrits dans
	cette documentation, une telle chaîne est recherchée pour une variable précise, si elle
	est trouvé, elle est \emph{remplacé} par son contenu ou par d'autres attributs.
	Ce remplacement est appelé \emph{substitution de variable}.

    Cela va être illustré par un exemple. Supposons cette configuration~:

\begin{example}
\begin{verbatim}
    # config/base.txt
    HOSTNAME='fli4l'
    # config/dns_dhcp.txt
    HOST_N='1' # Nombre d'hôtes
    HOST_1_NAME='client'
    HOST_1_IP4='192.168.1.1'
\end{verbatim}
\end{example}

    Ensuite, les chaînes de caractères sont réécrits comme ceci, si la substitution
	de variable est actif dans ce contexte~:

\begin{example}
\begin{verbatim}
    "Mon routeur s'appel $HOSTNAME"
    # --> "Mon routeur s'appel fli4l"
    "HOSTNAME fait partie du paquetage %{HOSTNAME}"
    # --> "HOSTNAME fait partie du paquetage base"
    "@HOST_N est $HOST_N"
    # --> "Nombre d'hôte est 1"
\end{verbatim}
\end{example}

    Comme vous pouvez le voir, il y a essentiellement trois options pour le remplacement~:
    \begin{itemize}
    \item \texttt{\$<Name>} ou bien \texttt{\$\{<Name>\}}~: remplace le nom de
	la variable par le contenu de la variable. C'est la forme la plus courante de
	substitution. Le nom doit être enfermé dans \texttt{\{...\}} si la chaîne est
	directement suivie d'un caractère qui peut être une partie valide d'un nom de variable
	soit une lettre, un chiffre ou un tire soulignement. Dans tous les autres cas,
	l'utilisation des accolades est possible, mais pas obligatoire.

    \item \texttt{\%<Name>} ou bien \texttt{\%\{<Name>\}}~: remplace le nom de
	la variable par le nom du paquetage, dans laquel la variable est définie. Cela
	ne fonctionne \emph{pas} dans le script via la variable affectées
	\jump{subsec:dev:assignment}{\texttt{set}} ou la variable de contrôle de boucle
	\jump{subsec:dev:control}{\texttt{foreach}}, étant donné que ces variables ne sont
	pas dans un paquetage, leur syntaxe est différent.

    \item \texttt{@<Name>} ou bien \texttt{@\{<Name>\}}~: remplace le nom de
	la variable par le commentaire qui est après la variable dans la configuration.
	Encore une fois, cela n'a pas de sens pour les variables définies par le script.
    \end{itemize}

    \textbf{Remarque~:} Les éléments de liste de variables ne peuvent \emph{pas}
	être intégré dans des chaînes de caractères de cette façon, parce qu'il n'y a
	aucune possibilité de fournir un index.

    En général, seule une \emph{constante} peut être utilisé pour la substitution
	de variable, les chaînes qui proviennet d'une variable restent inchangées.
	Un exemple permettra de clarifier cela - voici la configuration suivante~:

\begin{example}
\begin{verbatim}
    HOSTNAME='fli4l'
    TEST='${HOSTNAME}'
\end{verbatim}
\end{example}

    Ensuite, le code~:

\begin{example}
\begin{verbatim}
    warning "${TEST}"
\end{verbatim}
\end{example}

    Produit la sortie suivante~:

\begin{example}
\begin{verbatim}
    Warning: ${HOSTNAME}
\end{verbatim}
\end{example}

    Et \emph{pas} la sortie~:

\begin{example}
\begin{verbatim}
    Warning: fli4l
\end{verbatim}
\end{example}

    Dans les sections suivantes, on notera explicitement les conditions des
	chaînes qui font l'objet d'une substitution de variable.


\subsubsection{Définition d'un service avec un numéro de version
    associé~: \texttt{provides}}

    Cela permet par exemple un OPT peut déclarer et mettre à disposition
	un serveur d'impression ou un service Web. Un seul paquetage, fourni un certain
	service. Cela empêche par exemple d'installer deux serveurs web en parallèle, cela
	ne fonctionnerais pas pour des raisons évidentes, puisque les deux serveurs utiliseraient
	le port 80. En outre, la version actuelle du service est prévu pour est mise à jour
	régulièrement. Le numéro de version se compose de deux ou trois nombres
	séparés par des points, comme \og{}4.0\fg{} ou \og{}2.1.23\fg{}.

    Les services proviennent généralement de l'OPT, pas du paquetage. Par exemple, dans
	le paquetage \og{}tools\fg{} il y a une série de programmes, qui ont chacun leur propre
	instruction \texttt{provides} elle sera activée via \verb+OPT_...='yes'+.

    La syntaxe est~:

\begin{example}
\begin{verbatim}
    provides <Name> version <Version>
\end{verbatim}
\end{example}

    Exemple avec le paquetage \og{}easycron\fg{}:

\begin{example}
\begin{verbatim}
    provides cron version 3.10.0
\end{verbatim}
\end{example}

    Le numéro de version doit être incrémenté par le développeur de l'OPT dans le troisième
	volet, si les fonctionnalitées ont été seulement améliorations et que l'interface est
	toujours compatible avec l'OPT. Le numéro de version doit être augmentée avec le premier
	ou le deuxième chiffre, si l'interface a été modifiée et est incompatible (par exemple,
	en raison de variables renommées, des chemins changés, manquant ou le programme de service
	a été renommés, etc.).


\subsubsection{Définition une dépendance à un service avec une
    Version spécifique~: \texttt{depends}}

	Si un autre service est nécessaire pour assurer la fonction de votre propre service
	(par exemple un serveur web), on peut définir une dépendance par une version spécifique
	pour le service. La version peut être indiquée par deux (par ex. \og{}2.1\fg{}) ou trois
	chiffres (par ex. \og{}2.1.11\fg{}), la version à deux chiffres accepte toutes les versions
	à partir de ce nombre et la version à trois chiffres accepte seulement la version spécifié.
	En outre, vous pouvez spécifier une liste de numéros de version si plusieurs versions
	du service est compatible avec le paquetage.

    La syntaxe est la suivante~:

\begin{example}
\begin{verbatim}
    depends on <Name> version <Version>+
\end{verbatim}
\end{example}

    Exemple~: le paquetage \og{}serveur\fg{} contient~:
\begin{example}
\begin{verbatim}
    provides server version 1.0.1
\end{verbatim}
\end{example}

    Avec le paquetage \og{}client\fg{}. vous indiquez l'instruction \texttt{depends}
	dans cette exemple \footnote{bien sûr, un seul à la fois~!}

\begin{example}
\begin{verbatim}
    depends on server version 1.0       # OK, '1.0' s'adapte '1.0.1'
    depends on server version 1.0.1     # OK, '1.0.1' s'adapte '1.0.1'
    depends on server version 1.0.2     # Erreur, '1.0.2' s'adapte pas '1.0.1'
    depends on server version 1.1       # Erreur, '1.1' s'adapte pas '1.0.1'
    depends on server version 1.0 1.1   # OK, '1.0' s'adapte '1.0.1'
    depends on server version 1.0.2 1.1 # Erreur, ni '1.0.2' ni '1.1' ne s'adapte '1.0.1'
\end{verbatim}
\end{example}


\marklabel{subsec:dev:print}{
\subsubsection{Message pour l'utilisateur~: \texttt{warning}, \texttt{error}, \texttt{fatal\_error}}
}

    Avec l'aide de ces trois fonctions, on peut avertir et signaler
    l’utilisateur, d’une erreur ou interrompre immédiatement le traitement
    de contrôle. La syntaxe est la suivante~:

    \begin{itemize}
    \item \verb+warning "text"+
    \item \verb+error "text"+
    \item \verb+fatal_error "text"+
    \end{itemize}

    Toutes les chaînes utilisées avec ces fonctions sont soumis à une
    \jump{subsec:dev:string-rewrite}{substitution de variable}.


\marklabel{subsec:dev:assignment}{
\subsubsection{Affectation}
}

    Si vous avez besoin d’utiliser une variable temporaire pour une raison
    quelconque, vous pouvez la créer avec \og{}\texttt{set var [= value]}\fg{}.
	\emph{La variable ne peut pas être une variable de configuration~!} \footnote{C'est
	une restriction souhaitée~: le contrôle du skript ne sera \emph{pas} en mesure de
	modifier la configuration de l'utilisateur.} Si vous omettez \og{}= value\fg{} la
	variable sera alors simplement placée sur \og{}yes\fg{}, ensuite vous pouvez tester
	facilement la variable avec l’instruction \texttt{if}. Vous pouvez spécifier les
	données suivant après le signe égal, variable normale, variable indexée, nombre,
	chaîne de caractère, versions.

    Il convient de noter de par l'assignation le \emph{type} sera défini dans la variable
	temporaire. Si un numéro est attribué \var{mkfli4l} se \og{}souviendra\fg{} que la
	variable contient un numéro et permettra plus tard de faire un calcul avec celui-ci.
	Si vous essayez de faire un calcul avec une variable de type différent, il échouera.
	Exemple~:

\begin{example}
\begin{verbatim}
    set i=1   # OK, i est une variable numérique
    set j=i+1 # OK, j est une variable numérique et contient la valeur 2
    set i="1" # OK, i est maintenant une variable de chaîne
    set j=i+1 # Erreur "Vous ne pouvez pas utiliser un ID non numérique dans un
              #         contexte numérique. Vérifiez le type de l'opérande."
              # --> Pas de calculs avec des chaînes~!
\end{verbatim}
\end{example}

    Vous pouvez également créer des listes temporaires (voir ci-dessous). Exemple~:

\begin{example}
\begin{verbatim}
    set prim_%[1]=2
    set prim_%[2]=3
    set prim_%[3]=5
    warning "${prim_n}"
\end{verbatim}
\end{example}

    Le nombre de liste d'éléments est géré par \var{mkfli4l} et par la variable
	\var{prim\_n}. Le code ci-dessus conduit donc à la sortie suivante~:

\begin{example}
\begin{verbatim}
    Warning: 3
\end{verbatim}
\end{example}

    Si le côté droit de la cession est une constante de chaîne, elle est soumise à une
	\jump{subsec:dev:string-rewrite}{substitution de variable}. L'exemple
	suivant montre le code~:

\begin{example}
\begin{verbatim}
    set s="a"
    set v1="$s" # v1="a"
    set s="b"
    set v2="$s" # v2="b"
    if (v1 == v2)
    then
      warning "égal"
    else
      warning "pas égal"
    fi
\end{verbatim}
\end{example}

    La sortie produite n'est \og{}pas égale\fg{}, parce que les variables \var{v1} et
    \var{v2} sont remplacées par le contenu de la variable \var{s} déjà en cours de cession.

    \textbf{Remarque~:} Un ensemble de variables dans un script est visible lors du traitement
	des autres scripts~- actuellement il n'existe pas de principe de localisation pour ces
	variables introduites. L'ordre dans lequel les scripts sont traitées dans divers paquetages,
	n'est pas définie, vous ne devez pas compter sur une variable ayant des valeurs définies
	dans un autre paquetage.


\subsubsection{Liste (ou Tableau)}

    Si vous voulez accéder aux éléments d'une \%-variable (ou de la liste),
	vous devez utiliser le nom original de la variable comme mentionné dans le fichier
	\texttt{check/<PAQUETAGE>.txt} et d'ajouter un index pour chaque signe \og{}\%\fg{}
	en utilisant \og{}[\emph{Index}]\fg{}.

    Exemple~: Si vous voulez accéder aux éléments de la variable \var{PF\_USR\_CHAIN\_\%\_RULE\_\%}
	vous avez besoin de deux index car la variable a deux signes \og{}\%\fg{}. Tous
	les éléments peuvent être enregistrés par exemple en utilisant la (boucle \texttt{foreach}
	et \jump{subsec:dev:control}{voir ci-dessous})~:

\begin{example}
\begin{verbatim}
    foreach i in pf_usr_chain_n
    do
        # un seul index nécessaire, seul un '%' dans la variable
        set j_n=pf_usr_chain_%_rule_n[i]
        # Attention: a
        # foreach j in pf_usr_chain_%_rule_n[i]
        # n'est pas possible, d'où l'utilisation de j_n~!
        foreach j in j_n
        do
            # deux index nécessaires, deux '%' dans la variable
            set rule=pf_usr_chain_%_rule_%[i][j]
            warning "Rule $i/$j: ${rule}"
        done
    done
\end{verbatim}
\end{example}

    Exemple de configuration

\begin{example}
\begin{verbatim}
    PF_USR_CHAIN_N='2'
    PF_USR_CHAIN_1_NAME='usr-chain_a'
    PF_USR_CHAIN_1_RULE_N='2'
    PF_USR_CHAIN_1_RULE_1='ACCEPT'
    PF_USR_CHAIN_1_RULE_2='REJECT'
    PF_USR_CHAIN_2_NAME='usr-chain_b'
    PF_USR_CHAIN_2_RULE_N='1'
    PF_USR_CHAIN_2_RULE_1='DROP'
\end{verbatim}
\end{example}

    la sortie suivante est imprimé~:

\begin{example}
\begin{verbatim}
    Warning: Rule 1/1: ACCEPT
    Warning: Rule 1/2: REJECT
    Warning: Rule 2/1: DROP
\end{verbatim}
\end{example}

    Alternativement, vous pouvez parcourir directement toutes les valeurs du tableau,
	mais les indices exacts ne sont pas toujours connus (car se n'est pas nécessaire)~:

\begin{example}
\begin{verbatim}
    foreach rule in pf_usr_chain_%_rule_%
    do
        warning "Rule %{rule}='${rule}'"
    done
\end{verbatim}
\end{example}

    Cela produit la sortie suivante avec l'exemple de configuration à partir des information
	ci-dessus~:

\begin{example}
\begin{verbatim}
    Warning: Rule PF_USR_CHAIN_1_RULE_1='ACCEPT'
    Warning: Rule PF_USR_CHAIN_1_RULE_2='REJECT'
    Warning: Rule PF_USR_CHAIN_2_RULE_1='DROP'
\end{verbatim}
\end{example}

    Le deuxième exemple montre bien le sens de la syntaxe \texttt{\%{<Name>}}~:
	la chaîne \texttt{\%{rule}} est substitué par le \emph{nom} de la variable en
	question (par exemple \var{PF\_USR\_CHAIN\_1\_RULE\_1}), tandis que \texttt{\${rule}}
	est substitué par son \emph{contenu} (exemple \var{ACCEPT}).


\subsubsection{Crypter un mot de passe~: \texttt{crypt}}

Dans le fichier \texttt{rc.cfg} certaines variables contiennent des mots de passe qui
ne doivent pas apparaitre en clair. Ces variables peuvent être cryptées en
utilisant \texttt{crypt} et seront transférées dans un format qui est également utilisé
par le routeur. On utilise pour cela~:

\begin{example}
\begin{verbatim}
    crypt (<variable>)
\end{verbatim}
\end{example}

La fonction \texttt{crypt} est \emph{seulement} un endroit où la variable de configuration
peut être modifiée.


\marklabel{subsec:statdatei}{
\subsubsection{Contrôle des propriétés d'un fichier~: \texttt{stat}}
}

    \texttt{stat} vous permet de rechercher les propriétés d'un fichier. Il indique
    pour l'instant que la taille du fichier. Si vous souhaitez tester des fichiers
	de configuration dans le répertoire courant, vous pouvez utiliser la variable
	interne \var{config\_dir}. La syntaxe est~:

\begin{example}
\begin{verbatim}
    stat (<nom de fichier>, <clé>)
\end{verbatim}
\end{example}

    La commande ressemble à ceci (les paramètres utilisés ne sont que des exemples)~:

\begin{example}
\begin{verbatim}
    foreach i in openvpn_%_secret
    do
       stat("${config_dir}/etc/openvpn/$i.secret", keyfile)
       if (keyfile_res != "OK")
       then
          error "OpenVPN: missing secretfile <config>/etc/openvpn/$i.secret"
       fi
    done
\end{verbatim}
\end{example}

	L'exemple suivant vérifie si un fichier existe dans le répertoire de configuration
	actuelle. Si la variable \verb+OPENVPN_1_SECRET='test'+ est définie dans le fichier
	de configuration, lors du premier contrôle d'exécution la boucle vérifira l'existence du fichier
	\texttt{etc/openvpn/test.secret} dans le répertoire de configuration actuelle.

    Après l'exécution deux variables sont définies~:

    \begin{itemize}
    \item \texttt{<Clé>\_res}~:
    Résultat après l'exécution système \texttt{stat} (\og{}OK\fg{}, si l'exécution système
	est réussi, sinon il y aura message d'erreur de l'exécution système)
    \item \texttt{<Clé>\_size}~:
    taille du fichier
    \end{itemize}

    Cela pourrait alors ressembler à ceci~:

\begin{example}
\begin{verbatim}
    stat ("unix/Makefile", test)
    if ("$test_res" == "OK")
    then
            warning "test_size = $test_size"
    else
            error "Error '$test_res' while trying to get size of 'unix/Makefile'"
    fi
\end{verbatim}
\end{example}

    Un nom de fichier passé comme une constante de la chaîne est soumis à une
	\jump{subsec:dev:string-rewrite}{substitution de variable}.


\marklabel{subsec:fgrepdatei}{
\subsubsection{Rechercher dans les fichiers~: \texttt{fgrep}}
}

    Si vous voulez rechercher un fichier avec \og{}grep\fg{},
    \footnote{\og{}grep\fg{} est une commande du système d'exploitation Unix,
	pour les flux de texte de filtrage.} vous avez aussi la possibilité d'utiliser
	la commande \texttt{fgrep}, la syntaxe est~:

\begin{example}
\begin{verbatim}
    fgrep (<nomdefichier>, <regex>)
\end{verbatim}
\end{example}

    Si le fichier \texttt{<nom de fichier>} n'existe pas alors \var{mkfli4l} s’arrête
	et affiche une erreur fatale~! Si vous n'êtes pas sûr que le fichier est toujours
	présent, vous devez avant utiliser la commande \texttt{stat} pour savoir si le
	\texttt{<nom de fichier>} existe. Après avoir exécuté \texttt{fgrep} le résultat de
	recherche sera présent dans un tableau et \var{FGREP\_MATCH\_\%} sera disponible, avec
	l'indice \emph{x} comme d'habitude vous allez avoir \var{FGREP\_MATCH\_N}.
	\var{FGREP\_MATCH\_1} fait référence à toute les lignes l'expression régulière
	correspondantes, tandis que \var{FGREP\_MATCH\_2} pour \var{FGREP\_MATCH\_N}
	contient la nième \emph{n-1} partie entre parenthèses.

    Un premier exemple simple vous montrer comment l'utiliser. Le fichier
	\texttt{opt/etc/shells} contient la ligne~:

\begin{example}
\begin{verbatim}
/bin/sh
\end{verbatim}
\end{example}

    Le code est le suivant

\begin{example}
\begin{verbatim}
    fgrep("opt/etc/shells", "^/(.)(.*)/")
    foreach v in FGREP_MATCH_%
    do
      warning "%v='$v'"
    done
\end{verbatim}
\end{example}

    Produit la sortie suivante~:

\begin{example}
\begin{verbatim}
    Warning: FGREP_MATCH_1='/bin/'
    Warning: FGREP_MATCH_2='b'
    Warning: FGREP_MATCH_3='in'
\end{verbatim}
\end{example}

	Le RegEx correspond (seulement) à \og{}/bin/\fg{}, (seul) cette partie de ligne est
	contenue dans la variable \var{FGREP\_MATCH\_1}. La première partie entre les paranthéses
	de l'expression correspond au premier caractère après le premier signe \og{}/\fg{},
	c'est pourquoi \og{}b\fg{} est contenu dans \var{FGREP\_MATCH\_2}. La deuxième partie
	restante contient entre les paranthéses \og{}b\fg{} jusqu'au dernière signe \og{}/\fg{},
	donc \og{}in\fg{} est dans la variable \var{FGREP\_MATCH\_3}.

    Le deuxième exemple suivant vous montrer une utilisation pratique de \texttt{fgrep}
	avec le fichier \texttt{check/base.ext}. Il sera testé si toutes les références
	\texttt{tmpl:} indiquées, sont vraiment présent dans \var{PF\_FORWARD\_x}~:

\begin{example}
\begin{verbatim}
    foreach n in pf_forward_n
    do
      set rule=pf_forward_%[n]
      if (rule =~ "tmpl:([^[:space:]]+)")
      then
        foreach m in match_%
        do
          stat("$config_dir/etc/fwrules.tmpl/$m", tmplfile)
          if(tmplfile_res == "OK")
          then
            add_to_opt "etc/fwrules.tmpl/$m"
          else
            stat("opt/etc/fwrules.tmpl/$m", tmplfile)
            if(tmplfile_res == "OK")
            then
              add_to_opt "etc/fwrules.tmpl/$m"
            else
              fgrep("opt/etc/fwrules.tmpl/templates", "^$m[[:space:]]+")
              if (fgrep_match_n == 0)
              then
                error "Can't find tmpl:$m for PF_FORWARD_${n}='$rule'!"
              fi
            fi
          fi
        done
      fi
    done
\end{verbatim}
\end{example}

    La valeur du nom de fichier ainsi que l'expression régulière passée comme une constante
	de chaîne sont soumis à une \jump{subsec:dev:string-rewrite}{substitution de variable}.


\marklabel{subsec:split}{
\subsubsection{Désassemblage des paramètres~: \texttt{split}}
}

    Souvent plusieurs paramètres sont appliqués dans une variable, ensuite
    dans le script de démarrage ces paramètres sont désassemblés séparément.
    Si vous voulez effectuer des tests lors du désassemblage,
    c’est la commande \texttt{split} qu’il vous faut. La syntaxe est~:

\begin{example}
\begin{verbatim}
    split (<Chaîne>, <Liste>, <Séparateur>)
\end{verbatim}
\end{example}

    Une chaîne peut être spécifié par une variable ou directement en tant que constante.
	\var{mkfli4l} décompose la chaîne là ou apparaît un séparateur et génère un élément
	 pour chaque partie de la liste. Vous pouvez parcourir ces éléments plus tard et
	 effectuer des tests. Si rien n'est trouvé entre deux séparateurs un élément de la
	 liste est générée comme une valeur pour une chaîne vide. L'exception \og{} \fg{} est~:
	 tous les espaces sont supprimés et aucune variable vide n'est créé.

    Lors de la décomposition des éléments, si un contexte numérique apparait
    dans la variable (par exemple sous forme d'indice) cela doit être précisé dans
	la commande \texttt{split}. Vous devez ajouter un attribut supplémentaire 'numéric'.
	Cette exécution se présente comme ceci~:

\begin{example}
\begin{verbatim}
    split (<Chaîne>, <Liste>, <Séparateur>, numeric)
\end{verbatim}
\end{example}

   Voici un exemple~:

\begin{example}
\begin{verbatim}
    set bar="1.2.3.4"
    split (bar, tmp_%, '.', numeric)
    foreach i in tmp_%
    do
            warning "%i = $i"
    done
\end{verbatim}
\end{example}

    Produit la sortie suivante~:

\begin{example}
\begin{verbatim}
    Warning: TMP_1 = 1
    Warning: TMP_2 = 2
    Warning: TMP_3 = 3
    Warning: TMP_4 = 4
\end{verbatim}
\end{example}

    \textbf{Remarque~:} si vous utilisez une variable \og{}numeric\fg{} \var{mkfli4l} ne
	vérifira \emph{pas} les parties de chaîne générés si elle ne sont pas vraiment numérique~!
	Si vous utilisez une telle construction dans un contexte numérique \var{mkfli4l}
	déclenchera une erreur fatale si une telle variable n'est pas numérique. Exemple~:

\begin{example}
\begin{verbatim}
    set bar="a.b.c.d"
    split (bar, tmp_%, '.', numeric)
    # Fehler: invalid number 'a'
    set i=tmp_%[1]+1
\end{verbatim}
\end{example}

    Une valeur utilisée dans le première paramètre de la constant de chaîne est soumis à une
	\jump{subsec:dev:string-rewrite}{substitution de variable}.


\marklabel{subsec:addtoopt}{
\subsubsection{Ajouter des fichiers dans l'archive~: \texttt{add\_to\_opt}}
}

   Avec la fonction \texttt{add\_to\_opt} des fichiers supplémentaires peuvent
   être ajoutés dans une archive OPT ou dans le RootFS. Il est possible de sélectionner
   \emph{tous} les fichiers du sous-répertoire \texttt{opt/} ou à partir du répertoire
   de configuration. Il n'y a pas de restriction sur l’ajout de fichiers dans un paquetage.
   Si un fichier doit être à la fois dans \texttt{opt/} et dans le répertoire de configuration,
   \texttt{add\_to\_opt} choisira de copie les fichiers dans le répertoire de configuration.
   La fonction \texttt{add\_to\_opt} est complexe et est en règle général logique, la fonction
   décide quel fichier sera copié en premier dans l'archive.

   La syntaxe est la suivante~:

\begin{example}
\begin{verbatim}
    add_to_opt <nom de fichier> [<Flags>]
\end{verbatim}
\end{example}

   Le Flags est optionnel. Les valeurs par défaut de la table \ref{table:options}
   sont utilisés si aucun Flags n'est indiqué.

   Ci-après un exemple à partir du paquetage \og{}sshd\fg{}~:

\begin{example}
\begin{verbatim}
    if (opt_sshd)
    then
       foreach pkf in sshd_public_keyfile_%
       do
         stat("$config_dir/etc/ssh/$pkf", publickeyfile)
         if(publickeyfile_res == "OK")
         then
             add_to_opt "etc/ssh/$pkf" "mode=400 flags=utxt"
         else
             error "sshd: missing public keyfile %pkf=$pkf"
         fi
       done
    fi
\end{verbatim}
\end{example}

   Utiliser d'abord \jump{subsec:statdatei}{\texttt{stat}} pour vérifier si le
   fichier existe bien dans le répertoire config. Si le fichier existe, il sera
   ajouté à l'archive, sinon \var{mkfli4l} renvoie un message d'erreur.

   \textbf{Remarque~:} \var{mkfli4l} \jump{subsec:konfigspezdatei}{vérifie} aussi
   avec la fonction \texttt{add\_to\_opt} si le fichier à copier, se trouve bien
   dans le répertoire config.

   Les noms de fichiers et des Flags qui sont utilisés en tant que constantes de chaîne
   sont soumis à une \jump{subsec:dev:string-rewrite}{substitution de variable}.


\marklabel{subsec:dev:control}{
\subsubsection{Contrôle de flux}
}

\begin{example}
\begin{verbatim}
    if (expr)
    then
            statement
    else
            statement
    fi
\end{verbatim}
\end{example}

    Un cas classique de restinction, comme nous connaissons. Si la condition est vraie,
	alors, l’expression \texttt{then} est exécutée, si la condition est fausse, l’expression
	\texttt{else} sera exécutée.

    Si vous voulez effectuer des tests de \%-variable, il faudra tester chaque
    variable. Pour éviter cela, il y a la boucle \texttt{foreach} en deux variantes.

    \begin{enumerate}
    \item Itération sur une liste de variables~:

\begin{example}
\begin{verbatim}
    foreach <contrôle variable> in <Liste-Variable>
    do
            <instruction>
    done

    foreach <contrôle variable> in <liste-Variable-1> <liste-Variable-2> ...
    do
            <instruction>
    done
\end{verbatim}
\end{example}

    Cette boucle parcourt tous le tableau de variables spécifié, en commençant par
	le premier et le dernier élément, le nombre d'éléments de cette liste est extraites
	du N-variable associée à ce tableau. La contrôle variable prend les valeurs des
	variables du tableau respectifs. Il est à noter vous pouvez ajouter un processus optionnel
	pour les variables du tableau si une valeur n'est pas présents dans la configuration, un
	élément vide sera généré. Vous pourriez en tenir compte dans le script, par exemple
	comme ceci~:

\begin{example}
\begin{verbatim}
    foreach i in template_var_opt_%
    do
        if (i != "")
        then
            warning "%i is present (%i='$i')"
        else
            warning "%i is undefined (empty)"
        fi
    done
\end{verbatim}
\end{example}

    Comme vous pouvez le voir dans l'exemple, le \emph{nom} des variables du tableau
	respectives peut être déterminée avec l'instruction \texttt{\%<contrôle variable>}.

    L'instruction dans la boucle ci-dessus peut être l'un des éléments de contrôle ou
	de fonctions (\texttt{if}, \texttt{foreach}, \texttt{provides}, \texttt{depends},
	\ldots).

    Si vous souhaitez accéder exactement à un élément du tableau, vous pouvez y remédier
	en utilisant la syntaxe \texttt{<Liste>[<Index>]}. L'index peut être une variable normale,
	une constante numérique ou encore un tableau indexé.

    \item Itération sur N-variables~:

\begin{example}
\begin{verbatim}
    foreach <contrôle variable> in <N-Variable>
    do
            <instruction>
    done
\end{verbatim}
\end{example}

    Cette boucle s'exécute de 1 à la valeur qui est indiqué dans la N-variable.
	Vous pouvez utiliser contrôle variable dans un tableau de variables indexé.
	Donc, si vous voulez parcourir non seulement un tableau de variable, mais
	plusieurs tableaux de variables en même temps, tous contrôlés par la \emph{même}
	N-variable, vous prenez la variante de boucle et vous utilisez le contrôle variable
	pour l'indexation de plusieurs tableaux de variables. Exemple~:

\begin{example}
\begin{verbatim}
    foreach i in host_n
    do
        set name=host_%_name[i]
        set ip4=host_%_ip4[i]
        warning "$i: name=$name ip4=$ip4"
    done
\end{verbatim}
\end{example}

    Le résultat du contenu du tableua de la liste \var{HOST\_\%\_NAME} et de la liste
	\var{HOST\_\%\_IP4} pour cet exemple~:

\begin{example}
\begin{verbatim}
    Warning: 1: name=berry ip4=192.168.11.226
    Warning: 2: name=fence ip4=192.168.11.254
    Warning: 3: name=sandbox ip4=192.168.12.254
\end{verbatim}
\end{example}

    \end{enumerate}


    \subsubsection{Expressions}

    Une expression est liée à une valeur et un opérateur à une autre valeur.
	Cette valeur peut être une variable normale, un élément d'un tableau, ou une
	constante (nombre, chaîne de caractère ou numéro de version). Toutes les constantes
	de chaîne dans les expressions sont soumises à une
	\jump{subsec:dev:string-rewrite}{substitution de variable}.

    Les opérateurs vous permettent de fait à peu près tout ce que vous
    attendez d'un language de programmation. Un test pour l'égalité de deux
    variables pourrait donc ressembler à ceci~:

\begin{example}
\begin{verbatim}
    var1 == var2
    "$var1" == "$var"
\end{verbatim}
\end{example}

    À noter, que la comparaison selon le type de variables se li dans le fichier
	\texttt{check/<PAQUETAGE>.txt} où ils ont été créés. Si l'une des deux variables est
	\jump{subsec:dev:data-types}{numérique}, la comparaison se fait sur une base numérique,
	c'est-à-dire que les chaînes de caractère sont converties en nombres, puis comparées.
	La comparaison est fondée par une chaîne, si on compare \texttt{"05"\ == "5"}
	cela donne un résultat \og{}faux\fg{}, une comparaison \texttt{"18"\ < "9"} donne \og{}vrai\fg{}
	selon l'ordre lexicographique des chaînes de caractère~: le chiffre \og{}1\fg{} précède
	le chiffre \og{}9\fg{} dans le jeu de caractères ASCII.

    Pour introduire la comparaison des numéros de version de construction vous devez utiliser
	\texttt{numeric(version)}, cela génère la valeur numérique du numéro de version à des fins
	de comparaison. Ici on applique~:

\begin{example}
\begin{verbatim}
    numeric(version) := major * 10000 + minor * 1000 + sub
\end{verbatim}
\end{example}

    \og{}major\fg{} représente la première partie, \og{}minor\fg{} la deuxième partie et \og{}sub\fg{}
	la troisième partie du numéro de version. Si \og{}sub\fg{} est manquant le nombre dans l'addition
	ci-dessus sera omis (en d'autres termes \og{}sub\fg{} sera égale à zéro).

    Vous trouverez dans le tableau \ref{tab:expr} une liste complète de toutes les expressions.
	\og{}val\fg{} indique une valeur de n'importe quel type, \og{}number\fg{} une valeur numérique
	et \og{}string\fg{} une chaîne de caractère.

  \begin{table}[htb]
      \centering
      \caption{Expressions logiques}
      \label{tab:expr}
      \begin{tabular}{ll}
        \hline
        expression &                     vraie si\\
        \hline
        \hline
       id                    &  id == \og{}yes\fg{}\\
       val  == val           &  valeurs de type identique sont égaux\\
       val  != val           &  valeurs de type identique sont inégaux\\
       val  == number        &  numérique la valeur de val == nombre\\
       val  != number        &  numérique la valeur de val != nombre\\
       val  $<$  number      &  numérique la valeur de val $<$  number\\
       val  $>$  number      &  numérique la valeur de val $>$  number\\
       val  == version       &  numeric(val) == numeric(version)\\
       val  $<$  version     &  numeric(val) $<$  numeric(version)\\
       val  $>$  version     &  numeric(val) $>$  numeric(version)\\
       val  =\verb?~? string &  val correspond à une chaîne expression régulière\\
       ( expr )              &  l'expression entre les parenthèses est vraie\\
       expr \&\& expr        &  les deux expressions sont vraies\\
       expr || expr          &  au moins une des deux expressions est vraie\\
       copy\_pending(id)     &  voir la description ci-dessous\\
       samenet (string1, string2) & chaîne1 décrit le même réseau que la chaîne2\\
       subnet (string1, string2)  & chaîne1 décrit un sous-réseau de chaîne2\\
        \hline
      \end{tabular}
  \end{table}


\subsubsection{Opérateur \og{}match\fg{}}

Avec l'opérateur-match \verb?=~? vous pouvez vérifier, si une expression
régulière correspond à la valeur d'une variable. En outre,
l'opérateur peut également l’utiliser pour extraire une partie de l’expression
de la variable. Après avoir appliqué avec succès une expression régulière à
une variable, une table \var{MATCH\_\%} contiendra toutes les parties trouvées de
la variable. Cela pourrait par exemple ressembler à ceci~:

\begin{example}
\begin{verbatim}
    set foo="foobar12"
    if ( foo =~ "(foo)(bar)([0-9]*)" )
    then
            foreach i in match_%
            do
                    warning "match %i: $i"
            done
    fi
\end{verbatim}
\end{example}

Après une exécution de \var{mkfli4l}, cela conduira vers la sortie suivante~:

\begin{example}
\begin{verbatim}
    Warning: match MATCH_1: foo
    Warning: match MATCH_2: bar
    Warning: match MATCH_3: 12
\end{verbatim}
\end{example}

  Lors de l'utilisation de \verb?=~? on peut faire référence à toutes les expressions
  régulières existantes. Si l'on veut par exemple vérifier, si le pilote de la carte
  Ethernet-PCMCIA a été sélectionné, sans que la variable \var{OPT\_PCMCIA} soit sur le
  paramétrée sur \og{}yes\fg{}, vous pouvez écrire~:

\begin{example}
\begin{verbatim}
    if (!opt_pcmcia)
    then
        foreach i in net_drv_%
        do
           if (i =~ "^(RE:PCMCIA_NET_DRV)$")
           then
               error "If you want to use ..."
           fi
        done
    fi
\end{verbatim}
\end{example}

Comme démontré dans l'exemple, il est important \emph{d'ancrer} l'expression régulière avec
\texttt{\^} et \texttt{\$}, si vous avez l'intention d'appliquer une expression dans la variable
\emph{entière}. Sinon, l'expression renvoi une valeur \og{}vrai\fg{} si une \emph{partie} de
la variable est couverte par l'expression régulière, ce qui n'est certainement pas souhaitable
dans ce cas.


\subsubsection{Vérifier le fichier copié, en fonction de la valeur d'une variable~:
    \texttt{copy\_pending}}

        Avec le processus de vérification \texttt{copy\_pending} vous pouvez vérifier
        si le fichier a été copié ou non en fonction de la valeur d'une variable.
        On peut l’utiliser, par exemple pour tester un pilote spécifié par
        l'utilisateur, pour savoir s’il existe bien et s’il a bien été copié.
        \texttt{copy\_pending} accepte les noms à tester, sous forme d'une variable ou
        d'une chaîne. \footnote{Comme décrit dans la chaîne des objet de substitution de variable,
		c'est à dire via la \jump{subsec:dev:control}{boucle \texttt{foreach}} et le
		\jump{subsec:dev:string-rewrite}{ \texttt{\%<Name>} de substitution} tous les éléments
		de la liste (ou un tableau) peuvent être examinées.} \texttt{copy\_pending} vérifie si

        \begin{itemize}
        \item la variable est active (si l'opt est chargé, la variable-OPT doit
           être placé sur \og{}yes\fg{}),

         \item la variable a été référencée dans le fichier \texttt{opt/<PAQUETAGE>.txt}, et si

         \item le fichier a été copié en fonction de la valeur indiquée.
        \end{itemize}

        la fonction \texttt{copy\_pending} renverra \og{}vrai\fg{}, s'il ne détecte emph{pas}
		le fichier a copier lors de la dernière étape, le processus de copie sera donc
		(encore en \og{}attente\fg{}).

    Vous pouvez trouver un petit exemple de l'utilisation de toutes ces
    fonctions dans le fichier \texttt{check/base.ext}~:

\begin{example}
\begin{verbatim}
    foreach i in net_drv_%
    do
        if (copy_pending("%i"))
        then
            error "No network driver found for %i='$i', check config/base.txt"
        fi
    done
\end{verbatim}
\end{example}

    Tous les éléments de la liste \var{NET\_DRV\_\%} seront détectés pour lesquelles aucune copie n'a été
	faite car il n'existe pas de configuration correspondante dans le fichier \texttt{opt/base.txt}.

\subsubsection{Comparer des adresses réseaux avec~: \texttt{samenet} et
  \texttt{subnet}}

Pour vérifier la communication entre les réseaux, nous avons besoin d'un test,
pour savoir si les deux réseaux sont identiques ou si l'un des deux est un
sous-réseau donc différent. Pour cela, vous avez deux fonctions \texttt{samenet} et
\texttt{subnet} qui vous permez de vérifier le réseau.

\begin{example}
\begin{verbatim}
    samenet (netz1, netz2)
\end{verbatim}
\end{example}

Le retour est \og{}vrai\fg{}, si les deux réseaux sont identiques, et

\begin{example}
\begin{verbatim}
    subnet (netz1, netz2)
\end{verbatim}
\end{example}

Le retour est \og{}vrai\fg{}, si \og{}netz1\fg{} est un sous-réseau de \og{}netz2\fg{}.


\subsubsection{Extension du Kernel par ligne de commande}

Obligatoire pour certain OPT, elle est utilisée pour rajouter d'autres paramètres
dans le Kernel lors du Boot, on peut contrôler la variable \var{KERNEL\_BOOT\_OPTION},
pour savoir si les valeurs ont bien été incluses et au cas échéant un message d'avertissement
ou d'erreur sera envoyé. Maintenant avec la variable interne \var{KERNEL\_BOOT\_OPTION\_EXT}
vous pouvez ajouter une option nécessaire mais manquante directement dans le script-ext.
Un exemple tiré du fichier \texttt{check/base.ext}~:

\begin{example}
\begin{verbatim}
    if (powermanagement =~ "apm.*|none")
    then
        if ( ! kernel_boot_option =~ "acpi=off")
        then
            set kernel_boot_option_ext="${kernel_boot_option_ext} acpi=off"
        fi
    fi
\end{verbatim}
\end{example}

Le paramétre \og{}acpi=off\fg{} sera transmis au Kernel si aucune gestion d'alimentation ou aucun
type \og{}d'APM\fg{} n'est nécessaire.


\subsection{Supporte différent choix de version du Kernel}

Les différents choix de version du Kernel se distinguent souvent de quelques détails~:

\begin{itemize}
\item La modification des pilotes disponible, certains ont disparu,
   d'autres ont été ajoutés.
\item Une partie les modules ont ont un nom différent
\item Certains modules ont un aspect différent
\item Les modules se trouvent sur d’autres emplacement
\end{itemize}

Ces différences sont en grande partie traitées automatiquement par \var{mkfli4l}.
Pour définir ces modules disponibles, vous pouvez d’une part, les tester et
examiner en fonction de la version les
(\jump{sec:regexp-dependencies}{expressions régulières conditionnelles})
d'autre part, cela vous permez avec \var{mkfli4l} d'utilisée le fichier \texttt{opt/<PAQUETAGE>.txt}
\emph{selon la version utilisée}. ils seront ensuite renommés avec un tiret bas
\texttt{opt/<PAQUETAGE>\_<Kernel-Version>.txt}, ainsi, les composants selon la version
du Kernel seront séparés les uns des autres. Le fichier du paquetage \og{}base\fg{}
sera dans le répertoire \texttt{opt}~:

\begin{itemize}
\item \texttt{base.txt}
\item \texttt{base\_3\_18.txt}
\item \texttt{base\_3\_19.txt}
\end{itemize}

Le premier fichier (\texttt{base.txt}) est \emph{toujours} traité. Les deux autres
fichiers sont traités si la version du Kernel \og{}3.18(.*)\fg{} ou \og{}3.17(.*)\fg{} est utilisée. Comme
on peut le voir, certaines parties de la version peuvent être omis dans le nom du
fichier, si vous avez un groupe de Kernels les numéros de version seront \og{}écrasés\fg{}.
En supposant que l'on utilise la \verb+VERSION_KERNEL='3.18.9'+ les fichiers suivants
(si existant) seront lues et traitées pour plusieurs installation~:

\begin{itemize}
\item \texttt{<PAQUETAGE>.txt}
\item \texttt{<PAQUETAGE>\_3.txt}
\item \texttt{<PAQUETAGE>\_3\_18.txt}
\item \texttt{<PAQUETAGE>\_3\_18\_9.txt}
\end{itemize}


\subsection{Documentation}

    Les fichiers de la documentation sont placés dans

    \begin{itemize}
    \item \texttt{doc/<LANGUE>/opt/<PAQUETAGE>.txt}
    \item \texttt{doc/<LANGUE>/opt/<PAQUETAGE>.html}
    \end{itemize}

    Les fichiers HTML peuvent également être divisés, c'est-à-dire être inclus
    dans chaque OPT différent. Il faut tout de même qu'un \texttt{<PAQUETAGE>.html},
    soit le fichier référent pour tous les autres. Les
    modifications d'un paquetage doivent être documentées dans l'un des fichiers
    suivants~:

    \begin{itemize}
    \item \texttt{changes/<PAQUETAGE>.txt}
    \end{itemize}

    L'ensemble de la documentation, ne doit pas contenir des tabulateurs et doit
    contenir un maximum de 79 caractères après un retour à la ligne. Vous devez
    vous assurer que la documentation pourra être lue correctement avec
    un éditeur sans retour de ligne automatique.

    La documentation peut être produit dans le format \LaTeX{} et ensuite être
    transformé en format HTML et PDF. À titre d'exemple,
    cette documentation peut servir à fli4l. Dans la documentation qui se
    trouver dans le paquetage \og{}template\fg{} traite du minimum requis pour les
    Macros-\LaTeX. Vous pouvez voir une brève description générale dans
    les paragraphes suivantes.

    La documentation de fli4l est actuellement disponible dans les langues suivantes~:
	allemande, anglaise (\texttt{<LANGUE>} = \og{}english\fg{}) et française
	(\texttt{<LANGUE>} = \og{}french\fg{}). C'est le responsable du paquetage qui prend
	la décision pour documenter son paquetage dans n'importe quelle langue. Pour plus
	de clarté, il est recommandé de créer une documentation en allemand et/ou en anglais
	(idéalement dans les deux langues).


\subsubsection{Conditions pour créer une Documentation-\LaTeX}

  Pour créer des documents à partir des sources-\LaTeX, vous avez certaines
  exigences à respecter relatives à l'environnement~:

  \begin{itemize}
  \item Sous l’environnement Linux/OS~X~: Pour faciliter la production de la
    documentation, vous avez le programme Makefile, avec celui-ci toutes les
    commandes sont automatisés (Cygwin peut également fonctionner, mais n'est
	pas testée par l'équipe fli4l)
  \item Installer LaTeX2HTML pour les versions HTML
  \item Bien sûr pour le \LaTeX\ (\og{}TeX Live\fg{} pour Linux/OS~X et
  \og{}MiKTeX\fg{} pour Microsoft Windows sont recommandés) avec le programme
    \og{}pdftex\fg{} et les paquets \TeX suivant~:
    \begin{itemize}
    \item L'actuel Script-KOMA (au moins la version 2)
    \item Tous les paques nécessaires pour pdftex
    \item Le paquetage décompressé de la documentation fli4l fourmit le makefile et
	les styles-\TeX
  \end{itemize}
  \end{itemize}


\subsubsection{Nom de fichier}

Les fichiers de la documentation sont désignés selon le schéma suivant~:

\begin{description}
\item [\texttt{<PAQUETAGE>\_main.tex}~:] ce fichier contient la partie principale
  de la documentation. \texttt{<PAQUETAGE>} indique le nom du paquetage, il doit
  être écrit (en minuscules).
\item [\texttt{<PAQUETAGE>\_appendix.tex}~:] si vous souhaitez ajouter des
  commentaires supplémentaires au sujet du paquetage, ils seront placés
  ici dans l’annexe.
\end{description}

Ces fichiers sont stockés dans le répertoire \texttt{fli4l/<PAQUETAGE>/doc/<LANGUE>/tex/<PAQUETAGE>}.
Vous pouvez voir ci-dessous un exemple du paquetage \og{}sshd\fg{}~:

\begin{verbatim}
    $ ls fli4l/doc/deutsch/tex/sshd/
    Makefile sshd_appendix.tex  sshd_main.tex  sshd.tex
\end{verbatim}

Makefile est chargé de générer la documentation, le fichier \texttt{sshd.tex}
fournir la structure pour la documentation actuelle et son annexe, qui est situé
dans les deux autres fichiers. Vous pouvez consulter des exemples dans la documentation
du paquetage \og{}template\fg{}.


\subsubsection{Principes de base du \LaTeX}

\LaTeX\ fonctionne un peu comme HTML \og{}orienté balise\fg{}, seulement les balises
sont appelées ici \og{}commandes\fg{}, le format est le suivant~: \verb*?\commande?
ou \verb*?\begin{environnement}? ... \verb*?\end{environnement}?.

Dans la mesure du possible, vous devez utiliser ces commandes qui accentu \emph{l'importance}
du texte et moins sa \emph{présentation}. Il est donc avantageux de les utiliser par exemple.

\begin{example}
\verb*?\warning{ne fait ... pas}?
\end{example}

\noindent Au lieu d'utiliser

\begin{example}
\verb*?\emph{ne fait ... pas}?
\end{example}

\noindent cette commande.

Chaque commande ou environnement peut absorber plusieurs paramètres
supplémentaires, on peut écrire \verb*?\commande{paramètre1}{paramètre2}{paramètreN}?.

Certaines commandes ont des paramètres optionnels (au lieu des accolades) pour fermer
la commande vous utilisez les crochets~: \verb*?\kommando[optionalerParameter]{parameter1}?
\ldots\ Habituellement, un seul paramètre optionnel est utilisé, dans des cas plus
rares, il peut y en avoir plus.

Dans le document, certains paragraphes sont séparés par des lignes blanches. \LaTeX\
gardera ces sauts de ligne pour séparés les paragraphes dans le texte.

Les caractères suivants ont une signification spéciale dans \LaTeX\, ils
doivent être précédé du caractère \ dans le texte pour être écrit normalement~:
\# \$ \& \_ \% \{ \}. Le caractère \og{}\verb?~?\fg{} et \og{}\verb?^?\fg{}, doit
être écrit comme ceci~: \verb!\verb?~?! \verb!\verb?^?!.

Les principales commandes \LaTeX\ sont expliquées dans la documentation du paquetage
\og{}template\fg{}.


\subsection{Formats de fichier}

    Dans le paquetage tous les fichiers texte (la documentation et les scripts
    d’installation qui sont sur le routeur) doivent être placés au format DOS,
    c'est à dire avec un CR/LF au lieu d'un simplement fin de ligne LF. Cela garantit
	que les utilisateurs Windows pourront également lire la documentation avec
	\og{}Notepad\fg{} (ou bloc-notes), et pourront modifier les scripts sous Windows,
	ensuite ils seront toujours susceptibles de fonctionner sur le routeur.
    Les Scriptes sont convertis à la construction des archives dans le format
    utilisé par le routeur (voir la description des flags dans le tableau~\ref{table:options}).


\subsection{Développer la documentation}

    Si vous devez définir un programme pour une nouvelle interface, à partir
    d'un paquetage et qui sera utilisé par d'autres programmes, la documentation
    de cette interface, sera séparée du reste de la documentation et se trouvera
    dans \texttt{doc/dev/<PAQUETAGE>.txt}.


\subsection{Programme-Client}

    Si vous ajoutez un programme-client pour un paquetage supplémentaire,
    il sera placé dans le répertoire \texttt{windows/} pour les clients-Windows et
    dans le répertoire \texttt{unix/} pour les clients-Unix et Linux.


\subsection{Code source}

    Les Programmes personnalisés et les codes sources peuvent être récupéré
    dans le répertoire \texttt{src/<PACKAGE>/}. Le programme peut
    être construit comme le programme-fli4l, merci de jeter un œil à la
    documentation du \jump{buildroot}{paquetage \og{}src\fg{}}.


\marklabel{sec:script_names}{
  \subsection{Les autres fichiers}
}

    Tous les fichiers stockés sur le routeur sont dans le répertoire
    \texttt{opt/}. Voici une description~:

    \begin{itemize}
    \item Les scriptes dans \texttt{opt/etc/boot.d} et \texttt{opt/etc/rc.d} sont exécutés pour
      l'amorçage du système
    \item Les Scriptes dans \texttt{opt/etc/rc0.d} sont exécutés lors de l'arrêt du système
    \item Les Scriptes dans \texttt{opt/etc/ppp} sont exécutés pour appeler et
      raccrocher une connexion téléphonique par modem
    \item Les programmes exécutables et les autres fichiers dans \texttt{opt/}, sont utilisés
	  en fonction de leurs positions dans le fichier système (par exemple le fichier
	  \texttt{opt/bin/busybox} est placé tout en haut dans le répertoire \texttt{/bin})
    \end{itemize}

    Les scripts dans \texttt{opt/etc/boot.d/}, \texttt{opt/etc/rc.d/} et \texttt{opt/etc/rc0.d/}
	sont nommés de façon suivante~:

    \begin{example}
    \begin{verbatim}
    rc<numéro>.<nom>
    \end{verbatim}
    \end{example}

    Le numéro détermine l'ordre d'exécution de l’installation, le nom donne
    une indication, pour quel programme/paquetage le script est traité.



% Synchronized to r35952

\section{Creating Scripts for fli4l}


The following is \emph{not} a general introduction to shell scripts,
everyone can read about this topic on the Internet. It is only about the flil4-specific
things. Further information is available in the various *nix/Linux help pages.
The following links may be used as entry points to this topic:
\begin{itemize}
\item Introduction to shell scripts:
  \begin{itemize}
  \item \altlink{http://linuxcommand.org/writing_shell_scripts.php}
  \end{itemize}
 \item
   Help pages online:
   \begin{itemize}
   \item \altlink{http://linux.die.net/}
   \item \altlink{http://heapsort.de/man2web}
   \item \altlink{http://man.he.net/}
   \item \altlink{http://www.linuxcommand.org/superman_pages.php}
   \end{itemize}
\end{itemize}

\subsection{Structure}

    In the *nix world, it is necessary to begin a script with the name of the interpreter,
    hence the first line is:
\begin{example}
\begin{verbatim}
      #!/bin/sh
\end{verbatim}
\end{example}

    To easily recognize later what a script does and who created it,
    this line should now be followed by a short header, like so:

\begin{example}
\begin{verbatim}
      #--------------------------------------------------------------------
      # /etc/rc.d/rc500.dummy - start my cool dummy server
      #
      # Creation:     19.07.2001 Sheldon Cooper  <sheldon@nerd.net>
      # Last Update:  11.11.2001 Howard Wolowitz <howard@nerd.net>
      #--------------------------------------------------------------------
\end{verbatim}
\end{example}

    Now for the real stuff to start...



\marklabel{dev:sec:config-variables}{
\subsection{Handling of Configuration Variables}
}


    Packages are configured via the file \texttt{config/<PACKAGE>.txt}.
    The \jump{subsec:dev:var-check}{active variables} contained there are
    transferred to the file \texttt{rc.cfg} during creation of the medium.
    This file is processed during router boot before any rc-script
    (scripts under \texttt{/etc/rc.d/}) gets started. The script then
    may access all configuration variables by reading the content of
    \var{\$<variable name>}.

    If the values of configuration variables are needed after booting
    the may be extracted from \texttt{/etc/rc.cfg} to which the configuration
    of the boot medium was written during the boot process. If for example
    the value of the variable \texttt{OPT\_DNS} should be processed in a script
    this can be achieved as follows:

\begin{example}
\begin{verbatim}
    eval $(grep "^OPT_DNS=" /etc/rc.cfg)
\end{verbatim}
\end{example}

    This works efficiently also with multiple variables (with calling
    the \texttt{grep} program only once):

\begin{example}
\begin{verbatim}
    eval $(grep "^\(HOSTNAME\|DOMAIN_NAME\|OPT_DNS\|DNS_LISTEN_N\)=" /etc/rc.cfg)
\end{verbatim}
\end{example}

\marklabel{dev:sec:persistent-data}{
\subsection{Persistent Data Storage}
}

Occasionally a package needs the possibility to store data persistent,
surviving the reboot of the router. For this purpose, the function
\texttt{map2persistent} exists and can be called from a script in
\texttt{/etc/rc.d/}. It expects a variable that contains a path
and a subdirectory. The idea is that the variable is either describing
an actual path~-- then this path is used because the user explicitely
has done so, or the string ``auto''~-- then a subdirectory on a persistent
medium is created corresponding to the second parameter. The function returns
the result in the variable passed by name as the first parameter.
An example will make this clear. \var{VBOX\_SPOOLPATH} is a variable,
that 'contains a path or the string ``auto'' . The call

\begin{example}
\begin{verbatim}
    begin_script VBOX "Configuring vbox ..."
    [...]
    map2persistent VBOX_SPOOLPATH /spool
    [...]
    end_script
\end{verbatim}
\end{example}

results in the variable \var{VBOX\_SPOOLPATH} either not being changed at all
(if it contains a path), or being changed to \texttt{/var/lib/persistent/vbox/spool}
(if it contains the string ``auto''). \texttt{/var/lib/persistent} then points\footnote{
by the use of a so-called ``bind''-mount} to a directory on a non-volatile and writable
storage medium, \var{<SCRIPT>} is the name of the calling script in lowercase
(this name is derived from the first argument of the
\jump{subsec:dev:bug-searching}{\texttt{begin\_script}-call}).
If no suitable medium should exist (which may well be),
\texttt{/var/lib/persistent} is a directory in the RAM disk.

Please note that the path returned by \texttt{map2persistent} is \emph{not}
created automatically~-- The caller has to do that by himself (ie. by calling
\texttt{mkdir -p <path>}).

The file \texttt{/var/run/persistent.conf} allows for checking if
persistent data storage is possible. Example:

\begin{example}
\begin{verbatim}
    . /var/run/persistent.conf
    case $SAVETYPE in
    persistent)
        echo "persistent data storage is possible!"
        ;;
    transient)
        echo "persistent data storage is NOT possible!"
        ;;
    esac
\end{verbatim}
\end{example}

\marklabel{subsec:dev:bug-searching}{
\subsection{Debugging}
}

    For startup scripts it is often useful to run them in debug mode
    in a shell when you are in need to determine where they fail.
    For this purpose, insert the following at the beginning and at the end:

\begin{example}
\begin{verbatim}
      begin_script <OPT-Name> "start message"
      <script code>
      end_script
\end{verbatim}
\end{example}

At the start and at the end of the script the specified text will now appear,
preceded by ``finished''.

If you want to debug the scripts, you must do two things:

\begin{enumerate}

\item You have to set \jump{DEBUGSTARTUP}{\var{DEBUG\_\-STARTUP}} to ``yes''.
\item You have to activate debugging for the OPT. This is usually done by the
  entry
\begin{example}
\begin{verbatim}
      <OPT-Name>_DO_DEBUG='yes'
\end{verbatim}
\end{example}
in the config file.\footnote{Sometimes multiple start-scripts are used,
  which then have different names for their debug-variables. Have a quick look at the scripts for clarification.}
    What happens during runtime is now displayed in detail on screen.
\end{enumerate}


\subsubsection{Further Variables helpful for Debugging}

\begin{description}

  \config{DEBUG\_ENABLE\_CORE}{DEBUG\_ENABLE\_CORE}{DEBUGENABLECORE}

  This variable allows the creation of ``Core-Dumps''. If a program crashes
  due to an error an image of the current state in the file system is stored
  which can be used to analyse the problem. The core dumps are stored under
  \texttt{/var/log/dumps/}.

  \config{DEBUG\_IP}{DEBUG\_IP}{DEBUGIP}

  Activating this variable will log all calls of the program \texttt{ip}.

  \config{DEBUG\_IPUP}{DEBUG\_IPUP}{DEBUGIPUP}

  Setting this variable to ``yes'' will log all executed instructions of the
  \texttt{ip-up}- and \texttt{ip-down}-scripts to syslog.

  \config{LOG\_BOOT\_SEQ}{LOG\_BOOT\_SEQ}{LOGBOOTSEQ}

  Setting this variable to ``yes'' will cause \texttt{bootlogd} to log all console
  output during boot to the file \texttt{/var/tmp/boot.log}. This variable has
  ``yes'' as a default value.

  \config{DEBUG\_KEEP\_BOOTLOGD}{DEBUG\_KEEP\_BOOTLOGD}{DEBUGKEEPBOOTLOGD}

  Normally \texttt{bootlogd} is terminated at the end of the boot process.
  Activating this variable prevents this and thus allows for logging console
  output during the whole runtime.

  \config{DEBUG\_MDEV}{DEBUG\_MDEV}{DEBUGMDEV}

  Setting this variable generates a logfile for the \texttt{mdev}-daemon,
  which is responsible for creating device nodes under \texttt{/dev}.

\end{description}
\subsection{Hints}
\begin{itemize}
\item  It is \emph{always} better to use curly brackets ``\{\ldots\}'' instead of normal
      ones``(\ldots)''. However, care must be taken to ensure
      that after the opening bracket a space or a new line follows before
      the next command and before the closing brackets a Semicolon
      or a new line. For example:

\begin{example}
\begin{verbatim}
        { echo "cpu"; echo "quit"; } | ...
\end{verbatim}
\end{example}

      \noindent is equal to:

\begin{example}
\begin{verbatim}
        {
                echo "cpu"
                echo "quit"
        } | ...
\end{verbatim}
\end{example}


      \item A script may be stopped by an ``exit'' at any time. This is deadly for
	start scripts (\texttt{opt/etc/boot.d/...},
        \texttt{opt/etc/rc.d/...}), stop-scripts (\texttt{opt/etc/rc0.d/...}) and\\
        \texttt{ip-up}/\texttt{ip-down}-scripts (\texttt{opt/etc/ppp/...})
        because the following scripts will not be run as well. If in doubt,
        keep your fingers away.


      \item KISS~-- Keep it small and simple. You want to use Perl for
        scripting? The scripting abilities of fli4l are not enough for you?
        Rethink your attitude! Is your OPT really necessary? fli4l after all
        is ``only'' a router and a router should not really offer server services.


      \item The error message ``: not found'' usually means the script is
        still in DOS format. Another source of errors: the script is not executable.
        In both cases the \texttt{opt/<PACKAGE>.txt} file should be checked whether
        it contains the correct options for ``mode'', ``gid'', ``uid'' and Flags.
        If the script is created during boot, execute ``chmod +x <script name>''.

      \item Use the path \texttt{/tmp} for temporary files. However, it is important
	to keep in mind that there is little space there because it is situated in the
	rootfs-RAM disc! If more space is needed, you have to create and mount your own
	ramdisk. Detailed informations on this topic can be found in the section ``RAM-Disks''
        of this documentation.

    \item In order to create temporary files with unique names you should always append
      the current process-ID stored in the shell variable ``\$'' to the file name.
      \texttt{/tmp/<OPT-Name>.\$\$} hence is a perfect file name,
      \texttt{/tmp/<OPT-Name>} rather not (\texttt{<OPT-Name>} of course has
      to be replaced by the according OPT-Name).

\end{itemize}


% Synchronized to r49912

\providecommand{\fwaction}[1]{{\small\textsf{#1}}}
\providecommand{\fwchain}[1]{\texttt{#1}}
\providecommand{\fwtable}[1]{\textsc{#1}}
\providecommand{\fwmatch}[1]{\texttt{#1}}
\providecommand{\fwpktstate}[1]{\texttt{#1}}
\providecommand{\fwloglevel}[1]{\texttt{#1}}

\section{Using The Packet Filter}
\subsection{Adding Own Chains And Rules}

A set of routines is provided to manipulate the packet filter to add or delete
so-called ``chains'' and ``rules''. A chain is a named list of ordered rules.
There is a set of predefined chains (\fwchain{PREROUTING}, \fwchain{INPUT},
\fwchain{FORWARD}, \fwchain{OUTPUT}, \fwchain{POSTROUTING}), using this set of
routines more chains can be created as needed.

\begin{description}
\item [\texttt{add\_chain/add\_nat\_chain <chain>}:]
  Adds a chain to the ``filter-'' or ``nat-'' table.
\item [\texttt{flush\_chain/flush\_nat\_chain <chain>}:]
  Deletes all rules from a chain of the ``filter-'' or ``nat-'' table.
\item [\texttt{del\_chain/del\_nat\_chain <chain>}:]
  Deletes a chain from the ``filter-'' or ``nat-'' table. Chains must be empty
  prior to deleting and all references to them have to be deleted as well before.
  Such a reference i.e. can be a \fwaction{JUMP}-action with the chain defined
  as its target.
\item[\texttt{add\_rule/ins\_rule/del\_rule}:] Adds rules to the end
  (\texttt{add\_rule}) resp. at any place of a chain (\texttt{ins\_rule})
  or deletes rules from a chain (\texttt{del\_rule}). Use the syntax like here:

\begin{example}
\begin{verbatim}
    add_rule <table> <chain> <rule> <comment>
    ins_rule <table> <chain> <rule> <position> <comment>
    del_rule <table> <chain> <rule> <comment>
\end{verbatim}
\end{example}

  \noindent where the parameters have the following meaning:
  \begin{description}
  \item[table] The table in which the chain is
  \item[chain] The chain, in which the rule is to be inserted
  \item[rule] The rule which is to be inserted, the format
     corresponds to that used in the configuration file
  \item[position] The position at which the rule will be added (only
     in \texttt{ins\_rule})
  \item[comment] A comment that appears with the rule
     when somebody looks at the packet filter.
  \end{description}
\end{description}


\subsection{Integrating Into Existing Rules}

fli4l configures the packet filter with a certain default rule set.
If you want to add your own rules, you will usually want to insert
them after the default rule set. You will also need to know what the
action is desired by the user when dropping a packet. This information
can be obtained for
\fwchain{FORWARD}- and \fwchain{INPUT} chains by calling two functions,
\texttt{get\_defaults} and \texttt{get\_count}. After calling

\begin{example}
\begin{verbatim}
    get_defaults <chain>
\end{verbatim}
\end{example}

the following results are obtained:

\begin{description}
\item[\var{drop}:] This variable contains the chain to which is branched
   when a packet is discarded.
\item[\var{reject}:] This variable contains the chain to which is branched
   when a packet is rejected.
\end{description}

After calling

\begin{example}
\begin{verbatim}
    get_count <chain>
\end{verbatim}
\end{example}

the variable \var{res} contains the number of rules in the chain
\texttt{<chain>}. This position is of importance because you can \emph{not}
simply use \texttt{add\_rule} to add a rule at the end of the predefined
``filter''-chains \fwchain{INPUT}, \fwchain{FORWARD} and \fwchain{OUTPUT}.
This is because these chains are completed with a default rule valid for
all remaining packets depending on the content of the \var{PF\_<chain>\_POLICY}-variable.
Adding a rule \emph{after} this last rule hence has no effect. The function
\texttt{get\_count} instead allows to detect the position right \emph{in front of}
this last rule and to pass this position to the \texttt{ins\_rule}-function as a
parameter \texttt{<position>} in order to add the rule in the place at the end of the
appropriate chain, but right in front of this last default rule targeting
all remaining packets.

An example from the script \texttt{opt/etc/rc.d/rc390.dns\_dhcp} from the
package ``dns\_dhcp'' shall make this clear:

\begin{example}
\begin{verbatim}
    case $OPT_DHCPRELAY in
        yes)
            begin_script DHCRELAY "starting dhcprelay ..."

            idx=1
            interfaces=""
            while [ $idx -le $DHCPRELAY_IF_N ]
            do
                eval iface='$DHCPRELAY_IF_'$idx

                get_count INPUT
                ins_rule filter INPUT "prot:udp  if:$iface:any 68 67 ACCEPT" \
                    $res "dhcprelay access"

                interfaces=$interfaces' -i '$iface
                idx=`expr $idx + 1`
            done
            dhcrelay $interfaces $DHCPRELAY_SERVER

            end_script
        ;;
  esac
\end{verbatim}
\end{example}

Here you can see in the middle of the loop a call to \texttt{get\_count}
followed by a call to the \texttt{ins\_rule} function and, among other things,
the \var{res} variable is passed as \texttt{position} parameter.


\subsection{Extending The Packet Filter Tests}

fli4l uses the syntax \fwmatch{match:params} in packet filter rules
to add additional conditions for packet matching (see \fwmatch{mac:},
\fwmatch{limit:}, \fwmatch{length:}, \fwmatch{prot:}, \ldots). If you
want to add tests you have to do this as follows:

\begin{enumerate}
\item Define a suitable name. The first character of this name has
to be lower case a-z. The rest of the name can consist of any character
or digit.

\achtung{If the packet filter test is used within IPv6 rules it is
to make sure that the name is not a valid IPv6 address component!}

\item Creating a file \texttt{opt/etc/rc.d/fwrules-<name>.ext}.
The content of this file is something like this:

\begin{example}
\begin{verbatim}
    # IPv4 extension is available
    foo_p=yes

    # the actual IPv4 extension, adding matches to match_opt
    do_foo()
    {
        param=$1
        get_negation $param
        match_opt="$match_opt -m foo $neg_opt --fooval $param"
    }

    # IPv6 extension is available
    foo6_p=yes

    # the actual IPv6 extension, adding matches to match_opt
    do6_foo()
    {
        param=$1
        get_negation6 $param
        match_opt="$match_opt -m foo $neg_opt --fooval $param"
    }
\end{verbatim}
\end{example}

\mtr{The packet filter test does not have to be implemented for both IPv4 and
IPv6 (though this would be preferred if reasonable for both layer 3 protocols).}

\item Testing the extension:

\begin{example}
\begin{verbatim}
    $ cd opt/etc/rc.d
    $ sh test-rules.sh 'foo:bar ACCEPT'
    add_rule filter FORWARD 'foo:bar ACCEPT'
    iptables -t filter -A FORWARD -m foo --fooval bar -s 0.0.0.0/0 \
        -d 0.0.0.0/0 -m comment --comment foo:bar ACCEPT -j ACCEPT
\end{verbatim}
\end{example}

\item Adding the extension and all other needed files
(\texttt{iptables} components) to the archive using
the known mechanisms.
\item Allowing the extension in the configuration by extending
of \var{FW\_GENERIC\_MATCH} and/or \var{FW\_GENERIC\_MATCH6} in an exp-file,
for example:

\begin{example}
\begin{verbatim}
    +FW_GENERIC_MATCH(OPT_FOO) = 'foo:bar' : ''
    +FW_GENERIC_MATCH6(OPT_FOO) = 'foo:bar' : ''
\end{verbatim}
\end{example}
\end{enumerate}


% Synchronized to r36388

\section{CGI-Creation for Package \emph{httpd}}

\subsection{General information about the web server}

The web server used in fli4l is \texttt{mini\_httpd} by ACME
Labs. The sources can be found at
\altlink{http://www.acme.com/software/mini_httpd/}.
However, a few changes in the current version were made for fli4l.
The modifications are located in the \emph{src} package in the directory
\texttt{src/fbr/buildroot/package/mini\_httpd}.

\subsection{Script Names}

The script names should be self-explanatory in order to be
easy to distinguish from other scripts and even similar names
have to be avoided to differ from other OPTs.

To make scripts executable and convert DOS line breaks to UNIX ones
a corresponding entry has to be created in \texttt{opt/<PACKAGE>.txt}, see
Table~\jump{table:options}{\ref{table:options}}.

\subsection{Menu Entries}

To create an entry in the menu you have to enter it in the file
\texttt{/etc/httpd/menu}. This mechanism enables OPTs to change the menu during
runtime. This should only be done using the script \texttt{/etc/httpd/menu}
because this will check for valid file formatting. New menu items
are inserted as follows:

\begin{example}
\begin{verbatim}
    httpd-menu.sh add [-p <priority>] <link> <name> [section] [realm]
\end{verbatim}
\end{example}

Thus, an entry with the name \texttt{<name>} is inserted to the \texttt{[section]}.
If \texttt{[section]} is omitted, it will be inserted in the section ``OPT-Packages''
as default. \texttt{<link>} specifies the target of the new link. \texttt{<priority>}
specifies the priority of a menu item in its section. If not set, the default priority
used is 500. The priority should be a three digit number. The lower the
priority, the higher the link is placed in the section. If an entry should be
placed as far down as possible the priority to choose is e.g. 900. Entries with
the same priority are sorted by the target of the link.
In \texttt{[realm]} the range is specified for which a logged-in user must have
\var{view} rights so the item is displayed for him. If \texttt{[realm]} is not
specified, the menu item is always displayed.
For this, see also the section \jump{sec:rights}{``User access rights''}.

Example:

\begin{example}
\begin{verbatim}
    httpd-menu.sh add "newfile.cgi" "Click here" "Tools" "tools"
\end{verbatim}
\end{example}

This example creates a link named ``Click here'' with the target ``newfile.cgi''
in the section ``Tools'' which will be created if not present.

The script may also delete entries from a menu:

\begin{example}
\begin{verbatim}
    httpd-menu.sh rem <link>
\end{verbatim}
\end{example}

By executing this the entry containing the link \texttt{<link>} will be
deleted.

\wichtig{If several entries have the same link target file they
will all be removed from the menu.}

Since sections can have priorities they can also be created manually. If a section
was created automatically when adding a menu entry it defaults to priority 500.
The syntax for creating sections is as follows:

\begin{example}
\begin{verbatim}
    httpd-menu.sh addsec <priority> <name>
\end{verbatim}
\end{example}

\texttt{<priority>} should only be a three digit number.

In order to create meaningful priorities it is worthwhile to
have a look at the file\\ \texttt{/etc/httpd/menu} of fli4l during runtime,
priorities are placed in the second column.\\

A short description of the file format of the file menu follows for completeness.
Those satisfied with the function of \texttt{httpd-menu.sh} may skip this
section. The file \texttt{/etc/httpd/menu} is divided into four columns.
The first column is a letter identifying the line as a heading
or a menu entry. The second column is the sort priority. The third column
contains the target of the link for entries and for headlines a hyphen,
as this field has no meaning for headings. The rest of the line is the text
that will appear in the menu.

Headings use the letter ``t'', a new menu section will be started then. Normal
menu items use the letters ``e''.
An example:

\begin{example}
\begin{verbatim}
    t 300 - My beautiful OPT
    e 200 myopt1.cgi Do something beautiful
    e 500 myopt1.cgi?more=yes Do something even more beautiful
\end{verbatim}
\end{example}

When editing this file you have to be aware that the script \texttt{httpd-menu.sh} 
always stores the file sorted. The individual
sections are sorted and the entries in this section are sorted too.
The sorting algorithm can be stolen from \texttt{httpd-menu.sh}, however,
it would be better to expand the script itself with possible new
functions, so that all menu-editing takes place at a central location.

\subsection{Construction of a CGI script}

\subsubsection{The headers}
All web server scripts are simple shell scripts (interpreter as e.g.
Perl, PHP, etc. are much too big in filesize for fli4l). You should start with the
mandatory script header (reference to the interpreter, name, what does the script,
author, license).

\subsubsection{Helper Script \texttt{cgi-helper}}
After the header you should include the helper script \texttt{cgi-helper}
with the following call:

\begin{example}
\begin{verbatim}
    . /srv/www/include/cgi-helper
\end{verbatim}
\end{example}

A space between the dot and the slash is important!

This script provides several helper functions that should greatly simplify the
creation of CGIs for fli4l. With the integration some standard tasks are performed,
such as the parsing of variables that were passed via forms or via the URL or loading
of language and CSS files.

Here is a small function overview:

\begin{table}[htbp]
  \centering
  \caption{Functions of the \texttt{cgi-helper} script}
  \label{tab:dev:cgi-helper}
  \begin{small}
    \begin{tabular}{|l|p{0.8\textwidth}|}
      \hline
      Name                         & Function         \\
      \hline
      \texttt{check\_rights}      & Check for user access rights \\
      \texttt{http\_header}       & Creation of a standard HTTP header or a special header, e.g. for file download\\
      \texttt{show\_html\_header} & Creation of a complete page header (inc. HTTP header, headline and menu)\\
      \texttt{show\_html\_footer} & Creation of a footer for the HTML page\\
      \texttt{show\_tab\_header}  & Creation of a content window with tabs\\
      \texttt{show\_tab\_footer}  & Creation of a footer for the content window\\
      \texttt{show\_error}        & Creation of a box for error messages (background color: red)\\
      \texttt{show\_warn}         & Creation of a box for warning messages (background color: yellow)\\
      \texttt{show\_info}         & Creation of a box for information/ success messages (background color: green)\\
      \hline
    \end{tabular}
  \end{small}
\end{table}

\subsubsection{Contents of a CGI script}

To ensure consistency in design and especially the compatibility
with future versions of fli4l it is highly recommended to use the
functions of the \texttt{cgi-helper} script even if theoretically everything
in a CGI could be generated from scratch.

A simple CGI script might look like this:

\begin{example}
\begin{verbatim}
    #!/bin/sh
    # --------------------
    # Header (c) Author Date
    # --------------------
    # get main helper functions
    . /srv/www/include/cgi-helper

    show_html_header "My first CGI"
    echo '   <h2>Welcome</h2>'
    echo '   <h3>This is a CGI script example</h3>'
    show_html_footer

\end{verbatim}
\end{example}

\subsubsection{The Function \texttt{show\_html\_header}}

The \texttt{show\_html\_header} function expects a string as a parameter.
This string represents the title of the generated page. It automatically
generates the menu and includes associated CSS and language files
as long as they can be found in the directories \texttt{/srv/www/css} resp.\
\texttt{/srv/www/lang} and have the same name (but of course a different
extension) as the script. An example:

\begin{example}
\begin{verbatim}
    /srv/www/admin/OpenVPN.cgi
    /srv/www/css/OpenVPN.css
    /srv/www/lang/OpenVPN.de
\end{verbatim}
\end{example}

Both the use of language files and CSS files is optional. The files \texttt{css/main.css} and
\texttt{lang/main.<lang>} (where \texttt{<lang>} refers to the chosen language) are always included.

Additional parameters can be passed to the function \texttt{show\_html\_header}. A call with
all possible parameters might look like this:

\begin{example}
\begin{verbatim}
    show_html_header "Title" "refresh=$time;url=$url;cssfile=$cssfile;showmenu=no"
\end{verbatim}
\end{example}

Any additional parameters must, as shown in the example, be enclosed with quotation marks
and separated by a semicolon. The syntax will \emph{not} be checked! So it is necessary to pay
close attention to the exact parameter syntax.

Here is a brief overview of the function of the parameters:
\begin{itemize}
 \item \texttt{refresh=}\emph{time}: Time in seconds in which the page should be reloaded by the browser.
 \item \texttt{url=}\emph{url}: The URL which is reloaded on a refresh.
 \item \texttt{cssfile=}\emph{cssfile}: Name of a CSS file if it differs from the name of the CGI.
 \item \texttt{showmenu=no}: By using this the display of the menu and the header can be suppressed.
\end{itemize}

Other Content Guidelines:

\begin{itemize}
 \item Don't write novels, use short desriptions :-)
 \item Use clean HTML (SelfHTML\footnote{see
    \altlink{http://de.selfhtml.org/}} is a good starting point)
 \item Omit the bells and whistles (JavaScript is OK, if it
does not interfere and support the user, everything also has to
work without JavaScript)
\end{itemize}

\subsubsection{The Function \texttt{show\_html\_footer}}

The function \texttt{show\_html\_footer} closes the block from the CGI script which was
openend by the function \texttt{show\_html\_header}.

\subsubsection{The Function \texttt{show\_tab\_header}}

For good looking content of your generated webpage generated by the CGI
you may use the \texttt{cgi-helper} function \texttt{show\_tab\_header}.
It creates clickable ``Tabs'' in order to present your page divided into
multiple logically separated areas.

Parameters are always passed in pairs to the show\_tab\_header function.
The first value reflects the title of a tab, the second reflects the link.
If the string ``no'' is passed as a link only the title will be created
and the tab is not clickable (and blue).

In the following example a ``window'' with the title ``A great window''
is generated. In the window is ``foo bar'':

\begin{example}
\begin{verbatim}
    show_tab_header "A great window" "no"
    echo "foo"
    echo "bar"
    show_tab_footer
\end{verbatim}
\end{example}

In this example, two clickable selection tabs are generated that pass
the variable \var{action} to the script, each with a different value.

\begin{example}
\begin{verbatim}
    show_tab_header "1st selection tab" "$myname?action=dothis" \
                    "2nd selection tab" "$myname?action=dothat"
    echo "foo"
    echo "bar"
    show_tab_footer
\end{verbatim}
\end{example}

Now the script can change the content of the variable \var{FORM\_action} (see variable
evaluation below) and provide different content depending on the selection.
For the clicked tab to appear selected and not clickable anymore, a ``no''
would have to be passed to the function instead of the link. But there is an
easier way, if you hold to the convention in the following example:

\begin{example}
\begin{verbatim}
    _opt_dothis="1st selection tab"
    _opt_dothat="2nd selection tab"
    show_tab_header "$_opt_dothis" "$myname?action=opt_dothis" \
                    "$_opt_dothat" "$myname?action=opt_dothat"
    case $FORM_action in
        opt_dothis) echo "foo" ;;
        opt_dothat) echo "bar" ;;
    esac
    show_tab_footer
\end{verbatim}
\end{example}

Hence, if a variable whose name equals the content of the variable \var{action} with a
leading underscore (\texttt{\_}) is passed as the title then the tab will be displayed selected.

\subsubsection{The Function \texttt{show\_tab\_footer}}

The function \texttt{show\_tab\_footer} closes the block in the CGI script that was
opened by the function \texttt{show\_tab\_header}.

\subsubsection{Multi-Language Capabilities}

The helper script \texttt{cgi-helper} furthermore contains functions to create multi-langual CGI scripts.
You only have to use variables with a leading underscore (\texttt{\_}) for all text output. This variables have to
be defined in the respective language files.

Example:

Let \texttt{lang/opt.de} contain:

\begin{example}
\begin{verbatim}
    _opt_dothis="Eine Ausgabe"
\end{verbatim}
\end{example}

Let \texttt{lang/opt.en} contain:

\begin{example}
\begin{verbatim}
    _opt_dothis="An Output"
\end{verbatim}
\end{example}

Let \texttt{admin/opt.cgi} contain:

\begin{example}
\begin{verbatim}
    ...
    echo $_opt_dothis
    ...
\end{verbatim}
\end{example}


\subsubsection{Form Evaluation}

To process forms you have to know a few more things. Regardless of using the
form's \var{GET} or \var{POST} methods, after including the \texttt{cgi-helper} script (which in turn calls
the utility proccgi) the parameters can be accessed by variables named \var{FORM\_<Parameter>}. 
If i.e. the form field had the name ``input'' the CGI script can access
its content in the shell variable \var{\$FORM\_input}.

Further informations on the program \texttt{proccgi} can be found under
\altlink{http://www.fpx.de/fp/Software/ProcCGI.html}.

\marklabel{sec:rights}{
    \subsubsection{User access rights: The Function \texttt{check\_rights}}
}

At the beginning of a CGI scripts the \texttt{check\_rights} function
has to be called in order to check whether a user has sufficient rights to use
the script. Do this like here:

\begin{example}
\begin{verbatim}
    check_rights <Section> <Action>
\end{verbatim}
\end{example}

The CGI script will only be executed if the user, who is logged in at the moment
\begin{itemize}
\item has all rights (\verb+HTTPD_RIGHTS_x='all'+), or
\item has all rights for the current area
    (\verb+HTTPD_RIGHTS_x='<Bereich>:all'+), or
\item has the right to execute the function in the current area\\
    (\verb+HTTPD_RIGHTS_x='<Bereich>:<Aktion>'+).
\end{itemize}

\subsubsection{The Function \texttt{show\_error}}

This funtion displays an error message in a red box. It expects two
parameters: a title and a message. Example:

\begin{example}
\begin{verbatim}
    show_error "Error: No key" "No key was specified!"
\end{verbatim}
\end{example}

\subsubsection{The Function \texttt{show\_warn}}

This funtion displays a warning message in a yellow box. It expects two
parameters: a title and a message. Example:

\begin{example}
\begin{verbatim}
    show_info "Warning" "No connection at the moment!"
\end{verbatim}
\end{example}

\subsubsection{The Function \texttt{show\_info}}

This funtion displays an information or success message in a green box. It expects two
parameters: a title and a message. Example:

\begin{example}
\begin{verbatim}
    show_info "Info" "Action successfully executed!"
\end{verbatim}
\end{example}

\subsubsection{The Helper Script \texttt{cgi-helper-ip4}}

Right after \texttt{cgi-helper} the helper script \texttt{cgi-helper-ip4}
may be included by writing the following line:

\begin{example}
\begin{verbatim}
. /srv/www/include/cgi-helper-ip4
\end{verbatim}
\end{example}

A space between the dot and the slash is important!

The script provides helper funtions for checking IPv4 addresses.

\subsubsection{The Function \texttt{ip4\_isvalidaddr}}

This function checks if a valid IPv4 address was passed. Example:

\begin{example}
\begin{verbatim}
    if ip4_isvalidaddr ${FORM_inputip}
    then
        ...
    fi
\end{verbatim}
\end{example}

\subsubsection{The Function \texttt{ipv4\_normalize}}

This function removes leading zeros from the passed IPv4 address. Example:

\begin{example}
\begin{verbatim}
    ip4_normalize ${FORM_inputip}
    IP=$res
    if [ -n "$IP" ]
    then
        ...
    fi
\end{verbatim}
\end{example}
\subsubsection{The Function \texttt{ipv4\_isindhcprange}}

This function checks whether the passed IPv4 address is ranged between the
passed start and end addresses. Example:

\begin{example}
\begin{verbatim}
    if ip4_isindhcprange $FORM_inputip $ip_start $ip_end
    then
        ...
    fi
\end{verbatim}
\end{example}

\subsection{Miscellaneous}

This and that (yes, also important!):

\begin{itemize}
 \item \texttt{mini\_httpd} does not protect subdirectories with a password. Each directory
       must contain a \texttt{.htaccess} file or a link to another \texttt{.htaccess} file.
 \item KISS - Keep it simple, stupid!
 \item This information may change at any time without prior notice!
\end{itemize}

\subsection{Debugging}

To ease debugging of a CGI script you may activate the debugging mode
by sourcing the \texttt{cgi-helper} script. Set the variable \var{set\_debug}
to ``yes'' in order to do so. This will create a file \texttt{debug.log}
which may be loaded down with the URL \texttt{http://<fli4l-Host>/admin/debug.log}.
It contains all calls of the CGI script. The variable \texttt{set\_debug} is not a
global one, it has to be set anew for each CGI in question. Example:

\begin{example}
\begin{verbatim}
    set_debug="yes"
    . /srv/www/include/cgi-helper
\end{verbatim}
\end{example}

Furthermore, cURL\footnote{see \altlink{http://de.wikipedia.org/wiki/CURL}} is
ideal for troubleshooting, especially if the HTTP headers
are not assembled correctly or the browser displays only blank pages.
Also, the caching behavior of modern Web browser is obstructive when troubleshooting.

Example: Get a dump of the HTTP-Header with ("`\emph{d}ump"',
\texttt{-D}) and the normal output of the CGI \texttt{admin/my.cgi}.
The ``\emph{u}ser'' (\texttt{-u}) name here shall be ``admin''.

\begin{example}
\begin{verbatim}
    curl -D - http://fli4l/admin/my.cgi -u admin
\end{verbatim}
\end{example}


% Last Update: $Id$

\section{Hochfahren, Herunterfahren, Einwählen und Auflegen unter fli4l}

\subsection{Bootkonzept}

fli4l 2.0 sollte eine saubere Installation auf eine Festplatte oder ein
CompactFlash(TM)-Medium bieten, aber auch eine Installation auf ein
Zip-Medium oder die Erstellung einer bootfähigen CD-ROM sollte möglich
sein. Zusätzlich sollte die Festplattenversion sich nicht grundlegend von
einer Installation auf Diskette\footnote{Ursprünglich konnte fli4l auch
von einer Diskette betrieben werden. Da fli4l inzwischen dafür zu groß geworden
ist, wird dies nicht mehr unterstützt.} unterscheiden.

Diese Anforderungen wurden realisiert, indem die Dateien des
\texttt{opt.img}-Archivs aus der
bisherigen RAM-Disk auf eine anderes Medium verlagert werden können. Dies
kann eine Partition auf einer Festplatte bzw.\ einem CF-Medium sein. Dieses
zweite Volume wird unter \texttt{/opt} eingehängt, und dort liegende Programme
werden nur noch über symbolische Links in das RootFS integriert. Das entstehende Layout im
RootFS-Dateisystem entspricht dann dem im \texttt{opt}-Verzeichnis der
ausgepackten fli4l-Distribution mit einer Ausnahme~-- das \texttt{files}-Präfix
entfällt. Die Datei \texttt{opt/etc/rc} ist dann also direkt unter \texttt{/etc/rc} zu
finden, \texttt{opt/bin/busybox} unter
\texttt{/bin/busybox}. Dass diese Dateien unter Umständen nur symbolische Verknüpfungen
auf ein im Nur-Lese-Modus eingehängtes Volume sind, kann man ignorieren,
solange man die Dateien nicht modifizieren möchte. Will man dies tun,
muss man die Dateien vorher mit \texttt{mk\_writable} (s.\,u.) schreibbar machen.

\subsection{Start- und Stopp-Skripte}

Skripte, die beim Hochfahren des Systems ausgeführt werden sollen, befinden
sich in den Verzeichnissen \texttt{opt/etc/boot.d/} und \texttt{opt/etc/rc.d/}
und werden auch in dieser Reihenfolge ausgeführt. Des Weiteren befinden sich in
\texttt{opt/etc/rc0.d/} Skripte, die beim Herunterfahren des Systems ausgefühlt
werden.

\wichtig{Da zum Ausführen dieser Skripte kein separater Prozess
erzeugt wird, dürfen sie nicht mit "`exit"' abgeschlossen
werden. Ein solcher Befehl führt zum vorzeitigen Abbruch des Bootvorgangs!}

\subsubsection{Start-Skripte in \texttt{opt/etc/boot.d/}}

Skripte, die in diesem Verzeichnis liegen, werden als erstes
ausgeführt. Ihre Aufgabe ist es, das Boot-Volume einzuhängen, die
auf dem Boot-Medium liegende Konfigurationsdatei \texttt{rc.cfg} einzulesen und
das Opt-Archiv zu entpacken. Je nach \jump{BOOTTYPE}{Boot-Typ}
sind diese Skripte mehr oder weniger kompliziert und tun die folgenden
Dinge:

\begin{itemize}
\item Laden von Hardware-Treibern (optional)
\item Boot-Volume einhängen (optional)
\item Konfigurationsdatei \texttt{rc.cfg} vom Boot-Volume einlesen und
      in die Datei \texttt{/etc/rc.cfg} schreiben
\item Einhängen des Opt-Volumes (optional)
\item Extrahieren des Opt-Archivs (optional)
\end{itemize}

Damit die Skripte überhaupt eine Chance haben, etwas über die
fli4l-Konfiguration zu erfahren, wird die Konfigurationsdatei auch ins
RootFS-Archiv unter \texttt{/etc/rc.cfg} eingebunden; die
Konfigurationsvariablen in dieser Datei stehen den Start-Skripten in
\texttt{opt/etc/boot.d/} direkt zur Verfügung. Nach dem Einhängen des
Boot-Volumes wird die \texttt{/etc/rc.cfg} durch die Konfigurationsdatei
vom Boot-Volume ersetzt, so dass den Start-Skripten in \texttt{opt/etc/rc.d/}
(s.\,u.) die aktuelle Konfiguration vom Boot-Volume zur Verfügung steht.
\footnote{Normalerweise sind diese beiden Dateien identisch. Eine Abweichung
entsteht nur, wenn die Konfigurationsdatei auf dem Boot-Volume händisch
editiert wird, etwa um die Konfiguration nachträglich abzuändern, ohne die
fli4l-Archive neu zu bauen.}

\subsubsection{Start-Skripte in \texttt{opt/etc/rc.d/}}

Befehle, die immer beim Starten des Routers ausgeführt werden müssen,
können in Skripten im Verzeichnis \texttt{opt/etc/rc.d/} abgelegt werden. Hierbei
gelten folgende Konventionen:

\begin{enumerate}
  \item Es gilt folgende Namenskonvention:

\begin{example}
\begin{verbatim}
    rc<dreistellige Zahl>.<Name des OPTs>
\end{verbatim}
\end{example}
      
    Die Skripte werden in aufsteigender Reihenfolge der Zahlen
    gestartet. Ist mehreren Skripten dieselbe Zahl zugeordnet,
    entscheidet die alphabetische Sortierung nach dem Punkt. Falls
    der Start eines Pakets erst nach einem anderen erfolgen darf,
    wird das durch die Zahl festgelegt.

    Hier eine ungefähre Richtlinie, welche Nummern für welche Aufgaben
    verwendet werden sollten:

    \begin{table}[htbp]
    \centering
    \begin{tabular}{lp{0.8\textwidth}}
            \hline
            Nummer        &       Aufgabe         \\
            \hline
            \hline
            000-099       &       Grundsystem (Hardware, Zeitzone, Dateisystem) \\
            100-199       &       Kernel-Module (Treiber) \\
            200-299       &       externe Verbindungen (PPPoE, ISDN4Linux, PPtP) \\
            300-399       &       Netzwerk (Routing, Interfaces, Paketfilter) \\
            400-497       &       Server (DHCP, HTTPD, Proxy, etc.) \\
            498-499       &       reserviert (Aktivierung des Circuit-Systems) \\
            500-997       &       nach Belieben (ab hier können Skripte von Circuits
                                  kontrollierte Verbindungen nutzen und laufen
                                  möglicherweise parallel zu einer Einwahl) \\
            998-999       &       reserviert (bitte nicht benutzen!) \\
            \hline
    \end{tabular}
    \end{table}

  \item In diesen Skripten \emph{müssen} alle Funktionen, die das RootFS
    verändern, hinterlegt werden, etwa das Anlegen eines
    Verzeichnisses \texttt{/var/log/lpd}.

  \item In diesen Skripten dürfen \emph{keine} Schreibzugriffe auf
    Dateien erfolgen, die Teil des Opt-Archivs sein können, da diese Dateien
    auf einem im Nur-Lese-Modus eingehängten Volume liegen können.
    Muss man eine solche Datei modifizieren, muss man sie
    vorher mit Hilfe der Funktion \texttt{mk\_writable} (s.\,u.) beschreibbar
    machen. Durch den Aufruf der Funktion wird die Datei (wenn nötig) als
    beschreibbare Kopie im RootFS abgelegt. Ist die Datei bereits beschreibbar,
    bewirkt der \texttt{mk\_writable}-Aufruf nichts.
    
    \wichtig{\texttt{mk\_writable} muss direkt auf Dateien im RootFS angewandt
    werden, nicht über den Umweg des \texttt{opt}-Verzeichnisses. Will man
    also \texttt{/usr/local/bin/foo} modifizieren, ruft man
    \texttt{mk\_writable} mit dem Argument \texttt{/usr/local/bin/foo} auf.}

  \item Diese Skripte müssen vor der Ausführung der eigentlichen
    Befehle prüfen, ob das dazugehörige OPT auch aktiv ist. Das ist
    normalerweise durch eine einfache Fallunterscheidung erledigt:

\begin{example}
\begin{verbatim}
    if [ "$OPT_<OPT-Name>" = "yes" ]
    then
        ...
        # Hier OPT starten!
        ...
    fi
\end{verbatim}
\end{example}

  \item Um die Fehlersuche zu erleichtern, sollten die Skripte mit
    \texttt{begin\_script} und \texttt{end\_script} geklammert werden:

\begin{example}
\begin{verbatim}
    if [ "$OPT_<OPT-Name>" = "yes" ]
    then
        begin_script FOO "configuring foo ..."
        ...
        end_script
    fi
\end{verbatim}
\end{example}

    Die Fehlersuche einzelner Start-Skripte kann man dann einfach via
    \verb+FOO_DO_DEBUG='yes'+ aktivieren.

  \item Den Skripten stehen alle Konfigurationsvariablen
    direkt zur Verfügung. Im Abschnitt \jump{dev:sec:config-variables}{"`Umgang
    mit Konfigurationsvariablen"'} wird erklärt, wie man von anderen Skripten
    aus auf die Konfigurationsvariablen zugreifen kann.

  \item Der Pfad \texttt{/opt} darf auch nicht als Speicherplatz für
    Datenbestände der OPTs benutzt werden. Falls zusätzlicher
    Speicherplatz benötigt wird, sollte dem Benutzer über eine
    Konfigurationsvariable die Möglichkeit gegeben werden, einen geeigneten
    Pfad auszuwählen. Je nach Art der zu speichernden Daten (persistente oder
    transiente Daten) sind verschiedene Standard-Belegungen sinnvoll. So bieten
    sich für transiente Daten etwa Pfade unterhalb von \texttt{/var/run/} an;
    für persistente Daten sollte die Funktion
    \jump{dev:sec:persistent-data}{map2persistent} in Kombination mit einer
    geeigneten Konfigurationsvariable verwendet werden.

\end{enumerate}

\subsubsection{Stopp-Skripte in \texttt{opt/etc/rc0.d/}}

Jeder Rechner muss mal heruntergefahren oder neu gestartet werden. Nun kann
es vorkommen, dass man Vorgänge ausführen muss, bevor der Rechner
heruntergefahren oder neu gestartet wird. Zum Herunterfahren und Neustarten
sind die Befehle "`halt"' bzw.\ "`reboot"' zuständig. Diese Befehle werden auch
aufgerufen, wenn man im IMONC oder in der Web-GUI auf die entsprechenden
Schaltflächen klickt.

Alle Stopp-Skripte liegen im Verzeichnis \texttt{opt/etc/rc0.d/}. Ihre Dateinamen
werden analog zu den Start-Skripten gebildet. Sie werden ebenfalls in
\emph{auf}steigender Reihenfolge der Zahlen ausgeführt.

\subsection{Hilfsfunktionen}

In \texttt{/etc/boot.d/base-helper} werden verschiedene Funktionen
bereitgestellt, die von den Start- und anderen Skripten verwendet werden
können. Das betrifft Dinge wie Unterstützung zur Fehlersuche, Laden von
Kernel-Modulen oder Ausgabe von Meldungen.
Die einzelnen Funktionen werden im Folgenden aufgelistet und kurz beschrieben.

\subsubsection{Skript-Steuerung}

\begin{description}

\item[\texttt{begin\_script <Symbol> <Meldung>}:]
Gibt eine Meldung aus und aktiviert die Fehlersuche im Skript mittels
\texttt{set -x}, falls \var{<Symbol>\_DO\_DEBUG} auf "`yes"' steht.

\item[\texttt{end\_script}:]
Gibt eine Abschluss-Meldung aus und deaktiviert
die Fehlersuche, falls sie mit \texttt{begin\_script} aktiviert wurde. Für
jeden \texttt{begin\_script}-Aufruf muss es einen zugehörigen
\texttt{end\_script}-Aufruf geben (und umgekehrt).

\end{description}

\subsubsection{Laden von Kernel-Modulen}

\begin{description}

\item[\texttt{do\_modprobe [-q] <Modul> <Parameter>*}:]
Lädt ein Kernel-Modul inkl.
eventueller Parameter bei gleichzeitiger Auflösung der Modulabhängigkeiten.
Der Parameter "`-q"' verhindert, dass im Fehlerfall eine Meldung auf der Konsole
ausgegeben und ins Boot-Protokoll geschrieben wird.
Die Funktion liefert im Erfolgsfall den Rückgabewert null zurück und im
Fehlerfall einen Wert ungleich null. Damit lässt sich Code schreiben, der
ein Fehlschlagen des Ladens eines Kernel-Moduls geeignet behandelt:

\begin{example}
\begin{verbatim}
    if do_modprobe -q acpi-cpufreq
    then
        # kein CPU-Frequenzsteuerung via ACPI
        log_error "CPU-Frequenzsteuerung via ACPI nicht verfügbar!"
        # [...]
    else
        log_info "CPU-Frequenzsteuerung via ACPI aktiviert."
        # [...]
    fi
\end{verbatim}
\end{example}

\item[\texttt{do\_modrobe\_if\_exists [-q] <Modulpfad> <Modul> <Parameter>*}:]\mbox{}\\
Prüft, ob das Modul \texttt{/lib/modules/<Kernel-Version>/<Modulpfad>/<Modul>}
existiert und ruft bei Vorhandensein die Funktion \texttt{do\_modprobe} auf.

\wichtig{Das Modul muss tatsächlich unter dem angegebenen Modulnamen existieren,
der Modulname darf kein Alias sein. Bei einem Alias wird direkt
\texttt{do\_modprobe} aufgerufen.}

\end{description}

\subsubsection{Meldungen und Fehlerbehandlung}

\begin{description}

\item[\texttt{log\_info <Meldung>}:] Schreibt eine Meldung auf
die Konsole und nach \texttt{/bootmsg.txt}. Wird keine Meldung als
Parameter übergeben, liest \texttt{log\_info} von der Standard-Eingabe. Die
Funktion liefert als Rückgabewert immer null zurück.

\item[\texttt{log\_warn <Meldung>}:] Schreibt eine Warnmeldung auf
die Konsole und nach \texttt{/bootmsg.txt}, wobei vor die Meldung die
Zeichenkette \texttt{WARN:} gesetzt wird. Wird keine Meldung als
Parameter übergeben, liest \texttt{log\_warn} von der Standard-Eingabe.
Die Funktion liefert als Rückgabewert immer null zurück.

\item[\texttt{log\_error <Meldung>}:] Schreibt eine Fehlermeldung auf
die Konsole und nach \texttt{/bootmsg.txt}, wobei vor die Meldung die
Zeichenkette \texttt{ERR:} gesetzt wird. Wird keine Meldung als
Parameter übergeben, liest \texttt{log\_error} von der Standard-Eingabe.
Die Funktion liefert als Rückgabewert immer einen Wert ungleich null zurück.

\item[\texttt{set\_error <Meldung>}:] Gibt die Fehlermeldung aus und setzt
eine interne Fehlervariable, das später via \texttt{is\_error} geprüft werden
kann.

\item[\texttt{is\_error}:] Setzt die interne Fehlervariable zurück und liefert
wahr zurück, falls sie vorher durch \texttt{set\_error} gesetzt wurde.

\end{description}

\subsubsection{Netzwerk-Funktionen}

\begin{description}

\item[\texttt{translate\_ip\_net <Wert> <Variablenname> [<Ergebnisvariable>]}:]
Ersetzt symbolische Referenzen in
Parametern. Momentan finden folgende Übersetzungen statt:
\begin{description}
\item[Host- oder Netzwerk-Adressen] werden nicht übersetzt
\item[\texttt{any}] wird durch \texttt{0.0.0.0/0} ersetzt
\item[\texttt{dynamic}] wird durch die dynamische IPv4-Adresse des Routers
ersetzt; das ist die IPv4-Adresse, die der Router beim Einwählen über einen
Circuit mit einer Default-Route zugewiesen bekommt
\item[\var{IP\_NET\_x}] wird durch das in der Konfiguration stehende
Netzwerk ersetzt; referenziert die Variable einen Circuit, so wird das dem
Circuit zugewiesene IPv4-Netz zurückgegeben
\item[\var{IP\_NET\_x\_IPADDR}] wird durch die in der Konfiguration stehende
IP-Adresse ersetzt; referenziert die Variable einen Circuit, so wird die dem
Circuit zugewiesene IPv4-Adresse zurückgegeben
\item[\var{IP\_ROUTE\_x}] wird durch das in der Konfiguration stehende
geroutete Netzwerk ersetzt
\item[\texttt{@<Hostname>}] wird durch die in der Konfiguration für den
angegeben Host spezifizierte IP-Adresse ersetzt
\item[\texttt{\{<circuit>\}}] wird durch das dem Circuit zugewiesene IPv4-Netz
ersetzt
\end{description}

Das Ergebnis der Übersetzung wird in der Variable gespeichert, deren Name im
dritten Parameter übergeben wird; fehlt dieser Parameter, wird das Ergebnis in
der Variable \var{res} gespeichert. Der Variablenname, der im zweiten Parameter
übergeben wird, wird nur für Fehlermeldungen benutzt, falls die Übersetzung
fehlschlägt; hier kann also vom Aufrufer die Quelle des zu übersetzenden Wertes
angegeben werden. Im Fehlerfall wird dann eine Meldung wie

\begin{example}
\begin{verbatim}
    Unable to translate value '<Wert>' contained in <Variablenname>.
\end{verbatim}
\end{example}

ausgegeben.

Der Rückgabewert ist null, falls die Übersetzung erfolgreich war, und ungleich
null, falls ein Fehler aufgetreten ist.

\end{description}

\subsubsection{Diverses}

\begin{description}

\item[\texttt{mk\_writable <Datei>}:]
Stellt sicher, dass die übergebene Datei beschreibbar ist. Befindet sich die
Datei auf einem im Nur-Lese-Modus eingehängten Volume und ist lediglich über
eine symbolische Verknüpfung ins Dateisystem eingebunden, wird eine lokale Kopie
angelegt, die dann beschreibbar ist.

\item[\texttt{list\_unique <Liste>}:]
Entfernt Duplikate aus der übergebenen Liste. Das Resultat wird in die
Standard-Ausgabe geschrieben.

\end{description}

\subsection{mdev-Regeln}

Für OPTs ist es möglich, zusätzliche mdev-Regeln zu etablieren, die spezielle
Aktionen beim Erscheinen oder Verschwinden bestimmter Geräte vornehmen. Das
\var{OPT\_AUTOMOUNT} im hd-Paket verwendet beispielsweise eine solche Regel,
um auftauchende Speichermedien automatisch einzuhängen. Will man eine
zusätzliche mdev-Regel integrieren, muss man ein Skript der Form

\begin{verbatim}
    /etc/mdev.d/mdev<Nummer>.<Name>
\end{verbatim}

ins RootFS einbauen, wobei die Nummer ähnlich den Start- und Stopp-Skripten aus
drei Ziffern bestehen muss und der Name beliebig gewählt werden kann. Innerhalb
dieses Skriptes werden sämtliche Ausgaben an die Standardausgabe in die
resultierende \texttt{/etc/mdev.conf} integriert. Ein Beispiel aus dem oben
erwähnten \var{OPT\_AUTOMOUNT}:

\begin{small}
\begin{verbatim}
#!/bin/sh
#----------------------------------------------------------------------------
# /etc/mdev.d/mdev500.automount - mdev HD automounting rules     __FLI4LVER__
#
#
# Last Update:  $Id$
#----------------------------------------------------------------------------

cat <<"EOF"
#
# mdev500.automount
#

-SUBSYSTEM=block;DEVTYPE=partition;.+       0:0 660 */lib/mdev/automount

EOF
\end{verbatim}
\end{small}

Zu der Syntax der Regeln sei auf den Dateikopf der \texttt{/etc/mdev.conf}
sowie auf die mdev-Dokumentation unter
\altlink{http://git.busybox.net/busybox/plain/docs/mdev.txt} verwiesen. Falls
eine Regel ein Skript aufruft (wie \texttt{/lib/mdev/automount} im obigen
Beispiel), dann hat es Zugriff auf alle Variablen des auslösenden
Kernel-``uevent''s, insbesondere auf:

\begin{itemize}
\item \var{ACTION} (typischerweise \texttt{add} oder \texttt{remove}, seltener
\texttt{change})
\item \var{DEVPATH} (sysfs-Pfad zu der betroffenen Komponente)
\item \var{SUBSYSTEM} (das betroffene Kernel-Subsystem, siehe unten)
\item \var{DEVNAME} (die betroffene Gerätedatei unter \texttt{/dev}; fehlt, wenn
es nicht um zu erstellende oder löschende Geräte geht, sondern z.B. um Module)
\item \var{MDEV} (wird von mdev auf den Namen der letzlich erzeugten Gerätedatei
gesetzt)
\end{itemize}

Beispiele für Kernel-Subsysteme:

\begin{description}
\item[block] Blockgeräte (Speichermedien) wie \texttt{sda} (erste Festplatte),
             \texttt{sr0} (erstes CD-Laufwerk) oder \texttt{ram1} (zweite
             RAM-Disk)
\item[input] Geräte für Eingaben von Tastatur, Maus etc. wie
             \texttt{input/event0}; welche Gerätedateien welchen Geräten
             zugeordnet sind, ist nicht festgelegt und muss im sysfs ermittelt
             werden
\item[mem]   Geräte zum Zugriff auf den Speicher und Hardware-Ports wie
             \texttt{mem} und \texttt{ports}; hier fallen auch Pseudo-Geräte wie
             \texttt{zero} (liefert ununterbrochen das ASCII-Zeichen mit Wert
             null) und \texttt{null} (liefert nichts, verschluckt alles)
             darunter
\item[sound] diverse Geräte für die Tonausgabe, Benennung uneinheitlich
\item[tty]   Geräte zum Zugriff auf physische und virtuelle Konsolen wie
             \texttt{tty1} (erste virtuelle Konsole) oder \texttt{ttyS0} (erste
             serielle Konsole)
\end{description}

Ein Beispiel für die ersten beiden seriellen Schnittstellen:

\begin{scriptsize}
\begin{verbatim}
mdev[42]: 30.050644 add@/devices/pnp0/00:04/tty/ttyS0
mdev[42]:   ACTION=add
mdev[42]:   DEVPATH=/devices/pnp0/00:04/tty/ttyS0
mdev[42]:   SUBSYSTEM=tty
mdev[42]:   MAJOR=4
mdev[42]:   MINOR=64
mdev[42]:   DEVNAME=ttyS0
mdev[42]:   SEQNUM=613

mdev[42]: 30.051477 add@/devices/platform/serial8250/tty/ttyS1
mdev[42]:   ACTION=add
mdev[42]:   DEVPATH=/devices/platform/serial8250/tty/ttyS1
mdev[42]:   SUBSYSTEM=tty
mdev[42]:   MAJOR=4
mdev[42]:   MINOR=65
mdev[42]:   DEVNAME=ttyS1
mdev[42]:   SEQNUM=614
\end{verbatim}
\end{scriptsize}

Ein Beispiel für eine angeschlossene MF II-Tastatur:

\begin{scriptsize}
\begin{verbatim}
mdev[41]: 4.030653 add@/devices/platform/i8042/serio0/input/input0
mdev[41]:   ACTION=add
mdev[41]:   DEVPATH=/devices/platform/i8042/serio0/input/input0
mdev[41]:   SUBSYSTEM=input
mdev[41]:   PRODUCT=11/1/1/ab41
mdev[41]:   NAME="AT Translated Set 2 keyboard"
mdev[41]:   PHYS="isa0060/serio0/input0"
mdev[41]:   PROP=0
mdev[41]:   EV=120013
mdev[41]:   KEY=4 2000000 3803078 f800d001 feffffdf ffefffff ffffffff fffffffe
mdev[41]:   MSC=10
mdev[41]:   LED=7
mdev[41]:   MODALIAS=input:b0011v0001p0001eAB41-e0,1,4,11,14,k71,72,73,74,75,76,77,79,
7A,7B,7C,7D,7E,7F,80,8C,8E,8F,9B,9C,9D,9E,9F,A3,A4,A5,A6,AC,AD,B7,B8,B9,D9,E2,ram4,l0,
1,2,sfw
mdev[41]:   SEQNUM=604
\end{verbatim}
\end{scriptsize}

Ein Beispiel für ein geladenes USB-Kernelmodul (\texttt{uhci\_hcd}):

\begin{scriptsize}
\begin{verbatim}
mdev[41]: 6.537506 add@/module/uhci_hcd
mdev[41]:   ACTION=add
mdev[41]:   DEVPATH=/module/uhci_hcd
mdev[41]:   SUBSYSTEM=module
mdev[41]:   SEQNUM=633
\end{verbatim}
\end{scriptsize}

Ein Beispiel für eine angeschlossene Festplatte:

\begin{scriptsize}
\begin{verbatim}
mdev[41]: 7.267527 add@/devices/pci0000:00/0000:00:07.1/ata1/host0/target0:0:0/0:0:0:0/block/sda
mdev[41]:   ACTION=add
mdev[41]:   DEVPATH=/devices/pci0000:00/0000:00:07.1/ata1/host0/target0:0:0/0:0:0:0/block/sda
mdev[41]:   SUBSYSTEM=block
mdev[41]:   MAJOR=8
mdev[41]:   MINOR=0
mdev[41]:   DEVNAME=sda
mdev[41]:   DEVTYPE=disk
mdev[41]:   SEQNUM=688
\end{verbatim}
\end{scriptsize}

Dies ist eine ATA/IDE-Festplatte (ata1), die über den Gerätenamen \texttt{sda}
angesprochen werden sollte.

Ein Beispiel für ein entferntes Blockgerät (die Zuordnung einer Image-Datei zu
einer fli4l-VM wurde via \texttt{virsh detach-device} gelöst):

\begin{scriptsize}
\begin{verbatim}
mdev[42]: 52.600646 remove@/devices/pci0000:00/0000:00:0a.0/virtio5/block/vdb/vdb1
mdev[42]:   ACTION=remove
mdev[42]:   DEVPATH=/devices/pci0000:00/0000:00:0a.0/virtio5/block/vdb/vdb1
mdev[42]:   SUBSYSTEM=block
mdev[42]:   MAJOR=254
mdev[42]:   MINOR=17
mdev[42]:   DEVNAME=vdb1
mdev[42]:   DEVTYPE=partition
mdev[42]:   SEQNUM=776

mdev[42]: 52.644642 remove@/devices/virtual/bdi/254:16
mdev[42]:   ACTION=remove
mdev[42]:   DEVPATH=/devices/virtual/bdi/254:16
mdev[42]:   SUBSYSTEM=bdi
mdev[42]:   SEQNUM=777

mdev[42]: 52.644718 remove@/devices/pci0000:00/0000:00:0a.0/virtio5/block/vdb
mdev[42]:   ACTION=remove
mdev[42]:   DEVPATH=/devices/pci0000:00/0000:00:0a.0/virtio5/block/vdb
mdev[42]:   SUBSYSTEM=block
mdev[42]:   MAJOR=254
mdev[42]:   MINOR=16
mdev[42]:   DEVNAME=vdb
mdev[42]:   DEVTYPE=disk
mdev[42]:   SEQNUM=778

mdev[42]: 52.644777 remove@/devices/pci0000:00/0000:00:0a.0/virtio5
mdev[42]:   ACTION=remove
mdev[42]:   DEVPATH=/devices/pci0000:00/0000:00:0a.0/virtio5
mdev[42]:   SUBSYSTEM=virtio
mdev[42]:   MODALIAS=virtio:d00000002v00001AF4
mdev[42]:   SEQNUM=779

mdev[42]: 52.644973 remove@/devices/pci0000:00/0000:00:0a.0
mdev[42]:   ACTION=remove
mdev[42]:   DEVPATH=/devices/pci0000:00/0000:00:0a.0
mdev[42]:   SUBSYSTEM=pci
mdev[42]:   PCI_CLASS=10000
mdev[42]:   PCI_ID=1AF4:1001
mdev[42]:   PCI_SUBSYS_ID=1AF4:0002
mdev[42]:   PCI_SLOT_NAME=0000:00:0a.0
mdev[42]:   MODALIAS=pci:v00001AF4d00001001sv00001AF4sd00000002bc01sc00i00
mdev[42]:   SEQNUM=780
\end{verbatim}
\end{scriptsize}

Wie man sehen kann, sind bei einer solchen Entfernung diverse Kernel-Subsysteme
involviert (hier \texttt{block}, \texttt{bdi}, \texttt{virtio} und
\texttt{pci}).

\subsection{ttyI-Geräte}

Für die ttyI-Geräte (\texttt{/dev/ttyI0} \ldots \texttt{/dev/ttyI15}), über
welche die "`Modem-Emulation"' der ISDN-Karte genutzt werden kann, existiert
ein Zähler, um Konflikte zwischen verschiedenen Paketen, die diese Geräte
nutzen, zu vermeiden.
Hierzu wird beim Start des Routers die Datei \texttt{/var/run/next\_ttyI}
angelegt, die von den verschiedenen OPTs abgefragt und fortgezählt werden kann.
Mit dem folgenden Beispielskript kann dieser Wert abgefragt, um eins
erhöht und wieder für das nächste OPT exportiert werden.

\begin{example}
\begin{verbatim}
    ttydev_error=
    ttydev=$(cat /var/run/next_ttyI)
    if [ $ttydev -le 16 ]
    then                                    # ttyI device available? yes
        ttydev=$((ttydev + 1))              # ttyI device + 1
        echo $ttydev >/var/run/next_ttyI    # save it
    else                                    # ttyI device available? no
        log_error "No ttyI device for <Name deines OPTs> available!"
        ttydev_error=true                   # set error for later use
    fi

    if [ -z "$ttydev_error" ]               # start OPT only if next tty device
    then                                    # was available to minimize error
        ...                                 # messages and minimize the
                                            # risk of uncomplete boot
    fi
\end{verbatim}
\end{example}

\subsection{Skripte beim Einwählen und Auflegen}

\subsubsection{Allgemeines}

Nach dem Herstellen bzw.\ Trennen einer Circuit-Verbindung werden die Skripte
in \texttt{/etc/ppp/} abgearbeitet. Hier können OPTs Aktionen hinterlegen,
die nach dem Herstellen bzw.\ Auflegen der Verbindung nötig
sind. Benannt werden die Dateien wie folgt:

\begin{table}[htbp]
\centering
\begin{tabular}{l}
    \texttt{ip-up<dreistellige Zahl>.<OPT-Name>}\\
    \texttt{ip-down<dreistellige Zahl>.<OPT-Name>}\\
\end{tabular}
\end{table}

Dabei werden die \texttt{ip-up}-Skripte nach dem \emph{Aufbau} und die
\texttt{ip-down}-Skripte nach dem \emph{Abbau} der Verbindung ausgeführt.

\wichtig{In den \texttt{ip-down}-Skripten dürfen keine Aktionen ausgeführt
werden, die zu einer erneuten Einwahl führen, da dadurch nur ein
Dauer-Online-Zustand erreicht wird, was für Nicht-Flatrate-Benutzer
ein teures Unterfangen ist.}

\wichtig{Da für die einzelnen Skripte kein eigener Prozess erzeugt wird, dürfen
auch diese Skripte \emph{nicht} mit "`exit"' abgeschlossen werden!}

\subsubsection{Variablen}

Durch das spezielle Aufrufkonzept stehen die folgenden Variablen den
\texttt{ip-up}- und \texttt{ip-down}-Skripten zur Verfügung:

\begin{table}[htbp]
\centering
\begin{tabular}{lp{10cm}}

    \var{real\_interface}    & die aktuelle Schnittstelle, also z.\,B.
                               \texttt{ppp0}, \texttt{ippp0}, \ldots\\
    \var{interface}          & das IMOND-Interface, also \texttt{pppoe},
                               \texttt{ippp0}, \ldots\\
    \var{tty}                & verbundenes Terminal, möglicherweise leer!\\
    \var{speed}              & die Verbindunggeschwindigkeit, bei ISDN z.\,B.
                               64000\\
    \var{local}              & die eigene IP-Adresse\\
    \var{remote}             & die IP-Adresse des Point-To-Point-Partners\\
    \var{is\_default\_route} & gibt an, ob das aktuelle
                               \texttt{ip-up}/\texttt{ip-down} für die
                               Schnittstelle durchgeführt wird, über welche die
                               Default-Route geht (kann "`yes"' oder "`no"'
                               sein)\\
\end{tabular}
\end{table}

\subsubsection{Default-Route}

Seit Version 2.1.0 werden die \texttt{ip-up}/\texttt{ip-down}-Skripte nicht nur
für die Schnittstelle ausgeführt, über welche die Default-Route geht, sondern
für alle Verbindungen, welche die \texttt{ip-up}- und \texttt{ip-down}-Skripte
aufrufen. Um das alte Verhalten zu simulieren, muss in \texttt{ip-up}- und
\texttt{ip-down}-Skripten die folgende Abfrage eingefügt werden:

\begin{example}
\begin{verbatim}
    # is a default-route-interface going up?
    if [ "$is_default_route" = "yes" ]
    then
        # die eigentlichen Aktionen
    fi
\end{verbatim}
\end{example}

Natürlich darf das neue Verhalten auch für spezielle Aktionen
ausgenutzt werden.


\section{Paket "`template"'}

Um einiges von dem hier Beschriebenen etwas zu veranschaulichen, liegt der
fli4l-Distribution das Paket "`template"' bei. Dieses erklärt an kleinen
Beispielen, wie:

\begin{itemize}
\item eine Konfigurationsdatei auszusehen hat (\texttt{config/template.txt})
\item eine Check-Datei geschrieben wird (\texttt{check/template.txt})
\item die erweiterten Prüfmöglichekiten verwendet werden (\texttt{check/template.ext})
\item Konfigurationsvariablen für spätere Verwendung abgelegt werden
  können \\(\texttt{opt/etc/rc.d/rc999.template})
\item abgelegte Konfigurationsvariablen wieder ausgelesen werden \\(\texttt{opt/usr/bin/template\_show\_config})
\end{itemize}



\section{Aufbau des Boot-Datenträgers}

Seit Version 1.5 wird das Programm \texttt{syslinux} zum Booten verwendet.
Dieses hat den Vorteil, dass ein DOS-kompatibles Dateisystem auf dem
Datenträger zur Verfügung steht.

Der Boot-Datenträger enthält folgende Dateien:

\begin{table}[htbp]
\centering
\begin{tabular}[h!]{lp{10cm}}
\texttt{ldlinux.sys}             & der Urlader ("`Boot loader"') \texttt{syslinux} \\
\texttt{syslinux.cfg}            & Konfigurationsdatei für \texttt{syslinux} \\
\texttt{kernel}                  & Linux-Kernel\\
\texttt{rootfs.img}              & RootFS: enthält zum Booten nötige Programme\\
\texttt{opt.img}                 & Optionale Dateien: Treiber und Pakete\\
\texttt{rc.cfg}                  & Konfigurationsdatei mit den benutzten Variablen aus
den Dateien des Konfigurationsverzeichnisses\\
\texttt{boot.msg}                & Texte für das \texttt{syslinux}-Bootmenü \\
\texttt{boot\_s.msg}             & Texte für das \texttt{syslinux}-Bootmenü \\
\texttt{boot\_z.msg}             & Texte für das \texttt{syslinux}-Bootmenü \\
\texttt{hd.cfg}                  & Konfigurationsdatei zur Zuordnung der Partitionen \\
\end{tabular}
\end{table}

Durch das Skript \texttt{mkfli4l.sh} (bzw. \texttt{mkfli4l.bat}) werden zunächst
die Dateien \texttt{opt.img}, \texttt{syslinux.cfg} und \texttt{rc.cfg} sowie das \texttt{rootfs.img} erzeugt.
Die dafür nötigen Dateien ermittelt das Programm \var{mkfli4l} (im \texttt{unix}- bzw.
\texttt{windows}-Unterverzeichnis). In den beiden Archiven
sind die benötigten Kernel- und andere Pakete enthalten. Die Datei
\texttt{rc.cfg} befindet sich sowohl im Opt-Archiv als auch auf dem
Boot-Datenträger.\footnote{Die Fassung im Opt-Archiv ist während der frühen
Boot-Phase nötig, weil zu diesem Zeitpunkt das Boot-Volume noch nicht
eingehängt ist.}

Anschließend werden die Dateien \texttt{kernel}, \texttt{rootfs.img}, \texttt{opt.img}
und \texttt{rc.cfg} zusammen mit den \texttt{syslinux}-Dateien auf den Datenträger kopiert.

Beim Booten von fli4l wird über das Skript \texttt{/etc/rc} die \texttt{rc.cfg} ausgewertet und das
komprimierte \texttt{opt.img}-Archiv in die RootFS-RAM-Disk integriert (je nach
Installationstyp werden dabei die Dateien direkt in die RootFS-RAM-Disk entpackt
oder über symbolische Verknüpfungen eingebunden).
Danach werden die Skripte in \texttt{/etc/rc.d/} in alphanumerischer Reihenfolge ausgeführt
und somit die Treiber geladen und die Dienste gestartet.


\section{Konfigurationsdateien}

Hier werden die Dateien kurz aufgeführt, die vom fli4l-Router
on-the-fly beim Booten erzeugt werden.

\begin{enumerate}
\item Konfiguration Provider
  \begin{itemize}
  \item         \texttt{etc/ppp/pap-secrets}

  \item         \texttt{etc/ppp/chap-secrets}

  \end{itemize}
\item Konfiguration DNS
  \begin{itemize}
  \item         \texttt{etc/resolv.conf}

  \item         \texttt{etc/dnsmasq.conf}

  \item         \texttt{etc/dnsmasq\_dhcp.conf}

  \item         \texttt{etc/resolv.dnsmasq}

  \end{itemize}


\item Hosts-Datei
  \begin{itemize}
  \item         \texttt{etc/hosts}
  \end{itemize}



\item imond-Konfiguration
  \begin{itemize}
  \item \texttt{etc/imond.conf}
  \end{itemize}

\end{enumerate}


\subsection{Konfiguration Provider}

Für den ausgesuchten Provider wird in \texttt{etc/ppp/pap-secrets}
die User-ID und das Passwort angepasst.

Beispiel für Provider Planet-Interkom:

\begin{example}
\begin{verbatim}
# Secrets for authentication using PAP
# client        server  secret                  IP addresses
"anonymer"      *       "surfer"                *
\end{verbatim}
\end{example}

Dabei ist im Beispiel "`anonymer"' die USER-ID. Als Remote-Server wird
prinzipiell jeder erlaubt (deshalb "`*"'). "`surfer"' ist das Passwort für
den Provider Planet-Interkom.


\subsection{Konfiguration DNS}


Man kann den fli4l-Router als DNS-Server einsetzen. Warum dies
sinnvoll und bei Windows-Rechnern im LAN sogar zwingend notwendig
ist, wird in der Dokumentation des "`base"'-Pakets erläutert.

Die Resolver-Datei \texttt{etc/resolv.conf} enthält den Domainnamen und den zu
verwendenten Nameserver. Sie hat folgenden Inhalt (wobei "`domain.de"' nur
ein Platzhalter für den Wert der Konfigurationsvariable
\var{DOMAIN\_\-NAME} ist):

\begin{example}
\begin{verbatim}
        search domain.de
        nameserver 127.0.0.1
\end{verbatim}
\end{example}

Der DNS-Server dnsmasq wird über die Datei \texttt{etc/dnsmasq.conf}
konfiguriert. Sie wird beim Booten vom Skript \texttt{rc040.dns-local} sowie
\texttt{rc370.dnsmasq} automatisch erzeugt und könnte wie folgt aussehen:

\begin{example}
\begin{verbatim}
user=dns
group=dns
resolv-file=/etc/resolv.dnsmasq
no-poll
no-negcache
bogus-priv
log-queries
domain-suffix=lan.fli4l
local=/lan.fli4l/
domain-needed
expand-hosts
filterwin2k
conf-file=/etc/dnsmasq_dhcp.conf
\end{verbatim}
\end{example}


\subsection{Hosts-Datei}

    Diese Datei enthält eine Zuordnung von Host-Namen zu IP-Adressen. Diese
    Zuordnung ist jedoch nur lokal auf dem fli4l verwendbar, für andere Rechner
    im LAN ist sie nicht sichtbar. Diese Datei ist eigentlich überflüssig, wenn
    zusätzlich ein lokaler DNS-Server gestartet wird.



\subsection{imond-Konfiguration}

Die Datei \texttt{etc/imond.conf} wird unter anderem aus den Konfigurationsvariablen
\var{CIRC\_x\_NAME}, \var{CIRC\_x\_ROUTE}, \var{CIRC\_x\_CHARGEINT}
und \var{CIRC\_x\_TIMES} konstruiert. Sie kann aus bis zu 32 Zeilen
(ohne die Kommentarzeilen) bestehen. Jede Zeile besteht aus acht Spalten:

\begin{enumerate}
\item  Bereich Wochentag bis Wochentag
\item  Bereich Stunde bis Stunde
\item  Device (\texttt{ippp}\emph{X} oder \texttt{isdn}\emph{X})
\item  Circuit mit Default-Route: "`yes"'/"`no"'
\item  Telefonnummer
\item  Name des Circuits
\item  Telefonkosten pro Minute in Euro
\item  Zeittakt ("`Charge interval"') in Sekunden
\end{enumerate}

    Hier ein Beispiel:

\begin{example}
\begin{verbatim}
#day  hour  device  defroute  phone        name        charge  ch-int
Mo-Fr 18-09 ippp0   yes       010280192306 Addcom      0.0248  60
Sa-Su 00-24 ippp0   yes       010280192306 Addcom      0.0248  60
Mo-Fr 09-18 ippp1   yes       019160       Compuserve  0.019   180
Mo-Fr 09-18 isdn2   no        0221xxxxxxx  Firma       0.08    90
Mo-Fr 18-09 isdn2   no        0221xxxxxxx  Firma       0.03    90
Sa-Su 00-24 isdn2   no        0221xxxxxxx  Firma       0.03    90
\end{verbatim}
\end{example}

Weitere Erklärungen zum Least-Cost-Routing findet man in der Dokumentation des "`base"'-Pakets.

\subsection{Die \texttt{/etc/.profile}-Datei}

Die Datei \texttt{/etc/.profile} enthält benutzerdefinierte Einstellungen für
die Shell. Um die Stan\-dard-Einstellungen zu überschreiben, ist es nötig,
unterhalb seines Konfigurationsverzeichnisses eine Datei \texttt{etc/.profile}
zu erstellen. Dort können dann Einstellungen zum Prompt oder Abkürzungen
(so genannte "`Aliase"') eingetragen werden.

\wichtig{Diese Datei darf kein \texttt{exit} enthalten!}

Beispiel:

\begin{example}
\begin{verbatim}
alias ll='ls -al'
\end{verbatim}
\end{example}

\subsection{Skripte in \texttt{/etc/profile.d/}}

In dem Verzeichnis \texttt{/etc/profile.d/} können Skripte abgelegt werden, die
beim Starten einer Shell ausgeführt und damit die Umgebung für die Shell
beeinflussen können. Typischerweise platzieren OPTs dort Skripte, welche
spezielle Umgebungsvariablen setzen, die für die Programme des OPTs notwendig
sind.

Falls sich sowohl Skripte in \texttt{/etc/profile.d/} befinden als auch die
Datei \texttt{/etc/.profile} existiert, werden die Skripte in
\texttt{/etc/profile.d/} \emph{nach} dem Skript \texttt{/etc/.profile}
ausgeführt.

\section{Inkompatibilitäten zwischen 3.x und 4.x}

Beim Umstellen von Paketen von Version 3.x auf 4.x sind die folgenden Dinge
zu beachten:

\begin{itemize}
\item Die zugrunde liegende $\mu$Clibc-Bibliothek ist aktualisiert worden. Deswegen
sollten alle Binärprogramme, die gegen eine ältere $\mu$Clibc gebunden wurden, neu
übersetzt werden. Das dafür nötige FBR (\emph{f}li4l \emph{B}uild\emph{r}oot)
ist im \emph{src}-Paket zu finden.

\item Das Circuit-Konzept ist stark ausgebaut worden, so dass die Skripte in
\texttt{/etc/ppp} häufiger aufgerufen werden als vorher (z.B. auch für alle
konfigurierten Routen). Des Weiteren können jetzt dieselben Skripte für
verschiedene Circuits parallel ausgeführt werden. Greift ein \texttt{ip-up}-
oder ein \texttt{ip-down}-Skript auf eine globale Ressource zu (z.B. eine Datei,
die von allen Circuits gleichermaßen genutzt wird), muss diese Ressource für die
Dauer der Bearbeitung gesperrt und hinterher wieder freigegeben werden. Die
dafür nötigen Funktionen \texttt{sync\_lock\_resource} und
\texttt{sync\_unlock\_resource} werden in Abschnitt \ref{TODO} beschrieben.

\item Der Circuit und das Gerät \texttt{pppoe} existiert nur noch aus
Kompatibilitätsgründen und sollte nicht mehr benutzt werden. Es repräsentiert
den ersten konfigurierten PPPoE-Circuit bzw. die ihm zugrunde liegende
Netzwerkschnittstelle. PPP-Circuits, die \emph{keine} PPPoE-Circuits sind (dies
trifft z.B. auf PPTP-, Fritz!DSL- oder UMTS-Circuits zu), werden über diesen
Alias \emph{nicht} mehr repräsentiert. Des Weiteren sollte in den
Circuit-Skripten unterhalb von \texttt{/etc/ppp} die Variable
\texttt{real\_interface} nicht mehr genutzt werden. Statt dessen sollte
die Variable \texttt{interface} verwendet werden. Eine Abbildung von der
tatsächlichen PPP-Schnittstelle auf \texttt{pppoe} findet nicht mehr statt, da
nicht mehr davon ausgegangen werden darf, dass es höchstens einen DSL-Circuit
gibt. Es ist somit auch keine Prüfung mehr vorzunehmen, ob die Variable
\texttt{interface} den Wert \texttt{pppoe} beinhaltet.
\end{itemize}

