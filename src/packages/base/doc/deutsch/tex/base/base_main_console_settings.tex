% Last Update: $Id$
\marklabel{CONSOLESETTINGS}{\section{Konsolen-Einstellungen}}

fli4l kann auf verschiedenen Hardware-Plattformen betrieben werden.
Auf vielen dieser Plattformen ist es möglich, eine Tastatur und einen Monitor
anzuschließen, um mit dem fli4l zu interagieren; diese Kombination der Ein-
und Ausgabe wird generell \emph{Konsole} genannt.

fli4l kann aber auch gänzlich ohne Tastatur und Grafikkarte eingesetzt
werden. Damit man den Router dann auch ohne Netzwerkzugriff bedienen
sowie alle Boot-Meldungen des Kernels sehen kann, ist es u.\,a.\ möglich, über
die serielle Schnittstelle ebenfalls eine Konsole zu erhalten, indem die
Ein- und Ausgaben von der serielle Schnittstelle bezogen bzw. dorthin
ausgegeben werden. Dazu müssen die Variablen
\jump{SERCONSOLE}{\var{SER\_CONSOLE}},
\jump{SERCONSOLEIF}{\var{SER\_CONSOLE\_IF}} und
\jump{SERCONSOLERATE}{\var{SER\_CONSOLE\_RATE}} gesetzt bzw. angepasst werden.

Schließlich ist es möglich, zur selben Zeit sowohl über Tastatur und
Monitor als auch über die serielle Schnittstelle eine Konsole zur Verfügung
zu stellen.

Generell stellt fli4l auf \emph{jeder} Konsole die Möglichkeit zur
Anmeldung und somit eine \emph{Shell} zur Verfügung, an der Sie sich mit dem
Benutzer ``fli4l'' und dem über die Variable \jump{PASSWORD}{\var{PASSWORD}}
konfigurierten Passwort anmelden können.

\begin{description}
  \config{CONSOLE\_BLANK\_TIME}{CONSOLE\_BLANK\_TIME}{CONSOLEBLANKTIME}
  
  Standard-Einstellung: \var{CONSOLE\_BLANK\_TIME=''}
  
  Normalerweise aktiviert der Linux-Kern nach einer gewissen Zeit ohne
  Eingaben auf dem aktuellen Eingabegerät den Bildschirmschoner. Mit
  der Variable \var{CONSOLE\_BLANK\_TIME} kann man diese Zeit konfiguriereren
  bzw. den Bildschirmschonermodus ganz deaktivieren
  (\var{CONSOLE\_BLANK\_TIME}='0').

  \config{BEEP}{BEEP}{BEEP}
  
  Standard-Einstellung: \var{BEEP='yes'}
  
  {Signalton am Ende des Boot- und Shutdownprozesses ausgeben.

    Trägt man hier `yes' ein, ertönt
    ein Signalton am Ende des Boot- und Shutdownprozesses. Wer extremen
    Platzmangel auf dem Bootmedium hat oder keinen Signalton möchte,
    kann hier also `no' eingetragen lassen.}

  \config{SER\_CONSOLE}{SER\_CONSOLE}{SERCONSOLE}

    Standard-Einstellung: \var{SER\_CONSOLE='no}'

    Diese Variable aktiviert oder deaktiviert eine Konsole auf einer seriellen
    Schnittstelle. Die serielle Konsole kann in drei Modi betrieben werden:

      \begin{tabular}[h!]{|l|p{9cm}|}
        \hline
        \var{SER\_CONSOLE} & Konsolen-Ein-/Ausgabe \\
        \hline
        no & Ein- und Ausgabe (nur) über Tastatur und Monitor (tty0) \\
        yes & Ein- und Ausgabe (nur) über serielle Schnittstelle (ttyS0) \\
        primary & Ein- und Ausgabe sowohl über serielle Konsole als auch
        über Tastatur und Monitor, Ausgabe der Kernelmeldungen auf tty0 \\
        secondary & Ein- und Ausgabe sowohl über serielle Konsole als auch
        über Tastatur und Monitor, Ausgabe der Kernelmeldungen auf ttyS0 \\
        \hline
      \end{tabular}

    Wenn der Wert von \var{SER\_CONSOLE} verändert wird, wird diese
    Änderung nur bei der Erstellung eines neuen Bootmediums oder beim
    Remote-Update der syslinux.cfg wirksam.

    \wichtig{Achten Sie beim Ausschalten der seriellen Konsole darauf, sich
    einen alternativen Zugang zum Router (SSH oder direkt über Tastatur und
    Monitor) zu erhalten!}

    Weitere Informationen sind im Anhang unter
    \jump{SERIALCONSOLE}{Serielle Konsole} zu finden.


  \config{SER\_CONSOLE\_IF}{SER\_CONSOLE\_IF}{SERCONSOLEIF}
  
    Standard-Einstellung: \var{SER\_CONSOLE\_IF='0'}
    
    {Nummer der seriellen Schnittstelle für die serielle Konsole.

    Hier ist die Nummer der Schnittstelle einzutragen, an die die serielle
    Konsole angeschlossen wird.
    0 entspricht ttyS0 unter Linux bzw. COM1 unter Microsoft Windows.}


  \config{SER\_CONSOLE\_RATE}{SER\_CONSOLE\_RATE}{SERCONSOLERATE}
    
    Standard-Einstellung: \var{SER\_CONSOLE\_RATE='9600'}
    
    {Übertragungsrate der seriellen Schnittstelle für die Konsolenausgabe.

    Hier trägt man die Baud-Rate ein, mit der die Daten auf der
    seriellen Schnittstelle übertragen werden.  Sinnvolle Werte: 4800,
    9600, 19200, 38400, 57600, 115200.}

\end{description}
