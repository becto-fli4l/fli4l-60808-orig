% Last Update: $Id$
\section{Hilfen zum Einkreisen von Problemen und Fehlern}

fli4l loggt die gesamten Ausgaben des Bootvorganges in einer Datei
(\emph{/var/tmp/boot.log}). Diese Datei kann man sich am Ende des
Bootvorganges auf der Konsole oder über den entsprechenden
Menüpunkt im Web-Interface ansehen.

Manchmal ist es jedoch sinnvoll, bei Problemen einen ausführlicheren
Ablauf der Start-Sequenz zu generieren, um den Bootvorgang hinterher auf
Probleme untersuchen zu können. Dazu dient \var{DEBUG\_STARTUP}. Andere
Einstellungen unterstützen Entwickler beim Finden von Fehlern in bestimmten
Situationen; auch diese Einstellungen werden in diesem Abschnitt dokumentiert.

\begin{description}

  \config{DEBUG\_STARTUP}{DEBUG\_STARTUP}{DEBUGSTARTUP}
    
    Standard-Einstellung: \var{DEBUG\_STARTUP='no'}

    Steht dieser Wert auf `yes', wird beim Booten jedes ausgeführte
    Kommando vor seiner Ausführung auf den Schirm geschrieben. Da für
    das korrekte Funktionieren Änderungen an der syslinux.cfg
    vorgenommen werden müssen, gilt das für \var{SER\_CONSOLE}
    Gesagte auch hier. Wenn man die syslinux.cfg von Hand ergänzen will,
    ist es nötig, ein \verb+fli4ldebug=yes+ einzufügen. \var{DEBUG\_STARTUP} muss
    dann aber trotzdem auf `yes' stehen.

    \config{DEBUG\_MODULES}{DEBUG\_MODULES}{DEBUGMODULES} 
    
    Standard-Einstellung: \var{DEBUG\_MODULES='no'}
    
    Einige Module werden automatisch vom Kern geladen, ohne dass man das
    vorher erkennen kann. \var{DEBUG\_MODULES='yes'} aktiviert einen
    Modus, der einem die kompletten Modulladesequenzen zeigt, egal, ob
    sie von einem Skript oder vom Kern angestoßen werden.

    \config{DEBUG\_ENABLE\_CORE}{DEBUG\_ENABLE\_CORE}{DEBUGENABLECORE}
    
    Standard-Einstellung: \var{DEBUG\_ENABLE\_CORE='no'}
    
    Wird diese Option aktiviert, verursacht jeder Programmabsturz auf dem
    Router das Erzeugen einer so genannten ``core''-Datei, also eines
    Speicherabbilds des Prozesses direkt vor dem Absturz. Diese Dateien sind
    auf dem Router im Verzeichnis \texttt{/var/log/dumps} zu finden. Diese
    Dateien können dann genutzt werden, um den Programmfehler besser zu finden.
    Genaueres finden Sie hierzu im Abschnitt \jump{sec:debugging}{``Entwanzen
    von Programmen auf dem fli4l''} in der Dokumentation des SRC-Pakets.

    \config{DEBUG\_MDEV}{DEBUG\_MDEV}{DEBUGMDEV}
    
    Standard-Einstellung: \var{DEBUG\_MDEV='no'}
    
    Mit \var{DEBUG\_MDEV='yes'} werden alle Aktionen, die in Zusammenhang mit
    dem \texttt{mdev}-Dämon stehen und somit mit dem Hinzufügen oder Entfernen
    von Geräteknoten in \texttt{/dev} oder dem Laden von Firmware zu tun haben,
    in der Datei \texttt{/dev/mdev.log} protokolliert.

    \config{DEBUG\_IPTABLES}{DEBUG\_IPTABLES}{DEBUGIPTABLES}
    
    Standard-Einstellung: \var{DEBUG\_IPTABLES='no'}
    
    Mit \var{DEBUG\_IPTABLES='yes'} werden alle \texttt{iptables}-Aufrufe
    inklusive dem Rückgabewert in \texttt{/var/log/iptables.log} protokolliert.

    \config{DEBUG\_IP}{DEBUG\_IP}{DEBUGIP}
    
    Standard-Einstellung: \var{DEBUG\_IP='no'}
    
    Diese Variable aktiviert bei \var{DEBUG\_IP='yes'} das Protokollieren aller
    Aufrufe des Programms \texttt{/sbin/ip} in der Datei
    \texttt{/var/log/wrapper.log}.

\end{description}
