% Last Update: $Id$
\section{Zusätzliche Routen (IPv6)}

IPv6-Routen sind Wege für IPv6-Pakete. Damit der Router weiß, welches eingehende
Paket er wohin schicken soll, greift er auf eine Routing-Tabelle zurück, in der
genau diese Informationen zu finden sind. Im Falle von IPv6 ist es wichtig zu
wissen, wohin IPv6-Pakete geschickt werden, die nicht ins lokale Netz sollen.

\begin{description}
\config{IPV6\_ROUTE\_N}{IPV6\_ROUTE\_N}{IPV6ROUTEN}{
Diese Variable legt die Anzahl der zu spezifizierenden IPv6-Routen fest. In der
Regel werden keine zusätzlichen IPv6-Routen benötigt.

Standard-Einstellung:
}
\verb*?IPV6_ROUTE_N='0'?

\config{IPV6\_ROUTE\_x}{IPV6\_ROUTE\_x}{IPV6ROUTEx}{
Diese Variable enthält die Route in der Form `Zielnetz Gateway', wobei das
Zielnetz in der CIDR-Notation erwartet wird. Für die Default-Route muss als
Zielnetz \var{::/0} verwendet werden.

Beispiel:
}
\verb*?IPV6_ROUTE_1='2001:db8:1743:44::/64 2001:db8:1743:44::1'?
\end{description}
