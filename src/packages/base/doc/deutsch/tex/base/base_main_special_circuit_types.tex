\section{Spezielle Circuit-Typen im base-Paket}

\marklabel{sect:route-circuits}{\subsection{Circuits vom Typ ``route''}}

Ein Circuit vom Typ ``route'' dient dazu, eine vorkonfigurierte \emph{Route}
zu verwalten. Das Einwählen entspricht dem Aufbau, der Route, das Auflegen dem
Abbau. Ein einzelner route-Circuit kann das Routen mehrerer IPv4- und
IPv6-Netze (letzteres nur bei aktivierter IPv6-Unterstützung) gleichzeitig
steuern. Insofern ist ein route-Circuit flexibler als die Konfiguration einer
Route via \var{IP\_ROUTE\_x}, weil dort immer nur ein Netz pro Route angegeben
werden kann.

Neben den allgemeinen Circuit-Variablen, die in Abschnitt
\jump{sec:circuits:general}{``Circuits allgemein''} beschrieben sind, sind
zusätzlich die folgenden Angaben erforderlich:

\begin{description}

\config{CIRC\_x\_ROUTE\_DEV}{CIRC\_x\_ROUTE\_DEV}{CIRCxROUTEDEV}

In dieser Variablen ist die Netzwerk-Schnittstelle vermerkt, über welche die
Route geht. Diese Angabe ist optional, sofern mindestens eine Gateway-Angabe
(siehe unten) vorliegt und daraus die Netzwerk-Schnittstelle abgeleitet werden
kann. Es können Schnittstellen-Namen (``eth0''), Schnittstellen-Referenzen
(``IP\_NET\_1\_DEV'', ``IPV6\_NET\_1\_DEV''), Circuits (``ppp0'') und
Schlagwörter (``internet-v4'') genutzt werden, wobei Routen über dynamische,
von Circuits kontrollierte Schnittstellen eher esoterischer Natur sind und nur
dann verwendet werden sollten, wenn man sich über die auftretenden dynamischen
Effekte im Klaren ist.

Wenn diese Variable leer ist und beide Gateway-Adressen (IPv4 und IPv6) gesetzt
sind (s.\,u.), müssen diese derselben Schnittstelle zugeordnet sein. Das
liegt daran, dass jedem Circuit genau eine Netzwerkschnittstelle zugeordnet
sein muss (und nicht mehrere).

Beispiel 1: \verb+CIRC_x_ROUTE_DEV='IP_NET_2_DEV'+

Beispiel 2: \verb+CIRC_x_ROUTE_DEV='pppoe-v6'+

\config{CIRC\_x\_ROUTE\_GATEWAY\_IPV4}{CIRC\_x\_ROUTE\_GATEWAY\_IPV4}{CIRCxROUTEGATEWAYIPV4}

Diese Variable enthält die Adresse des Next-Hop-Rou\-ters (gemeinhin ``Gateway''
genannt) für die zu routenden IPv4-Netze. Sie kann nur dann leer sein, wenn
\var{CIRC\_x\_ROUTE\_DEV} gesetzt ist; in diesem Fall wird eine Route ohne
Gateway generiert. Die Gateway-Adresse kann mit einem Circuit kombiniert
werden, wobei Routen über dynamische, von Circuits kontrollierte Gateways eher
esoterischer Natur sind und nur dann verwendet werden sollten, wenn man sich
über die auftretenden dynamischen Effekte im Klaren ist.

Beispiel 1: \verb+CIRC_x_ROUTE_GATEWAY_IPV4='192.168.1.254'+

Beispiel 2 (esoterisch): \verb+CIRC_x_ROUTE_GATEWAY_IPV4='{dhcp}0.0.0.253'+

Im letzten Beispiel wird die Gateway-Adresse dynamisch aus dem per DHCP
zugewiesenen Netz (z.\,B. 192.168.12.0/24) und dem angegebenen Host-Anteil
gebildet (in diesem Fall also 192.168.12.253). Dies ist nur dann sinnvoll, wenn
man weiß, dass egal welches Netzpräfix dem fli4l-Router von dem DHCP-Server
zugewiesen wird, ein bestimmter Router an der Host-Adresse 253 zu finden ist.
Normalerweise sollte der via DHCP übermittelte Router alle Routen behandeln
können, somit sollte man die zu routenden Netze in einem solchen Fall direkt
im jeweiligen DHCP-Circuit vermerken und nicht einen route-Circuit dafür
verwenden.

\config{CIRC\_x\_ROUTE\_GATEWAY\_IPV6}{CIRC\_x\_ROUTE\_GATEWAY\_IPV6}{CIRCxROUTEGATEWAYIPV6}

Diese Variable enthält die Adresse des Next-Hop-Rou\-ters (gemeinhin ``Gateway''
genannt) für die zu routenden IPv6-Netze. Es gilt all das für
\var{CIRC\_x\_ROUTE\_GATEWAY\_IPV4} Gesagte. Weiterhin kann diese Variable nur
bei aktivierter IPv6-unterstützung genutzt werden.

Beispiel 1: \verb+CIRC_x_ROUTE_GATEWAY_IPV6='2001:db8:42::1'+

Beispiel 2 (esoterisch): \verb+CIRC_x_ROUTE_GATEWAY_IPV6='{dhcpv6}::0:0:0:254'+

\end{description}
