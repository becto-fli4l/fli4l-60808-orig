% Last Update: $Id$
\chapter{Basiskonfiguration}

Ab Version 2.0 ist die fli4l-Distribution modular aufgebaut und in
mehrere Pakete aufgeteilt, die extra heruntergeladen werden müssen. Im
Paket \texttt{fli4l-\version.tar.gz} ist lediglich die
Basis-Software für einen Ethernet-Router enthalten. Für DSL, ISDN und
weitere Software müssen die Pakete separat heruntergeladen werden und
ausgehend vom Verzeichnis \texttt{fli4l-\version/} (!) installiert
werden. Durch die Auswahlmöglichkeit des Betriebssystemkerns von fli4l
sind diese in die Kernel Pakete ausgelagert worden. Somit ist als Minimum
Basis und ein Kernel Paket erforderlich.
In Tabelle \ref{tab:zusatzpakete} finden Sie einen
Überblick über die Zusatzpakete.

\begin{table}[ht!]
 \caption{Übersicht über die (Zusatz-)Pakete}\marklabel{tab:zusatzpakete}{}
  \begin{center}
    \begin{tabular}{ll}
      \textbf{Download-Archiv}        &    \textbf{Paket} \\
      \hline
      \texttt{fli4l-\version}         &    BASIS, erforderlich!\\
      \verb*zkernel_4_19z             &    Linux-Kernel, erforderlich!\\
      \texttt{fli4l-\version-doc}     &    Komplette Dokumentation\\
      \verb*zadvanced_networkingz     &    Erweiterte Netzwerkkonfiguration\\
      \verb*zcertz                    &    Zertifikatsverwaltung\\
      \verb*zchronyz                  &    Time-Server/Client\\
      \verb*zdhcp_clientz             &    Verschiedene DHCP-Clients\\
      \verb*zdns_dhcpz                &    DNS- und DHCP-Server\\
      \verb*zdslmodemz                &    Unterstützung für interne DSL-Modems (z.B. AVM Fritz!DSL)\\
      \verb*zdyndnsz                  &    Unterstützung von DYNDNS-Diensten\\
      \verb*zeasycronz                &    Zeitplandienst\\
      \verb*zhdz                      &    Installation auf Festplatte\\
      \verb*zhttpdz                   &    Mini-Webserver für Status-Ausgaben\\
      \verb*zhwsuppz                  &    Unterstützung von Hardware\\
      \verb*zimonc_windowsz           &    Der Windows-Imonc\\
      \verb*zimonc_unixz              &    Der GTK-Unix-Imonc\\
      \verb*zipv6z                    &    Internet Protokoll Version 6\\
      \verb*zisdnz                    &    ISDN-Router\\
      \verb*zopenvpnz                 &    OpenVPN-Unterstützung\\
      \verb*zpcmciaz                  &    Unterstützung von PCMCIA-Karten\\
      \verb*zpppz                     &    PPP-Basispaket\\
      \verb*zpppoez                   &    DSL-Router (PPPoE)\\
      \verb*zproxyz                   &    Proxy-Server\\
      \verb*zqosz                     &    Quality of Service\\
      \verb*zsshdz                    &    SSH-Server\\
      \verb*ztoolsz                   &    Diverse Linux-Werkzeuge\\
      \verb*zumtsz                    &    Anbindung mittels UMTS an das Internet\\
      \verb*zusbz                     &    Unterstützung der USB-Schnittstelle\\
      \verb*zvpnz                     &    VPN (PPTP)\\
      \verb*zwlanz                    &    Unterstützung von WLAN-Karten
    \end{tabular}
  \end{center}
\end{table}

Die zur Konfiguration des fli4l-Routers verwendeten Dateien
befinden sich im Verzeichnis \texttt{config/} und werden hier im Folgenden
beschrieben.

Diese Dateien können mit einem \emph{einfachen} Text-Editor oder auch
mit einem speziell an fli4l angepassten Editor verändert werden. Diverse
Editoren sind unter

\par

\altlink{http://www.fli4l.de/download/zusatzpakete/addons/} zu finden.

Sind spezielle Anpassungen/Erweiterungen erforderlich, die über die
unten aufgeführten Einstellungsmöglichkeiten hinausgehen, benötigt man
ein lauffähiges Linux-System, um Anpassungen im RootFS vorzunehmen. In
diesem Fall hilft \verb+src/README+ weiter.

\newpage
