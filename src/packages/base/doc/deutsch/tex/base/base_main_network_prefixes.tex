% Last Update: $Id$
\section{Netzwerkpräfix-Konfiguration}

\begin{description}
  \config{OPT\_NET\_PREFIX}{OPT\_NET\_PREFIX}{OPTNETPREFIX}{
  Aktiviert die Unterstützung für selbstdefinierte Netzwerkpräfixe.

  Ein Netzwerkpräfix ist technisch nichts anderes als die Adresse eines
  Netzwerks, allerdings steht es i.~d.~R.\ für ein Netz, das weiter unterteilt
  werden soll. Sinnvoll ist dies vor allem dann, wenn ein fli4l-Router ein
  Netzwerk nicht alleine verwaltet, sondern Teilnetze daraus anderen Routern
  zur Verwaltung überlässt. Durch die Definition und somit die Benennung des
  insgesamt zur Verfügung und Verteilung stehenden Netzes ist es möglich, die
  Netzwerk-Adresse an mehreren Stellen zu verwenden, ohne den gemeinsamen
  Präfix immer wieder hinschreiben zu müssen.

  Konkrete Beispiele, wie man ein Netzwerkpräfix definiert, sind weiter unten
  bei den verschiedenen Typen von Netzwerkpräfixen zu finden.

  Standard-Einstellung:
  }
  \verb*?OPT_NET_PREFIX='yes'?

  \config{NET\_PREFIX\_x}{NET\_PREFIX\_x}{NETPREFIXx}{
  
  Über dieses Array werden die verschiedenen Netzwerkpräfixe definiert.
  Die einzelnen Komponenten werden im Anschluss erklärt.
  }

  \config{NET\_PREFIX\_x\_NAME}{NET\_PREFIX\_x\_NAME}{NETPREFIXxNAME}{
  Name des Netzwerkpräfixes.

  Diese Variable enthält den Namen des Präfixes. Dieser Name kann dann in
  Adressangaben verwendet werden, um das Präfix zu benutzen. Dabei wird der
  Name analog zu Circuit-Namen verwendet, d.~h.\ er muss in geschweiften
  Klammern geschrieben werden.
  }

  \config{NET\_PREFIX\_x\_TYPE}{NET\_PREFIX\_x\_TYPE}{NETPREFIXxTYPE}{
  Typ des Netzwerkpräfixes.

  Diese Variable enthält den Typ des Präfixes. Die unterstützten Typen werden
  in Tab.~\ref{base:net:prefix:types} erläutert.

  \begin{center}
      \begin{longtable}{|l|p{0.7\textwidth}|}
          \hline
          \multicolumn{1}{|l}{\textbf{Typ}} &
          \multicolumn{1}{|l|}{\textbf{Bedeutung}} \\
          \hline
          \endhead
          \hline
          \endfoot
          \endlastfoot
          static        & Das Netzwerkpräfix wird direkt als feste Adresse
                          angegeben.
                          \\
          generated-ula & Das Netzwerkpräfix ist eine vom fli4l generierte
                          ULA\footnote{``Unique Local Address''} gemäß RFC
                          4193.\footnote{\altlink{https://tools.ietf.org/html/rfc4193}}
                          Wenn der fli4l Zugriff auf persistenten Speicher hat,
                          dann wird das Präfix nur einmal generiert, so dass es
                          auch über Neustarts des Routers hinweg stabil bleibt.
                          \\
          \hline
          \caption{Typen von Netzwerkpräfixen}\label{base:net:prefix:types}
      \end{longtable}
  \end{center}
  }
\end{description}

\subsection{Netzwerkpräfixe vom Typ ``stable''}
Für Netzwerkpräfixe vom Typ ``stable'' gibt es die folgenden Einstellungen:

\begin{description}
  \configlabel{NET\_PREFIX\_x\_STATIC\_IPV6}{NETPREFIXxSTATICIPV6}
  \config{NET\_PREFIX\_x\_STATIC\_IPV4 NET\_PREFIX\_x\_STATIC\_IPV6}{NET\_PREFIX\_x\_STATIC\_IPV4}{NETPREFIXxSTATICIPV4}{
  Adresse(n) des Netzwerkpräfixes.

  Mit Hilfe dieser Einstellung kann die IPv4- und/oder die IPv6-Adresse des
  Netzwerkpräfixes angegeben werden.

  Beispiel:
  }
  \begin{example}
  \begin{verbatim}
    NET {
      PREFIX {
        [] {
          NAME='site'
          TYPE='static'
          STATIC {
            IPV4='10.1.0.0/16'
            IPV6='fdce:1c35:301f::/48'
          }
        }
      }
    }
  \end{verbatim}
  \end{example}

\end{description}

\subsection{Netzwerkpräfixe vom Typ ``generated-ula''}
Für Netzwerkpräfixe vom Typ ``generated-ula'' gibt es die folgenden
Einstellungen:

\begin{description}
  \config{NET\_PREFIX\_x\_ULA\_DEV}{NET\_PREFIX\_x\_ULA\_DEV}{NETPREFIXxULADEV}{
  Ethernet-Schnittstelle.

  Mit Hilfe dieser Einstellung wird die Ethernet-Schnittstelle angegeben, deren
  MAC-Adresse für die Generierung der ULA herangezogen wird.

  Beispiel:
  }
  \begin{example}
  \begin{verbatim}
    NET {
      PREFIX {
        [] {
          NAME='site'
          TYPE='generated-ula'
          ULA {
            DEV='eth0'
          }
        }
      }
    }
  \end{verbatim}
  \end{example}

\end{description}
