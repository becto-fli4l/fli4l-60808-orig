% Synchronized to r32677
\section {APCUPSD -- Daemon for APC USVs}

\subsection{Introduction}

  This package provides the APCUPSD daemon {[\ref{apcupsd}]} for monitoring
  APC-USVs on fli4l. All settings are taken from the orginal package
  {[\ref{apcupsd_manual}]}.
 
\subsection {Configuration of the APCUPSD package}
     
  The configuration is made, as of all fli4l packages, by adjusting the file\\
  \var{path/fli4l-\version/$<$config$>$/apcupsd.txt} to meet your own demands.
     
\begin{description}

\config {OPT\_APCUPSD}{OPT\_APCUPSD}{OPTAPCUPSD}

  The setting  \var{'no'} deactivates OPT\_APCUPSD ompletely. There will be no
  changes made on the fli4l boot medium or the archive \var{opt.img}.
  OPT\_APCUPSD does not overwrite other parts of the fli4l installation.
  To activate OPT\_APCUPSD set the variable \var{OPT\_APCUPSD} to \var{'yes'}.

\end{description}

\subsection {Communication settings}
\begin{description}

\config {APCUPSD\_UPSNAME}{APCUPSD\_UPSNAME}{APCUPSDUPSNAME}

  Use this to give your UPS a name in log files and such.
  
  This is particulary useful if you have multiple UPSes. This does not
  set the EEPROM. It should be 8 characters or less.


\config {APCUPSD\_UPSCABLE}{APCUPSD\_UPSCABLE}{APCUPSDUPSCABLE} 
 
  Defines the type of cable connecting the UPS to fli4l.
   
  Possible generic choices for \var{APCUPSD\_UPSCABLE} are:
 
   \var{'simple'}, \var{'smart'}, \var{'ether'} or \var{'usb'}
  
  Or a specific cable model number may be used:  

  \var{'940-0119A'}, \var{'940-0127A'}, \var{'940-0128A'}, \var{'940-0020B'},
  \var{'940-0020C'}, \var{'940-0023A'}, \var{'940-0024B'}, \var{'940-0024C'},
  \var{'940-1524C'}, \var{'940-0024G'}, \var{'940-0095A'}, \var{'940-0095B'},
  \var{'940-0095C'}or \var{'M-04-02-2000'}

\config {APCUPSD\_UPSTYPE} {APCUPSD\_UPSTYPE}{APCUPSDUPSTYPE}

  To get apcupsd to work, in addition to defining the 
  \smalljump{APCUPSDUPSCABLE}{\var{APCUPSD\_UPSCABLE}},
  you must also define a UPS type , which corresponds to
  the type of UPS you have set in 
  \smalljump{APCUPSDUPSDEVICE}{\var{APCUPSD\_UPSDEVICE}}. 

\config {APCUPSD\_UPSDEVICE} {APCUPSD\_UPSDEVICE}{APCUPSDUPSDEVICE}

  You must also specify a device, sometimes referred to as a port.
  
  For USB UPSes, please leave the \var{APCUPSD\_UPSDEVICE} directive blank. 
  For other UPS types, you must specify an appropriate port or address as 
  described in the following table:
   
\begin{tabular}{p{20mm}p{120mm}}
  UPS type & Device \\ & Description
  \\\\ 
  
  \var{'apcsmart'} & \var{'/dev/tty*'} \\ &
  Newer serial character device, appropriate for SmartUPS models using
  a serial cable (not USB).
  \\\\

  \var{'usb'} & \var{''} \\ &
  Most new UPSes are USB. A blank \var{APCUPSD\_UPSDEVICE} setting enables
  autodetection, which is the best choice for most installations. \\\\

  \var{'net'} & \var{'hostname:port'} \\ & 
  Network link to a master apcupsd through apcupsd's Network Information Server.
  This is used if the UPS powering the fli4l is connected to a 
  different computer for monitoring.
  \\\\

%   snmp & hostname:port:vendor:community \\ & 
%   SNMP network link to an SNMP-enabled UPS device.
%   Hostname is the ip address or hostname of the UPS 
%   on the network. Vendor can be can be 'APC' or 
%   'APC\_NOTRAP'. 'APC\_NOTRAP' will disable SNMP trap 
%   catching; you usually want 'APC'. Port is usually 
%   161. Community is usually 'private'.
%   (disabled in build from apcupsd binary) \\
% 
%   netsnmp & :port:vendor:community \\ &
%   OBSOLETE \\ &
%   Same as SNMP above but requires use of the 
%   net-snmp library. Unless you have a specific need
%   for this old driver, you should use 'snmp' instead.
%   (disabled in build from apcupsd binary) \\
% 
%   dumb  & /dev/tty** \\ &
%   Old serial character device for use with simple-signaling UPSes. \\

  \var{'pcnet'} & \var{'ipaddr:username:passphrase[:port]'} \\ &
  PowerChute Network Shutdown protocol which can be 
  used as an alternative to SNMP with the AP9617 
  family of smart slot cards.
  
  ipaddr is the IP address of the UPS management card. 
  
  username and passphrase are the credentials for which the card 
  has been configured. 
  
  port is the port number on which to listen for messages from the UPS, normally 
  3052. If this parameter is empty or missing, the 
  default of 3052 will be used.
  \\
\end{tabular}

\config {APCUPSD\_POLLTIME} {APCUPSD\_POLLTIME}{APCUPSDPOLLTIME}

  Interval in seconds at which apcupsd polls the UPS for status.
  
  This setting applies both to directly-attached UPSes 
  (\smalljump{APCUPSDUPSTYPE}{\var{APCUPSD\_UPSTYPE}} \var{'apcsmart'}, \var{'usb'}) 
  and networked UPSes (\smalljump{APCUPSDUPSTYPE}{\var{APCUPSD\_UPSTYPE}} 
  \var{'net'}, \var{'pcnet'}). 
  Lowering this setting will improve apcupsd's responsiveness to certain
  events at the cost of higher CPU utilization (default \var{'60'}).
  The default of 60 seconds is appropriate for most situations.

\end {description}

\subsection{Directory settings}
\begin {description}

\config {APCUPSD\_LOCKFILE} {APCUPSD\_LOCKFILE}{APCUPSDLOCKFILE}
  
  Path for device lock file (default \var{'/var/lock'}).
  

\config {APCUPSD\_SCRIPTDIR} {APCUPSD\_SCRIPTDIR}{APCUPSDSCRIPTDIR}
    
  Path to script directory in which apccontrol and event scripts are located.
  (default \var{'/etc'})


\config {APCUPSD\_PWRFAILDIR} {APCUPSD\_PWRFAILDIR}{APCUPSDPWRFAILDIR}

  Path to powerfail directory in which to write the powerfail flag file. 
  
  This file is created when apcupsd initiates a system shutdown and is
  checked in the OS halt scripts to determine if a killpower
  (turning off UPS output power) is required (default \var{'/etc'}).


\config {APCUPSD\_NOLOGINDIR} {APCUPSD\_NOLOGINDIR}{APCUPSDNOLOGINDIR}

  Path to nologin directory in which to write the nologin file.
  The existence of this flag file tells the OS to disallow new logins 
  (default \var{'/etc'}).

\end {description}

\subsection{Powerfail settings}
\begin {description}

\config {APCUPSD\_ONBATTERYDELAY} {APCUPSD\_ONBATTERYDELAY}{APCUPSDONBATTERYDELAY}
  
  The time in seconds from when a power failure is detected until we react
  to it with an onbattery event (default \var{'6'})

  This means that, apccontrol will be called with the powerout argument
  immediately when a power failure is detected.  However, the
  onbattery argument is passed to apccontrol only after the 
  \var{APCUPSD\_ONBATTERYDELAY} time.  
  If you don't want to be annoyed by short powerfailures, make sure that 
  apccontrol powerout does nothing i.e. comment out the wall.

\config {APCUPSD\_BATTERYLEVEL} {APCUPSD\_BATTERYLEVEL}{APCUPSDBATTERYLEVEL}
 
  If during a power failure, the remaining battery percentage
  (as reported by the UPS) is below or equal to \var{APCUPSD\_BATTERYLEVEL}, 
  apcupsd will initiate a system shutdown (default \var{'5'})

\config {APCUPSD\_MINUTES} {APCUPSD\_MINUTES}{APCUPSDMINUTES}
 
  If during a power failure, the remaining runtime in minutes 
  (as calculated internally by the UPS) is below or equal to 
  \var{APCUPSD\_MINUTES},  apcupsd, will initiate a system shutdown
  (default \var {'3'}).

\config {APCUPSD\_TIMEOUT} {APCUPSD\_TIMEOUT}{APCUPSDTIMEOUT}
 
  If during a power failure, the UPS has run on batteries for 
  \var{APCUPSD\_TIMEOUT} seconds or longer, 
  apcupsd will initiate a system shutdown (default \var{'0'}).
  A value of \var{'0'} disables this timer.

  Note, if you have a Smart UPS, you will most likely want to disable
  this timer by setting it to zero. That way, you UPS will continue
  on batteries until either the \% charge remaing drops to or below
  \var{APCUPSD\_BATTERYLEVEL}, or the remaining battery runtime drops to or
  below \var{APCUPSD\_MINUTES}.
  Of course, if you are testing, setting this to \var{'60'} causes a quick 
  system shutdown if you pull the power plug.   
  If you have an older dumb UPS, you will want to set this to less than
  the time you know you can run on batteries.
 
  Note: \var{APCUPSD\_BATTERYLEVEL}, \var{APCUPSD\_MINUTES} and 
  \var{APCUPSD\_TIMEOUT} work in conjunction, so
  the first that occurs will cause the initation of a shutdown.


\config {APCUPSD\_ANNOY} {APCUPSD\_ANNOY}{APCUPSDANNOY}
  
  Time in seconds between annoying users to signoff prior to
  system shutdown (default \var{'300'}). \var{'0'} disables. 


\config {APCUPSD\_ANNOYDELAY} {APCUPSD\_ANNOYDELAY}{APCUPSDANNOYDELAY}
 
  Initial delay after power failure before warning users to get
  off the system (default \var{'60'}).


\config {APCUPSD\_NOLOGON} {APCUPSD\_NOLOGON}{APCUPSDNOLOGON}
 
  The condition which determines when users are prevented from
  logging in during a power failure.
  \var{APCUPSD\_NOLOGON} has to be one of 
  \var{'disable'}, \var{'timeout'}, \var{'percent'}, \var{'minutes'} or
  \var{'always'} (default \var{'disable'}).


\config {APCUPSD\_KILLDELAY} {APCUPSD\_KILLDELAY}{APCUPSDKILLDELAY}
 
  If this value is non-zero, apcupsd will continue running after a
  shutdown has been requested, and after the specified time in
  seconds attempt to kill the power. This is for use on systems
  where apcupsd cannot regain control after a shutdown (default \var{'0'}).
  \var{'0'} disables.

\end {description}

\subsection{Network server settings}

\begin {description}

\config {APCUPSD\_NETSERVER} {APCUPSD\_NETSERVER}{APCUPSDNETSERVER}
 
  The value \var{'yes'} enables, \var{'no'} disables the network
  information server. If \var{'yes'}, a network information
  server process will be started for serving the STATUS and
  EVENT data over the network (used by CGI programs)
  (default \var{'no'}).


\config {APCUPSD\_NISIP} {APCUPSD\_NISIP}{APCUPSDNISIP}
 
  IP address on which NIS server will listen for incoming connections.
  This is useful if your server is multi-homed (has more than one
  network interface and IP address). Default value is \var{'0.0.0.0'} which
  means any incoming request will be serviced. Alternatively, you can
  configure this setting to any specific IP address of your server and 
  NIS will listen for connections only on that interface. Use the
  loopback address (\var{'127.0.0.1'}) to accept connections only from the
  local machine.


\config {APCUPSD\_NISPORT} {APCUPSD\_NISPORT}{APCUPSDNISPORT}
 
  Port to use for sending STATUS and EVENTS data over the network.
  It is not used unless \smalljump{APCUPSDNETSERVER}{\var{APCUPSD\_NETSERVER}}
  is \var{'on'}. If you change this port, you will need to change the
  corresponding value in the cgi directory and rebuild the cgi programs.
  Default is \var{'3551'} as registered with the IANA.


\config {APCUPSD\_EVENTSFILE} {APCUPSD\_EVENTSFILE}{APCUPSDEVENTSFILE}
 
  If you want the last few EVENTS to be available over the network
  by the network information server, you must define an EVENTSFILE.
  (default \var{'/var/log/apcupsd.events'})


\config {APCUPSD\_EVENTSFILEMAX} {APCUPSD\_EVENTSFILEMAX}{APCUPSDEVENTSFILEMAX}

  By default, the size of the 
  \smalljump{APCUPSDEVENTSFILE}{\var{APCUPSD\_EVENTSFILE}} 
  will be not be allowed to exceed 10 kilobytes.  
  When the file grows beyond this limit, older EVENTS will
  be removed from the beginning of the file (first in first out). 
  The parameter \var{APCUPSD\_EVENTSFILEMAX} can be set to a different kilobyte 
  value, or set to zero to allow the
  \smalljump{APCUPSDEVENTSFILE}{\var{APCUPSD\_EVENTSFILE}} to grow without limit.

\end {description}

\subsection{Configuration for share the UPS}

\begin {description}

\config {APCUPSD\_UPSCLASS} {APCUPSD\_UPSCLASS}{APCUPSDUPSCLASS}

   Normally \var{'standalone'} unless you share an UPS using an APC ShareUPS card.
   \var{APCUPSD\_UPSCLASS} may have an value of
   \var{'standalone'}, \var{'shareslave'} or \var{'sharemaster'} 
   (default \var{'standalone'}).


\config {APCUPSD\_UPSMODE} {APCUPSD\_UPSMODE}{APCUPSUPSMODE}

   Normally \var{'disable'} unless you share an UPS using an APC ShareUPS card.
   \var{APCUPSD\_UPSMODE} may have an value of
   \var{'disable'} or \var{'share'} 
   (default \var{'disable'}).

\end {description}

\subsection{Logging system}

\begin {description}

\config {APCUPSD\_STATTIME} {APCUPSD\_STATTIME}{APCUPSSTATTIME}
 
  Time interval in seconds between writing the STATUS file 
  (default \var{'0'}). \var{'0'} disables.


\config {APCUPSD\_STATFILE} {APCUPSD\_STATFILE}{APCUPSSTATFILE}
 
 Location of STATUS file (written only if 
 \smalljump{APCUPSDSTATTIME}{\var{APCUPSD\_STATFILE}} is non-zero)
 (default \var{'/var/log/apcupsd.status'}).


\config {APCUPSD\_LOGSTATS} {APCUPSD\_LOGSTATS}{APCUPSLOGSTATS}

  \var{'on'} enables, \var{'off'} disables the logging of status.
  
  Note! This generates a lot of output, so if turn this on, be sure that the
  file defined in syslog.conf for LOG\_NOTICE is a named pipe (default \var{'off'}).
  You probably do not want to set this to \var{'on'}.


\config {APCUPSD\_DATATIME} {APCUPSD\_DATATIME}{APCUPSDATATIME}
 
   Time interval in seconds between writing the DATA records to
   the log file (default \var{'0'}). \var{'0'} disables.
   


\config {APCUPSD\_FACILITY} {APCUPSD\_FACILITY}{APCUPSFACILITY}
 
  Defines the logging facility (class) for logging to syslog. 
  If not specified, it defaults to \var{'daemon'}.
  This is useful if you want to separate the data logged by apcupsd from other
  programs.

\end {description}

\subsection{Event Mail}

\begin {description}
% ##TRANSLATE## : FFL-1633 begin use mailsend
\config {OPT\_APCUPSD\_EVENTMAIL} {OPT\_APCUPSD\_EVENTMAIL}{OPTAPCUPSDEVENTMAIL}

  If set to \var{'yes'} event mails will be sent to the address in
  \smalljump{APCUPSDEVENTMAILTO}{\var{APCUPSD\_EVENTMAIL\_TO}} via the account 
  configured in   
  \smalljump{APCUPSDEVENTMAILACCOUNT}{\var{APCUPSD\_EVENTMAIL\_ACCOUNT}} 
  The package MAILSEND must be activated with \var{OPT\_MAILSEND='yes'}.
  
  (default \var{'no'}). 
  
\config {APCUPSD\_EVENTMAIL\_ACCOUNT} {APCUPSD\_EVENTMAIL\_ACCOUNT}{APCUPSDEVENTMAILACCOUNT}

  Optional MAILSEND account name used to transmit event mails.
  If the account name is not given, the account \var{'default'} is used.

\config {APCUPSD\_EVENTMAIL\_TO} {APCUPSD\_EVENTMAIL\_TO}{APCUPSDEVENTMAILTO}

  The email address receiving event mails is to be entered here.
  One or more comma separated recipient addresses can be entered here.
% ##TRANSLATE## : FFL-1633 end

\end {description}

% \subsection{Im moment nicht gesetzte Variablen von apcupsd}
% 
%  ========== Configuration statements used in updating the UPS EPROM =========
% 
% 
%  These statements are used only by apctest when choosing "Set EEPROM with conf
%  file values" from the EEPROM menu. THESE STATEMENTS HAVE NO EFFECT ON APCUPSD.
% 
%  UPS name, max 8 characters 
% UPSNAME UPS\_IDEN
% 
%  Battery date - 8 characters
% BATTDATE mm/dd/yy
% 
%  Sensitivity to line voltage quality (H cause faster transfer to batteries)  
%  SENSITIVITY H M L        (default = H)
% SENSITIVITY H
% 
%  UPS delay after power return (seconds)
%  WAKEUP 000 060 180 300   (default = 0)
% WAKEUP 60
% 
%  UPS Grace period after request to power off (seconds)
%  SLEEP 020 180 300 600    (default = 20)
% SLEEP 180
% 
%  Low line voltage causing transfer to batteries
%  The permitted values depend on your model as defined by last letter 
%   of FIRMWARE or APCMODEL. Some representative values are:
%     D 106 103 100 097
%     M 177 172 168 182
%     A 092 090 088 086
%     I 208 204 200 196     (default = 0 => not valid)
% LOTRANSFER  208
% 
%  High line voltage causing transfer to batteries
%  The permitted values depend on your model as defined by last letter 
%   of FIRMWARE or APCMODEL. Some representative values are:
%     D 127 130 133 136
%     M 229 234 239 224
%     A 108 110 112 114
%     I 253 257 261 265     (default = 0 => not valid)
% HITRANSFER 253
% 
%  Battery charge needed to restore power
%  RETURNCHARGE 00 15 50 90 (default = 15)
% RETURNCHARGE 15
% 
%  Alarm delay 
%  0 = zero delay after pwr fail, T = power fail + 30 sec, L = low battery, N = never
%  BEEPSTATE 0 T L N        (default = 0)
% BEEPSTATE T
% 
%  Low battery warning delay in minutes
%  LOWBATT 02 05 07 10      (default = 02)
% LOWBATT 2
% 
%  UPS Output voltage when running on batteries
%  The permitted values depend on your model as defined by last letter 
%   of FIRMWARE or APCMODEL. Some representative values are:
%     D 115
%     M 208
%     A 100
%     I 230 240 220 225     (default = 0 => not valid)
% OUTPUTVOLTS 230
% 
%  Self test interval in hours 336=2 weeks, 168=1 week, ON=at power on
%  SELFTEST 336 168 ON OFF  (default = 336)
% SELFTEST 336
