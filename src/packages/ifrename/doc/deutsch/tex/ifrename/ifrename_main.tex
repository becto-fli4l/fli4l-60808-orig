% Last Update: $Id$
% -------------------------------------------------------------------------------
% ifrename_main.tex: opt tex_documentation
% -------------------------------------------------------------------------------
% creation date: 2007/04/14 - <tobias@tb-home.de>
% last modified: 2011/04/17 - <tobias@tb-home.de>
% -------------------------------------------------------------------------------


\section {Abstract}

\textbf{ifrename} - Umbenennung von Netzwerk Interfacen basierend auf statischen Kriterien\\


Ifrename ist ein Hilfsprogramm, das es erlaubt konsistente Namen für jedes Netzwerk Interface
zu vergeben.\\
\\
Standardmässig sind Interface Namen dynamisch, jedes Interface kann den ersten verfügbaren Namen
bekommen (eth0, eth1...). Die Reihenfolge kann sich dabei ändern. Für eingebaute Interfaces kann die
Aktivierung durch den Kernel sich ändern, steckbare Interface können in beliebiger Reihenfolge
eingebaut werden.\\
\\
Ifrename erlaubt dem User zu entscheiden, welchen Namen ein Interface haben wird. Ifrename kann eine
Vielzahl von Kriterien benutzen um den Interface Namen daran fest zu machen, das geläufigste ist
die MAC Adresse des Interfaces.\\
\\
Ifrename muss ausgeführt werden bevor Interfaces aktiviert werden. Es ist am ehesten von Nutzen in scripten
(init, hotplug) aber selten direkt für den User. Standardmässig bennent ifrename alle vorhandenen
Interfaces des Systems nach Regeln aus /etc/iftab um.\\

\vspace{15pt}
\hrule
\achtung {ACHTUNG:\\
ifrename wurde zur Verwendung auf einem reinen Ethernet-Router entwickelt und nicht ausgiebig in Konfigurationen getestet, wo die Internetverbindung über Dial-Up Verbindungen hergestellt wird.\\
\\
Das Paket greift in die Bootsequenz ein, schaltet Ethernetinterfaces ab und fährt sie anschliessend wieder mit der neuen Bezeichnung hoch. Weiterhin ist zu beachten, das in sämtlichen Konfigurationsdateien des Routers die geänderten Bezeichner verwendet werden müssen.\\
\\
Aus diesem Grund an dieser Stelle der ausdrückliche Hinweis, das bei Verwendung des Paketes der Benutzer die Verantwortung trägt.\\
\\
Es ist also sinnvoll, die Funktion des OPTs zuerst in einer virtuellen Umgebung zu testen, bevor man eventuell ein Produktivsystem unbrauchbar macht!}
\hrule
\vspace{15pt}

\section {Konfiguration}

\begin{description}
\config{OPT\_IFRENAME:}{OPT\_IFRENAME}{OPTIFRENAME}{yes/no - Aktivierung, bzw. Deaktivierung des Paketes}
\config{IFRENAME\_DEBUG:}{IFRENAME\_DEBUG}{IFRENAMEDEBUG}{yes/no - Erweiterte Meldungen auf der Konsole ausgeben}
\config{IFRENAME\_ETH\_N:}{IFRENAME\_ETH\_N}{IFRENAMEETHN}{Anzahl der Ethernetschnittstellen, die vor der Aktivierung von ifrename heruntergefahren werden müssen}
\config{IFRENAME\_ETH\_x\_MAC:}{IFRENAME\_ETH\_x\_MAC}{IFRENAMEETHxMAC}{MAC-Adresse des Ethernet-Devices}
\config{IFRENAME\_ETH\_x\_NAME:}{IFRENAME\_ETH\_x\_NAME}{IFRENAMEETHxNAME}{Neuer Device Name des Ethernet-Devices}
\end{description}

