% Last Update: $Id$
\section{DSLMODEM - PPP über internes DSL-Modem}

Dieses Paket erlaubt es, PPP-Verbindungen über ein internes DSL-Modem wie den
AVM Fritz!DSL-Adapter aufzubauen. Dieses interne DSL-Modem wird dabei direkt an
den jeweiligen Netzzugang (in Deutschland an den Splitter) angeschlossen.
Technisch ``verpackt'' das DSL-Modem die Pakete je nach Provider in PPPoE- oder
PPPoA-Pakete; dies ist jedoch für den fli4l transparent. Hat man ein solches
DSL-Modem, kann mit diesem Paket somit eine DSL-Verbindung ins Internet
realisiert werden.

Ein solcher DSL-Zugang wird generell als PPP-Circuit konfiguriert (siehe
\jump{sect:ppp-circuits}{Circuits vom Typ ``ppp''}), d.\,h. es gilt:

\begin{example}
\begin{verbatim}
    CIRC_x_TYPE='ppp'
\end{verbatim}
\end{example}

Zusätzlich muss das \verb+OPT_PPP_DSLMODEM+ aktiviert werden:

\begin{description}
\config{OPT\_PPP\_DSLMODEM}{OPT\_PPP\_DSLMODEM}{OPTPPPDSLMODEM}

Diese Variable aktiviert die Unterstützung für PPP über ein internes DSL-Modem.
Damit auch tatsächlich eine solche Verbindung genutzt werden kann, muss
mindestens ein PPP-Circuit den Typ ``dslmodem'' besitzen, d.\,h. es muss
zusätzlich gelten

\begin{example}
\begin{verbatim}
    CIRC_x_TYPE='ppp'
    CIRC_x_PPP_TYPE='dslmodem'
\end{verbatim}
\end{example}

(wobei ``x'' einen gültigen Circuit-Index darstellt).

Standard-Einstellung: \verb+OPT_PPP_DSLMODEM='no'+

Beispiel: \verb+OPT_PPP_DSLMODEM='yes'+
\end{description}

\wichtig{Da die entsprechenden Treiber \emph{nicht} der GPL unterliegen, müssen
hierfür die \emph{nonfree}-Varianten des gewünschten Kernel-Pakets verwendet
werden.}

Zu den allgemeinen Circuit-Variablen kommen die folgenden, für
PPP/dslmodem-Circuits spezifischen Variablen hinzu:

\begin{description}
\config{CIRC\_x\_PPP\_DSLMODEM\_TYPE}{CIRC\_x\_PPP\_DSLMODEM\_TYPE}{CIRCxPPPDSLMODEMTYPE}

Es gibt verschiedene interne DSL-Modems, über die eine DSL-Anbindung erfolgen
kann. Das verwendete DSL-Modem wird mit Hilfe der Variablen
\var{CIRC\_x\_PPP\_DSLMODEM\_TYPE} eingestellt, wobei die in Tabelle
\ref{tab:internal-dsl-modems} aufgeführten Typen zur Verfügung stehen.

\begin{table}[htb]
  \centering
  \begin{tabular}{l|l}
    Kartentyp & Internes DSL-Modem \\
    \hline
    fcdsl & AVM Fritz!Card DSL \\
    fcdsl2 & AVM Fritz!Card DSLv2 \\
    fcdslusb & AVM Fritz!Card DSL USB \\
    fcdslusb2 & AVM Fritz!Card DSL USBv2 \\
    fcdslsl & AVM Fritz!Card DSL SL \\
    fcdslslusb & AVM Fritz!Card DSL SL USB \\
  \end{tabular}
  \caption{Unterstützte interne DSL-Modems}
  \label{tab:internal-dsl-modems}
\end{table}

Beispiel: \verb+CIRC_x_PPP_DSLMODEM_TYPE='fcdsl'+

\config{CIRC\_x\_PPP\_DSLMODEM\_PROVIDER}{CIRC\_x\_PPP\_DSLMODEM\_PROVIDER}{CIRCxPPPDSLMODEMPROVIDER}

Mit dieser Option wird der Typ der Gegenstelle eingestellt. Mögliche Optionen
sind U-R2, ECI, Siemens, Netcologne, oldArcor, Switzerland, Belgium, Austria1,
Austria2, Austria3 und Austria4. In Deutschland handelt es sich fast immer um
UR-2. Siemens und ECI kommen nur bei sehr alten Anschlüssen zum Einsatz.
Für Schweiz und Belgien sollten die Optionen selbsterklärend sein, und in
Österreich heißt es ausprobieren.

Sollte jemand für Österreich eine bessere Beschriftung der Optionen haben, möge
er diese bitte mitteilen.

Beispiel: \verb+CIRC_x_PPP_DSLMODEM_PROVIDER='U-R2'+

\end{description}

Beispiel (Internet-Zugang über DSL):

\begin{example}
\begin{verbatim}
    OPT_PPP='yes'
    OPT_PPP_DSLMODEM='yes'
    #
    CIRC_N='1'
    CIRC_1_NAME='DSL-Manitu'
    CIRC_1_TYPE='ppp'
    CIRC_1_ENABLED='yes'
    CIRC_1_NETS_IPV6_N='1'
    CIRC_1_NETS_IPV6_1='::/0'
    CIRC_1_CLASS_N='1'
    CIRC_1_CLASS_1='internet'
    CIRC_1_UP='yes'
    CIRC_1_TIMES='Mo-Su:00-24:0.0:Y'
    CIRC_1_USEPEERDNS='yes'
    CIRC_1_PPP_TYPE='dslmodem'
    CIRC_1_PPP_USERID='anonymer'
    CIRC_1_PPP_PASSWORD='surfer'
    CIRC_1_PPP_DSLMODEM_TYPE='fcdsl'
    CIRC_1_PPP_DSLMODEM_PROVIDER='U-R2'
    #
    CIRC_CLASS_N='1'
    CIRC_CLASS_1='internet'
\end{verbatim}
\end{example}
