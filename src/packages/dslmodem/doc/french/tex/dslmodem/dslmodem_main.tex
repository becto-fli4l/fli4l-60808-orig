% Synchronized to r34020

\section{DSLMODEM - PPP via un modem DSL interne}

Ce paquetage permet via un modem DSL interne de se connecter avec le protocole
PPP, en utilisant par exemple une carte Fritz!DSL du constructeur AVM. Ce modem
DSL interne est directement connecté au réseau tél (en Allemagne au répartiteur)
pour un accès Internet. Techniquement le modem DSL "emballe" les paquets selon
le fournisseur d'accès, avec le protocole PPPoE ou PPPoA, c'est donc, transparent
pour fli4l. Si vous avez un modem DSL de ce type, une liaison DSL peut être réalisée
avec ce paquetage pour un accés Internet.

Cet accés DSL est généralement configuré en tant que circuit PPP (voir
\jump{sect:ppp-circuits}{Circuit de type "ppp"}), c'est à dire~:

\begin{example}
\begin{verbatim}
    CIRC_x_TYPE='ppp'
\end{verbatim}
\end{example}

En outre, la variable \verb+OPT_PPP_DSLMODEM+ doit être activée~:

\begin{description}
\config{OPT\_PPP\_DSLMODEM}{OPT\_PPP\_DSLMODEM}{OPTPPPDSLMODEM}

Cette variable active le modem DSL interne qui doit supporter le protocole
PPP. Pour qu'une telle connexion soit utilisée, vous devez activer le circuit
PPP et le type "dslmodem", c'est à dire, vous devez configurer les variables

\begin{example}
\begin{verbatim}
    CIRC_x_TYPE='ppp'
    CIRC_x_PPP_TYPE='dslmodem'
\end{verbatim}
\end{example}

(le "x" représente l'indexation du circuit valide).

Paramètre par défaut : \verb+OPT_PPP_DSLMODEM='no'+

Exemple : \verb+OPT_PPP_DSLMODEM='yes'+
\end{description}

\wichtig{Les pilotes utilisés ne sont \emph{pas} sous license GPL, vous devez
installer le paquetage du kernel \emph{nonfree} pour ces pilotes.}

Pour un circuit PPP/dslmodem spécifique les variables ci-dessous sont à configurer,
les variables pour un circuit générale sont les suivantes~:

\begin{description}
\config{CIRC\_x\_PPP\_DSLMODEM\_TYPE}{CIRC\_x\_PPP\_DSLMODEM\_TYPE}{CIRCxPPPDSLMODEMTYPE}

Vous avez différents modems DSL internes sur lesquels une liaison DSL peut être
utilisé. Le modem DSL utilisé doit est paramétré dans la variable \var{CIRC\_x\_PPP\_DSLMODEM\_TYPE},
les types disponibles sont énumérés dans le tableau \ref{tab:Modem DSL interne}.

\begin{table}[htb]
  \centering
  \begin{tabular}{l|l}
    Type de carte & Modem DSL interne \\
    \hline
    fcdsl & AVM Fritz!Card DSL \\
    fcdsl2 & AVM Fritz!Card DSLv2 \\
    fcdslusb & AVM Fritz!Card DSL USB \\
    fcdslusb2 & AVM Fritz!Card DSL USBv2 \\
    fcdslsl & AVM Fritz!Card DSL SL \\
    fcdslslusb & AVM Fritz!Card DSL SL USB \\
  \end{tabular}
  \caption{Modems DSL internes supportés}
  \label{tab:internal-dsl-modems}
\end{table}

Exemple : \verb+CIRC_x_PPP_DSLMODEM_TYPE='fcdsl'+

\config{CIRC\_x\_PPP\_DSLMODEM\_PROVIDER}{CIRC\_x\_PPP\_DSLMODEM\_PROVIDER}{CIRCxPPPDSLMODEMPROVIDER}

Dans cette variable vous pouvez régler le type de fournisseur. Les options possibles
sont U-R2, ECI, Siemens, Netcologne, oldArcor, Switzerland, Belgium, Austria1,
Austria2, Austria3 et Austria4. En Allemagne, l'option UR-2 est presque toujours
utilisatée. Siemens et ECI sont utilisés que sur de très anciens fournisseurs.
Pour la Suisse et la Belgique, les options sont explicites et pour l'Autriche
vous devez essayer.

Pour l'Autriche, si quelqu'un a de meilleures options, merci de nous les faire parvenir.

Exemple : \verb+CIRC_x_PPP_DSLMODEM_PROVIDER='U-R2'+

\end{description}

Exemple (pour un accès à Internet via l'ADSL)~:

\begin{example}
\begin{verbatim}
    OPT_PPP='yes'
    OPT_PPP_DSLMODEM='yes'
    #
    CIRC_N='1'
    CIRC_1_NAME='DSL-Manitu'
    CIRC_1_TYPE='ppp'
    CIRC_1_ENABLED='yes'
    CIRC_1_NETS_IPV6_N='1'
    CIRC_1_NETS_IPV6_1='::/0'
    CIRC_1_CLASS_N='1'
    CIRC_1_CLASS_1='internet'
    CIRC_1_UP='yes'
    CIRC_1_TIMES='Mo-Su:00-24:0.0:Y'
    CIRC_1_USEPEERDNS='yes'
    CIRC_1_PPP_TYPE='dslmodem'
    CIRC_1_PPP_USERID='anonymer'
    CIRC_1_PPP_PASSWORD='surfer'
    CIRC_1_PPP_DSLMODEM_TYPE='fcdsl'
    CIRC_1_PPP_DSLMODEM_PROVIDER='U-R2'
    #
    CIRC_CLASS_N='1'
    CIRC_CLASS_1='internet'
\end{verbatim}
\end{example}
