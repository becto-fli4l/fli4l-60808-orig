% Last Update: $Id$
\marklabel{sec:opt-wlan }
{
\section {WLAN - Wireless-LAN Unterstützung}
}

Achten Sie in jedem Fall darauf, dass Sie beim Einsatz von PCI Karten
ein Mainboard benutzen, was mindestens die PCI 2.2 Spezifikationen
erfüllt. Auf älteren Mainboard die nur PCI 2.1 oder älter unterstützen
kann es zu den unterschiedlichsten Fehler kommen. Entweder startet der
Computer gar nicht (er läßt sich nicht einmal einschalten), oder die
WLAN-Karte wird beim PCI Scan nicht gefunden.

WLAN-Karten werden in der base.txt \var{IP\_NET\_X\_DEV} mit wlanX angesprochen.
Wenn nur eine WLAN-Karte im System ist, hat diese also den Namen wlan0.

\subsection{WLAN-Konfiguration}

\begin{description}
\config{OPT\_WLAN}{OPT\_WLAN}{OPTWLAN}

        Standard-Einstellung: \var{OPT\_\-WLAN}='no'

        Aktiviert das Wireless LAN Option Pack.

\config{WLAN\_REGDOMAIN}{WLAN\_REGDOMAIN}{WLANREGDOMAIN}

	Mit dieser Variable kann man die Landesspezifischen Einstellungen anpassen.
	Gültige Werte sind ISO 3166-1 alpha-2 Ländercodes wie z.B. 'DE'
	In verschiedenen Ländern gelten verschiedene Vorgaben für die Kanalauswahl
	und Sendeleistungen.

\config{WLAN\_N}{WLAN\_N}{WLANN}

        Anzahl der voneinander unabhängigen WLAN-Konfigurationen.
        Steht hier eine '1' so ist das Verhalten wie in früheren Versionen
        von fli4l. 
        
\config{WLAN\_x\_MAC}{WLAN\_x\_MAC}{WLANxMAC}

         MAC-Adresse der WLAN-Karte in dieser Schreibweise:
         
         XX:XX:XX:XX:XX:XX

         Jedes X ist ein Hex-Digit der Mac-Adresse der Karte, für
         die diese Konfiguration gelten soll. Sollte keine der hier
         eingetragenen Mac-Adressen zu einer Karte passen, so wird
         die Konfiguration \var{WLAN\_1\_*} auf diese Karte angewandt und es wird eine
         Warnmeldung ausgegeben, die auf den Umstand hinweist.
         Die Warnmeldung enthält die festgestellte MAC-Adresse der Karte.
         Diese ist in der Konfiguration einzutragen, damit auch das Web-Interface
         problemlos funktionieren kann.

\config{WLAN\_x\_MAC\_OVERRIDE}{WLAN\_x\_MAC\_OVERRIDE}{WLANxMACOVERRIDE}

         Ändert die MAC-Adresse der WLAN-Karte damit man als Client an ein WLAN
         mit MAC-Filter verbinden kann ohne dort den Filter anpassen zu müssen.
         Hilfreich bei WAN-Anbindungen, die z.B. auf die MAC-Adresse eines gelieferten
         WLAN-USB-Sticks gebunden sind.

\config{WLAN\_x\_ESSID}{WLAN\_x\_ESSID}{WLANxESSID}

        Die SSID ist der Name für das Funknetzwerk.
        Die auch ``Network Name'' genannte Zeichenfolge kann bis zu 32 Zeichen lang sein.
        Sie wird im AP eines WLAN konfiguriert und von allen Clients,
        die darauf Zugriff haben sollen, eingestellt.
        Auch bei Ad-Hoc muß die SSID auf allen teilnehmenden Nodes identisch sein.


\config{WLAN\_x\_MODE}{WLAN\_x\_MODE}{WLANxMODE}

        Stellt den zu verwendenden WLAN-Modus der Karte ein.

        Standard-Einstellung: \var{WLAN\_x\_\-MODE}='ad-hoc'

        Mögliche Werte:

        \begin{tabular}[h]{ll}
            ad-hoc        & für ein Funknetz ohne Access-Point \\
            managed       & gemanagedtes Funknetz mit mehreren Zellen\\
            master        & die WLAN-Karte arbeitet als Access-Point\\
        \end{tabular}

        \var{WLAN\_x\_MODE}='master' funktioniert nur mit einem geeigneten WLAN-Treiber.

\config{WLAN\_x\_NOESSID}{WLAN\_x\_NOESSID}{WLANxNOESSID}

        Ermöglicht das Abschalten der ESSID in den Beacon Frames.
        Nur möglich mit Treiber hostap\_* und Firmware $>$= 1.6.3 im WLAN\_MODE='master'

        Dieses Feature ist optional und muß manuell zur config/wlan.txt hinzugefügt werden.

\config{WLAN\_x\_CHANNEL}{WLAN\_x\_CHANNEL}{WLANxCHANNEL}

        Setzt den Übertragungskanal des Netzwerks.

        Standard-Einstellung: \var{WLAN\_x\_\-CHANNEL}='1'

        Mögliche Werte: 1-13 und 36,40,44,48,52,56,60,64,100,104,108,112,116,120,124,128,132,136,140

        Bitte lesen sie die Dokumentation Ihrer WLAN-Karte um herauszufinden,
        welche Kanäle in Ihrem Land erlaubt sind.
        Sollten sie hier einen nicht erlaubten Kanal einstellen, so sind sie alleine dafür
        verantwortlich. In Deutschland sind die Kanäle 1-13 im Frequenzband 2,4 GHz (Modi: b und g) erlaubt.
        Die Kanäle im Bereich 36-140 (siehe oben) sind im 5 GHz zulässig.
    
        Desweiteren ist der Wert '0' erlaubt, falls \var{WLAN\_x\_\-MODE}='managed' gesetzt ist.
        Dadurch wird kein expliziter Kanal eingestellt, sondern der AP auf allen
        verfügbaren Kanälen gesucht.
        Man kann dem Kanal-Wert auch einen Buchstaben a,b oder g anhängen (z.B. 5g), welcher dann
        den gewünschten Betriebsmodus/Frequenzband auswählt.

        Ein angehängtes 'n' oder 'N' selektiert bei entsprechenden WLAN-Karten die Nutzung von 802.11n.
        Kleingeschrieben bedeutet: 20 MHz Kanalbreite, grossgeschrieben: 40 MHz Kanalbreite.
        
        Großschreibung bei a/b/g sorgt bei einigen (aktuell nur ath\_pci) Treibern dafür, dass proprietäre 
        WLAN-Turbos aktiviert werden. Diese Option ist experimentell und kann auch wieder entfernt werden.

\config{WLAN\_x\_RATE}{WLAN\_x\_RATE}{WLANxRATE}

        Setzt die Übertragungsgeschwindigkeit des Netzwerks.

        Standard-Einstellung: \var{WLAN\_x\_\-RATE}='auto'

        Mögliche Werte: 1,2,5.5,11,auto - Angaben in Megabit/s
        je nach Karte können auch noch diese Raten ausgewählt werden: 6,9,12,18,24,36,48 und 54.
        Bei manchen 54 MBit-Karten kann die Rate nicht angegeben werden. Hier ist dann 'auto' einzutragen.
        
\config{WLAN\_x\_RTS}{WLAN\_x\_RTS}{WLANxRTS}

        Aktiviert RTS/CTS Handshake. Diese Option ist in grossen Wlans mit vielen sendenden Clients nuetzlich wenn sich die Clients gegenseitig nicht hoeren koennen sondern nur den AP.
        Ist diese Option aktiviert sendet der Client vor jedem Sendevorgang ein RTS mit der Bitte um Erlaubnis zum Senden und bekommt ein CTS, die Erlaubnis zum Senden, vom AP zurueck. Damit weiss jeder Client dass ein Client sendet auch wenn er diesen Client nicht hoert.
        Hierdurch werden Kollisionen vermindert weil sicher gestellt ist dass immer nur ein Client sendet. Diese Option macht nur unter der oben beschriebenen Situation Sinn weil sie zusaetzlichen overhead hinzufuegt und somit die Gesamtbandbreite verringert. Durch die Verringerung von Kollisionen kann sich die Bandbreite jedoch wieder erhoehen.

        Dieses Feature ist optional und muß manuell zur config/wlan.txt hinzugefügt werden.
        
\config{WLAN\_x\_ENC\_N (Überholt)}{WLAN\_x\_ENC\_N}{WLANxENCN}

        Legt die Anzahl der Wireless Encryption Key's fest (WEP).

        Mögliche Werte: 0-4

\config{WLAN\_x\_ENC\_x (Überholt)}{WLAN\_x\_ENC\_x}{WLANxENCx}

        Setzt die Wireless Encryption Keys.

        Mögliche Werte:

        \begin{tabular}[h]{ll}
        XXXX-XXXX-XXXX-XXXX-XXXX-XXXX-XX &       128 Bit Hex-Key (X=0-F) \\
        XXXX-XXXX-XX                     &        64 Bit Hex-Key (X=0-F) \\
        s:$<$5 Zeichen$>$                &        64 Bit\\
        s:$<$6-13 Zeichen$>$             &       128 Bit\\
        P:$<$1-64 Zeichen$>$             &       128 Bit\\
        \end{tabular}

        Das Verfahren der Key-Vergabe mit s:Text ist \textbf{nicht} mit der Passphrase
        der Windows-Treiber kompatibel. Hier bitte einen Hex-Key verwenden!
        Unter Windows wird der Hex-Key meist \textbf{ohne} die Bindestriche '-' verwendet.
        Die Angabe mittels P:$<$Text$>$ ist kompatibel zur Passphrase der meisten
        Windows WLAN-Treiber (wenn nicht allen) aber \textbf{nur} im 128 Bit Modus.
        Linux erlaubt es, verschiedene Schlüssellängen zu mischen. Windows-Treiber
        jedoch in der Regel \textbf{nicht}!

\config{WLAN\_x\_ENC\_ACTIVE (Überholt)}{WLAN\_x\_ENC\_ACTIVE}{WLANxENCACTIVE}

        Legt den aktiven Wireless Encryption Key fest.

        Mögliche Werte: 1-4
        
        Diese Variable ist aufzunehmen, wenn \var{WLAN\_x\_ENC\_N} > 0 gesetzt wird. Ansonsten optional.

\config{WLAN\_x\_ENC\_MODE (Überholt)}{WLAN\_x\_ENC\_MODE}{WLANxENCMODE}

        Aktiviert den Encryption Mode.

        Mögliche Werte:

        \begin{tabular}[h]{ll}
          on/off &         mit oder ohne Veschlüsselung \\
          open   &         nimmt auch unverschlüsselte Pakete an \\
          restricted &     nimmt nur veschlüsselte Pakete an \\
        \end{tabular}

        Sinnvoller Wert: 'restricted'
        
        Dieses Feature ist optional und muß manuell zur config/wlan.txt hinzugefügt werden.
        Ist die Variable nicht vorhanden, so wird als Default 'off' angenommen, wenn kein WEP-Key definiert ist und 'restricted' wenn mindestens ein Key definiert ist.

\config{WLAN\_x\_WPA\_KEY\_MGMT}{WLAN\_x\_WPA\_KEY\_MGMT}{WLANxWPAKEYMGMT}

Will man statt WEP-Verschlüsselung WPA verwenden, stellt man hier den
gewünschten WPA-Modus ein. Momentan wird nur WPA-PSK unterstützt, also
WPA mit einem Client und Access-Point vorab bekannten Schlüssel. Dieser
Schlüssel sollte sorgfältig gewählt werden und nicht zu kurz sein, da
er ansonsten auch anfällig gegen Wörterbuchattacken ist.

Unterstützt werden im \emph{managed}-Mode alle vom Wpa-Supplicanten
(\altlink{http://hostap.epitest.fi/wpa_supplicant/} 
und im \emph{master}-Mode alle vom Hostapd
(\altlink{http://hostap.epitest.fi/hostapd/}) unterstützten Karten.

Erfolgreich getestet wurden bereits Karten basierend auf den 
Chipsätzen von Atheros und vom hostap-Treiber
unterstützte Karten (sowohl im managed als auch im master mode). Theoretisch ist auch noch Unterstützung für
atmel-Karten und
einige andere möglich. Hier müssen die Ersteller der entsprechenden
Opts aber ihre Opt-Pakete noch entsprechend anpassen.

\config{WLAN\_x\_WPA\_PSK}{WLAN\_x\_WPA\_PSK}{WLANxWPAPSK}

      Hier wird der Schlüssel angegeben, der zur Kommunikation zwischen
      Client und Access-Point verwendet werden soll. Dieser Schlüssel wird in Form einer Passphrase 
      (eines Satzes) angegeben, die mindestens 8 Zeichen lang sein muß und bis zu 63 Zeichen lang sein kann. 
      Folgende Zeichen werden unterstützt:

      a-z A-Z 0-9 ! \# \$ \% \& ( ) * + , - . / : ; {\textless} = {\textgreater} ? @ [ {\textbackslash} ] \^{} \_ \`{} \{ | \} {\textasciitilde}

\config{WLAN\_x\_WPA\_TYPE}{WLAN\_x\_WPA\_TYPE}{WLANxWPATYPE}

      Zur Auswahl stehen hier 1 für WPA1, 2 für den WPA2 (IEEE 802.11i) Modus
      und 3 für beide - der Client kann dann entscheiden ob er WPA1 oder
      WPA2 nutzen möchte.
      Wenn die WLAN Hardware den Standard unterstützt, so ist dem sicheren WPA2 
      Verfahren den Vorzug zu geben.

\config{WLAN\_x\_WPA\_ENCRYPTION}{WLAN\_x\_WPA\_ENCRYPTION}{WLANxWPAENCRYPTION}

      Die Verschlüsselungsprotokolle TKIP und die erweiterte Version CCMP 
      (AES-CTR/CBC-MAC Protocol, manchmal auch nur AES genannt) stehen hier zur Verfügung. CCMP wird eventuell
      nicht von älterer WLAN-Hardware unterstützt. Es können auch beide
      gemeinsam angegeben werden.
      
\config{WLAN\_x\_WPA\_DEBUG (Experimentell)}{WLAN\_x\_WPA\_DEBUG}{WLANxWPADEBUG}

      Bei Problemen mit der WPA-Anbindung kann man diese Variable auf
      'yes' setzen, um den zuständigen daemon zu umfangreicheren
      Ausgaben zu veranlassen. Diese kann man dann zur Diagnose der
      Probleme verwenden.

\config{WLAN\_x\_AP}{WLAN\_x\_AP}{WLANxAP}

        Registriert diese Node bei einem Access-Point.

        Hier ist die MAC-Adresse des Access-Points anzugeben.
        Wenn man bereits den WLAN-Mode ``master'' ausgewählt hat, ist dieser Eintrag leer zu lassen.
        Diese Option ist nur dann sinnvoll, wenn der fli4l den AP nicht von selber finden kann oder an einen bevorzugten Access-Point gebunden werden soll.
        Nur zum Einsatz in WLAN-Mode ``managed'' gedacht.

        Dieses Feature ist optional und muß manuell zur config/wlan.txt hinzugefügt werden.

\config{WLAN\_x\_ACL\_POLICY}{WLAN\_x\_ACL\_POLICY}{WLANxACLPOLICY}

        Policy der ACL.

        Standard-Einstellung: \var{WLAN\_x\_ACL\_\-POLICY}='allow'

        Beschreibt eine Aktion, der die angegebenen MAC-Adressen unterliegen:

        \begin{tabular}[h]{ll}
          deny  & Keine der aufgelisteten MAC-Adressen erhält Zugang \\
          allow & Nur aufgelistete MAC-Adressen erhalten Zugang \\
          open  & Alle MAC-Adressen erhalten unabhängig vom Filter Zugang \\
        \end{tabular}
        
        Leider werden WLAN\_ACL's aktuell nur von einem Treiber sauber unterstützt: hostap\_*
        Als Alternative bieten sich die in 3.0.x deutlich erweiterten Firewall-Möglichkeiten an.

\config{WLAN\_x\_ACL\_MAC\_N}{WLAN\_x\_ACL\_MAC\_N}{WLANxACLMACN}

        AP-ACLs - Einschränkung der erlaubten WLAN-Stationen.

        Standard-Einstellung: \var{WLAN\_x\_ACL\_MAC\_N}='0'

        Eine Zahl größer 0 aktiviert die Access Control List (der MAC-Adressenfilter) und gibt die Anzahl der ACL-Einträge an.
        Die Access Control List ist eine Liste von MAC-Adressen,
        denen der Zugang zum Access Point (AP) erlaubt/verboten wird.
        Anzahl der Mac-Adressen, die definiert werden.


\config{WLAN\_x\_ACL\_MAC\_x}{WLAN\_x\_ACL\_MAC\_x}{WLANxACLMACx}

        Mac-Adressen in der Form: 00:00:E8:83:72:92

\config{WLAN\_x\_DIVERSITY}{WLAN\_x\_DIVERSITY}{WLANxDIVERSITY}

        Hiermit kann man einstellen, ob manuelle Antennen-Diversity aktiviert wird.

        Standard-Einstellung: \var{WLAN\_x\_DIVERSITY}='no' (automatische Wahl)

\config{WLAN\_x\_DIVERSITY\_RX}{WLAN\_x\_DIVERSITY\_RX}{WLANxDIVERSITYRX}

        Auswahl der Empfangsantenne.

        Standard-Einstellung: \var{WLAN\_x\_DIVERSITY\_RX}='1'

        \begin{tabular}[h]{ll}
        0 = Automatische Auswahl\\
        1 = Antenne 1\\
        2 = Antenne 2\\
        \end{tabular}

\config{WLAN\_x\_DIVERSITY\_TX}{WLAN\_x\_DIVERSITY\_TX}{WLANxDIVERSITYTX}

        Auswahl der Sendeantenne.

        Standard-Einstellung: \var{WLAN\_x\_DIVERSITY\_TX}='1'

\config{WLAN\_x\_WPS}{WLAN\_x\_WPS}{WLANxWPS}

        Aktiviert den WPS-Support. Push-Button und PIN ist möglich.
        Es ist sinnvoll, WLAN\_WEBGUI zu aktivieren, es sei denn es ist nur
        die Steuerung per Commandline gewünscht.

        Standard-Einstellung: \var{WLAN\_x\_WPS}='no'

\config{WLAN\_x\_PSKFILE}{WLAN\_x\_PSKFILE}{WLANxPSKFILE}

        Bei aktiviertem PSKFILE können neben dem unter \var{WLAN\_x\_WPA\_PSK}
        konfigurierten Preshared Key auch weitere Client-bezogene Keys genutzt werden.
        Aktuell nutzt die Funktion \var{WLAN\_x\_WPS} dieses File um darüber
        konfigurierten Clients individuelle Keys zu geben.

        Wird das File abgeschaltet sind auch bisher mit aktiviertem File verbundene
        WPS-Clients nicht mehr in der Lage mit dem AccessPoint zu verbinden.

        WPS-Clients, die mit abgeschaltetem File verbunden wurden, sind davon nicht
        betroffen.

        Standard-Einstellung: \var{WLAN\_x\_PSKFILE}='yes'

\config{WLAN\_x\_BRIDGE}{WLAN\_x\_BRIDGE}{WLANxBRIDGE}

        Alternativ zur Angabe im Paket ADVANCED\_NETWORKING kann hier umgekehrt
        angegeben werden an welche Bridge das WLAN gebunden werden soll.

        Beispiel: \var{WLAN\_x\_BRIDGE}='br0'

        Achtung: Entweder in Advanced-Network oder hier angeben! \textbf{Nicht} an beiden Stellen!

\end{description}

\subsection{Weboberfläche}

    Das Paket bringt außerdem eine Weboberfläche für den mini-httpd mit.
    Die Weboberfläche wird beim Setzen \var{OPT\_HTTPD='yes'} 
    automatisch mit aktiviert.
    

\subsection{Beispiele}
\subsubsection{Anbindung an einen Access Point via WPA}

\begin{example}
\begin{verbatim}
OPT_WLAN='yes'
WLAN_N='1'
WLAN_1_MAC='00:0F:A3:xx:xx:xx'
WLAN_1_ESSID='foo'
WLAN_1_MODE='managed'           # Anbindung an Access Point
WLAN_1_CHANNEL='1'
WLAN_1_RATE='auto'
#
# WPA Konfiguration
#
WLAN_1_ENC_N='0'                # kein WEP
WLAN_1_WPA_KEY_MGMT='WPA-PSK'   # WPA pre shared key
WLAN_1_WPA_TYPE='1'             # WPA 1
WLAN_1_WPA_ENCRYPTION='TKIP'
WLAN_1_WPA_PSK='Deine gute Passphrase (8-63 Zeichen)'
#
# irrelevant im WPA Kontext
#
WLAN_1_ENC_N='0'
WLAN_1_ENC_ACTIVE='1'
WLAN_1_ACL_POLICY='allow'
WLAN_1_ACL_MAC_N='0'
\end{verbatim}
\end{example}

\subsubsection{Access Point mit WPA2 Verschlüsselung}

\begin{example}
\begin{verbatim}
OPT_WLAN='yes'
WLAN_N='1'
WLAN_1_MAC='00:0F:A3:xx:xx:xx'
WLAN_1_ESSID='foo'
WLAN_1_MODE='master'            # Access Point
WLAN_1_CHANNEL='1g'             # Channel 1, Modus 'g' auf einer
                                # Atheros-Karte
WLAN_1_RATE='auto'
#
# WPA Konfiguration
#
WLAN_1_ENC_N='0'                # kein WEP
WLAN_1_WPA_KEY_MGMT='WPA-PSK'   # WPA pre shared key
WLAN_1_WPA_TYPE='2'             # WPA 2
WLAN_1_WPA_ENCRYPTION='CCMP'
WLAN_1_WPA_PSK='Deine gute Passphrase (8-63 Zeichen)'
#
# MAC basierte Zugriffkontrolle auf AP
#
WLAN_1_ACL_POLICY='allow'
WLAN_1_ACL_MAC_N='0'
#
# irrelevant im WPA Kontext
#
WLAN_1_ENC_ACTIVE='1'
\end{verbatim}
\end{example}

\subsubsection{Access Point mit WEP Verschlüsselung}
\begin{example}
\begin{verbatim}
OPT_WLAN='yes'
WLAN_N='1'
WLAN_1_MAC='00:0F:A3:xx:xx:xx'
WLAN_1_ESSID='foo'
WLAN_1_MODE='master'            # Access Point
WLAN_1_CHANNEL='1'
WLAN_1_RATE='auto'
#
# WEP Konfiguration
#
WLAN_1_WPA_KEY_MGMT=''          # kein WPA 
WLAN_1_ENC_N='4'                # 4 WEP-Keys
WLAN_1_ENC_1='...'
WLAN_1_ENC_2='...'
WLAN_1_ENC_3='...'
WLAN_1_ENC_4='...'
WLAN_1_ENC_ACTIVE='1'           # erster Schlüssel ist aktiv
#
# MAC basierte Zugriffkontrolle auf AP
#
WLAN_1_ACL_POLICY='allow'
WLAN_1_ACL_MAC_N='0'
# 
# irrelevant für WEP Konfiguration
#
WLAN_1_WPA_TYPE='2'
WLAN_1_WPA_ENCRYPTION='CCMP'
WLAN_1_WPA_PSK='...'

\end{verbatim}
\end{example}

\subsection{Virtual Accesspoint (VAP)(Experimentell)}

Bestimmte WLAN-Karten (Treiber: ath\_pci, ath5k, ath9k, ath9k\_htc) können auf bis zu 4 virtuelle WLAN-Karten aufgeteilt werden. (VAP)

Die WLAN-Konfiguration der virtuellen AP kann beliebig sein bis auf folgende Bedingung: Gleich sein muss: Kanal und MAC-Adresse.
Anhand der mehrfach verwendeten MAC-Adresse, wird die Karte identifiziert, die aufgesplittet werden soll.
Bei mehreren verbauten Karten kann dies auch mehrfach gemacht werden.

Das Basis-Device wird weiterhin wlan0 heissen (bei einer WLAN-Karte). Bei VAP dann wlan0v2 usw.
Zum binden an eine Bridge bitte hier WLAN\_x\_BRIDGE='br0' usw. verwenden!.

Das aktuelle Maximum ist: je nach Karte und Treiber bis zu 8x Master.

\subsection{Zeitgesteuertes ein-- und ausschalten mit easycron}

Mittels \jump{sec:opt-easycron}{\emph{easycron}}, einem anderen Paket, kann das WLAN ein--
und ausgeschaltet werden.

\begin{example}
\begin{verbatim}
EASYCRON_N='2'
EASYCRON_1_CUSTOM  = ''     # Jeden Abend um 24 Uhr ausschalten
EASYCRON_1_COMMAND = '/usr/sbin/wlanconfig.sh wlan0 down'
EASYCRON_1_TIME    = '* 24 * * *'

EASYCRON_2_CUSTOM  = ''     # und um 8 Uhr wieder an.
EASYCRON_2_COMMAND = '/usr/sbin/wlanconfig.sh wlan0'
EASYCRON_2_TIME    = '* 8 * * *'
\end{verbatim}
\end{example}

\subsection{Spendenhinweis}

Durch die großzügige Spende von 2 Ralink 2500 basierten WLAN-Karten können
WLAN-Karten mit dem RT25xx Chipsatz mit fli4l in den Modi ad-hoc und managed
verwendet werden. Als Treiber ist in der base.txt hierzu rt2500 auszuwählen.
Die Karten wurden gespendet von:

Computer Contor, Pilgrimstein 24a, 35037 Marburg
