% Do not remove the next line
% Synchronized to r45127

\marklabel{sec:opt-wlan }
{
\section {WLAN~- Supporte le WLAN (ou réseau sans fil)}
}

Faites attention au carte mère que vous utilisez et vérifier les spécifications
de la version du PCI il doit être au moins à 2.2. Sur des carte mères plus
ancienne la version du PCI sont seulement à 2.1, diverse erreurs viennent de là.
L'ordinateur ne démarre pas du tout (il ne peut même pas être allumé), ou la
carte WLAN n'est pas trouvée lors du scan PCI.

Le nom des cartes WLAN paramétré dans la variable \var{IP\_NET\_X\_DEV} du fichier
base.txt s'appel maintenant wlanX. Si une carte WLAN est dans le système, elle
aura le nom 'wlan0'. 


\subsection{Configuration du WLAN}

\begin{description}
\config{OPT\_WLAN}{OPT\_WLAN}{OPTWLAN}

      Installation par défaut~: \var{OPT\_\-WLAN}='no'

      Variable pour activer le Pack Wireless LAN indiquer ici 'yes'.

\config{WLAN\_REGDOMAIN}{WLAN\_REGDOMAIN}{WLANREGDOMAIN}

      Avec cette variable, nous pouvons paramétrer et ajuster le pays
      spécifique. Les valeurs des codes des pays utilisent la norme
      ISO 3166-1 alpha-2, tels que 'FR', 'DE'. La sélection des canaux et des
      niveaux de puissance sont différents selon les exigences des pays.

\config{WLAN\_N}{WLAN\_N}{WLANN}

      On indiquer ici le nombre de configurations pour le WLAN, elles sont
      indépendantes les unes aux autres. Si vous indiquez '1' le comportement de
      la carte sera identique à l'ancienne version fli4l.

\config{WLAN\_x\_MAC}{WLAN\_x\_MAC}{WLANxMAC}

      On indique ici l'adresse Mac de la carte WLAN, elle est écrite de cette
      manière~:

         XX:XX:XX:XX:XX:XX

      Chaque X représente un Hexadécimale de l'adresse Mac de la carte, elle
      doit être valide pour la configuration. Si aucune adresse Mac n'est
      paramétrée, la configuration de l'adresse Mac de la première carte dans la
      variable \var{WLAN\_1\_*} sera appliquée automatiquement. Un message
      d'avertissement apparaitra à la construction des archives, pour attirer
      votre attention. Ce message d'avertissement contient l'adresse Mac de la
      carte Wlan. Si vous voulez que l'interface web fonctionner correctement,
      vous devez enregistrer dans le fichier de configuration l'adresse Mac.

\config{WLAN\_x\_MAC\_OVERRIDE}{WLAN\_x\_MAC\_OVERRIDE}{WLANxMACOVERRIDE}

      Avec cette variable vous pouvez changer l'adresse MAC de la carte WLAN,
      ainsi vous pouvez vous connecter en tant que client WLAN à travers le
      filtre MAC, sans devoir régler ce filtre. Cela est utile lorque vous
      voulez vous connnecter au WAN, par ex. en paramétrant l'adresse Mac
      d'une clé USB WLAN.

\config{WLAN\_x\_ESSID}{WLAN\_x\_ESSID}{WLANxESSID}

      Le SSID est le nom du réseau sans fil. Le "nom du réseau" peut avoir
      une longueur maximum de 32 caractères à la suite. Il est configuré dans
      l'AP (ou Point Accès) du WLAN ainsi tous les clients qui utilise l'AP
      doivent s'indentifier avec ce nom. Le SSID à aussi pour but, d'indentifier
      une connexion Ad-Hoc les membres doivent avoir le même identifient.

\config{WLAN\_x\_MODE}{WLAN\_x\_MODE}{WLANxMODE}

      Modes à utiliser pour une carte WLAN.

      Installation par défaut~: \var{WLAN\_x\_\-MODE}='ad-hoc'

      Valeurs possibles~:

        \begin{tabular}[h]{ll}
            ad-hoc        & Pour un réseau Wlan sans Point d'accés \\
            managed       & (ou infrastructure) Gestion du réseau sans fil avec
            plusieurs Point d'accés \\
            master        & La carte WLAN fonctionne comme un Point d'accés \\
        \end{tabular}

      \var{WLAN\_x\_MODE}='master' fonctionne uniquement avec un pilote WLAN approprié.

\config{WLAN\_x\_NOESSID}{WLAN\_x\_NOESSID}{WLANxNOESSID}

      Cette variable vous permet de cacher le nom ESSID à l'écran. Seulement
      possible qu'avec le pilote hostap\_* et le Firmware $>$= 1.6.3 et aussi
      avec le mode WLAN\_MODE='master'.

      Cette fonctionnalité est optionnelle et doit être ajoutée manuellement
      dans le fichier config/wlan.txt.

\config{WLAN\_x\_CHANNEL}{WLAN\_x\_CHANNEL}{WLANxCHANNEL}

      On indique ici le canal de transmission du réseau.

      Installation par défaut~: \var{WLAN\_x\_\-CHANNEL}='1'

      Les valeurs possibles sont~: 1 à 13 et 36,40,44,48,52,56,60,64,100,104,
      108,112,116,120,124,128,132,136,140

      S'il vous plaît, lire la documentation de votre carte WLAN pour connaitre
      les canaux autorisés dans votre pays. Si vous paramétrez un canal non permis,
      vous êtes le seul responsable. En Allemagne et en France les canaux de 1 à 13
      sont permis sur la bande de fréquence 2,4 GHz~: b et g. Les canaux spécifiés
      de 36 à 140 (voir ci-dessus) sont autorisées sur la bande de fréquence 5 GHz.

      De plus la valeur '0' est permise, si cette variable est paramétrée sur
      \var{WLAN\_x\_\-MODE}='managed'. Ainsi aucun canal explicite n'est réglé,
      mais le wlan cherche l'AP sur tous les canaux disponibles. On peut ajouter
      à la valeur du canal une lettre a, b ou g (par exemple 5g) on sélectionne
      ainsi le mode de fonctionnement de la bande de fréquences souhaité.

      Maintenant vous pouvez ajouter la lettre 'n' ou 'N' pour les cartes Wlan
      qui correspondes à la nouvelle norme 802.11n. Si vous indiquez la lettre
      en minuscule~: la bande de fréquences utilisée sera de 20 MHz, si vous
      indiquez la lettre en majuscule~: la bande de fréquences utilisée sera de 40 MHz.

      Le paramétrage des majuscules a/b/g sont utilisés avec certains pilotes,
      (fonctionne actuellement seulement avec le pilote ath\_pci), pour activer
      le turbo WLAN affecté à la carte. Cette option est expérimentale et peut
      être ignorée.

\config{WLAN\_x\_RATE}{WLAN\_x\_RATE}{WLANxRATE}

      On indique ici la vitesse de transmission du réseau.

      Installation par défaut~: \var{WLAN\_x\_\-RATE}='auto'

      Les valeurs possibles sont~: 1, 2, 5.5, 11, auto~- Ils sont indiqué en
      mégabit/s, selon la carte utilisée les taux peuvent être de~: 6, 9, 12, 18,
      24, 36, 48 et 54. Avec certaines cartes le taux de 54 Mbit/s ne peut pas
      être indiqué. Vous devez alors inscrire 'auto'.

\config{WLAN\_x\_RTS}{WLAN\_x\_RTS}{WLANxRTS}

      Permets d'activer RTS/CTS Handshake (ou échange de donnée). Cette option
      est utile pour de grand Wlan avec de nombreux clients, si les clients
      entre eux ne s'entendent pas pour émettre sur l'AP (ou Point d'accès),
      alors vous pouvez activer cette variable. Si cette option est activée le
      client à chaque processus d'émission envoie un RTS, demande d'autorisation
      d'émettre et obtient en retour un CTS, autorisation d'émettre par l'AP.
      De cette façon, chaque client écoute et ne transmet pas si un client est
      en train de transmettre. Ainsi, les collisions sont réduites, parce qu'on
      est garantie qu'il y a toujours un seul client qui transmet. Cette option
      est valable uniquement si la situation qui est décrite plus haut a un sens,
      car elle ajoute des données supplémentaires dans l'en-tête du paquet et
      donc réduit l'ensemble de la bande passante. Mais par la réduction des
      collisions, cette bande passante peut augmente de nouveau.

      Cette fonctionnalité est optionnelle et doit être ajoutée manuellement
      dans le fichier config/wlan.txt.

\config{WLAN\_x\_ENC\_N (obsolète)}{WLAN\_x\_ENC\_N}{WLANxENCN}

      On indique ici le nombre de clé pour le cryptage du réseau sans fil.

      Valeurs possibles~: de 0 à 4

\config{WLAN\_x\_ENC\_x (obsolète)}{WLAN\_x\_ENC\_x}{WLANxENCx}

      On place ici les clés pour le cryptage du réseau sans fil.

      Valeurs possibles~:

        \begin{tabular}[h]{ll}
        XXXX-XXXX-XXXX-XXXX-XXXX-XXXX-XX &       128 Bit Hex-Key (X=0-F) \\
        XXXX-XXXX-XX                     &        64 Bit Hex-Key (X=0-F) \\
        s:$<$5 Caractères$>$             &        64 Bit \\
        s:$<$6-13 Caractères$>$          &       128 Bit \\
        P:$<$1-64 Caractères$>$          &       128 Bit \\
        \end{tabular}

      Procédure d'attribution de la Key-Hex avec l'option s: le texte pour la
      phrase mot de passe n'est \textbf{pas} compatible avec les pilotes Windows.
      Veuillez utiliser une Key-Hex pour Windows, la Key-Hex est utilisée
      généralement \textbf{sans} les traits d'union '-'. Procédure d'attribution de
      la Key-Hex avec l'option P: le $<$Texte$>$ pour la phrase mot de passe est
      compatible avec la plupart les pilotes WLAN de Windows (si c'est)
      \textbf{uniquement} dans le mode 128 bits. Linux permet de mélanger des longueurs
      de Key différentes. Mais dans la majorité des cas \textbf{pas} avec les pilotes
      WLAN de Windows~!

\config{WLAN\_x\_ENC\_ACTIVE (obsolète)}{WLAN\_x\_ENC\_ACTIVE}{WLANxENCACTIVE}

      On indique dans cette variable le nombre de Key-Hex à activer
      pour le réseau sans fil.

      Valeurs possibles~: 1-4

     Cette variable est à activer, si la variable \var{WLAN\_x\_ENC\_N} > 0 est
     suppérieur à zéro. Cette variable est optionnelle.

\config{WLAN\_x\_ENC\_MODE (obsolète)}{WLAN\_x\_ENC\_MODE}{WLANxENCMODE}

      On indique ici le mode de cryptage actif.

      Valeurs possibles~:

        \begin{tabular}[h]{ll}
          on/off &         Avec ou sans cryptage \\
          open   &         Accepte aussi les paquets non cryptés \\
          restricted &     Accepte seulement les paquets cryptés \\
        \end{tabular}

      Valeur logique~: 'restricted'

      Cette fonctionnalité est optionnelle et doit être ajoutée manuellement
      dans le fichier config/wlan.txt. Si la variable n'est pas dans le fichier
      elle est par défaut sur 'off' si aucune clé WEP n'est défini et si le
      paramètre 'restricted' est activé alors il faut au moins définir une clé Hex.

\config{WLAN\_x\_WPA\_KEY\_MGMT}{WLAN\_x\_WPA\_KEY\_MGMT}{WLANxWPAKEYMGMT}

      Si vous voulez utiliser WPA au lieu du cryptage WEP, vous pouvez paramétrer
      ici le mode WPA. En ce moment, il n'y a que le mode WPA-PSK qui est
      supporté, la clé WPA est reconnu entre le client et le Point d'accés. Cette
      clé doit être choisie avec soin et ne doit pas être trop courte, sinon elle
      sera vulnérable contre des attaques par recherche de mot "dictionnaire".

      Carte et pilote, supportant le mode-\emph{managed} avec Wpa-Supplicant
      (voir \altlink{http://hostap.epitest.fi/wpa_supplicant/} et le mode-\emph{master}
      avec le démon Hostapd (voir \altlink{http://hostap.epitest.fi/hostapd/}).

      Déjà quelques cartes ont été testées avec succès par ex. avec le chipset
      Atheros et le pilote hostap, ces cartes supportent (le mode managed et
      aussi master). Normalement il y a aussi les cartes atmel et quelques
      autres cartes. Il faut absolument que les développeurs d'Opt adaptent
      en conséquence leurs paquetages Opt.

\config{WLAN\_x\_WPA\_PSK}{WLAN\_x\_WPA\_PSK}{WLANxWPAPSK}

      Vous indiquez ici la clé qui doit être utilisée entre la communication du
      client et le point d'accès. Cette clé est enregistrée sous la forme d'une
      phrase mot de passe, cette (phrase) doit avoir une longueur minimum de 16
      caractères et un maximum de 63 caractères.
      Les caractères suivants sont supportés:

      a-z A-Z 0-9 ! \# \$ \% \& ( ) * + , - . / : ; {\textless} = {\textgreater} ? @ [ {\textbackslash} ] \^{} \_ \`{} \{ | \} {\textasciitilde}

\config{WLAN\_x\_WPA\_TYPE}{WLAN\_x\_WPA\_TYPE}{WLANxWPATYPE}

      Vous pouvez indiquer ici le type de cryptage, 1 pour WPA1, 2 pour WPA2
      (IEEE 802.11i) et 3 pour les deux modes~- Le client peut alors décider s'il
      veut utiliser WPA1 ou WPA2. Si le matériel WLAN supporte le cryptage
      standard, il est préférable d'indiquer le procédé WPA2.

\config{WLAN\_x\_WPA\_ENCRYPTION}{WLAN\_x\_WPA\_ENCRYPTION}{WLANxWPAENCRYPTION}

      Le protocole de cryptage TKIP et la version améliorée du CCMP (protocole
      Mac-AES CTR/CBC, parfois appelé AES) sont ici à votre disposition.
      Malheureusement le protocole CCMP n'est pas supporté par le matériel WLAN
      ancien. Il est également possible de spécifier les deux protocoles
      en même temps.

\config{WLAN\_x\_WPA\_DEBUG}{WLAN\_x\_WPA\_DEBUG}{WLANxWPADEBUG}

      Si vous avez des problèmes avec la connexion WPA, vous pouvez définir
      cette variable sur 'yes', ainsi le démon compétent peut enregistrer toutes
      les informations dans un journal. Vous pourrez ensuite l'utiliser pour
      diagnostiquer les problèmes.

\config{WLAN\_x\_AP}{WLAN\_x\_AP}{WLANxAP}

      Enregistré ici le n\oe{}ud du Point d'accés.

      Vous pouvez indiquer ici l'adresse Mac du Point d'accès. Si vous avez
      choisi le mode "Master" pour le WLAN, cette variable doit rester vide.
      Cette option est logique uniquement si fli4l ne trouve pas le PA par
      lui-même ou le Point d'accès qui est attaché. Uniquement destiné à être
      utilisé avec le mode "managed" pour le WLAN.

      Cette fonctionnalité est optionnelle et doit être ajoutée manuellement
      dans le fichier config/wlan.txt

\config{WLAN\_x\_ACL\_POLICY}{WLAN\_x\_ACL\_POLICY}{WLANxACLPOLICY}

      Politique pour l'ACL (ou Liste des Contrôles d'Accés).

      Installation par défaut~: \var{WLAN\_x\_ACL\_\-POLICY}='allow'

      Valeurs pour lesquelles les adresses MAC sont soumises~:

        \begin{tabular}[h]{ll}
          deny  & Les adresses Mac de la liste ne peuvent pas se connectées \\
          allow & Les adresses Mac de la liste peuvent se connectées \\
          open  & Toutes les adresses Mac reçoivent indépendamment un filtrage
          d'accès \\
        \end{tabular}

      Malheureusement WLAN\_ACL est actuellement pris en charge uniquement par
      le pilote hostap\_*. Comme alternative, vous avez la possibilité de
      paramétrer les options de contrôles d'accés dans le firewall, car des
      progrès significatifs ont été réalisés dans la version 3.0.x

\config{WLAN\_x\_ACL\_MAC\_N}{WLAN\_x\_ACL\_MAC\_N}{WLANxACLMACN}

      AP-ACLs~- Restriction des stations WLAN autorisées.

      Installation par défaut~: \var{WLAN\_x\_ACL\_MAC\_N}='0'

      En indiquant un chiffre supérieur à '0' on active la liste de contrôle
      d'accès (filtrage les adresses Mac) et indique le nombre d'entrées ACL. La
      liste de contrôle d'accès est une liste d'adresses MAC, qui autorise ou
      interdit l'accès au point d'accés. Le nombre défini les adresses Mac qui
      peut être activées.

\config{WLAN\_x\_ACL\_MAC\_x}{WLAN\_x\_ACL\_MAC\_x}{WLANxACLMACx}

      On indique ici les adresses Mac sous la forme~: 00:00:E8:83:72:92

\config{WLAN\_x\_DIVERSITY}{WLAN\_x\_DIVERSITY}{WLANxDIVERSITY}

      Cette variable vous permet, d'activer la diversité des antennes manuellement.

      Installation par défaut~: \var{WLAN\_x\_DIVERSITY}='no' (choix automatique)

\config{WLAN\_x\_DIVERSITY\_RX}{WLAN\_x\_DIVERSITY\_RX}{WLANxDIVERSITYRX}

      Ici vous sélectionnez l'antenne de réception.

      Installation par défaut~: \var{WLAN\_x\_DIVERSITY\_RX}='1'

        \begin{tabular}[h]{ll}
        0 = Sélection automatique\\
        1 = Antenne 1\\
        2 = Antenne 2\\
        \end{tabular}

\config{WLAN\_x\_DIVERSITY\_TX}{WLAN\_x\_DIVERSITY\_TX}{WLANxDIVERSITYTX}

      Ici vous sélectionnez l'antenne de transmission.

      Installation par défaut~: \var{WLAN\_x\_DIVERSITY\_TX}='1'

\config{WLAN\_x\_WPS}{WLAN\_x\_WPS}{WLANxWPS}

        Avec cette variable vous pouvez activer le support WPS (WiFi Protected Setup).
        Dans le WLAN\_WEBGUI il sera alors affiché un bouton de commande et un
        code PIN. Vous pouvez aussi contrôler le WPS par ligne de commande.

        Installation par défaut~: \var{WLAN\_x\_WPS}='no'

\config{WLAN\_x\_PSKFILE}{WLAN\_x\_PSKFILE}{WLANxPSKFILE}

        Si vous activez PSKFILE, vous allez en plus de la configuration de
        \var{WLAN\_x\_WPA\_PSK} pouvoir utiliser une clé pré-partagée pour se
        connecté à d'autres clients. Actuellement la configuration se fait par
        la fonction \var{WLAN\_x\_WPS}, qui utilise un fichier pour fournir les
        clés individuelle aux clients.

        Si le fichier est désactivé et si la connexion des clients WPS sont
        reliés à ce fichier, ils ne seront plus en mesure de se connecter au
        Point d'Accés.

        Les clients WPS qui se connectent après la désactivation de ce fichier
        ne seront pas affectés.

        Installation par défaut~: \var{WLAN\_x\_PSKFILE}='yes'

\config{WLAN\_x\_BRIDGE}{WLAN\_x\_BRIDGE}{WLANxBRIDGE}

        Dans cette variable vous pouvez indiquer le bridge qui sera rattaché au
        Wlan, alternativement avec le paquetage ADVANCED\_NETWORKING.

        Exemple~: \var{WLAN\_x\_BRIDGE}='br0'

        Attention~: soit vous indiquez la valeur ici soit dans Advanced-Network~!
                    Mais \textbf{pas} dans les deux fichier de configuration~!
\end{description}

\subsection {Web GUI (ou interface Web)}

	Ce paquetage met à disposition une interface Web qui fonctionne avec mini-httpd. 
    L'interface Web sera automatiquement activée si la variable \var{OPT\_HTTPD='yes'}
	du paquetage httpd est activée.

\subsection{Exemple}
\subsubsection{Connexion à un point d'accès via WPA}

\begin{example}
\begin{verbatim}
OPT_WLAN='yes'
WLAN_N='1'
WLAN_1_MAC='00:0F:A3:xx:xx:xx'
WLAN_1_ESSID='foo'
WLAN_1_MODE='managed'           # liaison au Point d'accés
WLAN_1_CHANNEL='1'
WLAN_1_RATE='auto'
#
# Configuration WPA
#
WLAN_1_ENC_N='0'                # Key WEP
WLAN_1_WPA_KEY_MGMT='WPA-PSK'   # WPA pre shared key
WLAN_1_WPA_TYPE='1'             # WPA 1
WLAN_1_WPA_ENCRYPTION='TKIP'
WLAN_1_WPA_PSK='phrase Mot de Passe solide entre (16-63 caractères)'
#
# Dans un contexte WPA insignifiant
#
WLAN_1_ENC_N='0'
WLAN_1_ENC_ACTIVE='1'
WLAN_1_ACL_POLICY='allow'
WLAN_1_ACL_MAC_N='0'
\end{verbatim}
\end{example}


\subsubsection{Connexion à un point d'accès avec le cryptage WPA2}

\begin{example}
\begin{verbatim}
OPT_WLAN='yes'
WLAN_N='1'
WLAN_1_MAC='00:0F:A3:xx:xx:xx'
WLAN_1_ESSID='foo'
WLAN_1_MODE='master'            # Point d'accés
WLAN_1_CHANNEL='1g'             # Channel 1, Mode sur 'g'
                                # Carte-Atheros
WLAN_1_RATE='auto'
#
# Configuration WPA
#
WLAN_1_ENC_N='0'                # Key WEP
WLAN_1_WPA_KEY_MGMT='WPA-PSK'   # WPA pre shared key
WLAN_1_WPA_TYPE='2'             # WPA 2
WLAN_1_WPA_ENCRYPTION='CCMP'
WLAN_1_WPA_PSK='Pass-phrase solide (16-63 caractères)'
#
# Contrôle d'accès des adresses MAC basé sur le AP
#
WLAN_1_ACL_POLICY='allow'
WLAN_1_ACL_MAC_N='0'
#
# Dans un contexte WPA insignifiant
#
WLAN_1_ENC_ACTIVE='1'
\end{verbatim}
\end{example}


\subsubsection{Connexion à un point d'accès avec le cryptage WEP}

\begin{example}
\begin{verbatim}
OPT_WLAN='yes'
WLAN_N='1'
WLAN_1_MAC='00:0F:A3:xx:xx:xx'
WLAN_1_ESSID='foo'
WLAN_1_MODE='master'            # Point d'accés
WLAN_1_CHANNEL='1'
WLAN_1_RATE='auto'
#
# Configuration WEP
#
WLAN_1_WPA_KEY_MGMT=''          # Key WPA
WLAN_1_ENC_N='4'                # 4 WEP-Keys
WLAN_1_ENC_1='...'
WLAN_1_ENC_2='...'
WLAN_1_ENC_3='...'
WLAN_1_ENC_4='...'
WLAN_1_ENC_ACTIVE='1'           # première clé actif
#
# Contrôle d'accès des adresses MAC basé sur le Point d'accès
#
WLAN_1_ACL_POLICY='allow'
WLAN_1_ACL_MAC_N='0'
# 
# Configuration WEP insignifiant
#
WLAN_1_WPA_TYPE='2'
WLAN_1_WPA_ENCRYPTION='CCMP'
WLAN_1_WPA_PSK='...'
\end{verbatim}
\end{example}


\subsection{Point d'accés virtuel (VAP) (Expérimental)}

Certaines cartes WLAN avec les (pilotes~: ath\_pci, ath5k, ath9k, ath9k\_htc)
peuvent distribuer jusqu'à 4 cartes WLAN virtuelles (VAP).

La configuration du WLAN pour un point d'accès virtuel (VAP) doit avoir les
conditions suivantes~: avoir le même canal et la même adresse MAC. Au moyen de
l'adresse MAC utilisée plusieurs fois, la carte sera fractionnée et identifiée.
Si vous avez plusieurs cartes, cette opération peut être faite plusieurs fois.

Le périphérique de base s'appellera wlan0 (pour la carte WLAN). Pour le VAP
wlan0v2 etc... Si vous utilisez un bridge pour faire un lien, indiquer S.V.P.
WLAN\_x\_BRIDGE='br0' etc...

Actuellement la configuration maximum pour le VAP est de 8x Master en fonction
de la carte et du pilote.


\subsection{Réglage de l'heure pour l'arrêt du WLAN avec easycron}

Au moyen du paquetage \jump{sec:opt-easycron}{\emph{easycron}}, vous pouvez
arrêter et redémarrer votre carte WLAN à une heure précise.

\begin{example}
\begin{verbatim}
EASYCRON_N='2'
EASYCRON_1_CUSTOM  = ''     # Arrêt tous les soirs à 24 heures.
EASYCRON_1_COMMAND = '/usr/sbin/wlanconfig.sh wlan0 down'
EASYCRON_1_TIME    = '* 24 * * *'

EASYCRON_2_CUSTOM  = ''     # Reprise le matin à 8 heures.
EASYCRON_2_COMMAND = '/usr/sbin/wlanconfig.sh wlan0'
EASYCRON_2_TIME    = '* 8 * * *'
\end{verbatim}
\end{example}


\subsection{Remarque et dons}

      Les cartes WLAN avec le chipset RT25xx peuvent être utilisées avec fli4l
      dans les modes ad-hoc et managed, grâce à la donation généreuse de 2
      cartes WLAN Ralink 2500. Le pilote a pu être ajouté dans le fichier
      base.txt sous le nom rt2500. Ces cartes ont été données par~:

      Computer Contor, Pilgrimstein 24a, 35037 Marburg, Allemagne
