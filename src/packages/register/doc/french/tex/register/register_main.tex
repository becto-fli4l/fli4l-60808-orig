% Do not remove the next line
% Synchronized to r54830

\section{REGISTER - Envoie les informations système à l'équipe fli4l}

Le démon de register recueille des données du routeur et les envoie
anonymement à l'équipe fli4l pour des statistiques. Les données suivantes
seront transmises~:

\begin{itemize}
\item L'identifiant unique du routeur
\item L'heure de la collecte des données
\item Le numéro exact de la version 
\item La version du kernel (ou noyau)
\item L'architecture du système
\item Tous les OPT activés avec le paquetage les contenant
\end{itemize}

Aucune autre donnée ne sera transmise. En particulier, aucun autre paramétre
de configuration ne sera envoyé à l'équipe fli4l.

La motivation de ce paquetage est de permettre à l'équipe fli4l de déterminer
quels paquetages et quels OPT sont utilisés sur les routeurs fli4l. Ces informations
sont utilisées pour déterminer les paquetages les moins utilisés (les moins testés)
et donc ils seront peut être plus tard inutiles. D'autre part, les paquetages qui
sont utilisés par un plus grande nombre d'utilisateurs peuvent être identifiés pour
une plus grande attention.

L'identifiant unique est utilisé pour identifier et affecter plusieurs transferts de
données provenant du même routeur sur la page de l'équipe fli4l. Pour ce faire, l'UUID
configuré dans la variable \var{FLI4L\_UUID} sera transmis. La configuration de cette
variable est donc une condition préalable à l'utilisation de ce paquetage.

Le résultat des l'analyses des données sont disponibles sur le site Web
\altlink{https://register.fli4l.de}

\subsection{Conditions}

Le paquetage \var{OPT\_REGISTER} nécessite les OPT suivants pour fonctionner~:
\begin{itemize}
\item \var{OPT\_CURL} (Paquetage "tools")
\item \var{OPT\_CERT\_X509} (Paquetage "cert")
\end{itemize}

\subsection{Installation}

\begin{description}

\config{OPT\_REGISTER}{OPT\_REGISTER}{OPTREGISTER}

Paramètre par défaut~: \var{OPT\_REGISTER='no'}

Avec ce paramètre \var{OPT\_REGISTER='yes'} vous activez le démon register pour fli4l.

\config{REGISTER\_INTERVAL}{REGISTER\_INTERVAL}{REGISTERINTERVAL}

Paramètre par défaut~: \var{REGISTER\_INTERVAL='86400'}

Dans cette variable vous indiquez la fréquence à la quelle le démon register transfère
les données du routeur à l'équipe fli4l. Vous pouvez s'il vous plaît définir la durée en
secondes entre deux transferts. Le premier transfert de données est initié directement
après le processus de démarrage. La valeur par défaut est 86400 secondes, cela déclenche
une transmission de données par jour.

Pourquoi transférer les données régulièrement, comme cela il sera possible de suivre la
capture des données d'un routeur pour la dernière fois et donc mieux juger leur pertinence.

Si vous indiquez la valeur '0', le transfert de données sera lancé une seule fois directement
après le processus de démarrage.

La plus petite valeur possible (à par du '0') est 600 secondes.

\config{REGISTER\_NUM\_ATTEMPTS}{REGISTER\_NUM\_ATTEMPTS}{REGISTERNUMATTEMPTS}

Paramètre par défaut~: \var{REGISTER\_NUM\_ATTEMPTS='5'}

Dans cette variable vous indiquez le nombre maximal de tentatives de transfert de données,
cette variable est utilisée pour un transfert de données planifié.

La plus petite valeur possible est '1'.

\config{REGISTER\_RETRY\_INTERVAL}{REGISTER\_RETRY\_INTERVAL}{REGISTERRETRYINTERVAL}

Paramètre par défaut~: \var{REGISTER\_RETRY\_INTERVAL='60'}

Dans cette variable vous indiquez le délai en secondes entre deux tentatives de transfert
de données.

La plus petite valeur possible est de 60 secondes.

\end{description}
