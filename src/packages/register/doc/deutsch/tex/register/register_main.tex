% Last Update: $Id$
\section{REGISTER - Systeminformationen an das fli4l-Team schicken}

Der Register-Dämon stellt Daten über den Router zusammen und sendet sie
anonymisiert an das fli4l-Team zu statistischen Zwecken. Die folgenden Daten
werden übermittelt:

\begin{itemize}
\item die eindeutige Kennung des Routers
\item der Zeitpunkt der Datenerhebung
\item die genaue Versionsnummer
\item die Kernelversion
\item die Architektur
\item alle aktivierten OPTs zusammen mit dem Paket, in dem sie enthalten sind
\end{itemize}

Es werden keine weiteren Daten übertragen. Insbesondere werden keine weiteren
Konfigurationseinstellungen an das fli4l-Team gesendet.

Die Motivation dieses Pakets ist, dass das fli4l-Team feststellen kann, welche
Pakete und welche OPTs auf aktiven fli4l-Routern im Einsatz sind. Diese
Informationen werden verwendet, um festzustellen, welche Pakete nur wenig
genutzt (und somit getestet) werden und somit langfristig unter Umständen
entbehrlich sind, und welche Pakete häufig im Einsatz sind und somit unter
Umständen einer verstärkten Pflege bedürfen.

Die eindeutige Kennung dient dazu, verschiedene Datenübertragungen auf der
Seite des fli4l-Teams demselben Router zuordnen zu können. Es wird dazu die
UUID übermittelt, die in der Variable \var{FLI4L\_UUID} konfiguriert wird.
Das Konfigurieren dieser Variable ist deshalb eine Voraussetzung für die
Nutzung dieses Pakets.

Die Ergebnisse der Datenanalyse können unter \altlink{https://register.fli4l.de}
eingesehen werden.

\subsection{Voraussetzungen}

\var{OPT\_REGISTER} setzt folgende OPTs voraus:
\begin{itemize}
\item \var{OPT\_CURL} (im Paket ``tools'')
\item \var{OPT\_CERT\_X509} (im Paket ``cert'')
\end{itemize}

\subsection{Installation}

\begin{description}

\config{OPT\_REGISTER}{OPT\_REGISTER}{OPTREGISTER}

Standard-Einstellung: \var{OPT\_REGISTER='no'}

Mit \var{OPT\_REGISTER='yes'} wird der Register-Dämon auf dem fli4l aktiviert.

\config{REGISTER\_INTERVAL}{REGISTER\_INTERVAL}{REGISTERINTERVAL}

Standard-Einstellung: \var{REGISTER\_INTERVAL='86400'}

Hier kann man einstellen, wie oft der Register-Dämon die Daten über den Router
an das fli4l-Team senden soll. Es wird die Anzahl der Sekunden zwischen zwei
Übertragungen angegeben. Die erste Datenübertragung wird direkt nach dem
Bootvorgang eingeleitet. Der Standardwert von 86400 Sekunden hat zur Folge,
dass eine Datenübertragung einmal täglich erfolgt.

Die regelmäßige Übertragung der Daten ist dadurch motiviert, dass man darüber
verfolgen kann, wann die Daten eines Routers zum letzten Mal erfasst worden
sind, und somit deren Relevanz besser bewerten kann.

Wird hier '0' angegeben, wird die Datenübertragung nur einmalig direkt nach
Abschluss des Bootvorgangs angestoßen.

Der kleinste mögliche Wert (abgesehen von null) ist 600 Sekunden.

\config{REGISTER\_NUM\_ATTEMPTS}{REGISTER\_NUM\_ATTEMPTS}{REGISTERNUMATTEMPTS}

Standard-Einstellung: \var{REGISTER\_NUM\_ATTEMPTS='5'}

Mit dieser Einstellung wird die maximale Anzahl der Datenübertragungsversuche
für eine geplante Datenübertragung angegeben.

Der kleinste mögliche Wert ist 1.

\config{REGISTER\_RETRY\_INTERVAL}{REGISTER\_RETRY\_INTERVAL}{REGISTERRETRYINTERVAL}

Standard-Einstellung: \var{REGISTER\_RETRY\_INTERVAL='60'}

Mit dieser Einstellung wird die Anzahl von Sekunden eingestellt, die zwischen
zwei Datenübertragungsversuchen gewartet werden soll.

Der kleinste mögliche Wert ist 60 Sekunden.

\end{description}
