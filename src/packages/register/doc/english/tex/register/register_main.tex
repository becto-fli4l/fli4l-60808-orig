% Synchronized to r49399
\section{REGISTER - Send system information to the fli4l team}

The register daemon collects data about the router and sends it 
anonymised to the fli4l team for statistical purposes. The following data
will be transmitted:

\begin{itemize}
\item the unique identifier of the router
\item the time of data collection
\item the exact version number
\item the kernel version
\item system architecture
\item all activated OPTs together with the package containing them
\end{itemize}

No further data will be transmitted. In particular, no further
configuration settings will be sent to the fli4l team.

The motivation for this package is to enable the fli4l team to determine which 
packages and which OPTs are used on fli4l routers. These informations are used
to determine which packages are less used (and thus less tested) and so in the
long term may be unnecessary. On the other hand packages which are used by a
larger user base may be identified in order to get additional attention.

The unique identifier is used to identify and assign multiple data transfers
coming from the same router on the fli4l team page. To achieve that the
UUID that is configured in the variable \var{FLI4L\_UUID} will be transmitted.
The configuration of this variable hence is a prerequisite for the use of
this package.

The results of data analysis can be found
at \altlink{https://register.fli4l.de}.

\subsection{Requirements}

\var{OPT\_REGISTER} requires the following OPTs:
\begin{itemize}
\item \var{OPT\_CURL} (Package ``tools'')
\item \var{OPT\_CERT\_X509} (Package ``cert'')
\end{itemize}

\subsection{Installation}

\begin{description}

\config{OPT\_REGISTER}{OPT\_REGISTER}{OPTREGISTER}

Default Setting: \var{OPT\_REGISTER='no'}

\var{OPT\_REGISTER='yes'} activates the Register daemon for fli4l.

\config{REGISTER\_INTERVAL}{REGISTER\_INTERVAL}{REGISTERINTERVAL}

Default Setting: \var{REGISTER\_INTERVAL='86400'}

This specifies how often the register daemon transfers router data to the
fli4l team. Please set this as the timespan in seconds between two
transfers. The first data transfer is initiated directly after the
boot process. The default value of 86400 seconds fires one data
transmission per day.

The motivation for the regular data transfer is that it is possible to track
at what time a router's data was last captured and thus better judge its relevance.

If set to '0' the data transfer is only initiated once directly after the boot process.

The smallest value possible (besides '0') is 600 seconds.

\config{REGISTER\_NUM\_ATTEMPTS}{REGISTER\_NUM\_ATTEMPTS}{REGISTERNUMATTEMPTS}

Default Setting: \var{REGISTER\_NUM\_ATTEMPTS='5'}

This setting determines the maximum number of data transfer attempts
specified for a planned data transfer.

The smallest value possible is 1.

\config{REGISTER\_RETRY\_INTERVAL}{REGISTER\_RETRY\_INTERVAL}{REGISTERRETRYINTERVAL}

Default Setting: \var{REGISTER\_RETRY\_INTERVAL='60'}

This setting is used to specify the timespan in seconds to wait between two data transmission attempts.

The smallest value possible is 60 seconds.

\end{description}
