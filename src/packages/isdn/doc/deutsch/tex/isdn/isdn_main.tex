% Last Update: $Id$
\section {ISDN - Kommunikation über aktive und passive ISDN-Karten}
\configlabel{OPT\_ISDN}{OPTISDN}

fli4l ist vornehmlich zum Einsatz als ISDN- und/oder DSL-Router
gedacht.  Mit der Einstellung \var{OPT\_\-ISDN='yes'} wird das ISDN-Paket
aktiviert.  Voraussetzung ist eine ISDN-Karte, die von fli4l
unterstützt wird.
    
Soll kein ISDN verwendet werden, kann mit der Einstellung
\var{OPT\_\-ISDN}='no' die ISDN-Installation abgeschaltet werden. Dann werden
alle in diesem Kapitel folgenden ISDN-Variablen ignoriert.
    
Standard-Einstellung: \var{OPT\_\-ISDN='no'}


\subsection {Herstellen einer ISDN-Verbindung}
\begin{sloppypar}
Das Einwählverhalten von fli4l wird von drei verschiedenen Variablen
bestimmt, \mbox{\var{DIALMODE},} \var{ISDN\_\-CIRC\_\-X\_\-ROUTE\_X},
\var{ISDN\_\-CIRC\_\-X\_\-TIMES}. Es wird von
\jump{DIALMODE}{\var{DIALMODE}} (in $<$config$>$/base.txt) bestimmt, ob bei Eintreffen eines Paketes auf
einem aktiven Circuit automatisch eine Verbindung aufgebaut werden
soll oder nicht. \var{DIALMODE} kann folgende Werte annehmen:
\end{sloppypar}

\begin{description}
\item[auto] Trifft ein Paket auf einem ISDN-Circuit (bzw. dem daraus
  abgeleiteten ISDN-Interface ippp*) ein, wird automatisch eine Verbindung
  aufgebaut. Ob und wann ein Paket auf einem ISDN-Circuit eintrifft, wird von
  \var{ISDN\_\-CIRC\_\-X\_\-ROUTE\_X} und \var{ISDN\_\-CIRC\_\-X\_\-TIMES}
  bestimmt.
       
\item[manual] Im manuellen Modus muß der Verbindungsaufbau über
  imond/imonc angestoßen werden. Wie das geht, steht im Abschnitt über
  imonc/imond.

 \item[off] Es werden keine ISDN-Verbindungen hergestellt.
\end{description}

Auf welchem der konfigurierten Circuits Pakete eintreffen und damit eine
Einwahl auslösen können, wird über \var{ISDN\_\-CIRC\_\-X\_\-ROUTE\_X}
definiert. Standardmäßig ist es auf '0.0.0.0/0', die sogenannte 'default route'
gesetzt. Das heißt, dass alle Pakete, die das lokale Netz verlassen, über diesen
Circuit gehen, wenn er aktiv ist. Ob und wann er aktiv ist, wird dabei von
\var{ISDN\_\-CIRC\_\-X\_\-TIMES} bestimmt, da fli4l über die definierten
Circuits ein \emph{least cost routing} durchführt (siehe Abschnitt
\jump{sec:leastcostrouting}{Least-Cost-Routing - Funktionsweise} in der
Dokumentation des Grundpaketes). Möchte man nicht alle Pakete über diesen
Circuit leiten, sondern nur Pakete in ein bestimmtes Netz (z.B.  Firmennetz),
kann man hier ein oder mehrere Netze angeben. Dann richtet fli4l eine Route
über das dem Circuit zugeordnete ISDN-Interface ein, die permanent aktiv ist.
Wird nun ein Paket in dieses Netz geschickt, erfolgt automatisch der 
Verbindungsaufbau.

Wie schon erwähnt, beschreibt \var{ISDN\_\-CIRC\_\-X\_\-TIMES} neben den
Verbindungskosten eines Circuits auch, ob und wann ein Circuit mit
einer 'default route' aktiv ist und damit einen Verbindungsaufbau
auslösen kann. Das 'wann' spezifiert man über die Zeit, die ersten
beiden Elemente einer time-info (z.B. Mo-Fr:09-18), das 'ob' durch den
vierten Parameter lc-default-route (y/n). fli4l (bzw. imond) sorgt
dann dafür, dass die Pakete, die das lokale Netz verlassen, immer über
den zu diesem Zeitpunkt aktiven Circuit gehen und damit ein Herstellen
der Verbindung zum Internet-Provider auslösen.

Zusammenfassend kann man also für die Standardanwendungsfälle
folgendes sagen: 

\begin{itemize}
\item Will man einfach nur ins Internet, stellt man \var{DIALMODE}
auf auto, definiert 1-n Circuits, die als erste Route '0.0.0.0/0' haben
und deren Zeiten (Zeiten mit lc-default-route = y) die gesamte Woche
abdecken.
\begin{small}
\begin{example}
\begin{verbatim}
        ISDN_CIRC_%_ROUTE_N='1'
        ISDN_CIRC_%_ROUTE_1='0.0.0.0/0'
        ISDN_CIRC_%_TIMES='Mo-Su:00-24:0.0148:Y'
\end{verbatim}
\end{example}
\end{small}

\item Will man nebenbei noch über eine spezielle Nummer ins
Firmennetz, definiert man sich einen Circuit (oder mehrere Circuits)
mit Route ungleich '0.0.0.0/0' und hat damit einen permanent aktiven
Zugang zum Firmennetz.
\begin{small}
\begin{example}
\begin{verbatim}
        ISDN_CIRC_%_ROUTE_N='1'
        ISDN_CIRC_%_ROUTE_1='network/netmaskbits'
        ISDN_CIRC_%_TIMES='Mo-Su:00-24:0.0148:Y'
\end{verbatim}
\end{example}
\end{small}
\end{itemize}

\subsection{ISDN-Karte}

Der fli4l-Router unterstützt generell die gleichzeitige Verwendung mehrerer
ISDN-Karten. Dies erfordert jedoch, dass alle ISDN-Karten \emph{denselben
Treibertyp} benötigen. Der Treibertyp leitet sich aus der Gruppe in der unten
stehenden Tabelle ab, in welcher der betreffende Treiber zu finden ist. Es ist
also beispielsweise kein Problem, mehrere mISDN-getriebene Adapter oder mehrere
HiSax-getriebene Adapter zu verwenden (sofern genügend Ressourcen vorhanden
sind), es ist aber nicht möglich, eine mISDN-getriebene und eine
HiSax-getriebene Karte zur selben Zeit zu benutzen.

\begin{description}
\configlabel{ISDN\_\%\_IO}{ISDNxIO}
\configlabel{ISDN\_\%\_IO0}{ISDNxIO0}
\configlabel{ISDN\_\%\_IO1}{ISDNxIO1}
\configlabel{ISDN\_\%\_MEM}{ISDNxMEM}
\config{ISDN\_\%\_TYPE ISDN\_\%\_IO ISDN\_\%\_IO0 ISDN\_\%\_IO1 ISDN\_\%\_MEM ISDN\_\%\_IRQ} {ISDN\_\%\_TYPE} {ISDNxTYPE}

  Hier sind die technischen Daten für die ISDN-Karte anzugeben.
  
  Die im Beispiel aufgeführten Werte funktionieren für eine TELES
  16.3, wenn die Karte auf IO-Adresse 0xd80 eingestellt ist (über
  Dip-Switches).  Bei einer anderen Einstellung der Karte muss der
  Wert geändert werden.

  \achtung{Häufig gemachter Fehler (Beispiel):}
  \begin{example}
  \begin{verbatim}
          ISDN_1_IO='240' -- richtig wäre: ISDN_1_IO='0x240'
  \end{verbatim}
  \end{example}

  Bei IRQ 12 muss man einen eventuell vorhandenen PS/2-Maus-Anschluss im
  BIOS abschalten. Sonst besser einen anderen IRQ wählen!
  \glqq{}Gute\grqq{} sind meist 5, 10 und 11.
  
  \var{ISDN\_\%\_\-TYPE} entspricht prinzipiell den Typ-Nummern für den
  HiSax-Treiber. Ausnahme: nicht-HiSax-Karten wie z.B. AVM-B1. Für diese
  wurde der Nummernkreis für die Typen erweitert (siehe unten). Die Liste
  der möglichen HiSax-Typen basiert auf \\
  \var{linux-2.x.y/Documentation/isdn/README.HiSax}.

\begin{small}
  \begin{longtable}{|r|p{60mm}|p{62mm}|}
    \hline
    \multicolumn{1}{|c}{\textbf{Typ}} & \multicolumn{1}{|c}{\textbf{Karte}} &
    \multicolumn{1}{|c|}{\textbf{Benötigte Parameter}} \\
    \hline\hline
    \endhead

    \multicolumn{3}{|l|}{Dummy Typ-Nummer:} \\
    \hline
      0 &  no driver (dummy)             & none \\

    \hline\hline
    \multicolumn{3}{|l|}{Typ-Nummern für Remote-CAPI-Treiber:} \\
    \hline

    160 & AVM Fritz!Box Remote CAPI   & ip,port \\
    161 & Melware Remote CAPI (rcapi) & ip,port \\

    \hline\hline
    \multicolumn{3}{|l|}{Typ-Nummern für mISDN-Treiber:} \\
    \hline

    301 & HFC-4S/8S/E1 multiport cards         & no parameter \\
    302 & HFC-PCI based cards                  & no parameter \\
    303 & HFCS-USB Adapters                    & no parameter \\
    304 & AVM FritZ!Card PCI (v1 and v2) cards & no parameter \\
    305 & cards based on Infineon (former Siemens) chips: \newline
          - Dialogic Diva 2.0 \newline
          - Dialogic Diva 2.0U \newline
          - Dialogic Diva 2.01 \newline
          - Dialogic Diva 2.02 \newline
          - Sedlbauer Speedwin \newline
          - HST Saphir3 \newline
          - Develo (former ELSA) Microlink PCI (Quickstep 1000) \newline
          - Develo (former ELSA) Quickstep 3000 \newline
          - Berkom Scitel BRIX Quadro \newline
          - Dr.Neuhaus (Sagem) Niccy           & no parameter \\
    306 & NetJet TJ 300 and TJ320 cards        & no parameter \\
    307 & Sedlbauer Speedfax+ cards            & no parameter \\
    308 & Winbond 6692 based cards             & no parameter \\
    \hline
  \end{longtable}
\end{small}
  
  Meine Karte ist eine Teles 16.3 NON-PNP ISA, also ist Type=3.

  Für eine ICN-2B-Karte müssen IO und MEM gesetzt werden, zum Beispiel
  \var{ISDN\_1\_\-IO}='0x320', \var{ISDN\_1\_\-MEM}='0xd0000'.

  Bei neueren Teles-PCI-Karten muss type=20 (statt 21) verwendet
  werden.  Die Dinger melden sich bei ``cat /proc/pci'' mit ``tiger'' oder
  so ähnlich.  Sonst kann ich nichts zu diesen Werten beitragen,
  sorry.

  Um die ISDN-Typen 105 bis 114 verwenden zu können, ist es vorher nötig, die
  passenden Treiberdateien von \altlink{http://www.fli4l.de/download/stabile-version/avm-treiber/}
  herunterzuladen und in das fli4l-Verzeichnis zu entpacken. Da diese Treiber
  nicht der GPL unterliegen, können sie leider nicht mit dem ISDN Paket
  mitgeliefert werden.\\

  Für den ISDN-Typ 303 ist es nötig USB Support zu 
  installieren und zu aktivieren. Siehe dazu 
  \jump{sec:opt-usb }{USB - Support für USB-Geräte}.\\

  Tips zu den Typ-Nummern bekommt man auch über die i4l-FAQ oder
  Mailingliste, wenn man wirklich nicht weiß, was da für eine Karte im
  PC steckt.

  Die Kartentypen, die mit \glqq{}from isapnp setup\grqq{} gekennzeichnet sind,
  müssen mit dem PnP-Tool isapnp initialisiert werden - wenn es sich
  tatsächlich um eine PnP-Karte handelt. Siehe dazu die Beschreibung im Kapitel
  \jump{OPTPNP}{OPT\_PNP - Installation von isapnp tools}.

  Der ISDN-Typ 0 wird dann benötigt, wenn man das ISDN-Paket installieren will
  ohne ISDN-Karte; z.B. um imond verwenden zu können bei einem Netzwerkrouter.

\configlabel{ISDN\_\%\_IP}{ISDNxIP}
\configlabel{ISDN\_\%\_PORT}{ISDNxPORT}
\config{ISDN\_\%\_IP ISDN\_\%\_PORT} {ISDN\_\%\_IP} {ISDNxIP}

  Für die ISDN-Typen 160 und 161 werden über diese Variablen die IP-Adresse
  (\var{ISDN\_\%\_IP}) bzw. der Port (\var{ISDN\_\%\_PORT}) des Gerätes
  eingestellt, das eine entfernte CAPI-Schnittstelle anbietet. Die IP-Adresse
  ist zwingend erforderlich, die Port-Nummer kann hingegen weggelassen werden:
  Je nach gewähltem Typ wird dann ein Standard-Port eingestellt (Typ 160: 5031,
  Typ 161: 2662).

  Beispiel:
  \begin{example}
  \begin{verbatim}
          ISDN_1_TYPE='160' # AVM Fritz!Box
          ISDN_1_IP='192.168.177.1'
  \end{verbatim}
  \end{example}

\config{ISDN\_DEBUG\_LEVEL}{ISDN\_DEBUG\_LEVEL}{ISDNDEBUGLEVEL} 
  
  Dieses gibt den Debug-Level für den HiSaX-Treiber an.
  Der Debug-Level setzt sich dabei durch Addition der folgenden Werte
  zusammen (Zitat aus der Orginal-Doku):\\
  
  \begin{tabular}[h!]{r|l}
   \multicolumn{1}{c|}{\textbf{Number}} & \multicolumn{1}{c}{\textbf{Debug-Information}} \\
   \hline
      1 & Link-level $<$--$>$ hardware-level communication \\
      2 & Top state machine \\
      4 & D-Channel Q.931 (call control messages) \\
      8 & D-Channel Q.921 \\
     16 & B-Channel X.75 \\
     32 & D-Channel l2 \\
     64 & B-Channel l2 \\
    128 & D-Channel link state debugging \\
    256 & B-Channel link state debugging \\
    512 & TEI debug \\
   1024 & LOCK debug in callc.c \\
   2048 & More debug in callc.c (not for normal use) \\
  \end{tabular}\latex{\\}

  Die Standardeinstellung (\var{ISDN\_DEBUG\_LEVEL}='31') sollte den meisten reichen.
  
\config{ISDN\_VERBOSE\_LEVEL}{ISDN\_VERBOSE\_LEVEL}{ISDNVERBOSELEVEL}
  
  Hiermit kann man die ``Geschwätzigkeit'' des ISDN-Subsystems im
  fli4l-Kernel einstellen. Jeder Verbose-Level schließt die Level mit
  niedrigerer Nummer mit ein. Die Verbose-Level sind:

  \begin{tabular}[h!]{lp{10cm}}
    '0' & keine zusätzlichen Informationen \\
    '1' & Es wird protokolliert, was eine ISDN-Verbindung ausgelöst hat\\
    '2' und '3' & Anrufe werden protokolliert\\
    '4' und mehr & Die Datentransferrate wird regelmäßig protokolliert. \\
        % Beim letzten bin ich mir nicht ganz sicher, siehe
        % linux-kernel-source/drivers/isdn/isdn_net.c, suche nach "dev->net_verbose > 3"
  \end{tabular}

  Die Meldungen werden über das Kernel-Logging-Interface ausgegeben, erscheinen
  also bei aktiviertem \jump{OPTSYSLOGD}{\var{OPT\_SYSLOGD}} dort.

  \wichtig{Sollen Anrufe mit telmond protokolliert werden, diesen Wert nicht
  kleiner als 2 einstellen, da telmond sonst die Informationen fehlen, um
  Anrufe zu protokollieren.}
  
  Standardeinstellung: \var{ISDN\_VERBOSE\_LEVEL}='2'

\begin{sloppypar}
\config{ISDN\_FILTER}{ISDN\_FILTER}{ISDNFILTER}

Aktiviert den Filtermechanismus des Kerns, um ein ordnungsgemäßes
Auflegen nach dem angegebenen Hangup-Timeout zu gewährleisten. Siehe
\altlink{http://www.fli4l.de/hilfe/howtos/basteleien/hangup-problem-loesen/} 
für genauere Informationen.
\end{sloppypar}

\config{ISDN\_FILTER\_EXPR}{ISDN\_FILTER\_EXPR}{ISDNFILTEREXPR}

Hier steht der zu nutzende Filter, wenn \var{ISDN\_FILTER} auf `yes' gesetzt ist.

\end{description}


\subsection{OPT\_ISDN\_COMP (EXPERIMENTAL)}
\configlabel{OPT\_ISDN\_COMP}{OPTISDNCOMP}

Mit \var{OPT\_\-ISDN\_\-COMP}='yes' werden die LZS- und die BSD-Kompression
aktiviert. Das Kompressionspaket hat freundlicherweise Arwin Vosselman
(\email{arwin(at)xs4all(dot)nl}) zusammengestellt. Dieses Zusatzpaket hat
Experimental-Status.

Standard-Einstellung: \var{OPT\_\-ISDN\_\-COMP}='no'

Hier im Einzelnen die erforderlichen Parameter für LZS-Kompression:

\begin{description}

\config{ISDN\_LZS\_DEBUG (EXPERIMENTAL)}{ISDN\_LZS\_DEBUG}{ISDNLZSDEBUG}

  Debug-Level-Einstellung:

  \begin{tabular}[h!]{ll}
    '0' & keine Debugging Information \\
    '1' & normale Debugging Information\\
    '2' & erweiterte Debugging Information\\
    '3' & schwere Debugging Information (inkl. Dumping der Datenpakete)\\
  \end{tabular}

  Standard-Einstellung: \var{ISDN\_\-LZS\_\-DEBUG}='1'
  
  Wer bei Problemen mit der Komprimierung noch mehr Debugmeldungen
  sehen möchte, setzt diese Variable auf '2'.


\config{ISDN\_LZS\_COMP (EXPERIMENTAL)}{ISDN\_LZS\_COMP}{ISDNLZSCOMP}
  
  Stärke der Kompression (nicht Dekompression!). Bitte erst einmal auf
  dem Wert '8' stehen lassen. Werte von 0 bis 9 sind möglich.
  
  Grössere Zahlen geben bessere Kompression, 9 erzeugt aber
  übermässige CPU-Last.
  
  Standard-Einstellung: \var{ISDN\_\-LZS\_\-COMP}='8'

\config{ISDN\_LZS\_TWEAK (EXPERIMENTAL)}{ISDN\_LZS\_TWEAK}{ISDNLZSTWEAK}

  Auch diese Variable erst einmal auf '7' stehen lassen.

  Standard-Einstellung: \var{ISDN\_\-LZS\_\-TWEAK}='7'
  
  Außer diesen 3 Werten muss noch die Variable
  \var{ISDN\_\-CIRC\_\-x\_\-FRAMECOMP} angepasst werden, s. nächstes Kapitel.

\end{description}




\subsection{ISDN-Circuits}

In der fli4l-Konfiguration können mehrere Verbindungen über ISDN
definiert werden. Davon sind maximal 2 Verbindungen auch zur gleichen
Zeit über eine ISDN-Karte möglich.

Die Definition solcher Verbindungen geschieht über sogenannte
Circuits.  Dabei wird pro Verbindung ein Circuit verwendet.

In der Beispiel-Datei config.txt sind zwei solcher Circuits definiert:

\begin{itemize}
\item Circuit 1: Dialout über Internet-By-Call-Provider Microsoft Network, Sync-PPP
  
\item Circuit 2: Dialin/Dialout zu einem ISDN-Router (das könnte beispielsweise 
  auch fli4l sein)
  über Raw-IP, z.B. als Zugang zum Firmen-Netz von irgendwo. Bei
  mir konkret ist das eine Linux-Box mit isdn4linux als ``Gegner''.

\end{itemize}

Soll der fli4l-Router lediglich als Internet-Gateway dienen, ist nur
ein Circuit notwendig. Ausnahme: Man will die
Least-Cost-Router-Features von fli4l nutzen. Dann sind sämtliche
erlaubten Circuits für verschiedene Zeitbereiche zu definieren, siehe
unten.

\begin{description}

\config{ISDN\_CIRC\_N}{ISDN\_CIRC\_N}{ISDNCIRCN}
  
  Gibt die Anzahl der verwendeten ISDN-Circuits an. Wird fli4l
  lediglich als ISDN-Anrufmonitor eingesetzt, ist einzustellen:

\begin{example}
\begin{verbatim}
        ISDN_CIRC_N='0'
\end{verbatim}
\end{example}
  
  Soll der fli4l-Router lediglich als einfaches ISDN-Gateway in das
  Internet verwendet werden, reicht ein Circuit. Ausnahme: LC-Routing,
  siehe unten.


\config{ISDN\_CIRC\_x\_NAME}{ISDN\_CIRC\_x\_NAME}{ISDNCIRCxNAME} 
  
  Hier sollte ein Name für den Circuit vergeben werden - max. 15
  Stellen lang. Dieser wird dann im imon-Client \texttt{imonc.exe} statt der
  angewählten Telefonnummer gezeigt. Erlaubte Zeichen sind die
  Buchstaben 'A' bis 'Z' (Klein- und Großschreibung), die Zahlen '0'
  bis '9' und der Bindestrich '-'., wie z.B.

\begin{example}
\begin{verbatim}
        ISDN_CIRC_x_NAME='msn'
\end{verbatim}
\end{example}

  Der Name kann außerdem im Paketfilter oder bei OpenVPN
  benutzt werden. Wenn  z.B. der Paketfilter einen ISDN Circuit
  regeln soll, muß ein 'circuit\_' dem Circuit Namen vorangesetzt werden.
  Heißt ein ISDN Circuit z.B. 'willi', so wird daraus folgendes im Paketfilter:

\begin{example}
\begin{verbatim}
PF_INPUT_3='if:circuit_willi:any prot:udp 192.168.200.226 192.168.200.254:53 ACCEPT'
\end{verbatim}
\end{example}

\config{ISDN\_CIRC\_x\_USEPEERDNS}{ISDN\_CIRC\_x\_USEPEERDNS}{ISDNCIRCxUSEPEERDNS}

  Hiermit wird festgelegt, ob die vom Internet-Provider bei der
  Einwahl übergebenen Nameserver für die Dauer der Onlineverbindung in
  die Konfigurationsdatei des lokalen Nameservers eingetragen
  werden sollen.  Sinnvoll ist die Nutzung dieser Option also nur bei
  Circuits für Internet-Provider.  Inzwischen unterstützen fast alle
  Provider diese Art der Übergabe.
  
  Nachdem die Nameserver-IP-Adressen übertragen wurden, werden die in
  der base.txt unter \emph{\var{DNS\_\-FORWARDERS}} eingetragenen Nameserver
  aus der Konfigurationsdatei des lokalen Nameservers entfernt und die vom
  Provider vergebenen IP-Adressen als Forwarder eingetragen. Danach wird
  der lokale Nameserver veranlaßt, seine Konfiguration neu einzulesen.
  Dabei gehen bis dahin aufgelöste Namen nicht aus dem Nameserver-Cache
  verloren.
  
  Diese Option bietet den Vorteil, immer mit den am nächsten liegenden
  Nameservern arbeiten zu können, sofern der Provider die korrekten
  IP-Adressen übermittelt - dadurch geht die Namensauflösung
  schneller.
  
  Im Falle eines Ausfalls eines DNS-Servers beim Provider werden in
  der Regel die übergebenen DNS-Server-Adressen sehr schnell vom
  Provider korrigiert.
  
  Trotz allem ist vor jeder ersten Einwahl die Angabe eines gültigen
  Nameservers in \emph{\var{DNS\_\-FORWARDERS}} der base.txt zwingend
  erforderlich, da sonst die erste Anfrage nicht korrekt aufgelöst
  werden kann. Außerdem wird beim Beenden der Verbindung die originale
  Konfiguration des lokalen Nameservers wieder hergestellt.
  
  Standard-Einstellung: \var{ISDN\_\-CIRC\_\-x\_\-USEPEERDNS}='yes'


\config{ISDN\_CIRC\_x\_TYPE}{ISDN\_CIRC\_x\_TYPE}{ISDNCIRCxTYPE}
  
  \var{ISDN\_\-CIRC\_\-x\_\-TYPE} gibt den Typ der x-ten IP-Verbindung an. Dabei
  sind folgende Werte möglich:

  \begin{tabular}[h!]{ll}
        'raw' &           RAW-IP\\
        'ppp' &           Sync-PPP\\
  \end{tabular}
  
  In den meisten Fällen wird PPP verwendet, jedoch ist Raw-IP etwas
  effizienter, da hier der PPP-Overhead entfällt. Eine
  Authentifizierung ist zwar bei Raw-IP nicht möglich, es kann jedoch
  über die Variable \var{ISDN\_\-CIRC\_\-x\_\-DIALIN} (s.u.) eine
  Zugangsbeschränkung auf ganz bestimmte ISDN-Nummern (Stichwort
  ``Clip'') eingestellt werden.  Wird \var{ISDN\_\-CIRC\_\-x\_\-TYPE}
  auf 'raw' gestellt wird analog zu den PPP up/down Scripten in
  /etc/ppp ein raw up/down Script ausgeführt.

\config{ISDN\_CIRC\_x\_BUNDLING}{ISDN\_CIRC\_x\_BUNDLING}{ISDNCIRCxBUNDLING}
  
  Für die ISDN-Kanalbündelung wird das verbreitete MPPP-Protokoll nach
  RFC 1717 verwendet. Damit gelten folgende Einschränkungen, die aber
  in der Praxis meist nicht relevant sind:
  \begin{itemize}
  \item Nur mit PPP-Verbindungen möglich, nicht bei Raw-Circuits
  \item Kanalbündelung nach neuerer RFC 1990 (MLP) ist nicht möglich
  \end{itemize}
  
  Der 2. Kanal kann entweder mit dem Client imonc manuell
  hinzugeschaltet werden oder über die Bandbreitenanpassung
  automatisch aktiviert werden, siehe auch die Beschreibung zu
  \var{ISDN\_\-CIRC\_\-x\_\-BANDWIDTH}.
  
  Standard-Einstellung: \var{ISDN\_\-CIRC\_\-x\_\-BUNDLING}='no'
  
  Vorsicht: bei Verwendung von Kanalbündelung zusammen mit Kompression
  kann es zu Problemen kommen, siehe auch die Beschreibung zu
  \var{ISDN\_\-CIRC\_\-x\_\-FRAMECOMP}.

\config{ISDN\_CIRC\_x\_BANDWIDTH}{ISDN\_CIRC\_x\_BANDWIDTH}{ISDNCIRCxBANDWIDTH}
  
Ist die ISDN-Kanalbündelung über \linebreak
\var{ISDN\_\-CIRC\_\-x\_\-BUNDLING}='yes' aktiviert, kann mit dieser Variablen
eine automatische Hinzuschaltung des 2. ISDN-Kanals eingestellt
werden.  Dabei sind 2 numerische Parameter anzugeben:
  \begin{enumerate}
  \item  Schwellenwert in Byte/Sekunde (S)
  \item  Zeitintervall in Sekunden (Z)
  \end{enumerate}
  
  Wird der Schwellenwert S für Z Sekunden überschritten, schaltet
  der Steuerprozeß imond den 2. Kanal automatisch hinzu. Wird der
  Schwellenwert S für Z Sekunden unterschritten, wird der 2.
  Kanal automatisch wieder deaktiviert. Die automatische
  Bandbreitenanpassung kann mit \var{ISDN\_\-CIRC\_\-1\_\-BANDWIDTH}=''
  abgeschaltet werden. Dann ist lediglich eine manuelle Kanalbündelung
  über den Client imonc möglich.
  
  Beispiele:
  \begin{itemize}
  \item \var{ISDN\_\-CIRC\_\-1\_\-BANDWIDTH}='6144 30'
    
    Überschreitet die Transferrate den Wert von 6 kibibyte/second für 30
    Sekunden, wird der 2. Kanal hinzugeschaltet.
    
  \item \var{ISDN\_\-CIRC\_\-1\_\-BANDWIDTH}='0 0'
    
    Der zweite ISDN-Kanal wird sofort, spätestens 10 Sekunden nach dem
    Verbindungsaufbau hinzugeschaltet und bleibt solange bestehen, bis
    die komplette Verbindung beendet wird.
    
  \item \var{ISDN\_\-CIRC\_\-1\_\-BANDWIDTH}=''
    
    Der zweite ISDN-Kanal kann nur manuell hinzugeschaltet werden,
    Voraussetzung ist jedoch weiterhin, dass
    \var{ISDN\_\-CIRC\_\-1\_\-BUNDLING}='yes' eingestellt ist.
    
  \item \var{ISDN\_\-CIRC\_\-1\_\-BANDWIDTH}='10000 30'
    
    Eigentlich sollte hiermit der 2. Kanal nach 30 Sekunden
    hinzugeschaltet werden, wenn während dieser Zeitspanne 10000
    B/s erreicht werden. Dazu wird es aber nicht kommen, da mit
    ISDN nicht mehr als 8 kB/s erreicht werden können.

  \end{itemize}
  
  Ist \var{ISDN\_\-CIRC\_\-x\_\-BUNDLING}='no' eingestellt, ist der Wert in der
  Variablen \linebreak \var{ISDN\_\-CIRC\_\-x\_\-BANDWIDTH} belanglos.
  
  Standard-Einstellung: \var{ISDN\_\-CIRC\_\-x\_\-BANDWIDTH}=''


\config{ISDN\_CIRC\_x\_LOCAL}{ISDN\_CIRC\_x\_LOCAL}{ISDNCIRCxLOCAL}
    
  In dieser Variablen wird die lokale IP-Adresse auf
  der ISDN-Seite hinterlegt.
  
  Bei dynamischer Adress\-zu\-wei\-sung sollte
  dieser Wert \textbf{leer} sein. Beim Verbindungsaufbau wird dann die
  IP-Adresse ausgehandelt. In den meisten Fällen vergeben Internet-Provider
  diese Adresse
  dynamisch. Soll jedoch eine fest vergebene IP-Adresse verwendet
  werden, ist diese hier einzutragen. Diese Variable ist optional
  und muss bei Bedarf in das Konfigfile eingetragen werden.

\config{ISDN\_CIRC\_x\_REMOTE}{ISDN\_CIRC\_x\_REMOTE}{ISDNCIRCxREMOTE}
    
  In dieser Variablen wird die entfernte IP-Adresse und die zugehörige
  Netzmaske auf der ISDN-Seite hinterlegt. Dazu muss die CIDR
  (Classles Inter-Domain Routing)
  Schreibweise benutzt werden. Details zu \jump{IPNETx}{CIDR} ist in
  der Dokumentation des Baseispaketes bei IP\_NET\_x zu finden.
  
  Bei dynamischer Adress\-zu\-wei\-sung sollte
  dieser Wert \textbf{leer} sein. Beim Verbindungsaufbau wird dann die
  IP-Adresse ausgehandelt. In den meisten Fällen vergeben Internet-Provider
  diese Adresse
  dynamisch. Soll jedoch eine fest vergebene IP-Adresse verwendet
  werden, ist diese hier einzutragen. Diese Variable ist optional
  und muss bei Bedarf in das Konfigfile eingetragen werden.

  Die angegebene Netzmaske wird bei der Configuration des Interfaces nach
  der Einwahl verwendet. Während dieser Configuration wird auch
  eine Route zum Host erzeugt, in den man sich einwählt. Da man
  diese Route in der Regel nicht braucht, ist es günstig, hier nur
  eine Route direkt zum Einwahlrechner zu erzeugen. Dazu setzt man
  die Netzmaske auf /32, indem man hier 32 als Anzahl
  der gesetzten Bits in der Netzmaske spezifiziert. Für Details siehe
  \jump{sec:route-details}{Kapitel: Technische Details zum Dialin}.


\configlabel{ISDN\_CIRC\_x\_MTU}{ISDNCIRCxMTU}
\config{ISDN\_CIRC\_x\_MTU ISDN\_CIRC\_x\_MRU}{ISDN\_CIRC\_x\_MRU}{ISDNCIRCxMRU}
  
  Mit diesen optionalen Variablen können die sog. \textbf{MTU} (maximum
  transmission unit) und die \textbf{MRU} (maximum receive unit) eingestellt
  werden. Optional bedeutet, die Variable muß nicht in der Konfigurationsdatei
  stehen, sie ist bei Bedarf durch den Benutzer einzufügen! \\ 
  Normal beträgt die MTU 1500 und die MRU 1524. Diese Einstellung sollte
  nur in Sonderfällen geändert werden!

\config{ISDN\_CIRC\_x\_CLAMP\_MSS}{ISDN\_CIRC\_x\_CLAMP\_MSS}{ISDNCIRCxCLAMPMSS}

Hier sollte man ein yes setzen, wenn man synchrones ppp verwendet
(\var{ISDN\_CIRC\_x\_TYPE}='ppp') und eines der folgenden Symptome
auftritt:
\begin{itemize}
\item der Webbrowser verbindet sich mit dem Webserver, aber es wird
  keine Seite angezeigt und kommt auch keine Fehlermeldung; es
  passiert einfach nichts mehr,
\item das Versenden kleiner \mbox{E-Mails} funktioniert, bei großen gibt es
  Probleme oder
\item ssh funktioniert, scp hängt nach dem initialen
  Verbindungsaufbau.
\end{itemize}

Provider, bei denen solche Probleme auftreten, sind z.B. Compuserve
und andere Mediaways basierte Zugänge.

  Standard-Einstellung: \var{ISDN\_\-CIRC\_\-x\_\-CLAMP\_MSS}='no'

\config{ISDN\_CIRC\_x\_HEADERCOMP}{ISDN\_CIRC\_x\_HEADERCOMP}{ISDNCIRCxHEADERCOMP}
  
  Mit \var{ISDN\_\-CIRC\_\-x\_\-HEADERCOMP}='yes' kann die
  Van-Jacobson-Komprimierung oder Headerkomprimierung eingestellt
  werden. Nicht alle Provider unterstützen das. Sollte es daher bei
  eingeschalteter Komprimierung zu Einwahlproblemen kommen, sollte
  \var{ISDN\_\-CIRC\_\-x\_\-HEADERCOMP}='no' eingestellt werden.
  
  Standard-Einstellung: \var{ISDN\_\-CIRC\_\-x\_\-HEADERCOMP}='yes'


\config{ISDN\_CIRC\_x\_FRAMECOMP (EXPERIMENTAL)}{ISDN\_CIRC\_x\_FRAMECOMP}{ISDNCIRCxFRAMECOMP}
  
  Dieser Parameter wird nur berücksichtigt, wenn \var{OPT\_\-ISDN\_\-COMP}='yes'
  eingestellt wird. Er regelt die Frame-Kom\-pri\-mierung.
  
  Folgende Werte sind möglich:

  \begin{tabular}[h!]{ll}
        'no' &                    Keine Frame-Komprimierung \\
        'default' &               LZS according RFC1974(std) and
        BSDCOMP 12 \\
        'all' &                   Negotiate lzs and bsdcomp \\
        'lzs' &                   Negotiate lzs only \\
        'lzsstd' &                LZS according RFC1974 Standard Mode
                                (``Sequential Mode'') \\
        'lzsext' &                LZS according RFC1974 Extended Mode \\
        'bsdcomp' &               Negotiate bsdcomp only \\
        'lzsstd-mh' &             LZS Multihistory according RFC1974
                                  Standard Mode (``Sequential Mode``)
  \end{tabular}
  
  Welcher Wert für den jeweiligen Provider verwendet werden kann, muss
  ausprobiert werden. So weit bekannt geht bei T-Online nur 'lzsext'.
  Bei den meisten anderen Providern sollte man mit 'default'
  auskommen.
  
  Vorsicht: Bei verwendung von Kanalbündelung in zusammenhang mit
  'lzsext' kann es zu Problemen kommen. Diese Probleme sind, so weit
  bekannt, Einwahlserverspezifisch und damit meistens
  Providerspezifisch. Es können aber bei einem Provider auch
  verschiedene Typen Einwahlserver im Einsatz sein, es kann in dem
  Fall zu Unterschieden zwischen Einwahlknoten kommen.

  'lzsstd-mh' ist für Router-zu-Routerbetrieb (r2r) gedacht. Das Verfahren
  wird von Providern nicht eingesetzt aber bringt bei Verwendung zwischen
  zwei fli4l-Router erhebliche Verbesserung bei gleichzeitigen Übertragung
  von mehreren Dateien. Die Headerkompression ist dazu erforderlich und wird
  deshalb automatisch eingeschaltet.

\config {ISDN\_CIRC\_x\_REMOTENAME}{ISDN\_CIRC\_x\_REMOTENAME}{ISDNCIRCxREMOTENAME}
  
  Diese Variable ist normalerweise lediglich bei der Konfiguration von
  fli4l als Einwahlrouter relevant. Hier kann ein Name des
  Remote-Hosts eingetragen werden, muß aber nicht.
  
  Standard-Einstellung: \var{ISDN\_\-CIRC\_\-x\_\-REMOTENAME}=''

\configlabel{ISDN\_CIRC\_x\_USER}{ISDNCIRCxUSER}
\config {ISDN\_CIRC\_x\_PASS}{ISDN\_CIRC\_x\_PASS}{ISDNCIRCxPASS}
  
  Hier sind die Provider-Daten einzutragen. Im Beispiel oben handelt
  es sich um die Daten des Providers Microsoft Network.
  
  \var{ISDN\_\-CIRC\_\-x\_\-USER} enthält die Benutzerkennung, \var{ISDN\_\-CIRC\_\-x\_\-PASS}
  das Password.
  
  WICHTIG:  Für einen T-Online-Zugang ist folgendes zu beachten:
  
  Der Username AAAAAAAAAAAATTTTTT\#MMMM setzt sich aus der
  zwölfstelligen Anschlußkennung, der T-Online-Nummer und der
  Mitbenutzernummer zusammen.  Hinter der T-Online-Nummer muß ein '\#'
  angegeben werden, wenn die Länge der T-Online-Nummer kürzer als 12
  Zeichen ist.
  
  Sollte dies in Einzelfällen nicht zum Erfolg führen (offenbar
  abhängig von der Vermittlungsstelle), muß zusätzlich zwischen der
  Anschlußkennung und der T-Online-Nummer ein weiteres '\#'-Zeichen
  eingefügt werden.
  
  Ansonsten (T-Online-Nr ist 12stellig) sind keine '\#'-Zeichen
  anzugeben.
  
  Beispiel:  \var{ISDN\_\-CIRC\_\-1\_\-USER}='123456\#123'

  
  Bei Raw-IP-Circuits haben diese Variablen keine Bedeutung.

\config {ISDN\_CIRC\_x\_ROUTE\_N}{ISDN\_CIRC\_x\_ROUTE\_N}{ISDNCIRCxROUTEN}

  Die Anzahl der Routen die dieser ISDN Circuit bedient. Wenn dieser
  Circuit eine Default-Route definiert muss der Eintrag auf '1'
  gesetzt werden.

\config {ISDN\_CIRC\_x\_ROUTE\_X}{ISDN\_CIRC\_x\_ROUTE\_X}{ISDNCIRCxROUTEx}
  
  Die Route oder die Routen für diesen Circuit. Für einen
  Internet-Zugang sollte man hier im ersten Eintrag '0.0.0.0/0' (default
  route) angeben. Das Format ist immer 'network/netmaskbits'. Eine
  Hostroute würde z.B. so aussehen: '192.168.199.1/32'. Bei Einwahl in
  den Firmen- oder Uni-Router ist lediglich das oder die Netze
  anzugeben, die man dort erreichen will, z.B.

\begin{example}
\begin{verbatim}
        ISDN_CIRC_%_ROUTE_N='2'
        ISDN_CIRC_%_ROUTE_1='192.168.8.0/24'
        ISDN_CIRC_%_ROUTE_2='192.168.9.0/24'
\end{verbatim}
\end{example}
  
  Bei mehreren  Netzen müssen diese jeweils in  einen eigenen Eintrag,
  also  für  jede  Route  muss eine  ISDN\_CIRC\_x\_ROUTE\_y=''  Zeile
  angelegt werden.
  
  Möchte man die LC-Routing-Features von fli4l nutzen, kann *mehreren*
  Circuits eine Default-Route zugewiesen werden. Welcher Circuit dann
  tatsächlich verwendet wird, wird über \var{ISDN\_\-CIRC\_\-x\_\-TIMES}
  eingestellt, siehe unten.

\config {ISDN\_CIRC\_x\_DIALOUT}{ISDN\_CIRC\_x\_DIALOUT}{ISDNCIRCxDIALOUT}

  \var{ISDN\_\-CIRC\_\-x\_\-DIALOUT} gibt die zu wählende Telefonnummer an. Es ist
  möglich, hier mehrere Nummern anzugeben (falls eine besetzt ist, wird
  die nächste angewählt) - die Trennung erfolgt dabei durch Leerzeichen.
  Laut Berichten in der Newsgroup dürfen maximal fünf Nummern angegeben
  werden.

\config {ISDN\_CIRC\_x\_DIALIN}{ISDN\_CIRC\_x\_DIALIN}{ISDNCIRCxDIALIN}

  Soll der Circuit (auch) zum Einwählen genutzt werden, wird in
  \var{ISDN\_\-CIRC\_\-x\_\-DIALIN} die Rufnummer des Anrufenden eingesetzt - und
  zwar mit Vorwahl, aber *ohne* die erste 0. Bei Anschlüssen hinter
  Telefonanlagen kann dies anders sein, eventuell müssen dann eine
  oder sogar zwei führende Nullen angegeben werden.
  
  Soll es mehreren Teilnehmern ermöglicht werden, sich über diesen
  Circuit einzuwählen, können diese Nummern durch Leerzeichen getrennt
  angegeben werden. Besser ist es aber, jedem Gegner einen extra
  Circuit zuzuweisen. Sonst könnte es bei Einwahl von zwei Gegnern zu
  gleicher Zeit (ist über die 2 ISDN-Kanäle durchaus möglich) auf
  demselben Circuit zu Kollisionen bzgl. IP-Adressen kommen.
  
  Falls der Anrufende keine Rufnummer überträgt, hier eine '0' setzen.
  Aber Vorsicht: damit wird jedem eine Anwahl gestattet, der keine
  Rufnummer überträgt!
  
  Möchte man eine Einwahl unabhängig von der MSN
  des Anrufenden realisieren, ist als Wert '*' anzugeben.

  In den beiden letzten Fällen ist ein Authentifizierungsverfahren 
  (siehe \var{ISDN\_CIRC\_x\_AUTH}) unumgänglich.

\config {ISDN\_CIRC\_x\_CALLBACK}{ISDN\_CIRC\_x\_CALLBACK}{ISDNCIRCxCALLBACK}
  
  Einstellung Callbackverfahren, mögliche Werte:

  \begin{tabular}[h!]{ll}
        'in' &     fli4l wird angerufen und ruft zurück \\
        'out' &    fli4l ruft an, hängt jedoch wieder ein und wartet auf
        Rückruf \\
        'off' &    kein Callback\\
        'cbcp' &   CallBack Control Protocol\\
        'cbcp0' &  CallBack Control Protocol 0\\
        'cbcp3' &  CallBack Control Protocol 3\\
        'cbcp6' &  CallBack Control Protocol 6\\

  \end{tabular}

  Bei den CallBack Control Protokolle (auch 'Microsoft CallBack' genannt)
  ist cbcp6 das meist übliche Protokoll.

  Standard-Wert: 'off'

\config {ISDN\_CIRC\_x\_CBNUMBER}{ISDN\_CIRC\_x\_CBNUMBER}{ISDNCIRCxCBNUMBER}

  Hier kann man für die Protokolle cbcp, cbcp3 und cbcp6 eine Rückrufnummer
  einsetzen (bei cbcp3 Pflicht).

\config {ISDN\_CIRC\_x\_CBDELAY}{ISDN\_CIRC\_x\_CBDELAY}{ISDNCIRCxCBDELAY}
  
  Diese Variable gibt eine Verzögerung in Sekunden an, die bei
  Callback gewartet werden soll. Je nachdem, in welcher Richtung der
  Callback erfolgen soll, hat diese Variable eine etwas andere
  Bedeutung:
  
  \begin{itemize}
  \item  \var{ISDN\_\-CIRC\_\-x\_\-CALLBACK}='in':
    
    Wird fli4l angerufen und soll zurückrufen, ist
    \var{ISDN\_\-CIRC\_\-x\_\-CBDELAY} die Wartezeit, bis der Rückruf erfolgen
    soll. Ein guter Erfahrungswert ist \linebreak \var{ISDN\_\-CIRC\_\-x\_\-CBDELAY}='3'. Je
    nach ``Gegner'' kann aber auch ein geringerer Wert funktionieren,
    welches dann den Verbindungsaufbau beschleunigen kann.

  \item \var{ISDN\_\-CIRC\_\-x\_\-CALLBACK}='out':
    
    In diesem Fall gibt \var{ISDN\_\-CIRC\_\-x\_\-CBDELAY} die Zeit an, wie lange
    das ``Anklingeln des Gegners'' erfolgen soll, bis auf den Rückruf
    gewartet wird. Auch hier ist \var{ISDN\_\-CIRC\_\-x\_\-CBDELAY}='3' ein guter
    Erfahrungswert. Was mir dazu aufgefallen ist: Bei Fernverbindungen
    muß man bis zu 3 Sekunden ``klingeln'' lassen, bevor der andere
    Router überhaupt den Anruf sieht. Bei Ortsverbindungen kann meist
    dieser Wert kleiner gewählt werden. Im Zweifel: Ausprobieren!
  \end{itemize}

  
  Ist die Variable \var{ISDN\_\-CIRC\_\-x\_\-CALLBACK}='off' eingestellt, wird
  \linebreak \var{ISDN\_\-CIRC\_\-x\_\-CBDELAY} ignoriert.
  Auch beim CallBack Control Protocol hat diese Variable keine Bedeutung.


\config {ISDN\_CIRC\_x\_EAZ}{ISDN\_CIRC\_x\_EAZ}{ISDNCIRCxEAZ}

  Im Beispiel ist die MSN (hier EAZ genannt) auf 81330 gesetzt. Hier
  sollte die eigene MSN *ohne* Vorwahl eingetragen werden.
  
  Bei Anschlüssen hinter einer Telefonanlage mit Anlagenanschluss ist
  meistens nur die Durchwahl anzugeben. Ich habe aber auch schon
  gelesen, dass eine '0' weiterhelfen kann, wenn es Probleme mit der
  verwendeten Telefonanlage geben sollte.
  
  Dieses Feld kann auch leer sein. Dies
  soll bei besonders hartnäckigen TK-Anlagen helfen. Um Feedback wird
  an dieser Stelle gebeten.

\config {ISDN\_CIRC\_x\_SLAVE\_EAZ}{ISDN\_CIRC\_x\_SLAVE\_EAZ}{ISDNCIRCxSLAVEEAZ}
  
  Ist der fli4l-Router am internen S0-Bus einer Telefonanlage
  angeschlossen und möchte man Kanalbündelung verwenden, ist bei
  manchen Telefonanlagen die Angabe einer 2. Durchwahlnummer für den
  Slave-Kanal einzutragen.
  
  Im Normalfall kann diese Variable jedoch leer bleiben.

\config {ISDN\_CIRC\_x\_DEBUG}{ISDN\_CIRC\_x\_DEBUG}{ISDNCIRCxDEBUG}
  
  Soll ipppd zusätzliche Debug-Informationen ausgeben, muss man
  \var{ISDN\_\-CIRC\_\-x\_\-DEBUG} auf 'yes' setzen. In diesem Fall schreibt ipppd
  zusätzlichen Informationen über die syslog-Schnittstelle.
  
  WICHTIG: Damit diese auch über syslogd ausgegeben werden, muss die
  Variable \var{OPT\_\-SYSLOGD} 
  (Siehe \jump{OPTSYSLOGD}{OPT\_SYSLOGD - Programm zum Protokollieren von Systemfehlermeldungen})
  ebenso auf 'yes' gesetzt sein.\\
  Weil manche Meldungen über klog ausgegeben werden sollte man beim Debuggen
  von ISDN auch \var{OPT\_\-KLOGD} (Siehe \jump{OPTKLOGD}{OPT\_KLOGD - Kernel-Message-Logger})
  auf 'yes' setzen.
  
  Bei Raw-IP-Circuits hat \var{ISDN\_\-CIRC\_\-x\_\-DEBUG} keine Bedeutung.

\config {ISDN\_CIRC\_x\_AUTH}{ISDN\_CIRC\_x\_AUTH}{ISDNCIRCxAUTH}
  
  Wird dieser Circuit auch zum Einwählen verwendet und soll eine
  Authentifizierung über PAP oder CHAP vom einwählenden ``Gegner''
  gefordert werden, ist \var{ISDN\_\-CIRC\_\-x\_AUTH} auf 'pap' oder 'chap' zu
  setzen - und *nur* dann. Anderenfalls immer leer lassen!
  
  Grund: Ein angewählter Internet-Provider wird es immer ablehnen,
  sich selbst auszuweisen! Ausnahmen bestätigen die Regel, wie ich
  erst kürzlich in der i4l-Mailingliste las ...
  
  Als Benutzername und Passwort werden die Einträge von
  \var{ISDN\_\-CIRC\_\-x\_USER} und \var{ISDN\_\-CIRC\_\-x\_PASS}
  benutzt.

  Bei Raw-IP-Circuits hat diese Variable keine Bedeutung.

\config {ISDN\_CIRC\_x\_HUP\_TIMEOUT}{ISDN\_CIRC\_x\_HUP\_TIMEOUT}{ISDNCIRCxHUPTIMEOUT}
  
  Mit der Variablen \var{ISDN\_\-CIRC\_\-x\_\-HUP\_\-TIMEOUT} wird die Zeit
  gesteuert, nach der der fli4l-Rechner die Verbindung zum Provider
  beenden soll, wenn nichts mehr über die Leitung geht. Im Beispiel
  wird die Verbindung nach 40 Sekunden Idle-Time abgebaut, um Geld zu
  sparen. Bei einem erneuten Zugriff in's Netz wird die Verbindung in
  Sekundenschnelle wieder aufgebaut. Sinnvoll bei Providern, die
  sekundengenau abrechnen!
  
  Man sollte zumindest in der Testphase das automatische
  Wählen/Einhängen des fli4l-Routers beobachten (entweder auf der
  Console oder im imon-Client für Windows). Nicht, dass durch eine
  fehlerhafte Konfiguration der ISDN-Anschluss zur Standleitung wird.
  
  Wird der Wert auf '0' gestellt, wird keine Idle-Zeit mehr
  berücksichtigt, d.h. fli4l hängt von sich aus die Leitung nicht mehr
  ein. Bitte mit Vorsicht anwenden.
  
\config {ISDN\_CIRC\_x\_CHARGEINT}{ISDN\_CIRC\_x\_CHARGEINT}{ISDNCIRCxCHARGEINT}
  
  Charge-Interval: Hier ist der Zeittakt in Sekunden anzugeben. Dieser
  wird dann für die Kosten-Berechnung verwendet.
  
  Die meisten Provider rechnen minutengenau ab. In diesem Fall ist der
  Wert '60' richtig. Compuserve verwendet einen 3-Minuten-Takt (Stand
  Juni 2000), also \linebreak \var{ISDN\_\-CIRC\_\-x\_\-CHARGEINT}='180'. Bei Providern mit
  sekundengenauer Abrechnung (z.B. Planet-Interkom) setzt man besser
  \var{ISDN\_\-CIRC\_\-x\_\-CHARGEINT} auf '1'.
  
  Erweiterung für \var{ISDN\_\-CIRC\_\-x\_\-CHARGEINT} $>$= 60 Sekunden:
  
  Wurde \var{ISDN\_\-CIRC\_\-x\_\-HUP\_\-TIMEOUT} Sekunden lang kein Traffic
  bemerkt, wird ca. 2 Sekunden vor Ablauf des Taktes eingehängt. Die
  vom Provider berechnete Zeit wird also fast komplett ausgenutzt. Ein
  wirklich tolles Feature von isdn4linux!
  
  Bei sekundengenau abgerechneten Verbindungen hat das natürlich
  keinen Sinn - daher gilt diese Regelung erst ab Zeittakten von 60
  Sekunden.

\config {ISDN\_CIRC\_x\_TIMES}{ISDN\_CIRC\_x\_TIMES}{ISDNCIRCxTIMES}
  
  Hier werden die Zeiten angegeben, wann dieser Circuit aktiviert
  werden soll und wann er wieviel kostet. Dadurch wird es möglich, zu
  verschiedenen Zeiten verschiedene Circuits mit Default-Routen zu
  verwenden (Least-Cost-Routing). Dabei kontrolliert der Daemon imond
  die Routen-Zuweisung.
  
  Aufbau der Variablen:

\begin{example}
\begin{verbatim}
        ISDN_CIRC_x_TIMES='times-1-info [times-2-info] ...'
\end{verbatim}
\end{example}

  
  Jedes Feld times-?-info besteht aus 4 Unterfeldern - durch
  Doppelpunkt (':') getrennt.
  \begin{enumerate}
  \item Feld: W1-W2
    
    Wochentag-Zeitraum, z.B. Mo-Fr oder Sa-Su usw. Schreibweise in
    Englisch oder deutsch möglich. Soll ein einzelner Wochentag
    eingetragen werden, ist zu W1-W1 schreiben, also z.B. Su-Su.
    
  \item Feld: hh-hh
    
    Stunden-Bereich, z.B. 09-18 oder auch 18-09. 18-09 ist
    gleichbedeutend mit 18-24 plus 00-09. 00-24 meint den ganzen Tag.

  \item  Feld: Charge
    
    Hier werden in Euro-Werten die Kosten pro Minute angegeben, z.B.
    0.032 für 3.2 Cent pro Minute. Diese werden unter Berücksichtigung
    der Taktzeit umgerechnet für die tatsächlich anfallenden Kosten,
    welche dann im imon-Client angezeigt werden.

    
  \item  Feld: LC-Default-Route
    
    Der Inhalt kann Y oder N sein. Dabei bedeutet:

    \begin{itemize}
    \item Y: Der angegebene Zeitbereich wird beim LC-Routing als
      Default-Route verwendet. Wichtig: In diesem Fall muss auch
      \var{ISDN\_\-CIRC\_\-x\_\-ROUTE}='0.0.0.0/0' sein!
        
    \item N: Der angegebene Zeitbereich dient nur zum Berechnen von
      Kosten, er wird beim automatischen LC-Routing jedoch nicht
      weiter verwendet.
    \end{itemize}

  \end{enumerate}

    Beispiel:

\begin{small}
\begin{example}
\begin{verbatim}
    ISDN_CIRC_1_TIMES='Mo-Fr:09-18:0.049:N Mo-Fr:18-09:0.044:Y Sa-Su:00-24:0.044:Y'
    ISDN_CIRC_2_TIMES='Mo-Fr:09-18:0.019:Y Mo-Fr:18-09:0.044:N Sa-Su:00-24:0.044:N'
\end{verbatim}
\end{example}
\end{small}
    
    Die Bedeutung ist dabei wie folgt: Circuit 1 (Provider
    Planet-Interkom) soll abends an Werktagen und komplett am
    Wochenende verwendet werden, jedoch tagsüber an Werktagen der
    Circuit 2 (Provider Compuserve).

    \begin{description}
    \item \wichtig{Die bei \var{ISDN\_CIRC\_x\_TIMES} angegebenen Zeiten
     muessen die ganze Woche abdecken.
     Ist das nicht der Fall, kann keine gültige Konfiguration
     erzeugt werden.}
      
     \emph{Wenn die Zeitbereiche aller LC-Default-Route-Circuits (``Y'')
     zusammengenommen nicht die komplette Woche beinhalten, gibt's zu
     diesen Lückenzeiten keine Default-Route. Damit ist dann Surfen
     im Internet zu diesen Zeiten ausgeschlossen!}

        
    \item Beispiel:
\begin{example}
\begin{verbatim}
    ISDN_CIRC_1_TIMES='Sa-Su:00-24:0.044:Y Mo-Fr:09-18:0.049:N Mo-Fr:18-09:0.044:N'
    ISDN_CIRC_2_TIMES='Sa-Su:00-24:0.044:N Mo-Fr:09-18:0.019:Y Mo-Fr:18-09:0.044:N'
\end{verbatim}
\end{example}
      
      Hier wurde für die Werktage von 18-09 Uhr ``N'' gesetzt. Zu diesen Zeiten
      gibt es keine Route in's Internet: Surfen verboten!

      
    \item Noch ein ganz einfaches Beispiel:

\begin{example}
\begin{verbatim}
      ISDN_CIRC_1_TIMES='Mo-Su:00-24:0.0:Y'        
\end{verbatim}
\end{example}
      
      für diejenigen, die eine Flatrate nutzen.


      
    \item Und noch eine letzte Bemerkung zum LC-Routing:
      
      Deutsche Feiertage werden wie Sonntage behandelt.
    \end{description}

\end{description}

\subsection {OPT\_TELMOND - telmond-Konfiguration}
\configlabel{OPT\_TELMOND}{OPTTELMOND}

Mit \var{OPT\_\-TELMOND} kann man einstellen, ob der telmond-Server aktiviert
werden soll. Dieser horcht auf eingehende Telefonanrufe und teilt über
TCP-Port 5001 die anrufende und die angerufene Telefonnummer mit.
Diese Information kann vom Windows- und Unix/Linux-Client imonc
abgefragt und angezeigt werden (s.a. Kapitel
``Client-/Server-Schnittstelle imond'').

Zwingende Voraussetzung ist hierfür die Installation einer ISDN-Karte
und die Konfiguration von \var{OPT\_\-ISDN} und den dazugehörenden
Konfigurations-Variablen.

Im laufenden Betrieb kann die korrekte Funktion von telmond unter
Linux/Unix/Windows überprüft werden mit:

\begin{example}
\begin{verbatim}
        telnet fli4l 5001        
\end{verbatim}
\end{example}


Dann sollte der letzte Anruf gezeigt und anschließend die
telnet-Verbindung sofort wieder geschlossen werden.

Der Port 5001 ist nur vom LAN aus erreichbar. Standardmäßig wird ein
Zugriff von außen über die Firewall-Konfiguration abgeblockt. Möchte
man jedoch auch den Zugriff innerhalb des LANs anders regeln, kann
dies über die weitere telmond-Konfigurationsvariablen eingestellt
werden, siehe unten.

Standard-Einstellung: \var{START\_\-TELMOND}='yes'

\begin{description}

\config {TELMOND\_PORT}{TELMOND\_PORT}{TELMONDPORT}
  
  TCP/IP-Port, auf dem telmond auf Verbindungen horcht.  Der
  Standardwert '5001' sollte nur in Ausnahmefällen geändert werden.


\config {TELMOND\_LOG}{TELMOND\_LOG}{TELMONDLOG}
  
  Bei \var{TELMOND\_\-LOG}='yes' werden sämtliche einkommenden Telefonanrufe
  in der Datei /var/log/telmond.log gespeichert. Der Inhalt der Datei
  kann mit dem imond-Client imonc unter Unix/Linux und Windows
  abgefragt werden.
  
  Abweichende Pfade bzw. nach Clients aufgeteilte Log-Dateien können
  weiter unten konfiguriert werden.
  
  Standard-Einstellung: \var{TELMOND\_\-LOG}='no'

\config {TELMOND\_LOGDIR}{TELMOND\_LOGDIR}{TELMONDLOGDIR}
  
  Ist das Protokollieren eingeschaltet, kann über \var{TELMOND\_\-LOGDIR} ein
  alternatives Verzeichnis statt /var/log angegeben werden, z.B.
  '/boot'. Dann wird die LOG-Datei telmond.log auf dem Bootmedium
  angelegt. Dazu muß dann das Bootmedium auch Read/Write ``gemounted''
  sein. Wird hier 'auto' angegeben, befindet sich das Logfile je nach Konfiguration
  unter /boot/persistent/isdn oder an einem anderen durch \var{FLI4L\_UUID} bestimmten
  Pfad. Wenn /boot nicht Read/Write gemountet ist, wird das File in /var/run angelegt.

\config {TELMOND\_MSN\_N}{TELMOND\_MSN\_N}{TELMONDMSNN}
  
  Sollen bestimmte Anrufe nur auf einigen PC-Clients im imonc sichtbar
  werden, kann ein Filter eingestellt werden, mit dem Anrufe auf
  spezielle MSNs nur für diese PC-Clients protokolliert werden.
  
  Ist so etwas notwendig, z.B. in einer WG, wird die Variable
  \var{TELMOND\_\-MSN\_\-N} auf die Anzahl der MSN-Filter eingestellt.
  
  Standard-Einstellung: \var{TELMOND\_\-MSN\_\-N}='0'

\config {TELMOND\_MSN\_x}{TELMOND\_MSN\_x}{TELMONDMSNx}
  
  Für jeden MSN-Filter ist eine Liste von IP-Adressen anzugeben, für
  welche die Anrufe auf eingetragene MSN sichtbar werden sollen.
  
  Die Variable \var{TELMOND\_\-MSN\_\-N} bestimmt die Anzahl solcher
  Konfigurationen, siehe oben.
  
  Der Aufbau der Variblen ist:
\begin{example}
\begin{verbatim}
        TELMOND_MSN_x='MSN IP-ADDR-1 IP-ADDR-2 ...'
\end{verbatim}
\end{example}

  
  Ein einfaches Beispiel:

\begin{example}
\begin{verbatim}
        TELMOND_MSN_1='123456789 192.168.6.2'            
\end{verbatim}
\end{example}

  
  Soll ein Anruf auf eine bestimmte MSN auf mehreren Rechnern sichtbar
  werden, z.B. Fax, sind die IP-Adressen der Rechner hintereinander
  anzugeben, z.B.

\begin{example}
\begin{verbatim}
        TELMOND_MSN_1='123456789 192.168.6.2 192.168.6.3'            
\end{verbatim}
\end{example}

\config {TELMOND\_CMD\_N}{TELMOND\_CMD\_N}{TELMONDCMDN}
  
  Sobald ein Telefonanruf (Voice) auf einer bestimmten MSN
  hereinkommt, können optional bestimmte Kommandos auf dem
  fli4l-Router ausgeführt werden. Mit \var{TELMOND\_\-CMD\_\-N} gibt man die
  Anzahl der konfigurierten Kommandos an.


\config {TELMOND\_CMD\_x}{TELMOND\_CMD\_x}{TELMONDCMDx}
  
  Mit \var{TELMOND\_\-CMD\_\-1} bis \var{TELMOND\_\-CMD\_\-n} können Kommandos angegeben
  werden, welche ausgeführt werden, wenn ein Telefonanruf
  eintrifft.
  
  Die Variable \var{TELMOND\_\-CMD\_\-N} bestimmt die Anzahl solcher Kommandos,
  siehe oben.
  
  Die Variable hat folgenden Aufbau:

\begin{example}
\begin{verbatim}
        MSN CALLER-NUMBER  COMMAND ...
\end{verbatim}
\end{example}
  
  Dabei ist die MSN ohne Vorwahl einzutragen. Als CALLER-NUMBER gibt
  man die komplette Telefonnummer - also mit Vorwahl - an. Schreibt
  man als Wert für CALLER-NUMBER ein einfaches Sternchen (*), wird von
  telmond die Telefonnummer des Anrufers nicht ausgewertet.
  
  Hier ein Beispiel:

\begin{example}
\begin{verbatim}
        TELMOND_CMD_1='1234567 0987654321 sleep 5; imonc dial'
        TELMOND_CMD_2='1234568 * switch-on-coffee-machine'
\end{verbatim}
\end{example}
  
  Im ersten Fall wird die Kommandofolge ``sleep 5; imonc dial''
  durchgeführt, wenn der Anrufer mit der Telefonnummer 0987654321 die
  MSN 1234567 anruft. Tat\-säch\-lich werden hier 2 Kommandos ausgeführt.
  Zunächst wird 5 Sekunden gewartet, damit der ISDN-Kanal wieder frei
  wird, auf dem der Anruf hereinkam. Anschließend wird der
  fli4l-Client imonc mit dem Argument ``dial'' gestartet. imonc gibt
  dieses Kommando 1:1 an den Server imond weiter, welcher dann auf dem
  Default-Circuit eine Netzverbindung herstellt, z.B. ins Internet.
  Welche Kommandos das imonc-Client-Programm an den Server imond
  weitergeben kann, ist im Kapitel ``Client-/Server-Schnittstelle imond''
  erklärt. Damit diese  Einstellung funktioniert, muss \var{OPT\_IMONC}
  aus dem Paket ``tools'' installiert sein.
  
  Das zweite Kommando ``switch-on-coffee-machine'' wird ausgeführt,
  wenn ein Anruf auf der MSN 1234568 hereinkommt, unabhängig, woher
  der Anruf kam. Na\-tür\-lich gibt es das Kommando
  ``switch-on-coffee-machine'' (noch) nicht für fli4l!
  
  Beim Aufruf der Kommandos können folgende Platzhalter verwendet
  werden:

  \begin{tabular}[h!]{cll}
  \hline
            \%d  &    date   &     Datum \\
            \%t  &    time   &     Uhrzeit \\
            \%p  &    phone  &     Telefonnummer des Anrufers \\
            \%m  &    msn     &    Eigene MSN \\
            \%\%  &    percent &    das Prozentzeichen selbst\\
  \end{tabular}
  
  Diese Daten können dann von den aufgerufenen Programmen weiter
  verwendet werden, z.B. zum Verschicken per \mbox{E-Mail}.

\config {TELMOND\_CAPI\_CTRL\_N}{TELMOND\_CAPI\_CTRL\_N}{TELMONDCAPICTRLN}

  Wenn Sie einen CAPI-fähigen ISDN-Adapter oder eine Remote-CAPI (Typ 160 oder
  161) verwenden, kann es sein, dass Sie die CAPI-Controller, an denen telmond
  auf Anrufe horcht, genauer konfigurieren wollen. Beispielsweise bietet die
  Fritz!Box Zugriff auf teilweise bis zu fünf verschiedene Controller an, von
  denen manche sich nicht unterscheiden (siehe hierzu die Informationen unter
  \altlink{http://www.wehavemorefun.de/fritzbox/CAPI-over-TCP\#Virtuelle_Controller}).
  Um die zu verwendenden Controller einzuschränken, können Sie hier die Anzahl
  der zu nutzenden Controller angeben. In der nachfolgenden Array-Variable
  \var{TELMOND\_CAPI\_CTRL\_\%} können Sie dann angeben, welche Controller
  genutzt werden sollen.

  Wenn Sie diese Variable nicht nutzen, horcht telmond auf \emph{allen}
  verfügbaren CAPI-Controllern.

\config {TELMOND\_CAPI\_CTRL\_x}{TELMOND\_CAPI\_CTRL\_x}{TELMONDCAPICTRLx}

  Wenn sie \var{TELMOND\_CAPI\_CTRL\_N} ungleich Null gesetzt haben, müssen Sie
  in den Einträgen dieses Arrays die Indizes der CAPI-Controller angeben, an
  denen telmond auf eingehende Anrufe horchen soll.

  Beispiel für die Remote-CAPI einer Fritz!Box mit ``echtem'' ISDN-Anschluss:

\begin{example}
\begin{verbatim}
        TELMOND_CAPI_CTRL_N='2'
        TELMOND_CAPI_CTRL_1='1' # horche auf eingehende ISDN-Anrufe
        TELMOND_CAPI_CTRL_2='3' # horche auf Anrufe auf dem internen S0-Bus
\end{verbatim}
\end{example}

  Beispiel für die Remote-CAPI einer Fritz!Box mit analogem Anschluss und
  SIP-Weiterleitung:

\begin{example}
\begin{verbatim}
        TELMOND_CAPI_CTRL_N='2'
        TELMOND_CAPI_CTRL_1='4' # horche auf eingehende analoge Anrufe
        TELMOND_CAPI_CTRL_2='5' # horche auf eingehende SIP-Anrufe
\end{verbatim}
\end{example}

\end{description}

\subsection{OPT\_RCAPID - Remote CAPI Dämon}
\configlabel{OPT\_RCAPID}{OPTRCAPID}

Dieses OPT konfiguriert auf dem fli4l-Router das Programm rcapid, das Zugriff
auf die ISDN-CAPI-Schnittstelle des Routers übers Netzwerk anbietet. Geeignete
Anwendungen können somit die ISDN-Karte des Routers übers Netzwerk so nutzen,
als ob sie lokal zur Verfügung stünde. Die Funktionalität ähnelt somit dem
Paket ``mtgcapri''. Der Unterschied besteht jedoch darin, dass ``mtgcapri'' nur
Windows-Systeme als Klienten unterstützt, während die Netzwerk-Schnittstelle
des rcapid nach Kenntnis des Autors zur Zet nur Linux-Systeme nutzen können.
Insofern ergänzen sich in gemischten Umgebungen mit Windows- und Linux-Systemen
beide Pakete ideal.

\subsubsection{Konfiguration des Routers}

\begin{description}
\config{OPT\_RCAPID}{OPT\_RCAPID}{OPTRCAPID}{
Diese Variable aktiviert das Anbieten der auf dem Router verfügbaren ISDN-CAPI
für entfernte Klienten. Mögliche Werte sind ``yes'' und ``no''. Wird diese
Variable auf ``yes'' gesetzt, wird der Internet-Dämon inetd so konfiguriert,
dass auf Anfragen am rcapid-Port 6000 der rcapid-Dämon gestartet wird (der Port
kann mit Hilfe der Variablen \var{RCAPID\_PORT} geändert werden).

Beispiel:
}
\verb*?OPT_RCAPID='yes'?

\config{RCAPID\_PORT}{RCAPID\_PORT}{RCAPIDPORT}{
Diese Variable enthält den TCP-Port, den der rcapid-Dämon benutzen soll.

Standard-Einstellung:
}
\verb*?RCAPID_PORT='6000'?
\end{description}

\subsubsection{Konfiguration der Linux-Klienten}
Um auf einem Linux-Rechner die entfernte CAPI-Schnittstelle nutzen zu können,
muss die modulare libcapi20-Bibliothek verwendet werden. Aktuelle
Linux-Distributionen installieren bereits eine solche CAPI-Bibliothek (z.B.
Debian Wheezy). Falls nicht, können die Quellen von
\altlink{http://ftp.de.debian.org/debian/pool/main/i/isdnutils/isdnutils_3.25+dfsg1.orig.tar.bz2}
heruntergeladen werden; nach dem Auspacken und dem Wechsel ins Verzeichnis
``capi20'' kann die CAPI-Bibliothek mit dem üblichen Dreierschritt
``./configure'', ``make'' und ``sudo make install'' übersetzt und installiert
werden. Ist die Bibliothek erst einmal installiert, muss man nur noch in der
Konfigurationdatei \texttt{/etc/capi20.conf} vermerken, auf welchem Rechner das
rcapid-Programm läuft. Ist der Router beispielsweise unter dem Namen ``fli4l''
erreichbar, sieht die Konfigurationsdatei folgendermaßen aus:

\begin{example}
\begin{verbatim}
REMOTE fli4l 6000
\end{verbatim}
\end{example}

Das war's! Ist auf dem Linux-Klienten das Programm ``capiinfo'' installiert
(Teil des capi4k-utils-Pakets vieler Distributionen), dann kann man sofort die
entfernte CAPI-Schnittstelle testen:

\begin{example}
\begin{verbatim}
kristov@peacock ~ $ capiinfo 
Number of Controllers : 1
Controller 1:
Manufacturer: AVM Berlin
CAPI Version: 1073741824.1229996355
Manufacturer Version: 2.2-00  (808333856.1377840928)
Serial Number: 0004711
BChannels: 2
[...]
\end{verbatim}
\end{example}

Unter ``Number of Controllers'' wird die Anzahl der ISDN-Karten vermerkt, die
auf dem Klienten nutzbar sind. Steht hier ``0'', dann funktioniert zwar die
Verbindung zum rcapid-Programm, aber auf dem Router wird/werden die
ISDN-Karte(n) nicht erkannt. Funktioniert die Verbindung zum rcapid-Programm
gar nicht (z.B. weil \var{OPT\_RCAPID} auf ``no'' steht), dann erscheint die
Fehlermeldung ``capi not installed - Connection refused (111)''. In diesem Fall
sollte man die Konfiguration noch einmal überprüfen.
