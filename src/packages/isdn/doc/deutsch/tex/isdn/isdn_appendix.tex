% Last Update: $Id$
\section{ISDN}
\marklabel{sec:route-details}
{
  \subsection{Technische Details zu Einwahl und Routing bei ISDN}
}

Dieses Kapitel ist nur für Leute interessant, die ein wenig verstehen
wollen, was intern passiert, die spezielle Konfigurationswünsche haben
oder die nach der Lösung für Probleme suchen. Andere sollten dieses
Kapitel bitte \emph{ nicht} lesen.

Nach dem Herstellen einer Verbindung zum Provider konfiguriert der
ipppd-Daemon, der diese Verbindung hergestellt hat, das Interface neu,
um die ausgehandelten IP-Adressen zu setzen. Dabei werden vom
Linux-Kern automatisch auch Routen gesetzt, die der Remote-IP und der
Netzmaske entsprechen und vorhandene, spezielle Routen werden
gelöscht. Wird keine Netzmaske vorgegeben, leitet der ipppd aus der
Remote-IP die Netzmaske ab (er benutzt dazu die Überholte Einteilung
in Class A,B und C Subnetze). Das Verschwinden der Routen und die
automatisch gesetzten neuen Routen haben immer wieder für Probleme
gesorgt:
\begin{itemize}
\item Firmennetze waren nicht mehr erreichbar, weil die Routen
  verschwunden waren oder von den gesetzten neuen Routen überlagert
  wurden
\item Interfaces wählten sich scheinbar ohne Grund ein, da ein Paket
  statt über die default Route über die vom Kern generierte Route auf
  ein anderes Interface ging
\item ...
\end{itemize}

Daher wird nunmehr versucht, diese unerwünschten Routen zu verhindern.


Dazu werden folgende Dinge geändert:
\begin{itemize}

\item remote ip wird auf 0.0.0.0 gesetzt, wenn nichts anderes spezifiert
  ist. Dadurch verschwinden die Routen, die beim Konfigurieren des
  Interfaces vom Kern eingerichtet werden.

\item zusätzlich angegebene Routen über den Circuit werden in einer Datei
  zwischengespeichert

\item wird eine Netzwerkmaske für den Circuit angegeben, wird diese an den
  ipppd weitergereicht, damit der sie nach Aushandeln der IP für die
  Konfiguration des Interfaces (und damit für die Generierung von
  Routen) nutzt.

\item nach dem Einwählen werden die zwischengespeicherten Routen des
  Circuits ausgelesen und wieder gesetzt (sie wurden vom Kern beim
  Neukonfigurieren des Interfaces durch ipppd gelöscht)

\item nach dem Auflegen wird das Interface wieder neu konfiguriert und die
  Routen werden neu gesetzt um die Ausgangssituation wieder
  herzustellen.
\end{itemize}

Die Konfiguration der Circuits sieht dann wie folgt aus:

\begin{itemize}
\item 
  item default route
  \begin{small}
\begin{example}
\begin{verbatim}
    ISDN_CIRC_%_ROUTE='0.0.0.0'
\end{verbatim}
\end{example}
  \end{small}
  
  Ist der Circuit ein lcr circuit und gerade ``aktiv'', wird eine
  default route auf diesen Circuit (bzw. das dazugehörige Interface)
  gesetzt.  Nach dem Einwählen erscheint eine Host-Route zum Provider,
  die nach dem Auflegen wieder verschwindet.


  \item spezielle Routen
  \begin{small}
\begin{example}
\begin{verbatim}
    ISDN_CIRC_%_ROUTE='network/netmaskbits'
\end{verbatim}
\end{example}
  \end{small}
  
  Es werden die angegeben Routen auf den Circuit (bzw. das
  dazugehörige Interface) eingerichtet.  Nach dem Einwählen werden die
  von Kern gelöschten Routen wieder hergestellt und es gibt eine
  Host-Route zum Einwahlknoten. Nach dem Auflegen wird der
  Originalzustand wieder hergestellt.



  \item remote ip
  \begin{small}
\begin{example}
\begin{verbatim}
    ISDN_CIRC_%_REMOTE='ip address/netmaskbits'
    ISDN_CIRC_%_ROUTE='network/netmaskbits'
\end{verbatim}
\end{example}
  \end{small}
  
  Beim Konfigurieren des Interfaces erscheinen Routen in das Zielnetz
  (entsprechend ip-adress AND netmask).  Wird die spezifizierte IP
  nach dem Einwählen beibehalten (dass heißt, es wird keine andere ip
  während des Verbindungsaufbaus ausgehandelt), bleibt die Route
  bestehen.
  
  Wird allerdings beim Einwählen eine andere IP ausgehandelt, ändert
  sich die Route entsprechend (new ip AND netmask).
  
  Für die zusätzlichen Routen gilt das oben gesagte.


\end{itemize}

Das löst hoffentlich vorläufig \emph{alle} Probleme, die mit speziellen
Routen auftraten. Die Form der Korrektur mag sich in Zukunft noch
ändern, an dem Prinzip ändert sich hoffentlich nichts mehr.

\marklabel{sec:isdn-cause}
{
  \subsection{Fehlermeldungen des ISDN-Subsystems (englisch, i4l-Dokumentation)}
}

Im folgenden ein Auszug aus der Isdn4Linux Dokumentation (man 7
isdn\_cause). 

Cause messages are 2-byte information elements, describing the state
transitions of an ISDN line. Each cause message describes its
origination (location) in one byte, while the cause code is described
in the other byte. Internally, when EDSS1 is used, the first byte
contains the location while the second byte contains the cause code.
When using 1TR6, the first byte contains the cause code while the
location is coded in the second byte. In the Linux ISDN subsystem, the
cause messages visible to the user are unified to avoid confusion. All
user visible cause messages are displayed as hexadecimal strings.
These strings always have the location coded in the first byte,
regardless if using 1TR6 or EDSS1. When using EDSS1, these strings are
preceeded by the character 'E'.


\begin{description}
\item [LOCATION] 
 
  The following location codes are defined when using EDSS1:

  \begin{small}
  \begin{longtable}{lp{12cm}}

  00 &   Message generated by user. \\
  01 &   Message generated by private network serving the local user. \\
  02 &   Message generated by public network serving the local user. \\
  03 &   Message generated by transit network. \\
  04 &   Message generated by public network serving the remote user. \\
  05 &   Message generated by private network serving the remote
  user. \\
  07 &   Message generated by international network. \\
  0A &   Message generated by network beyond inter-working point. \\
  \end{longtable}
  \end{small}

\item  [CAUSE]

  The following cause codes are defined when using EDSS1:

  \begin{small}
  \begin{longtable}{lp{12cm}}

  01 &   Unallocated (unassigned) number. \\
  02 &   No route to specified transit network. \\
  03 &   No route to destination. \\
  06 &   Channel unacceptable. \\
  07 &   Call awarded and being delivered in an established channel. \\
  10 &   Normal call clearing. \\
  11 &   User busy. \\
  12 &   No user responding. \\
  13 &   No answer from user (user alerted). \\
  15 &   Call rejected. \\
  16 &   Number changed. \\
  1A &   Non-selected user clearing. \\
  1B &   Destination out of order. \\
  1C &   Invalid number format. \\
  1D &   Facility rejected. \\
  1E &   Response to status enquiry. \\
  1F &   Normal, unspecified. \\
  22 &   No circuit or channel available. \\
  26 &   Network out of order. \\
  29 &   Temporary failure. \\
  2A &   Switching equipment congestion. \\
  2B &   Access information discarded. \\
  2C &   Requested circuit or channel not available. \\
  2F &   Resources unavailable, unspecified. \\
  31 &   Quality of service unavailable. \\
  32 &   Requested facility not subscribed. \\
  39 &   Bearer capability not authorised. \\
  3A &   Bearer capability not presently available. \\
  3F &   Service or option not available, unspecified. \\
  41 &   Bearer capability not implemented. \\
  42 &   Channel type not implemented. \\
  45 &   Requested facility not implemented. \\
  46 &   Only restricted digital information bearer. \\
  4F &   Service or option not implemented, unspecified. \\
  51 &   Invalid call reference value. \\
  52 &   Identified channel does not exist. \\
  53 &   A suspended call exists, but this call identity does not. \\
  54 &   Call identity in use. \\
  55 &   No call suspended. \\
  56 &   Call having the requested call identity. \\
  58 &   Incompatible destination. \\
  5B &   Invalid transit network selection. \\
  5F &   Invalid message, unspecified. \\
  60 &   Mandatory information element is missing. \\
  61 &   Message type non-existent or not implemented. \\
  62 &   Message not compatible with call state or message 
        or message type non existent or not implemented. \\
  63 &   Information element non-existent or not implemented. \\
  64 &   Invalid information element content. \\
  65 &   Message not compatible. \\
  66 &   Recovery on timer expiry. \\
  6F &   Protocol error, unspecified. \\
  7F &   Inter working, unspecified. \\
  \end{longtable}
  \end{small}
\end{description}
