% Do not remove the next line
% Synchronized to r29817

\section {ISDN~- Communication avec les cartes ISDN (ou Numéris) actives et passives}
\configlabel{OPT\_ISDN}{OPTISDN}

fli4l a été pensé principalement pour l'application ISDN (ou Numéris)
et/ou pour un routeur DSL. Avec le paramétrage de la variable 
\var{OPT\_\-ISDN='yes'} le programme ISDN devient actif. A condition
que la carte ISDN soit supporté par fli4l.

Si vous n'utilisé pas d'ISDN, le paramètre de la variable sera alors
\var{OPT\_\-ISDN}='no' aucune carte ISDN ne sera installée. Vous
pouvez alors ignorer la suite de ce chapitre.

Configuration par défaut~: \var{OPT\_\-ISDN='no'}

\subsection {Établir une connexion par ISDN}
\begin{sloppypar}
La sélection du comportement de fli4l est déterminé par trois variables
différentes, \mbox{\var{DIALMODE},} \var{ISDN\_\-CIRC\_\-X\_\-ROUTE\_X},
\var{ISDN\_\-CIRC\_\-X\_\-TIMES}. Lorsqu'un paquet arrive sur le circuit 
\jump{DIALMODE}{\var{DIALMODE}} actif (dans le fichier $<$config$>$/base.txt),
une connexion doit se construire automatiquement ou pas. La variable
\var{DIALMODE} peut accepter les valeurs suivantes~:
\end{sloppypar}

\begin{description}
\item[auto] Si un paquet arrive sur le circuit ISDN (il traverse
l'interface ISDN puis il est changé en ippp*) il ouvre automatiquement une connexion.
Quand le paquet arrive sur le circuit ISDN les deux variables sont utilisées
\var{ISDN\_\-CIRC\_\-X\_\-ROUTE\_X} et \var{ISDN\_\-CIRC\_\-X\_\-TIMES}.

\item[manual] En mode manuel, la connexion doit être déclenchés avec
le client imonc/imond. La documentation se trouve dans la section imonc/imond.

\item[off] Aucune connexion ISDN n'est établie.
\end{description}

Les paquets qui arrivent sur le réseau, sont défini dans la variable
\var{ISDN\_\-CIRC\_\-X\_\-ROUTE\_X} qui est prés configuré. Avec le réglage
'0.0.0.0/0', on appelle ce réglage 'default route' (ou route par défaut) avec
pondération. Cela signifie aussi que tous les paquets qui quittent le réseau
local passent par ce réseau, s'il est actif. Quand le réseau est actif, la
variable \var{ISDN\_\-CIRC\_\-X\_\-TIMES} est analysée, puis fli4l li la variable
\emph{least cost routing} "moindre coût de routage" (voir la section
\jump{sec:leastcostrouting}{Least-Cost-Routing~- Mode de fonctionnement}
dans la documentation du paquetage base). Si l'on ne veut pas que les paquets
passent par tous les réseaux, mais uniquement sur certains réseaux (par ex. le
réseau d'entreprise), on pourra indiquer ici un ou plusieurs réseaux différent.
fli4l gère l'interface ISDN qui est en permanence actif et le réseau informatique
qui lui est assigné. Maintenant si un paquet est envoyé par un PC du réseau local,
une connexion automatiquement se crée.

Comme déjà mentionné, la variable \var{ISDN\_\-CIRC\_\-X\_\-TIMES} sert à
activer le côut des connexions pour le circuit ISDN, lorsque le réseau
'default route' (ou routage par défaut) est activé le tableau des
frais de connexion peut être déclenchés. 'Si' dans la variable on a
spécifié la date, les deux premiers paramètres de time-info
(par ex. Mo-Fr:09-18), et 'si' le quatrième paramètre de
lc-default-route est sur (y/n). fli4l (par le démon imond) s'occupe
alors des paquets qui quittent le réseau local, passent toujours par
le réseau actif puis par l'interface ISDN et établi une connexion avec
le fournisseur d'accès Internet à la date/heure indiqué.

En résumé selon l'utilisation par défaut on peut dire suivant les cas~:

\begin{itemize}
\item Si on veut uniquement Internet, on met auto dans \var{DIALMODE},
on définit 1-n circuit, comme le premier route '0.0.0.0/0' et comme temps
(avec lc-default-route = Y) pour toute la semaine.

\begin{small}
\begin{example}
\begin{verbatim}
        ISDN_CIRC_%_ROUTE_N='1'
        ISDN_CIRC_%_ROUTE_1='0.0.0.0/0'
        ISDN_CIRC_%_TIMES='Mo-Su:00-24:0.0148:Y'
\end{verbatim}
\end{example}
\end{small}

\item Si on veut utiliser une connexion spécial pour un réseau
d'entreprise, on définit le réseau (ou plusieurs réseaux)
avec une route différente de '0.0.0.0/0', ainsi vous avez en
permanence un accès actif pour un réseau d'entreprise spécifique.
\begin{small}
\begin{example}
\begin{verbatim}
        ISDN_CIRC_%_ROUTE_N='1'
        ISDN_CIRC_%_ROUTE_1='network/netmaskbits'
        ISDN_CIRC_%_TIMES='Mo-Su:00-24:0.0148:Y'
\end{verbatim}
\end{example}
\end{small}
\end{itemize}

\subsection{Carte ISDN}

Le routeur fli4l supporte l'utilisation simultanée de plusieurs cartes ISDN.
Cependant, cela nécessite que toutes les cartes ISDN requièrent \emph{le même
type de pilote}. Le type de pilote est choisi dans le tableau ci-dessous, vous pouvez
trouver tous les pilotes correspondant aux cartes définis. Par exemple, vous pouvez utilisation
plusieurs cartes avec un pilote mISDN ou plusieurs cartes avec un pilote HiSax
(s'il y a suffisamment de ressources disponibles) cela ne pose pas de problème, mais
il n'est pas possible d'utiliser simultanément des cartes alimentés par mISDN et HiSax.

\begin{description}
\configlabel{ISDN\_\%\_IO}{ISDNxIO}
\configlabel{ISDN\_\%\_IO0}{ISDNxIO0}
\configlabel{ISDN\_\%\_IO1}{ISDNxIO1}
\configlabel{ISDN\_\%\_MEM}{ISDNxMEM}
\config{ISDN\_\%\_TYPE ISDN\_\%\_IO ISDN\_\%\_IO0 ISDN\_\%\_IO1 ISDN\_\%\_MEM ISDN\_\%\_IRQ} {ISDN\_\%\_TYPE} {ISDNxTYPE}

  On mentionne ici les données techniques pour la carte ISDN (ou RNIS).

  Les valeurs mentionnées dans l'exemple fonctionnent avec une carte
  TELES 16.3, elle est réglée sur l'Adresse IO 0xd80 (en utilisant les Switches Dip).
  Si vous changez le réglage de la carte, l'adresse doit être modifiée.

  \achtung{Erreur fréquemment faite (exemple)~:}
  \begin{example}
  \begin{verbatim}
          ISDN_1_IO='240' -- au lieu de~: ISDN_1_IO='0x240'
  \end{verbatim}
  \end{example}

  Si vous utilisez l'IRQ 12 vous devez couper la souris PS/2 qui
  est éventuellement disponible dans le BIOS. Mieux vaut choisir
  un autre IRQ! \flqq{}Les bons\frqq{} IRQ sont généralement 5, 10 et 11.

  \var{ISDN\_\%\_\-TYPE} par principe le type correspond au numéro de
  pilote HiSax. Excepté~: les cartes non HiSax comme par exemple AVM-B1,
  la numérotation de ces types de cartes a été élargie. La liste de tous
  des types HiSax sont indiquées dans \\
  \var{linux-2.x.y/Documentation/isdn/README.HiSax}.

\begin{small}
  \begin{longtable}{|r|p{60mm}|p{62mm}|}
    \hline
    \multicolumn{1}{|c}{\textbf{Type}} & \multicolumn{1}{|c}{\textbf{Carte}} &
    \multicolumn{1}{|c|}{\textbf{Paramètres nécessaires}} \\
    \hline\hline
    \endhead

    \multicolumn{3}{|l|}{Numéro du type pour pilote factice~:} \\
    \hline
      0 &  no driver (dummy)             & none \\

    \hline\hline
    \multicolumn{3}{|l|}{Numéro du type pour pilote Remote-CAPI~:} \\
    \hline

    160 & AVM Fritz!Box Remote CAPI   & ip,port \\
    161 & Melware Remote CAPI (rcapi) & ip,port \\

    \hline\hline
    \multicolumn{3}{|l|}{Numéro du type pour pilote mISDN~:} \\
    \hline

    301 & HFC-4S/8S/E1 multiport cards         & no parameter \\
    302 & HFC-PCI based cards                  & no parameter \\
    303 & HFCS-USB Adapters                    & no parameter \\
    304 & AVM FritZ!Card PCI (v1 and v2) cards & no parameter \\
    305 & cards based on Infineon (former Siemens) chips: \newline
          - Dialogic Diva 2.0 \newline
          - Dialogic Diva 2.0U \newline
          - Dialogic Diva 2.01 \newline
          - Dialogic Diva 2.02 \newline
          - Sedlbauer Speedwin \newline
          - HST Saphir3 \newline
          - Develo (former ELSA) Microlink PCI (Quickstep 1000) \newline
          - Develo (former ELSA) Quickstep 3000 \newline
          - Berkom Scitel BRIX Quadro \newline
          - Dr.Neuhaus (Sagem) Niccy           & no parameter \\
    306 & NetJet TJ 300 and TJ320 cards        & no parameter \\
    307 & Sedlbauer Speedfax+ cards            & no parameter \\
    308 & Winbond 6692 based cards             & no parameter \\
    \hline
  \end{longtable}
\end{small}

  Ma carte est une Teles 16.3 NON-PNP ISA, elle est donc du Type=3.

  Par exemple pour une carte ICN-2B les paramètres IO et MEM doivent être,
  \var{ISDN\_1\_\-IO}='0x320', \var{ISDN\_1\_\-MEM}='0xd0000'.

  Les nouvelles cartes Teles-PCI doivent être indiqués avec le type=20 (au lieu de 21).
  Au sujet des paramètres ils sont indiqués avec la commande "cat /proc/pci" avec "tiger"
  ou semblable. Si vous ne trouvez pas cette valeur, je ne peux rien faire pour vous, désolé.

  Pour la configuration des types ISDN 105 à 114, il est nécessaire au préalable de télécharger
  les pilotes ici \altlink{http://www.fli4l.de/fr/telechargement/version-stable/pilote-avm/}
  et de décompresser les fichiers dans le répertoire fli4l. malheureusement ces
  pilotes ne sont pas sous licence GPL, c'est pour cela qu'il ne sont pas fournis
  avec le paquetage ISDN.\\

  Pour l'utilisation des type ISDN 303 il est nécessaire d'installer et
  d'activer le support USB. Voir la section
  \jump{sec:opt-usb }{USB~- Gestion des périphériques USB}.

  Quelques conseils sont disponibles sur fi4l FAQ ou sur Mailing-liste au sujet
  des numéros de type pour les pilotes de cartes ISDN, si vous ne savait pas exactement
  quel Type de carte est dans votre PC.

  Certains types de cartes sont identifiés avec la fonction \flqq{}from isapnp setup\frqq{}
  ils doivent être initialisés avec l'outil PnP Tool isapnp~- S'il s'agit réellement
  d'une carte PnP. Voir la documentation dans le chapitre
  \jump{OPTPNP}{OPT\_PNP~- Outil d'installation pour isapnp}.

  Le Type ISDN 0 est nécessaire, si l'on veut installer le paquetage ISDN \textbf{sans}
  carte ISDN dans votre PC, pour pouvoir par exemple utiliser imond sur le routeur
  et le client imonc sur le réseau local.

\configlabel{ISDN\_\%\_IP}{ISDNxIP}
\configlabel{ISDN\_\%\_PORT}{ISDNxPORT}
\config{ISDN\_\%\_IP ISDN\_\%\_PORT} {ISDN\_\%\_IP} {ISDNxIP}

  Pour l'utilisation des types ISDN 160 et 161, vous devez définir dans la variable
  (\var{ISDN\_\%\_IP}) l'adresse IP et dans \var{ISDN\_\%\_PORT}) le port du périphérique
  offrant une interface avec le démon CAPI distant. L'adresse IP est obligatoire, mais le numéro
  de port peut être omis~: selon le type sélectionné, un port standard sera alors défini, pour
  le (type 160: 5031, type 161: 2662).

  Exemple~:
  \begin{example}
  \begin{verbatim}
          ISDN_1_TYPE='160' # AVM Fritz!Box
          ISDN_1_IP='192.168.177.1'
  \end{verbatim}
  \end{example}

\config{ISDN\_DEBUG\_LEVEL}{ISDN\_DEBUG\_LEVEL}{ISDNDEBUGLEVEL}

  Pour activer Debug-Level avec les cartes HiSaX. Pour vous aidez, Debug-Level
  (ou niveau de débogage) se compose de valeurs suivantes (documentation original)~:\\

  \begin{tabular}[h!]{r|l}
   \multicolumn{1}{c|}{\textbf{Number}} & \multicolumn{1}{c}{\textbf{Debug-Information}} \\
   \hline
      1 & Link-level $<$--$>$ hardware-level communication \\
      2 & Top state machine \\
      4 & D-Channel Q.931 (call control messages) \\
      8 & D-Channel Q.921 \\
     16 & B-Channel X.75 \\
     32 & D-Channel l2 \\
     64 & B-Channel l2 \\
    128 & D-Channel link state debugging \\
    256 & B-Channel link state debugging \\
    512 & TEI debug \\
   1024 & LOCK debug in callc.c \\
   2048 & More debug in callc.c (not for normal use) \\
  \end{tabular}\latex{\\}

  Valeur par défaut (\var{ISDN\_DEBUG\_LEVEL}='32') devrait être la plus adaptée.

\config{ISDN\_VERBOSE\_LEVEL}{ISDN\_VERBOSE\_LEVEL}{ISDNVERBOSELEVEL}

  Avec cette variable on peut régler "divers" fonction dans le sous système ISDN
  du Kernel fli4l. Dans Verbose-Level chaque numéro correspond à un niveau du
  plus bas au plus haut. Voici les Verbose-Level~:

  \begin{tabular}[h!]{lp{10cm}}
    '0' & Enregistrement d'aucune information supplémentaire \\
    '1' & Enregistrement à la moindre connexion ISDN\\
    '2' et '3' & Les appels Tél. son enregistrés dans un journal\\
    '4' et plus & enregistrement régulier du taux de transfert de données. \\
        % Beim letzten bin ich mir nicht ganz sicher, siehe
        % linux-kernel-source/drivers/isdn/isdn_net.c, suche nach "dev->net_verbose > 3"
  \end{tabular}

  Pour visualiser les messages du Kernel-Logging-Interface, il faut
  activer la variable \jump{OPTSYSLOGD}{\var{OPT\_SYSLOGD}}.

  \wichtig{si les appels Tél. sont activer avec telmond, le réglage pour
  l'enregistrement des appel Tél. ne doit pas être inférieur à 2 autrement
  aucun appel ne sera enregistré.}

  Configuration par défaut~: \var{ISDN\_VERBOSE\_LEVEL}='2'

\begin{sloppypar}
\config{ISDN\_FILTER}{ISDN\_FILTER}{ISDNFILTER}

Active le mécanisme de filtrage du Kernel, afin d'assurer le bon fonctionnement
du Hangup-Timeout. Voir pour de plus amples informations
\altlink{http://www.fli4l.de/hilfe/howtos/basteleien/hangup-problem-loesen/}
\end{sloppypar}
\end{description}

\subsection{OPT\_ISDN\_COMP (Expérimantal)}
\configlabel{OPT\_ISDN\_COMP}{OPTISDNCOMP}

Si vous avez activez la variable \var{OPT\_\-ISDN\_\-COMP}='yes'
la compression de LZS et de BSD sera possible. Les paquets seront compressés
lorsque la connexion sera établie. Merci à Arwin Vosselman 
(\email{arwin(at)xs4all(dot)nl}). Cette variable supplémentaire à un statut expérimental.

Configuration par défaut~: \var{OPT\_\-ISDN\_\-COMP}='no'

Détail des paramètres pour DEBUG, nécessaires avec la compression LZS~:

\begin{description}
\config{ISDN\_LZS\_DEBUG (Expérimental)}{ISDN\_LZS\_DEBUG}{ISDNLZSDEBUG}

  Réglage-Debug-Level~:

  \begin{tabular}[h!]{ll}
    '0' & aucune Information de Debogage \\
    '1' & Information normale de Debogage \\
    '2' & Information élargie de Debogage \\
    '3' & Information total de Debogage  (incl. dumping des paquets de
    données)\\
  \end{tabular}

  Configuration par défaut~: \var{ISDN\_\-LZS\_\-DEBUG}='1'

  en cas de problèmes de compression, pour avoir plus détail sur des messages
de débogage, mettais la variable sur '2'.

\config{ISDN\_LZS\_COMP (EXPERIMENTAL)}{ISDN\_LZS\_COMP}{ISDNLZSCOMP}

  Puissance de compression (pas de la décompression!). Restez sur la valeur '8'.
  Les valeurs possibles sont de 0 à 9

  Plus le nombre est grand meilleure est la compression, cependant '9'
  est excessif, le CPU est trop sollicité.

  Configuration par défaut~: \var{ISDN\_\-LZS\_\-COMP}='8'

\config{ISDN\_LZS\_TWEAK (EXPERIMENTAL)}{ISDN\_LZS\_TWEAK}{ISDNLZSTWEAK}

  Dans cette variable vous pouvez laisser la valeur sur '7'.

  Configuration par défaut~: \var{ISDN\_\-LZS\_\-TWEAK}='7'

  En plus de la configuration des 3 dernières variables, la variable
  \var{ISDN\_\-CIRC\_\-x\_\-FRAMECOMP} doit être configurée,
  voir le chapitre suivant.
\end{description}

\subsection{Circuits ISDN}

  Dans la configuration de fli4l on peut définir plusieurs connexions via ISDN.
  Au maximum 2 connexions égal sont possibles sur une mêmes carte ISDN.

  La définition de ces connexions, se nomme circuit dans la configuration de fli4l.
  Un Circuit est utilisé par connexion.

  Dans notre fichier d'exemple config.txt on a définis deux Circuits~:

\begin{itemize}
\item Circuit 1~: Dialout sur Internet-By-Call-Provider Microsoft Network, Sync-PPP

\item Circuit 2~: Dialin/Dialout sur le routeur ISDN (par exemple on pourrait
également indiquer, fli4l)

  Avec Raw IP (ce fonctionnement interne, utilise uniquement les numéros de
  téléphone pour établir une connexion), par exemple pour accèder au réseau
  d'entreprise de n'importe où. Concrètement chez moi, c'est un Linux-Box avec
  isdn4linux comme "adversaire".
\end{itemize}

  Si le routeur fli4l sert uniquement de gateway (ou passerelle) Internet, un seul
  circuit est nécessaire. Exception~: si vous utilisez le routeur fli4l avec
  Least-Cost-Router-Features (ou calcul des coûts des connexions téléphoniques).
  Tous les circuits autorisés sont à définir dans des domaines différent,
  voir ci-dessous.

\begin{description}
\config{ISDN\_CIRC\_N}{ISDN\_CIRC\_N}{ISDNCIRCN}

  Dans cette variable on indique le nombre de circuit ISDN à utiliser.
  Si vous utilisez uniquement fli4l comme écran d'affichage pour des appels
  Tél. avec ISDN vous pouvez paramétrer la variable~:

\begin{example}
\begin{verbatim}
        ISDN_CIRC_N='0'
\end{verbatim}
\end{example}

  Si le routeur fli4l sert uniquement de gateway (ou passerelle) Internet,
  un seul circuit est nécessaire. exception~: pour le LC-Routing (ou calcul des coûts
  des connexions téléphoniques), voir ci-dessous.

\config{ISDN\_CIRC\_x\_NAME}{ISDN\_CIRC\_x\_NAME}{ISDNCIRCxNAME}

  Dans cette variable on indique le nom du circuit maximum 15 caractères. Le nom
  sera visible sur le client imonc \texttt{imonc.exe} au lieu du numéro de téléphone.
  Les caractère autorisés sont de 'A' à 'Z' (minuscule et majuscule), et les chiffre
  de '0' à '9' et aussi le trait d'union '-', par exemple.

\begin{example}
\begin{verbatim}
        ISDN_CIRC_x_NAME='msn'
\end{verbatim}
\end{example}

  Le nom de circuit peut être utilisé pour configurer le filtrage de paquet ou
  OpenVPN. par ex. lorsque l'on configure le filtrage de paquet pour le circuit ISDN,
  il faut indiquer d'abord le prémis 'circuit\_' puis le nom de circuit. Si le
  circuit s'appelle 'willi', on peut écrire dans le filtrage de paquet~:

\begin{example}
\begin{verbatim}
PF_INPUT_3='if:circuit_willi:any prot:udp 192.168.200.226 192.168.200.254:53 ACCEPT'
\end{verbatim}
\end{example}

\config{ISDN\_CIRC\_x\_USEPEERDNS}{ISDN\_CIRC\_x\_USEPEERDNS}{ISDNCIRCxUSEPEERDNS}

  Il est établi que les fournisseurs d'accès Internet utilise un serveur de
  nom (ou DNS) et nous devons enregistrer ce serveur de nom dans notre réseau
  local pour la durée de connexion. Rationnellement cette option est seulement
  utiliser pour la connexion au fournisseur d'accès d'Internet. En même temps,
  presque tous les fournisseurs supporte ce type de transfert.

  Vous devez enregistrer les adresses IP du serveur de nom donné par votre FAI
  dans le fichier base.txt à la variable \emph{\var{DNS\_\-FORWARDERS}} et vous
  devez supprimer le serveur de nom qui est configuré sur votre PC du réseau
  local. Ensuite vous devez mettre à la place l'adresse IP de votre routeur. Avec
  se réglage la résolution des noms ne se perd pas dans le cache du serveur de nom.

  Cette option offre l'avantage de toujours pouvoir travailler avec un serveur de
  nom le plus proche, dans la mesure où le fournisseur d'accès à une adresse IP
  correcte~- Ainsi, la résolution de nom sera plus rapide.

  En cas d'une défaillance d'un serveur de DNS du fournisseur d'accès, on pourra
  en régle général corriger plus rapidement les adresses des serveurs DNS du
  fournisseur d'accès.

  Malgré tout, il est nécessaire d'indiquer un serveur de nom valide dans la
  variable \emph{\var{DNS\_\-FORWARDERS}} du fichier base.txt pour se connecter,
  autrement lors de la première connexion Internet la demande ne pourra pas être
  résolue correctement. En outre, la configuration originale du serveur de nom
  local est restaurée à la fin de la connexion.

  Configuration par défaut~: \var{ISDN\_\-CIRC\_\-x\_\-USEPEERDNS}='yes'

\config{ISDN\_CIRC\_x\_TYPE}{ISDN\_CIRC\_x\_TYPE}{ISDNCIRCxTYPE}

  Dans cette variable \var{ISDN\_\-CIRC\_\-x\_\-TYPE} on indique le type de
  connexion IP. Les valeurs possibles sont les suivantes~:

  \begin{tabular}[h!]{ll}
        'raw' &           RAW-IP\\
        'ppp' &           Sync-PPP\\
  \end{tabular}

  Dans la plus par des cas on utilise PPP, mais Raw IP est un peu plus efficace,
  étant donné que PPP-Overhead est supprimé. Cependant une authentification de
  Raw IP n'est pas possible, toutefois, on peut indiquer dans la variable
  \var{ISDN\_\-CIRC\_\-x\_\-DIALIN} un accès limité à un numéro ISDN (mot-clé "Clip").
  Si on paramètre la variable \var{ISDN\_\-CIRC\_\-x\_\-TYPE} avec 'raw' on peut
  créer un script analogique PPP up/down et un script raw up/down dans le dossier /etc/ppp.

\config{ISDN\_CIRC\_x\_BUNDLING}{ISDN\_CIRC\_x\_BUNDLING}{ISDNCIRCxBUNDLING}

  Dans cette variable avec le protocole MPPP (Multilink Protocol) RFC 1717, on
  permet l'agrégation des canaux ISDN. Dans la pratique plupart du temps elle ne
  sont pas pertinentes, pour que ces restrictions soient applicables il faut~:
  \begin{itemize}
  \item que se la soit possible uniquement avec la liaison PPP et non avec un circuits Raw
  \item que l'agrégation des canaux avec la nouvelle RFC 1990 (MLPPP) est pas possible
  \end{itemize}

  Le 2ème canal peut est activé manuellement avec le client imonc ou activé
  automatiquement par rapport à la bande passante, pour cela voir la variable
  \var{ISDN\_\-CIRC\_\-x\_\-BANDWIDTH}.

  Configuration par défaut~: \var{ISDN\_\-CIRC\_\-x\_\-BUNDLING}='no'

  Attention~: lors de l'utilisation des canaux, associer à la compression cela
  peut occasionner des problèmes, voir aussi la description de la variable
  \var{ISDN\_\-CIRC\_\-x\_\-FRAMECOMP}.

\config{ISDN\_CIRC\_x\_BANDWIDTH}{ISDN\_CIRC\_x\_BANDWIDTH}{ISDNCIRCxBANDWIDTH}

  Si l'agrégation des canaux-ISDN est activer dans\linebreak
  \var{ISDN\_\-CIRC\_\-x\_\-BUNDLING}='yes', vous pouvez paramètrer cette variable
  pour automatiser le 2ème canal-ISDN. Il y a 2 paramètres numériques à respecter~:
  \begin{enumerate}
  \item  La valeur du seuil en Octet/Seconde (S)
  \item  Interval temps en Seconde (Z)
  \end{enumerate}

  Si la valeur du seuil (S) est dépassé pendant un interval temps (Z) en seconde,
 Imond active le processus de commutation du 2ème canal automatiquement. Si la
 valeur du seuil (S) est inférieur pendant un interval temps (Z) le 2ème canal se
 déactive automatiquement. Pour ne pas activer la bande passante automatiquement,
 il ne faut rien indiquer dans la variable \var{ISDN\_\-CIRC\_\-1\_\-BANDWIDTH}='',
 si vous voulez activer le 2ème canal il faut le faire manuellement avec
 le client imonc.

  Exemple~:
  \begin{itemize}
  \item \var{ISDN\_\-CIRC\_\-1\_\-BANDWIDTH}='6144 30'

   Si la valeur de transfert dépasse 6 Kilo-octets/seconde pendant 30
   secondes le 2ème canal s'active.

  \item \var{ISDN\_\-CIRC\_\-1\_\-BANDWIDTH}='0 0'

    Le deuxième canal ISDN sera immédiatement activé après 10 secondes
    au plus tard, sur une connexion Internet et restera active jusqu'à
    la coupure de la connexion.

  \item \var{ISDN\_\-CIRC\_\-1\_\-BANDWIDTH}=''

    Le deuxième canal ISDN peut être uniquement activé manuellement, à
    condition que la variable \var{ISDN\_\-CIRC\_\-1\_\-BUNDLING}='yes'
    soit configurée.

  \item \var{ISDN\_\-CIRC\_\-1\_\-BANDWIDTH}='10000 30'

    Normalement le deuxième canal sera activé lorsque la valeur de transfert
    atteint les 10 Ko/s pendant 30 secondes. Mais le deuxième canal ne s'activera
    jamais, car le maximun de transfert par canal est de 8 Ko/s.
  \end{itemize}

  Si la variable est sur \var{ISDN\_\-CIRC\_\-x\_\-BUNDLING}='no', la valeur de
  la variable \linebreak \var{ISDN\_\-CIRC\_\-x\_\-BANDWIDTH} est sans intérêts.

  Configuration par défaut~: \var{ISDN\_\-CIRC\_\-x\_\-BANDWIDTH}=''

\config{ISDN\_CIRC\_x\_LOCAL}{ISDN\_CIRC\_x\_LOCAL}{ISDNCIRCxLOCAL}

  On enregistre dans cette variable l'adresse IP du fournisseur d'accés Internet
  pour la partie ISDN. la variable n'est pas dans le fichier de configuration elle
  est à rajouter.

  Si l'\-assi\-gna\-tion de adresse IP est dynamique, le paramètre doit être
  \textbf{vide}. C'est au moment de la connexion, que l'adresse IP est négocier.
  Dans la plus par des cas les fournisseurs d'accès Internet donne une adresse IP
  dynamique. Cependant, si l'on doit attribuer une adresse IP, c'est ici que
  l'on doit l'inscrire. Cette variable est optionnelle et doit être configuré
  qu'en cas de besoin.

\config{ISDN\_CIRC\_x\_REMOTE}{ISDN\_CIRC\_x\_REMOTE}{ISDNCIRCxREMOTE}

  On enregistre dans cette variable le l'adresse IP distant (renvoyer vers) et
  le masque de sous réseau pour la partie ISDN. le masque doit être écrit sous
  la forme CIDR (Classles Inter-Domain Routing). Vous trouverez plus d'information
  sur \jump{IPNETx}{CIDR} dans la documentation du paquetage Base à IP\_NET\_x.

  Si l'\-assi\-gna\-tion de adresse IP est dynamique, le paramètre doit être
  \textbf{vide}. C'est au moment de la connexion, que l'adresse IP est négocier.
  Dans la plus par des cas les fournisseurs d'accès Internet donne une adresse IP
  dynamique. Cependant, si l'on doit attribuer une adresse IP, c'est ici que
  l'on doit l'inscrire. Cette variable est optionnelle et doit être configuré
  qu'en cas de besoin.

  Le numéro utilisé pour le masque de sous réseau est indiqué dans la configuration
  de l'interface. Il sera utilisé pour configurer les réseaux vers des hôtes, pour
  se connecter. En règle générale nous n'avons pas besoin de ce circuit, il est plus
  favorable, de créer uniquement un circuit directement à l'ordinateur de connexion.
  On place le masque de sous réseau sur /32, ici 32 est le nombre de bits du masque
  de sous réseau. Pour plus de détails, voir
  \jump{sec:route-details}{chapitre~: Détails techniques sur la Connexion}.

\configlabel{ISDN\_CIRC\_x\_MTU}{ISDNCIRCxMTU}
\config{ISDN\_CIRC\_x\_MTU ISDN\_CIRC\_x\_MRU}{ISDN\_CIRC\_x\_MRU}{ISDNCIRCxMRU}

  Ces variables sont optionnelles, ont paramètres avec celles-ci le \textbf{MTU}
  (maximum transmission unit) et le \textbf{MRU} (maximum receive unit).
  Optionnelle signifie que les variables ne sont pas dans le fichier de configuration,
  Elles sont à insérer par l'utilisateur si besoin! \\
  Normalement le réglage est~: MTU 1500 et MRU 1524. Ce réglage doit être
  modifier uniquement dans des cas exceptionnels!

\config{ISDN\_CIRC\_x\_CLAMP\_MSS}{ISDN\_CIRC\_x\_CLAMP\_MSS}{ISDNCIRCxCLAMPMSS}

  On devrais mettre cette variable sur 'yes' si on utilise la synchronisation PPP
  (\var{ISDN\_CIRC\_x\_TYPE}='ppp') et si l'un des symptômes suivants se produit~:
\begin{itemize}
\item Si le navigateur du PC se connecte à un serveur web et qu'il n'y pas la page
  qui s'affichée et aucun message d'erreur, plus simplement il ne se passe rien.
\item Lorsque vous envoyez de petits courriels cela fonctionne, mais si vous avez des
  problèmes pour envoyer des courriels plus important.
\item Avec la fonction ssh, réinitialisation du scp après avoir établi une connexion.
\end{itemize}

  Avec certain FAI, ces problèmes peuvent se produire par exemple Compuserve et
  aussi Mediaways (FAI Allemand).

  Configuration par défaut~: \var{ISDN\_\-CIRC\_\-x\_\-CLAMP\_MSS}='no'

\config{ISDN\_CIRC\_x\_HEADERCOMP}{ISDN\_CIRC\_x\_HEADERCOMP}{ISDNCIRCxHEADERCOMP}

  SI vous paramétrez cette variable sur \var{ISDN\_\-CIRC\_\-x\_\-HEADERCOMP}='yes'
  vous compressez les en-têtes proposé par (Van Jacobson TCP/IP Header Compression).
  Certain FAI ne supporte pas cette fonction. Si vous avez des problèmes avec la
  compression des en-têtes vous devez paramétrer la variable sur
  \var{ISDN\_\-CIRC\_\-x\_\-HEADERCOMP}='no'.

  Configuration par défaut~: \var{ISDN\_\-CIRC\_\-x\_\-HEADERCOMP}='yes'

\config{ISDN\_CIRC\_x\_FRAMECOMP (EXPERIMENTAL)}{ISDN\_CIRC\_x\_FRAMECOMP}{ISDNCIRCxFRAMECOMP}

  Cette variable est uniquement pris en compte, que si la variable
  \var{OPT\_\-ISDN\_\-COMP}='yes' est activée. Cette variable réglemente
  les Frame-Com\-pres\-sion (compression des en-têtes).

  Les paramètres suivants sont possibles~:

  \begin{tabular}[h!]{ll}
        'no' &                    Aucune Compression de Frame (ou Trame)\\
        'default' &               LZS according RFC1974(std) and
        BSDCOMP 12 \\
        'all' &                   Negotiate lzs and bsdcomp \\
        'lzs' &                   Negotiate lzs only \\
        'lzsstd' &                LZS according RFC1974 Standard Mode
                                (``Sequential Mode'') \\
        'lzsext' &                LZS according RFC1974 Extended Mode \\
        'bsdcomp' &               Negotiate bsdcomp only \\
        'lzsstd-mh' &             LZS Multihistory according RFC1974
                                  Standard Mode (``Sequential Mode``)
  \end{tabular}

  Quelle son les paramètres que l'on doit utilisé pour chaque fournisseur, on doit
  essayées. Avec le plus connu T-Online (FAI Allemand) on paramètre uniquement 'lzsext'.
  Dans la plupart des cas, on peut se débrouiller avec le paramètre par défaut
  'default' pour tous les autres fournisseurs d'accès.

  Attention~: lors de l'utilisation de l'agrégation des canaux en relation avec
  'lzsext' il peut en résulter des problèmes. Ces problèmes sont largement connues,
  à la connexion au serveur spécifique, et plus particulièrement à des FAI spécifique.
  cependant les problèmes ne provient pas uniquement des fournisseurs d'accès, mais
  peuvent provenir des différents noeuds de connexion.

  le paramètre 'lzsstd-mh' a été pensé pour la communication de routeur à routeur.
  cette procédure n'est pas utilisé par les FAI mais lors de l'utilisation de deux
  routeurs fli4l amélioration considérable sur le transfert simultané de plusieurs
  fichiers. La compression des en-têtes est nécessaire et sera donc automatiquement activé.

\config {ISDN\_CIRC\_x\_REMOTENAME}{ISDN\_CIRC\_x\_REMOTENAME}{ISDNCIRCxREMOTENAME}

  Cette variable est importante, uniquement pour la configuration de fli4l comme
  routeur de connexion distant. Vous pouvez enregistrer ici un nom hôte distant,
  normalement nous n'en avons pas besoin.

  Configuration par défaut~: \var{ISDN\_\-CIRC\_\-x\_\-REMOTENAME}=''

\configlabel{ISDN\_CIRC\_x\_USER}{ISDNCIRCxUSER}
\config {ISDN\_CIRC\_x\_PASS}{ISDN\_CIRC\_x\_PASS}{ISDNCIRCxPASS}

  Dans ces variables on indique les données du fournisseur d'accès. Il s'agit
  Dans l'exemple ci-dessous, des données du fournisseur de Microsoft Network.

  \var{ISDN\_\-CIRC\_\-x\_\-USER} l'identification de l'utilisateur,
  \var{ISDN\_\-CIRC\_\-x\_\-PASS} le mot de passe.

  ATTENTION~: pour un accès au FAI T-Online (pour l'Allemagne) il est à noter~:

  Le nom d'utilisateur AAAAAAAAAAAATTTTTT\#MMMM est composé, du numéro
  co-utilisateur, puis du numéro T-online de 12 chiffres et de
  l'identification. Le dernier chiffre du numéro T-Online doit se terminer
  par '\#' si le numéro de T-Online ne comporte pas les 12 chiffres.

  Avec ca si cela ne fonctionne pas! (évidemment cela peut provenir du centrale
  téléphonique), le caractère '\#' doit être placé entre le numéro T-Online
  et l'identification.

  Evidemment si le (numéro T-Online comporte les 12 chiffres) il n'y a pas
  besoin de mettre le caractère '\#'.

  Exemple~: \var{ISDN\_\-CIRC\_\-1\_\-USER}='123456\#123'

  Avec le type de Circuit-Raw-IP ces variables n'ont aucune signification.

\config {ISDN\_CIRC\_x\_ROUTE\_N}{ISDN\_CIRC\_x\_ROUTE\_N}{ISDNCIRCxROUTEN}

  Dans cette variable on configure le nombre de réseau a utiliser pour le
  circuit ISDN. Si vous avez qu'un routage par Défaut vous devez placez '1'.

\config {ISDN\_CIRC\_x\_ROUTE\_X}{ISDN\_CIRC\_x\_ROUTE\_X}{ISDNCIRCxROUTEx}

  Dans cette variable vous indiquez le réseau ou les réseaux de routages pour
  le circuit ISDN. ici est enregistré pour la première variable tous les réseaux
  0.0.0.0/0 (default route) ou (routage par défaut). le format est toujours
  'network/netmaskbits' par exemple pour un réseau on écrira~: '192.168.199.1/32'.
  Une société ou une université qui à plusieurs réseaux et veulent se connecter
  au routeur par exemple pour un accès Internet on configurera la variable~:

\begin{example}
\begin{verbatim}
        ISDN_CIRC_%_ROUTE_N='2'
        ISDN_CIRC_%_ROUTE_1='192.168.8.0/24'
        ISDN_CIRC_%_ROUTE_2='192.168.9.0/24'
\end{verbatim}
\end{example}

  Lorsque vous avez plusieurs réseaux, vous devez paramétrer chaque réseau dans
  une variable ISDN\_CIRC\_x\_ROUTE\_y='' une ligne par réseau.

  Si vous voulez utiliser LC-Routing-Features de fli4l, on peut configurer
  *plusieurs* réseaux comme Default-Route (ou routage par défaut). On pourra
  paramétrer un des circuits à utiliser pour le LC- dans la variable 
  \var{ISDN\_\-CIRC\_\-x\_\-TIMES} voir ci-dessous.

\config {ISDN\_CIRC\_x\_DIALOUT}{ISDN\_CIRC\_x\_DIALOUT}{ISDNCIRCxDIALOUT}

  Dans cette variable \var{ISDN\_\-CIRC\_\-x\_\-DIALOUT} on indique ici le numéro
  de téléphone par exemple celui du FAI. Il est possible d'indiquer plusieurs
  Numéros de Tél. (au cas où si l'un est occupé)~- On sépare les numéros par un
  espace. Selon un rapport des newsgroup il est possible d'indiquer un maximun
  de cinq numéros de Tél.

\config {ISDN\_CIRC\_x\_DIALIN}{ISDN\_CIRC\_x\_DIALIN}{ISDNCIRCxDIALIN}

  Dans cette variable \var{ISDN\_\-CIRC\_\-x\_\-DIALIN} on indique le numéro
  de Tél. personnel, si le circuit-ISDN est utilisé pour les appels Tél. -
  avec son indicatif, mais *sans* le premier 0. Par rapport au raccordement
  téléphonique derrière l'installation cela ne peut en être autrement,
  éventuellement si le premier voire le deuxième sont des numéros principaux.

  Si le circuit le permet vous pouvez indiquer plusieurs numéros de correspondant
  ces numéros seront séparés par un espace. Le mieux est de rajouter un circuit
  par correspondant. Autrement cela pourrait à la connexion appeler les deux
  correspondants et créer des collisions bzgl sur le même circuit (mais c'est
  tout à fait possible avec l'ouverture des 2 canaux ISDN). C'est comme les
  adresses IP.

  Si vous n'arrivez pas à avoir votre correspondant, vous pouvez essayer de mettre
  le '0' avant le numéro. Mais prudence~: c'est permis uniquement, si le numéro
  d'appel ne peut pas être transmis!

  Si on voulait réaliser une connexion indépendamment de MSN (ou Multiple numéros
  d'abonnés) et du correspondant, vous pouvez placer comme paramètre le caractère '*'.

  Dans les deux derniers cas, une procédure d'authentification est indispensable
  (voir \var{ISDN\_CIRC\_x\_AUTH})

\config {ISDN\_CIRC\_x\_CALLBACK}{ISDN\_CIRC\_x\_CALLBACK}{ISDNCIRCxCALLBACK}

  Réglage et processus Callback (ou rappel Téléphonique), valeurs possibles~:

  \begin{tabular}[h!]{ll}
        'in' &     fli4l appel et rappel Tél.\\
        'out' &    fli4l appel, puis raccoche, attent et rappel de nouveau\\
        'off' &    pas de Callback (rappel Tél.)\\
        'cbcp' &   CallBack Control Protocol\\
        'cbcp0' &  CallBack Control Protocol 0\\
        'cbcp3' &  CallBack Control Protocol 3\\
        'cbcp6' &  CallBack Control Protocol 6\\
  \end{tabular}

  Les protocoles de Contrôles CallBack (aussi appelé 'Microsoft CallBack'),
  le plus souvent utilisé est le protocole cbcp6.

  Paramètre par défaut~: 'off'

\config {ISDN\_CIRC\_x\_CBNUMBER}{ISDN\_CIRC\_x\_CBNUMBER}{ISDNCIRCxCBNUMBER}

  Ici, on peut placer un numéro de rappel pour utilisation les protocoles
  cbcp, cbcp3 et cbcp6 (si on utilise cbcp3 le numéro est obligatoire).

\config {ISDN\_CIRC\_x\_CBDELAY}{ISDN\_CIRC\_x\_CBDELAY}{ISDNCIRCxCBDELAY}

  Dans cette variable on paramètre le délais en seconde du Callback (ou rappel Tél.).
  Selon le paramètre du rappel Tél. qui doit résulter, cette variable a une
  autre signification~:

  \begin{itemize}
  \item  \var{ISDN\_\-CIRC\_\-x\_\-CALLBACK}='in'~:

    fli4l appel et rappel le numéro si occupé, on indique ici
    \var{ISDN\_\-CIRC\_\-x\_\-CBDELAY} le temps en seconde entre les deux
    appels Tél. La bonne valeur est \linebreak \var{ISDN\_\-CIRC\_\-x\_\-CBDELAY}='3'
    pour rappeler le correspondant. Avec un valeur plus petite cela peu aussi
    fonctionner, la connexion téléphonique sera accélérer.

  \item \var{ISDN\_\-CIRC\_\-x\_\-CALLBACK}='out'~:

    Dans ce cas fli4l appel et raccroche si occupé, on indique ici
    \var{ISDN\_\-CIRC\_\-x\_\-CBDELAY} le temps attente en seconde pour appeler
    de nouveau. Là encore, \var{ISDN\_\-CIRC\_\-x\_\-CBDELAY}='3' est une bonne
    valeur. Ce qui ma étonné~: a l'appel Téléphonique, il suffit de 3 secondes
    pour "faire sonner", et avant que la serveur téléphonique réponde a la connexion
    Tél. en ville. Cette valeur ne doit jamais être plus basse. En cas de doute~: Testez!
  \end{itemize}

  Si vous paramétrez cette variable sur \var{ISDN\_\-CIRC\_\-x\_\-CALLBACK}='off',
  la variable \linebreak \var{ISDN\_\-CIRC\_\-x\_\-CBDELAY} sera ignoré. De même,
  la variable Callback Control Protocol, n'aura pas d'importance.

\config {ISDN\_CIRC\_x\_EAZ}{ISDN\_CIRC\_x\_EAZ}{ISDNCIRCxEAZ}

  Dans cette variable on paramètre l'indicatif régional, dans notre exemple MSN
  (l'appel du EAZ) est le 81330. Avec votre configuration MSN (ou Multiple numéros
  d'abonnés), vous avez peut être *pas* besoin d'indicatif à paramétrer.

  Seul une ligne direct est configurer le plus souvent derrière une installation
  téléphonique sans indicatif régional. Cependant indiquer un '0' comme valeur peut
  aider, si vous avez des problèmes avec votre installation téléphonique. Avoir des
  remarques sur cette variable serais appréciable.

\config {ISDN\_CIRC\_x\_SLAVE\_EAZ}{ISDN\_CIRC\_x\_SLAVE\_EAZ}{ISDNCIRCxSLAVEEAZ}

  Si le routeur fli4l est équipé d'un Bus-S0 interne pour un deuxième numéro de
  téléphone et si l'on utiliser l'agrégation des canaux, le deuxième numéros est
  a indiquer ici. Numéro du Tél. de diffusion sur le canal esclave.

  Donc normalement cette variable peut rester vide.

\config {ISDN\_CIRC\_x\_DEBUG}{ISDN\_CIRC\_x\_DEBUG}{ISDNCIRCxDEBUG}

  Pour avoir des informations débogage supplémentaire par exemple sur ipppd, vous devez
  activez la variable \var{ISDN\_\-CIRC\_\-x\_\-DEBUG} régler celle-ci sur 'yes'.
  Ces informations supplémentaire sur ipppd seront enregistrées avec interface syslogd

  IMPORTANT~: pour que le démon syslogd fonctionne, il faut activer la variable
  \var{OPT\_\-SYSLOGD} régler celle-ci sur 'yes'
  (Voir \jump{OPTSYSLOGD}{OPT\_SYSLOGD~- Enregistrement des messages erreur système}).\\
  Il faut aussi activer klog pour déboguer certains messages par exemple sur ISDN,
  il faut activer la variable \var{OPT\_\-KLOGD} régler celle-ci sur 'yes'
  (Voir \jump{OPTKLOGD}{OPT\_KLOGD~- Kernel-Message-Logger}).

  Avec le Circuit-Raw-IP la variable \var{ISDN\_\-CIRC\_\-x\_\-DEBUG} n'a pas d'importance.

\config {ISDN\_CIRC\_x\_AUTH}{ISDN\_CIRC\_x\_AUTH}{ISDNCIRCxAUTH}

  Avec cette variable vous pouvez avoir une authentification PAP ou CHAP lors de la
  connexion et la communication avec votre correspondant, pour cela paramétrer
  \var{ISDN\_\-CIRC\_\-x\_AUTH} sur 'pap' ou 'chap'~- Et ensuite *nur*. Dans la
  plus par des cas vous pouvez laisser cette variable vide!

  La raison~: les fournisseurs accès internet refuse souvent l'authentification,
  là rejette même! L'exceptions confirment la règle, j'ai récemment lu dans
  le i4l-Mailingliste ...

  Bien sur, il faut entrer le nom d'utilisateur et mot de passe, dans les variables
  \var{ISDN\_\-CIRC\_\-x\_USER} et \var{ISDN\_\-CIRC\_\-x\_PASS} 
  pour l'utilisation.

  Avec le circuits-Raw-IP, cette variable n'a aucune importance.

\config {ISDN\_CIRC\_x\_HUP\_TIMEOUT}{ISDN\_CIRC\_x\_HUP\_TIMEOUT}{ISDNCIRCxHUPTIMEOUT}

  Avec cette variable \var{ISDN\_\-CIRC\_\-x\_\-HUP\_\-TIMEOUT} on paramètre
  le temps après lequel ordinateur fli4l doit se déconnecter du fournisseur
  d'accès, s'il n'y a aucune transmission sur le réseau. Dans notre exemple
  la déconnexion Idle-Time se fait après 40 secondes, pour économiser de l'argent.
  S'il y a de nouveau une transmission sur le réseau la connexion se rétablira
  en une fraction de seconde. Il faut aussi que le FAI calcul à la seconde près!

  Il faudrait, au moins dans la phase de test de choisir/la déconnexion
  automatique du routeur fli4l (soit sur la console ou soit avec le client
  imonc pour Windows), et de vérifier, si vous avez pas une configuration
  défectueuse du raccordement ISDN.

  Si vous avez paramétré la valeur '0' aucun temps Idle-Time n'est pris en
  compte, c.-à-d. que fli4l ne se déconnectera plus de lui-même. S'il vous
  plaît appliquer cette valeur avec prudence.

\config {ISDN\_CIRC\_x\_CHARGEINT}{ISDN\_CIRC\_x\_CHARGEINT}{ISDNCIRCxCHARGEINT}

  On utilise cette variable pour placez un espace temps~: on paramètre ici le
  coût par unité téléphonique en seconde. Pour avoir le prix total des communications.

  En Allemagne les plus grands FAI facture l'unité Tél. exactement à la minute, le
  paramètre correct dans la variable est donc '60'. Compuserve facture l'unité
  toute les 3 minutes (juin 2000), on paramètre alors la variable
  \linebreak \var{ISDN\_\-CIRC\_\-x\_\-CHARGEINT}='180'. Certain FAI facture l'unité
  exactement à la seconde (par ex. Planet-Interkom) dans ce cas la variable
  \var{ISDN\_\-CIRC\_\-x\_\-CHARGEINT} sera de '1'.

  La variable est à \var{ISDN\_\-CIRC\_\-x\_\-CHARGEINT} $>$= 60 Secondes~:

  Cette variable \var{ISDN\_\-CIRC\_\-x\_\-HUP\_\-TIMEOUT} se paramètre en seconde
  pour la coupure de la connexion si aucun trafic. Raccroche 2 secondes avant la fin
  de la pulsation téléphonique. Le calcul du temps d'accès sera donc presque entièrement
  exploités. Une fonctionnalité vraiment fantastique de isdn4linux!

  Si la facturation de l'unité est calculé à la seconde, bien sûr cette variable n'a pas
  de sens~- Donc cette règle ne s'applique qu'à partir de 60 secondes par unité de Tél.

\config {ISDN\_CIRC\_x\_TIMES}{ISDN\_CIRC\_x\_TIMES}{ISDNCIRCxTIMES}

  Dans cette variable on paramètre temps activation et l'arrêt de la connexion,
  aussi le prix de l'unité Tél. Il est possible d'activer des circuits 'Default-Route'
  différents et aussi l'utilisation de (Least-Cost-Routing). contrôle la route
  affectation avec le démon (ou programme) imond.

  Structure de la variable~:

\begin{example}
\begin{verbatim}
        ISDN_CIRC_x_TIMES='times-1-info [times-2-info] ...'
\end{verbatim}
\end{example}


  Il y a dans chaque times-?-info 4 sous-paramètres~- Cet sous-paramètres sont séparés
  par deux points (':').
  \begin{enumerate}
  \item Sous-paramètre~: W1-W2

    On indique ici les périodes des jours ouvrables, par ex. Mo-Fr ou Sa-Su, il est
    possible décrire les jours en Anglais ou en Allemand. Si l'on paramètre un seul
    jour, il sera écrit W1-W1 par ex. Su-Su.

  \item Sous-paramètre~: hh-hh

    On indique ici la période horaire, par ex. 09-18 ou 18-09. De 18-09 est synonyme
    à 18-24 plus 00-09. de 00-24 correspond à toute une journée.

  \item  Sous-paramètre~: Charge

    On indique ici le prix par minute de connexion ou par unité téléphonique en euro,
    par ex. 0.032 correspond à 3.2 Centimes par minute. Les unités téléphonique, sont
    calculées en tenant compte du temps de conversion pour un coût réel, et seront
    alors affiché dans le client-imonc.

  \item  Sous-paramètre~: LC-Default-Route

    Le contenu de ce sous-paramètre peut être Y ou N. cela signifie~:

    \begin{itemize}
    \item Y~: autorise la plage horaire et LC-Routing (ou calcul des frais)
      avec Default-Route (ou routage par défaut). Important~: Dans ce cas,
      il faut aussi que la variable soit réglée \var{ISDN\_\-CIRC\_\-x\_\-ROUTE}='0.0.0.0/0'
      comme ceci!

    \item N~: autorise la plage horaire et le calcul des frais Tél automatiquement
      avec LC-Routing, il n'est pas utilisé pour autre chose.
    \end{itemize}
  \end{enumerate}

    Exemple~:

\begin{small}
\begin{example}
\begin{verbatim}
    ISDN_CIRC_1_TIMES='Mo-Fr:09-18:0.049:N Mo-Fr:18-09:0.044:Y Sa-Su:00-24:0.044:Y'
    ISDN_CIRC_2_TIMES='Mo-Fr:09-18:0.019:Y Mo-Fr:18-09:0.044:N Sa-Su:00-24:0.044:N'
\end{verbatim}
\end{example}
\end{small}

    Interprétation de l'exemple ci-dessus~: Circuit 1 (FAI Planet-Interkom) est utilisé
    le soir des jours ouvrables et toute la journée en fin de semaine, mais durant la
    journée les jours ouvrables de la Circuit 2 (Provider Compuserve) est utilisé.

    \begin{description}
    \item \wichtig{les paramètres de la variable \var{ISDN\_CIRC\_x\_TIMES} doit
     couvrir toute la semaine. si ce n'est pas le cas, aucune connexion valide ne
     peut se produire.}

     \emph{Si vous avez placé le paramètre ("Y") pour LC-Default-Route-Circuits et que vous
     n'avez pas réglé la semaine complète, il y aura des interruptions dans la période
     de la semaine avec Default-Route. Alors il sera impossible de surfer sur Internet
     pendant cet périodes!}

    \item Exemple~:
\begin{example}
\begin{verbatim}
    ISDN_CIRC_1_TIMES='Sa-Su:00-24:0.044:Y Mo-Fr:09-18:0.049:N Mo-Fr:18-09:0.044:N'
    ISDN_CIRC_2_TIMES='Sa-Su:00-24:0.044:N Mo-Fr:09-18:0.019:Y Mo-Fr:18-09:0.044:N'
\end{verbatim}
\end{example}

     Dans cette exemple les jours ouvrables de 18-09 Heure son paramétrés sur "N".
     Il n'y a pas de défaut route pour Internet~: le surf est interdit!

    \item Encore un exemple simple~:

\begin{example}
\begin{verbatim}
      ISDN_CIRC_1_TIMES='Mo-Su:00-24:0.0:Y'        
\end{verbatim}
\end{example}

      Cette exemple est pour ceux qui utilise un Flatrate (ou forfait d'accès
      internet illimité).

    \item Encore une derrière remarque pour le LC-Routing~:

      Les jours fériés allemands sont traités comme un dimanche.
    \end{description}

\end{description}

\subsection {OPT\_TELMOND~- Configuration telmond}
\configlabel{OPT\_TELMOND}{OPTTELMOND}

On utilise la variable \var{OPT\_\-TELMOND} pour activer le serveur-telmond.
Il écoute via le port TCP 5001 les appels téléphonique entrant et enregistre
les informations, numéro de Tél. et le nom du correspondant. Ces informations
pourront être visualisées avec le client imonc sous Windows et Unix/Linux
(voir le chapitre "Client-/Interface-Serveur imond").

Condition impérative~: avoir installé une carte ISDN (carte numéris), et
avoir configuré les variables du paquetage \var{OPT\_\-ISDN}.

On peut vérifier le fonctionnement en cours de telmond sous Linux/Unix/Windows
avec la commande~:

\begin{example}
\begin{verbatim}
        telnet fli4l 5001
\end{verbatim}
\end{example}

Vous devez voir le dernier appel Tél. puis vous verrez la liaison-telnet se fermer.

Le port 5001 est uniquement accessible depuis le LAN (réseau local). Par défaut
la configuration du Firewall bloque l'accés de ce port de l'extérieur. Si vous
voulez modifier accés du port pour le réseau LAN, cela est possible utilisé la
variable de configuration de telmonc, voir ci-dessous.

Configuration par défaut~: \var{START\_\-TELMOND}='yes'

\begin{description}
\config {TELMOND\_PORT}{TELMOND\_PORT}{TELMONDPORT}

  Telmond écoute via le Port-TCP/IP. La valeur par défaut est '5001' et devrait
  être modifié seulement dans certains cas exceptionnels.

\config {TELMOND\_LOG}{TELMOND\_LOG}{TELMONDLOG}

  Si on paramètre la variable sur \var{TELMOND\_\-LOG}='yes' l'ensemble des appels
  téléphonique seront sauvegardés dans le fichier /var/log/telmond.log. Le contenu
  du fichier d'imond peut être disponible avec le Client-imonc sous Unix/Linux
  et Windows.

  Vous pouvez configurez des chemin différent pour le fichier log, voir ci-dessous.

  Configuration par défaut~: \var{TELMOND\_\-LOG}='no'

\config {TELMOND\_LOGDIR}{TELMOND\_LOGDIR}{TELMONDLOGDIR}

  Si vous avez activé les sauvegarde-log, vous pouvez paramétrer cette variable
  \var{TELMOND\_\-LOGDIR} et configurer un autre répertoire d'origine /var/log, Par ex.
  '/boot'. Alors, les fichiers LOG de telmond.log serons sauvegardés sur un
  support de boot, il faut qu'il soit "monté" et paramétré en Read/Write.
  Si vous indiquez 'auto', le fichier log sera enregistré en fonction de la
  configuration dans /boot/persistant/isdn ou un autre chemin d'accès spécifique
  avec la variable \var{FLI4L\_UUID}. Si le chemin /boot n'est pas monté en
  lecture/écriture, le fichier log sera créé dans le répertoire /var/run.

\config {TELMOND\_MSN\_N}{TELMOND\_MSN\_N}{TELMONDMSNN}

  Ici on peut filtrer les appels téléphonique uniquement pour certain PC client
  il seront visibles dans imonc, pour chaque appel spécial, le MSN (ou le numéros
  de téléphones internes) sera suivi du protocole PC client.

  Si c'est nécessaire, Par ex. une collocation du Bat, vous pouvez paramétrer le
  nombre de filtre MSN dans la variable \var{TELMONS\_\-MSN\_\-N}.

  Configuration par défaut~: \var{TELMOND\_\-MSN\_\-N}='0'

\config {TELMOND\_MSN\_x}{TELMOND\_MSN\_x}{TELMONDMSNx}

  Vous devez indiquer pour chaque Filtre-MSN (ou le numéros de téléphones internes)
  une adresse IP, les appels seront ainsi enregistrés et visibles.

  Si la variable \var{TELMOND\_\-MSN\_\-N} est configurée avec le nombre de filtre MSN.

  Stucture de la variable~:
\begin{example}
\begin{verbatim}
        TELMOND_MSN_x='MSN IP-ADDR-1 IP-ADDR-2 ...'
\end{verbatim}
\end{example}

  Exemple simple~:

\begin{example}
\begin{verbatim}
        TELMOND_MSN_1='123456789 192.168.6.2'
\end{verbatim}
\end{example}

  Si vous voulez un appel MSN (ou Tél. interne) sur plusieurs ordinateurs visible,
  par ex. envoyer un Fax sur plusieurs PCs, écrire les adresses-IP et les séparer
  par un espace, exemple~:

\begin{example}
\begin{verbatim}
        TELMOND_MSN_1='123456789 192.168.6.2 192.168.6.3'
\end{verbatim}
\end{example}

\config {TELMOND\_CMD\_N}{TELMOND\_CMD\_N}{TELMONDCMDN}

  Dès qu'un appel téléphonique MSN entre, certaines commandes facultatif peuvent
  être exécutées sur le routeur fli4l. On configure ici le nombre de commande dans
  la variable \var{TELMOND\_\-CMD\_\-N}.

\config {TELMOND\_CMD\_x}{TELMOND\_CMD\_x}{TELMONDCMDx}

  Avec la variable \var{TELMOND\_\-CMD\_\-1} bis \var{TELMOND\_\-CMD\_\-n} on peut
  configurer les commandes, elles seront exécutées si un appel téléphonique entre.

  Si la variable \var{TELMOND\_\-CMD\_\-N} est configurée avec le nombre de commande.

  Stucture de la variable~:

\begin{example}
\begin{verbatim}
        MSN CALLER-NUMBER COMMAND ...
\end{verbatim}
\end{example}

  Le numéro MSN doit être paramétré sans séparé l'indicatif. A la place de
  CALLER-NUMBER on indique le numéro de Tél. complet~- C'est-à-dire le numéro de
  téléphone avec l'indicatif. Si on écrit à la place de CALLER-NUMBER le caractère
  astérisque (*), telmond n'exploite aucun numéro de téléphone du correspondant.

  Voir l'exemple~:

\begin{example}
\begin{verbatim}
        TELMOND_CMD_1='1234567 0987654321 sleep 5; imonc dial'
        TELMOND_CMD_2='1234568 * switch-on-coffee-machine'
\end{verbatim}
\end{example}

  Dans le premier cas, la commande "sleep 5; imonc dial" est exécuté si le
  correspondant avec numéro de Tél 0987654321 appel numéro MSN 1234567.
  \-En \-fait il y a 2 commandes. Tout d'abord, on attend 5 secondes, de sorte à
  libérer canal ISDN, sur lequel l'appel Téléphonique entrera. Ensuite, le Client imonc
  de fli4l démarre avec l'argument "dial". Imonc transmet la commande 1:1 sur le
  serveur imond lequel produit une connexion réseau par défaut, par ex. sur Internet.
  Quelles sont les autres commandes que le programme Client imonc peut envoyer
  vers le serveur imond, elles sont décrient dans le chapitre "Interface
  Client-/Serveur imond". Pour que cette option fonctionne, il faut installer
  \var{OPT\_IMONC} dans le paquetage "tools".

  Dans le deuxième cas la commande "switch-on-coffee-machine" est exécuté, si un appel
  MSN 1234568 entre, quel que soit, d'où l'appel provient. Na\-turelle\-ment
  la commande "switch-on-coffee-machine" ne fonctionne pas encore avec fli4l!

  lors d'une commande vous pouvez utiliser les jokers suivants~:

  \begin{tabular}[h!]{cll}
  \hline
            \%d   &   date    &    Date \\
            \%t   &   time    &    Heure \\
            \%p   &   phone   &    Numéro de Tél du correspondant \\
            \%m   &   msn     &    MSN spécifique \\
            \%\%  &   percent &    Pourcentage \\
  \end{tabular}

  Ces données peuvent ensuite être utilisées par un programme, par exemple
  envoyer par \mbox{E-Mail}.

\config {TELMOND\_CAPI\_CTRL\_N}{TELMOND\_CAPI\_CTRL\_N}{TELMONDCAPICTRLN}

  Si vous utilisez un adaptateur ISDN sous CAPI ou un CAPI distant du
  (type 160 ou 161), il sera peut être nécessaire de configurer le contrôleur
  CAPI pour que telmond écoute des appels de façon plus explicite. Par exemple,
  la Fritz!Box offre un accès avec un maximum de cinq contrôleurs différents
  qui ne peuvent même pas être différenciés (voir les informations sur
  \altlink{http://www.wehavemorefun.de/fritzbox/CAPI-over-TCP\#Virtuelle_Controller}).
  Pour limiter le nombre de contrôleurs à utiliser vous pouvez définir la quantité,
  dans le tableau les variables suivantes \var{TELMOND\_CAPI\_CTRL\_\%} vous
  pouvez réglé les contrôleurs qui doivent être utilisés.

  Si vous n'utilisez pas la variable telmond pour écouter sur \emph{tous} les
  contrôleurs CAPI disponible.

\config {TELMOND\_CAPI\_CTRL\_x}{TELMOND\_CAPI\_CTRL\_x}{TELMONDCAPICTRLx}

  Si la variable \var{TELMOND\_CAPI\_CTRL\_N} est égal à zéro, l'indice pour
  les contrôleurs CAPI doit être spécifié pour que telmond surveiller les
  appels entrants.

   Exemple pour un CAPI distant et avec une Fritz!Box pour une "réel" connexion ISDN~:

\begin{example}
\begin{verbatim}
        TELMOND_CAPI_CTRL_N='2'
        TELMOND_CAPI_CTRL_1='1' # listen to incoming ISDN calls
        TELMOND_CAPI_CTRL_2='3' # listen to calls on the internal S0-Bus
\end{verbatim}
\end{example}

  Exemple pour un CAPI distant avec une Fritz!Box pour une connexion analogique
  et SIP-Forwarding~:

\begin{example}
\begin{verbatim}
        TELMOND_CAPI_CTRL_N='2'
        TELMOND_CAPI_CTRL_1='4' # listen to incoming analog calls
        TELMOND_CAPI_CTRL_2='5' # listen to incoming SIP-calls
\end{verbatim}
\end{example}
\end{description}

\subsection{OPT\_RCAPID~- Le démon CAPI distant}
\configlabel{OPT\_RCAPID}{OPTRCAPID}

  Avec cette OPT vous pouvez configurer le programme rcapid sur le routeur fli4l
  qui offre un accès à Interface par le ISDN sous CAPI via des routeurs sur
  le réseau. Les outils appropriés peuvent accéder sur la carte ISDN du routeur
  via le réseau comme s'il était installé localement. Ceci est similaire au
  paquetage "mtgcapri". La différence est que les systèmes Windows peuvent
  utiliser "mtgcapri" comme un client alors que l'interface réseau de rcapid
  supporte seulement les systèmes Linux au moment de l'écriture. Ainsi, les
  deux paquetages sont complémentaires idéaux dans les environnements mixtes
  Windows et Linux.

\subsubsection{Configuration du routeur}

\begin{description}
\config{OPT\_RCAPID}{OPT\_RCAPID}{OPTRCAPID}{
  Cette variable permet d'activer un ISDN sous CAPI sur le routeur pour les
  clients distants. Les valeurs possibles sont "yes" et "no". Si la valeur est
  sur "yes", si le démon inetd sur Internet est configuré, si les demandes de
  requêtes rcapid sur le port 6000 fonctionne, alors le démon rcapid démarre
  (peut être modifié en utilisant la variable \var{RCAPID\_PORT}).

Exemple~:
}
\verb*?OPT_RCAPID='yes'?

\config{RCAPID\_PORT}{RCAPID\_PORT}{RCAPIDPORT}{
  Cette variable contient le port TCP qui est utilisé par le démon rcapid.

Configuration par défaut~:
}
\verb*?RCAPID_PORT='6000'?
\end{description}

\subsubsection{Configuration du client Linux}

  Pour utiliser l'interface CAPI distant sur un PC Linux vous devez utiliser 
  le module de bibliothèque libcapi20. Une telle bibliothèque CAPI se touve
  dans les derrières distributions Linux (par ex. Debian Wheezy). Sinon vous
  devez télécharger les sources à partir du lien 
  \altlink{http://ftp.de.debian.org/debian/pool/main/i/isdnutils/isdnutils_3.25+dfsg1.orig.tar.bz2}.
  Après le dépaquetage et le chargement dans le répertoire "capi20" de la
  bibliothèque CAPI, il peut être compilé après les trois étapes "configure"
  "make" et "sudo make install" comme d'habitude. Lorque la bibliothèque est
  installé le fichier de configuration \texttt{/etc/capi20.conf} doit être paramétré
  pour spécifier le client sur lequel rcapid tourne. Par exemple si le routeur est
  accessible par le nom "fli4l" le fichier conf se présentera comme ceci~:

\begin{example}
\begin{verbatim}
REMOTE fli4l 6000
\end{verbatim}
\end{example}

C'est tout~! Pour le client Linux, le programme "capiinfo" est installé (fait
parte du paquetage capi4k-utils de nombreuses distributions), vous pouvez tester
immédiatement l'interface CAPI distant~:

\begin{example}
\begin{verbatim}
kristov@peacock ~ $ capiinfo 
Number of Controllers : 1
Controller 1:
Manufacturer: AVM Berlin
CAPI Version: 1073741824.1229996355
Manufacturer Version: 2.2-00  (808333856.1377840928)
Serial Number: 0004711
BChannels: 2
[...]
\end{verbatim}
\end{example}

Dans "Number of Controllers" la quantité de cartes RNIS est affiché qui peuvent
être utilisés par le client. Si vous lisez "0" la connexion au programme rcapid
fonctionne mais la carte RNIS n'est pas reconnu sur le routeur. Si la connexion
au programme rcapid ne fonctionne pas du tout (peut-être la variable
\var{OPT\_RCAPID} est sur "no"), un message d'erreur "capi not installed~-
Connection refused (111)" sera affiché. Dans ce cas, recontrôler votre configuration.

