% Synchronized to r50201
\section {ISDN - Communication Over Active And Passive ISDN-Cards}
\configlabel{OPT\_ISDN}{OPTISDN}

fli4l is mainly aiming to be used as an ISDN- and/or DSL-router. 
By setting \var{OPT\_\-ISDN='yes'} the ISDN package is activated. 
Precondition is a ISDN-card supported by fli4l.
    
Default setting: \var{OPT\_\-ISDN='no'}


\subsection {Establishing An ISDN Connection}
\begin{sloppypar}
fli4l's behaviour during dial-in is determined by three variables
\mbox{\var{DIALMODE},} \var{ISDN\_\-CIRC\_\-X\_\-ROUTE\_X},
\var{ISDN\_\-CIRC\_\-X\_\-TIMES}. \jump{DIALMODE}{\var{DIALMODE}} 
(in $<$config$>$/base.txt) determines wether a connection will be 
automatically established on an active circuit on packet arrival 
or not. \var{DIALMODE} may have the following values:
\end{sloppypar}

\begin{description}
\item[auto] If a packet reaches an ISDN-circuit (res. the ISDN 
  interface derived from it - ippp*) a connection will be established 
  automatically. If and when a packet reaches an ISDN-circuit is determined 
  by \var{ISDN\_\-CIRC\_\-X\_\-ROUTE\_X} and \var{ISDN\_\-CIRC\_\-X\_\-TIMES}.
       
\item[manual] In manual mode the connection has to established via 
  imond/imonc. How this is done see imonc/imond.

 \item[off] No ISDN connections will be established.
\end{description}

Which circuits packets will trigger a dial-in is defined by 
\var{ISDN\_\-CIRC\_\-X\_\-ROUTE\_X}. Normally this uses '0.0.0.0/0' as the 
'default route'. This means that all packets that leave the local net are 
using this circuit if it is active. If and when it is active is determined 
by \var{ISDN\_\-CIRC\_\-X\_\-TIMES} for fli4l is doing \emph{least cost routing} 
over a predefinded circuit (see \jump{sec:leastcostrouting}{Least-Cost-Routing 
- Functionality} in the base documentation). If not all but only packets for 
a certain net should be routed over this circuit (i.e. a company net) additional 
nets can be given here. fli4l will then set a permanently active ISDN route over 
the interface set for this circuit. If a packet is sent to this net a connection 
will be automatically established.

As said before \var{ISDN\_\-CIRC\_\-X\_\-TIMES} besides the connection costs for 
a circuit describes also if and when a circuit with a default route is active 
and can trigger a connection. 'When' is defined by the time-info, the first 
two elements (i.e. Mo-Fr:09-18), 'if' is given by the forth parameter 
lc-default-route (y/n). fli4l (res. imond) will trigger a connection to the 
internet provider and assure that all packets leaving the local net are 
routed over the circuit that is active at this time. 

The standard use cases in summary: 

\begin{itemize}
\item If simply a connection to the internet is intended \var{DIALMODE} 
is set to auto, 1-n circuits are to be defined which have an initial route 
of '0.0.0.0/0' and whose times (times with lc-default-route = y) cover 
the whole week.
\begin{small}
\begin{example}
\begin{verbatim}
        ISDN_CIRC_%_ROUTE_N='1'
        ISDN_CIRC_%_ROUTE_1='0.0.0.0/0'
        ISDN_CIRC_%_TIMES='Mo-Su:00-24:0.0148:Y'
\end{verbatim}
\end{example}
\end{small}

\item If besides that a connection to a company net should be used another 
(or more) circuits have to be defined. The route should differ from '0.0.0.0/0' 
to accomplish a permanently active route.
\begin{small}
\begin{example}
\begin{verbatim}
        ISDN_CIRC_%_ROUTE_N='1'
        ISDN_CIRC_%_ROUTE_1='network/netmaskbits'
        ISDN_CIRC_%_TIMES='Mo-Su:00-24:0.0148:Y'
\end{verbatim}
\end{example}
\end{small}
\end{itemize}

\subsection{ISDN Card}

The fli4l router generally supports the use of multiple ISDN cards
simultaneously. However, this requires all ISDN cards to be driven by the
\emph{same driver type}. The driver type can be derived from the group in the
table below the driver in question is part of. Consequently, it is e.~g.\ no
problem to use multiple mISDN-driven adapters or multiple HiSax-driven adapters
(as long as sufficient resources are available), but it is not possible to use
an mISDN-driven card and a HiSax-driven card at the same time.

\begin{description}
\configlabel{ISDN\_\%\_IO}{ISDNxIO}
\configlabel{ISDN\_\%\_IO0}{ISDNxIO0}
\configlabel{ISDN\_\%\_IO1}{ISDNxIO1}
\configlabel{ISDN\_\%\_MEM}{ISDNxMEM}
\config{ISDN\_\%\_TYPE ISDN\_\%\_IO ISDN\_\%\_IO0 ISDN\_\%\_IO1 ISDN\_\%\_MEM ISDN\_\%\_IRQ} {ISDN\_\%\_TYPE} {ISDNxTYPE}

  Some technical data about the ISDN card is specified here.
  
  The values in the example work for a TELES 16.3 set to IO-address 0xd80 
  via Dip-switches. For other settings of the switches the values have 
  to be changed.

  \achtung{Common error (example):}
  \begin{example}
  \begin{verbatim}
          ISDN_1_IO='240' -- right value would be: ISDN_1_IO='0x240'
  \end{verbatim}
  \end{example}

  Using IRQ 12 eventually a PS/2 mouse has to be deactivated in 
  BIOS. Better choose another IRQ!
  \glqq{}Good ones\grqq{} are mostly 5, 10 and 11.
  
  \var{ISDN\_\%\_\-TYPE} in principle follows the type numbers for HiSax 
  drivers. Exception: non-HiSax-cards like i.e. the AVM-B1. For those 
  the type numbers were extended (see below). The list of possible 
  HiSax-types is based on \\
  \var{linux-2.x.y/Documentation/isdn/README.HiSax}.

\begin{small}
  \begin{longtable}{|r|p{60mm}|p{62mm}|}
    \hline
    \multicolumn{1}{|c}{\textbf{Typ}} & \multicolumn{1}{|c}{\textbf{Karte}} &
    \multicolumn{1}{|c|}{\textbf{Needed parameters}} \\
    \hline\hline
    \endhead

    \multicolumn{3}{|l|}{Dummy Type-Number:} \\
    \hline
      0 &  no driver (dummy)             & none \\

    \hline\hline
    \multicolumn{3}{|l|}{Type numbers for remote CAPI drivers:} \\
    \hline

    160 & AVM Fritz!Box Remote CAPI   & ip,port \\
    161 & Melware Remote CAPI (rcapi) & ip,port \\

    \hline\hline
    \multicolumn{3}{|l|}{Type numbers for mISDN-drivers:} \\
    \hline

    301 & HFC-4S/8S/E1 multiport cards         & no parameter \\
    302 & HFC-PCI based cards                  & no parameter \\
    303 & HFCS-USB Adapters                    & no parameter \\
    304 & AVM FritZ!Card PCI (v1 and v2) cards & no parameter \\
    305 & cards based on Infineon (former Siemens) chips: \newline
          - Dialogic Diva 2.0 \newline
          - Dialogic Diva 2.0U \newline
          - Dialogic Diva 2.01 \newline
          - Dialogic Diva 2.02 \newline
          - Sedlbauer Speedwin \newline
          - HST Saphir3 \newline
          - Develo (former ELSA) Microlink PCI (Quickstep 1000) \newline
          - Develo (former ELSA) Quickstep 3000 \newline
          - Berkom Scitel BRIX Quadro \newline
          - Dr.Neuhaus (Sagem) Niccy           & no parameter \\
    306 & NetJet TJ 300 and TJ320 cards        & no parameter \\
    307 & Sedlbauer Speedfax+ cards            & no parameter \\
    308 & Winbond 6692 based cards             & no parameter \\
    \hline
  \end{longtable}
\end{small}
  
  My card is a Teles 16.3 NON-PNP ISA, this is Type=3.

  For a ICN-2B-card IO and MEM have to be set, for example 
  \var{ISDN\_1\_\-IO}='0x320', \var{ISDN\_1\_\-MEM}='0xd0000'.

  For newer Teles-PCI-card type=20 (instead of 21) has to be used. 
  Those are shown by ``cat /proc/pci'' as ``tiger'' or similar. 
  
  For ISDN type 303 it is necessary to activate USB support. 
  See \jump{sec:opt-usb }{USB - Support for USB-devices}.\\

  If you really don't know what card is in your PC you can get tips for type numbers 
  also from the i4l-FAQ or mailing list.

  Card types that are signed \glqq{}from isapnp setup\grqq{} have to be initialized 
  by the PnP tool isapnp - if they really are PnP cards. See 
  \jump{OPTPNP}{OPT\_PNP - Installation of isapnp tools}.

  ISDN type 0 is used if the ISDN package should be installed without an ISDN card, 
  for example to use imond in a network router.

\configlabel{ISDN\_\%\_IP}{ISDNxIP}
\configlabel{ISDN\_\%\_PORT}{ISDNxPORT}
\config{ISDN\_\%\_IP ISDN\_\%\_PORT} {ISDN\_\%\_IP} {ISDNxIP}

  For ISDN types 160 and 161 the variables (\var{ISDN\_\%\_IP}) and (\var{ISDN\_\%\_PORT})
  set IP address resp. port of the device offered by a remote CAPI interface.
  The IP address is mandatory while the port number may be omitted:
  Depending on the chosen type a standard port will be set (Type 160: 5031,
  Type 161: 2662).

  Example:
  \begin{example}
  \begin{verbatim}
          ISDN_1_TYPE='160' # AVM Fritz!Box
          ISDN_1_IP='192.168.177.1'
  \end{verbatim}
  \end{example}

\config{ISDN\_DEBUG\_LEVEL}{ISDN\_DEBUG\_LEVEL}{ISDNDEBUGLEVEL} 
  
  Sets the debug level for the HiSaX driver. Debug level is concatenated by addition 
  of the following values (cited from the orginal docs):\\
  
  \begin{tabular}[h!]{r|l}
   \multicolumn{1}{c|}{\textbf{Number}} & \multicolumn{1}{c}{\textbf{Debug-Information}} \\
   \hline
      1 & Link-level $<$--$>$ hardware-level communication \\
      2 & Top state machine \\
      4 & D-Channel Q.931 (call control messages) \\
      8 & D-Channel Q.921 \\
     16 & B-Channel X.75 \\
     32 & D-Channel l2 \\
     64 & B-Channel l2 \\
    128 & D-Channel link state debugging \\
    256 & B-Channel link state debugging \\
    512 & TEI debug \\
   1024 & LOCK debug in callc.c \\
   2048 & More debug in callc.c (not for normal use) \\
  \end{tabular}\latex{\\}

  The default setting (\var{ISDN\_DEBUG\_LEVEL}='31') should be enough for most purposes.
  
\config{ISDN\_VERBOSE\_LEVEL}{ISDN\_VERBOSE\_LEVEL}{ISDNVERBOSELEVEL}
  
  Sets the ``verbosity'' of the ISDN subsystem in fli4l kernel. Each verbose-level 
  includes levels with lower numbers. Verbose levels are:

  \begin{tabular}[h!]{lp{10cm}}
    '0' & no additional informations \\
    '1' & events triggering an ISDN connection will be logged\\
    '2' and '3' & Calls are logged\\
    '4' and more & Data transfer rates will be logged\\
        % Beim letzten bin ich mir nicht ganz sicher, siehe
        % linux-kernel-source/drivers/isdn/isdn_net.c, suche nach "dev->net_verbose > 3"
  \end{tabular}

  Messages are sent over the kernel logging interface activated by 
  \jump{OPTSYSLOGD}{\var{OPT\_SYSLOGD}}.

  \wichtig{If calls should be logged with telmond don't set this value lower 
  than 2 otherwise telmond would lack informations for logging.}
  
  Default setting: \var{ISDN\_VERBOSE\_LEVEL}='2'

\begin{sloppypar}
\config{ISDN\_FILTER}{ISDN\_FILTER}{ISDNFILTER}

Activates filtering mechanism of the kernel to achieve hangup after the 
specified hangup timeout. See 
\altlink{http://www.fli4l.de/hilfe/howtos/basteleien/hangup-problem-loesen/} 
for additional informations.
\end{sloppypar}

\config{ISDN\_FILTER\_EXPR}{ISDN\_FILTER\_EXPR}{ISDNFILTEREXPR}

Specifies the filter to use if \var{ISDN\_FILTER} is set to `yes'.

\end{description}


\subsection{OPT\_ISDN\_COMP (EXPERIMENTAL)}
\configlabel{OPT\_ISDN\_COMP}{OPTISDNCOMP}

\var{OPT\_\-ISDN\_\-COMP}='yes' activates LZS- and BSD-compression. Credits for 
this go to  Arwin Vosselman (\email{arwin(at)xs4all(dot)nl}). This addon package 
is in experimental state.

Default setting: \var{OPT\_\-ISDN\_\-COMP}='no'

The needed parameters for LZS-compression in detail:

\begin{description}

\config{ISDN\_LZS\_DEBUG (EXPERIMENTAL)}{ISDN\_LZS\_DEBUG}{ISDNLZSDEBUG}

  Debug-level-settings:

  \begin{tabular}[h!]{ll}
    '0' & no debugging informations \\
    '1' & normal debugging informations\\
    '2' & enhanced debugging informations\\
    '3' & extreme debugging informations (incl. dumping of data packets)\\
  \end{tabular}

  Default setting: \var{ISDN\_\-LZS\_\-DEBUG}='1'
  
\config{ISDN\_LZS\_COMP (EXPERIMENTAL)}{ISDN\_LZS\_COMP}{ISDNLZSCOMP}
  
  Compression level (not decompression!). Please use value 8. Values from 
  0 to 9 are possible.
  
  Higher numbers will compress better at the cost of higher CPU load with 9 
  being disproportional excessive.
  
  Default setting: \var{ISDN\_\-LZS\_\-COMP}='8'

\config{ISDN\_LZS\_TWEAK (EXPERIMENTAL)}{ISDN\_LZS\_TWEAK}{ISDNLZSTWEAK}

  Keep this value at '7' at the moment.

  Default setting: \var{ISDN\_\-LZS\_\-TWEAK}='7'
  
  Beside this three values the variable \var{ISDN\_\-CIRC\_\-x\_\-FRAMECOMP} 
  has to be set (see next chapter).

\end{description}




\subsection{ISDN-Circuits}

More connections over ISDN can be defined in fli4l configuration. A maximum of 
two at a time is possible over one ISDN card.

Definition of connections is done by so-called circuits. One circuit is used 
per connection.

In the config.txt example two circuits are defined:

\begin{itemize}
\item Circuit 1: Dialout over Internet-by-call provider Microsoft Network, Sync-PPP
  
\item Circuit 2: Dialin/Dialout to an ISDN-router (maybe another fli4l)
  over Raw-IP, i.e. as a connection to a company net somewhere.
  
\end{itemize}

If fli4l is simply used as an internet gateway only one circuit is needed. Exception: 
fli4l's least-cost features should be used. In this case define different circuits 
for all allowed timespans, see below.

\begin{description}

\config{ISDN\_CIRC\_N}{ISDN\_CIRC\_N}{ISDNCIRCN}
  
  Sets the number of used ISDN circuits. If fli4l is used only to monitor 
  incoming ISDN calls set:

\begin{example}
\begin{verbatim}
        ISDN_CIRC_N='0'
\end{verbatim}
\end{example}
  
  If fli4l is simply used as an internet gateway one circuit is enough. 
  Exception: LC-routing, see below.


\config{ISDN\_CIRC\_x\_NAME}{ISDN\_CIRC\_x\_NAME}{ISDNCIRCxNAME} 
  
  Set a name for the circuit - maximum length is 15 characters. 
  The imon client \texttt{imonc.exe} will show this instead of the 
  telephone number dialed. Possible characters are 'A' to 'Z' (Capitals 
  are possible), number '0' to '9' and hyphens '-'. Example:

\begin{example}
\begin{verbatim}
        ISDN_CIRC_x_NAME='msn'
\end{verbatim}
\end{example}

  This name can be used in the packet filter or with OpenVPN. If for example 
  the packet filter should control an ISDN circuit a 'circuit\_' has to prefix 
  the circuit name. If an ISDN circuit is called 'willi' the packet filter has 
  to be set like this:

\begin{example}
\begin{verbatim}
PF_INPUT_3='if:circuit_willi:any prot:udp 192.168.200.226 192.168.200.254:53 ACCEPT'
\end{verbatim}
\end{example}

\config{ISDN\_CIRC\_x\_USEPEERDNS}{ISDN\_CIRC\_x\_USEPEERDNS}{ISDNCIRCxUSEPEERDNS}

  This determines wether the name servers transferred by the internet provider 
  during dial-in should be filled in the configuration file of the local name 
  server for the duration of the connection. This only makes sense for circuits 
  used for connecting to an internet provider. Nearly all providers support 
  this name server tansfer.
  
  After name server IP addresses have been transferred name servers entries from 
  base.txt's \emph{\var{DNS\_\-FORWARDERS}} are removed from the configuration 
  file of the local name server and the transferred ones are filled in as 
  forwarders. After this the name server is forced to relaod is configuration. 
  The name server cache will be preserved and names already resolved are kept.
  
  This option has the advantage to work with the nearest possible name servers 
  at any time, as far as the provider transmits correct IP addresses - name 
  resolution is faster then.
  
  In case of failing DNS servers at the provider side transmitted DNS server
  addresses usually are corrected rapidly by the provider.
  
  After all it is absolutely necessary for the first dial-in to provide a valid 
  name server in base.txt's \emph{\var{DNS\_\-FORWARDERS}}. Otherwise the first 
  request can not be resolved correctly. In addition the initial configuration 
  of the name server will be restored on hangup.
  
  Default setting: \var{ISDN\_\-CIRC\_\-x\_\-USEPEERDNS}='yes'


\config{ISDN\_CIRC\_x\_TYPE}{ISDN\_CIRC\_x\_TYPE}{ISDNCIRCxTYPE}
  
  \var{ISDN\_\-CIRC\_\-x\_\-TYPE} specifies the type of connection x. Possible 
  values are:

  \begin{tabular}[h!]{ll}
        'raw' &           RAW-IP\\
        'ppp' &           Sync-PPP\\
  \end{tabular}
  
  In most cases PPP is used, Raw-IP is a little more efficient because 
  of the missing PPP overhead. Authentification is not possible with 
  Raw-IP but with variable \var{ISDN\_\-CIRC\_\-x\_\-DIALIN} (see below) 
  limitiations to explicit ISDN numbers (``Clip'') can be accomplished. 
  If \var{ISDN\_\-CIRC\_\-x\_\-TYPE} is set to'raw' /etc/ppp a raw up/down 
  script will be executed in analog to the PPP up/down scripts.

\config{ISDN\_CIRC\_x\_BUNDLING}{ISDN\_CIRC\_x\_BUNDLING}{ISDNCIRCxBUNDLING}
  
  For ISDN channel bundeling the MPPP protocol according to RFC 1717 is used. 
  This inherits the following mostly irrelevant limitiations:
  \begin{itemize}
  \item Only possible with PPP connections, not with raw circuits
  \item channel bundeling according to newer RFC 1990 (MLP) is not possible
  \end{itemize}
  
  The second channel either can be added manually using imonc client or 
  automatically by bandwidth adaption, see description for
  \var{ISDN\_\-CIRC\_\-x\_\-BANDWIDTH}.
  
  Default setting: \var{ISDN\_\-CIRC\_\-x\_\-BUNDLING}='no'
  
  Caution: using channel bundeling together with compression can trigger 
  some problems, see description for \var{ISDN\_\-CIRC\_\-x\_\-FRAMECOMP}.

\config{ISDN\_CIRC\_x\_BANDWIDTH}{ISDN\_CIRC\_x\_BANDWIDTH}{ISDNCIRCxBANDWIDTH}
  
If ISDN channel bundeling is activated by \linebreak
\var{ISDN\_\-CIRC\_\-x\_\-BUNDLING}='yes' an automatical addition of 
the second ISDN channel can be configured here. Two parameters have to be set:
  \begin{enumerate}
  \item  threshold level in bytes/second (S)
  \item  time interval in seconds (Z)
  \end{enumerate}
  
  If treshold level S is exceeded for Z seconds imond will add a second 
  channel automatically. If treshold level S is underrun for Z seconds 
  imond will deactivate the second channel again. Automatic bandwidth 
  adaption may be deactivated with \var{ISDN\_\-CIRC\_\-1\_\-BANDWIDTH}=''. 
  After that channel bundeling can only be accomplished manually by the 
  imonc client.
  
  Examples:
  \begin{itemize}
  \item \var{ISDN\_\-CIRC\_\-1\_\-BANDWIDTH}='6144 30'
    
    If the transfer rate exceeds 6 kibibyte/second for 30 seconds 
    the second channel will be added.
    
  \item \var{ISDN\_\-CIRC\_\-1\_\-BANDWIDTH}='0 0'
    
    The second ISDN channel will be added immedeately (not later than 10 
    seconds after connection establishment and stays active until the connection 
    terminates completely.
    
  \item \var{ISDN\_\-CIRC\_\-1\_\-BANDWIDTH}=''
    
    The second ISDN channel only can be added manually, furthermore
    \var{ISDN\_\-CIRC\_\-1\_\-BUNDLING}='yes' has to be set.
    
  \item \var{ISDN\_\-CIRC\_\-1\_\-BANDWIDTH}='10000 30'
    
    This is intended to add a second channel after 30 seconds if 10000
    B/s were reached during that timespan. This won't work because ISDN 
    has a maximum tranfer rate of 8 kB/s.

  \end{itemize}
  
  If \var{ISDN\_\-CIRC\_\-x\_\-BUNDLING}='no' is set the value in
  variable \linebreak \var{ISDN\_\-CIRC\_\-x\_\-BANDWIDTH} is irrelevant.
  
  Default setting: \var{ISDN\_\-CIRC\_\-x\_\-BANDWIDTH}=''


\config{ISDN\_CIRC\_x\_LOCAL}{ISDN\_CIRC\_x\_LOCAL}{ISDNCIRCxLOCAL}
    
  This variable holds the local IP address on the ISDN side.
  
  This value should be \textbf{empty} if using dynamical address assignment. 
  The IP address will be negotiated during connection establishment. In 
  most cases internet providers hand out dynamic addresses. If a fixed IP 
  address is used specify it here. This variable is optional and has to 
  be added to the config file.

\config{ISDN\_CIRC\_x\_REMOTE}{ISDN\_CIRC\_x\_REMOTE}{ISDNCIRCxREMOTE}
    
  This variable holds the remote IP address and netmask on the ISDN side. 
  Classes Inter-Domain routing (CIDR) notation has to be used. Details for 
  \jump{IPNETx}{CIDR} can be found in the base documentation for IP\_NET\_x.
  
  With dynamic address negotiation this should \textbf{empty}. The IP address 
  will be negotiated on connection establishment. In most cases internet 
  providers hand out dynamic addresses. If a fixed IP address is used 
  specify it here. This variable is optional and has to be added to 
  the config file.

  The netmask provided will be used for interface configuration after dial-in. 
  A route to the dial-in host itself will be generated as well. As you most 
  probably won't need this route it is best to generate a direct route to 
  the dial-in host by setting the netmask to /32. For details see
  \jump{sec:route-details}{Chapter: Technical Details For Dialin}.


\configlabel{ISDN\_CIRC\_x\_MTU}{ISDNCIRCxMTU}
\config{ISDN\_CIRC\_x\_MTU ISDN\_CIRC\_x\_MRU}{ISDN\_CIRC\_x\_MRU}{ISDNCIRCxMRU}
  
  With this optional variable the so-called \textbf{MTU} (maximum transmission unit) 
  and \textbf{MRU} (maximum receive unit) can be set. Optional means that the variable 
  has to be added manually to the configuration file by the user! \\ 
  Usually MTU is 1500 and MRU 1524. This settings should only be changed in
  rare special cases!

\config{ISDN\_CIRC\_x\_CLAMP\_MSS}{ISDN\_CIRC\_x\_CLAMP\_MSS}{ISDNCIRCxCLAMPMSS}

Set this to 'yes' when using synchronous ppp \\ (\var{ISDN\_CIRC\_x\_TYPE}='ppp') and 
one of the following symptoms occurs:
\begin{itemize}
\item Webbrowser connects to the webserver without error messages but no pages 
  are displayed and nothing happens,
\item sending of small \mbox{E-mails} is working but bigger ones trigger problems or
\item ssh works but scp hangs after initial connection.
\end{itemize}

  Default setting: \var{ISDN\_\-CIRC\_\-x\_\-CLAMP\_MSS}='no'

\config{ISDN\_CIRC\_x\_HEADERCOMP}{ISDN\_CIRC\_x\_HEADERCOMP}{ISDNCIRCxHEADERCOMP}
  
  \var{ISDN\_\-CIRC\_\-x\_\-HEADERCOMP}='yes' activates Van-Jacobson compression or 
  header compression. Not all providers are supporting this. If activated compression 
  leads to problems while dialing in set this to 'no'.
  
  Default setting: \var{ISDN\_\-CIRC\_\-x\_\-HEADERCOMP}='yes'


\config{ISDN\_CIRC\_x\_FRAMECOMP (EXPERIMENTAL)}{ISDN\_CIRC\_x\_FRAMECOMP}{ISDNCIRCxFRAMECOMP}
  
  This parameter is only used if \\ \var{OPT\_\-ISDN\_\-COMP} is set to 'yes'. 
  It handles frame compression.
  
  The following values are possible:

  \begin{tabular}[h!]{ll}
        'no' &                    no frame compression\\
        'default' &               LZS according to RFC1974(std) and
        BSDCOMP 12 \\
        'all' &                   Negotiate lzs and bsdcomp \\
        'lzs' &                   Negotiate lzs only \\
        'lzsstd' &                LZS according to RFC1974 Standard Mode
                                (``Sequential Mode'') \\
        'lzsext' &                LZS according to RFC1974 Extended Mode \\
        'bsdcomp' &               Negotiate bsdcomp only \\
        'lzsstd-mh' &             LZS Multihistory according to RFC1974
                                  Standard Mode (``Sequential Mode``)
  \end{tabular}
  
  You have to find out by yourself which value is supported by the provider. 
  T-Online supports only 'lzsext' as far as I know. With most other providers 'default' 
  should work.
  
  Attention: using channel bundeling together with 'lzsext' can lead to problems specific 
  to the dial-in server and provider. As providers use different types of dial-in servers 
  there can be differences between dial-in servers of the same provider.

  'lzsstd-mh' is meant for router-to-router usage (r2r). It is not used by providers 
  but using it between two fli4l routers leads to significant improvements while transferring 
  more files in parallel. Header compression is needed here and therefore will be activated 
  automatically.

\config {ISDN\_CIRC\_x\_REMOTENAME}{ISDN\_CIRC\_x\_REMOTENAME}{ISDNCIRCxREMOTENAME}
  
  This variable normally is only relevant when configuring fli4l as a dial-in router. 
  Set the name of a remote hosts if you want but this is not needed.
  
  Default setting: \var{ISDN\_\-CIRC\_\-x\_\-REMOTENAME}=''

\configlabel{ISDN\_CIRC\_x\_USER}{ISDNCIRCxUSER}
\config {ISDN\_CIRC\_x\_PASS}{ISDN\_CIRC\_x\_PASS}{ISDNCIRCxPASS}
  
  Enter provider data here. In the example data for Microsoft Network is used.
  
  \var{ISDN\_\-CIRC\_\-x\_\-USER} holds the user-id, \var{ISDN\_\-CIRC\_\-x\_\-PASS}
  the password.
  
  Note for T-Online:
  
  Username AAAAAAAAAAAATTTTTT\#MMMM is composed from a 12 digit 'Anschlußkennung'
  plus T-Online-Number and 'Mitbenutzernummer'. Put a '\#' after the T-Online-Nummer 
  if it is shorter than 12 characters.
  
  In rare cases another '\#' character has to be inserted between 'Anschlußkennung' and 
  T-Online-Number.
  
  For T-Online-Numbers with 12 characters no additional '\#' is needed.
  
  Example:  \var{ISDN\_\-CIRC\_\-1\_\-USER}='123456\#123'

  
  For Raw-IP circuits this variable has no meaning.

\config {ISDN\_CIRC\_x\_ROUTE\_N}{ISDN\_CIRC\_x\_ROUTE\_N}{ISDNCIRCxROUTEN}

  Number of routes of this ISDN circuit. If the circuit defines a default 
  route you must set this to '1'.

\config {ISDN\_CIRC\_x\_ROUTE\_X}{ISDN\_CIRC\_x\_ROUTE\_X}{ISDNCIRCxROUTEx}
  
  Route(s) for this circuit. For Internet access the first entry should be 
  '0.0.0.0/0' (default route). Format is always 'network/netmaskbits'. A 
  host route for example would look like this: '192.168.199.1/32'. If dialing 
  in to company or university routers name only the net you want to reach there. 
  Examples:

\begin{example}
\begin{verbatim}
        ISDN_CIRC_%_ROUTE_N='2'
        ISDN_CIRC_%_ROUTE_1='192.168.8.0/24'
        ISDN_CIRC_%_ROUTE_2='192.168.9.0/24'
\end{verbatim}
\end{example}
  
  All nets must have an explicit entry hence for each route a new 
  ISDN\_CIRC\_x\_ROUTE\_y='' line has to be provided.
  
  For using fli4l's LC routing features a default route can be assigned to 
  *several* circuits. Which circuit is used is driven by \var{ISDN\_\-CIRC\_\-x\_\-TIMES}, 
  see below.

\config {ISDN\_CIRC\_x\_DIALOUT}{ISDN\_CIRC\_x\_DIALOUT}{ISDNCIRCxDIALOUT}

  \var{ISDN\_\-CIRC\_\-x\_\-DIALOUT} specifies the telephone number to be dialed. It is 
  possible to put in several numbers (if one is busy the next is chosen) - numbers have 
  to be separated by blanks. A maximum of five numbers can be used.

\config {ISDN\_CIRC\_x\_DIALIN}{ISDN\_CIRC\_x\_DIALIN}{ISDNCIRCxDIALIN}

  If the circuit (also) serves for dial-in \var{ISDN\_\-CIRC\_\-x\_\-DIALIN} keeps the 
  phone number of the callee - with a region prefix but *without* a leading 0. Ports 
  behind telephone systems may have to specify one or even two zeros.
  
  If more users should be able to dial in over a circuit more numbers may be added separated 
  by blanks. It is advised to assign a separate circuit for each caller although. Otherwise two callers 
  trying to dial in at the same time (which is absolutetely feasible with 2 ISDN channels) 
  could collide on behalf of IP addresses assigned.
  
  If callers don't transfer a number during calling '0' could be used.
  Caution: everyone not transferring a number is allowed to call in then!
  
  If number-independent dial-in should be realized set '*'.

  In both cases a separate authentification (see \var{ISDN\_CIRC\_x\_AUTH}) is unavoidable.

\config {ISDN\_CIRC\_x\_CALLBACK}{ISDN\_CIRC\_x\_CALLBACK}{ISDNCIRCxCALLBACK}
  
  Settings for callback, possible values:

  \begin{tabular}[h!]{ll}
        'in' &     fli4l is called and calls back\\
        'out' &    fli4l calls, hangs up and waits for callback \\
        'off' &    no callback\\
        'cbcp' &   callback control protocol\\
        'cbcp0' &  callback control protocol 0\\
        'cbcp3' &  callback control protocol 3\\
        'cbcp6' &  callback control protocol 6\\

  \end{tabular}

  CallBack control protocol (aka 'Microsoft CallBack')
  cbcp6 is the protocol mostly used.

  Default setting: 'off'

\config {ISDN\_CIRC\_x\_CBNUMBER}{ISDN\_CIRC\_x\_CBNUMBER}{ISDNCIRCxCBNUMBER}

  Set a callback number for protocol cbcp, cbcp3 and cbcp6 here (mandatory for cbcp3).

\config {ISDN\_CIRC\_x\_CBDELAY}{ISDN\_CIRC\_x\_CBDELAY}{ISDNCIRCxCBDELAY}
  
  This variable sets a delay in seconds to be waited until triggering callback. 
  The meaning differs depending on the direction of callback:
  
  \begin{itemize}
  \item  \var{ISDN\_\-CIRC\_\-x\_\-CALLBACK}='in':
    
    If fli4l is called and should call back \var{ISDN\_\-CIRC\_\-x\_\-CBDELAY} specifies 
    the waiting time until calling back. Use \var{ISDN\_\-CIRC\_\-x\_\-CBDELAY}='3' 
    as a rule of thumb. A lower value may work and speed up connection establishment then 
    depending on whom to call back.
    
  \item \var{ISDN\_\-CIRC\_\-x\_\-CALLBACK}='out':
    
    In this case \var{ISDN\_\-CIRC\_\-x\_\-CBDELAY} is the ringing timespan for the other 
    party until fli4l waits for callback. \var{ISDN\_\-CIRC\_\-x\_\-CBDELAY}='3' 
    is a good rule of thumb here either. On long distance calls it takes up 
    to 3 seconds untils the other router is even recognizing the call. If in 
    doubt simply try.
    
  \end{itemize}

  
  If setting \var{ISDN\_\-CIRC\_\-x\_\-CALLBACK}='off',
  \var{ISDN\_\-CIRC\_\-x\_\-CBDELAY} is ignored.
  This variable is ignored with CallBack Control Protocol as well.


\config {ISDN\_CIRC\_x\_EAZ}{ISDN\_CIRC\_x\_EAZ}{ISDNCIRCxEAZ}

  In the example the MSN (called EAZ here) is set to 81330. Set your 
  own MSN *without* area code here.
  
  Depending on your telephony system only the extension station number 
  could be necessary. Setting an additional '0' may also help sometimes. 
  
  This variable may be empty which can help with some telephony systems as well. 
  
\config {ISDN\_CIRC\_x\_SLAVE\_EAZ}{ISDN\_CIRC\_x\_SLAVE\_EAZ}{ISDNCIRCxSLAVEEAZ}
  
  If fli4l is connected on the internal S0-Bus of a telephony system and you 
  want to use channel bundeling you may have to specify a second extension 
  station number for the slave channel.
  
  Normally this variable can stay empty.

\config {ISDN\_CIRC\_x\_DEBUG}{ISDN\_CIRC\_x\_DEBUG}{ISDNCIRCxDEBUG}
  
  If ipppd should display additional debug informations set 
  \var{ISDN\_\-CIRC\_\-x\_\-DEBUG} to 'yes'. Ipppd will log
  these informations to syslog then.
  
  IMPORTANT: To use syslogd for logging \var{OPT\_\-SYSLOGD} has to be set to 'yes'.\\
  (See \jump{OPTSYSLOGD}{OPT\_SYSLOGD - Program For Protocolling System Messsages})
  Some messages are logged to klogd so \var{OPT\_\-KLOGD} 
  (See \jump{OPTKLOGD}{OPT\_KLOGD - Kernel-Message-Logger})
  should be set to 'yes' as well for debugging ISDN.
  
  For Raw-IP-Circuits \var{ISDN\_\-CIRC\_\-x\_\-DEBUG} has no meaning.

\config {ISDN\_CIRC\_x\_AUTH}{ISDN\_CIRC\_x\_AUTH}{ISDNCIRCxAUTH}
  
  If the circuit is also used for dial-in and an authentification over 
  PAP or CHAP should be used by the calling party set 
  \var{ISDN\_\-CIRC\_\-x\_AUTH} to 'pap' or 'chap' - but *only* then. 
  In all other cases this variable has to be empty!
  
  Cause: An Internet provider will never authentificate with you - but 
  there may be exceptions to this rule.
  
  Use the entries \var{ISDN\_\-CIRC\_\-x\_USER} and \var{ISDN\_\-CIRC\_\-x\_PASS}
  for username and password.

  For Raw-IP-Circuits this variable has no meaning.

\config {ISDN\_CIRC\_x\_HUP\_TIMEOUT}{ISDN\_CIRC\_x\_HUP\_TIMEOUT}{ISDNCIRCxHUPTIMEOUT}
  
  \var{ISDN\_\-CIRC\_\-x\_\-HUP\_\-TIMEOUT} sets the time after that 
  fli4l disconnects from the provider if no traffic is detected over 
  the connection. In the example the connection will be hung up after 
  40 seconds idle time to save money. On new accesses fli4l connects 
  again in a short timespan. This makes sense especially with providers 
  that have seconds charge intervals!
  
  At least while testing you should keep an eye on fli4l's automatic 
  dial-in and hangup using either console or imon-client. In case of 
  faulty configuration the ISDN connection could become an unwanted 
  permanent line.
  
  Setting this to '0' means that fli4l doesn't use idle time and won't hang 
  up by itself anymore. Use with care!
  
\config {ISDN\_CIRC\_x\_CHARGEINT}{ISDN\_CIRC\_x\_CHARGEINT}{ISDNCIRCxCHARGEINT}
  
  Set charge interval in seconds which will be used for calculating online costs.
  
  Most providers charge by minute intervals. In this case use the value '60'. 
  For providers that charge in seconds set \var{ISDN\_\-CIRC\_\-x\_\-CHARGEINT} to '1'.
  
  Addition for \var{ISDN\_\-CIRC\_\-x\_\-CHARGEINT} $>$= 60 seconds:
  
  If no traffic was detected for \var{ISDN\_\-CIRC\_\-x\_\-HUP\_\-TIMEOUT} 
  seconds the connection will be terminated approx. 2 seconds before reaching 
  the chargint timespan. Charged time is used nearly complete this way. A 
  really neat feature of fli4l.
  
  With charging intervals of under a minute this does not make sense so this 
  feature is only used for charging intervals of more than 60 seconds.
  
\config {ISDN\_CIRC\_x\_TIMES}{ISDN\_CIRC\_x\_TIMES}{ISDNCIRCxTIMES}
  
  Specify here when and at what charge the circuit should be active. 
  This makes it possible to use different circuits as default routes at 
  different daytimes (least-cost-routing).The imond daemon controls 
  route-allocation.
  
  Composition of the variable:

\begin{example}
\begin{verbatim}
        ISDN_CIRC_x_TIMES='times-1-info [times-2-info] ...'
\end{verbatim}
\end{example}

  
  Each times-?-info field consists of 4 subfields separated by colons (':').
  \begin{enumerate}
  \item Field: W1-W2
    
    Weekday-timespan, for example Mo-Fr or Sa-Su. English and german notation 
    are possible. If a single weekday should be specified write W1-W1, for 
    example Su-Su.
    
  \item Field: hh-hh
    
    Daytime-timespan, for example 09-18 or also 18-09. 18-09 is equal to 
    18-24 plus 00-09. 00-24 means the whole day.

  \item  Field: Charge
    
    Costs per minute in currency units, for example 0.032 for 3.2 
    Cent per minute. The real costs are calculated in consideration of charging 
    intervals and displayed in imon-client then.
    
  \item  Field: LC-default-route
    
    May be Y or N. Meaning:

    \begin{itemize}
    \item Y: The timespan specified will be used as default route for LC-routing. 
      Important: in this case 
      \var{ISDN\_\-CIRC\_\-x\_\-ROUTE}='0.0.0.0/0' must be set in addition!
        
    \item N: The timespan specified only serves for calculating online costs but 
      won't be used for LC-Routing.
    \end{itemize}

  \end{enumerate}

    Example:

\begin{small}
\begin{example}
\begin{verbatim}
    ISDN_CIRC_1_TIMES='Mo-Fr:09-18:0.049:N Mo-Fr:18-09:0.044:Y Sa-Su:00-24:0.044:Y'
    ISDN_CIRC_2_TIMES='Mo-Fr:09-18:0.019:Y Mo-Fr:18-09:0.044:N Sa-Su:00-24:0.044:N'
\end{verbatim}
\end{example}
\end{small}
    
    Read as follows: Circuit 1 should be used on labour days evenings/nights and 
    on the complete weekends, Circuit 2 is used on labour days from 9 AM to 6 PM.

    \begin{description}
    \item \wichtig{timespans specified in \var{ISDN\_CIRC\_x\_TIMES} have to cover 
     the whole week. Without that no valid configuration can be generated.}
      
     \emph{If timespans of all LC-default-route circuits (``Y'') don't cover the 
     complete week no default route exists during the missing times. Therefore 
     no internet connections are possible!}

        
    \item Example:
\begin{example}
\begin{verbatim}
    ISDN_CIRC_1_TIMES='Sa-Su:00-24:0.044:Y Mo-Fr:09-18:0.049:N Mo-Fr:18-09:0.044:N'
    ISDN_CIRC_2_TIMES='Sa-Su:00-24:0.044:N Mo-Fr:09-18:0.019:Y Mo-Fr:18-09:0.044:N'
\end{verbatim}
\end{example}
      
      Here for labour days from 18-09 ``N'' is set. At this times no route to 
      the internet exists - surfing forbidden!

      
    \item Another simple example:

\begin{example}
\begin{verbatim}
      ISDN_CIRC_1_TIMES='Mo-Su:00-24:0.0:Y'        
\end{verbatim}
\end{example}
      
      for those using a flatrate.


      
    \item A last note concerning LC-Routing:
      
      Bank holidays are treated as sundays.
    \end{description}

\end{description}

\subsection {OPT\_TELMOND - telmond-Configuration}
\configlabel{OPT\_TELMOND}{OPTTELMOND}

\var{OPT\_\-TELMOND} determines wheter the telmond server is activated 
or not. It listens to incoming telephone calls and signals on TCP port 
5001 the caller id transmitted and called. This information can be queried 
and displayed by i.e. windows- or Unix/Linux imon-client (see chapter 
``Client-/Server-interface imond'').

An installed ISDN card is mandatory as well as as valid configuration 
of \var{OPT\_\-ISDN}.

Testing the correct function of telmond is possible with:

\begin{example}
\begin{verbatim}
        telnet fli4l 5001        
\end{verbatim}
\end{example}

This should display the last call and immedeately close the telnet 
connection.

Port 5001 is only reachable from LAN. Access from outside is blocked 
by the firewall per default. Further access limitations are configurable 
via telmond variables, see below.

Default setting: \var{START\_\-TELMOND}='yes'

\begin{description}

\config {TELMOND\_PORT}{TELMOND\_PORT}{TELMONDPORT}
  
  TCP/IP-Port on which telmond listens for connections.  The default 
  setting '5001' should only be changed in rare exceptions.


\config {TELMOND\_LOG}{TELMOND\_LOG}{TELMONDLOG}
  
  \var{TELMOND\_\-LOG}='yes' activates saving of all incoming calls in 
  a file called /var/log/telmond.log. The content of this file can be 
  queried with imond-Client imonc under Unix/Linux and Windows.
  
  Different paths or logfiles splitted by clients may be configured 
  below.
  
  Default setting: \var{TELMOND\_\-LOG}='no'

\config {TELMOND\_LOGDIR}{TELMOND\_LOGDIR}{TELMONDLOGDIR}
  
  If protocolling is active \var{TELMOND\_\-LOGDIR} can set a different path 
  instead of /var/log, for example '/boot'. The file telmond.log will be saved on 
  the boot media (which has to be mounted Read/Write ``rw'') then. If 'auto' is set 
  the logfile is created in /boot/persistent/isdn or at another path specified by 
  \var{FLI4L\_UUID}. If /boot is not mounted Read/Write the file is created in 
  /var/run.

\config {TELMOND\_MSN\_N}{TELMOND\_MSN\_N}{TELMONDMSNN}
  
  If certain calls should only be visible on some client PC's imonc a filter 
  can be set to achieve that MSNs are only protocolled for those PCs.
  
  If this is necessary (for example with flat sharing) the variable 
  \var{TELMOND\_\-MSN\_\-N} holds the number of MSN filters.
  
  Default setting: \var{TELMOND\_\-MSN\_\-N}='0'

\config {TELMOND\_MSN\_x}{TELMOND\_MSN\_x}{TELMONDMSNx}
  
  For each MSN filter a list of IP addresses has to be set which should be able 
  to view the call informations.
  
  The variable \var{TELMOND\_\-MSN\_\-N} determines the number of those 
  configurations, see above.
  
  Composition of the variable:
\begin{example}
\begin{verbatim}
        TELMOND_MSN_x='MSN IP-ADDR-1 IP-ADDR-2 ...'
\end{verbatim}
\end{example}

  
  A simple example:

\begin{example}
\begin{verbatim}
        TELMOND_MSN_1='123456789 192.168.6.2'            
\end{verbatim}
\end{example}

  If a call for a certain MSN should be displayed on more computers 
  their IP addresses have to be specified one after the other.

\begin{example}
\begin{verbatim}
        TELMOND_MSN_1='123456789 192.168.6.2 192.168.6.3'            
\end{verbatim}
\end{example}

\config {TELMOND\_CMD\_N}{TELMOND\_CMD\_N}{TELMONDCMDN}
  
  If a telephone call (Voice) is coming in for a MSN some commands can be 
  executed on the fli4l router optionally. \var{TELMOND\_\-CMD\_\-N} holds 
  the number of commands to be executed.

\config {TELMOND\_CMD\_x}{TELMOND\_CMD\_x}{TELMONDCMDx}
  
  \var{TELMOND\_\-CMD\_\-1} bis \var{TELMOND\_\-CMD\_\-n} holds commands to be 
  executed for an incoming phone call.
  
  Variable \var{TELMOND\_\-CMD\_\-N} specifies the quantity of commands,
  see above.
  
  Composition of the variable:

\begin{example}
\begin{verbatim}
        MSN CALLER-NUMBER  COMMAND ...
\end{verbatim}
\end{example}
  
  Set the MSN without area prefix. CALLER-NUMBER takes the complete caller id 
  with area prefix. If CALLER-NUMBER is set to an asterisk (*) telmond won't 
  pay attention to the caller id.
  
  An example:

\begin{example}
\begin{verbatim}
        TELMOND_CMD_1='1234567 0987654321 sleep 5; imonc dial'
        TELMOND_CMD_2='1234568 * switch-on-coffee-machine'
\end{verbatim}
\end{example}
  
  In the first case the command sequence ``sleep 5; imonc dial'' is executed 
  if caller with id 0987654321 calls MSN 1234567. Two commands are executed.
  At first fli4l will wait for 5 seconds for the ISDN channel to become available. 
  After that the fli4l client imonc is started with the argument ``dial''. 
  imonc passes this command to the telmond server which will establish a 
  network connection on the default route circuit. Which commands the imonc 
  client can pass to the imond server is described in chapter ``Client-/Server 
  interface imond''. \var{OPT\_IMONC} from the package ``tools'' has to be 
  installed to get this working.
  
  The second command ``switch-on-coffee-machine'' will be executed if a call 
  on MSN 1234568 comes in independent on caller id. The command 
  ``switch-on-coffee-machine'' does not exist for fli4l (at the moment)!
  
  On execution of command the following placeholders may be used:

  \begin{tabular}[h!]{cll}
  \hline
            \%d  &    date   &     Date \\
            \%t  &    time   &     Time \\
            \%p  &    phone  &     Caller ID \\
            \%m  &    msn     &    Own MSN \\
            \%\%  &    percent &    the percent sign\\
  \end{tabular}
  
  This data can be used by the programs called, for example for sending \mbox{E-Mail}.

\config {TELMOND\_CAPI\_CTRL\_N}{TELMOND\_CAPI\_CTRL\_N}{TELMONDCAPICTRLN}

  If using a CAPI capable ISDN adapter or a remote CAPI (type 160 or 161) 
  it may be necessary to configure the CAPI controller on which telmond is 
  listening for calls more explicitly. For example the Fritz!Box offers access 
  to up to five different controllers which may not even differ (see informations 
  at \altlink{http://www.wehavemorefun.de/fritzbox/CAPI-over-TCP\#Virtuelle_Controller}).
  To limit the number of controllers to be used you may set the quantity. 
  In the following array-variables \var{TELMOND\_CAPI\_CTRL\_\%} it may be set 
  which controllers are to be used.

  If you don't use this variable telmond will listen on \emph{all} available CAPI 
  controllers.

\config {TELMOND\_CAPI\_CTRL\_x}{TELMOND\_CAPI\_CTRL\_x}{TELMONDCAPICTRLx}

  If \var{TELMOND\_CAPI\_CTRL\_N} is unequal to zero the indices for the CAPI 
  controllers have to be specified on which telmond should monitor incoming calls.

  Example for the remote CAPI of a Fritz!Box with ``real'' ISDN connection:

\begin{example}
\begin{verbatim}
        TELMOND_CAPI_CTRL_N='2'
        TELMOND_CAPI_CTRL_1='1' # listen to incoming ISDN calls
        TELMOND_CAPI_CTRL_2='3' # listen to calls on the internal S0-Bus
\end{verbatim}
\end{example}

  Example for the remote CAPI of a Fritz!Box with analog connection and
  SIP-Forwarding:

\begin{example}
\begin{verbatim}
        TELMOND_CAPI_CTRL_N='2'
        TELMOND_CAPI_CTRL_1='4' # listen to incoming analog calls
        TELMOND_CAPI_CTRL_2='5' # listen to incoming SIP-calls
\end{verbatim}
\end{example}

\end{description}

\subsection{OPT\_RCAPID - Remote CAPI Daemon}
\configlabel{OPT\_RCAPID}{OPTRCAPID}

This OPT configures the program rcapid on the fli4l router which offers access 
to the ISDN CAPI interface of the routers via network. Appropritate tools can 
access the ISDN card of the routers via network as if it was installed locally. 
This is similar to the package ``mtgcapri''. The difference is that only Windows 
systems can use ``mtgcapri'' as a client while the network interface of rcapid 
is only supporting linux systems at the time of writing. So both packages are 
ideal complements in mixed Windows- and Linux environments.

\subsubsection{Konfiguration des Routers}

\begin{description}
\config{OPT\_RCAPID}{OPT\_RCAPID}{OPTRCAPID}{
This variable activates offering of the router's ISDN-CAPI for remote clients. 
Possible values are ``yes'' and ``no''. If set to ``yes'' the internet daemon 
inetd will start the rcapid daemon on incoming queries on rcapid port 6000 
(port my be changed using variable \var{RCAPID\_PORT}).

Example:
}
\verb*?OPT_RCAPID='yes'?

\config{RCAPID\_PORT}{RCAPID\_PORT}{RCAPIDPORT}{
This variable holds the TCP port to be used by the rcapid daemon.

Default setting:
}
\verb*?RCAPID_PORT='6000'?
\end{description}

\subsubsection{Configuration Of Linux Clients}
To use the remote CAPI interface on a Linux PC the modular libcapi20 library 
must be used. Actual Linux distributions install such a CAPI library (i.e.
Debian Wheezy). If not the sources may be downloaded from
\altlink{http://ftp.de.debian.org/debian/pool/main/i/isdnutils/isdnutils_3.25+dfsg1.orig.tar.bz2}.
After unpacking and changing to the directory ``capi20'' the CAPI library may be 
compiled and installed with ``./configure'', ``make'' and ``sudo make install'' 
as usual. If the library is installed the configuration file \texttt{/etc/capi20.conf} 
has to be adapted to specifiy the client on which rcapid is running. If the router for 
example is reached by the name of ``fli4l'' the conf file will looks as follows:

\begin{example}
\begin{verbatim}
REMOTE fli4l 6000
\end{verbatim}
\end{example}

That's all! If the program ``capiinfo'' is installed on the linux client (part of 
capi4k-utils-package of many distributions) you can test the remote CAPI interface:

\begin{example}
\begin{verbatim}
kristov@peacock ~ $ capiinfo 
Number of Controllers : 1
Controller 1:
Manufacturer: AVM Berlin
CAPI Version: 1073741824.1229996355
Manufacturer Version: 2.2-00  (808333856.1377840928)
Serial Number: 0004711
BChannels: 2
[...]
\end{verbatim}
\end{example}

Under ``Number of Controllers'' the quantity of ISDN cards is displayed which are 
usable on the client. If this reads ``0'' the connection to the rcapid program is 
working but the ISDN card is not recognized on the router. If the connection to the 
rcapid program is not working at all (maybe \var{OPT\_RCAPID} is set to ``no'') 
an error message ``capi not installed - Connection refused (111)'' will be displayed. 
In this case check your configuration once more.
