% Synchronized to r29817
\section{ISDN}
\marklabel{sec:route-details}
{
  \subsection{Technical Details About Dial-In And Routing With ISDN}
}

This chapter is interesting for those who want to know what is happening 
'under the hood', having special wishes for configuration or simply 
looking for solutions to their problems. All others are \emph{ not} 
encouraged to read this chapter.

After establishing a connection to the provider the ipppd daemon 
that has made the connection newly configures the interface to 
set the negotiated IP addresses. The linux kernel automatically sets 
corresponding routes for remote IP address and netmask. Existing 
special routes will be deleted. If no netmask is given the ipppd 
derives it from the remote IP netmask (Class A, B and C subnets 
will be used here). The vanishing of existing and appearing of new 
routes has raised problems time and time again:

\begin{itemize}
\item Company networks were unaccessible because of routes vanishing 
  or being overlayed by newly set ones
\item Interfaces were dialing in apparently without cause because a 
  packet was routed by the kernel to another interface instead of 
  the default route 
\item ...
\end{itemize}

Because of that it is tried to avoid unwanted routes.

The following things will be changed to achieve this:
\begin{itemize}

\item remote IP will be set to 0.0.0.0 if nothing else is specified. 
  Hence the routes configured by the kernel while initializing the 
  interface will vanish.
  
\item additionally set routes will be saved in a file

\item if a netmask is given for the circuit it will be transferred to 
  the ipppd in order to use it for the configuration of the interface 
  (and therefore route generation) after negotiation of an IP.

\item after dial-in the saved routes of the circuit will be reloaded 
  from the file and set again (they were deleted by the kernel while 
  ipppd was reconfiguring the interface)
  
\item after hangup the interface will be reconfigured and routes are 
  set anew to restore the initial situation.
\end{itemize}

Configuration of a circuits looks like this in that case:

\begin{itemize}
\item 
  item default route
  \begin{small}
\begin{example}
\begin{verbatim}
    ISDN_CIRC_%_ROUTE='0.0.0.0'
\end{verbatim}
\end{example}
  \end{small}
  
  If the circuit is a lcr circuit and ``active'' in the moment a default 
  route will be set towards the circuit (res. the according interface).
  After dial-in a host-route to the provider appears that vanishes after 
  hanging up.

  \item special routes
  \begin{small}
\begin{example}
\begin{verbatim}
    ISDN_CIRC_%_ROUTE='network/netmaskbits'
\end{verbatim}
\end{example}
  \end{small}
  
  The given routes for the circuit (res. the according interface) will 
  be set. After dial-in the routes deleted by the kernel will be restored 
  and a host-route to the dial-in node exists. After hangup the initial 
  state will be restored.


  \item remote ip
  \begin{small}
\begin{example}
\begin{verbatim}
    ISDN_CIRC_%_REMOTE='ip address/netmaskbits'
    ISDN_CIRC_%_ROUTE='network/netmaskbits'
\end{verbatim}
\end{example}
  \end{small}
  
  While configurating the interface routes to the target net appear 
  (according to IP address AND netmask). If the specified IP is kept 
  after dial-in (meaning no other IP is negotiated during connection 
  establishment) the route will be kept as well. 
  
  If another IP was negotiated during dial-in the route will change 
  accordingly (new IP AND netmask).
  
  For additional routes see above.


\end{itemize}

This will hopefully solve \emph{all} problems raised by special routes. 
The way of correction may change in the future but the principle won't.

\marklabel{sec:isdn-cause}
{
  \subsection{Error Messages Of The ISDN-Subsystem (i4l-Documentation)}
}

Following is an excerpt from the Isdn4Linux Documentation (man 7
isdn\_cause). 

Cause messages are 2-byte information elements, describing the state
transitions of an ISDN line. Each cause message describes its
origination (location) in one byte, while the cause code is described
in the other byte. Internally, when EDSS1 is used, the first byte
contains the location while the second byte contains the cause code.
When using 1TR6, the first byte contains the cause code while the
location is coded in the second byte. In the Linux ISDN subsystem, the
cause messages visible to the user are unified to avoid confusion. All
user visible cause messages are displayed as hexadecimal strings.
These strings always have the location coded in the first byte,
regardless if using 1TR6 or EDSS1. When using EDSS1, these strings are
preceeded by the character 'E'.


\begin{description}
\item [LOCATION] 
 
  The following location codes are defined when using EDSS1:

  \begin{small}
  \begin{longtable}{lp{12cm}}

  00 &   Message generated by user. \\
  01 &   Message generated by private network serving the local user. \\
  02 &   Message generated by public network serving the local user. \\
  03 &   Message generated by transit network. \\
  04 &   Message generated by public network serving the remote user. \\
  05 &   Message generated by private network serving the remote
  user. \\
  07 &   Message generated by international network. \\
  0A &   Message generated by network beyond inter-working point. \\
  \end{longtable}
  \end{small}

\item  [CAUSE]

  The following cause codes are defined when using EDSS1:

  \begin{small}
  \begin{longtable}{lp{12cm}}

  01 &   Unallocated (unassigned) number. \\
  02 &   No route to specified transit network. \\
  03 &   No route to destination. \\
  06 &   Channel unacceptable. \\
  07 &   Call awarded and being delivered in an established channel. \\
  10 &   Normal call clearing. \\
  11 &   User busy. \\
  12 &   No user responding. \\
  13 &   No answer from user (user alerted). \\
  15 &   Call rejected. \\
  16 &   Number changed. \\
  1A &   Non-selected user clearing. \\
  1B &   Destination out of order. \\
  1C &   Invalid number format. \\
  1D &   Facility rejected. \\
  1E &   Response to status enquiry. \\
  1F &   Normal, unspecified. \\
  22 &   No circuit or channel available. \\
  26 &   Network out of order. \\
  29 &   Temporary failure. \\
  2A &   Switching equipment congestion. \\
  2B &   Access information discarded. \\
  2C &   Requested circuit or channel not available. \\
  2F &   Resources unavailable, unspecified. \\
  31 &   Quality of service unavailable. \\
  32 &   Requested facility not subscribed. \\
  39 &   Bearer capability not authorised. \\
  3A &   Bearer capability not presently available. \\
  3F &   Service or option not available, unspecified. \\
  41 &   Bearer capability not implemented. \\
  42 &   Channel type not implemented. \\
  45 &   Requested facility not implemented. \\
  46 &   Only restricted digital information bearer. \\
  4F &   Service or option not implemented, unspecified. \\
  51 &   Invalid call reference value. \\
  52 &   Identified channel does not exist. \\
  53 &   A suspended call exists, but this call identity does not. \\
  54 &   Call identity in use. \\
  55 &   No call suspended. \\
  56 &   Call having the requested call identity. \\
  58 &   Incompatible destination. \\
  5B &   Invalid transit network selection. \\
  5F &   Invalid message, unspecified. \\
  60 &   Mandatory information element is missing. \\
  61 &   Message type non-existent or not implemented. \\
  62 &   Message not compatible with call state or message 
        or message type non existent or not implemented. \\
  63 &   Information element non-existent or not implemented. \\
  64 &   Invalid information element content. \\
  65 &   Message not compatible. \\
  66 &   Recovery on timer expiry. \\
  6F &   Protocol error, unspecified. \\
  7F &   Inter working, unspecified. \\
  \end{longtable}
  \end{small}
\end{description}
