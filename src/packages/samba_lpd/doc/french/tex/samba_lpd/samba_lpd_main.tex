% Do not remove the next line
% Synchronized to r35642

\section{SAMBA\_LPD - Support pour service d'impression et de données pour Windows dans le réseau fli4l}

Le paquetage \var{SAMBA\_\-LPD} est composé de plusieurs paquetages-OPT.

\begin{itemize}
\item \var{OPT\_\-SAMBA} - Samba en tant que serveur de fichiers et d'impression
\item \var{OPT\_\-SAMBATOOLS} - Outils qui tourne autour de Samba, par ex.
  des outils pour envoyer des messages à des clients Windows, autorise des
  intégrations par le réseau dans le système de fichiers du routeur, ...
\item \var{OPT\_\-NMBD} - Un serveur de nom NETBIOS (supporte les autorisations sur le réseau)
\item \var{OPT\_\-LPD} - Un support pour imprimante distante
\end{itemize}

Bien que l'installation des quatre paquetages-OPT est combinée en un seul
paquetage, il est possible d'activer ou de désactiver individuellement chaque
paquetage. Lorque vous désactivez l'un d'eux, vous perdez naturellement le
fonctionnement du paquetage-OPT correspondant. Une exception le paquetage
\var{OPT\_\-NMBD} fonctionne que si \var{OPT\_\-SAMBA} est activé.

\wichtig{Si vous avez activé \var{OPT\_\-LPD} vous ne devez absolument pas
  activer la variable \var{OPT\_\-LPDSRV}='no'~!}

Les paquetages-OPT sont décrits dans les pagragraphes suivants.


\marklabel{sec:OPTSAMBA}{\subsection {OPT\_SAMBA - Samba serveur de fichiers et d'impression}}
\configlabel{OPT\_SAMBA}{OPTSAMBA}

    Si vous activez \var{OPT\_\-SAMBA}='yes' les clients Windows peuvent imprimer
    directement en utilisant le protocole SMB. Aucun autre logiciel client n'est
    nécessaire (juste le pilote de l'imprimante).

   Cependant la condition la plus importante pour l'impression avec samba est
   de toujours paramétrer la variable sur \smalljump{sec:OPTLPD}{\var{OPT\_LPD}}='yes'~!

    En outre, ce paquetage optionnel fournit des fonctions rudimentaires pour
    le serveur de fichiers. Rudimentaire parce fli4l n'a pas de gestion des
    utilisateurs, et donc le partage réseau est assujettie aux restrictions. Il
    aurait besoin d'un serveur de fichier plus mature, il vaudrait mieux utiliser~:

        \altlink{http://www.eisfair.org/}

    Ce système intégre une version de Samba toujours à jour, de plus il n'a aucun
    problème avec les grandes partitions, ni avec l'utilisation du contrôleur
    principal de domaine (PDC). La configuration de ce sytème est calqué sur
    fli4l et donc, tout aussi simple à installer. Samba pour fli4l permet
    principalement de configurer les imprimantes plus facilement sous Windows.

    Il est possible d'installer Samba sans NMBD (Serveur de nom NetBios,
    voir \smalljump{sec:OPTNMBD}{\var{OPT\_NMBD}}), les deux paquetages réunis
    utilisent beaucoup d'espace et vous ne pourrez plus ajouter d'autres programmes
    optionnels sur une disquette de 1,44 Mo.

    C'est pour cela que le programme est fournis avec des paquetages-opt
    individuel - \var{OPT\_\-SAMBA} et \smalljump{sec:OPTNMBD}{\var{OPT\_NMBD}}.
    Dans ce gros paquetage, il y a aussi le paquetage \smalljump{sec:OPTLPD}{\var{OPT\_LPD}}
    il peut être activé individuellement. Une exception, pour le paquetage
    \var{OPT\_\-NMBD} il ne fonctionne pas sans \var{OPT\_\-SAMBA}.

    Sous Windows, si vous voulez renoncer à l'affichage dans les favoris réseau vous
    devez paramétrer la variable sur \var{OPT\_\-SAMBA}='yes' et \smalljump{sec:OPTNMBD}{\var{OPT\_NMBD}}='no'.
    Le partage de l'imprimante ne sera pas affichée, néanmoins son accès sera
    possible dans Windows si vous connaissez le chemin exact. Si dessous vous
    avez une description plus précise pour "installer un client SMB pour Windows
    avec Samba activé (OPT\_SAMBA='yes')".

    Pour ceux qu'ils ne veulent pas abandonner l'affichage dans l'environnement
    réseau, vous devez paramétrer les variables
\begin{example}
\begin{verbatim}
    OPT_SAMBA='yes' et OPT_NMBD='yes'.
\end{verbatim}
\end{example}

	Au sujet du firewall (ou pare-feu)~: si vous activez la variable
	\verb+PF_INPUT_ACCEPT_DEF='yes'+ (ou \verb+PF6_INPUT_ACCEPT_DEF='yes'+ pour IPv6)
	les règles dans la chaîne INPUT seront installées, ainsi les ports pour Samba
	(137--139 et 445) seront configurés de manière appropriée pour qu'ils soient
	accessibles sur le réseau, (voir les variables \jump{SAMBABINDALL}{\var{SAMBA\_BIND\_ALL}},
	\jump{SAMBABINDIPV4x}{\var{SAMBA\_BIND\_IPV4\_x}} et
	\jump{SAMBABINDIPV6x}{\var{SAMBA\_BIND\_IPV6\_x}}). Si vous ne l'activez pas
	\verb+PF_INPUT_ACCEPT_DEF='no'+ (ou \verb+PF6_INPUT_ACCEPT_DEF='no'+ pour IPv6)
	vous devez veiller à ce que les ordinateurs puissent accéder indépendamment au
	serveur Samba à partir de votre réseau.


\begin{description}
\config{SAMBA\_WORKGROUP}{SAMBA\_WORKGROUP}{SAMBAWORKGROUP}

    Pour que le partage de l'imprimante soit visible, vous devez définir un
    groupe de travail dans Windows, le groupe de travail pour Samba doit
    correspondre au groupe de travail défini dans Windows. Donc, si le groupe
    de travail dans Windows est "workgroup", vous devez définir cette variable
    comme ceci~:

\begin{example}
\begin{verbatim}
    SAMBA_WORKGROUP='workgroup'
\end{verbatim}
\end{example}

    C'est aussi le paramètre par défaut.


\config{SAMBA\_TRUSTED\_NETS}{SAMBA\_TRUSTED\_NETS}{SAMBATRUSTEDNETS}

    Quels sont les réseaux autorisant l'accés à Samba~?

    Cette variable est utilisée pour paramétrer les réseaux qui autorise
    l'accés à Samba.
    Le réglage de Samba prendra en compte les paramétrés des réseaux internes
    et des réglages de la base fli4l lors de la création d'une nouvelle
    configuration. Pour des raisons de sécurité, seuls les ordinateurs de ces
    réseaux auront un accès autorisé. Il suffit de préciser ici les réseaux
    supplémentaires. Le réglage sera dans le format suivant.

\begin{example}
\begin{verbatim}
    NUMERO DU RESEAU/NUMERO DU MASQUE DE SOUS-RESEAU
\end{verbatim}
\end{example}

    Par exemple pour un réseau de la forme 192.168.x.0~:

\begin{example}
\begin{verbatim}
     SAMBA_TRUSTED_NETS='192.168.6.0/24'
\end{verbatim}
\end{example}

    Paramètre par défaut~: \var{SAMBA\_\-TRUSTED\_\-NETS}=''


\config{SAMBA\_LOG}{SAMBA\_LOG}{SAMBALOG} Enregistre les erreurs dans les
fichiers log.smb et log.nmb~: 'yes' ou 'no'

    Cette variable est utilisée pour paramétrer l'enregistrement des partages
    dans le fichier log.smb et log.nmb. Dans quel répertoire ces fichiers sont
    écrits~? Il sera déterminé dans la variable \smalljump{SAMBALOGDIR}{\var{SAMBA\_LOGDIR}}
    et par rapport au type d'installation que vous avez choisi pour fli4l. Cette
    variable doit être réglée sur 'yes' uniquement pour le débogage, car les
    fichiers log (ou journal) sont écrits en fonction du réglage de \smalljump{SAMBALOGDIR}{\var{SAMBA\_LOGDIR}}
    si l'option sélectionnée est un disque RAM, il peut y avoir un risque de
    débordement à un momment donné. \var{SAMBA\_\-LOG} s'applique aux paquetages \smalljump{sec:OPTSAMBA}{\var{OPT\_SAMBA}}
    et \smalljump{sec:OPTNMBD}{\var{OPT\_NMBD}}, je l'ai déjà dit, \var{OPT\_\-NMBD}
    ne fonction pas sans \var{OPT\_\-SAMBA}. Si vous paramétrez \var{SAMBA\_\-LOG}='no',
    vous deviez lire absolument la paragraphe suivant sur les observations de la
    variable \smalljump{SAMBALOGDIR}{\var{SAMBA\_LOGDIR}}.

    Paramètre par défaut~: \var{SAMBA\_\-LOG}='no'


\config{SAMBA\_LOGDIR}{SAMBA\_LOGDIR}{SAMBALOGDIR}

    Répertoire log pour les fichiers log.smb et log.nmb

    Dans cette variable vous indiquez le répertoire dans lequel les fichiers
    log.smb et log.nmb seront enregistrés. Soit cette variable reste vide, soit
    vous indiquez un chemin d'accès absolu. Le répertoire devra être accessible
    en écriture et doit déjà exister dans l'arborescence. Le chemin d'accès
    commence toujours par le symbole '/'. Si la variable est vide, il y aura
    une écriture dans la partition de donnée monter sous /data qui doit être
    pré définie, où les fichiers log seront stockés~:

    S'il n'existe pas de partition de donnée monter sous /data pour l'écriture
    (généralement tous les types d'installation, sauf le type B) et si la variable
    \var{SAMBA\_\-LOGDIR} est vide, les données seront enregistrées dans /var/log
    (du disque RAM).

    S'il existe une partition de donnée monter sous /data pour l'écriture
    (typique à l'installation du type B) et si la variable \var{SAMBA\_\-LOGDIR}
    est vide, les données seront enregistrées dans /data (la partition de données).

    Si vous avez paramétré la variable, les fichiers log.smb et log.nmb seront
    écrit dans le répertoire spécifié si ce répertoire est accessible en écriture.
    Il n'y a aucun sens à intégrer ici une partition optionnelle en lecture seule.
    Si les fichiers log ne peuvent pas être écrits, Samba ne démarra pas. Donc,
    si vous avez une raison valable vous devez réfléchir très sérieusement avant
    de remplir la variable \var{SAMBA\_\-LOGDIR}.

    Si vous avez défini la variable \smalljump{SAMBALOG}{\var{SAMBA\_LOG}}='no',
    la variable \var{SAMBA\_\-LOGDIR} doit être vide ou faisant référence à un
    répertoire qui sera basé sur un fichiers système Linux comme (minix, ext2,
    ext3), étant donné que \smalljump{SAMBALOG}{\var{SAMBA\_LOG}}='no' log.smb
    et log.nmb seront envoyé vers /dev/null, ce lien symbolique fonctionnera
    alors convenablement.
    Donc, si vous ne voulez absolument pas de fichiers log pour Samba et pas de
    liens vers /var/log, vous devez par exemple paramétrer

\begin{example}
\begin{verbatim}
     SAMBA_LOG='no'
     SAMBA_LOGDIR='/tmp'
\end{verbatim}
\end{example}

    Dans la plupart des cas, \var{SAMBA\_\-LOGDIR}='' est la bonne décision,
    c'est aussi le réglage par défaut.

    Paramètre par défaut~: \var{SAMBA\_\-LOGDIR}=''


\config{SAMBA\_TDBPATH}{SAMBA\_TDBPATH}{SAMBATDBPATH}

    Dans cette variable vous indiquez le répertoire persistante pour stocker
	les fichiers dits TDB pour le serveur Samba. Ces fichiers seront entre autres
	des pilotes d'imprimante qui ont été téléchargés sur le serveur fli4l, reportez-vous
	à la section \jump{sec:OPTSAMBAPOINTANDPRINT}{"Point'n'Print"} pour plus de
	détails. Ces  pilotes d'imprimante seront placés dans ce répertoire. Vous pouvez
	indiquer 'auto' dans cette variable, dans ce cas fli4l va créer et monter un
	répertoire sur un support de stockage persistant voir \texttt{/var/lib/persistent/samba/db}.

    Paramètre par défaut~: \verb+SAMBA_TDBPATH='auto'+

    Exemple~: \verb+SAMBA_TDBPATH='/data/samba/tdb'+


\config{SAMBA\_SPOOLPATH}{SAMBA\_SPOOLPATH}{SAMBASPOOLPATH}

    Dans cette variable vous indiquez le répertoire spool (ou répertoire tampon) pour
	les travaux d'impression entrants. Lors de l'impression via le protocole Samba, les
	données d'impression seront d'abord mis en attente dans le répertoire défini ici avant
	d'être transmis au serveur d'impression LPD. Vous pouvez indiquer 'auto' dans cette
	variable, ensuite fli4l va créer et monter un répertoire spool sur un support de
	stockage persistant voir \texttt{/var/lib/persistent/samba/spool}.

    Notez s'il vous plaît, que le contenu de ce répertoire sera effacé lors du démarrage
	de votre routeur fli4l. Donc, vous ne devez pas spécifier dans ce répertoire d'autres
	données importantes~!

    Paramètre par défaut~: \verb+SAMBA_SPOOLPATH='auto'+

    Exemple~: \verb+SAMBA_SPOOLPATH='/data/samba/spool'+


\config{SAMBA\_BIND\_ALL}{SAMBA\_BIND\_ALL}{SAMBABINDALL}

    Si vous activez cette variable \verb+SAMBA_BIND_ALL='yes'+ le serveur
	Samba "écoutera" les réquêtes sur \emph{toutes} les interfaces disponibles du
	réseau local. Si vous ne souhaitez pas que Samba écoute sur tous le réseau, vous
	devez indiquer \verb+SAMBA_BIND_ALL='no'+; En outre, vous devez configurer l'ensemble
	du réseaux avec les variables \var{SAMBA\_BIND\_IPV4\_\%} et \var{SAMBA\_BIND\_IPV6\_\%}
	pour que le serveur Samba puisse répondre aux réquêtes sur le réseau.

    Paramètre par défaut~: \verb+SAMBA_BIND_ALL='no'+


\configlabel{SAMBA\_BIND\_IPV4\_N}{SAMBABINDIPV4N}
\config{SAMBA\_BIND\_IPV4\_x}{SAMBA\_BIND\_IPV4\_x}{SAMBABINDIPV4x}

    Si vous avez indiqué \verb+SAMBA_BIND_ALL='no'+, vous devez configurer l'ensemble
	du réseaux IPv4 sur lesquels le serveur Samba répondra aux réquêtes.

    Exemple~:
\begin{example}
\begin{verbatim}
    SAMBA_BIND_IPV4_N='1'
    SAMBA_BIND_IPV4_1='IP_NET_1'
\end{verbatim}
\end{example}

    Paramètre par défaut~: \verb+SAMBA_BIND_IPV4_N='0'+


\configlabel{SAMBA\_BIND\_IPV6\_N}{SAMBABINDIPV6N}
\config{SAMBA\_BIND\_IPV6\_x}{SAMBA\_BIND\_IPV6\_x}{SAMBABINDIPV6x}

    Si vous avez indiqué \verb+SAMBA_BIND_ALL='no'+, vous devez configurer l'ensemble
	du réseaux IPv6 sur lesquels le serveur Samba répondra aux réquêtes.

    Exemple~:
\begin{example}
\begin{verbatim}
    SAMBA_BIND_IPV6_N='1'
    SAMBA_BIND_IPV6_1='IPV6_NET_1'
\end{verbatim}
\end{example}

    Paramètre par défaut~: \verb+SAMBA_BIND_IPV6_N='0'+


\config{LPD\_PARPORT\_x\_SAMBA\_NAME}{LPD\_PARPORT\_x\_SAMBA\_NAME}{LPDPARPORTxSAMBANAME}

    Dans cette variable vous indiquez le nom de l'imprimante fli4l qui peut
    être installée dans l'environnement réseau, les imprimantes sont branchées
    sur x port parallèle (\smalljump{LPDPARPORTxIO}{\var{LPD\_PARPORT\_x\_IO}}).
    Cela nécessite bien sûr que la variable

\begin{example}
\begin{verbatim}
    OPT_NMBD='yes'
\end{verbatim}
\end{example}

    soit activé, autrement rien ne s'affichera dans l'environnement réseau. Le
    nom que vous indiquez peut contenir plus de 8 caractères et vous pourrez
    mélanger des lettres et des chiffres. Le tréma et les caractères spéciaux
    comme ä, ö, ü, ß, \_, @, etc, ne sont pas autorisés~!

    Si la variable n'est pas paramétrée, un nom par défaut sera utilisé pour le
    nom de l'imprimante. Le nom par défaut pour l'imprimante locale sur le port
    parallèle sera indiqué prx, le x sera remplacé soit par 1, 2, 3, etc, donc
    pour le premier, deuxième, troisième, etc, port parallèle de connexion disponible.

    Paramètre par défaut~: \var{LPD\_\-PARPORT\_\-1\_\-SAMBA\_\-NAME}=''


\config{LPD\_PARPORT\_x\_SAMBA\_NET}{LPD\_PARPORT\_x\_SAMBA\_NET}{LPDPARPORTxSAMBANET}

    Cette variable est utilisée pour contrôler les hôtes qui seront autorisés
    à utiliser l'imprimante locale sur x port parallèle de fli4l. Vous pouvez
    utiliser cette variable pour restreindre individuellement l'accès aux
    ordinateurs ou à différents sous-réseaux. Par défaut, la variable est vide.
    Elle permet à tous les ordinateurs du réseau interne (y compris tous les
    sous-réseaux) d'accéder à la x-ième imprimante (voir \smalljump{LPDPARPORTxIO}{\var{LPD\_PARPORT\_x\_IO}}).
    Si vous avez besoin de deux imprimantes locales sur fli4l, connectées sur
    deux ports parallèles vous devez indiquer \textbf{\var{LPD\_\-PARPORT\_\-1\_\-SAMBA\_\-NET}}
    et \textbf{\var{LPD\_\-PARPORT\_\-2\_\-SAMBA\_\-NET}}

    Pour paramétrer la variable~:

\begin{itemize}
\item Vous pouvez saisie des adresses IP sur une seule ligne les l'une derrière
        les l'autre en les séparant par un espace, par exemple~:

\begin{example}
\begin{verbatim}
    LPD_PARPORT_1_SAMBA_NET='192.168.6.2 192.168.0.1'
\end{verbatim}
\end{example}

    Si vous avez deux réseaux sous la forme 192.168.141.0/ 255.255.255.0 et
    192.168.142.0/ 255.255.255.0 pour partager une imprimante sur le premier
    port parallèle~:

\item vous pouvez saisir une plage d'adresses IP sans la partie hôte~:

\begin{example}
\begin{verbatim}
    LPD_PARPORT_1_SAMBA_NET='192.168.141. 192.168.142.'
\end{verbatim}
\end{example}

    ou mieux

\begin{example}
\begin{verbatim}
    LPD_PARPORT_1_SAMBA_NET='192.168.'
\end{verbatim}
\end{example}

    Il est important de faite attention au point à la fin de l'adresse~!
\end{itemize}

    Paramètre par défaut~: \var{LPD\_\-PARPORT\_\-1\_\-SAMBA\_\-NET}=''


\config{LPD\_USBPORT\_x\_SAMBA\_NAME}{LPD\_USBPORT\_x\_SAMBA\_NAME}{LPDUSBPORTxSAMBANAME}

    Dans cette variable vous indiquez le nom de l'imprimante fli4l qui peut
    être installée dans l'environnement réseau, les imprimantes sont branchées
    sur x port USB. Cela nécessite bien sûr que la variable

\begin{example}
\begin{verbatim}
    OPT_NMBD='yes'
\end{verbatim}
\end{example}

    soit activé, autrement rien ne s'affichera dans l'environnement réseau. Le
    nom que vous indiquez peut contenir plus de 8 caractères et vous pourrez
    mélanger des lettres et des chiffres. Le tréma et les caractères spéciaux
    comme ä, ö, ü, ß, \_, @, etc, ne sont pas autorisés~!

    Si la variable n'est pas paramétrée, un nom par défaut sera utilisé pour le
    nom de l'imprimante. Le nom par défaut pour l'imprimante locale sur le port
    USB sera indiqué usbprx, le x sera remplacé soit par 1, 2, 3, etc, donc
    pour le premier, deuxième, troisième, etc, port USB de connexion disponible.

    Paramètre par défaut~: \var{LPD\_\-USBPORT\_\-1\_\-SAMBA\_\-NAME}=''


\config{LPD\_USBPORT\_x\_SAMBA\_NET}{LPD\_USBPORT\_x\_SAMBA\_NET}{LPDUSBPORTxSAMBANET}

    Cette variable est utilisée pour contrôler les hôtes qui seront autorisés
    à utiliser l'imprimante locale sur x port USB de fli4l. Vous pouvez utiliser
    cette variable pour restreindre individuellement l'accès aux ordinateurs ou
    à différents sous-réseaux. Par défaut, la variable est vide. Elle permet à
    tous les ordinateurs du réseau interne (y compris tous les sous-réseaux)
    d'accéder à la x-ième imprimante USB.
    Si vous voulez enregistrer explicitement des hôtes ou des réseaux pour
    partager les imprimantes, voir les paramètres de la variable
    \smalljump{LPDPARPORTxSAMBANET}{\var{LPD\_PARPORT\_x\_SAMBA\_NET}} déjà décrit.

    Paramètre par défaut~: \var{LPD\_\-USBPORT\_\-1\_\-SAMBA\_\-NET}=''


\config{LPD\_REMOTE\_x\_SAMBA\_NAME}{LPD\_REMOTE\_x\_SAMBA\_NAME}{LPDREMOTExSAMBANAME}

    Dans cette variable vous indiquez le nom de l'imprimante fli4l qui sera
    installée dans l'environnement réseau, les imprimantes distante sont branchées
    sur x \smalljump{LPDREMOTExIP}{\var{LPD\_REMOTE\_x\_IP}}. Cela nécessite
    bien sûr que la variable

\begin{example}
\begin{verbatim}
    OPT_NMBD='yes'
\end{verbatim}
\end{example}

    soit activé, autrement rien ne s'affichera dans l'environnement réseau. Le
    nom que vous indiquez peut contenir plus de 8 caractères et vous pourrez
    mélanger des lettres et des chiffres. Le tréma et les caractères spéciaux
    comme ä, ö, ü, ß, \_, @, etc, ne sont pas autorisés~!

    Si la variable n'est pas paramétrée, un nom par défaut sera utilisé pour le
    nom de l'imprimante. Le nom par défaut pour l'imprimante distante sera
    indiqué reprx, le x sera remplacé soit par 1, 2, 3, etc, donc pour
    la premiere, deuxième, troisième, etc, connexion distante pour l'imprimante.

    Paramètre par défaut~: \var{LPD\_\-REMOTE\_\-1\_\-SAMBA\_\-NAME}=''


\config{LPD\_REMOTE\_x\_SAMBA\_NET}{LPD\_REMOTE\_x\_SAMBA\_NET}{LPDREMOTExSAMBANET}

    Cette variable est utilisée pour contrôler les hôtes qui seront autorisés
    à utiliser l'imprimante distante depuis fli4l. Vous pouvez utiliser cette
    variable pour restreindre individuellement l'accès aux ordinateurs ou à
    différents sous-réseaux. Par défaut, la variable est vide. Elle permet à
    tous les ordinateurs du réseau interne de fli4l (y compris tous les sous-réseaux)
    d'accéder à la x-ième imprimante distante (voir \smalljump{LPDREMOTExIP}{\var{LPD\_REMOTE\_x\_IP}}).
    Si vous voulez enregistrer explicitement des hôtes ou des réseaux pour
    partager les imprimantes, voir les paramètres de la variable
    \smalljump{LPDPARPORTxSAMBANET}{\var{LPD\_PARPORT\_x\_SAMBA\_NET}} déjà décrit.

    Paramètre par défaut~: \var{LPD\_\-REMOTE\_\-1\_\-SAMBA\_\-NET}=''


\config{LPD\_SMBREMOTE\_x\_SAMBA\_NAME}{LPD\_SMBREMOTE\_x\_SAMBA\_NAME}{LPDSMBREMOTExSAMBANAME}

    Dans cette variable vous indiquez le nom de l'imprimante fli4l qui peut
    être installée dans l'environnement réseau, les imprimantes SMB distante
    sont branchées sur x \smalljump{LPDSMBREMOTExSERVER}{\var{LPD\_SMBREMOTE\_x\_SERVER}}.
    Cela nécessite bien sûr que la variable

\begin{example}
\begin{verbatim}
    OPT_NMBD='yes'
\end{verbatim}
\end{example}

    soit activé, autrement rien ne s'affichera dans l'environnement réseau. Le
    nom que vous indiquez peut contenir plus de 8 caractères et vous pourrez
    mélanger des lettres et des chiffres. Le tréma et les caractères spéciaux
    comme ä, ö, ü, ß, \_, @, etc, ne sont pas autorisés~!

    Si la variable n'est pas paramétrée, un nom par défaut sera utilisé pour le
    nom de l'imprimante. Le nom par défaut pour l'imprimante SMB distante sera
    indiqué smbreprx, le x sera remplacé soit par 1, 2, 3, etc, donc pour
    la premiere, deuxième, troisième, etc, connexion distante SMB pour l'imprimante.

    Paramètre par défaut~: \var{LPD\_\-SMBREMOTE\_\-1\_\-SAMBA\_\-NAME}=''


\config{LPD\_SMBREMOTE\_x\_SAMBA\_NET}{LPD\_SMBREMOTE\_x\_SAMBA\_NET}{LPDSMBREMOTExSAMBANET}

    Cette variable est utilisée pour contrôler les hôtes qui seront autorisés
    à utiliser l'imprimante SMB distante depuis fli4l. Vous pouvez utiliser cette
    variable pour restreindre individuellement l'accès aux ordinateurs ou à
    différents sous-réseaux. Par défaut, la variable est vide. Elle permet à
    tous les ordinateurs du réseau interne de fli4l (y compris tous les sous-réseaux)
    d'accéder à la x-ième imprimante SMB distante (voir \smalljump{LPDSMBREMOTExSERVER}{\var{LPD\_SMBREMOTE\_x\_SERVER}}).
    Si vous voulez enregistrer explicitement des hôtes ou des réseaux pour
    partager les imprimantes, voir les paramètres de la variable
    \smalljump{LPDPARPORTxSAMBANET}{\var{LPD\_PARPORT\_x\_SAMBA\_NET}} déjà décrit.

    Paramètre par défaut~: \var{LPD\_\-SMBREMOTE\_\-1\_\-SAMBA\_\-NET}=''

\end{description}


\begin{description}
\config{SAMBA\_ADMINIP}{SAMBA\_ADMINIP}{SAMBAADMINIP}

        Si vous indiquez dans cette variable une adresse IP ou une plage d'adresses
        du réseau local, les ordinateurs correspondant auront un accès complet
        sur le disque virtuel de fli4l par le réseau.
        Lorsque vous activez la variable \var{OPT\_NMBD='yes'} vous pouvez accéder
        à l'ordinateur fli4l via l'environnement réseau de Windows.

        Voici un exemple pour l'adresse IP 192.168.6.2~:

\begin{example}
\begin{verbatim}
    SAMBA_ADMINIP='192.168.6.2'
\end{verbatim}
\end{example}

        Si vous permettez à plusieurs ordinateurs un tel accès, vous aurez
        plusieurs options~:

        - Vous pouvez saisie des adresses IP sur une seule ligne les l'une
        derrière les l'autre en les séparant par un espace, par exemple~:

          \var{SAMBA\_ADMINIP='192.168.6.2 192.168.6.3'}

        - Saisir une plage d'adresses IP sans la partie hôte~:

          \var{SAMBA\_ADMINIP='192.168.'}

          Il est important de faite attention au point à la fin de l'adresse~!

        Cette variable doit être configurée pour des raisons de sécurité et
        pour la recherche d'erreur~!

        Par défaut, le disque virtuel de fli4l dans un environnement réseau est
        non visible et non accessible.

        Paramètre par défaut~: \var{SAMBA\_ADMINIP=''}
\end{description}


\begin{description}
\config{SAMBA\_SHARE\_N}{SAMBA\_SHARE\_N}{SAMBASHAREN}

    Création d'un certain nombre de partage réseau~: par ex. '2'

        Dans la variable \var{SAMBA\_SHARE\_N} vous indiquez le nombre de partage
        réseau à utiliser. Si le partage réseau n'existent pas, il sera créé
        automatiquement, et s'il existe, il sera simplement utilisé. La création
        du partage réseau est normalement utile en conjonction avec un support
        monter comme un disque dur, un lecteur de CD-ROM ou un disque Compact-Flash
        (voir \var{OPT\_MOUNT}).

        Vous pouvez voir dans les variables ci-dessous 2 partages réseaux

        \var{SAMBA\_SHARE\_1\_NAME}

        \var{SAMBA\_SHARE\_1\_RW}

        \var{SAMBA\_SHARE\_1\_BROWSE}

        \var{SAMBA\_SHARE\_1\_PATH}

        \var{SAMBA\_SHARE\_1\_NET}

        et

        \var{SAMBA\_SHARE\_2\_NAME}

        \var{SAMBA\_SHARE\_2\_RW}

        \var{SAMBA\_SHARE\_2\_BROWSE}

        \var{SAMBA\_SHARE\_2\_PATH}

        \var{SAMBA\_SHARE\_2\_NET}

        elle sont configurées avec des valeurs significatives.

        Paramètre par défaut~: \var{SAMBA\_SHARE\_N='0'}

\end{description}


\begin{description}
\config{SAMBA\_SHARE\_x\_NAME}{SAMBA\_SHARE\_x\_NAME}{SAMBASHARE_NAMEX}

        Dans la variable \var{SAMBA\_SHARE\_x\_NAME} vous indiquez le x-ième
        nom pour le partage réseau. Vous pouvez accéder au nom de partage
        réseau si vous avez activé la variable \var{OPT\_NMBD} le nom de partage
        sera visible sur les ordinateurs Windows dans l'environnement réseau,
        (voir aussi \var{SAMBA\_SHARE\_x\_BROWSE} ci-dessous).

        Vous pouvez indiquer pour le nom 12 caractères avec le tréma sous Windows,
        néanmoins il est préférable d'indiquer 8 caractères sans accent sous Dos,
        par exemple

\begin{example}
\begin{verbatim}
    SAMBA_SHARE_1_NAME='share1'
\end{verbatim}
\end{example}

        Le nom du partage doit être unique, il ne doit pas être
        dupliqué. Ce nom sera automatiquement ajouté au chemin

        \var{SAMBA\_SHARE\_x\_PATH}

        du répertoire de fli4l. il sera rajouté dans le chemin créé dans cette
        variable, le nom du répertoire de partage sera nommé "share1" s'il
        n'existe pas. Il est impératif que le chemin dans la partition soit monter
        en écriture. Si ce n'est pas le cas, il y aura une erreur lors du boot
        de fli4l. Si Le répertoire existe déjà, il ne sera pas remplacé, les
        données déjà enregistrées seront conservées.

        Paramètre par défaut~: \var{SAMBA\_SHARE\_1\_NAME='share1'}

\end{description}


\begin{description}
\config{SAMBA\_SHARE\_x\_RW}{SAMBA\_SHARE\_x\_RW}{SAMBASHARERWX}

    Ce répertoire doit il être accessible en écriture~: 'yes' ou 'no'

        Dans la variable \var{SAMBA\_SHARE\_x\_RW} vous devez indiquer si le
        x-ième nom de partage doit être accessible en écriture.

        Si vous avez sélectionné 'no', les fichiers pourront être lus à partir
        de ce partage, mais pas sauvegardés. Cela est particulièrement utile
        pour les fichiers que vous souhaitez rendre disponibles aux autres
        utilisateur, mais que vous souhaitez éviter que ces fichiers soit
        modifiés, voire supprimés.

        Si vous avez sélectionné 'yes', tous se qui est configuré dans la variable
        ci-dessous pour le partage de répertoire

        \var{SAMBA\_SHARE\_x\_NET]}

        c'est à dire, les adresses IP ou de réseaux ou si elle est vide, tous
        les ordinateurs du réseau interne (y compris tous les sous-réseaux)
        sera en lecture et en écriture.

        Paramètre par défaut~: \var{SAMBA\_SHARE\_1\_RW='yes'}

\end{description}


\begin{description}
\config{SAMBA\_SHARE\_x\_BROWSE}{SAMBA\_SHARE\_x\_BROWSE}{SAMBASHAREBROWSEX} (vous avez besoin \var{OPT\_NMBD='yes'})

    Si le x-ième nom de partage doit être visible~: 'yes' ou 'no'

        Dans la variable \var{SAMBA\_SHARE\_x\_BROWSE} vous indiquez le x-ième
        nom de partage qui doit être visible ou non dans l'environnement réseau,
        \var{OPT\_NMBD} doit être activée.
        Si vous voulez empêcher que d'autres utilisateurs dans l'environnement
        réseau, voient le nom de partage et avoir accès à celui-ci, vous devez
        indiquer

        \var{SAMBA\_SHARE\_x\_BROWSE='no'}

        Les utilisateurs savent que le partage existe, ils pourront toujours
        y accéder, par exemple vous indiquez dans Démarrer/Exécuter

            $\backslash\backslash$fli4l$\backslash$nom de partage

        puis Entré. bien sûr \flqq{}fli4l\frqq{} doit est remplacé par le nom de
        votre routeur fli4l - sans les guillemets - et "nom de partage" doit est
        remplacé par le nom que vous avez entré dans \var{SAMBA\_SHARE\_x\_NAME}.

        Paramètre par défaut~: \var{SAMBA\_SHARE\_1\_BROWSE='yes'}

\end{description}


\begin{description}
\config{SAMBA\_SHARE\_x\_PATH}{SAMBA\_SHARE\_x\_PATH}{SAMBASHAREPATHX}

     Nom de partage vers le x-ième chemin d'accés

        Dans la variable \var{SAMBA\_SHARE\_x\_PATH} vous indiquez le x-ième
        chemin pour accéder au nom du répertoire de partage.

        Voici un exemple des variables, il faut que les variables du
        paquetage \var{OPT\_\-HD} soit configurées.

        \var{OPT\_EXTMOUNT='yes'}

        \var{EXTMOUNT\_N='1'}

        \var{EXTMOUNT\_1\_VOLUMID='hda4'}

        \var{EXTMOUNT\_1\_MOUNTPOINT='/usr/local/data'}

        \var{EXTMOUNT\_1\_FILESYSTEM='ext2'}
%        \var{MOUNT\_1\_CHECK='yes'}

        \var{EXTMOUNT\_1\_OPTIONS='rw'}

        Si vous avez monter la quatrième partition primaire du premier disque dur
        et avec un système de fichiers et aussi le chemin /usr/local/data,
        maintenant, vous pouvez indiquez le nom de partage avec les variables

        \var{SAMBA\_SHARE\_N='1'}

        \var{SAMBA\_SHARE\_1\_NAME='share1'}

        \var{SAMBA\_SHARE\_1\_RW='yes'}

        \var{SAMBA\_SHARE\_1\_BROWSE='yes'}

        vous pouvez indiquer le chemin

        \var{SAMBA\_SHARE\_1\_PATH='/usr/local/data'}

        dans le chemin /usr/local/data le répertoire "share1" sera créé pour
        le nom de partage. Le nom du répertoire de partage est contenu dans
        la variable

        \var{SAMBA\_SHARE\_1\_NAME}

        dans cette exemple se sera

        share1

        Si le répertoire n'existe pas, il sera créé automatiquement et s'il
        existe, il sera simplement utilisée. Il n'existe actuellement aucun
        moyen de supprimer le répertoire une fois créé dans samba\_lpd.txt,
        si vous avez fait une erreur de configuration, les fichiers déjà stockés
        seront supprimés. Les fichiers situés dans ce répertoire peuvent être
        visible en configurant

        \var{OPT\_NMBD}

        vous pourrez ajouter ou supprimer des fichiers via l'explorateur,
        avec la variable

        \var{SAMBA\_SHARE\_1\_RW}

        elle doit être définie en écriture, vous pourrez aussi accéder au
        répertoire par ligne de commande.

        Paramètre par défaut~: \var{SAMBA\_SHARE\_1\_PATH='/usr/local/data'}

\end{description}


\begin{description}
\config{SAMBA\_SHARE\_x\_NET}{SAMBA\_SHARE\_x\_NET}{SAMBASHARENETX}

        Cette variable est utilisée pour contrôler les hôtes qui pourront accéder
        au x-ième nom de partage. Vous pouvez l'utiliser pour restreindre
        individuellement l'accès aux ordinateurs ou des différents sous-réseaux.
        Par défaut, la variable est vide. Pour accéder au partage vous devez
        indiquer tous les ordinateurs du réseau interne (y compris tous les
        sous-réseaux).

        La variable \var{SAMBA\_ADMINIP} peut aussi être configurée.

        - Vous pouvez saisie des adresses IP sur une seule ligne les l'une derrière
        les l'autre en les séparant par un espace, par exemple~:

          \var{SAMBA\_SHARE\_1\_NET='192.168.6.2 192.168.0.1'}

        Si vous avez deux réseaux sous la forme 192.168.141.0/255.255.255.0 et
        192.168.142.0/255.255.255.0

        - vous pouvez saisir la plage d'adresses IP sans la partie hôte~:

          \var{SAMBA\_SHARE\_1\_NET='192.168.141. 192.168.142.'}

          ou mieux

          \var{SAMBA\_SHARE\_1\_NET='192.168.'}

          Il est important de faite attention au point à la fin de l'adresse~!

        Paramètre par défaut~: \var{SAMBA\_SHARE\_1\_NET=''}

\end{description}


\begin{description}
\config{SAMBA\_CDROM\_N}{SAMBA\_CDROM\_N}{SAMBACDROMN}

    Création d'un nombre déterminé de lecteur CD-ROM dans le voisinage réseau, par ex. '2'

        Dans la variable \var{SAMBA\_CDROM\_N} vous pouvez indiquer le nombre
        de lecteur CD-ROM installé sur le routeur pour le voisinage réseau.
        Les variables suivantes et les extensions appartenant au script du
        rc.samba ont été créées pour partager sur le réseau le CDROMs, si
        vous avez des erreurs soyez un peu indulgent.
        Vous avez peut être essayé la version 2.0pre2 pour le partage réseau du
        lecteur CD-ROM, il y avait des problèmes~! Le CD-ROM n'était pas monter
        ou le chemin d'accès était incorrect pour le partage, ce qui bien sûr
        s'est mal passé.
        Les nouveaux scripts sont en relation direct avec les variables, les CDROMs
        seront déjà monter avec le point de montage défini dans la variable \var{OPT\_MOUNT}
        ou si le CD-ROM n'est pas encore monter avec le point de montage, il
        sera activé par la suite.

        Avec la dernière version, le lecteur est accessible uniquement (par
        l'intermédiaire du voisinage réseau) et en fonction des besoins

        /mnt/cdromx

        Le x représente le x-ième montage de lecteur de CD-ROM. Avec ce
        paramètre il faut faire en sorte que vos propres points de montage
        n'entrent pas en collision. Si personne n'accède à ce partage après un
        certain temps, le lecteur sera automatiquement démonté. Ainsi, vous pouvez
        démonter le CD-ROM, sans avoir à le faire manuellement, c'est surtout
        utile pour un serveur de CDROM avec plusieurs lecteurs, vous pourrez
        changer plus fréquemment le CD.

        Si vous voulez paramétrer 2 partages dans \var{SAMBA\_CDROM\_N}, les
        variables suivant~:

        \var{SAMBA\_CDROM\_1\_DEV}
        \var{SAMBA\_CDROM\_1\_NET}

        et

        \var{SAMBA\_CDROM\_2\_DEV}
        \var{SAMBA\_CDROM\_2\_NET}

        doivent être présentes et paramétrées avec des valeurs significatives.

        Paramètre par défaut~: \var{SAMBA\_CDROM\_N='0'}

\end{description}


\begin{description}
\config{SAMBA\_CDROM\_x\_DEV}{SAMBA\_CDROM\_x\_DEV}{SAMBACDROMDEVX}

    Nom du lecteur de CDROM par exemple~: 'sdc'

        Vous indiquez dans cette variable, le matétiel qui sera partagé. Les
        conventions pour les noms de périphériques peuvent être trouvées dans
        la documentation de \var{OPT\_mount}.

        \var{SAMBA\_CDROM\_1\_DEV='hdc'}

        Paramètre par défaut~: \var{SAMBA\_CDROM\_1\_DEV='hdc'}

\end{description}


\begin{description}
\config{SAMBA\_CDROM\_x\_NET}{SAMBA\_CDROM\_x\_NET}{SAMBACDROMNETX}

        Cette variable est utilisée pour contrôler les hôtes qui seront autorisés
        à utiliser le x-ième lecteur de CDROM sur fli4l. Vous pouvez utiliser cette
        variable pour restreindre individuellement l'accès aux ordinateurs ou à
        différents sous-réseaux. Par défaut, la variable est vide. Elle permet à
        tous les ordinateurs du réseau interne de fli4l (y compris tous les
        sous-réseaux) d'accéder au x-ième lecteur de CD-ROM sur fli4l. Si vous
        avez besoin de deux réseaux pour se connecter sur le lecteurs de CD-ROM
        de fli4l vous devez indiquez~:

          \var{SAMBA\_CDROM\_1\_NET}

          et

          \var{SAMBA\_CDROM\_2\_NET}

        La variable \var{SAMBA\_ADMINIP} peut aussi être paramétrée.

        - Vous pouvez saisie des adresses IP sur une seule ligne les l'une derrière
        les l'autre en les séparant par un espace, par exemple~:

          \var{SAMBA\_CDROM\_1\_NET='192.168.6.2 192.168.0.1'}

        Si vous avez deux réseaux sous la forme 192.168.141.0/255.255.255.0 et
        192.168.142.0/255.255.255.0 et un lecteur de CD-ROM

        - vous pouvez saisir une plage d'adresses IP sans la partie hôte~:

          \var{SAMBA\_CDROM\_1\_NET='192.168.141. 192.168.142.'}

          ou mieux

          \var{SAMBA\_CDROM\_1\_NET='192.168.'}

          Il est important de faite attention au point à la fin de l'adresse~!

        Paramètre par défaut~: \var{SAMBA\_CDROM\_1\_NET=''}

\end{description}


\marklabel{sec:OPTSAMBATOOLS}{\subsection {OPT\_SAMBATOOLS - Outils spéciaux pour Samba}}
\configlabel{OPT\_SAMBATOOLS}{OPTSAMBATOOLS}
    Installer des outils supplémentaires pour Samba~: 'yes' ou 'no'

    Il m'a été demandé à plusieurs reprises s'il était possible d'envoyer des
    messages aux clients Windows ou de monter les ordinateurs Windows pour le
    partage réseau sur fli4l, j'ai décidé de fournir des outils appropriés. Vous
    pouvez installer ces outils supplémentaires pour Samba.

    Vous pouvez avoir des problèmes en utilisant ces outils sans les connaitre.
    Vous ne savez pas les dangers qui vous menace, par exemple quand vous
    montez un partage pour le réseau fli4l, vous devez éviter le truc, j'ai
	essayé de supprimer certaines sources d'erreur dans les scripts qui gèrent
	le montage et le démontage d'un partage réseau ou d'envoyer un ou à plusieurs
	messages aux clients Windows.

    En outre, vous devez lire le support de description, qui n'est pas de moi~!

    Voici les fichiers supplémentaires suivants~:

\begin{example}
\begin{verbatim}
    smbfs.o
    nls_iso8859-1.o
    nls_cp850.o
    nmblookup
    samba-netsend
    smbclient
    smbstatus
\end{verbatim}
\end{example}

    Le plus important d'entre eux est le script qui est expliqué ici.

    samba-netsend

    Avec ce script, vous pouvez envoyer des messages de manière interactif sur
    des hôtes avec le protocole SMB. Lorque vous aurez démarré le script sur la
    console, il s'affichera les informations suivante~:

\begin{verbatim}
Send Message to SMB Hosts

To which SMB Hosts the message should be send?

Choice 1
--------
All SMB Hosts on configured Subnets on fli4l - type 'all'.

Choice 2
--------
fli4l Samba Clients with active connections - type 'active'.

Choice 3
--------
One ore more active SMB Hosts, type NETBIOS Names
separated with a blank, for instance 'client1 client2':
\end{verbatim}

Comme vous pouvez le voir, vous pouvez sélection une des 3 possibilités dans
la première étape~:

\begin{enumerate}
\item
Envoi des messages à tous les hôtes avec le protocole SMB sur les réseaux configurés
dans ordinateur fli4l. Depuis Samba et avec la création du fichier de configuration

\begin{example}
\begin{verbatim}
   /config/base.txt
\end{verbatim}
\end{example}

l'accès pour envoi des messages se fera sur toutes les cartes réseaux qui sont
configurés dans le fichier ci-dessus et même celles qui sont configurées dans
Samba. Les informations désormais obtenu, sont utilisé pour recherchées les
ordinateurs SMB dans le réseau, l'envoie du message à tous ces ordinateurs reste
à définir (il sera envoyé par l'ordinateur fli4l). Les adresses de diffusion et
les noms des ordinateurs NetBIOS sont calculées pour envoyé le message.

Pour sélectionner ces options,

\begin{example}
\begin{verbatim}
   all
\end{verbatim}
\end{example}

devrat être indiquée dans la console.

\item
Envoyer un message de Samba à tout les clients fli4l avec le lien actif sur
fli4l - de sorte que seule les liens encore ouverts sur fli4l peuvent recevoir
le message de Samba.

pour cela,

\begin{example}
\begin{verbatim}
   active
\end{verbatim}
\end{example}

devrat être indiquée dans la console.

\item
Envoyer un message à un ou plusieurs hôtes SMB actifs. vous devez indiquer les
noms des l'ordinateurs NetBIOS. Plusieurs ordinateurs doivent être spécifiés,
vous devez les séparer par un espace.

Si l'information nécessaire est donnée, nous allons arriver à la deuxième étape~:

Send Message to SMB Hosts

Which Message should be send?
For instance 'fli4l-Samba-Server is going down in 3 Minutes ...':

Ici, vous devez maintenant indiquer un message qui sera envoyé. Ce message est
émis uniquement aux clients qui ont un service de messagerie compatibles. Le
service de messagerie est normalement activé sur Windows-NT, Windows-2000 et
Windows x, dans le cas contraire il doit être installé/activé.
Pour les clients Windows-9x comme Windows 98 ou Windows ME, vous devez exécuter
le programme de WinPopUp.

\end{enumerate}

    Paramètre par défaut~: \var{OPT\_\-SAMBATOOLS}='no'


\marklabel{sec:OPTNMBD}{\subsection {OPT\_NMBD - NETBIOS serveur de nom}}
\configlabel{OPT\_NMBD}{OPTNMBD}

    Ce programme sert à visualiser une ressourse partager dans un environnement réseau
    sous Windows, il est nécessaire de paramétrer (la variable \smalljump{sec:OPTSAMBA}{\var{OPT\_SAMBA}}='yes').
    Pour permettre \smalljump{sec:OPTSAMBA}{\var{OPT\_SAMBA}} de partager un disque
    virtuel ou une imprimante sur fli4l et pour qu'il soit visible dans le voisinage
    réseau, vous devez activer la variable \var{OPT\_\-NMBD}='yes'.

    Le serveur de noms SMB nécessite 100 Ko supplémentaires sur le média fli4l
    pour l'installation. Si vous n'avez pas assez de place sur le média, vous
    devriez essayer de vous en passer, et intégrer l'imprimante en saisissant
	directement le chemin d'accès du réseau, par exemple \verb+\\fli4l\pr1+.

    Une description plus détaillée de l'interaction des deux programmes optionnels
    se trouve ici \smalljump{sec:OPTSAMBA}{\var{OPT\_SAMBA}}.

    Paramètre par défaut~: \var{OPT\_\-NMBD}='no'


\begin{description}
\config{NMBD\_MASTERBROWSER}{NMBD\_MASTERBROWSER}{NMBDMASTERBROWSER}

    Samba en tant que master browser~: 'yes' ou 'no'

    Dans de nombreux cas l'ordinateur fli4l fonctionne en permanence, il
    serait aventageux d'utiliser cette ordinateur en tant que master browser.
    Un ordinateur master browser dans un réseau Windows, (tous les ordinateurs
    Windows auront activés le partage de fichier et d'imprimante) maintient une
    liste de tous les serveurs SMB disponible. Ainsi, les clients Windows peuvent
    connaître à partir du master browser, les ordinateurs qui ont le partage de
    fichier et d'imprimante activé sur le réseau. cela évite que chaque système
    génère des interrogations broadcast. Dans un réseau avec un serveur NT,
    il est préférable de laisser cette tâche au serveur NT. Dans un réseau avec
    quelques ordinateurs Windows 9x, fli4l peut bien sûr effectuer cette tâche.

    Si vous indiquez \var{NMBD\_\-MASTERBROWSER}='yes' fli4l sera utilisé
    en tant que master browser en opposition de toutes les autres machines Windows.

    Paramètre par défaut~: \var{NMBD\_\-MASTERBROWSER}='no'


\config{NMBD\_DOMAIN\_MASTERBROWSER}{NMBD\_DOMAIN\_MASTERBROWSER}{NMBDDOMAINMASTERBROWSER} (vous avez besoin NMBD\_MASTERBROWSER='yes')

    Samba en tant que domaine master browser~: 'yes' ou 'no'

    J'ai longtemps résisté à inclure cette variable dans la configuration, car
    elle peut être dangereux lors d'une mauvaise utilisation. Si vous activez
    cette option à savoir un contrôleur de domaine dans un réseau, en même temps
    se n'est qu'un domaine master browser et c'est un moyen fiable pour saboter
    le contrôleur de domaine. Si vous l'utilisez, des effets étranges peuvent se
    produire. D'autre part, un domaine master browser est le moyen le plus sûr
    de réaliser un parcours du voisinage réseau (voir cipe-HOWTO).

    Lorsque pour installez un domaine master browser, on ne peut pas l'expliqué
    en un ou deux mots. Heureusement, d'autres personnes ont pris la peine de
    d'expliquer plus clairement la fonction~:

        \altlink{http://samba.sernet.de/dokumentation/browsing-2.html}

    \wichtig{Un serveur WINS est nécessaire il doit être configuré et vu de tous
    les ordinateurs du réseau, pour que tout fonctionne~!
    }

    Si vous choisissez d'activer la variable \var{NMBD\_\-DOMAIN\_\-MASTERBROWSER}='yes'
    pour fli4l, vous pouvez utiliser le domaine master browser, pour que cela
    fonctionne il faut qu'aucun autre domaine master browser existe dans
    l'environnement réseau. Si un autre domaine master browser existe dans
    l'environnement réseau, des dérangements dans le réseau se produiront et
    une soi-disant guerres des browser apparaîtra, l'un des deux ordinateurs
    tentera de prendre le dessus. Donc, si vous ne savez pas exactement si un
    autre domaine master browser fonctionnant déjà dans le réseau, vous devez
    garder la valeur du paramètre par défaut~!

    Paramètre par défaut~: \var{NMBD\_\-DOMAIN\_\-MASTERBROWSER}='no'


\config{NMBD\_WINSSERVER}{NMBD\_WINSSERVER}{NMBDWINSSERVER}

    Samba en tant que serveur WINS~: 'yes' ou 'no'

    Vous avez deux possibilités pour résoudre les noms NETBIOS dans un réseau
    Windows. La première utiliser une résolution statique dans le fichier lmhosts,
    de plus, il est difficile de maintenir à jour le fichier hosts pour la
    résolution de nom pour le DNS. Par conséquent, Microsoft à développé le WINS~:
    \textbf{W}indows \textbf{I}nternet \textbf{N}ame \textbf{S}ervice

    WINS a l'avantage pour la résolution de nom NETBIOS de s'adressé directement
    au serveur WINS et non par le Broadcasts. La base de données WINS est construit
    sur un serveur dynamique, il présente un inconvénient, les propriétés du
    serveur doit être enregistré sur chaque client avec le protocole TCP/IP.
    Samba a mis partiellement en place ce type de serveur sur fli4l, ainsi il
    peut également être disponible sur le réseau.

    Pour que fli4l fonctionne en tant que serveur fli4l, vous devez paramétrer
    les variables \smalljump{sec:OPTSAMBA}{\var{OPT\_SAMBA}}, \smalljump{sec:OPTNMBD}{\var{OPT\_NMBD}}
    et \var{NMBD\_\-WINSSERVER} sur yes et dans les propriétés du protocole
    TCP/IP de la carte réseau à l'onglet WINS vous devez sélectionner "Activer
    la résolution WINS".

    Toujours dans l'onglet WINS, pour la recherche de serveur WINS, vous devez
    placer l'adresse IP de l'ordinateur fli4l en cliquant sur la fonction "Ajouter"

    Bien que l'on a ici le choix entre WINS OU DHCP, si la spécification de
    l'adresse IP n'est pas correct pour le serveur WINS sur fli4l pour la
    configuration TCP/IP, vous pouvez soit spécifier l'adresse IP de chaque
    client soit utiliser le DHCP.

    Dans les réseaux avec un serveur NT, le service du serveur WINS est activé,
    il est préférable de laisser cette tâche à NT. Dans les réseaux avec
    quelques ordinateurs Windows 9x, fli4l peut également faire ce travail, si
    cette fonction est activé.

\begin{example}
\begin{verbatim}
    NMBD_WINSSERVER='yes'
\end{verbatim}
\end{example}

    Si vous avez installé et activé \var{OPT\_\-DHCP}, vous activez

\begin{example}
\begin{verbatim}
    NMBD_WINSSERVER='yes'
\end{verbatim}
\end{example}

    l'adresse IP de l'ordinateur fli4l sera transmis en tant que adresse IP
    du serveur WINS à tous les clients.

    Paramètre par défaut~: \var{NMBD\_\-WINSSERVER}='no'


\config{NMBD\_EXTWINSIP}{NMBD\_EXTWINSIP}{NMBDEXTWINSIP} (vous avez besoin NMBD\_WINSSERVER='no')

    Adresse IP du serveur WINS distant pour Samba

    comme mentionné ci-dessus, si dans le réseau vous avez un serveur NT qui
    fonctionne, vous devez laisser gérer la base de données WINS par ce serveur.
    Ici vous pouvez configurer fli4l en tant que client WINS. L'ordinateur fli4l
    tentera ensuite de s'enregistré auprès du serveur WINS distant qui sera
    configuré. Il faudra s'assurer que fli4l n'est pas configuré en tant
    que serveur et client en même temps - Les variables

\begin{example}
\begin{verbatim}
     NMBD_WINSSERVER='yes'
\end{verbatim}
\end{example}

    et

\begin{example}
\begin{verbatim}
    NMBD_EXTWINSIP='Adresse-IP'
\end{verbatim}
\end{example}

    doivent être éliminées de la configuration. La construction du média fli4l
    ne fonctionnera pas pour des raisons de sécurité dans une telle configuration.
    Dans ce mode, Samba agira également en tant que proxy WINS. Ceci est utile
    lorsque les clients WINS ne sont pas seulement sur le réseau, le serveur WINS
    sera sur un réseau différent qui n'est pas accessible par le Broadcast et
    les clients non WINS demande également une résolution de nom NETBIOS. Ici,
    l'ordinateur fli4l Broadcasts les clients non WINS, le serveur WINS enregistré
    ici est interrogé et enverra la réponse par Broadcast à l'ordinateur qui a
    fait la demande.

    Si vous souhaitez exécuter l'ordinateur fli4l en tant que client WINS, vous
    devez lui indiquer l'adresse IP du serveur WINS distant sur lequel il doit
    s'enregistrer. Vous devez paramétrer la variable sur \smalljump{NMBDWINSSERVER}{\var{NMBD\_WINSSERVER}}='no'.

    Voici un exemple avec l'adresse IP 192.168.6.11~:

\begin{example}
\begin{verbatim}
    NMBD_EXTWINSIP='192.168.6.11'
\end{verbatim}
\end{example}

    Une fois \var{OPT\_\-DHCP} installé et activé, l'adresse IP configuré ici
    sera en tant qu'adresse IP du serveur WINS pour tous les clients.

    Paramètre par défaut~: \var{NMBD\_\-EXTWINSIP}=''

\end{description}


\marklabel{sec:OPTLPD}{\subsection {OPT\_LPD - Serveur d'impression avec le protocole LPD/LPR}}
\configlabel{OPT\_LPD}{OPTLPD}

    Si vous activez la variable \var{OPT\_\-LPD}='yes' vous pouvez utiliser fli4l
    en tant que serveur d'impression. Le service LPD installera une ou plusieurs
    files d'attente pour l'impression (en fonction du nombre d'imprimante
    connectés à fli4l) dans un disque-RAM (ou disque virtuel) sur le
    root-Filesystems (ou racine du fichier système) ou sur un disque dur.

    \wichtig{Pour que l'impression fonctionne sans problème dans un environnement
    multi-utilisateur, le service LPD utilisera un spooler (ou Mis en attente
    dans une mémoire tampon). Les données à imprimer sont stockés dans un
    répertoire spool. un répertoire spool est créé pour chaque imprimante.
    Ces répertoires spool sont situés dans un disque virtuel à la racine du fichiers
    système de la mémoire principale ou sur un disque dur s'il est installé
    et disponible. Lors du démarrage de fli4l, s'il n'y a pas de descripteur
    minix avec une partition ext2 ou ext3 dans lequel le répertoire /data sera monté,
    chaque imprimante configuré utilisera alors le disque-RAM sur le root-Filesystems.
    Si vous voulez faire une impression sur 3 imprimantes simultanément installées
    sur fli4l, les fichiers seront placer dans le spooler du disque-RAM pour les
    3 imprimantes. Il convient de noter, si on utilise Samba pour imprimer même
    avec de petit fichiers texte sous Windows, il y a un gros travail d'impression
    car le programme conditionne en peu de temps 2 fois le disque virtuel.
    Pour qu'une impression fonctionne bien avec le spooler sur le disque-RAM,
    vous devez avoir suffisamment de mémoire RAM dans l'ordinateur fli4l - plus
    serait encore mieux. Pour ceux qui doivent imprimer souvent de gros documents,
    vous devez avoir pas moins de 4 Mio de mémoire RAM seulement pour l'impression,
    vous devez absolument faire une installation sur un disque dur, afin de ne pas
    perturber le fonctionnement du routeur par un débordement du disque-RAM. Oui,
    vous avez bien lu. Pour un travail d'impression le disque-RAM ne correspond
    pas, car le routeur ne pourra plus routé ...}

    \emph{Lorsque vous avez installé un spooler sur le disque, le traitement des travaux
    d'impression est limitée par l'espace disponible sur le disque dur. Pour cela
    le type B doit avoir été sélectionné pour une installation de fli4l sur le
    disque dur, sur celui-ci une partition de données ext3 doit être créée avec
    le répertoire /data il doit être monté.}

    \emph{Si vous avez des problèmes lors de l'impression de gros fichiers et si vous
    n'avez pas utilisé de disque dur pour l'installation et si vous n'avez pas
    assez de mémoire RAM dans le routeur.}

    \emph{Règle de base~: fli4l en tant que routeur par défaut nécessite environ 10 Mo
    d'espace disque (8-12, selon la configuration). Pour 32 Mo de mémoire RAM installé
    dans le routeur, le reste sera disponible pour l'impression - donc 32-10 = 22 Mo.
    Lors de l'impression via Samba, l'espace disponible est encore réduit de moitié,
    car au moment du travail d'impression le disque-RAM est utiliser 2 fois~:
    Pour un travail d'impression d'un maximum de 11 Mo est utiliser 2 fois sur
    le disque-RAM, le routeur arrêtera routage ...}

    \emph{Je vous recommande pour l'impression une installation sur un disque dur.}

    Paramètre par défaut~: \var{OPT\_\-LPD}='no'

\begin{description}


\config{LPD\_DEBUG}{LPD\_DEBUG}{LPDDEBUG}

    Avec cette variable vous pouvez activer ou désactiver la fonction de
	journalisation pour le serveur d'impression LPD. Il est recommandé d'activer
	l'enregistrement de journalisation, que si vous constatez un dysfonctionnement
	du service d'impression, car cette fonction utilise beaucoup d'espace sur
	le support.

	Vous pouvez indiquez dans cette variable les paramètres suivants 'yes', 'no' ou
	un nombre compris entre 1 et 5, la valeur la plus élevée indique une journalisation
	plus détaillée. Le paramètre 'yes' est équivalent à la valeur 1.

    Paramètre par défaut~: \verb+LPD_DEBUG='no'+

    Exemple~: \verb+LPD_DEBUG='2'+


\config{LPD\_DEBUG\_FILE}{LPD\_DEBUG\_FILE}{LPDDEBUGFILE}

    Dans cette variable vous indiquez le chemin du fichier journal, dans lequel sera
	sauvegardé les actions du serveur d'impression LPD et les fonctions protocolaires actif
	(voir la description de la variable \jump{LPDDEBUG}{\var{LPD\_DEBUG}}). En outre, il
	est possible d'indiquez le paramètre 'auto', ainsi le système choisira l'emplacement
	par défaut \texttt{/var/log/lpd.log} du fichier journal.

    Si vous n'avez pas activé la variable \verb+LPD_DEBUG='no'+, le contenu de la variable
	ci-dessus n'a pas besoin d'être configurée.

    Paramètre par défaut~: \verb+LPD_DEBUG_FILE='auto'+

    Exemple~: \verb+LPD_DEBUG_FILE='/data/log/lpd.log'+


\config{LPD\_SPOOLPATH}{LPD\_SPOOLPATH}{LPDSPOOLPATH}

    Dans cette variable vous indiquez le répertoire spool (ou répertoire tampon) pour
	les travaux d'impression entrants. Toutes les données d'impression seront d'abord
	mis en attente dans le répertoire défini ici, avant d'être transmis au serveur
	d'impression LPD. Vous pouvez indiquer 'auto' dans cette variable, dans ce cas fli4l
	va créer et monter un répertoire spool sur un support de stockage persistant voir
	\texttt{/var/lib/persistent/lpd/spool}.

	Notez s'il vous plaît, que le contenu de ce répertoire sera effacé lors du démarrage
	de votre routeur fli4l. Donc, vous ne devez pas spécifier dans ce répertoire d'autres
	données importantes~! Notez également que le chemin d'accés configuré ici, ne doit pas
	être le même que dans la variable \var{SAMBA\_SPOOLPATH} (excepté si les deux variables
	ont la valeur 'auto') parce que les deux répertoires spool sont utilisés à des fins
	différentes et donc, ils ont besoin d'avoir une configuration différent.

    Paramètre par défaut~: \verb+LPD_SPOOLPATH='auto'+

    Exemple~: \verb+LPD_SPOOLPATH='/data/lpd/spool'+


\config{LPD\_NETWORK\_N}{LPD\_NETWORK\_N}{LPDNETWORKN}

    Dans cette variable vous indiquez le nombre de variable \var{LPD\_NETWORK\_x}
	à configurer (voir ci-dessous).

    Exemple~: \verb+LPD_NETWORK_N='1'+


\config{LPD\_NETWORK\_x}{LPD\_NETWORK\_x}{LPDNETWORKx}

    Chaque configuration de la liste indiquera un hôte ou une adresse réseau qui sera
	autorisé à imprimer via le protocole LPD \footnote{voir RFC 1179}. Vous pouvez indiquez
	des adresses IPv4 \verb+192.168.1.0/24+, des adresses symbolique \verb+IP_NET_1+ et
	des références à des hôtes \verb+@peacock+.

    Il convient de noter, que ce paramètre n'est \emph{pas} nécessaire si vous voulez
	accéder seulement à une imprimante via le serveur Samba~! Ce paramètre est
	\emph{seulement} pertinent, si vous souhaitez imprimer avec le protocole LPD et
	plus particulièrement intéressant pour les ordinateurs Linux et Mac. Pour les
	ordinateurs sous Windows, il est plus commode d'imprimer via le service Samba,
	en règle générale, pour les services d'impression UNIX, vous devez faire une
	installation séparée.

    Exemple~:
\begin{example}
\begin{verbatim}
    LPD_NETWORK_1='IP_NET_1'
    LPD_NETWORK_2='192.168.1.0/24'
    LPD_NETWORK_3='@client'
\end{verbatim}
\end{example}


\config{OPT\_LPD\_PARPORT}{OPT\_LPD\_PARPORT}{OPTLPDPARPORT}

    Si vous activez la variable \var{OPT\_\-LPD\_\-PARPORT}='yes' vous allez
    utiliser le port parallèle local pour installer l'imprimante. Si vous souhaitez
    utiliser une imprimante USB ou une imprimante à distance, vous pouvez laisser
    la variable à la valeur par défaut~:

    Paramètre par défaut~: \var{OPT\_\-LPD\_\-PARPORT}='no'


\config{LPD\_PARPORT\_N}{LPD\_PARPORT\_N}{LPDPARPORTN} (vous avez besoin OPT\_LPD\_PARPORT='yes')

    Avec la variable \var{LPD\_\-PARPORT\_\-N} vous indiquez le nombre de port
    parallèle à utiliser pour les imprimantes local. Si vous utilisez une
    imprimante sur le premier port parallèle, vous devez configuré cette variable
    dans le fichier samba\_lpd.txt comme ceci

\begin{example}
\begin{verbatim}
    LPD_PARPORT_N='1'
\end{verbatim}
\end{example}

    Pour 2 ports d'imprimantes vous incrémenter la variable \var{LPD\_\-PARPORT\_\-N}

\begin{example}
\begin{verbatim}
    LPD_PARPORT_N='2'
\end{verbatim}
\end{example}

    Vous aurez besoin également les paramètres correspondants
        \emph{\var{LPD\_\-PARPORT\_\-1\_\-IO}},
        \emph{\var{LPD\_\-PARPORT\_\-1\_\-IRQ}}
        \emph{\var{LPD\_\-PARPORT\_\-1\_\-DMA}}
    et
        \emph{\var{LPD\_\-PARPORT\_\-2\_\-IO}},
        \emph{\var{LPD\_\-PARPORT\_\-2\_\-IRQ}}
        \emph{\var{LPD\_\-PARPORT\_\-2\_\-DMA}}

    et, aussi les paramètres pour l'utilisation de Samba

        \emph{\var{LPD\_\-PARPORT\_\-1\_\-SAMBA\_\-NET}},
        \emph{\var{LPD\_\-PARPORT\_\-2\_\-SAMBA\_\-NET}},

    et, vous devez aussi donner un nom aux imprimantes Samba,

        \emph{\var{LPD\_\-PARPORT\_\-1\_\-SAMBA\_\-NAME}},
        \emph{\var{LPD\_\-PARPORT\_\-2\_\-SAMBA\_\-NAME}},

    pour qu'elles soivent disponible.

    Paramètre par défaut~: \var{LPD\_\-PARPORT\_\-N}='1'


\config{LPD\_PARPORT\_x\_IO}{LPD\_PARPORT\_x\_IO}{LPDPARPORTxIO}

    Avec la variable \var{LPD\_\-PARPORT\_\-x\_\-IO} vous réglez le x-ième port
    parallèle pour l'imprimante local. Si vous avez 2 imprimantes, 2 ports
    parallèles doivent être paramétrés sur fli4l, voici les valeurs possibles
    qui peuvent existées~:

\begin{itemize}
\item 0x3bc
\item 0x378 ou
\item 0x278
\end{itemize}

    Vous pouvez par exemple indiquer dans cette variable

\begin{example}
\begin{verbatim}
    LPD_PARPORT_1_IO='0x378'
\end{verbatim}
\end{example}

    et

\begin{example}
\begin{verbatim}
    LPD_PARPORT_2_IO='0x278'
\end{verbatim}
\end{example}

    \wichtig{
    Jusqu'à présent, seules les interfaces parallèles sur la carte mère ou les
    interfaces sur une carte ISA sont prisent en charge avec les valeurs possibles
    décrites ci-dessus. Les cartes PCI avec ports parallèles ne peuvent pas être
    utilisées.}

    \emph{Pour certaines versions de cartes PCI avec le chipset NETMOS, la configuration
    des ports parallèles peuvent également être configurées. Vous devez utiliser
    la fonction pour}

\begin{example}
\begin{verbatim}
        cat /proc/pci
\end{verbatim}
\end{example}
    \emph{pour voir l'utilisation des périphériques PCI détectés sur l'ordinateur. Ici
    on cherche le Vendor-ID convenant à l'appareil, vous selectionnez ensuite le
    Device-ID pour le saisir dans l'adresse-IO, à partir des entrées suivantes~:}

        \begin{itemize}
        \item Nm9705CV  (Vendor id=9710, Device id=9705, Port1 1. Saisir)
        \item Nm9735CV  (Vendor id=9710, Device id=9735, Port1 3. Saisir)
        \item Nm9805CV  (Vendor id=9710, Device id=9805, Port1 1. Saisir)
        \item Nm9715CV  (Vendor id=9710, Device id=9815, Port1 1. Saisir, Port2 3. Saisir)
        \item Nm9835CV  (Vendor id=9710, Device id=9835, Port1 3. Saisir)
        \item Nm9755CV  (Vendor id=9710, Device id=9855, Port1 1. Saisir, Port2 3. Saisir)
        \end{itemize}

    \emph{Cette configuration a été paramétré, sans matériel disponible et donc sans
    les tests adéquat. Par conséquent, vous devez considéré cette fonction comme
    expérimentale. Si vous avez des erreurs s'il vous plaît fournissez des
    informations détaillées pour les poster sur le forum~!}

    Vous devez d'abord vérifier les adresses IO des interfaces intégrées s'il
    sont bien définies, parce qu'il est nécessaire de configurer cette variable,
    car l'impression ne fonctionnera pas. Les adresses IO sont définies, soit
    dans le BIOS de l'ordinateur ou soit elles sont affichées au démarrage, sur
    des vieux ordinateurs pas configurables. En outre, les ports intégrés sur
    l'ensemble des cartes-IO peuvent généralement être définies par un cavalier
    pour configurer l'imprimante, cela est décrit dans la (documentation espérons
    qu'elle est toujours existante).

    De plus il convient de s'assurer que l'ensemble les adresses IO paramètrées
    dans le fichier samba\_lpd.txt n'entrent pas en collision avec les adresses
    que vous avez peut être configurées dans la variable \var{LCD\_\-ADDRESS},
    si bien sûr, vous avez activer le paquetage \var{OPT\_\-LCD}='yes'.
	Ce conflit empêchera la création du média de boot~!

    Paramètre par défaut~: \var{LPD\_\-PARPORT\_\-1\_\-IO}='0x378'


\config{LPD\_PARPORT\_x\_IRQ}{LPD\_PARPORT\_x\_IRQ}{LPDPARPORTxIRQ}

        {Si vous paramétrez cette variable \var{LPD\_\-PARPORT\_\-x\_\-IRQ} vous
        pouvez imprimer et soulager le processeur en indiquant une interruption.
        A cet effet, le mode ECP/EPP doit être configuré pour les interfaces sur
        la carte mère dans le BIOS de l'ordinateur, sur la carte ISA en modifiant
        un cavalier. SI vous activez le mode d'interruption~:

\begin{example}
\begin{verbatim}
        LPD_PARPORT_1_IRQ='yes'
\end{verbatim}
\end{example}

        Si vous ne voulez pas utiliser ce mode

\begin{example}
\begin{verbatim}
        LPD_PARPORT_1_IRQ='no'
\end{verbatim}
\end{example}

        vous devez configurer dans tous les cas le mode Normal ou SPP pour les
        interfaces sur la carte mère dans le BIOS de l'ordinateur ou sur la carte
        ISA en modifiant un cavalier. Si quelque chose ne fonctionne pas, vous
        devez testé sans activer l'interruption~!

\begin{example}
\begin{verbatim}
        LPD_PARPORT_1_IRQ='no'
\end{verbatim}
\end{example}

        Paramètre par défaut~: \var{LPD\_\-PARPORT\_\-1\_\-IRQ}='no'}


\config{LPD\_PARPORT\_x\_DMA}{LPD\_PARPORT\_x\_DMA}{LPDPARPORTxDMA}

        {Si vous paramétrez cette variable \var{LPD\_\-PARPORT\_\-x\_\-DMA} vous
        pouvez imprimer et soulager le processeur en activant la fonction DMA.
        A cet effet, le mode ECP/EPP doit être configuré pour les interfaces sur
        la carte mère dans le BIOS de l'ordinateur ou sur la carte ISA en modifiant
        un cavalier. Le mode interruption dont aussi être activé~:

\begin{example}
\begin{verbatim}
        LPD_PARPORT_1_DMA='yes'
\end{verbatim}
\end{example}

        C'est la condition préalable avec

\begin{example}
\begin{verbatim}
        LPD_PARPORT_1_IRQ='yes'
\end{verbatim}
\end{example}

        Si vous ne voulez pas utiliser ce mode

\begin{example}
\begin{verbatim}
        LPD_PARPORT_1_DMA='no'
\end{verbatim}
\end{example}

        vous devez configurer dans tous les cas le mode Normal ou SPP pour les
        interfaces sur la carte mère dans le BIOS de l'ordinateur ou sur la carte
        ISA en modifiant un cavalier. Si quelque chose ne fonctionne pas, vous
        devez testé sans activer le DMA~!

\begin{example}
\begin{verbatim}
        LPD_PARPORT_1_DMA='no'
\end{verbatim}
\end{example}

        Paramètre par défaut~: \var{LPD\_\-PARPORT\_\-1\_\-DMA}='no'}


\config{OPT\_LPD\_USBPORT}{OPT\_LPD\_USBPORT}{OPTLPDUSBPORT}

    Si vous paramétrez cette variable \var{OPT\_\-LPD\_\-USBPORT}='yes' vous
    déterminez l'imprimante qui sera utilisée sur le port USB local.

    De plus pour utiliser une imprimante USB vous devez avoir paramétré le
    paquetage \var{OPT\_\-USB}. Voici les paramètres à définir~:

\begin{example}
\begin{verbatim}
    OPT_USB='yes'
    USB_LOWLEVEL='uhci'
    USB_PRINTER='yes'
\end{verbatim}
\end{example}

        ou bien~:

\begin{example}
\begin{verbatim}
    OPT_USB='yes'
    USB_LOWLEVEL='usb-ohci'
    USB_PRINTER='yes'
\end{verbatim}
\end{example}

    \wichtig{L'option de configuration pour imprimante USB a été installé, sans
    matériel disponible et donc sans les tests adéquat. Par conséquent, cette
    fonction doit être considéré comme expérimentale. Si vous avez des erreurs
    s'il vous plaît fournissez des informations détaillées pour les poster
    sur le forum~!
    Beaucoup d'imprimantes USB sont les imprimantes GDI. Les imprimantes GDI ne
    peuvent pas être pris en charge. Je ne répondrais pas aux questions concernant
    des problèmes avec des imprimantes USB, si vous avez exclu que l'imprimante
    concernée est une imprimante GDI~!
    }

    Si vous souhaitez utiliser une seule imprimante avec un port parallèle
    ou une imprimante à distance , vous pouvez laisser cette variable sur
    la valeur par défaut~:

    Paramètre par défaut~: \var{OPT\_\-LPD\_\-USBPORT}='no'


\config{LPD\_USBPORT\_N}{LPD\_USBPORT\_N}{LPDUSBPORTN} (vous avez besoin OPT\_LPD\_USBPORT='yes')

    Avec la variable \var{LPD\_\-USBPORT\_\-N} vous indiquez le nombre de port
    USB pour les imprimantes local. Pour une imprimante connectée au premier
    port USB vous indiquez

\begin{example}
\begin{verbatim}
    LPD_USBPORT_N='1'
\end{verbatim}
\end{example}

    Pour 2 imprimantes sur ports USB vous incrémenter la variable \var{LPD\_\-USBPORT\_\-N}

\begin{example}
\begin{verbatim}
    LPD_USBPORT_N='2'
\end{verbatim}
\end{example}

    En outre, vous allez avoir besoin pour utiliser Samba, les paramètres correspondants
    \var{LPD\_\-USBPORT\_\-1\_\-SAMBA\_\-NET} et \var{LPD\_\-USBPORT\_\-2\_\-SAMBA\_\-NET}
    vous devez aussi attribuer nom aux imprimante pour Samba, les paramètres
    \var{LPD\_\-USBPORT\_\-1\_\-SAMBA\_\-NAME} et \var{LPD\_\-USBPORT\_\-2\_\-SAMBA\_\-NAME}
    doivent être également présents.

    \wichtig{Si vous utilisez plus d'une imprimante USB, il faut s'assurer de
    l'ordre de démarrage des imprimantes qui est déterminant, Quelle imprimante
    sera la première et quelle imprimante sera la deuxième sur les ports USB pour
    imprimer. La deuxième imprimante USB sera automatiquement la première imprimante,
    lorsque la première imprimante USB n'est pas démarrée. Si différents modèles
    d'imprimantes sont installées, différents pilotes seront nécessaire pour le
    client, il peut donc arriver que l'imprimante sélectionnée émet uniquement
    une "salade de caractères", c'est parce que le travail d'impression a été
    formatée dans un langage d'une autre imprimante.
    }

    Paramètre par défaut~: \var{LPD\_\-USBPORT\_\-N}='1'


\config{OPT\_LPD\_REMOTE}{OPT\_LPD\_REMOTE}{OPTLPDREMOTE}

    Si vous paramétrez cette variable \var{OPT\_\-LPD\_\-REMOTE}='yes' vous pouvez
    déterminer une imprimante distante (sur un autre réseau) qui sera utilisée. Si
    vous souhaitez exécuter le travail d'impression uniquement sur une imprimante
    parallèle ou sur un port USB local, vous pouvez laisser cette variable
    avec la valeur par défaut~:

    Paramètre par défaut~: \var{OPT\_\-LPD\_\-REMOTE}='no'


\config{LPD\_REMOTE\_N}{LPD\_REMOTE\_N}{LPDREMOTEN} (vous avez besoin OPT\_LPD\_REMOTE='yes')

    Avec cette variable \var{LPD\_REMOTE\_\-N} vous indiquez le nombre
    d'imprimante distante à installer. Cela permet d'envoyer un travail d'impression
    à partir d'un client fli4l, qui à son tour transmet la tâche d'impression à
    un serveur d'impression distant compatible LPD. Le tout doit fonctionner en
    conjonction avec Samba. Si vous souhaitez travailler avec une imprimante
    distante à partir de fli4l sur un serveur d'impression distant, vous indiquez

\begin{example}
\begin{verbatim}
    LPD_REMOTE_N='1'
\end{verbatim}
\end{example}

    Si vous avez 2 imprimantes distante ou serveur d'impression distant avec
    2 files d'attente, vous devez incrémenter la variable \var{LPD\_\-REMOTE\_\-N}
    et indiquez

\begin{example}
\begin{verbatim}
    LPD_REMOTE_N='2'
\end{verbatim}
\end{example}

    De plus vous allez également avoir besoin des paramètres correspondants

\begin{itemize}
\item \var{LPD\_\-REMOTE\_\-1\_\-IP}
\item \var{LPD\_\-REMOTE\_\-1\_\-PORT}
\item \var{LPD\_\-REMOTE\_\-1\_\-QUEUENAME}
\item \var{LPD\_\-REMOTE\_\-2\_\-IP}
\item \var{LPD\_\-REMOTE\_\-2\_\-PORT}
\item \var{LPD\_\-REMOTE\_\-2\_\-QUEUENAME}
\end{itemize}

    et si vous utilisez en plus Samba, ces variables

\begin{itemize}
\item \var{LPD\_\-REMOTE\_\-1\_\-SAMBA\_\-NAME}
\item \var{LPD\_\-REMOTE\_\-1\_\-SAMBA\_\-NET}
\item \var{LPD\_\-REMOTE\_\-2\_\-SAMBA\_\-NAME}
\item \var{LPD\_\-REMOTE\_\-2\_\-SAMBA\_\-NET}
\end{itemize}

    doivent également être présentes.

    Paramètre par défaut~: \var{LPD\_\-REMOTE\_\-N}='0'


\config{LPD\_REMOTE\_x\_IP}{LPD\_REMOTE\_x\_IP}{LPDREMOTExIP}

    Avec cette variable \var{LPD\_\-REMOTE\_\-x\_\-IP} vous indiquez la x-ième
    adresse IP pour le serveur d'impression à distant.

    Dans l'installation par défaut on suppose qu'un deuxième ordinateur fli4l
    est accessible par l'adresse IP 192.168.6.99 configuré pour l'impression.

    Paramètre par défaut~: \var{LPD\_\-REMOTE\_\-1\_\-IP}='192.168.6.99'


\config{LPD\_REMOTE\_x\_PORT}{LPD\_REMOTE\_x\_PORT}{LPDREMOTExPORT}

    Avec cette variable \var{LPD\_\-REMOTE\_\-x\_\-PORT} vous indiquez le x-ième
    port pour l'impression.
    Vous utilisez cette variable seulement si vous souhaitez imprimer sur
    un serveur d'impression qui permet d'envoyer des données via le ftp ou le netcat.
    Si vous construisez un serveur d'impression qui utilise le protocole LPD, alors
    cette variable doit être vide et paramétrer à la place la variable
    \smalljump{LPDREMOTExQUEUENAME}{\var{LPD\_REMOTE\_x\_QUEUENAME}}. Donc
    SOIT la variable \var{LPD\_\-REMOTE\_\-x\_\-PORT} ou SOIT la variable
    \smalljump{LPDREMOTExQUEUENAME}{\var{LPD\_REMOTE\_x\_QUEUENAME}} sera
    paramétrée, mais jamais les deux en même temps~! L'une des deux variables
    doivent être paramétrée.

    Que votre serveur d'impression appartient à l'un ou à l'autre catégorie,
    pouvez trouver s'il vous plaît les information dans le manuel ou sur le site
    Web du fabricant. Un aperçu incomplet peut être trouvé ici

        \altlink{http://www.lprng.com/LPRng-Reference/LPRng-Reference.html\#AEN4990}

    Pardonnez-moi mais je n'ai pas le temps de vous chercher ces informations,
    donc s'il vous plaît recherche les vous même.

    Dans l'installation par défaut le nom de la troisième imprimante à distance
    est repr3, l'IP 192.168.6.100 est utilisé pour le serveur d'impression HP
    JetDirect (carte d'interface), qui est accessible via le port 9100 (comme
    vous pouvez le lire avec le lien ci-dessus, cependant elle peut aussi être
    accessible par le nom de la file d'attente raw ...).

    Voici un conseil~:
    Si au moment de l'impression l'imprimante correspondant n'est peut pas
    accessible ou si l'impression est bloquée dans la file d'attente sur le
    serveur d'impression LPD et que le travail ne peut pas être traitée. Vous
    pouvez supprimer ce travail avec la commande lprm, autrement il restera en
    place dans la file d'attente jusqu'au redémarrage du routeur~!

    Paramètre par défaut~: \var{LPD\_\-REMOTE\_\-3\_\-PORT}='9100'


\config{LPD\_REMOTE\_x\_QUEUENAME}{LPD\_REMOTE\_x\_QUEUENAME}{LPDREMOTExQUEUENAME}

    Avec cette variable \var{LPD\_\-REMOTE\_\-x\_\-QUEUENAME} vous indiquez le
    x-ième nom de la file d'attente de l'imprimante distante

    Vous utilisez cette variable seulement si vous souhaitez imprimer sur un
    serveur d'impression qui comprent le protocole LPD.
    Si vous construisez un serveur d'impression qui permet d'envoyer des données
    via le ftp ou le netcat, alors cette variable doit être vide et paramétrer à la
    place la variable \smalljump{LPDREMOTExPORT}{\var{LPD\_REMOTE\_x\_PORT}}. donc
    SOIT la variable \var{LPD\_\-REMOTE\_\-x\_\-QUEUENAME} ou SOIT la variable
    \smalljump{LPDREMOTExPORT}{\var{LPD\_REMOTE\_x\_PORT}} sera paramétrée,
    mais jamais les deux en même temps~! L'une des deux variables doivent être
    paramétrée.

    Que votre serveur d'impression appartient à l'un ou à l'autre catégorie,
    pouvez trouver s'il vous plaît les information dans le manuel ou sur le site
    Web du fabricant. Un aperçu incomplet peut être trouvé ici

        \altlink{http://www.lprng.com/LPRng-Reference/LPRng-Reference.html\#AEN4990}

    Pardonnez-moi mais je n'ai pas le temps de vous chercher ces informations,
    donc s'il vous plaît recherche les vous même.

    Dans l'installation par défaut, on suppose que le nom de la première file
    d'attente de l'imprimante est pr1 sur le deuxième ordinateur fli4l.

    Paramètre par défaut~: \var{LPD\_\-REMOTE\_\-1\_\-QUEUENAME}='pr1'


\config{OPT\_LPD\_SMBREMOTE}{OPT\_LPD\_SMBREMOTE}{OPTLPDSMBREMOTE}

    Si vous activez cette variable \var{OPT\_\-LPD\_\-SMBREMOTE}='yes' vous pouvez
    déterminer une imprimante SMB distante (via un partage SMB) qui sera utilisée

    \wichtig{
    La configuration pour cette imprimante n'a de sens que si l'imprimante SMB
    distante est allumées pour le travail d'impression - un spooler est utilisé
    pour stoker le travail d'impression, jusqu'au redémarrage de ordinateur
    distant, l'ordinateur distant qui a le partage d'imprimante n'est possible qu'a
    une condition, la réalisation d'un script de préfiltre pour le lpd.
    }

    Si vous souhaitez installer uniquement une imprimante sur le port parallèle
    locale, ou sur ports USB local, ou une imprimante LPD distante, vous pouvez
    laisser la valeur de cette variable par défaut~:

    Paramètre par défaut~: \var{OPT\_\-LPD\_\-SMBREMOTE}='no'


\config{LPD\_SMBREMOTE\_DEBUGLEVEL}{LPD\_SMBREMOTE\_DEBUGLEVEL}{LPDSMBREMOTEDEBUGLEVEL} (vous avez besoin OPT\_LPD\_SMBREMOTE='yes')

    Avec cette variable \var{LPD\_SMBREMOTE\_\-DEBUGLEVEL} vous indiquez le nombre
    de messages pour le débogage de l'impression, lorsque vous êtes connecté à
    une imprimante SMB distante. Il faut que le fichier d'impression soit enregistré,
    le fichier log (ou journal) dans /tmp/smb-print.log sera écrasé à chaque nouvel
    impression. Avec \var{LPD\_SMBREMOTE\_DEBUGLEVEL}='0' il n'y a pas
    d'enregistrement. Si des problèmes surviennent, vous devez régler une valeur
    plus élevée de manière à limiter l'erreur dans le fichier /tmp/smb-print.log

    Paramètre par défaut~: \var{LPD\_\-SMBREMOTE\_\-DEBUGLEVEL}='0'


\config{LPD\_SMBREMOTE\_N}{LPD\_SMBREMOTE\_N}{LPDSMBREMOTEN} (vous avez besoin OPT\_LPD\_SMBREMOTE='yes')

    Avec cette variable \var{LPD\_SMBREMOTE\_\-N} vous indiquez le nombre
    d'imprimante SMB distante à installer. Cela permet d'envoyer un travail
    d'impression à partir d'un client fli4l, qui à son tour transmet cette
    information à une imprimante SMB distant partagée.

    Le tout fonctionne aussi en conjonction avec Samba. Si vous souhaitez travailler
    avec une imprimante SMB distante pour un ordinateur distant avec Windows
    ou avec samba sur fli4l, vous devez indiquer

\begin{example}
\begin{verbatim}
    LPD_SMBREMOTE_N='1'
\end{verbatim}
\end{example}

    Si vous souhaitez travailler avec 2 imprimantes SMB distante pour un ordinateur
    distant avec Windows ou avec samba, vous devez incrémenter la variable
    \var{LPD\_\-SMBREMOTE\_\-N}, c'est à dire

\begin{example}
\begin{verbatim}
    LPD_SMBREMOTE_N='2'
\end{verbatim}
\end{example}

    De plus vous allez également avoir besoin les paramètres correspondants

\begin{itemize}
\item \var{LPD\_\-SMBREMOTE\_\-1\_\-SERVER}
\item \var{LPD\_\-SMBREMOTE\_\-1\_\-SERVICE}
\item \var{LPD\_\-SMBREMOTE\_\-1\_\-USER}
\item \var{LPD\_\-SMBREMOTE\_\-1\_\-PASSWORD}
\item \var{LPD\_\-SMBREMOTE\_\-1\_\-IP}
\item \var{LPD\_\-SMBREMOTE\_\-2\_\-SERVER}
\item \var{LPD\_\-SMBREMOTE\_\-2\_\-SERVICE}
\item \var{LPD\_\-SMBREMOTE\_\-2\_\-USER}
\item \var{LPD\_\-SMBREMOTE\_\-2\_\-PASSWORD}
\item \var{LPD\_\-SMBREMOTE\_\-2\_\-IP}
\end{itemize}

    et si vous utilisez en plus Samba, ces variables

\begin{itemize}
\item \var{LPD\_\-SMBREMOTE\_\-1\_\-SAMBA\_\-NAME}
\item \var{LPD\_\-SMBREMOTE\_\-1\_\-SAMBA\_\-NET}
\item \var{LPD\_\-SMBREMOTE\_\-2\_\-SAMBA\_\-NAME}
\item \var{LPD\_\-SMBREMOTE\_\-2\_\-SAMBA\_\-NET}
\end{itemize}

    doivent également être présentes.

    Paramètre par défaut~: \var{LPD\_\-SMBREMOTE\_\-N}='0'


\config{LPD\_SMBREMOTE\_x\_SERVER}{LPD\_SMBREMOTE\_x\_SERVER}{LPDSMBREMOTExSERVER}

    Dans cette variable \var{LPD\_\-SMBREMOTE\_\-x\_\-SERVER} vous indiquez le
    x-ième nom NetBIOS de l'ordinateur qui partage l'imprimante. Ce nom est
    nécessaire, si un client smb veut imprimé.

    Le nom NetBIOS "canard" du paramètre par défaut est un ordinateur NT.

    Paramètre par défaut~: \var{LPD\_\-SMBREMOTE\_\-1\_\-SERVER}='canard'


\config{LPD\_SMBREMOTE\_x\_SERVICE}{LPD\_SMBREMOTE\_x\_SERVICE}{LPDSMBREMOTExSERVICE}

    Dans cette variable \var{LPD\_\-SMBREMOTE\_\-x\_\-SERVICE} vous indiquez le
    x-ième nom de l'imprimante partagée, pour utiliser l'imprimante SMB distante.
    Par défaut on suppose que le nom de l'imprimante partagée est "pr2"

    Paramètre par défaut~: \var{LPD\_\-SMBREMOTE\_\-1\_\-SERVICE}='pr2'


\config{LPD\_SMBREMOTE\_x\_USER}{LPD\_SMBREMOTE\_x\_USER}{LPDSMBREMOTExUSER}

    Dans cette variable \var{LPD\_\-SMBREMOTE\_\-x\_\-USER} vous indiquez le
    x-ième nom d'utilisateur pour accèder à l'imprimante partagée.
    Par défaut on suppose que le nom d'utilisateur est "king".

    Paramètre par défaut~: \var{LPD\_\-SMBREMOTE\_\-1\_\-USER}='king'


\config{LPD\_SMBREMOTE\_x\_PASSWORD}{LPD\_SMBREMOTE\_x\_PASSWORD}{LPDSMBREMOTExPASSWORD}

    Dans cette variable \var{LPD\_\-SMBREMOTE\_\-x\_\-PASSWORD} vous indiquez le
    x-ième mot de passe utilisateur pour accèder à l'imprimante partagée.
    Par défaut on suppose que le mot de passe utilisateur est "kong"

    Paramètre par défaut~: \var{LPD\_\-SMBREMOTE\_\-1\_\-PASSWORD}='kong'


\config{LPD\_SMBREMOTE\_x\_IP}{LPD\_SMBREMOTE\_x\_IP}{LPDSMBREMOTExIP}

    Dans cette variable \var{LPD\_\-SMBREMOTE\_\-x\_\-IP} vous indiquez le
    x-ième adresse IP de l'ordinateur Windows ou Samba pour le partage l'imprimante.
    Par défaut on suppose que l'ordinateur NT sera accessible par l'IP "192.168.0.6"

    Paramètre par défaut~: \var{LPD\_\-SMBREMOTE\_\-1\_\-IP}='192.168.0.6'

\end{description}


\marklabel{sec:OPTSAMBAPOINTANDPRINT}{
\subsection{OPT\_SAMBA\_POINT\_AND\_PRINT - Gestion des pilotes d'imprimante pour Windows sur le serveur}}

Le Point'n'Print est une technologie Windows pour la gestion des pilotes d'imprimante
côté serveur. L'idée est simple~: si un serveur Windows est en même temps un serveur
d'impression, il est alors raisonnable qu'il offre les pilotes d'imprimante approprié
pour le ou les imprimantes connectées. Car il est inutile et coûteux d'installer sur
chaque client Windows les pilotes d'imprimante approprié. Point'n'Print fait exactement
là même chose~: d'abord un administrateur charge autant de pilotes que d'imprimantes installées
sur l'architectures du serveur d'impression. Un utilisateur normal peut se connecter
en cas de besoin à une imprimante sur ce serveur (il faut donc utiliser analogiquement le
fichier partagé par rapport à l'imprimante partagée) ensuite le client Windows récupère
automatiquement le pilote d'imprimante approprié dans le serveur d'impression et l'installe
localement sur le client. Ainsi, l'utilisation et l'installation de l'imprimante réseau
partagée est rapide et sans trop d'effort.

Dans \jump{sec:OPTSAMBAPOINTANDPRINT:XP}{l'annexe} vous avez une documentation pour
mettre en place la configuration du Point'n'Print à l'aide d'un client Windows XP.

\begin{description}


\config{OPT\_SAMBA\_POINT\_AND\_PRINT}{OPT\_SAMBA\_POINT\_AND\_PRINT}{OPTSAMBAPOINTANDPRINT}

Cette variable permet d'activer la fonction Point'n'Print. L'activation de cette
variable nécessite l'activation de la variable \verb+OPT_SAMBA='yes'+ et
\verb+OPT_LPD='yes'+.

Paramètre par défaut~: \verb+OPT_SAMBA_POINT_AND_PRINT='no'+

Exemple~: \verb+OPT_SAMBA_POINT_AND_PRINT='yes'+


\config{SAMBA\_PRINT\_ADMIN\_NAME}{SAMBA\_PRINT\_ADMIN\_NAME}{SAMBAPRINTADMINNAME}

Pour que les utilisateurs n'installe ou désinstalle pas à volonté les pilotes
d'imprimante (c'est en règle générale indésirable), cette fonction doit être utilisée
que par \emph{l'administrateur d'imprimante}. Le nom de l'administrateur qui utilisera
cette fonction pour les comptes Windows, sera paramétré ici.

Exemple~: \verb+SAMBA_PRINT_ADMIN_NAME='pradmin'+


\config{SAMBA\_PRINT\_ADMIN\_PASSWORD}{SAMBA\_PRINT\_ADMIN\_PASSWORD}{SAMBAPRINTADMINPASSWORD}

Dans cette variable vous indiquez le mot de passe de l'administrateur d'imprimante, qui
prendra en charge les comptes Windows.

Exemple~: \verb+SAMBA_PRINT_ADMIN_PASSWORD='secret'+

\end{description}


\marklabel{sec:DRUCKEREINRICHTUNG}{\subsection {Configuration de l'imprimante sur le client}}

    L'installation de l'imprimante fli4l sur les clients dépendent de façon
    significative de l'activation ou pas du paquetage \smalljump{sec:OPTSAMBA}{\var{OPT\_SAMBA}}
    et si \smalljump{sec:OPTSAMBA}{\var{OPT\_SAMBA}} a été activé, de la même façon
    si \smalljump{sec:OPTNMBD}{\var{OPT\_NMBD}} a été activé ou pas.
    En outre, on doit aussi faire attention aux différents systèmes d'exploitation
    des clients et de leurs possibilités. Par conséquent, nous avons écrit un
    article pour chaque option de configuration.


\subsubsection{Configuration avec OPT\_SAMBA déactivé}
\begin{enumerate}
\item \textbf{Avec un équipement NT}

    Si Samba n'est pas utilisé, vous devez installer le service d'impression LPD
    pour Unix, il est nécessaires pour Windows NT 4.0/2000/XP pour l'accès au
    serveur d'impression fli4l, l'utilisation du standard TCP/IP ports pour
    Windows est inapproprié.

    Service d'impression pour Unix allez dans

    Démarrer/Paramètres:Panneau de configuration/Ajouter ou Suppression de
    programmes/Ajouter ou supprimer des composants Windows/Autres services de
    fichiers et d'impression en réseau/Détails/Services d'impression pour UNIX

    et ajoutez

    Il s'agit d'un nouveau port d'imprimante qui est appelé "Port LPR". Maintenant,
    dans l'assistant d'imprimante nous allons configurer une nouvelle imprimante
    sous Windows NT 4.0/2000/XP avec son pilote, celle-ci sera rattachée à fli4l.

    Démarrer/Paramètres/Imprimantes

    faite un double clic sur "Ajouter une imprimante". nous allons confirmé
    la mise en place en cliquant sur "Suivant" sélectionnez "Imprimante locale"
    désactivée "Détection et installation automatique de l'imprimante Plug and Play"
    et confirmé "Suivant". Dans "Choisir un port d'imprimante" en-dessous vous
    activez "Créer un nouveau port" et dans "Type de port" sélectionnez le
    "Port LPR" créé ci-dessus. Une fois que vous avez confirmé ces paramètres en
    cliquant sur "Suivant", vous devez configurer dans le champ "Nom ou adresse
    du serveur fournissant le LPD" l'adresse IP correcte de l'ordinateur fli4l
    et indiquez dans le champ "Nom de l'imprimante ou de la file d'attente sur
    ce serveur" le nom de la file d'attente de l'imprimante. le nom est "prx" pour
    l'imprimante locale sur le port parallèle, "usbprx" pour l'imprimante locale
    sur un port USB, "reprx" pour l'imprimante distante et "smbprx" pour l'imprimante
    SMB distante. A la place du "x" vous indiquez 1, 2, 3, pour le premier, deuxième,
    troisième port etc.

    Dans la fenêtre Installer le logiciel d'impression vous sélectionnez
    sur le côté gauche le fabricant, l'imprimante attaché à fli4l et sur le côté
    droit le type correspondant et confirmé à nouveau avec "Suivant". Vous pouvez
    maintenant spécifier un nom pour l'imprimante Dans le champ "Nom de l'imprimante".
    Dans "Partage d'Imprimante" on choisit de ne pas partager cette imprimante,
    car l'imprimante est partagée sur l'ordinateur fli4l. Après avoir cliqué sur
    "Suivant" on répond non à la question à savoir si une page de test doit être
    imprimer, parce que tous les paramètres serons définis et confirmés lorsque
    l'on aura à nouveau cliqué sur "Suivant". Une fenêtre apparaît avec le
    résumé de la configuration. Si tout a été correctement saisi, appuyez sur "Terminer".

    Après avoir copié les pilotes d'imprimante, une nouvelle icône pour
    l'imprimante apparaît dans le dossier Imprimantes. Avec le bouton droit de
    la souris vous cliquez sur l'icône de l'imprimante fli4l et choisissez dans
    le menu contextuel "Propriétés". Dans l'onglet "Ports"  déactivez "Activer
    la gestion du mode bidirectionnel". Dans l'onglet "Avancé" cliquez sur "Processeur
    d'impression" et entrez dans "Processeur d'impression" et activez "WinPrint"
    puis sous "Type de donnée par défaut" acrivez "RAW" et quitte la boîte de
    dialogue avec "OK". (Pour Windows NT 4.0 vous avez encore une case à coche
    "Toujours spooler les données au format RAW"). toujours dans l'onglet "Avancé"
    activez "Spouler l'impression des documents pour qu'elle se termine plus rapidement"
    et "Commencer l'impression après le trensfert de la dernière page dans le spouler".
    puis déactivez "Activer les fonctionnalités d'implession avancées". Maintenant,
    vous allez accepter tous les réglages effectués jusqu'à présent avec le bouton
    "Appliquer" et laisse la fenêtre de configuration affichée, car si vous cliquez
    sur "OK" Windows NT 4.0/2000/xp ne sauvegarde pas correctement les réglages effectués.

\item \textbf{Installation 9x}

    Si vous ne voulez pas imprimer via Samba, avec un seul serveur d'impression
    pour Unix, vous devez paramétrer la variable \var{OPT\_\-LPD} pour les clients
    Linux et Windows NT, car seul ces systèmes d'exploitation apporte un logiciel
    client approprié.

    Cependant, il est aussi possible d'imprimer avec Windows9x/Me en utilisant
    le client LPR avec la version freeware sans devoir à installer SAMBA usant
    beaucoup de place sur le support de média.

    Pour le téléchargement du client LPR pour Windows (je ne garanti pas que
    les pages sont encore accessibles)~:\\
%   \altlink{http://utep.el.utwente.nl/diensten/ftd/pdf/instlpr.exe}
    \altlink{ftp://ftp.informatik.uni-hamburg.de/pub/os/unix/utils/LPRng/WINDOWS/acitsplr/instlpr.exe}

    C'est la dernière version 3.4f (gratuite) du client lpr pour une utilisateur
    privé. La version actuelle est payante et trouve ici~:

    \altlink{http://www.utexas.edu/academic/otl/software/lpr/}

    L'installation et la configuration de ce logiciel pour Windows 9x/Me est
    décrite sur le site et pour Windows NT 4.0/2000/XP la documentation de
    l'OPT\_LPDSRV ne sera pas abordées ici.
    Dans le paragaphe suivante, nous nous limiterons aux systèmes d'exploitation
    qui intègrent déjà la fonctionnalité du client LPR.

\end{enumerate}


\subsubsection{Configuration avec OPT\_SAMBA activé}

    La mise en place d'un client Windows pour l'impression via Samba fonctionne
    différemment, selon la configutation de la variable \smalljump{sec:OPTNMBD}{\var{OPT\_NMBD}}='no'
    ou \smalljump{sec:OPTNMBD}{\var{OPT\_NMBD}}='yes'.

\begin{enumerate}
\item \textbf{OPT\_NMBD='no'}

    Avec la configuration \smalljump{sec:OPTNMBD}{\var{OPT\_NMBD}}='no' les
    imprimantes sur fli4l ne seront pas vu sur les PC Windows dans l'environnement
    réseau. Néanmoins, vous pouvez vous connecter à l'aide du chemin d'accès UNC

    En plus il est nécessaire d'enregistrer le routeur fli4l dans le fichier hosts.
    Un exemple de ce fichier se trouve dans le répertoire Windows 95, Windows 98
    et Windows Me il porte le nom host.sam, lors de l'installation par défaut de
    \verb+C:\WINDOWS+, la fin du fichier .sam signifiant sample, qui se traduit
    exemple. Pour Windows NT 4.0/2000/XP le fichier est dans un autre répertoire
    \verb+SYSTEM32\DRIVERS\ETC+, et lors de l'installation par défaut de Windows
    \verb+C:\WINNT\SYSTEM32\DRIVERS\ETC+.

    Ici le contenu du fichier de Windows 2000~:
\begin{example}
\begin{verbatim}
    # Copyright (c) 1993-1999 Microsoft Corp.
    #
    # This is a sample HOSTS file used by Microsoft TCP/IP for Windows.
    #
    # This file contains the mappings of IP addresses to host names. Each
    # entry should be kept on an individual line. The IP address should
    # be placed in the first column followed by the corresponding host name.
    # The IP address and the host name should be separated by at least one
    # space.
    #
    # Additionally, comments (such as these) may be inserted on individual
    # lines or following the machine name denoted by a '#' symbol.
    #
    # For example:
    #
    #      102.54.94.97     rhino.acme.com          # source server
    #       38.25.63.10     x.acme.com              # x client host

    127.0.0.1       localhost
\end{verbatim}
\end{example}

    Ici à la fin de la documentation vous pouvez enregistrer le routeur fli4l.
    Qui est indiqué dans le fichier base.txt vous trouverez aussi l'adresse IP
    de la carte réseau configurée, pour le réseau interne de fli4l par exemple

\begin{example}
\begin{verbatim}
    IP_NET_1='192.168.6.1/24'
\end{verbatim}
\end{example}

    et le nom du routeur fli4l qui à

\begin{example}
\begin{verbatim}
    HOST_1='192.168.6.1 fli4l'
\end{verbatim}
\end{example}

    été enregistré pour fli4l, placer suivante l'adresse IP et le nom fli4l
    dans le fichier host~:

    192.168.6.1     fli4l

    Maintenant, vous devez enregistrer le fichier sous le nom \textbf{hosts}. Si vous
    avez utiliser le Bloc-notes pour ouvrir le fichier il sera enregistré entant
    que hosts.txt (Pour vérifier, il est nécessaire de déactiver "Masquer
    les extensions de fichier pour les types de fichiers connus" sous Windows,
    sinon vous n'allez pas voir cette particularité ennuyeux du Bloc-notes.) Comme
    nous avons besoin que le fichier s'appelle "hosts", nous allons le débaptiser
    et l'enregistrer avec le nom "hosts". vous devez redémarrer Windows, pour
    terminer les préparatifs.

    Création d'une nouvelle imprimante (Démarrer/Paramètres/Imprimantes/Ajouter
    une imprimante) et sélectionnez "Imprimante réseau". Dans "Connexion à cette
    imprimante" ajoutez dans le nom \verb+\\NOMFLI4L\NOMIPRIMANTE+. vous remplacer
    "NOMFLI4L" par le nom du routeur fli4l et "NOMIPRIMANTE" par le nom de l'imprimante.
     "NOMIPRIMANTE" est différent en fonction du type de connexion (parallel, USB,
     Remote, SMB-Remote). Les valeurs sont généralement~:
    "prx" pour l'imprimante locale sur le port parallèle, "usbprx" pour l'imprimante locale
    sur un port USB, "reprx" pour l'imprimante distante et "smbprx" pour l'imprimante
    SMB distante. A la place du "x" vous indiquez 1, 2, 3, pour le premier, deuxième,
    troisième port etc. Pour sélectionner la première imprimante locale sur un
    port parallèle vous indiquez \verb+\\fli4l\pr1+ si le serveur fli4l a vraiment
    le nom fli4l. Vous avez peut être donné dans \var{LPD\_\-PARPORT\_\-x\_\-SAMBA\_\-NAME},
    \var{LPD\_\-USBPORT\_\-x\_\-SAMBA\_\-NAME}, \var{LPD\_\-REMOTE\_\-x\_\-SAMBA\_\-NAME}
    et \var{LPD\_\-SMBREMOTE\_\-x\_\-SAMBA\_\-NAME} un nom propre à l'imprimante
    pour Windows, dans se cas vous devez indiquer ce nom à la place "NOMIPRIMANTE".
    Pour les imprimantes déjà installées, selectionnez propriétés de l'imprimante
    puis dans l'onglet "détails" les procédures décritent précédemments seront
    analogues, pour la sélection d'un nouveau port, ensuite dans "ports pour
    l'imprimante" indiquez. Les autres paramètres dépendent du système
    d'exploitation~:

    Suite pour Windows 9x/Me~:

     Dans l'onglet "Détail" puis dans "Paramètre du Spooler" modifier, activer
     "Mettre les travaux d'impression dans la file d'attente" (impression est
     plus rapide) et "Lancer l'impression après la dernière page". Dans format
     de données activez "RAW" et activez egalement "Support bidirectionnel
     Désactiver".

    Suite pour Windows NT 4.0/2000/XP~:

    Dans l'onglet "Ports" déactivez "Activer la gestion du mode bidirectionnel".
    Dans l'onglet "Avancé" cliquez sur "Processeur d'impression" et entrez dans
    "Processeur d'impression" et activez "WinPrint" puis sous "Type de donnée par
    défaut" acrivez "RAW" et quitte la boîte de dialogue avec "OK". (Pour Windows
    NT 4.0 vous avez encore une case à coche "Toujours spooler les données au
    format RAW"). toujours dans l'onglet "Avancé" activez "Spouler l'impression
    des documents pour qu'elle se termine plus rapidement" et "Commencer l'impression
    après le trensfert de la dernière page dans le spouler". puis déactivez "Activer
    les fonctionnalités d'implession avancées". Maintenant, vous allez accepter
    tous les réglages effectués jusqu'à présent avec le bouton "Appliquer" et
    laisse la fenêtre de configuration affichée, car si vous cliquez sur "OK"
    Windows NT 4.0/2000/XP ne sauvegarde pas correctement les réglages effectués.

\item \textbf{OPT\_NMBD='yes'}

    Avec cette variable \smalljump{sec:OPTNMBD}{\var{OPT\_NMBD}}='yes' les
    imprimantes sur fli4l seront visible sur les PC Windows dans un environnement
    réseau.

    Création d'une nouvelle imprimante (Démarrer/Paramètres/Imprimantes/Ajouter
    une imprimante) sélectionner "imprimante réseau". Dans "sélectionner une
    imprimante partagéepar nom " vous pouvez utiliser le bouton "Parcourir". Dans
    le fichier base.txt vous trouverez le nom du routeur fli4l (HOSTNAME='fli4l')
    et le nom pour le partage réseau "prx", "usbprx", "reprx" ou "smbreprx".
    "prx" pour l'imprimante locale sur le port parallèle, "usbprx" pour l'imprimante
    locale sur un port USB, "reprx" pour l'imprimante distante et "smbprx" pour
    l'imprimante SMB distante. A la place du "x" vous indiquez 1, 2, 3, pour le
    premier, deuxième, troisième port etc. Vous avez peut être donné dans
    \var{LPD\_\-PARPORT\_\-x\_\-SAMBA\_\-NAME}, \var{LPD\_\-USBPORT\_\-x\_\-SAMBA\_\-NAME},
    \var{LPD\_\-REMOTE\_\-x\_\-SAMBA\_\-NAME} et \var{LPD\_\-SMBREMOTE\_\-x\_\-SAMBA\_\-NAME}
    un nom propre à l'imprimante pour Windows, dans se cas vous devez indiquer ce nom.
    Pour les imprimantes déjà installées, selectionnez propriétés de l'imprimante
    puis dans l'onglet "Détail" les procédures décritent précédemments seront
    analogues, pour la sélection d'un nouveau port, ensuite dans "ports pour
    l'imprimante" indiquez. Les autres paramètres dépendent du système
    d'exploitation~:

    Suite pour Windows 9x/Me~:

     Dans l'onglet "Détail" puis dans "Paramètre du Spooler" modifier, activer
     "Mettre les travaux d'impression dans la file d'attente" (impression est
     plus rapide) et "Lancer l'impression après la dernière page". Dans format
     de données activez "RAW" et activez egalement "Support bidirectionnel
     Désactiver".

    Suite pour Windows NT 4.0/2000/XP~:

    Dans l'onglet "Ports" déactivez "Activer la gestion du mode bidirectionnel".
    Dans l'onglet "Avancé" cliquez sur "Processeur d'impression" et entrez dans
    "Processeur d'impression" et activez "WinPrint" puis sous "Type de donnée par
    défaut" acrivez "RAW" et quitte la boîte de dialogue avec "OK". (Pour Windows
    NT 4.0 vous avez encore une case à coche "Toujours spooler les données au
    format RAW"). toujours dans l'onglet "Avancé" activez "Spouler l'impression
    des documents pour qu'elle se termine plus rapidement" et "Commencer l'impression
    après le trensfert de la dernière page dans le spouler". puis déactivez "Activer
    les fonctionnalités d'implession avancées". Maintenant, vous allez accepter
    tous les réglages effectués jusqu'à présent avec le bouton "Appliquer" et
    laisse la fenêtre de configuration affichée, car si vous cliquez sur "OK"
    Windows NT 4.0/2000/XP ne sauvegarde pas correctement les réglages effectués.

    Une autre remarque à ce sujet~:

    Sur l'ordinateur Windows, le protocole de réseau TCP/IP doit être installé
    et configuré. Le réglage par défaut doit être activé sur Windows "NETBIOS avec
    TCP/IP"  le protocole que Samba utilise.

\end{enumerate}


\subsubsection{Configuration d'un client Linux LPR}

    Sur un ordinateur Linux, vous pouvez configurer l'imprimante réseau de fli4l
    avec le fichier /etc/printcap. Voir ci-dessous pour un autre système
	d'impression tels que CUPS.

    Exemple avec (le nom de l'imprimante "imprimante")~:

\begin{example}
\begin{verbatim}
    drucker:\
            :lp=:\
            :rm=fli4l:\
            :rp=pr1:\
            :sd=/var/spool/lpd/imprimante:\
            :sh:mx#0:
\end{verbatim}
\end{example}

    Dans "rm=fli4l" vous indiquez le nom du routeur fli4l. Vous paramétrez selon
    vos besion, si la file d'attente de l'imprimante sous Linux s'appelle autrement.
    Ici le nom est "imprimante".

    Le nom du port d'imprimante distante est "rp = PR1" vous pouvez aussi indiquer~:

    \begin{description}
    \item[:rp=pr1:\ ] Pour la première imprimante parallèle sur fli4l
    \item[:rp=pr2:\ ] Pour la deuxième imprimante parallèle sur fli4l

    \item[:rp=usbpr1:\ ] Pour la première imprimante USB sur fli4l
    \item[:rp=usbpr2:\ ] Pour la première imprimante USB sur fli4l

    \item[:rp=repr1:\ bzw. :rp=repr2:\ ] Pour configurer une connexion sur un
    serveur d'impression distant

    \item[:rp=smbrepr1:\ bzw. :rp=smbrepr2:\ ] Pour configurer une connexion sur
    un serveur d'impression SMB distant

    \end{description}

    \wichtig{Après la configuration du fichier /etc/printcap, vous devez créer
    le répertoire /var/spool/lpd/imprimante avec la commande mkdir.}

    Maintenant avec la commande "lpr -P imprimante Nom du fichier" vous pouvez
    imprimer un fichier de l'ordinateur Linux via fli4l.

Beaucoup de distributions plus récents utilisent des systèmes d'impression
alternatifs et des outils de configuration personnalisables qui échoue lors
de la configuration décrite ci-dessus. Pour cette raison Belle Peter a contribué
à la documentation de la distribution SuSE (version 8.1) en Allemagne~:

SuSE 8.1 avec le système d'impression CUPS standard, le dispositif est très
confortable à utilisé.

Sous YAST2 vous choisissez dans la section Matériel de configuration de
l'imprimante. Si vous avez déjà configuré l'imprimante locale, vous pouvez
sauter la détection automatique en toute sécurité ;-)
Dans la fenêtre "Configuration de l'imprimante" choisir le bouton "Configurer ...''
ensuite sélectionner la rubrique  "Montrer plusieurs types de connexion ..." et
confirmé par "Suivant". Différents types d'imprimantes sont affichées. Comme il
s'agit d'un paquet LPD compatible, vous choisissez la première entrée "Filtrage
LPD et transfert de la file d'attente". Après une nouvelle confirmation avec
"Suivant", vous arrivez à la réelle configuration~:
Vous pouvez enregistrer ici le routeur, si vous n'êtes pas sûr du nom du routeur
utilisé le bouton "Rechercher" - laissez "serveur LPD" et faite une entrée
automatiquement, ou saisissez directement l'adresse IP du routeur.
Le nom de la file d'attente pour l'impression est dans le second champ. Pour
la première imprimante parallèle connectée à fli4l "pr1", pour la deuxième "pr2",
pour la troisième "pr3", si l'imprimantes est connectée à fli4l via le port USB
"usbpr1", "usbpr2" etc., si l'imprimante distante est contrôlée par fli4l "repr1",
 "repr2" etc., si l'imprimante SMB distante est contrôlée par fli4l "smbrepr1",
"smbrepr2". Si vous cliquez sur le bouton "Test LPD accès à distance" Vous pouvez
voir si les paramètres sont corrects. Pour pouvez confirmer, vous pouvez poursuivre
dans boite de dialogue avec "Suivant". Dans la fenêtre suivante vous indiquez
le nom avec lequel l'imprimante sera utilisée pour imprimer à partir d'applications,
attribué. Les champs "Description de l'imprimante" et "Emplacement de l'imprimante"
restent vides. On continu avec "Suivant" ...
Maintenant, vous sélectionnez l'imprimante connectée au routeur, confirmez la
sélection et choisit le bon pilote, ensuite ferme toute la configuration en
cliquant sur le bouton "Quitté" et en confirmant "Oui".
L'imprimante est maintenant entièrement installée et il sera possible d'imprimer
à partir de la plupart des applications.


\subsubsection{Configuration d'un client Mac (MacOSX 10.3.2)}

    Vous ouvrez "Préférences Système" puis dans le menu "Configuration d'imprimante"
    cliquez sur "Ajouter". Ensuite dans "Imprimante - TCP/IP" et sélectionné comme
    type d'imprimante "LPD/LPR". Puis dans "Adresse d'imprimante" entrez l'adresse
    IP du routeur fli4l. Maintenant, vous devez préciser le "nom de la file d'attente".
    la première imprimante parallèle connectée à fli4l "pr1", pour la deuxième
    "pr2", pour la troisième "pr3", si l'imprimantes est connectée à fli4l via
    le port USB "usbpr1", "usbpr2" etc., si l'imprimante distante est contrôlée
    par fli4l "repr1", "repr2" etc., si l'imprimante SMB distante est contrôlée
    par fli4l "smbrepr1", "smbrepr2". Ensuite, vous choisissez le modèle
    d'imprimante dans la liste et vous cliquez sur "Ajouter".

