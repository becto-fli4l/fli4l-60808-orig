% Synchronized to r35642
\section{SAMBA\_LPD - Support For Windows Print- And File Services In A fli4l Network}

The package \var{SAMBA\_\-LPD} consist of the OPT packages

\begin{itemize}
\item \var{OPT\_\-SAMBA} - Samba as a file and print server
\item \var{OPT\_\-SAMBATOOLS} - tools for Samba like i.e.
  for sending messages to Windows clients, mounting of
  network shares to the router's file system, ...
\item \var{OPT\_\-NMBD} - NETBIOS name server (support for network shares)
\item \var{OPT\_\-LPD} - printer support
\end{itemize}

Although all four OPT packages are combined into one installation package,
they may be enabled or disabled all individually. When disabling you loose the
functionality of the according OPT. An exception is \var{OPT\_\-NMBD}, which does
not work without \var{OPT\_\-SAMBA}.

\wichtig{When using \var{OPT\_\-LPD} you have to disable \var{OPT\_\-LPDSRV}!}

The individual OPT packages are described in the following sections.

\marklabel{sec:OPTSAMBA}{\subsection {OPT\_SAMBA - Samba As A File- And Print Server}}
\configlabel{OPT\_SAMBA}{OPTSAMBA}


    With \var{OPT\_\-SAMBA}='yes' Windows clients may print directly over the SMB protocol.
    No additional client software (except for the printer driver) is needed.

    The most important precondition for printing over Samba is always setting
     \smalljump{sec:OPTLPD}{\var{OPT\_LPD}}='yes'!

    Furthermore, this optional package provides rudimentary
    file server functions.
    Rudimentary is because fli4l has no user management, and therefore the
    shares are not restricted. If a full-blown file server is needed, better use

        \altlink{http://www.eisfair.org/}.

    The version of Samba integrated in this system is always up
    to date and has no problems neither with large partitions nor with
    its use as a Primary Domain Controller (PDC). The configuration is similar
    to fli4l's and thus simple. Samba for fli4l is primarily intended for a simpler
    printer configuration under Windows.

    It is possible to install Samba without Nmbd (NetBios Name Server, see \smalljump{sec:OPTNMBD}{\var{OPT\_NMBD}})
    because both together need a lot of free space on the boot medium.

    Hence the programs normally belonging together have been split into single OPT-
    packages - \var{OPT\_\-SAMBA} und \smalljump{sec:OPTNMBD}{\var{OPT\_NMBD}}.
    Although this OPT packages are tied to a big Package together with \smalljump{sec:OPTLPD}{\var{OPT\_LPD}}
    they may be activated individually.
    An exception is \var{OPT\_\-NMBD}, which does not work without \var{OPT\_\-SAMBA}.

    Those that do not need network neighborhood browsing under Windows set
    \var{OPT\_\-SAMBA}='yes' and \smalljump{sec:OPTNMBD}{\var{OPT\_NMBD}}='no'.
    Although the printer shares are not shown under Windows, they may as well be
    accessed if the path to them is known. A detailed description can be found later
    under ``Setting up a Windows-SMB client with Samba activated (OPT\_SAMBA='yes')''.

    Those that need network neighborhood browsing under Windows set
\begin{example}
\begin{verbatim}
    OPT_SAMBA='yes' and OPT_NMBD='yes'.
\end{verbatim}
\end{example}

    Regarding the firewall: if \verb+PF_INPUT_ACCEPT_DEF='yes'+ (resp.
    \verb+PF6_INPUT_ACCEPT_DEF='yes'+ for IPv6) is set, then rules will be added
    to the the INPUT chain which will open the Samba ports (137--139 and 445)
    for access from the nets configured (see variables
    \jump{SAMBABINDALL}{\var{SAMBA\_BIND\_ALL}},
    \jump{SAMBABINDIPV4x}{\var{SAMBA\_BIND\_IPV4\_x}} and
    \jump{SAMBABINDIPV6x}{\var{SAMBA\_BIND\_IPV6\_x}}). If specifying
    \verb+PF_INPUT_ACCEPT_DEF='no'+ (resp. \verb+PF6_INPUT_ACCEPT_DEF='no'+ for
    IPv6) you will have to ensure access for your clients manually.

\begin{description}
\config{SAMBA\_WORKGROUP}{SAMBA\_WORKGROUP}{SAMBAWORKGROUP}

    In order for the printer shares to become visible in the workgroup defined
    in Windows, the workgroup name for Samba must match the Windows' workgroup name.
    If for example on Windows the workgroup name is ``workgroup'', this variable must
    be defined as follows:

\begin{example}
\begin{verbatim}
    SAMBA_WORKGROUP='workgroup'
\end{verbatim}
\end{example}

    This is also the default setting.

\config{SAMBA\_TRUSTED\_NETS}{SAMBA\_TRUSTED\_NETS}{SAMBATRUSTEDNETS}

    Which nets are allowed to access Samba?

    This parameter defines which nets should be able to access Samba.
    Samba determines and considers the internal networks from fli4l's
    basic settings when creating a new configuration. For security reasons
    only computers from these networks are permitted access.
    If computers from other networks should be able to access Samba, they
    must be configured separately here. It is sufficient to specify the
    additional networks like this:

\begin{example}
\begin{verbatim}
    NETWORK/NETMASK
\end{verbatim}
\end{example}

     for example the net 192.168.6.0:

\begin{example}
\begin{verbatim}
     SAMBA_TRUSTED_NETS='192.168.6.0/24'
\end{verbatim}
\end{example}

    Default Setting: \var{SAMBA\_\-TRUSTED\_\-NETS}=''


\config{SAMBA\_LOG}{SAMBA\_LOG}{SAMBALOG} Logging errors in log.smb and log.nmb:
    'yes' or 'no'

    This variable specifies whether actions are recorded in the files
    log.smb and log.nmb or not. To which directory these files are written
    is decided using \smalljump{SAMBALOGDIR}{\var{SAMBA\_LOGDIR}}
    or by the selected installation type. The variable should only
    be set to 'yes' for debugging, because the log files may be written
    to the RAM disk, depending on the setting for \smalljump{SAMBALOGDIR}{\var{SAMBA\_LOGDIR}}
    and cause overflows there. \var{SAMBA\_\-LOG} applies to both
    \smalljump{sec:OPTSAMBA}{\var{OPT\_SAMBA}} and \smalljump{sec:OPTNMBD}{\var{OPT\_NMBD}},
    because they both depend on and won't run without each other.
    If you set \var{SAMBA\_\-LOG}='no' we urge you to read what to obey for
    \smalljump{SAMBALOGDIR}{\var{SAMBA\_LOGDIR}}.

    Default Setting: \var{SAMBA\_\-LOG}='no'


\config{SAMBA\_LOGDIR}{SAMBA\_LOGDIR}{SAMBALOGDIR}

    The log directory for the files log.smb and log.nmb

    This variable sets the directory to which the
    files log.smb and log.nmb are written. The variable can
    either be empty or must hold an absolute path to a
    writable directory. An absolute path always starts
    with a '/'. The directory has to exist already.
    If the variable is empty, the presence of a writable
    partition mounted under /data determines
    where the log files are stored:

    If no writable partition mounted under /data exists
    (typically all installation types except for B) the files are
    written to /var/log on the RAM disk.

    If it exists the files will be written to /data (the data partition).

    If the variable is not empty the files log.smb and log.nmb
    will be written to the directory specified (if it is writable). Setting this
    to a read only directory (i.e. the opt partition) does not make sense.
    If the log files can not be written Samba may not start. So please think
    carefully about the content of \var{SAMBA\_\-LOGDIR}.

    If setting \smalljump{SAMBALOG}{\var{SAMBA\_LOG}}='no' the variable
    \var{SAMBA\_\-LOGDIR} has to be either empty or point to a directory
    on a linux file system (minix, ext2/3/4) because with
    \smalljump{SAMBALOG}{\var{SAMBA\_LOG}}='no' log.smb and log.nmb will
    be symlinked to /dev/null which only works there.
    If you definitely do not want to have Samba log files or symlinks to
    /var/log you may specify for example

\begin{example}
\begin{verbatim}
     SAMBA_LOG='no'
     SAMBA_LOGDIR='/tmp'
\end{verbatim}
\end{example}

    In most use cases \var{SAMBA\_\-LOGDIR}='' is the right decision,
    this is why this is the default setting.

    Default Setting: \var{SAMBA\_\-LOGDIR}=''

\config{SAMBA\_TDBPATH}{SAMBA\_TDBPATH}{SAMBATDBPATH}

    This variable configures the directory where the Samba server stores
    its data in so-called TDB files. Among other things these files
    contain informations about which printer drivers have been uploaded
    to the fli4l printer server (for details refer to section
    \jump{sec:OPTSAMBAPOINTANDPRINT}{``Point'n'Print''}). The
    actual printer drivers are also stored in this directory.
    You can set this variable to `auto '; in this case fli4l
    will create a directory on a persistent storage medium
    and mount it under \texttt{/var/lib/persistent/samba/db}.

    Default Setting: \verb+SAMBA_TDBPATH='auto'+

    Example: \verb+SAMBA_TDBPATH='/data/samba/tdb'+

\config{SAMBA\_SPOOLPATH}{SAMBA\_SPOOLPATH}{SAMBASPOOLPATH}

    This variable configures the so-called spool directory for
    incoming print jobs. When printed through the Samba protocol,
    the print data will first be spooled to the directory set here before
    being passed to the LPD print server. You may set this variable to
    `auto', in this case fli4l will create a spool directory on a persistent
    storage medium and mount it to \texttt{/var/lib/persistent/samba/spool}.

    Please note that the content of this directory will be erased when starting
    your fli4l router. So you should not specify a directory here that contains
    other important data!

    Default Setting: \verb+SAMBA_SPOOLPATH='auto'+

    Example: \verb+SAMBA_SPOOLPATH='/data/samba/spool'+

\config{SAMBA\_BIND\_ALL}{SAMBA\_BIND\_ALL}{SAMBABINDALL}

    With \verb+SAMBA_BIND_ALL='yes'+ the Samba server will listen for queries
    on \emph{all} available local network interfaces. If this is not desired
    you have to set \verb+SAMBA_BIND_ALL='no'+ and in addition configure
    the nets to be processed by the Samba server using the arrays \var{SAMBA\_BIND\_IPV4\_\%}
    resp. \var{SAMBA\_BIND\_IPV6\_\%}.

    Default Setting: \verb+SAMBA_BIND_ALL='no'+

\configlabel{SAMBA\_BIND\_IPV4\_N}{SAMBABINDIPV4N}
\config{SAMBA\_BIND\_IPV4\_x}{SAMBA\_BIND\_IPV4\_x}{SAMBABINDIPV4x}

    If \verb+SAMBA_BIND_ALL='no'+, you have to use this array to configure the IPv4 nets
    for which the Samba server will answer on queries.

    Example:
\begin{example}
\begin{verbatim}
    SAMBA_BIND_IPV4_N='1'
    SAMBA_BIND_IPV4_1='IP_NET_1'
\end{verbatim}
\end{example}

    Default Setting: \verb+SAMBA_BIND_IPV4_N='0'+

\configlabel{SAMBA\_BIND\_IPV6\_N}{SAMBABINDIPV6N}
\config{SAMBA\_BIND\_IPV6\_x}{SAMBA\_BIND\_IPV6\_x}{SAMBABINDIPV6x}

    If \verb+SAMBA_BIND_ALL='no'+, you have to use this array to configure the IPv6 nets
    for which the Samba server will answer on queries.

    Example:
\begin{example}
\begin{verbatim}
    SAMBA_BIND_IPV6_N='1'
    SAMBA_BIND_IPV6_1='IPV6_NET_1'
\end{verbatim}
\end{example}

    Default Setting: \verb+SAMBA_BIND_IPV6_N='0'+

\config{LPD\_PARPORT\_x\_SAMBA\_NAME}{LPD\_PARPORT\_x\_SAMBA\_NAME}{LPDPARPORTxSAMBANAME}

    Set the fli4l printer name in network neighborhood for the x'th printer on fli4l's parallel port
    (\smalljump{LPDPARPORTxIO}{\var{LPD\_PARPORT\_x\_IO}}) here. Of course

\begin{example}
\begin{verbatim}
    OPT_NMBD='yes'
\end{verbatim}
\end{example}

    has to be set else it won't show up there. Valid names have a maximum of 8 characters
    and only consist of chars or numbers. Umlauts and special characters like ä, ö, ü,
    ß, \_, @, a.s.o. are not allowed!

    If the variable is left empty the predefined printer name is used. This is
    prx for local printers at the parallel port, x stands for 1, 2, 3 a.s.o., meaning the
    first, second, third ... port.

    Default Setting: \var{LPD\_\-PARPORT\_\-1\_\-SAMBA\_\-NAME}=''


\config{LPD\_PARPORT\_x\_SAMBA\_NET}{LPD\_PARPORT\_x\_SAMBA\_NET}{LPDPARPORTxSAMBANET}

    This variable can be used to control which hosts may use the local
    printer at fli4l's x-th parallel port. You can use this to restrict access
    to individual computers or individual subnets. By default the variable remains empty.
    This enables all computers of the internal network (including all subnets) to print
    on this printer (see \smalljump{LPDPARPORTxIO}{\var{LPD\_PARPORT\_x\_IO}}). With
    two printers on fli4l's parallel ports
    \textbf{\var{LPD\_\-PARPORT\_\-1\_\-SAMBA\_\-NET}} and
    \textbf{\var{LPD\_\-PARPORT\_\-2\_\-SAMBA\_\-NET}} have to exist.

    Set the variable like this:

\begin{itemize}
\item Specification of IP addresses in one line separated by spaces:

\begin{example}
\begin{verbatim}
    LPD_PARPORT_1_SAMBA_NET='192.168.6.2 192.168.0.1'
\end{verbatim}
\end{example}

    With two nets 192.168.141.0/255.255.255.0 and
    192.168.142.0/255.255.255.0 and one printer on the first parallel port:

\item  Specification of an IP range without host part:

\begin{example}
\begin{verbatim}
    LPD_PARPORT_1_SAMBA_NET='192.168.141. 192.168.142.'
\end{verbatim}
\end{example}

    or better

\begin{example}
\begin{verbatim}
    LPD_PARPORT_1_SAMBA_NET='192.168.'
\end{verbatim}
\end{example}

    Please note the dot at the end!
\end{itemize}

    Default Setting: \var{LPD\_\-PARPORT\_\-1\_\-SAMBA\_\-NET}=''


\config{LPD\_USBPORT\_x\_SAMBA\_NAME}{LPD\_USBPORT\_x\_SAMBA\_NAME}{LPDUSBPORTxSAMBANAME}

    Set the fli4l network neighborhood name for the x'th printer on fli4l's USB port here.
    Of course

\begin{example}
\begin{verbatim}
    OPT_NMBD='yes'
\end{verbatim}
\end{example}

    has to be set else it won't show up there. Valid names have a maximum of 8 characters
    and only consist of chars or numbers. Umlauts and special characters like ä, ö, ü,
    ß, \_, @, a.s.o. are not allowed!

    If the variable is left empty the predefined printer name is used. This is
    usbprx for local printers at the USB port, x stands for 1, 2, 3 a.s.o., meaning the
    first, second, third ... port.

    Default Setting: \var{LPD\_\-USBPORT\_\-1\_\-SAMBA\_\-NAME}=''

\config{LPD\_USBPORT\_x\_SAMBA\_NET}{LPD\_USBPORT\_x\_SAMBA\_NET}{LPDUSBPORTxSAMBANET}

    This variable can be used to control which hosts may use the local
    printer at fli4l's x-th USB port. You can use this to restrict access
    to individual computers or individual subnets. By default the variable remains empty.
    This enables all computers of the internal network (including all subnets) to print
    on this printer (see \smalljump{LPDPARPORTxIO}{\var{LPD\_PARPORT\_x\_IO}}). With
    two printers on fli4l's USB ports
    \textbf{\var{LPD\_\-USBPORT\_\-1\_\-SAMBA\_\-NET}} and
    \textbf{\var{LPD\_\-USBPORT\_\-2\_\-SAMBA\_\-NET}} have to exist.

    See \smalljump{LPDPARPORTxSAMBANET}{\var{LPD\_PARPORT\_x\_SAMBA\_NET}} for restricting
    access to individual hosts or subnets.

    Default Setting: \var{LPD\_\-USBPORT\_\-1\_\-SAMBA\_\-NET}=''


\config{LPD\_REMOTE\_x\_SAMBA\_NAME}{LPD\_REMOTE\_x\_SAMBA\_NAME}{LPDREMOTExSAMBANAME}

    Set the fli4l network neighborhood name for the x'th printer on fli4l's
    \smalljump{LPDREMOTExIP}{\var{LPD\_REMOTE\_x\_IP}} here. Of course

\begin{example}
\begin{verbatim}
    OPT_NMBD='yes'
\end{verbatim}
\end{example}

    has to be set else it won't show up there. Valid names have a maximum of 8 characters
    and only consist of chars or numbers. Umlauts and special characters like ä, ö, ü,
    ß, \_, @, a.s.o. are not allowed!

    If the variable is left empty the predefined printer name is used. This is
    reprx for remote printers, x stands for 1, 2, 3 a.s.o., meaning the
    first, second, third ... port.

    Default Setting: \var{LPD\_\-REMOTE\_\-1\_\-SAMBA\_\-NAME}=''


\config{LPD\_REMOTE\_x\_SAMBA\_NET}{LPD\_REMOTE\_x\_SAMBA\_NET}{LPDREMOTExSAMBANET}

    This variable can be used to control which hosts may use fli4l's remote
    printers. You can use this to restrict access
    to individual computers or individual subnets. By default the variable remains empty.
    This enables all computers of the internal network (including all subnets) to print
    on this printer (see \smalljump{LPDREMOTExIP}{\var{LPD\_REMOTE\_x\_IP}}).

    See \smalljump{LPDPARPORTxSAMBANET}{\var{LPD\_PARPORT\_x\_SAMBA\_NET}} for restricting
    access to individual hosts or subnets.

    Default Setting: \var{LPD\_\-REMOTE\_\-1\_\-SAMBA\_\-NET}=''


\config{LPD\_SMBREMOTE\_x\_SAMBA\_NAME}{LPD\_SMBREMOTE\_x\_SAMBA\_NAME}{LPDSMBREMOTExSAMBANAME}

    Set the fli4l network neighborhood name for the x'th printer on fli4l's
    \smalljump{LPDSMBREMOTExSERVER}{\var{LPD\_SMBREMOTE\_x\_SERVER}} here. Of course

\begin{example}
\begin{verbatim}
    OPT_NMBD='yes'
\end{verbatim}
\end{example}

    has to be set else it won't show up there. Valid names have a maximum of 8 characters
    and only consist of chars or numbers. Umlauts and special characters like ä, ö, ü,
    ß, \_, @, a.s.o. are not allowed!

    If the variable is left empty the predefined printer name is used. This is
    smbreprx for SMB remote printers, x stands for 1, 2, 3 a.s.o., meaning the
    first, second, third ... port.

    Default Setting: \var{LPD\_\-SMBREMOTE\_\-1\_\-SAMBA\_\-NAME}=''


\config{LPD\_SMBREMOTE\_x\_SAMBA\_NET}{LPD\_SMBREMOTE\_x\_SAMBA\_NET}{LPDSMBREMOTExSAMBANET}

    This variable can be used to control which hosts may use fli4l's smb remote
    printers. You can use this to restrict access
    to individual computers or individual subnets. By default the variable remains empty.
    This enables all computers of the internal network (including all subnets) to print
    on this printer (see \smalljump{LPDSMBREMOTExSERVER}{\var{LPD\_SMBREMOTE\_x\_SERVER}}).

    See \smalljump{LPDPARPORTxSAMBANET}{\var{LPD\_PARPORT\_x\_SAMBA\_NET}} for restricting
    access to individual hosts or subnets.

    Default Setting: \var{LPD\_\-SMBREMOTE\_\-1\_\-SAMBA\_\-NET}=''

\end{description}

\begin{description}
\config{SAMBA\_ADMINIP}{SAMBA\_ADMINIP}{SAMBAADMINIP}

    If an IP address or an address range from the local network
    is specified here, the according computers have full access to
    the fli4l ramdisk over the network.
    If using \var{OPT\_NMBD='yes'} those computers can access
    fli4l over Windows' network neighborhood.

    An example with the IP address 192.168.6.2:

\begin{example}
\begin{verbatim}
    SAMBA_ADMINIP='192.168.6.2'
\end{verbatim}
\end{example}

    If you want to allow such access for multiple computers there are
    different ways:

    - Specify the IP addresses in a line separated by spaces:

    \var{SAMBA\_ADMINIP='192.168.6.2 192.168.6.3'}

    - Specify an IP range without host part:

    \var{SAMBA\_ADMINIP='192.168.'}

    Please note the dot at the end!

    Due to security concerns this variable should only be used for debugging!

    In the default settings fli4l's Ram disk is not accessable in
    network neighborhood.

    Default Setting: \var{SAMBA\_ADMINIP=''}
\end{description}

\begin{description}
\config{SAMBA\_SHARE\_N}{SAMBA\_SHARE\_N}{SAMBASHAREN}

    Creation of a defined number of shares: i.e. '2'

    By \var{SAMBA\_SHARE\_N} the number of shares to be created resp. mounted is set.
    If shares do not exist they will be created, if the do they will be simply used.
    Creating shares makes only sense with a mounted medium like a harddisk, a CD ROM
    drive or a Compact-Flash disk (see \var{OPT\_MOUNT}).

    If i.e. 2 is set here, the variables\\

        \var{SAMBA\_SHARE\_NAME\_1}\\
        \var{SAMBA\_SHARE\_RW\_1}\\
        \var{SAMBA\_SHARE\_BROWSE\_1}\\
        \var{SAMBA\_SHARE\_PATH\_1}\\
        \var{SAMBA\_SHARE\_NET\_1}\\

        and\\

        \var{SAMBA\_SHARE\_NAME\_2}\\
        \var{SAMBA\_SHARE\_RW\_2}\\
        \var{SAMBA\_SHARE\_BROWSE\_2}\\
        \var{SAMBA\_SHARE\_PATH\_2}\\
        \var{SAMBA\_SHARE\_NET\_2}\\

        have to be defined with meaningful content.

        Default Setting: \var{SAMBA\_SHARE\_N='0'}

\end{description}

\begin{description}
\config{SAMBA\_SHARE\_NAME\_x}{SAMBA\_SHARE\_NAME\_x}{SAMBASHARE_NAMEX}

        With \var{SAMBA\_SHARE\_NAME\_x} the name of the x'th share is defined.
        The share can be accessed by this name resp. with \var{OPT\_NMBD}
        activated can be seen in network neighborhood on Windows clients
        (see \var{SAMBA\_SHARE\_BROWSE\_x} below).

        Although under Windows more characters and umlauts are allowed
        you should stick to 8 characters length without umlauts for
        compatibility with DOS,
        i.e.

\begin{example}
\begin{verbatim}
    SAMBA_SHARE_NAME_1='share1'
\end{verbatim}
\end{example}

  In addition share names have to be unique on the network and should not
  appear twice.
  This name will be automatically added to the path setting from

  \var{SAMBA\_SHARE\_PATH\_x}.

  Hence fli4l will try to create a directory by the name of ``share'' if it does
  not exist.
  It is absolutely necessary that the partition on this path is mounted
  writable. If this is not the case an error message on boot will occur.
  If the directory already exists it will not be overwritten to preserve
  the data stored there.

  Default Setting: \var{SAMBA\_SHARE\_NAME\_1='share1'}

\end{description}

\begin{description}
\config{SAMBA\_SHARE\_RW\_x}{SAMBA\_SHARE\_RW\_x}{SAMBASHARERWX}

  Should the share be writable: 'yes' or 'no'

  With \var{SAMBA\_SHARE\_RW\_x} it is defined whether the x'th share
  should be writable or not.

  If choosing 'no' files from this share can only be read but nothing can
  be stored or changed in there. This makes sense for files that should
  be shared with others but should not be deleted or altered in any way.

  If 'yes' is set the share is read/writable for all IP addresses or
  networks defined in

  \var{SAMBA\_SHARE\_NET\_x]}

  or if this variable is empty then for all computers of the internal network
  (including all subnets).

  Default Setting: \var{SAMBA\_SHARE\_RW\_1='yes'}

\end{description}

\begin{description}
\config{SAMBA\_SHARE\_BROWSE\_x}{SAMBA\_SHARE\_BROWSE\_x}{SAMBASHAREBROWSEX} (needs \var{OPT\_NMBD='yes'})

  Should the x'th share be browsable: 'yes' or 'no'

  By \var{SAMBA\_SHARE\_BROWSE\_x} is defined whether the x'th share should be
  displayed in network neighborhood when \var{OPT\_NMBD} is activated or not.
  If you want to prevent other users in the network environment from seeing
  and accessing the share, specify

  \var{SAMBA\_SHARE\_BROWSE\_x='no'}

  Users knowing of the existence of the share may access it nevertheless by i.e.
  using

      $\backslash\backslash$fli4l$\backslash$sharename

  as the address in an explorer window. ``fli4l'' has to be replaced by your fli4l router's
  name - if different - and "sharename" has to be replaced by the name
  defined in \var{SAMBA\_SHARE\_NAME\_x}.

  Default Setting: \var{SAMBA\_SHARE\_BROWSE\_1='yes'}

\end{description}

\begin{description}
\config{SAMBA\_SHARE\_PATH\_x}{SAMBA\_SHARE\_PATH\_x}{SAMBASHAREPATHX}

  The path to the x'th share.

  By \var{SAMBA\_SHARE\_PATH\_x} the path to the x'th share is defined.

  An example: By\\

  \var{OPT\_MOUNT='yes'}\\
  \var{MOUNT\_N='1'}\\
  \var{MOUNT\_1\_DEV='sda4'}\\
  \var{MOUNT\_1\_POINT='/usr/local/data'}\\
  \var{MOUNT\_1\_FS='ext2'}\\
  \var{MOUNT\_1\_CHECK='yes'}\\
  \var{MOUNT\_1\_OPTION='rw'}\\

  the fourth primary partition of the first harddisk is mounted to
  the file system under /usr/local/data.

  A single share is defined with\\

  \var{SAMBA\_SHARE\_N='1'}\\
  \var{SAMBA\_SHARE\_NAME\_1='share1'}\\
  \var{SAMBA\_SHARE\_RW\_1='yes'}\\
  \var{SAMBA\_SHARE\_BROWSE='yes'}\\

  and you may create and mount the directory "share1" under /usr/local/data by defining

  \var{SAMBA\_SHARE\_PATH\_1='/usr/local/data'}

  The share name is taken from

  \var{SAMBA\_SHARE\_NAME\_1}

  i.e. in this case ``share1''.

  If the directory does not exist it will be created automatically, if it does
  it will be simply used. At the moment there is no option to delete created
  directories again via samba\_lpd.txt, an error could delete files still needed.
  If \var{OPT\_NMBD} is installed and configured files existing in the share can
  be deleted by a file manager if the share is defined writable in \var{SAMBA\_SHARE\_RW\_1}.
  The directory itself can only be deleted on the router console.

  Default Setting: \var{SAMBA\_SHARE\_PATH\_1='/usr/local/data'}

\end{description}

\begin{description}
\config{SAMBA\_SHARE\_NET\_x}{SAMBA\_SHARE\_NET\_x}{SAMBASHARENETX}

  This variable can be used to control which computers
  may use the x'th share. You can restrict access to
  individual computers or individual subnets.
  By default, the variable remains empty. By this
  all computers in the internal network (including all subnets)
  can access the share.

  The variable can be defined like \var{SAMBA\_ADMINIP}.

  - Specify the IP addresses in a line separated by spaces:

  \var{SAMBA\_SHARE\_NET\_1='192.168.6.2 192.168.0.1'}

  - Specify an IP range without host part:

  \var{SAMBA\_SHARE\_NET\_1='192.168.141. 192.168.142.'}

  For two nets 192.168.141.0/255.255.255.0 and
  192.168.142.0/255.255.255.0

  or better

  \var{SAMBA\_SHARE\_NET\_1='192.168.'}

  Please note the dot at the end!

  Default Setting: \var{SAMBA\_SHARE\_NET\_1=''}

\end{description}

\begin{description}
\config{SAMBA\_CDROM\_N}{SAMBA\_CDROM\_N}{SAMBACDROMN}

  Creation of a number of shares for CDROMs: i.e. '2'

  \var{SAMBA\_CDROM\_N} defines the number of shares for CD-ROM drives.

  This and the following variables and the extensions in
  the script rc.samba were created to make sharing of CD-ROMs
  more fault-tolerant. The script shares CDROMs
  mounted at the mount point defined in \var{OPT\_MOUNT}
  or generates a mount point for CDROMs not yet mounted
  and shares it afterwards.

  In the latter case the drive is mounted to

  /mnt/cdromx

  when needed (access via network neighborhood), the x representing
  the x'th CD-ROM. Pay attention to avoid collisions with own mount
  points with the same name. If nobody accesses the share anymore the
  drive gets unmounted automatically after some time in order to be able to
  eject the CDROM without having to unmount it manually. This eases handling
  on CDROM servers with more drives where CDs have to be swapped from time to time.

  If \var{SAMBA\_CDROM\_N} is set to the variables

  \var{SAMBA\_CDROM\_DEV\_1}
  \var{SAMBA\_CDROM\_NET\_1}

  and

  \var{SAMBA\_CDROM\_DEV\_2}
  \var{SAMBA\_CDROM\_NET\_2}

  have to be defined with meaningful values.

  Default Setting:     \var{SAMBA\_CDROM\_N='0'}

\end{description}

\begin{description}
\config{SAMBA\_CDROM\_DEV\_x}{SAMBA\_CDROM\_DEV\_x}{SAMBACDROMDEVX}

  Device name of the CDROM drive: i.e. 'sdc'

  Specify here the device to be shared. Conventions in device naming
  are explained in the documentation of \var{OPT\_MOUNT}.

\end{description}

\begin{description}
\config{SAMBA\_CDROM\_NET\_x}{SAMBA\_CDROM\_NET\_x}{SAMBACDROMNETX}

  This variable can be used to control which computers
  may use the x'th CDROM drive. You can restrict access to
  individual computers or individual subnets.
  By default, the variable remains empty. By this
  all computers in the internal network (including all subnets)
  can access the share.

  With two CDROM drives connected to fli4l

  \var{SAMBA\_CDROM\_NET\_1}

  and

  \var{SAMBA\_CDROM\_NET\_2}

  have to exist.

  The variable can be defined like \var{SAMBA\_ADMINIP}.

  - Specify the IP addresses in a line separated by spaces:

  \var{SAMBA\_CDROM\_NET\_1='192.168.6.2 192.168.0.1'}

  - Specify an IP range without host part:

  \var{SAMBA\_CDROM\_NET\_1='192.168.141. 192.168.142.'}

  For two nets 192.168.141.0/255.255.255.0 and
  192.168.142.0/255.255.255.0

  or better

  \var{SAMBA\_CDROM\_NET\_1='192.168.'}

  Please note the dot at the end!

  Default Setting: \var{SAMBA\_CDROM\_NET\_1=''}

\end{description}

\marklabel{sec:OPTSAMBATOOLS}{\subsection {OPT\_SAMBATOOLS - Special Tools For Samba }}
\configlabel{OPT\_SAMBATOOLS}{OPTSAMBATOOLS}
  Activate installation of additional Samba tools: 'yes' or 'no'.

  By setting this additional tools for Samba will be installed.
  It was asked repeatedly whether it is possible to send messages to
  the Windows clients or mount shares on Windows clients on fli4l so
  we decided to provide the appropriate tools.

  Using this tools without further knowledge is error prone.
  Who does not know the dangers, for example, of mounting
  shares on fli4l, should avoid the stuff here. I have
  tried to exclude some common sources of error with scripts
  that mount and unmount shares or send messages to one, some
  or all available Windows clients.

  Further support will not be given, hence read the documentation
  thoroughly!

  Additional files will be copied to the router:

\begin{example}
\begin{verbatim}
    smbfs.o
    nls_iso8859-1.o
    nls_cp850.o
    nmblookup
    samba-netsend
    smbclient
    smbstatus
\end{verbatim}
\end{example}

    Most important are the scripts which are exlained below.\\

\begin{verbatim}
    samba-netsend
\end{verbatim}

    With this script you may send messages to SMB hosts interactively.
    Entering this on the console the following message appears:

\begin{verbatim}
Send Message to SMB Hosts

To which SMB Hosts the message should be send?

Choice 1
--------
All SMB Hosts on configured Subnets on fli4l - type 'all'.

Choice 2
--------
fli4l Samba Clients with active connections - type 'active'.

Choice 3
--------
One ore more active SMB Hosts, type NETBIOS Names
separated with a blank, for instance 'client1 client2':
\end{verbatim}

As you see, there are three choices in the first step:

\begin{enumerate}
\item
Sending of messages to all SMB hosts in the nets configured on
the fli4l. Samba accesses the file

\begin{example}
\begin{verbatim}
   /config/base.txt
\end{verbatim}
\end{example}

while writing its configuration and all network interfaces configured there are
automatically inserted in the Samba configuration. From this the information is
generated in what nets should be searched for SMB clients and the message is sent
to all of these (except for fli4l itself). The broadcast addresses and the
NetBIOS names of the computers to which the message has to be sent will be issued.

To choose this option, enter

\begin{example}
\begin{verbatim}
   all
\end{verbatim}
\end{example}

here.

\item
Sending messages to all fli4l Samba clients with active
connections to fli4l - this only means those actually having
open connections to fli4l's Samba.

To choose this enter

\begin{example}
\begin{verbatim}
   active
\end{verbatim}
\end{example}

here.

\item
Sending messages to one or more active SMB hosts.
The computers have to be specified with their NETBIOS names. In case
of more computers they have to be separated by a space.

After giving the relevant information the second step is reached:

Send Message to SMB Hosts

Which Message should be send?
For instance 'fli4l-Samba-Server is going down in 3 Minutes ...':

Enter the message to be sent here.
This message is displayed on clients with news service enabled.
The news service on Windows NT, Windows 2000 and Windows XP is
normally activated, in other cases it has to be activated/installed later.
On Windows 9x clients such as Windows 98 or Windows ME, the program WinPopUp
must be runnning.

\end{enumerate}

  Default Setting: \var{OPT\_\-SAMBATOOLS}='no'

\marklabel{sec:OPTNMBD}{\subsection {OPT\_NMBD - NETBIOS Name Server }}
\configlabel{OPT\_NMBD}{OPTNMBD}

  This is the program for displaying shares in Windows' network
  neighborhood (needs \smalljump{sec:OPTSAMBA}{\var{OPT\_SAMBA}}='yes').
  To browse the shared ramdisk from \smalljump{sec:OPTSAMBA}{\var{OPT\_SAMBA}}
  or own shares, set \var{OPT\_\-NMBD}='yes'.

  The SMB name server needs an additional 100 KB on fli4l's boot medium. In case
  of shortage you should try to get by without it and bind your printers by
  direct use of the network path, i.e. as \verb+\\fli4l\pr1+.

  A detailed description of the interaction of both programs can be found
  here: \smalljump{sec:OPTSAMBA}{\var{OPT\_SAMBA}}.

  Default Setting: \var{OPT\_\-NMBD}='no'


\begin{description}
\config{NMBD\_MASTERBROWSER}{NMBD\_MASTERBROWSER}{NMBDMASTERBROWSER}

  Samba as the master browser: 'yes' or 'no'

  Since fli4l in most cases runs continuously, it sometimes
  makes sense to use it as a master browser. A master browser
  in Windows networks is a computer that holds a list of available SMB
  servers (including all Windows computers with activated file and
  printer sharing). So the Windows clients can learn from
  the master browser which computers with activated file and
  printer sharing  exist in the network. In networks with a dedicated
  Server it is better to leave this task to it. In networks with
  a few Windows computers fli4l can do the job nicely.

  With \var{NMBD\_\-MASTERBROWSER}='yes' fli4l wins the master browser
  election against all Windows machines.

  Default Setting: \var{NMBD\_\-MASTERBROWSER}='no'


\config{NMBD\_DOMAIN\_MASTERBROWSER}{NMBD\_DOMAIN\_MASTERBROWSER}{NMBDDOMAINMASTERBROWSER} (needs NMBD\_MASTERBROWSER='yes')

  Samba as the domain master browser: 'yes' or 'no'

  I have resisted long to include this variable in the configuration
  because it is dangerous when improperly used.
  If you activate this option in a network with a domain controller
  which always is also the domain master browser at the same time,
  this is a reliable means to sabotage this domain controller.
  In this case, the strangest effects can occur.
  On the other hand, a domain master browser is the best solution
  to realize browsing over subnets (see cipe HOWTO).

  In what cases it is necessary to configure a domain master browser can
  not be explained in one or two words. Fortunately, however, others made this:

  \altlink{http://samba.sernet.de/dokumentation/browsing-2.html}

  \wichtig{For proper operation you need a configured WINS server
  that has to be known by all clients!
  }

  With \var{NMBD\_\-DOMAIN\_\-MASTERBROWSER}='yes' fli4l attempts to win
  the domain master browser election, but only succeeds if no other
  domain master browser exists. Is there another domain master browser
  and this setting is set, errors will definitely occur on the network
  caused by so-called browse wars, where both computers attempt to win.
  So if you do not know exactly whether your network is already running
  a domain master browser or not, keep your fingers off and stick with
  the default!

  Default Setting: \var{NMBD\_\-DOMAIN\_\-MASTERBROWSER}='no'


\config{NMBD\_WINSSERVER}{NMBD\_WINSSERVER}{NMBDWINSSERVER}

  Samba as a WINS server: 'yes' or 'no'

  In order to resolve NetBIOS names in Windows networks, there are two
  possibilities. The first one uses a static resolution with the file
  lmhosts and is hard to maintain like the DNS Name Resolution with hosts files.
  Hence, Microsoft developed WINS:
  \textbf{W}indows \textbf{I}nternet \textbf{N}ame \textbf{S}ervice

  WINS has the advantage that the NETBIOS name resolution is achieved
  via direct requests to a WINS server and not by broadcasts. The
  WINS database is created dynamically by the server, but has the
  disadvantage that the server has to be entered in the TCP/IP protocol
  properties of every client. Samba has this server implemented and
  thus it is available also for fli4l.

  To run fli4l as a WINS server, \smalljump{sec:OPTSAMBA}{\var{OPT\_SAMBA}},
  \smalljump{sec:OPTNMBD}{\var{OPT\_NMBD}} and \var{NMBD\_\-WINSSERVER}
  must be set to ``yes'' and in the TCP/IP protocol properties of the
  client's network card the ``WINS-resoltuion'' has to be activated.

  Under WINS server search order the IP address of the fli4 machine
  has to be added.

  Although only the choice between WINS OR DHCP is possible, specifying the
  IP address of the fli4l WINS server does not complete a correct TCP/IP
  configuration. You will still have to configure either the IP address of each client
  or DHCP.

  In networks with a Windows server on which the WINS Server service
  is enabled, you should leave this task to Windows. But in
  Networks with a few Windows computers also fli4l can fulfil this task
  easily.

\begin{example}
\begin{verbatim}
    NMBD_WINSSERVER='yes'
\end{verbatim}
\end{example}

  activates this function.
  With \var{OPT\_\-DHCP} installed and activated
\begin{example}
\begin{verbatim}
    NMBD_WINSSERVER='yes'
\end{verbatim}
\end{example}

  the IP address of fli4l will be transferred as the WINS server's IP
  to the clients.

  Default Setting: \var{NMBD\_\-WINSSERVER}='no'


\config{NMBD\_EXTWINSIP}{NMBD\_EXTWINSIP}{NMBDEXTWINSIP} (needs NMBD\_WINSSERVER='no')

  The IP address of the remote WINS server for Samba

  If, as noted above, in networks with a Windows server,
  the task to manage the WINS database should be done by it.
  In this variable you can configure fli4l as a WINS client. the
  fli4l computer then tries to register with the configured WINS server.
  It should be ensured that flil4 can not be configured as server and
  client at the same sime - the options

\begin{example}
\begin{verbatim}
  NMBD_WINSSERVER='yes'
\end{verbatim}
\end{example}

  and

\begin{example}
\begin{verbatim}
  NMBD_EXTWINSIP='IP-Adresse'
\end{verbatim}
\end{example}

  exclude each other. Generation of a boot medium will not work with this
  configuration. In this mode Samba works as a WINS-Proxy. This is of
  advantage if the net does not consist only of WINS clients, the WINS
  server is located in another subnet not reachable via broadcast
  and the Non-WINS clients are in need of a NETBIOS name resolution.
  Then the fli4l answers broadcasts of Non-WINS clients by querying the
  WINS server and redirecting the answer via broadcast to the client.

  If you want to run fli4l as a WINS client it has to know the
  IP address of the external WINS servers to register with. Precondition
  for this is \smalljump{NMBDWINSSERVER}{\var{NMBD\_WINSSERVER}}='no'.

  An example with IP address 192.168.6.11:

\begin{example}
\begin{verbatim}
  NMBD_EXTWINSIP='192.168.6.11'
\end{verbatim}
\end{example}

  With \var{OPT\_\-DHCP} installed and activated the IP address configured
  there will be transferred to the clients as the IP address of the WINS servers.

  Default Setting: \var{NMBD\_\-EXTWINSIP}=''


\end{description}

\marklabel{sec:OPTLPD}{\subsection {OPT\_LPD - Print Server For The lpr/lpd-Protocol}}
\configlabel{OPT\_LPD}{OPTLPD}

  With \var{OPT\_\-LPD}='yes' fli4l may also be used as a print server. Then
  lpd and one or more printer queues (depending on the number of printers
  connected to fli4l) will be installed to the Ram disk of the root file
  systems or to the harddisk.

  \wichtig{To make printing work seamlessly even in a multi-user environment,
  the lpd spooler is used. The print data will be stored in a spool directory.
  A separate spool directory is created for each printer. This spool directories
  are located in the ramdisk of the root file system or, if available,
  will use the hard disk. If at the start of fli4l no writable minix, ext2/3/4
  partition mounted to /data is found, each printer configured uses the ramdisk
  of the root file system. With three printers connected and simultaneously
  printing the spooled files for all three printers are created in the ramdisk.
  Note that even at small text file already causes large print jobs on Windows
  and they shortly will exists twice in memory requiring two times the ram space
  needed when printing via Samba . To make print spooling on the ramdisk work
  seamlessly you should have enough memory in the fli4l computer - the more, the
  better. For those who often have to print large documents without having at
  least 4 MB of memory left for printing, it is absolutely essential to
  use a hard drive installation to avaoid afffecting the function of the router
  by an overflowing ramdisk. Yes, you got that right. If a print job does not
  fit in the ramdisk, the result could be a router that no longer routes...}

  \emph{When spooling to disk, the processing of print jobs is limited by the space
  available on the hard disk. For this purpose, the hard drive installation
  type B should be used with an ext data partition mounted to /data.}

  \emph{If problems occur when printing large files on a non-harddisk installation
  there is simply not enough memory in the router.}

  \emph{Rule of thumb: fli4l as a router needs about 10 MB of memory (8 - 12,
  depending on the configuration). With 32 MB memory in the router the rest
  is available for printing - hence 32 - 10 = 22 MB.
  When printing with Samba the available space is reduced by half, because
  print jobs are stored twice in ram while printing:
  This means a print job can have a maximum size of 11 MB and with two jobs
  at the same time the router stops routing...
  Therefore we strongly advice a harddisk installation for printing.}

  Default Setting: \var{OPT\_\-LPD}='no'


\begin{description}

\config{LPD\_DEBUG}{LPD\_DEBUG}{LPDDEBUG}

  This variable activates or deactivates the log function for the
  LPD print server. It is recommended to activate logging only in
  case of debuging because the protocol is quite big in file size.

  The variable can either have the values `yes', `no' or a number
  between 1 and 5, where a higher value causes more log details.
  The setting `yes' is equivalent to the value 1.

  Default Setting: \verb+LPD_DEBUG='no'+

  Example: \verb+LPD_DEBUG='2'+

\config{LPD\_DEBUG\_FILE}{LPD\_DEBUG\_FILE}{LPDDEBUGFILE}

  This variable contains the path to the logfile, to which the logs
  of the LPD print server are written in case of an activated protocol
  function (see description of variable \jump{LPDDEBUG}{\var{LPD\_DEBUG}}).
  It is also possible to set the the value to `auto' to select the
  default location of the log file \texttt{/var/log/lpd.log}.

  The content of this variable will be ignored if logging is disabled
  by specifying \verb+LPD_DEBUG='no'+.

  Default Setting: \verb+LPD_DEBUG_FILE='auto'+

  Beispiel: \verb+LPD_DEBUG_FILE='/data/log/lpd.log'+

\config{LPD\_SPOOLPATH}{LPD\_SPOOLPATH}{LPDSPOOLPATH}

  This variable configures the so-called spool directory of the LPD print
  server for incoming print jobs. All print data will first be spooled to
  the directory set here before being passed to printers by the print server.
  You may set this variable to `auto', in this case fli4l will create a spool
  directory on a persistent storage medium and mount it to
  \texttt{/var/lib/persistent/samba/spool}.

  Please note that the content of this directory will be erased when starting
  your fli4l router. So you should not specify a directory here that contains
  other important data! Also note that the path configured here may not be the
  same as in \var{SAMBA\_SPOOLPATH} (with the exception that both variables
  have the value 'auto') because the two spool directories serve for different
  purposes and hence need different access settings.

  Default Setting: \verb+LPD_SPOOLPATH='auto'+

  Example: \verb+LPD_SPOOLPATH='/data/lpd/spool'+

\config{OPT\_LPD\_PARPORT}{OPT\_LPD\_PARPORT}{OPTLPDPARPORT}

  \var{OPT\_\-LPD\_\-PARPORT}='yes' specifies that local parallel printer
  ports should be used. If only USB- or remote printers are used,
  you may keep the default of the variable:

  Default Setting: \var{OPT\_\-LPD\_\-PARPORT}='no'

\config{LPD\_NETWORK\_N}{LPD\_NETWORK\_N}{LPDNETWORKN}

  This variable contains the number of entries in the array
  \var{LPD\_NETWORK\_x} (see below).

  Example: \verb+LPD_NETWORK_N='1'+

\config{LPD\_NETWORK\_x}{LPD\_NETWORK\_x}{LPDNETWORKx}

  Each entry in this array specifies a host- or network address for which
  printing via the LPD protocol \footnote{see RFC 1179} is allowed. Possible
  entries are IPv4 addresses like \verb+192.168.1.0/24+, symbolic addresses
  like \verb+IP_NET_1+ and host references like \verb+@peacock+.

  Please note that this settings are \emph{not} necessary if printer access
  is only accomplished via the Samba server! This is \emph{only} relevant if printing
  via the LPD protocol is desired and hence is mostly of interest for Linux- and Mac
  machines. It is more convenient for Windows machines to print over Samba because
  LPD needs the additional installation of the ``UNIX print services for Windows''.

  Example:
\begin{example}
\begin{verbatim}
    LPD_NETWORK_1='IP_NET_1'
    LPD_NETWORK_2='192.168.1.0/24'
    LPD_NETWORK_3='@client'
\end{verbatim}
\end{example}

\config{LPD\_PARPORT\_N}{LPD\_PARPORT\_N}{LPDPARPORTN} (needs OPT\_LPD\_PARPORT='yes')

  By \var{LPD\_\-PARPORT\_\-N} the number of local parallel ports in use
  is determined. With a printer at the first parallel port configured in
  samba\_lpd.txt set

\begin{example}
\begin{verbatim}
    LPD_PARPORT_N='1'
\end{verbatim}
\end{example}
  For two printer ports increment \var{LPD\_\-PARPORT\_\-N},
  hence

\begin{example}
\begin{verbatim}
    LPD_PARPORT_N='2'
\end{verbatim}
\end{example}

  The corresponding settings in\\
      \emph{\var{LPD\_\-PARPORT\_\-1\_\-IO}}\\
      \emph{\var{LPD\_\-PARPORT\_\-1\_\-IRQ}}\\
      \emph{\var{LPD\_\-PARPORT\_\-1\_\-DMA}}\\
  and\\
      \emph{\var{LPD\_\-PARPORT\_\-2\_\-IO}}\\
      \emph{\var{LPD\_\-PARPORT\_\-2\_\-IRQ}}\\
      \emph{\var{LPD\_\-PARPORT\_\-2\_\-DMA}}\\

  and, if using Samba in addition, also\\

      \emph{\var{LPD\_\-PARPORT\_\-1\_\-SAMBA\_\-NET}}\\
      \emph{\var{LPD\_\-PARPORT\_\-2\_\-SAMBA\_\-NET}}\\

  and, if Samba printer names should be used, also\\

      \emph{\var{LPD\_\-PARPORT\_\-1\_\-SAMBA\_\-NAME}}\\
      \emph{\var{LPD\_\-PARPORT\_\-2\_\-SAMBA\_\-NAME}}\\

  have to exist.\\

  Default Setting: \var{LPD\_\-PARPORT\_\-N}='1'


\config{LPD\_PARPORT\_x\_IO}{LPD\_PARPORT\_x\_IO}{LPDPARPORTxIO}

  \var{LPD\_\-PARPORT\_\-x\_\-IO} configures the x'th local parallel printer port.
  For two printers at fli4l's parallel ports two entries with the valid values of

\begin{itemize}
\item 0x3bc
\item 0x378 or
\item 0x278
\end{itemize}

  have to exist, i.e.

\begin{example}
\begin{verbatim}
    LPD_PARPORT_1_IO='0x378'
\end{verbatim}
\end{example}

  and

\begin{example}
\begin{verbatim}
    LPD_PARPORT_2_IO='0x278'
\end{verbatim}
\end{example}

  \wichtig{
  Parallel ports on the mainboard or on ISA interface cards are supported with
  the values mentioned above. PCI cards with certain NETMOS chips providing parallel
  ports are also possible. You have to determine the device via}

\begin{example}
\begin{verbatim}
        cat /proc/pci
\end{verbatim}
\end{example}
  \emph{. Search for the device with matching Vendor-ID and Device-ID and choose the
  IO-addresses from the following list:}

      \begin{itemize}
      \item Nm9705CV  (Vendor id=9710, Device id=9705, Port1 1st entry)
      \item Nm9735CV  (Vendor id=9710, Device id=9735, Port1 3rd entry)
      \item Nm9805CV  (Vendor id=9710, Device id=9805, Port1 1st entry)
      \item Nm9715CV  (Vendor id=9710, Device id=9815, Port1 1st entry, Port2 3rd entry)
      \item Nm9835CV  (Vendor id=9710, Device id=9835, Port1 3rd entry)
      \item Nm9755CV  (Vendor id=9710, Device id=9855, Port1 1st entry, Port2 3rd entry)
      \end{itemize}

  \emph{This was included without matching hardware to test. Consider this an
  experimental feature. In case of errors please report to the newsgroups
  and provide all possible informations!}

  To make printing work ensure yourself to which io addresses the built-in
  interfaces are set. The io addresses can either be set in the BIOS of the
  computer or in very old computers are not configurable, but only displayed
  at boot time. Additional ports can usually be configured via jumpers on the
  card which are desribed in the (hopefully still existing) documentation.

  In case of \var{OPT\_\-LCD}='yes' in addition make sure that the address
  set there in \var{LCD\_\-ADDRESS} does not collide with the address set
  in samba\_lpd.txt. A boot media can not be build with this conflict!

  Default Setting: \var{LPD\_\-PARPORT\_\-1\_\-IO}='0x378'


\config{LPD\_PARPORT\_x\_IRQ}{LPD\_PARPORT\_x\_IRQ}{LPDPARPORTxIRQ}

  {\var{LPD\_\-PARPORT\_\-x\_\-IRQ} specifies whether interrupt mode is used
  for printing to reduce processor load or not. To use this mode the
  interfaces on the mainboard or addon cards have to be set to ECP/EPP
  mode either via BIOS or Jumpers. Activate interrupt mode like this:

\begin{example}
\begin{verbatim}
        LPD_PARPORT_1_IRQ='yes'
\end{verbatim}
\end{example}

  deativate it like this:

\begin{example}
\begin{verbatim}
        LPD_PARPORT_1_IRQ='no'
\end{verbatim}
\end{example}

  and configure normal or SPP mode for the interfaces in mainboard or addon
  cards. If errors occur and somthing goes wrong first try to test with

\begin{example}
\begin{verbatim}
        LPD_PARPORT_1_IRQ='no'
\end{verbatim}
\end{example}

  Default Setting: \var{LPD\_\-PARPORT\_\-1\_\-IRQ}='no'}


\config{LPD\_PARPORT\_x\_DMA}{LPD\_PARPORT\_x\_DMA}{LPDPARPORTxDMA}

  {\var{LPD\_\-PARPORT\_\-x\_\-DMA} specifies whether DMA mode is used
  for printing to reduce processor load or not. To use this mode the
  interfaces on the mainboard or addon cards have to be set to ECP/EPP
  mode either via BIOS or Jumpers. Activate interrupt mode like this:

\begin{example}
\begin{verbatim}
        LPD_PARPORT_1_DMA='yes'
\end{verbatim}
\end{example}

  Precondition is to set

\begin{example}
\begin{verbatim}
  LPD_PARPORT_1_IRQ='yes'
\end{verbatim}
\end{example}

  deativate it like this:

\begin{example}
\begin{verbatim}
  LPD_PARPORT_1_DMA='no'
\end{verbatim}
\end{example}

  and configure normal or SPP mode for the interfaces in mainboard or addon
  cards. If errors occur and somthing goes wrong first try to test with

\begin{example}
\begin{verbatim}
  LPD_PARPORT_1_DMA='no'
\end{verbatim}
\end{example}

  Default Setting: \var{LPD\_\-PARPORT\_\-1\_\-DMA}='no'}


\config{OPT\_LPD\_USBPORT}{OPT\_LPD\_USBPORT}{OPTLPDUSBPORT}

  \var{OPT\_\-LPD\_\-USBPORT}='yes' specifies the use of local
  USB printer ports.

  Support for USB printers has to be activated in package \var{OPT\_\-USB}
  in addition. Depending on the driver this might look like this:

\begin{example}
\begin{verbatim}
    OPT_USB='yes'
    USB_LOWLEVEL='uhci'
    USB_PRINTER='yes'
\end{verbatim}
\end{example}

    or like this:

\begin{example}
\begin{verbatim}
    OPT_USB='yes'
    USB_LOWLEVEL='usb-ohci'
    USB_PRINTER='yes'
\end{verbatim}
\end{example}

  \wichtig{The configuration option for USB printers has been integrated,
  without having appropriate hardware available for testing. Hence this
  has to be regarded as an experimental feature. In case of errors, please
  post detailed information in the newsgroup!
  Many USB printers are GDI printers. GDI printers can not be addressed.
  Questions about problems with USB printers will only be answered if you
  communicate that you have ensured that the affected printer is not a GDI printer!
  }

  If you only want to use printers on parallel ports or remote printers
  the variable can be left at its default:

  Default Setting: \var{OPT\_\-LPD\_\-USBPORT}='no'


\config{LPD\_USBPORT\_N}{LPD\_USBPORT\_N}{LPDUSBPORTN} (needs OPT\_LPD\_USBPORT='yes')

  In \var{LPD\_\-USBPORT\_\-N} set the number of local USB printer ports to be used.
  A printer on the first USB interface is set like this

\begin{example}
\begin{verbatim}
    LPD_USBPORT_N='1'
\end{verbatim}
\end{example}

  For two USB printer ports \var{LPD\_\-USBPORT\_\-N} has to be incremented:

\begin{example}
\begin{verbatim}
  LPD_USBPORT_N='2'
\end{verbatim}
\end{example}

  When using Samba the corresponding settings in\\
  \var{LPD\_\-USBPORT\_\-1\_\-SAMBA\_\-NET}\\
  \var{LPD\_\-USBPORT\_\-2\_\-SAMBA\_\-NET}\\
  and, if Samba printer names have been assigned,\\
  also \var{LPD\_\-USBPORT\_\-1\_\-SAMBA\_\-NAME}\\
  and \var{LPD\_\-USBPORT\_\-2\_\-SAMBA\_\-NAME}\\
  have to exist.\\

  \wichtig{If more than one USB printer is used, note that the order
  of turning on the printers determines which is the first and the second
  USB printer. The second USB printer will automatically become the first
  printer if the first USB printer is not turned on. If the printers
  are different printer models and require different drivers on the client
  it can thus happen that printing to the selected printer produces garbage
  because of using the wrong printer language for the  print job.
  }

  Default Setting: \var{LPD\_\-USBPORT\_\-N}='1'

\config{OPT\_LPD\_REMOTE}{OPT\_LPD\_REMOTE}{OPTLPDREMOTE}

  \var{OPT\_\-LPD\_\-REMOTE}='yes' specifies that remote printers
  (not local) should be used. If only printers on local parallel or USB
  interfaces should be used leave the default untouched here:

  Default Setting: \var{OPT\_\-LPD\_\-REMOTE}='no'


\config{LPD\_REMOTE\_N}{LPD\_REMOTE\_N}{LPDREMOTEN} (needs OPT\_LPD\_REMOTE='yes')

  By \var{LPD\_REMOTE\_\-N} the number of remote printers in use
  is determined. When using this the print job from the client is sent to fli4
  and then forwarded to a remote LPD print server.

  This will also work together with Samba. If a remote printer on a non-local
  print server should be accessed over fli4l

\begin{example}
\begin{verbatim}
    LPD_REMOTE_N='1'
\end{verbatim}
\end{example}

  has to be set. For two remote print servers with two printer queues
  increment \var{LPD\_\-REMOTE\_\-N}, hence

\begin{example}
\begin{verbatim}
    LPD_REMOTE_N='2'
\end{verbatim}
\end{example}

  In addition the corresponding settings in

\begin{itemize}
\item \var{LPD\_\-REMOTE\_\-1\_\-IP}
\item \var{LPD\_\-REMOTE\_\-1\_\-PORT}
\item \var{LPD\_\-REMOTE\_\-1\_\-QUEUENAME}
\item \var{LPD\_\-REMOTE\_\-2\_\-IP}
\item \var{LPD\_\-REMOTE\_\-2\_\-PORT}
\item \var{LPD\_\-REMOTE\_\-2\_\-QUEUENAME}
\end{itemize}

  and, if Samba is used, also

\begin{itemize}
\item \var{LPD\_\-REMOTE\_\-1\_\-SAMBA\_\-NAME}
\item \var{LPD\_\-REMOTE\_\-1\_\-SAMBA\_\-NET}
\item \var{LPD\_\-REMOTE\_\-2\_\-SAMBA\_\-NAME}
\item \var{LPD\_\-REMOTE\_\-2\_\-SAMBA\_\-NET}
\end{itemize}

  have to exist.

  Default Setting: \var{LPD\_\-REMOTE\_\-N}='0'


\config{LPD\_REMOTE\_x\_IP}{LPD\_REMOTE\_x\_IP}{LPDREMOTExIP}

  By \var{LPD\_\-REMOTE\_\-x\_\-IP} the IP of the x'th remote print server is set.

  The default is a second fli4 computer accessable at 192.168.6.99.

  Default Setting: \var{LPD\_\-REMOTE\_\-1\_\-IP}='192.168.6.99'


\config{LPD\_REMOTE\_x\_PORT}{LPD\_REMOTE\_x\_PORT}{LPDREMOTExPORT}

  By \var{LPD\_\-REMOTE\_\-x\_\-PORT} the port of the x'th remote printer is set.
  Define this variable only if printing to print servers able to receive print
  jobs via ftp or netcat. If using print servers able to understand the lpd
  protocol leave this variable empty and use \smalljump{LPDREMOTExQUEUENAME}{\var{LPD\_REMOTE\_x\_QUEUENAME}}
  instead.
  Hence use EITHER \var{LPD\_\-REMOTE\_\-x\_\-PORT} OR
  \smalljump{LPDREMOTExQUEUENAME}{\var{LPD\_REMOTE\_x\_QUEUENAME}} and never
  both at the same time! One of the two variables has to be defined.

  Consult the manual of your print server or the manufacturer's website to find
  out which category your print server belongs to.
  An incomplete overview can be found at

  \altlink{http://www.lprng.com/LPRng-Reference/LPRng-Reference.html\#AEN4990}

  The default defines a third remote printer repr3 on a HP JetDirect print server
  (Interface card) with the IP 192.168.6.100 reached via port 9100 (as the link above
  shows this would also be reachable on the print queue by the name raw...).

  Here's a hint:
  If the corresponding printer is not reachable or turned off at the time of print
  job creation lpd will reach a timeout and the job can not be processed. This job
  can not be deleted with the lprm command and remains in the queue until fli4 is
  rebooted!

  Default Setting: \var{LPD\_\-REMOTE\_\-3\_\-PORT}='9100'


\config{LPD\_REMOTE\_x\_QUEUENAME}{LPD\_REMOTE\_x\_QUEUENAME}{LPDREMOTExQUEUENAME}

  By \var{LPD\_\-REMOTE\_\-x\_\-QUEUENAME} the printer queue name of the x'th
  remote printer is specified.

  Define this variable only if you want to print to print servers that
  understand the lpd protocol.
  If you want to access print servers which allow sending data via ftp or netcat
  this variable should be left blank and instead
  \smalljump{LPDREMOTExPORT}{\var{LPD\_REMOTE\_x\_PORT}} should be used.
  Hence use EITHER \var{LPD\_\-REMOTE\_\-x\_\-QUEUENAME} OR
  \smalljump{LPDREMOTExPORT}{\var{LPD\_REMOTE\_x\_PORT}} and never both
  at the same time! One of the two variables has to be defined.

  Consult the manual of your print server or the manufacturer's website to find
  out which category your print server belongs to.
  An incomplete overview can be found at

  \altlink{http://www.lprng.com/LPRng-Reference/LPRng-Reference.html\#AEN4990}

  The default defines a second fli4 computer with a printer queue by the name
  of pr1.

  Default Setting: \var{LPD\_\-REMOTE\_\-1\_\-QUEUENAME}='pr1'


\config{OPT\_LPD\_SMBREMOTE}{OPT\_LPD\_SMBREMOTE}{OPTLPDSMBREMOTE}

  \var{OPT\_\-LPD\_\-SMBREMOTE}='yes' defines the use of remote SMB printers.
  This may be shared printers on Windows- or Samba- computers.

  \wichtig{
  The configuration of such printers only makes sense if these remote
  SMB printers are turned on at printing time - spooling and storing the print
  jobs until the remote computer is online again is not possible due to the
  realization of this function as a pre-filter script for lpd.
  }

  If you only want to use printers on local parallel or USB ports or remote
  LPD printers, the variable can be left at its default:

  Default Setting: \var{OPT\_\-LPD\_\-SMBREMOTE}='no'


\config{LPD\_SMBREMOTE\_DEBUGLEVEL}{LPD\_SMBREMOTE\_DEBUGLEVEL}{LPDSMBREMOTEDEBUGLEVEL} (needs OPT\_LPD\_SMBREMOTE='yes')

  \var{LPD\_SMBREMOTE\_\-DEBUGLEVEL} sets the number of debug messages to be
  logged when printing to remote SMB printers. Only printing of one Job will
  be logged to the file /tmp/smb-print.log and it will be overwritten with every
  new job. With \var{LPD\_SMBREMOTE\_DEBUGLEVEL}='0' logging is disabled.
  If problems occur try a higher value here for debugging by the help of the
  messages in /tmp/smb-print.log.

  Default Setting: \var{LPD\_\-SMBREMOTE\_\-DEBUGLEVEL}='0'

\config{LPD\_SMBREMOTE\_N}{LPD\_SMBREMOTE\_N}{LPDSMBREMOTEN} (needs OPT\_LPD\_SMBREMOTE='yes')

  \var{LPD\_SMBREMOTE\_\-N} sets the number of SMB remote printers to be configured.
  In this way it is possible for a client to send a print job to fli4l which in turn
  will forward this print job to a remote SMB printer share.
  This also works in conjunction with Samba. If you want to access a
  SMB remote printer on a remote Windows or Samba host over flil4 set

\begin{example}
\begin{verbatim}
  LPD_SMBREMOTE_N='1'
\end{verbatim}
\end{example}

  here. For two remote Windows- or Samba computers or one remote Windows-
  or Samba computer with two printer shares \var{LPD\_\-SMBREMOTE\_\-N}
  has to be incremented, hence

\begin{example}
\begin{verbatim}
  LPD_SMBREMOTE_N='2'
\end{verbatim}
\end{example}

  In addition the corresponding settings in

\begin{itemize}
\item \var{LPD\_\-SMBREMOTE\_\-1\_\-SERVER}
\item \var{LPD\_\-SMBREMOTE\_\-1\_\-SERVICE}
\item \var{LPD\_\-SMBREMOTE\_\-1\_\-USER}
\item \var{LPD\_\-SMBREMOTE\_\-1\_\-PASSWORD}
\item \var{LPD\_\-SMBREMOTE\_\-1\_\-IP}
\item \var{LPD\_\-SMBREMOTE\_\-2\_\-SERVER}
\item \var{LPD\_\-SMBREMOTE\_\-2\_\-SERVICE}
\item \var{LPD\_\-SMBREMOTE\_\-2\_\-USER}
\item \var{LPD\_\-SMBREMOTE\_\-2\_\-PASSWORD}
\item \var{LPD\_\-SMBREMOTE\_\-2\_\-IP}
\end{itemize}

  and, if Samba is used, also

\begin{itemize}
\item \var{LPD\_\-SMBREMOTE\_\-1\_\-SAMBA\_\-NAME}
\item \var{LPD\_\-SMBREMOTE\_\-1\_\-SAMBA\_\-NET}
\item \var{LPD\_\-SMBREMOTE\_\-2\_\-SAMBA\_\-NAME}
\item \var{LPD\_\-SMBREMOTE\_\-2\_\-SAMBA\_\-NET}
\end{itemize}

  have to exist.

  Default Setting: \var{LPD\_\-SMBREMOTE\_\-N}='0'


\config{LPD\_SMBREMOTE\_x\_SERVER}{LPD\_SMBREMOTE\_x\_SERVER}{LPDSMBREMOTExSERVER}

  By \var{LPD\_\-SMBREMOTE\_\-x\_\-SERVER} the NETBIOS name of the computer
  with the x'th printer share is specified. This name is necessary because
  smbclient is used to print.

  Default is a Windows computer with the NETBIOS name ``ente''.

  Default Setting: \var{LPD\_\-SMBREMOTE\_\-1\_\-SERVER}='ente'

\config{LPD\_SMBREMOTE\_x\_SERVICE}{LPD\_SMBREMOTE\_x\_SERVICE}{LPDSMBREMOTExSERVICE}

  With \var{LPD\_\-SMBREMOTE\_\-x\_\-SERVICE} the name of the x'th
  printer share of the SMB remote printer is set. Default is the printer
  share name ``pr2''.

  Default Setting: \var{LPD\_\-SMBREMOTE\_\-1\_\-SERVICE}='pr2'

\config{LPD\_SMBREMOTE\_x\_USER}{LPD\_SMBREMOTE\_x\_USER}{LPDSMBREMOTExUSER}

  \var{LPD\_\-SMBREMOTE\_\-x\_\-USER} sets the username for accessing the
  x'th printer share. Default is the username ``king''.

  Default Setting: \var{LPD\_\-SMBREMOTE\_\-1\_\-USER}='king'


\config{LPD\_SMBREMOTE\_x\_PASSWORD}{LPD\_SMBREMOTE\_x\_PASSWORD}{LPDSMBREMOTExPASSWORD}

  \var{LPD\_\-SMBREMOTE\_\-x\_\-PASSWORD} sets the password of the x'th
  user for accessing the printer share. Default is the password ``kong''.

  Default Setting: \var{LPD\_\-SMBREMOTE\_\-1\_\-PASSWORD}='kong'


\config{LPD\_SMBREMOTE\_x\_IP}{LPD\_SMBREMOTE\_x\_IP}{LPDSMBREMOTExIP}

  \var{LPD\_\-SMBREMOTE\_\-x\_\-IP} sets the IP of the Windows- or
  Samba-computer with the x'th printer share. Default is a Windows
  computer reachable at the IP 192.168.0.6.

  Default Setting: \var{LPD\_\-SMBREMOTE\_\-1\_\-IP}='192.168.0.6'


\end{description}

\marklabel{sec:OPTSAMBAPOINTANDPRINT}{
\subsection{\var{OPT\_SAMBA\_POINT\_AND\_PRINT} -- Server-side Management of Windows Printer Drivers}}

Point'n'Print is a Windows technology for server-side management of
printer drivers. The idea is simple: If a Windows server is also a
print server offering printing services, then it is reasonable that
he also offers printer drivers for the printers connected to avoid
manually copying those drivers to each Windows client.
Point'n'Print does exactly that: First, an administrator uploads
printer drivers for all printers and computer architectures in question
to the print server. A normal user can now connect to a printer on the
server (like using file shares on the server) and the Windows client
automatically retrieves the appropriate printer driver from the print
server and installs it locally on the client. Thus, using the shared
printer on the network is possible quickly and without much
installation effort.

In the \jump{sec:OPTSAMBAPOINTANDPRINT:XP}{Appendix} it is shown how to
set up a Point'n'Print Configuration using a Windows XP clients.

\begin{description}

\config{OPT\_SAMBA\_POINT\_AND\_PRINT}{OPT\_SAMBA\_POINT\_AND\_PRINT}{OPTSAMBAPOINTANDPRINT}

This variable activates Point'n'Print functionality. For activation
\verb+OPT_SAMBA='yes'+ and \verb+OPT_LPD='yes'+  are mandatory.

Default Setting: \verb+OPT_SAMBA_POINT_AND_PRINT='no'+

Example: \verb+OPT_SAMBA_POINT_AND_PRINT='yes'+

\config{SAMBA\_PRINT\_ADMIN\_NAME}{SAMBA\_PRINT\_ADMIN\_NAME}{SAMBAPRINTADMINNAME}

In order to avoid installing and uninstalling of printer drivers by any user
(in most cases undesirable), this is only allowed to \emph{Printer Administrators}.
The name of the Windows account with such privileges is noted here.

Example: \verb+SAMBA_PRINT_ADMIN_NAME='pradmin'+

\config{SAMBA\_PRINT\_ADMIN\_PASSWORD}{SAMBA\_PRINT\_ADMIN\_PASSWORD}{SAMBAPRINTADMINPASSWORD}

Specify the password of the Windows account with printer administrator privileges.

Example: \verb+SAMBA_PRINT_ADMIN_PASSWORD='secret'+

\end{description}

\marklabel{sec:DRUCKEREINRICHTUNG}{\subsection {Printer Setup On The Clients}}

  Setting up a fli4l printer on the clients is depending on activation of
  \smalljump{sec:OPTSAMBA}{\var{OPT\_SAMBA}} and with activated
  \smalljump{sec:OPTSAMBA}{\var{OPT\_SAMBA}} whether \smalljump{sec:OPTNMBD}{\var{OPT\_NMBD}}
  is activated in addition. Also differences of client operating systems
  concerning configuration and options have to be obeyed. Therefore,
  there is a section for each configuration option.

\subsubsection{If OPT\_SAMBA Is Deactivated}
\begin{enumerate}
\item \textbf{NT4.0/2000/XP/Win7}

  If Samba is not used, install the Print Services for Unix in Windows NT4.0/2000/XP
  to access to fli4l's LPD, because Windows Standard TCP printing is using wrong ports.
  This service is called LPD Printservice in Windows 7 and above.

  The Print Services for Unix can be added with

  Start/Settings/System settings/Software/Add Windows Components/Additional
  File An d Print Service for Networks/Details/
  Print services for UNIX.

  This will crate a new printer port by the name of ``LPR Port''. Now
  install and configure a new printer with the driver for the printer
  connected to fli4l. Use

  Start/Settings/Printer

  double click ``New Printer''. In the introduction click ``proceed'', choose
  ``Local Printer'', deactivate ``Automatic Printer Recognition and Installation
  of Plug \& Play-Printers'' and click ``Proceed'' again. Under ``Choose Printer Port''
  activate ``Create a new port'' and as ``Type'' choose the den ``LPR Port''
  created above. After confirming these settings with clicking ``Proceed''
  insert into ``Name or Address of the LPD Server PD'' the IP address of the
  fli4l computer and write the name of it's printer queue into the field
  ``Name of the printer or printer queue on the server''. Use ``prx'' for local
  printers at the parallel ports, ``usbprx'' for local printers at the USB ports,
  ``reprx'' for remote printers and ``smbprx'' for SMB remote printers, where
  ``x'' stands for 1, 2, 3 for the first, second, third port a.s.o.
  In the next screen choose the manufacturer of the printer connected to fli4l
  on the left side and on the right side the according type, then click ``Proceed''
  again. Now specify a printer name for the printer in the field ``Printer Name''.
  Under Printer sharing choose ``Don't share this printer'', because it is already
  shared by fli4l. After clicking ``Next'' deny the question for printing a test page
  because not all settings are made by now and click ``Proceed''. Now a window
  appears showing a summary of the configuration made up to now. If everything
  was entered correctly press ``Finish''. After copying the printer driver a new
  icon will appear for your printer in the printer folder. Right click on it and
  choose ``Properties'' from the context menu. On the tab ``Port'' deactivate
  ``Activate Bidirectionale Printing''. On the tab ``Advanced'' click on ``Print
  processor'' and set it to ``WinPrint'', ``Standard data type'' to ``RAW'', now
  leave the dialogue with ``OK'' (for Windows NT 4.0 make an additional tick at
  ``always spool Raw-Data''). Back on the tab ``Advanced'' activate ``Use Spooler
  to speed up printing'' and ``Start printing after spooling of last page''.
  Untick ``activate advanced printing functions'' to disable its use. Now store
  all settings made by now by clicking ``Apply'' and leave the configuration
  window with ``OK'', for else Windows NT 4.0/2000 will not save the settings
  correctly.

\end{enumerate}
\subsubsection{If OPT\_SAMBA Is Activated}

  Setting up printing over Samba on a Windows client is different depending
  on whether \smalljump{sec:OPTNMBD}{\var{OPT\_NMBD}} is set to 'yes' or 'no'.

\begin{enumerate}
\item \textbf{OPT\_NMBD='no'}

  If \smalljump{sec:OPTNMBD}{\var{OPT\_NMBD}} is 'no' fli4l printers can not be
  browsed in the network neighborhood of a Windows computer. They may be installed
  using the UNC path though.

  For this it is necessary to create an entry in the hosts file of the client.
  You may find a sample file host.sam in Windows 95, Windows 98 and Windows Me
  in \verb+C:\WINDOWS+ in a standard Installation. In Windows NT 4.0/2000/XP
  the file can be found in the Windows directory under \verb+SYSTEM32\DRIVERS\ETC+.

  Add an entry for the router at the end of the file with a text editor.
  If in base.txt the IP address of the network interface for fli4l's internal
  net was set to

\begin{example}
\begin{verbatim}
    IP_NET_1='192.168.6.1/24'
\end{verbatim}
\end{example}

  and the name of the fli4l routers was set to

\begin{example}
\begin{verbatim}
  HOST_1='192.168.6.1 fli4l'
\end{verbatim}
\end{example}

  the following entry for the IP address and the name of the router should be added:

  192.168.6.1     fli4l

  Save the file again with the filename \textbf{hosts}. If using i.e. Notepad the
  file will get a suffix .txt added, which has to be removed to make this work.
  You may have to deactivate the explorer option ``Hide file name suffixes for
  known file extensions'' otherwise the suffix is not shown. Rename the file
  to ``hosts'' if necessary. After rebooting Windows the preparations are complete.

  When creating a new printer (Start/Settings/Printer/New Printer) choose ``Network
  printer''. For ``Network path or queue name'' specify \verb+\\fli4lNAME\PRINTERNAME+.
  Change ``fli4lNAME'' to the name of the fli4l router and PRINTERNAME to the name
  of the printer. PRINTERNAME is different depending on the port (parallel, USB, Remote).
  In general applies:
  ``prx'' for local printers on parallel ports, ``usbprx'' for
  local printers on USB ports, ``reprx'' for remote printers and ``smbreprx''
  for SMB remote printers, x standing for 1, 2, 3, hence for the first, second,
  third port a.s.o.
  To choose the first local printer at the parallel port \verb+\\fli4l\pr1+ would
  have to be set if your router really is named fli4l.
  Maybe you have set another Windows printer name by using \var{LPD\_\-PARPORT\_\-x\_\-SAMBA\_\-NAME},
  \var{LPD\_\-USBPORT\_\-x\_\-SAMBA\_\-NAME}, \var{LPD\_\-REMOTE\_\-x\_\-SAMBA\_\-NAME}
  and \var{LPD\_\-SMBREMOTE\_\-x\_\-SAMBA\_\-NAME}, obviously you have to use it then.
  For printers already installed you may use the tab ``Details'' in the printer properties
  to add a new port in analog to the procedere described above to assign it later with
  ``Port for Printing''. The following depends on the OS version used:

  For Windows 9x/Me:

  On the tab ``Details'' edit the ``Spool Settings'' by ticking ``Spool print jobs
  (faster printing)'' and ``Start printing after last page''.
  As data type choose ``RAW'' and tick ``Deactivated bidirectional printing''.

  For Windows NT 4.0/2000/XP:

  On the tab ``Ports'' deactivate ``Activate bidirectional printing''.
  On the tab ``Advanced'' click ``Print processor'' and set it to ``WinPrint'',
  under ``Standard data type'' to ``RAW'' leave the dialogue with
  ``OK'' (for Windows NT 4.0 make an additional tick at ``always spool Raw-Data'').
  Back on the tab ``Advanced'' activate ``Print to spooler to speed up
  printing'' and ``Start printing after spooling of last page''. Untick
  ``Activate advanced print functions'' to avoid using this setting. Now store
  all settings made by now by clicking ``Apply'' and leave the configuration
  window with ``OK'', for else Windows NT 4.0/2000 will not save the settings
  correctly.

\item \textbf{OPT\_NMBD='yes'}


  With \smalljump{sec:OPTNMBD}{\var{OPT\_NMBD}}='yes' fli4l's printers are
  displayed in the network neighborhood of Windows clients.

  When creating a new printer (Start/Settings/Printer/New Printer) choose ``Network
  printer''. For ``Network path or queue name'' you may use the ``Search'' button.
  You will find the fli4l by its name defined in base.txt (HOSTNAME='fli4l') and
  printer shares like ``prx'', ``usbprx'', ``reprx'' or ``smbreprx''.
  ``prx'' for local printers on parallel ports, ``usbprx'' for
  local printers on USB ports, ``reprx'' for remote printers and ``smbreprx''
  for SMB remote printers, x standing for 1, 2, 3, hence for the first, second,
  third port a.s.o.
  To choose the first local printer at the parallel port \verb+\\fli4l\pr1+ would
  have to be set if your router really is named fli4l.
  Maybe you have set another Windows printer name by using \var{LPD\_\-PARPORT\_\-x\_\-SAMBA\_\-NAME},
  \var{LPD\_\-USBPORT\_\-x\_\-SAMBA\_\-NAME}, \var{LPD\_\-REMOTE\_\-x\_\-SAMBA\_\-NAME}
  and \var{LPD\_\-SMBREMOTE\_\-x\_\-SAMBA\_\-NAME}, obviously you have to use it then.
  For printers already installed you may use the tab ``Details'' in the printer properties
  to add a new port in analog to the procedere described above to assign it later with
  ``Port for Printing''. The following depends on the OS version used:

  For Windows 9x/Me:

  On the tab ``Details'' edit the ``Spool Settings'' by ticking ``Spool print jobs
  (faster printing)'' and ``Start printing after last page''.
  As data type choose ``RAW'' and tick ``Deactivated bidirectional printing''.

  For Windows NT 4.0/2000/XP:

  On the tab ``Ports'' deactivate ``Activate bidirectional printing''.
  On the tab ``Advanced'' click ``Print processor'' and set it to ``WinPrint'',
  under ``Standard data type'' to ``RAW'' leave the dialogue with
  ``OK'' (for Windows NT 4.0 make an additional tick at ``always spool Raw-Data'').
  Back on the tab ``Advanced'' activate ``Print to spooler to speed up
  printing'' and ``Start printing after spooling of last page''. Untick
  ``Activate advanced print functions'' to avoid using this setting. Now store
  all settings made by now by clicking ``Apply'' and leave the configuration
  window with ``OK'', for else Windows NT 4.0/2000 will not save the settings
  correctly.

  Hint:

  The Network protocol TCP/IP must be installed and configured on the Windows machine.
  As the default Windows activates ``NETBIOS over TCP/IP'' which is the protocol Samba uses.

\end{enumerate}
\subsubsection{Setting Up A Linux LPR Client}

  On a Linux client fli4l's network printers may be inserted to the file
  /etc/printcap. For CUPS (common for newer distributions) see below.

  Example (name of the printer: ``drucker''):

\begin{example}
\begin{verbatim}
    drucker:\
            :lp=:\
            :rm=fli4l:\
            :rp=pr1:\
            :sd=/var/spool/lpd/drucker:\
            :sh:mx#0:
\end{verbatim}
\end{example}

  ``rm=fli4l'' sets the name of the fli4l router and has to be adapted
  to your settings. If the Linux printing queue is named different
  adapt ``drucker'' as well.

  The name of the remote queue in ``rp=pr1'' is as follows:

  \begin{description}
  \item[:rp=pr1:\ ] for fli4l's first parallel printer
  \item[:rp=pr2:\ ] for fli4l's second parallel printer

  \item[:rp=usbpr1:\ ] for fli4l's first USB printer
  \item[:rp=usbpr2:\ ] for fli4l's second USB printer

  \item[:rp=repr1:\ bzw. :rp=repr2:\ ] for configured remote-
  printer server ports

  \item[:rp=smbrepr1:\ bzw. :rp=smbrepr2:\ ] for configured SMB remote-
  printer server ports
  \end{description}

  \wichtig{After inserting this entry to etc/printcap
  the directory /var/spool/lpd/drucker has to be created
  with the mkdir command.}

  By executing ``lpr -P drucker FILENAME'' you may now print
  files from the Linux client on fli4l.

  Many newer distributions use alternative printing systems and own
  configuration tools, where the above described process will fail.
  For this reason Peter Schöne has contributed a description for the
  OpenSuSE distribution:

  Use YAST2 and choose Hardware -> Printer configuration. If local printers
  are already configured you may skip automatic detection. In the window
  ``Printer Setup'' choose the button ``Configure...'', and next the
  Option ``Show more connections...'' now click ``Proceed''.
  Now some different printer types will be shown. As we are using a LPD
  compatible package pick the first entry ``LPD-Prefilter- and -Forward-Queue''.
  After clicking ``Proceed'' we get to the configuration:
  Here you may search for the right router name with the button ``Lookup''-``LPD-Servers''
  and have it entered automatically else set the IP address of the router directly.
  The second box is for the name of the printer queue. Use one of the names
  explained above, i.e. ``pr1'', ``usbpr1'', ``repr1'', ``smbrepr1''
  etc. A click on the button ``Test remote LPD'' checks the settings for correctness.
  If the test is passsed click ``Proceed''. In the next window a name has to be set
  under which the printer is accessable from programs. The fields ``Description''
  and ``Location of the printer'' may be empty. Again, ``Proceed''...
  Now pick the printer type connected to the router and the right driver and
  store all your settings by clicking ``Save'' and a confirmation with ``Yes''.
  Now the printer has been set up completetly and should be accessable
  from most programs.

\subsubsection{Einrichtung eines Mac-Clients (For MacOSX 10.3 And Up)}

  Open the ``Printer assistent'' in ``System Settings'' and click ``Add''.
  Now choose ``TCP/IP - Printer'' and ``LPD/LPR'' as printer type. Specify the
  IP address of the router under ``Printer address''. Now add the name of
  the queue like described above, i.e. ``pr1'', ``usbpr1'', ``repr1'', ``smbrepr1''
  etc. Now choose the printer model from the list and click ``Add''.

