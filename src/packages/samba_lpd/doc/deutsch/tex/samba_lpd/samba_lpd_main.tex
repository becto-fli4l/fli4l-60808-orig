% Last Update: $Id$
\section{SAMBA\_LPD - Unterstützung von Windows Druck- und Dateidiensten in einem fli4l-Netzwerk}

Das Paket \var{SAMBA\_\-LPD} besteht aus den einzelnen OPT-Paketen

\begin{itemize}
\item \var{OPT\_\-SAMBA} - Samba als Datei- und Druckserver
\item \var{OPT\_\-SAMBATOOLS} - Tools rund um Samba wie
  z.B. Tools zum Senden von Nachrichten an Windows-Klienten, Einbinden
  von Netzwerkfreigaben in das Filesystem des Routers, ...
\item \var{OPT\_\-NMBD} - NETBIOS-Nameserver (Unterstützung von Netzwerkfreigaben)
\item \var{OPT\_\-LPD} - Druckerunterstützung
\end{itemize}

Obwohl die vier OPT-Pakete zu einem Installations-Paket zusammengefasst sind,
ist es möglich, sie einzeln zu aktivieren oder deaktivieren. Beim Deaktivieren
verliert man natürlich die Funktionalität des entsprechenden OPT-Paketes.
Eine Ausnahme ist \var{OPT\_\-NMBD}, welches nicht ohne \var{OPT\_\-SAMBA} läuft.



\wichtig{Bei Aktivierung von \var{OPT\_\-LPD} ist unbedingt \var{OPT\_\-LPDSRV}='no' zu setzen!}

Die einzelnen OPT-Pakete sind in den folgenden Abschnitten
beschrieben.

\marklabel{sec:OPTSAMBA}{\subsection {OPT\_SAMBA - Samba als Datei- und Druckerserver}}
\configlabel{OPT\_SAMBA}{OPTSAMBA}


    Mit \var{OPT\_\-SAMBA}='yes' können Windows-Clients direkt über das SMB-Protokoll
    drucken. Es ist dann keine weitere Client-Software (bis auf den
    Druckertreiber) notwendig.

    Wichtigste Voraussetzung zum Drucken über Samba ist aber immer die
    Einstellung von \smalljump{sec:OPTLPD}{\var{OPT\_LPD}}='yes' !

    Weiterhin ermöglicht dieses optionale Paket rudimentäre
    Fileserverfunktionen.
    Rudimentär deshalb, da fli4l keine Nutzerverwaltung hat und deshalb die
    Freigaben keinen Beschränkungen unterliegt. Wer einen ausgewachsenen
    File-Server benötigt, sollte dafür besser

        \altlink{http://www.eisfair.org/}

    verwenden. Die für dieses System intergrierte Samba-Version ist immer auf
    dem aktuellen Stand und hat weder Probleme mit grossen Partitionen noch mit
    dem Einsatz als Primary Domain Controller (PDC). Die Konfiguration lehnt
    sich an fli4l an und ist daher ähnlich einfach.
    Samba für fli4l ist vorrangig dafür gedacht, eine einfachere
    Druckerkonfiguration unter Windows zu ermöglichen.

    Es ist möglich, Samba ohne Nmbd (NetBios NameServer, siehe \smalljump{sec:OPTNMBD}{\var{OPT\_NMBD}})
    zu installieren, da beide zusammen sehr viel Platz verbrauchen.

    Deshalb wurden die an sich zusammengehörigen Programme auf einzelne OPT-
    Pakete verteilt - \var{OPT\_\-SAMBA} und \smalljump{sec:OPTNMBD}{\var{OPT\_NMBD}}.
    Auch wenn diese OPT-Pakete zusammen mit \smalljump{sec:OPTLPD}{\var{OPT\_LPD}} in einem
    grossen Paket stecken, können sie so einzeln aktiviert werden.
    Eine Ausnahme ist \var{OPT\_\-NMBD}, welches nicht ohne \var{OPT\_\-SAMBA} läuft.

    Wer auf die Anzeige in der Netzwerkumgebung unter Windows verzichten will,
    setzt \var{OPT\_\-SAMBA}='yes' und \smalljump{sec:OPTNMBD}{\var{OPT\_NMBD}}='no'.
    Trotzdem die Druckerfreigaben dann unter Windows nicht angezeigt werden, ist ein
    Zugriff darauf möglich, wenn man den genauen Pfad kennt. Dazu gibt es weiter
    unten unter ``Einrichtung eines Windows-SMB-Clients bei aktiviertem Samba
    (OPT\_SAMBA='yes')'' eine genauere Beschreibung.

    Wer auf die Anzeige in der Netzwerkumgebung nicht verzichten möchte, setzt
\begin{example}
\begin{verbatim}
    OPT_SAMBA='yes' und OPT_NMBD='yes'.
\end{verbatim}
\end{example}

    Zur Firewall: Ist \verb+PF_INPUT_ACCEPT_DEF='yes'+ (bzw.
    \verb+PF6_INPUT_ACCEPT_DEF='yes'+ für IPv6) gesetzt, dann werden in die
    INPUT-Kette Regeln eingebaut, welche die Samba-Ports (137--139 und 445)
    für den Zugriff aus den entsprechend konfigurierten Netzen (siehe hierzu die
    Variablen \jump{SAMBABINDALL}{\var{SAMBA\_BIND\_ALL}},
    \jump{SAMBABINDIPV4x}{\var{SAMBA\_BIND\_IPV4\_x}} und
    \jump{SAMBABINDIPV6x}{\var{SAMBA\_BIND\_IPV6\_x}}) öffnen. Bei
    \verb+PF_INPUT_ACCEPT_DEF='no'+ (bzw. \verb+PF6_INPUT_ACCEPT_DEF='no'+ für
    IPv6) müssen Sie sich selbständig darum kümmern, dass die Rechner aus Ihren
    Netzwerken auf den Samba-Server zugreifen können.

\begin{description}
\config{SAMBA\_WORKGROUP}{SAMBA\_WORKGROUP}{SAMBAWORKGROUP}

    Damit die Druckerfreigaben in der unter Windows definierten Arbeitsgruppe
    sichtbar werden, muss die Arbeitsgruppe für Samba mit der unter Windows
    definierten Arbeitsgruppe übereinstimmen. Wenn also unter Windows die
    Arbeitsgruppe ``workgroup'' heisst, muss diese Variable folgendermassen
    definiert werden:

\begin{example}
\begin{verbatim}
    SAMBA_WORKGROUP='workgroup'
\end{verbatim}
\end{example}

    Dies ist auch die Standard-Einstellung.

\config{SAMBA\_TRUSTED\_NETS}{SAMBA\_TRUSTED\_NETS}{SAMBATRUSTEDNETS}

    Von welchen Netzen aus ist der Zugriff auf Samba gestattet?

    Mit diesem Parameter wird eingestellt, welche Netze auf Samba zugreifen
    dürfen.
    Samba ermittelt und berücksichtigt beim Erstellen einer neuen
    Konfiguration die internen Netze aus den Grundeinstellungen von
    fli4l. Aus Sicherheitsgründen wird nur Rechnern aus diesen Netzen
    ein Zugriff gestattet.
    Sollen Rechner aus anderen Netzen zugreifen dürfen, müssen diese hier
    gesondert konfiguriert werden. Es reicht dabei aus, nur die
    zusätzlichen Netze anzugeben.
    Die Angabe hat dabei in der Form

\begin{example}
\begin{verbatim}
    NETZWERKNUMMER/ANZAHL-DER-GESETZTEN-BITS-IN-NETMASK
\end{verbatim}
\end{example}

     zu erfolgen, also z.B. fuer Netze der Form 192.168.x.0:

\begin{example}
\begin{verbatim}
     SAMBA_TRUSTED_NETS='192.168.6.0/24'
\end{verbatim}
\end{example}

    Standard-Einstellung: \var{SAMBA\_\-TRUSTED\_\-NETS}=''


\config{SAMBA\_LOG}{SAMBA\_LOG}{SAMBALOG} Logging von Fehlern in log.smb und log.nmb:
    'yes' oder 'no'

    Mit dieser Variable kann eingestellt werden, ob Aktionen in den Dateien
    log.smb und log.nmb aufgezeichnet werden sollen. In welches Verzeichnis
    diese Dateien geschrieben werden, wird mittels \smalljump{SAMBALOGDIR}{\var{SAMBA\_LOGDIR}}
    bzw. durch den gewählten Installationstyp bestimmt. Die Variable sollte nur
    für die Fehlersuche auf 'yes' gesetzt werden, da die Log-Dateien je nach
    Einstellung von \smalljump{SAMBALOGDIR}{\var{SAMBA\_LOGDIR}} oder gewählter
    Installationsvariante in die RAM-Disk geschrieben werden und deshalb die
    Gefahr besteht, dass diese irgendwann überläuft. \var{SAMBA\_\-LOG} gilt
    gleichermassen für \smalljump{sec:OPTSAMBA}{\var{OPT\_SAMBA}} und \smalljump{sec:OPTNMBD}{\var{OPT\_NMBD}},
    da \var{OPT\_\-NMBD} ohne \var{OPT\_\-SAMBA} nicht läuft.
    Wenn Ihr \var{SAMBA\_\-LOG}='no' setzt, solltet Ihr dringend nachlesen, was
    dabei bei der Variable \smalljump{SAMBALOGDIR}{\var{SAMBA\_LOGDIR}} zu beachten ist.

    Standard-Einstellung: \var{SAMBA\_\-LOG}='no'


\config{SAMBA\_LOGDIR}{SAMBA\_LOGDIR}{SAMBALOGDIR}

    Das Log-Verzeichnis für die Dateien log.smb und log.nmb

    Mit dieser Variable kann eingestellt werden, in welches Verzeichnis die
    Dateien log.smb und log.nmb geschrieben werden. Dabei kann die Variable
    entweder leer bleiben oder sie muss mit einem absoluten Pfad zu einem
    beschreibbaren Verzeichnis gefüllt werden. Ein absoluter Pfad beginnt immer
    mit einem '/'. Das Verzeichnis muss ausserdem bereits existieren.
    Bleibt die Variable leer, so entscheidet das Vorhandensein einer schreibbar
    unter /data eingehängten Partition darüber, wo die Logdateien abgelegt
    werden:

    Existiert keine schreibbar unter /data eingehängte Partition
    (typischerweise alle Installationstypen ausser B), so wird bei leerer
    Variable \var{SAMBA\_\-LOGDIR} nach /var/log (in die Ramdisk) geschrieben.

    Existiert eine schreibbar unter /data eingehängte Partition
    (typischerweise Installationstyp B), so wird bei leerer Variable
    \var{SAMBA\_\-LOGDIR} nach /data (die Datenpartition) geschrieben.

    Wenn die Variable gefüllt ist, dann werden die Dateien log.smb und log.nmb
    in das angegebene Verzeichnis geschrieben, wenn dieses Verzeichnis
    beschreibbar ist. Es macht keinen Sinn, hier eventuell eine nur lesbar
    eingebundene Opt-Partition anzugeben. Wenn die Logdateien nicht geschrieben
    werden können, startet Samba eventuell nicht. Ihr solltet also einen
    triftigen Grund haben und sehr genau nachdenken, bevor Ihr \var{SAMBA\_\-LOGDIR}
    füllt.

    Wenn Ihr \smalljump{SAMBALOG}{\var{SAMBA\_LOG}}='no' setzt, muss die Variable
    \var{SAMBA\_\-LOGDIR} entweder leer gelassen werden oder aber auf ein
    Verzeichnis verweisen, welches auf einem Linux-Dateisystem
    (minix, ext2, ext3) liegt, da bei \smalljump{SAMBALOG}{\var{SAMBA\_LOG}}='no' log.smb
    und log.nmb nach /dev/null gelinkt werden und diese symbolischen Links nur
    dann sauber funktionieren.
    Wenn man also definitiv keine Samba-Logdateien und auch keine Links zu
    solchen in /var/log haben möchte, setzt man beispielsweise

\begin{example}
\begin{verbatim}
     SAMBA_LOG='no'
     SAMBA_LOGDIR='/tmp'
\end{verbatim}
\end{example}

    In den meisten Fällen ist \var{SAMBA\_\-LOGDIR}='' die richtige Entscheidung,
    deshalb ist das auch die Standard-Einstellung.

    Standard-Einstellung: \var{SAMBA\_\-LOGDIR}=''

\config{SAMBA\_TDBPATH}{SAMBA\_TDBPATH}{SAMBATDBPATH}

    Diese Variable konfiguriert das Verzeichnis, in dem persistente Daten des
    Samba-Servers in so genannten TDB-Dateien gespeichert werden. In diesen
    Dateien wird u.\,a. vermerkt, welche Druckertreiber zum fli4l-Server
    hochgeladen wurden, siehe hierzu den Abschnitt
    \jump{sec:OPTSAMBAPOINTANDPRINT}{``Point'n'Print''} für Details. Auch die
    eigentlichen Druckertreiber werden unterhalb dieses Verzeichnisses
    abgelegt. Man kann diese Variable mit `auto' belegen; in diesem Fall wird
    vom fli4l ein passendes Verzeichnis auf einem persistenten Speichermedium
    erstellt und unterhalb von \texttt{/var/lib/persistent/samba/db}
    eingehängt.

    Standard-Einstellung: \verb+SAMBA_TDBPATH='auto'+

    Beispiel: \verb+SAMBA_TDBPATH='/data/samba/tdb'+

\config{SAMBA\_SPOOLPATH}{SAMBA\_SPOOLPATH}{SAMBASPOOLPATH}

    Diese Variable konfiguriert das so genannte Spool-Verzeichnis für
    eingehende Druckaufträge. Wird über das Samba-Protokoll gedruckt, landen
    die Druckdaten zuerst in dem hier eingestellten Verzeichnis bevor sie dann
    an den LPD-Druckserver weitergegeben werden. Man kann diese Variable mit
    `auto' belegen; in diesem Fall wird vom fli4l ein passendes Verzeichnis auf
    einem persistenten Speichermedium erstellt und unterhalb von
    \texttt{/var/lib/persistent/samba/spool} eingehängt.

    Zu beachten ist, dass dieses Verzeichnis beim Starten des fli4l-Routers
    geleert wird. Man sollte also kein Verzeichnis einstellen, das noch andere,
    wichtige Daten enthält!

    Standard-Einstellung: \verb+SAMBA_SPOOLPATH='auto'+

    Beispiel: \verb+SAMBA_SPOOLPATH='/data/samba/spool'+

\config{SAMBA\_BIND\_ALL}{SAMBA\_BIND\_ALL}{SAMBABINDALL}

    Mit \verb+SAMBA_BIND_ALL='yes'+ wird der Samba-Server an \emph{allen}
    verfügbaren lokalen Netzwerk-Schnittstellen auf Anfragen ``horchen''. Falls
    dies nicht gewünscht wird, muss \verb+SAMBA_BIND_ALL='no'+ gesetzt werden;
    zusätzlich muss man über die Arrays \var{SAMBA\_BIND\_IPV4\_\%} bzw.
    \var{SAMBA\_BIND\_IPV6\_\%} die Netze konfigurieren, auf denen der
    Samba-Server Anfragen beantworten soll.

    Standard-Einstellung: \verb+SAMBA_BIND_ALL='no'+

\configlabel{SAMBA\_BIND\_IPV4\_N}{SAMBABINDIPV4N}
\config{SAMBA\_BIND\_IPV4\_x}{SAMBA\_BIND\_IPV4\_x}{SAMBABINDIPV4x}

    Ist \verb+SAMBA_BIND_ALL='no'+, so werden über dieses Array die IPv4-Netze
    konfiguriert, innerhalb deren der Samba-Server Anfragen beantwortet.

    Beispiel:
\begin{example}
\begin{verbatim}
    SAMBA_BIND_IPV4_N='1'
    SAMBA_BIND_IPV4_1='IP_NET_1'
\end{verbatim}
\end{example}

    Standard-Einstellung: \verb+SAMBA_BIND_IPV4_N='0'+

\configlabel{SAMBA\_BIND\_IPV6\_N}{SAMBABINDIPV6N}
\config{SAMBA\_BIND\_IPV6\_x}{SAMBA\_BIND\_IPV6\_x}{SAMBABINDIPV6x}

    Ist \verb+SAMBA_BIND_ALL='no'+, so werden über dieses Array die IPv6-Netze
    konfiguriert, innerhalb deren der Samba-Server Anfragen beantwortet.

    Beispiel:
\begin{example}
\begin{verbatim}
    SAMBA_BIND_IPV6_N='1'
    SAMBA_BIND_IPV6_1='IPV6_NET_1'
\end{verbatim}
\end{example}

    Standard-Einstellung: \verb+SAMBA_BIND_IPV6_N='0'+

\config{LPD\_PARPORT\_x\_SAMBA\_NAME}{LPD\_PARPORT\_x\_SAMBA\_NAME}{LPDPARPORTxSAMBANAME}

    Hier kann der fli4l-Druckername des Druckers am x'ten parallelen Druckerport
    (\smalljump{LPDPARPORTxIO}{\var{LPD\_PARPORT\_x\_IO}}) in der Netzwerkumgebung
    eingestellt werden. Dazu muss selbstverständlich

\begin{example}
\begin{verbatim}
    OPT_NMBD='yes'
\end{verbatim}
\end{example}

    gesetzt sein, da sonst in der Netzwerkumgebung nichts angezeigt wird.
    Es dürfen hier Namen hinterlegt werden, die maximal 8 Zeichen lang sind und
    aus Buchstaben oder Zahlen bestehen. Umlaute und Sonderzeichen wie ä, ö, ü,
    ß, \_, @, usw. sind nicht erlaubt!

    Wenn die Variable leer bleibt, wird als Druckername der
    voreingestellte Name verwendet. Der voreingestellte Name für lokale Drucker
    an parallelen Ports ist prx, wobei das x für 1, 2, 3 usw., also für den
    ersten, zweiten, dritten Anschluss usw. steht.

    Standard-Einstellung: \var{LPD\_\-PARPORT\_\-1\_\-SAMBA\_\-NAME}=''


\config{LPD\_PARPORT\_x\_SAMBA\_NET}{LPD\_PARPORT\_x\_SAMBA\_NET}{LPDPARPORTxSAMBANET}

    Mit dieser Variable kann gesteuert werden, welche Rechner den lokalen
    Drucker am x'ten parallelen Port von fli4l nutzen dürfen. Man kann damit den
    Zugriff auf einzelne Rechner oder einzelne Subnetze beschränken. In der
    Standardeinstellung bleibt die Variable leer. Hiermit können alle Rechner
    des internen Netzwerkes (inclusive aller Subnetze) auf den x'ten Drucker
    drucken (siehe \smalljump{LPDPARPORTxIO}{\var{LPD\_PARPORT\_x\_IO}}). Bei zwei an fli4l
    angeschlossenen lokalen Druckern an parallelen Ports müssen
    \textbf{\var{LPD\_\-PARPORT\_\-1\_\-SAMBA\_\-NET}} und
    \textbf{\var{LPD\_\-PARPORT\_\-2\_\-SAMBA\_\-NET}} vorhanden sein.

    Die Variable kann so gefüllt werden:

\begin{itemize}
\item Eingabe der IP-Adressen in einer Zeile hintereinander durch Leerzeichen
      getrennt:

\begin{example}
\begin{verbatim}
    LPD_PARPORT_1_SAMBA_NET='192.168.6.2 192.168.0.1'
\end{verbatim}
\end{example}

    Bei zwei Netzen der Form 192.168.141.0/ 255.255.255.0 und
    192.168.142.0/ 255.255.255.0 und einem Drucker am ersten parallelen
    Anschluss:

\item  Eingabe eines IP-Bereiches ohne Hostanteil:

\begin{example}
\begin{verbatim}
    LPD_PARPORT_1_SAMBA_NET='192.168.141. 192.168.142.'
\end{verbatim}
\end{example}

    oder besser

\begin{example}
\begin{verbatim}
    LPD_PARPORT_1_SAMBA_NET='192.168.'
\end{verbatim}
\end{example}

    Hierbei ist unbedingt auf den Punkt am Ende zu achten!
\end{itemize}

    Standard-Einstellung: \var{LPD\_\-PARPORT\_\-1\_\-SAMBA\_\-NET}=''


\config{LPD\_USBPORT\_x\_SAMBA\_NAME}{LPD\_USBPORT\_x\_SAMBA\_NAME}{LPDUSBPORTxSAMBANAME}

    Hier kann der fli4l-Druckername des Druckers am x'ten USB-Druckerport in der
    Netzwerkumgebung eingestellt werden. Dazu muss selbstverständlich

\begin{example}
\begin{verbatim}
    OPT_NMBD='yes'
\end{verbatim}
\end{example}

    gesetzt sein, da sonst in der Netzwerkumgebung nichts angezeigt wird.
    Es dürfen hier Namen hinterlegt werden, die maximal 8 Zeichen lang sind und
    aus Buchstaben oder Zahlen bestehen. Umlaute und Sonderzeichen wie ä, ö, ü,
    ß, \_, @, usw. sind nicht erlaubt!

    Wenn die Variable leer bleibt, wird als Druckername der voreingestellte Name
    verwendet. Der voreingestellte Name für lokale Drucker an USB-Ports ist
    usbprx, wobei das x für 1, 2, 3 usw., also für den ersten, zweiten, dritten
    Anschluss usw. steht.

    Standard-Einstellung: \var{LPD\_\-USBPORT\_\-1\_\-SAMBA\_\-NAME}=''


\config{LPD\_USBPORT\_x\_SAMBA\_NET}{LPD\_USBPORT\_x\_SAMBA\_NET}{LPDUSBPORTxSAMBANET}

    Mit dieser Variable kann gesteuert werden, welche Rechner den lokalen
    Drucker am x'ten USB-Port von fli4l nutzen dürfen. Man kann damit den
    Zugriff auf einzelne Rechner oder einzelne Subnetze beschränken. In der
    Standardeinstellung bleibt die Variable leer. Hiermit können alle Rechner
    des internen Netzwerkes (inclusive aller Subnetze) auf den x'ten USB-Drucker
    drucken.
    Möchte man explizit Hosts oder Netze für die Ausgabe auf diese Drucker
    eintragen, gilt das unter \smalljump{LPDPARPORTxSAMBANET}{\var{LPD\_PARPORT\_x\_SAMBA\_NET}} beschriebene.

    Standard-Einstellung: \var{LPD\_\-USBPORT\_\-1\_\-SAMBA\_\-NET}=''


\config{LPD\_REMOTE\_x\_SAMBA\_NAME}{LPD\_REMOTE\_x\_SAMBA\_NAME}{LPDREMOTExSAMBANAME}

    Hier kann der fli4l-Druckername des Druckers an der x'ten
    \smalljump{LPDREMOTExIP}{\var{LPD\_REMOTE\_x\_IP}} in der Netzwerkumgebung
    eingestellt werden. Dazu muss selbstverständlich

\begin{example}
\begin{verbatim}
    OPT_NMBD='yes'
\end{verbatim}
\end{example}

    gesetzt sein, da sonst in der Netzwerkumgebung nichts angezeigt wird.
    Es dürfen hier Namen hinterlegt werden, die maximal 8 Zeichen lang sind und
    aus Buchstaben oder Zahlen bestehen. Umlaute und Sonderzeichen wie ä, ö, ü,
    ß, \_, @, usw. sind nicht erlaubt!

    Wenn die Variable leer bleibt, wird als Druckername der voreingestellte Name
    verwendet. Der voreingestellte Name für Remote-Drucker ist reprx, wobei das
    x für 1, 2, 3 usw., also für den ersten, zweiten, dritten Anschluss usw.
    steht.

    Standard-Einstellung: \var{LPD\_\-REMOTE\_\-1\_\-SAMBA\_\-NAME}=''


\config{LPD\_REMOTE\_x\_SAMBA\_NET}{LPD\_REMOTE\_x\_SAMBA\_NET}{LPDREMOTExSAMBANET}

    Mit dieser Variable kann gesteuert werden, welche Rechner den Remote-Drucker
    von fli4l nutzen dürfen. Man kann damit den Zugriff auf einzelne Rechner
    oder einzelne Subnetze beschränken. In der Standardeinstellung bleibt die
    Variable leer. Hiermit können alle Rechner des internen Netzwerkes
    (inclusive aller Subnetze) auf den x'ten Remote-Drucker an fli4l drucken
    (siehe \smalljump{LPDREMOTExIP}{\var{LPD\_REMOTE\_x\_IP}}).
    Möchte man explizit Hosts oder Netze für die Ausgabe auf diese Drucker
    eintragen, gilt das unter \smalljump{LPDPARPORTxSAMBANET}{\var{LPD\_PARPORT\_x\_SAMBA\_NET}} beschriebene.

    Standard-Einstellung: \var{LPD\_\-REMOTE\_\-1\_\-SAMBA\_\-NET}=''


\config{LPD\_SMBREMOTE\_x\_SAMBA\_NAME}{LPD\_SMBREMOTE\_x\_SAMBA\_NAME}{LPDSMBREMOTExSAMBANAME}

    Hier kann der fli4l-Druckername des Druckers am x'ten \smalljump{LPDSMBREMOTExSERVER}{\var{LPD\_SMBREMOTE\_x\_SERVER}}
    in der Netzwerkumgebung eingestellt werden. Dazu muss selbstverständlich

\begin{example}
\begin{verbatim}
    OPT_NMBD='yes'
\end{verbatim}
\end{example}

    gesetzt sein, da sonst in der Netzwerkumgebung nichts angezeigt wird.
    Es dürfen hier Namen hinterlegt werden, die maximal 8 Zeichen lang sind und
    aus Buchstaben oder Zahlen bestehen. Umlaute und Sonderzeichen wie ä, ö, ü,
    ß, \_, @, usw. sind nicht erlaubt!

    Wenn die Variable leer bleibt, wird als Druckername der voreingestellte Name
    verwendet. Der voreingestellte Name für SMB-Remote-Drucker ist smbreprx,
    wobei das x für 1, 2, 3 usw., also für den ersten, zweiten, dritten
    Anschluss usw. steht.

    Standard-Einstellung: \var{LPD\_\-SMBREMOTE\_\-1\_\-SAMBA\_\-NAME}=''


\config{LPD\_SMBREMOTE\_x\_SAMBA\_NET}{LPD\_SMBREMOTE\_x\_SAMBA\_NET}{LPDSMBREMOTExSAMBANET}

    Mit dieser Variable kann nun gesteuert werden, welche Rechner den
    SMB-Remote-Drucker von fli4l nutzen dürfen. Man kann damit den Zugriff auf
    einzelne Rechner oder einzelne Subnetze beschränken. In der
    Standardeinstellung bleibt die Variable leer. Hiermit können alle Rechner
    des internen Netzwerkes (inclusive aller Subnetze) auf den x'ten
    SMB-Remote-Drucker an fli4l drucken (siehe \smalljump{LPDSMBREMOTExSERVER}{\var{LPD\_SMBREMOTE\_x\_SERVER}}).
    Möchte man explizit Hosts oder Netze für die Ausgabe auf diese Drucker
    eintragen, gilt das unter \smalljump{LPDPARPORTxSAMBANET}{\var{LPD\_PARPORT\_x\_SAMBA\_NET}} beschriebene.

    Standard-Einstellung: \var{LPD\_\-SMBREMOTE\_\-1\_\-SAMBA\_\-NET}=''

\end{description}

\begin{description}
\config{SAMBA\_ADMINIP}{SAMBA\_ADMINIP}{SAMBAADMINIP}

        Wenn hier eine IP-Adresse oder ein Adressbereich aus dem lokalen Netz
        hinterlegt wird, haben die entsprechenden Rechner vollen Zugriff auf
        die fli4l-Ramdisk über das Netzwerk.
        Bei Verwendung von \var{OPT\_NMBD='yes'} können diese Rechner über die
        Netzwerkumgebung von Windows auf fli4l zugreifen.

        Hier ein Beispiel mit der IP-Adresse 192.168.6.2:

\begin{example}
\begin{verbatim}
    SAMBA_ADMINIP='192.168.6.2'
\end{verbatim}
\end{example}

        Will man mehreren Rechnern diesen Zugriff gestatten, hat man
        verschiedene Möglichkeiten:

        - Eingabe der IP-Adressen in einer Zeile hintereinander
          durch Leerzeichen getrennt:

          \var{SAMBA\_ADMINIP='192.168.6.2 192.168.6.3'}

        - Eingabe eines IP-Bereiches ohne Hostanteil:

          \var{SAMBA\_ADMINIP='192.168.'}

          Hierbei ist unbedingt auf den Punkt am Ende zu achten!

        Diese Variable sollte aus Sicherheitsgründen möglichst nur für die
        Fehlersuche gefüllt werden!

        In der Standardeinstellung ist die fli4l-Ramdisk über die
        Netzwerkumgebung nicht sichtbar und nicht im Zugriff.

        Standard-Einstellung: \var{SAMBA\_ADMINIP=''}
\end{description}

\begin{description}
\config{SAMBA\_SHARE\_N}{SAMBA\_SHARE\_N}{SAMBASHAREN}

    Erstellung einer bestimmten Anzahl von Freigaben: z.B. '2'

        Über \var{SAMBA\_SHARE\_N} wird die Anzahl der zu erstellenden bzw. zu
        benutzenden Freigaben eingestellt. Wenn die Freigaben nicht existieren,
        werden sie automatisch angelegt und wenn sie existieren, werden sie
        einfach benutzt.
        Das Erstellen von Freigaben ist normalerweise nur sinnvoll in Verbindung
        mit einem eingehängten Medium wie einer Festplatte, einem CD-ROM-Laufwerk
        oder einer Compact-Flash-Disk (siehe \var{OPT\_MOUNT}).

        Wird hier eine 2 eingegeben, müssen die folgenden Variablen

        \var{SAMBA\_SHARE\_NAME\_1}

        \var{SAMBA\_SHARE\_RW\_1}

        \var{SAMBA\_SHARE\_BROWSE\_1}

        \var{SAMBA\_SHARE\_PATH\_1}

        \var{SAMBA\_SHARE\_NET\_1}


        und


        \var{SAMBA\_SHARE\_NAME\_2}

        \var{SAMBA\_SHARE\_RW\_2}

        \var{SAMBA\_SHARE\_BROWSE\_2}

        \var{SAMBA\_SHARE\_PATH\_2}

        \var{SAMBA\_SHARE\_NET\_2}

        vorhanden und mit sinnvollen Werten gefüllt sein.

        Standard-Einstellung: \var{SAMBA\_SHARE\_N='0'}

\end{description}

\begin{description}
\config{SAMBA\_SHARE\_NAME\_x}{SAMBA\_SHARE\_NAME\_x}{SAMBASHARE_NAMEX}

        Mit \var{SAMBA\_SHARE\_NAME\_x} wird der Name der x'ten Freigabe eingestellt.
        Unter diesem Namen ist die Freigabe zu erreichen bzw. bei aktiviertem
        \var{OPT\_NMBD} in der Netzwerkumgebung von Windows-Rechnern sichtbar
        (siehe auch \var{SAMBA\_SHARE\_BROWSE\_x} weiter unten).

        Trotzdem unter Windows 12 Buchstaben und Umlaute für den Freigabenamen
        hinterlegbar sind, sollte man sich bei den Namen aus
        DOS-Kompatibilitätsgründen auf 8 Buchstaben ohne Umlaute beschränken,
        z.B.

\begin{example}
\begin{verbatim}
    SAMBA_SHARE_NAME_1='share1'
\end{verbatim}
\end{example}


        Außerdem sollten Freigabenamen im Netzwerk eindeutig sein, also nicht
        doppelt vorkommen.
        Dieser Name wird automatisch von fli4l an die Pfadangabe aus

        \var{SAMBA\_SHARE\_PATH\_x}

        angehängt. Im Pfad aus dieser Variablen wird also versucht, ein
        Verzeichnis mit dem Namen "share" zu erstellen, wenn ein solches noch
        nicht existiert.
        Es ist zwingend erforderlich, dass die Partition, die auf diesen Pfad
        eingehängt ist, schreibbar eingebunden ist. Wenn das nicht der Fall ist,
        gibt es beim Booten eine Fehlermeldung.
        Existiert das Verzeichnis schon, wird es nicht überschrieben, damit
        schon abgelegte Daten erhalten bleiben.

        Standard-Einstellung: \var{SAMBA\_SHARE\_NAME\_1='share1'}

\end{description}

\begin{description}
\config{SAMBA\_SHARE\_RW\_x}{SAMBA\_SHARE\_RW\_x}{SAMBASHARERWX}

    Soll die Freigabe beschreibbar sein: 'yes' oder 'no'

        Über \var{SAMBA\_SHARE\_RW\_x} wird eingestellt, ob die x'te Freigabe
        beschreibbar sein soll.

        Wird hier 'no' gewählt, können Dateien von dieser Freigabe gelesen aber
        nicht dorthin gespeichert werden. Das ist vor allem bei Dateien
        sinnvoll, die man anderen zur Verfügung stellen möchte aber dabei
        unbedingt verhindern will, dass diese Dateien verändert oder sogar
        gelöscht werden.

        Wird 'yes' gewählt, ist diese Freigabe für alle in der Variable

        \var{SAMBA\_SHARE\_NET\_x]}

        eingestellten IP-Adressen oder Netzwerke oder wenn diese leer ist, für
        alle Rechner des internen Netzwerkes (inclusive aller Subnetze)
        les- und beschreibbar.

        Standard-Einstellung: \var{SAMBA\_SHARE\_RW\_1='yes'}

\end{description}

\begin{description}
\config{SAMBA\_SHARE\_BROWSE\_x}{SAMBA\_SHARE\_BROWSE\_x}{SAMBASHAREBROWSEX} (benötigt \var{OPT\_NMBD='yes'})

    Soll die x'te Freigabe sichtbar sein: 'yes' oder 'no'

        Mit \var{SAMBA\_SHARE\_BROWSE\_x} wird eingestellt, ob die x'te Freigabe bei
        aktiviertem \var{OPT\_NMBD} in der Netzwerkumgebung sichtbar sein soll oder
        nicht.
        Möchte man verhindern, dass andere User in der Netzwerkumgebung die
        Freigabe sehen und dadurch darauf zugreifen können, setzt man

        \var{SAMBA\_SHARE\_BROWSE\_x='no'}

        Nutzer, die wissen, dass die Freigabe existiert, können trotzdem
        darauf zugreifen, indem sie z.B. unter Start/Ausführen

            $\backslash\backslash$fli4l$\backslash$sharename

        eingeben. Dabei ist \dq{}fli4l\dq{} durch den Namen des fli4l-Routers zu
        ersetzen - wenn er davon abweicht - und "sharename" mit dem Namen,
        den man in \var{SAMBA\_SHARE\_NAME\_x} eingetragen hat.

        Standard-Einstellung: \var{SAMBA\_SHARE\_BROWSE\_1='yes'}

\end{description}

\begin{description}
\config{SAMBA\_SHARE\_PATH\_x}{SAMBA\_SHARE\_PATH\_x}{SAMBASHAREPATHX}

    Der Pfad zur x'ten Freigabe

        Über \var{SAMBA\_SHARE\_PATH\_x} der Pfad wird der Pfad zur x'ten Freigabe
        eingestellt.

        Dazu ein Beispiel. Wenn man mittels


        \var{OPT\_MOUNT='yes'}

        \var{MOUNT\_N='1'}

        \var{MOUNT\_1\_DEV='hda4'}

        \var{MOUNT\_1\_POINT='/usr/local/data'}

        \var{MOUNT\_1\_FS='ext2'}

        \var{MOUNT\_1\_CHECK='yes'}

        \var{MOUNT\_1\_OPTION='rw'}


        die vierte primäre Partition der ersten Festplatte unter
        /usr/local/data in das Dateisystem eingehängt hat und eine
        einzelne Freigabe mit


        \var{SAMBA\_SHARE\_N='1'}

        \var{SAMBA\_SHARE\_NAME\_1='share1'}

        \var{SAMBA\_SHARE\_RW\_1='yes'}

        \var{SAMBA\_SHARE\_BROWSE='yes'}


        erstellt hat, kann man mit


        \var{SAMBA\_SHARE\_PATH\_1='/usr/local/data'}


        das Verzeichnis "share1" unter /usr/local/data erstellen
        und freigeben. Als Verzeichnisname wird der Inhalt der Variable

        \var{SAMBA\_SHARE\_NAME\_1}

        also in diesem Fall

        share1

        benutzt. Wenn das Verzeichnis nicht existiert, wird es automatisch
        angelegt und wenn es existiert, wird es einfach benutzt.
        Es gibt im Moment keine Möglichkeit, einmal erstellte Verzeichnisse über
        die samba\_lpd.txt zu löschen, da bei einer Fehleingabe die schon abgelegten
        Dateien gelöscht werden würden.
        Die im Verzeichnis liegenden Dateien können bei aktiviertem und
        konfigurierten

        \var{OPT\_NMBD}

        über den Explorer gelöscht werden, wenn die Freigabe in der Variablen

        \var{SAMBA\_SHARE\_RW\_1}

        schreibbar definiert wurde; das Verzeichnis selbst nur über die
        Kommandozeile.

        Standard-Einstellung: \var{SAMBA\_SHARE\_PATH\_1='/usr/local/data'}

\end{description}

\begin{description}
\config{SAMBA\_SHARE\_NET\_x}{SAMBA\_SHARE\_NET\_x}{SAMBASHARENETX}

        Mit dieser Variable kann gesteuert werden, welche Rechner die
        x'te Freigabe nutzen dürfen. Man kann damit den Zugriff auf
        einzelne Rechner oder einzelne Subnetze beschränken.
        In der Standardeinstellung bleibt die Variable leer. Hiermit können
        alle Rechner des internen Netzwerkes (inclusive aller Subnetze)
        auf die Freigabe zugreifen.

        Die Variable kann wie \var{SAMBA\_ADMINIP} gefüllt werden.

        - Eingabe der IP-Adressen in einer Zeile hintereinander
          durch Leerzeichen getrennt, z.B.:

          \var{SAMBA\_SHARE\_NET\_1='192.168.6.2 192.168.0.1'}

        Bei zwei Netzen der Form 192.168.141.0/255.255.255.0 und
        192.168.142.0/255.255.255.0

        - Eingabe eines IP-Bereiches ohne Hostanteil:

          \var{SAMBA\_SHARE\_NET\_1='192.168.141. 192.168.142.'}

          oder besser

          \var{SAMBA\_SHARE\_NET\_1='192.168.'}

          Hierbei ist unbedingt auf den Punkt am Ende zu achten!

        Standard-Einstellung: \var{SAMBA\_SHARE\_NET\_1=''}

\end{description}

\begin{description}
\config{SAMBA\_CDROM\_N}{SAMBA\_CDROM\_N}{SAMBACDROMN}

    Erstellung einer bestimmten Anzahl von Freigaben für CDROMs: z.B. '2'

        Über \var{SAMBA\_CDROM\_N} wird die Anzahl der zu erstellenden Freigaben
        für eingebaute CD-ROM-Laufwerke eingestellt.

        Diese, die folgenden Variablen und die dazugehörigen Erweiterungen
        des Scripts rc.samba wurden geschaffen, um die Freigabe von CDROMs
        etwas fehlertoleranter zu gestalten.
        Hatte man in Version 2.0pre2 versucht, ein CDROM-Laufwerk freizugeben,
        welches nicht eingehängt war oder hatte man einen falschen Pfad für
        die Freigabe angegeben, ging das natürlich schief.
        Das neue Script gibt im Zusammenhang mit den folgenden Variablen schon
        eingehängte CDROMs unter dem in \var{OPT\_MOUNT} definierten Mountpoint
        frei oder erzeugt für noch nicht eingehängte CDROMs einen Mountpoint,
        und gibt den anschliessend frei.

        Bei der letzten Variante wird das Laufwerk erst bei Bedarf
        (Zugriff über die Netzwerkumgebung) unter

        /mnt/cdromx

        eingehängt, wobei das x für das x'te CD-ROM steht. Man sollte deshalb
        darauf achten, dass eigene Mountpoints nicht mit diesen Angaben
        kollidieren.
        Wenn niemand mehr auf diese Freigabe zugreift, wird nach einiger Zeit
        das Laufwerk automatisch ausgehängt. Damit kann man die CDROM entnehmen,
        ohne sie manuell aushängen zu müssen, was sich vor allem bei
        CDROM-Servern mit mehreren Laufwerken anbietet, bei denen die CDs öfter
        mal gewechselt werden.


        Wird bei \var{SAMBA\_CDROM\_N} eine 2 eingegeben, müssen die folgenden Variablen

        \var{SAMBA\_CDROM\_DEV\_1}
        \var{SAMBA\_CDROM\_NET\_1}

        und

        \var{SAMBA\_CDROM\_DEV\_2}
        \var{SAMBA\_CDROM\_NET\_2}

        vorhanden und mit sinnvollen Werten gefüllt sein.

        Standard-Einstellung:     \var{SAMBA\_CDROM\_N='0'}

\end{description}

\begin{description}
\config{SAMBA\_CDROM\_DEV\_x}{SAMBA\_CDROM\_DEV\_x}{SAMBACDROMDEVX}

    Gerätename des CDROM-Laufwerks: z.B. 'sdc'

        Hier wird das Gerät angegeben, welches freigegeben werden soll.
        Die Konventionen für Gerätenamen können in der Dokumentation zu
        \var{OPT\_MOUNT} nachgelesen werden.

\end{description}

\begin{description}
\config{SAMBA\_CDROM\_NET\_x}{SAMBA\_CDROM\_NET\_x}{SAMBACDROMNETX}

        Mit dieser Variable kann gesteuert werden, welche Rechner das x'te
        CDROM-Laufwerk von fli4l nutzen dürfen. Man kann damit den Zugriff auf
        einzelne Rechner oder einzelne Subnetze beschränken.
        In der Standardeinstellung bleibt die Variable leer. Hiermit können
        alle Rechner des internen Netzwerkes ( inclusive aller Subnetze )
        auf das x'te CDROM-Laufwerk an fli4l zugreifen.
        Bei zwei an fli4l angeschlossenen CDROM-Laufwerken müssen

          \var{SAMBA\_CDROM\_NET\_1}

          und

          \var{SAMBA\_CDROM\_NET\_2}

        vorhanden sein.

        Die Variable kann wie \var{SAMBA\_ADMINIP} gefüllt werden.

        - Eingabe der IP-Adressen in einer Zeile hintereinander
          durch Leerzeichen getrennt:

          \var{SAMBA\_CDROM\_NET\_1='192.168.6.2 192.168.0.1'}

        Bei zwei Netzen der Form 192.168.141.0/255.255.255.0 und
        192.168.142.0/255.255.255.0 und einem CDROM-Laufwerk

        - Eingabe eines IP-Bereiches ohne Hostanteil:

          \var{SAMBA\_CDROM\_NET\_1='192.168.141. 192.168.142.'}

          oder besser

          \var{SAMBA\_CDROM\_NET\_1='192.168.'}

          Hierbei ist unbedingt auf den Punkt am Ende zu achten!

        Standard-Einstellung: \var{SAMBA\_CDROM\_NET\_1=''}

\end{description}

\marklabel{sec:OPTSAMBATOOLS}{\subsection {OPT\_SAMBATOOLS - Spezielle Tools für Samba }}
\configlabel{OPT\_SAMBATOOLS}{OPTSAMBATOOLS}
    Installiere zusätzliche Samba-Tools: 'yes' oder 'no'

    Hiermit werden zusätzliche Tools für Samba installiert.
    Da immer wieder gefragt wurde, ob es möglich ist, Nachrichten an
    Windows-Clients zu senden oder Freigaben von Windows-Rechnern auf fli4l
    einzubinden, habe ich mich entschlossen, die entsprechenden Tools
    bereitzustellen.

    Man kann bei Nutzung der Tools ohne weitergehende Kenntnisse einiges falsch
    machen. Wer nicht weiss, welche Gefahren zum Beispiel beim Einbinden von
    Freigaben auf fli4l drohen, sollte die Finger davon lassen. Ich habe mich
    bemüht, einige Fehlerquellen mit Skripten auszuschliessen, die das Ein-
    und Aushängen von Freigaben erledigen oder Nachrichten an einen, mehrere
    oder alle erreichbaren Windows-Clients senden.

    Weiteren Support wird es dafür von mir aber nicht geben, lest daher die
    Beschreibung hier genau!

    Es werden folgende zusätzliche Dateien bereitgestellt:

\begin{example}
\begin{verbatim}
    smbfs.o
    nls_iso8859-1.o
    nls_cp850.o
    nmblookup
    samba-netsend
    smbclient
    smbstatus
\end{verbatim}
\end{example}

    Die wichtigsten davon sind die Skripte, welche hier erklärt werden.

    samba-netsend

    Mit diesem Skript kann man interaktiv Nachrichten an SMB-Hosts versenden.
    Bei der Eingabe auf der Konsole erscheint folgende Ausgabe:

\begin{verbatim}
Send Message to SMB Hosts

To which SMB Hosts the message should be send?

Choice 1
--------
All SMB Hosts on configured Subnets on fli4l - type 'all'.

Choice 2
--------
fli4l Samba Clients with active connections - type 'active'.

Choice 3
--------
One ore more active SMB Hosts, type NETBIOS Names
separated with a blank, for instance 'client1 client2':
\end{verbatim}

Wie man hier sieht, gibt es im ersten Schritt 3 Auswahlmoeglickeiten:

\begin{enumerate}
\item
Das Versenden von Nachrichten an alle SMB-Rechner an den konfigurierten Netzen
des fli4l-Rechners. Da Samba bei Erstellung seiner Konfiguration auf die Datei

\begin{example}
\begin{verbatim}
   /config/base.txt
\end{verbatim}
\end{example}

zugreift, werden alle Netzwerkkarten, die hier konfiguriert sind, auch in die
Samba-Konfiguration eingetragen. Hieraus wird nun die Information gewonnen, in
welchen Netzen nach SMB-Rechnern gesucht werden muss und an alle diese Rechner
(bis auf den fli4l-Rechner selbst) die noch zu definierende Nachricht verschickt.
Dabei werden die ermittelten Broadcast-Adressen und die NETBIOS-Namen der
Rechner ausgegeben, an die die Nachricht verschickt wurde.

Um diese Option zu waehlen, ist

\begin{example}
\begin{verbatim}
   all
\end{verbatim}
\end{example}

einzugeben.

\item
Das Versenden von Nachrichten an alle fli4l-Samba-Clients mit aktiven
Verbindungen zu fli4l - damit sind nur die gemeint, die wirklich noch
Verbindungen zu fli4l-Samba offen halten.

Hierzu ist

\begin{example}
\begin{verbatim}
   active
\end{verbatim}
\end{example}

einzugeben.

\item
Das Versenden von Nachrichten an einen oder mehrere aktive SMB-Hosts.
Die Rechner sind mit ihren NETBIOS-Namen anzugeben. Mehrere Rechner muessen
durch ein Leerzeichen getrennt angegeben werden.


Wurden die erforderlichen Angaben gemacht, kommen wir zum 2. Schritt:

Send Message to SMB Hosts


Which Message should be send?
For instance 'fli4l-Samba-Server is going down in 3 Minutes ...':

Hier gibt man nun die Nachricht ein, die gesendet werden soll.
Diese Nachricht wird nur auf Clients mit aktiviertem Nachrichtendienst
ausgegeben. Der Nachrichtendienst ist auf Windows-NT, Windows-2000 und
Windows XP normalerweise aktiviert und muss anderenfalls
nachinstalliert/aktiviert werden.
Unter Windows-9x-Clients wie Windows 98 oder Windows ME muss dazu das
Programm WinPopUp laufen.


\end{enumerate}


    Standard-Einstellung: \var{OPT\_\-SAMBATOOLS}='no'


\marklabel{sec:OPTNMBD}{\subsection {OPT\_NMBD - NETBIOS Nameserver }}
\configlabel{OPT\_NMBD}{OPTNMBD}

    Das ist das Programm zur Anzeige von Freigaben in der Netzwerkumgebung von
    Windows (benötigt \smalljump{sec:OPTSAMBA}{\var{OPT\_SAMBA}}='yes'). Um die in \smalljump{sec:OPTSAMBA}{\var{OPT\_SAMBA}}
    freigegebene Ramdisk, die fli4l-Drucker oder eigene Freigaben sichtbar zu
    machen, ist \var{OPT\_\-NMBD}='yes' zu setzen.

    Der SMB-Nameserver benötigt auf dem fli4l-Medium weitere 100 KB. Wenn der
    Platz knapp wird, sollte man versuchen, ohne ihn
    auszukommen und die Drucker über direkte Eingabe des Netzwerkpfades
    einbinden, z.B. als \verb+\\fli4l\pr1+.

    Eine genauere Beschreibung des Zusammenspiels der beiden optionalen
    Programme findet sich unter \smalljump{sec:OPTSAMBA}{\var{OPT\_SAMBA}}.

    Standard-Einstellung: \var{OPT\_\-NMBD}='no'


\begin{description}
\config{NMBD\_MASTERBROWSER}{NMBD\_MASTERBROWSER}{NMBDMASTERBROWSER}

    Samba als Masterbrowser: 'yes' oder 'no'

    Da der fli4l-Rechner bei vielen durchgehend läuft, ist es mitunter
    sinnvoll, ihn auch als Masterbrowser einzusetzen. Ein Masterbrowser ist
    in Windows-Netzwerken der Rechner, der eine Liste aller verfügbaren SMB-
    Server (wozu alle Windows-Rechner mit aktivierter Datei- und
    Druckerfreigabe gehören) führt. Die Windows-Clients erfahren also vom
    Masterbrowser, welche Rechner mit aktivierter Datei- und
    Druckerfreigabe sich im Netzwerk befinden. In Netzwerken mit einem NT-
    Server sollte man lieber NT diese Aufgabe überlassen. In Netzwerken mit
    ein paar WIN9x-Rechnern kann fli4l diese Aufgabe problemlos übernehmen.

    Bei \var{NMBD\_\-MASTERBROWSER}='yes' gewinnt fli4l die Wahl zum Masterbrowser
    gegen alle anderen Windowsmaschinen.

    Standard-Einstellung: \var{NMBD\_\-MASTERBROWSER}='no'


\config{NMBD\_DOMAIN\_MASTERBROWSER}{NMBD\_DOMAIN\_MASTERBROWSER}{NMBDDOMAINMASTERBROWSER} (benötigt NMBD\_MASTERBROWSER='yes')

    Samba als Domänen-Masterbrowser: 'yes' oder 'no'

    Ich habe mich lange gesträubt, diese Variable mit in die Konfiguration
    aufzunehmen, da sie bei unsachgemässer Anwendung gefährlich ist.
    Aktiviert man diese Option nämlich in einem Netzwerk mit einem
    Domänen-Controller, der gleichzeitig immer auch Domänen-Masterbrowser ist,
    so ist das ein zuverlässiges Mittel, diesen Domänen-Controller zu sabotieren.
    In diesem Fall können die seltsamsten Effekte auftreten.
    Andererseits ist ein Domänen-Masterbrowser das sicherste Mittel, um
    netzübergreifendes Browsing zu realisieren (siehe cipe-HowTo).

    Wann es notwendig wird, einen Domänen-Masterbrowser zu konfigurieren, ist
    nicht mit ein oder zwei Worten zu erklären. Zum Glück haben sich aber schon
    andere die Mühe gemacht, das verständlich darzustellen:

        \altlink{http://samba.sernet.de/dokumentation/browsing-2.html}

    \wichtig{Es ist ein konfigurierter WINS-Server notwendig, der allen
    beteiligten Rechnern bekannt sein muss, damit das Ganze funktioniert!
    }

    Bei \var{NMBD\_\-DOMAIN\_\-MASTERBROWSER}='yes' versucht fli4l die Wahl zum
    Domänen-Masterbrowser zu gewinnen, was aber nur gelingt, wenn kein anderer
    Domänen-Masterbrowser existiert. Existiert ein anderer Domänen-Masterbrowser
    und diese Einstellung wurde gesetzt, werden definitiv Störungen im Netzwerk
    durch sogenannte Browse-Wars auftreten, bei denen beide Rechner versuchen,
    die Oberhand zu gewinnen. Wer also nicht genau weiss, ob in den beteiligten
    Netzen schon ein Domänen-Masterbrowser läuft, sollte die Finger von der
    Standardeinstellung lassen!

    Standard-Einstellung: \var{NMBD\_\-DOMAIN\_\-MASTERBROWSER}='no'


\config{NMBD\_WINSSERVER}{NMBD\_WINSSERVER}{NMBDWINSSERVER}

    Samba als WINS-Server: 'yes' oder 'no'

    Um NETBIOS-Namen in Windows-Netzen aufzulösen, gibt es zwei
    Möglichkeiten. Die erste benutzt eine statische Auflösung mit der Datei
    lmhosts und ist wie die DNS-Namensauflösung mit der Datei hosts schwer
    zu pflegen. Deshalb wurde von Microsoft WINS entwickelt:
    \textbf{W}indows \textbf{I}nternet \textbf{N}ame \textbf{S}ervice

    WINS hat den Vorteil, dass die NETBIOS-Namensauflösung per gerichteter
    Anfrage an einen WINS-Server passiert und nicht durch Broadcasts. Die
    WINS-Datenbank wird vom Server dynamisch aufgebaut, hat aber den
    Nachteil, dass der Server in den TCP/IP-Protokolleigenschaften auf jedem
    Client eingetragen werden muss. Samba hat diesen Server teilweise
    implementiert und damit steht er fli4l auch zur Verfügung.

    Um fli4l als WINS-Server zu betreiben, ist \smalljump{sec:OPTSAMBA}{\var{OPT\_SAMBA}}, \smalljump{sec:OPTNMBD}{\var{OPT\_NMBD}} und
    \var{NMBD\_\-WINSSERVER} auf yes zu setzen und in den TCP/IP-
    Protokolleigenschaften der Netzwerkkarte auf der Lasche WINS-Konfiguration
    ``WINS-Auflösung aktivieren'' auszuwählen.

    Unter WINS-Server Suchreihenfolge ist dabei die IP-Adresse des fli4-
    Rechners zu hinterlegen, welche mit ``Hinzufügen'' übernommen werden muss.

    Obwohl man hier nur die Wahl zwischen WINS ODER DHCP hat, entbindet die
    Angabe der IP-Adresse des fli4l-WINS-Servers nicht von einer korrekten
    TCP/IP-Konfiguration, entweder über Angabe der IP-Adresse jedes Clients
    oder über DHCP.

    In Netzwerken mit einem NT-Server, auf dem der WINS-Serverdienst
    aktiviert ist, sollte man lieber NT diese Aufgabe überlassen. Aber in
    Netzwerken mit ein paar WIN9x-Rechnern kann auch fli4l diese Aufgabe
    problemlos übernehmen.

\begin{example}
\begin{verbatim}
    NMBD_WINSSERVER='yes'
\end{verbatim}
\end{example}

    aktiviert diese Funktion.
    Bei installierten und aktiviertem \var{OPT\_\-DHCP} wird bei
\begin{example}
\begin{verbatim}
    NMBD_WINSSERVER='yes'
\end{verbatim}
\end{example}

    die IP-Adresse des fli4l-Rechners als IP-Adresse des Wins-Servers an die
    Clients übergeben.

    Standard-Einstellung: \var{NMBD\_\-WINSSERVER}='no'


\config{NMBD\_EXTWINSIP}{NMBD\_EXTWINSIP}{NMBDEXTWINSIP} (benötigt NMBD\_WINSSERVER='no')

    Die IP-Adresse des externen WINS-Servers für Samba

    Wenn man, wie oben bereits erwähnt, in Netzwerken mit einem NT-Server
    arbeitet, sollte man diesem die Aufgabe überlassen, die WINS-Datenbank
    zu verwalten. Dabei kann man fli4l als WINS-Client konfigurieren. Der
    fli4l-Rechner versucht dann, sich bei dem konfigurierten WINS-Server zu
    registrieren. Hierbei ist darauf zu achten, dass fli4l nicht
    gleichzeitig als Server und Client konfiguriert werden kann - die
    Optionen

\begin{example}
\begin{verbatim}
     NMBD_WINSSERVER='yes'
\end{verbatim}
\end{example}

    und

\begin{example}
\begin{verbatim}
    NMBD_EXTWINSIP='IP-Adresse'
\end{verbatim}
\end{example}

    schliessen einander also aus. Die Erstellung eines Bootmediums funktioniert
    sicherheitshalber bei einer solchen Konfiguration nicht.
    In diesem Modus arbeitet Samba ausserdem als WINS-Proxy. Das ist von
    Vorteil, wenn sich nicht nur WINS-Clients im Netzwerk befinden, der
    WINS-Server in einem anderen Netzwerk liegt und nicht per Broadcast
    erreichbar ist und die Nicht-WINS-Clients aber eine
    NETBIOS-Namensauflösung benötigen. Hierbei fängt der fli4l-Rechner
    Broadcasts der Nicht-WINS-Clients auf, fragt den eingetragenen
    WINS-Server ab und schickt die Antwort per Broadcast an den
    anfragenden Rechner.

    Wenn man den fli4l-Rechner als WINS-Client betreiben möchte, muss man
    ihm die IP-Adresse des externen WINS-Servers bekanntmachen, bei dem er
    sich registrieren soll. Voraussetzung dafür ist \smalljump{NMBDWINSSERVER}{\var{NMBD\_WINSSERVER}}='no'.

    Hier ein Beispiel mit der IP-Adresse 192.168.6.11:

\begin{example}
\begin{verbatim}
    NMBD_EXTWINSIP='192.168.6.11'
\end{verbatim}
\end{example}

    Bei installierten und aktiviertem \var{OPT\_\-DHCP} wird die hier konfigurierte IP-Adresse
    als IP-Adresse des Wins-Servers an die Clients übergeben.

    Standard-Einstellung: \var{NMBD\_\-EXTWINSIP}=''


\end{description}

\marklabel{sec:OPTLPD}{\subsection {OPT\_LPD - Druckerserver für lpr/lpd-Protokoll}}
\configlabel{OPT\_LPD}{OPTLPD}

    Mit \var{OPT\_\-LPD}='yes' kann man fli4l auch als Druckerserver verwenden. Dabei
    werden lpd und eine oder mehrere Druckerqueues (je nach Anzahl der an fli4l
    angeschlossenen Drucker) in der Ramdisk des root-Filesystems oder auf die
    Festplatte installiert.

    \wichtig{Damit das Drucken auch in einer Multi-User-Umgebung reibungslos
    funktioniert, wird der lpd-Spooler verwendet. Dabei werden die zu
    druckenden Daten in einem Spool-Verzeichnis zwischengelagert. Es wird
    für jeden Drucker ein gesondertes Spool-Verzeichnis erstellt. Diese
    Spoolverzeichnisse befinden sich im Hauptspeicher zusammen in der
    Ramdisk des root-Filesystems oder, wenn vorhanden, auf einer eingebauten
    Festplatte. Wenn beim Start von fli4l keine beschreibbare minix-, ext2- oder
    ext3-Partition gefunden wird, welche nach /data eingehängt wurde, benutzt
    jeder konfigurierte Drucker also die Ramdisk des root-Filesystems.
    Bei gleichzeitigem Druck über 3 an fli4l hängende Drucker
    werden die Spooldateien für alle 3 Drucker in dieser Ramdisk angelegt.
    Man sollte dabei beachten, dass selbst bei kleinen Textdateien schon
    grosse Druckjobs unter Windows entstehen und diese kurzzeitig
    programmbedingt 2 mal in der Ramdisk liegen, wenn über Samba gedruckt wird.
    Damit das Drucken beim Spoolen in der Ramdisk reibungslos funktioniert,
    sollte man also genug Speicher im fli4l-Rechner haben - je mehr, desto
    besser. Wer oft grosse Dokumente drucken muss oder nicht wenigstens 4 MB
    Speicher allein für das Drucken übrig hat, sollte unbedingt eine
    Festplatteninstallation benutzen, um die Funktion des Routers nicht durch
    eine überlaufende Ramdisk zu beeinträchtigen.
    Ja, Ihr habt richtig gelesen. Passt ein Druckjob nicht in die Ramdisk, kann
    das dazu führen, dass der Router nicht mehr routet...}

    \emph{Beim Spoolen auf die Festplatte wird die Verarbeitung der Druckjobs
    durch den auf der Festplatte verfügbaren Speicherplatz limitiert. Dazu
    muss aber bei der Festplatteninstallation die Variante B ausgewählt worden
    sein, bei der eine ext3-Datenpartition angelegt und diese nach /data
    eingehängt wird.}

    \emph{Sollten Probleme beim Druck grosser Dateien auftreten und benutzt Ihr keine
    Festplatteninstallation, so habt Ihr zu wenig Speicher im Router.}

    \emph{Daumenregel: fli4l als Standard-Router benötigt ca. 10 MB Speicherplatz
    (8 - 12, je nach Konfiguration). Bei 32 MB Arbeitsspeicher im Router steht
    der Rest für das Drucken zur Verfügung - also 32 - 10 = 22 MB.
    Beim Druck über Samba verringert sich der zur Verfügung stehende Platz noch
    einmal um die Hälfte, da die Jobs zur Druckzeit 2 mal in der Ramdisk liegen:
    Damit darf ein Job maximal 11 MB gross sein und bei 2 Jobs zur selben Zeit
    hört der Router auf zu routen...
    Ich empfehle für den Druck also dringend eine Festplatteninstallation.}

    Standard-Einstellung: \var{OPT\_\-LPD}='no'


\begin{description}

\config{LPD\_DEBUG}{LPD\_DEBUG}{LPDDEBUG}

    Diese Variable aktiviert oder deaktiviert die Protokollfunktion für den
    LPD-Druckserver. Es wird empfohlen, das Protokollierung nur dann zu
    aktivieren, wenn einer Fehlfunktion auf den Grund gegangen werden soll,
    weil das Protokoll recht viel Platz einnimmt.

    Die Variable kann entweder die Werte `yes' oder `no' annehmen oder einen
    Zahlenwert zwischen 1 und 5, wobei ein höherer Wert eine detailliertere
    Protokollierung bedeutet. Die Einstellung `yes' ist äquivalent zu dem Wert
    1.

    Standard-Einstellung: \verb+LPD_DEBUG='no'+

    Beispiel: \verb+LPD_DEBUG='2'+

\config{LPD\_DEBUG\_FILE}{LPD\_DEBUG\_FILE}{LPDDEBUGFILE}

    In dieser Variable wird der Pfad der Datei vermerkt, in der die Aktionen
    des LPD-Druckservers bei eingeschalteter Protokollfunktion (siehe die
    Beschreibung der Variable \jump{LPDDEBUG}{\var{LPD\_DEBUG}}) gespeichert
    werden sollen. Des Weiteren ist der Wert `auto' möglich, der den
    Standardort der Protokolldatei, \texttt{/var/log/lpd.log}, auswählt.

    Ist die Protokollierung mit Hilfe von \verb+LPD_DEBUG='no'+ deaktiviert,
    so ist der Inhalt dieser Variable nicht von Bedeutung.

    Standard-Einstellung: \verb+LPD_DEBUG_FILE='auto'+

    Beispiel: \verb+LPD_DEBUG_FILE='/data/log/lpd.log'+

\config{LPD\_SPOOLPATH}{LPD\_SPOOLPATH}{LPDSPOOLPATH}

    Diese Variable konfiguriert das so genannte Spool-Verzeichnis des
    LPD-Druckservers für eingehende Druckaufträge. Alle Druckaufträge landen
    in dem hier eingestellten Verzeichnis, bevor sie vom LPD-Druckserver an den
    Drucker weitergegeben werden. Man kann diese Variable mit `auto' belegen;
    in diesem Fall wird vom fli4l ein passendes Verzeichnis auf einem
    persistenten Speichermedium erstellt und unterhalb von
    \texttt{/var/lib/persistent/lpd/spool} eingehängt.

    Zu beachten ist, dass dieses Verzeichnis beim Starten des fli4l-Routers
    geleert wird. Man sollte also kein Verzeichnis einstellen, das noch andere,
    wichtige Daten enthält! Ferner ist zu beachten, dass der hier konfigurierte
    Pfad nicht derselbe sein darf wie der unter \var{SAMBA\_SPOOLPATH}
    eingestellte (mit der Ausnahme, dass beide Variablen den Wert `auto' haben
    können), weil die beiden Spool-Verzeichnisse unterschiedliche Aufgaben
    erfüllen und unterschiedliche Zugriffseinstellungen benötigen.

    Standard-Einstellung: \verb+LPD_SPOOLPATH='auto'+

    Beispiel: \verb+LPD_SPOOLPATH='/data/lpd/spool'+

\config{LPD\_NETWORK\_N}{LPD\_NETWORK\_N}{LPDNETWORKN}

    Diese Variable beinhaltet die Anzahl der Einträge im
    \var{LPD\_NETWORK\_x}-Array (siehe unten).

    Beispiel: \verb+LPD_NETWORK_N='1'+

\config{LPD\_NETWORK\_x}{LPD\_NETWORK\_x}{LPDNETWORKx}

    Jeder Eintrag in diesem Array spezifiziert eine Host- oder Netzwerkadresse,
    von der das Drucken über das LPD-Protokoll\footnote{siehe RFC 1179} erlaubt
    ist. Es sind direkte IPv4-Adressen wie \verb+192.168.1.0/24+, symbolische
    Adressen wie \verb+IP_NET_1+ und Host-Referenzen wie \verb+@peacock+
    möglich.

    Zu beachten ist, dass diese Einstellung \emph{nicht} nötig ist, wenn
    lediglich über den Samba-Server auf den Drucker zugegriffen werden soll!
    Diese Einstellung ist \emph{nur} relevant, wenn über das LPD-Protokoll
    gedruckt werden soll, und ist vornehmlich nur für Linux- und Mac-Rechner
    interessant. Für Windows-Rechner ist es bequemer, über Samba zu drucken,
    da die ``UNIX-Druckdienste'' i.\,d.\,R.\ separat installiert werden
    müssen.

    Beispiel:
\begin{example}
\begin{verbatim}
    LPD_NETWORK_1='IP_NET_1'
    LPD_NETWORK_2='192.168.1.0/24'
    LPD_NETWORK_3='@client'
\end{verbatim}
\end{example}

\config{OPT\_LPD\_PARPORT}{OPT\_LPD\_PARPORT}{OPTLPDPARPORT}

    Mit \var{OPT\_\-LPD\_\-PARPORT}='yes' wird bestimmt, dass lokale parallele
    Druckerports genutzt werden sollen. Wenn man nur USB-Drucker oder Remote-Drucker
    betreiben möchte, kann man die Variable auf ihrer Voreinstellung belassen:

    Standard-Einstellung: \var{OPT\_\-LPD\_\-PARPORT}='no'


\config{LPD\_PARPORT\_N}{LPD\_PARPORT\_N}{LPDPARPORTN} (benötigt OPT\_LPD\_PARPORT='yes')

    Über \var{LPD\_\-PARPORT\_\-N} wird die Anzahl der zu benutzenden lokalen
    parallelen Druckerports eingestellt. Bei einem Drucker an der ersten
    parallelen in der samba\_lpd.txt konfigurierten Schnittstelle ist

\begin{example}
\begin{verbatim}
    LPD_PARPORT_N='1'
\end{verbatim}
\end{example}

    einzutragen. Bei 2 Druckerports ist \var{LPD\_\-PARPORT\_\-N} zu inkrementieren,
    also

\begin{example}
\begin{verbatim}
    LPD_PARPORT_N='2'
\end{verbatim}
\end{example}

    Weiterhin müssen dann auch die korrespondierenden Einstellungen
        \emph{\var{LPD\_\-PARPORT\_\-1\_\-IO}},
        \emph{\var{LPD\_\-PARPORT\_\-1\_\-IRQ}}
        \emph{\var{LPD\_\-PARPORT\_\-1\_\-DMA}}
    und
        \emph{\var{LPD\_\-PARPORT\_\-2\_\-IO}},
        \emph{\var{LPD\_\-PARPORT\_\-2\_\-IRQ}}
        \emph{\var{LPD\_\-PARPORT\_\-2\_\-DMA}}

    und, wenn zusätzlich Samba genutzt wird, auch

        \emph{\var{LPD\_\-PARPORT\_\-1\_\-SAMBA\_\-NET}},
        \emph{\var{LPD\_\-PARPORT\_\-2\_\-SAMBA\_\-NET}},

    und, wenn Samba-Druckernamen vergeben werden sollen, auch

        \emph{\var{LPD\_\-PARPORT\_\-1\_\-SAMBA\_\-NAME}},
        \emph{\var{LPD\_\-PARPORT\_\-2\_\-SAMBA\_\-NAME}},

    vorhanden sein.

    Standard-Einstellung: \var{LPD\_\-PARPORT\_\-N}='1'


\config{LPD\_PARPORT\_x\_IO}{LPD\_PARPORT\_x\_IO}{LPDPARPORTxIO}

    Mit \var{LPD\_\-PARPORT\_\-x\_\-IO} wird der x'te lokale parallele Druckerport
    eingestellt.
    Bei 2 Druckern an 2 parallelen Schnittstellen von fli4l müssen 2 Einträge mit den
    möglichen Werten

\begin{itemize}
\item 0x3bc
\item 0x378 oder
\item 0x278
\end{itemize}

    existieren, also z.B.

\begin{example}
\begin{verbatim}
    LPD_PARPORT_1_IO='0x378'
\end{verbatim}
\end{example}

    und

\begin{example}
\begin{verbatim}
    LPD_PARPORT_2_IO='0x278'
\end{verbatim}
\end{example}

    \wichtig{
    Bisher wurden nur parallele Schnittstellen auf dem Mainboard
    oder auf ISA-Schnittstellenkarten mit den oben beschriebenen möglichen
    Werten unterstützt. PCI-Karten mit parallelen Schnittstellen konnten
    nicht verwendet werden.}

    \emph{Diese Version hier erlaubt auch die Konfiguration von parallelen
    Schnittstellen auf bestimmten PCI-Karten mit NETMOS-Chips.
    Hierzu muss man sich mittels}

\begin{example}
\begin{verbatim}
        cat /proc/pci
\end{verbatim}
\end{example}
    \emph{die erkannten PCI-Geräte anzeigen lassen. Hier sucht man das Gerät mit
    der passenden Vendor-ID und Device-ID und wählt als io-Adresse den oder
    die folgenden Einträge aus:}

        \begin{itemize}
        \item Nm9705CV  (Vendor id=9710, Device id=9705, Port1 1. Eintrag)
        \item Nm9735CV  (Vendor id=9710, Device id=9735, Port1 3. Eintrag)
        \item Nm9805CV  (Vendor id=9710, Device id=9805, Port1 1. Eintrag)
        \item Nm9715CV  (Vendor id=9710, Device id=9815, Port1 1. Eintrag, Port2 3. Eintrag)
        \item Nm9835CV  (Vendor id=9710, Device id=9835, Port1 3. Eintrag)
        \item Nm9755CV  (Vendor id=9710, Device id=9855, Port1 1. Eintrag, Port2 3. Eintrag)
        \end{itemize}

    \emph{Die Konfigurationsmöglichkeit wurde eingebaut, ohne entsprechende
    Hardware zum Testen zur Verfügung zu haben. Daher ist das als
    experimentelles Feature zu betrachten. Bei Fehlern bitte ausführliche
    Informationen in die Newsgroup posten!}

    Man sollte sich vor der Konfiguration unbedingt vergewissern, auf
    welche io-Adressen die eingebauten Schnittstellen eingestellt sind, da
    der Druck sonst nicht funktioniert. Die io-Adressen kann man entweder
    im BIOS seines Rechners einstellen oder sie sind bei sehr alten
    Rechnern nicht konfigurierbar, werden aber beim Booten angezeigt.
    Zusätzlich verbaute Ports lassen sich meist über Jumper auf der io-
    Karte einstellen und werden in der (hoffentlich noch vorhandenen
    Dokumentation) zur Einstellung der Druckerports beschrieben.

    Ausserdem ist darauf zu achten, dass bei \var{OPT\_\-LCD}='yes' die hier
    eingestellten Adressen in der samba\_lpd.txt nicht mit der dort
    eingestellten io-Adresse in \var{LCD\_\-ADDRESS} kollidieren.
    Dieser Konflikt verhindert die Erstellung eines Bootmediums!

    Standard-Einstellung: \var{LPD\_\-PARPORT\_\-1\_\-IO}='0x378'


\config{LPD\_PARPORT\_x\_IRQ}{LPD\_PARPORT\_x\_IRQ}{LPDPARPORTxIRQ}

        {Über \var{LPD\_\-PARPORT\_\-x\_\-IRQ} wird eingestellt, ob im
        Interruptbetrieb gedruckt werden soll, was den Prozessor entlastet. Dazu
        muss bei Schnittstellen auf dem Mainboard oder auf ISA-Karten aber im
        Rechnerbios oder per Jumperbelegung in jedem Fall der ECP/EPP-Modus
        konfiguriert werden. Der Interruptbetrieb wird so aktiviert:

\begin{example}
\begin{verbatim}
        LPD_PARPORT_1_IRQ='yes'
\end{verbatim}
\end{example}

        Will man diesen Modus nicht nutzen, so ist

\begin{example}
\begin{verbatim}
        LPD_PARPORT_1_IRQ='no'
\end{verbatim}
\end{example}

        zu setzen und bei Schnittstellen auf dem Mainboard oder auf ISA-Karten
        im Rechnerbios oder per Jumperbelegung in jedem Fall der Normal- oder
        SPP-Modus zu konfigurieren.
        Wenn etwas nicht funktioniert, sollte auf jeden Fall erst einmal mit

\begin{example}
\begin{verbatim}
        LPD_PARPORT_1_IRQ='no'
\end{verbatim}
\end{example}

        getestet werden!

        Standard-Einstellung: \var{LPD\_\-PARPORT\_\-1\_\-IRQ}='no'}


\config{LPD\_PARPORT\_x\_DMA}{LPD\_PARPORT\_x\_DMA}{LPDPARPORTxDMA}

        {Über \var{LPD\_\-PARPORT\_\-x\_\-DMA} wird eingestellt, ob im
        DMA-Betrieb gedruckt werden soll, was den Prozessor entlastet. Dazu
        muss bei Schnittstellen auf dem Mainboard oder auf ISA-Karten aber im
        Rechnerbios oder per Jumperbelegung in jedem Fall der ECP/EPP-Modus
        konfiguriert werden. Der Interruptbetrieb wird so aktiviert:

\begin{example}
\begin{verbatim}
        LPD_PARPORT_1_DMA='yes'
\end{verbatim}
\end{example}

        Voraussetzung dafür ist

\begin{example}
\begin{verbatim}
        LPD_PARPORT_1_IRQ='yes'
\end{verbatim}
\end{example}

        Will man diesen Modus nicht nutzen, so ist

\begin{example}
\begin{verbatim}
        LPD_PARPORT_1_DMA='no'
\end{verbatim}
\end{example}

        zu setzen und bei Schnittstellen auf dem Mainboard oder auf ISA-Karten
        im Rechnerbios oder per Jumperbelegung in jedem Fall der Normal- oder
        SPP-Modus zu konfigurieren.
        Wenn etwas nicht funktioniert, sollte auf jeden Fall erst einmal mit

\begin{example}
\begin{verbatim}
        LPD_PARPORT_1_DMA='no'
\end{verbatim}
\end{example}

        getestet werden!

        Standard-Einstellung: \var{LPD\_\-PARPORT\_\-1\_\-DMA}='no'}


\config{OPT\_LPD\_USBPORT}{OPT\_LPD\_USBPORT}{OPTLPDUSBPORT}

    Mit \var{OPT\_\-LPD\_\-USBPORT}='yes' wird bestimmt, dass lokale
    USB-Druckerports genutzt werden sollen.

    Ausserdem muss die generelle Unterstützung für USB-Drucker im separaten
    Paket \var{OPT\_\-USB} eingeschaltet werden. Je nach Treiber sieht das
    so aus:

\begin{example}
\begin{verbatim}
    OPT_USB='yes'
    USB_LOWLEVEL='uhci'
    USB_PRINTER='yes'
\end{verbatim}
\end{example}

        oder so:

\begin{example}
\begin{verbatim}
    OPT_USB='yes'
    USB_LOWLEVEL='usb-ohci'
    USB_PRINTER='yes'
\end{verbatim}
\end{example}

    \wichtig{Die Konfigurationsmöglichkeit für USB-Drucker wurde eingebaut,
    ohne entsprechende Hardware zum Testen zur Verfügung zu haben. Daher
    ist das als experimentelles Feature zu betrachten. Bei Fehlern bitte
    ausführliche Informationen in die Newsgroup posten!
    Viele USB-Drucker sind GDI-Drucker. GDI-Drucker können nicht angesprochen
    werden. Ich werde nur Fragen zu Problemen mit USB-Druckern beantworten,
    aus denen hervorgeht, dass Ihr ausgeschlossen habt, dass der betroffene
    Drucker ein GDI-Drucker ist!
    }

    Wenn man nur Drucker an parallelen Schnittstellen oder Remote-Drucker
    betreiben möchte, kann man die Variable auf ihrer Voreinstellung
    belassen:

    Standard-Einstellung: \var{OPT\_\-LPD\_\-USBPORT}='no'


\config{LPD\_USBPORT\_N}{LPD\_USBPORT\_N}{LPDUSBPORTN} (benötigt OPT\_LPD\_USBPORT='yes')

    Über \var{LPD\_\-USBPORT\_\-N} wird die Anzahl der zu benutzenden lokalen
    USB-Druckerports eingestellt. Bei einem Drucker an der ersten
    USB-Schnittstelle ist

\begin{example}
\begin{verbatim}
    LPD_USBPORT_N='1'
\end{verbatim}
\end{example}

    einzutragen. Bei 2 USB-Druckerports ist \var{LPD\_\-USBPORT\_\-N} zu inkrementieren,
    also

\begin{example}
\begin{verbatim}
    LPD_USBPORT_N='2'
\end{verbatim}
\end{example}

    Weiterhin müssen bei Nutzung von Samba die korrespondierenden
    Einstellungen \var{LPD\_\-USBPORT\_\-1\_\-SAMBA\_\-NET} und \var{LPD\_\-USBPORT\_\-2\_\-SAMBA\_\-NET}
    und, wenn Samba-Druckernamen vergeben werden sollen, auch
    \var{LPD\_\-USBPORT\_\-1\_\-SAMBA\_\-NAME} und \var{LPD\_\-USBPORT\_\-2\_\-SAMBA\_\-NAME}
    vorhanden sein.

    \wichtig{Wenn mehr als ein USB-Drucker verwendet wird, ist darauf zu
    achten, dass durch die Reihenfolge des Einschaltens der Drucker bestimmt
    wird, welcher Drucker der erste und welcher Drucker der zweite
    USB-Drucker wird.
    Der zweite USB-Drucker wird ausserdem automatisch zum ersten Drucker,
    wenn der erste USB-Drucker überhaupt nicht eingeschaltet wird.
    Wenn es sich um unterschiedliche Drucker-Modelle handelt,
    die unterschiedliche Treiber auf dem Client benötigen, kann es dadurch
    passieren, dass auf dem gewählten Drucker nur Zeichensalat ausgegeben
    wird, da der Druckjob in der Druckersprache des anderen Druckers
    formatiert wurde.
    }

    Standard-Einstellung: \var{LPD\_\-USBPORT\_\-N}='1'


\config{OPT\_LPD\_REMOTE}{OPT\_LPD\_REMOTE}{OPTLPDREMOTE}

    Mit \var{OPT\_\-LPD\_\-REMOTE}='yes' wird bestimmt, dass Remote-Drucker
    (entfernte Drucker) genutzt werden sollen. Wenn man nur Drucker an lokalen
    parallelen oder an lokalen USB-Schnittstellen betreiben möchte, kann man die
    Variable auf ihrer Voreinstellung belassen:

    Standard-Einstellung: \var{OPT\_\-LPD\_\-REMOTE}='no'


\config{LPD\_REMOTE\_N}{LPD\_REMOTE\_N}{LPDREMOTEN} (benötigt OPT\_LPD\_REMOTE='yes')

    Über \var{LPD\_REMOTE\_\-N} wird die Anzahl der zu konfigurierenden Remote-Drucker
    eingestellt. Damit ist es möglich, einen Druckauftrag von einem Client
    an fli4l zu schicken, der diesen Druckauftrag seinerseits an einen
    entfernten LPD-kompatiblen Printserver weiterleitet.

    Das Ganze funktioniert auch im Zusammenspiel mit Samba. Wenn  man einen
    Remote-Drucker über einen entfernten Printserver über fli4l ansprechen
    möchte, ist

\begin{example}
\begin{verbatim}
    LPD_REMOTE_N='1'
\end{verbatim}
\end{example}

    einzutragen. Bei 2 entfernten Printservern oder einem entfernten
    Printserver mit 2 Druckerwarteschlangen ist \var{LPD\_\-REMOTE\_\-N} zu
    inkrementieren, also

\begin{example}
\begin{verbatim}
    LPD_REMOTE_N='2'
\end{verbatim}
\end{example}

    Weiterhin müssen dann auch die korrespondierenden Einstellungen

\begin{itemize}
\item \var{LPD\_\-REMOTE\_\-1\_\-IP}
\item \var{LPD\_\-REMOTE\_\-1\_\-PORT}
\item \var{LPD\_\-REMOTE\_\-1\_\-QUEUENAME}
\item \var{LPD\_\-REMOTE\_\-2\_\-IP}
\item \var{LPD\_\-REMOTE\_\-2\_\-PORT}
\item \var{LPD\_\-REMOTE\_\-2\_\-QUEUENAME}
\end{itemize}

    und, wenn zusätzlich Samba genutzt wird, auch

\begin{itemize}
\item \var{LPD\_\-REMOTE\_\-1\_\-SAMBA\_\-NAME}
\item \var{LPD\_\-REMOTE\_\-1\_\-SAMBA\_\-NET}
\item \var{LPD\_\-REMOTE\_\-2\_\-SAMBA\_\-NAME}
\item \var{LPD\_\-REMOTE\_\-2\_\-SAMBA\_\-NET}
\end{itemize}

    vorhanden sein.

    Standard-Einstellung: \var{LPD\_\-REMOTE\_\-N}='0'


\config{LPD\_REMOTE\_x\_IP}{LPD\_REMOTE\_x\_IP}{LPDREMOTExIP}

    Mit \var{LPD\_\-REMOTE\_\-x\_\-IP} wird die IP des x'ten Remote-Printservers
    eingestellt.

    In der Standardeinstellung wird von einem zweiten fli4-Rechner
    ausgegangen, der unter der IP 192.168.6.99 erreichbar ist.

    Standard-Einstellung: \var{LPD\_\-REMOTE\_\-1\_\-IP}='192.168.6.99'


\config{LPD\_REMOTE\_x\_PORT}{LPD\_REMOTE\_x\_PORT}{LPDREMOTExPORT}

    Mit \var{LPD\_\-REMOTE\_\-x\_\-PORT} wird der Port des x'ten Remote-Druckers
    eingestellt.
    Diese Variable ist nur zu füllen, wenn auf Printserver gedruckt werden soll,
    die es erlauben, Daten per ftp oder netcat an sie zu schicken.
    Möchte man Printserver ansteuern, die das lpd-Protokoll verstehen, ist
    diese Variable leer zu lassen und statt dessen \smalljump{LPDREMOTExQUEUENAME}{\var{LPD\_REMOTE\_x\_QUEUENAME}}
    zu füllen.
    Es ist also ENTWEDER \var{LPD\_\-REMOTE\_\-x\_\-PORT} ODER
    \smalljump{LPDREMOTExQUEUENAME}{\var{LPD\_REMOTE\_x\_QUEUENAME}} zu füllen und niemals
    beides gleichzeitig! Eine von beiden Variablen muss aber gefüllt werden.

    Ob Euer Printserver zu der einen oder anderen Kategorie gehört, entnehmt Ihr
    bitte dem mitgelieferten Handbuch oder der Webseite des Herstellers.
    Eine unvollständige Übersicht findet Ihr unter

        \altlink{http://www.lprng.com/LPRng-Reference/LPRng-Reference.html\#AEN4990}

    Ich habe nicht die Zeit, Euch diese Informationen herauszusuchen,
    recherchiert also bitte selbst.

    In der Standardeinstellung wird beim dritten Remoteprinter repr3 von einem
    HP JetDirect Printserver (Interfacekarte) mit der IP 192.168.6.100
    ausgegangen, der über den Port 9100 erreicht werden kann (wie man unter dem
    obigen Link nachlesen kann, wäre der aber auch über den Warteschlangennamen
    raw zu erreichen...).

    Hier noch ein Hinweis:
    Ist der entsprechende Drucker zum Zeitpunkt der Druckjob-Erzeugung nicht
    erreichbar oder ist der Printserver, an dem der Drucker hängt, abgeschaltet,
    so läuft der lpd auf einen Timeout und der Job kann nicht verarbeitet
    werden. Dieser Job kann mit dem lprm-Kommando nicht gelöscht werden und
    verbleibt bis zu Neustart des Routers in der Warteschlange!

    Standard-Einstellung: \var{LPD\_\-REMOTE\_\-3\_\-PORT}='9100'


\config{LPD\_REMOTE\_x\_QUEUENAME}{LPD\_REMOTE\_x\_QUEUENAME}{LPDREMOTExQUEUENAME}

    Mit \var{LPD\_\-REMOTE\_\-x\_\-QUEUENAME} wird der Warteschlangenname des x'ten
    Remote-Druckers eingestellt.

    Diese Variable ist nur zu füllen, wenn auf Printserver gedruckt werden soll,
    die das lpd-Protokoll verstehen.
    Möchte man Printserver ansteuern, die es erlauben, Daten per ftp oder netcat
    an sie zu schicken, ist diese Variable leer zu lassen und statt dessen
    \smalljump{LPDREMOTExPORT}{\var{LPD\_REMOTE\_x\_PORT}} zu füllen.
    Es ist also ENTWEDER \var{LPD\_\-REMOTE\_\-x\_\-QUEUENAME} ODER
    \smalljump{LPDREMOTExPORT}{\var{LPD\_REMOTE\_x\_PORT}} zu füllen und niemals
    beides gleichzeitig! Eine von beiden Variablen muss aber gefüllt werden.

    Ob Euer Printserver zu der einen oder anderen Kategorie gehört, entnehmt Ihr
    bitte dem mitgelieferten Handbuch oder der Webseite des Herstellers.
    Eine unvollständige Übersicht findet Ihr unter

        \altlink{http://www.lprng.com/LPRng-Reference/LPRng-Reference.html\#AEN4990}

    Ich habe nicht die Zeit, Euch diese Informationen herauszusuchen,
    recherchiert also bitte selbst.

    In der Standardeinstellung wird von einem zweiten fli4-Rechner
    ausgegangen, dessen Warteschlangenname des ersten Druckers pr1 lautet.

    Standard-Einstellung: \var{LPD\_\-REMOTE\_\-1\_\-QUEUENAME}='pr1'


\config{OPT\_LPD\_SMBREMOTE}{OPT\_LPD\_SMBREMOTE}{OPTLPDSMBREMOTE}

    Mit \var{OPT\_\-LPD\_\-SMBREMOTE}='yes' wird bestimmt, dass
    SMB-Remote-Drucker (entfernte SMB-Drucker) genutzt werden sollen. Das können
    freigegebene Drucker an Windows- oder Samba-Rechnern sein.

    \wichtig{
    Die Konfiguration dieser Drucker macht nur einen Sinn, wenn diese entfernten
    SMB-Drucker zum Druckzeitpunkt eingeschaltet sind - ein Spoolen und
    Abspeichern der Druckjobs bis zur Wiederinbetriebname der entfernten Rechner
    mit diesen Druckfreigaben ist bedingt durch die Realisierung als
    Vorfilterskript für den lpd leider nicht möglich.
    }

    Wenn man nur Drucker an lokalen parallelen, lokalen USB-Schnittstellen oder
    an Remote-LPD-Printern betreiben möchte, kann man die Variable auf ihrer
    Voreinstellung belassen:

    Standard-Einstellung: \var{OPT\_\-LPD\_\-SMBREMOTE}='no'


\config{LPD\_SMBREMOTE\_DEBUGLEVEL}{LPD\_SMBREMOTE\_DEBUGLEVEL}{LPDSMBREMOTEDEBUGLEVEL} (benötigt OPT\_LPD\_SMBREMOTE='yes')

    Über \var{LPD\_SMBREMOTE\_\-DEBUGLEVEL} wird die Anzahl von Debugmeldungen
    eingestellt, die beim Druck auf SMB-Remote-Printer geloggt werden sollen.
    Es wird immer nur der Druck einer Datei geloggt, die Logdatei
    /tmp/smb-print.log wird bei jedem Job neu überschrieben.
    Bei \var{LPD\_SMBREMOTE\_DEBUGLEVEL}='0' wird nicht geloggt. Wenn Probleme
    auftreten, sollte ein höherer Wert eingestellt werden, um anhand der
    Meldungen in /tmp/smb-print.log den Fehler eingrenzen zu können.

    Standard-Einstellung: \var{LPD\_\-SMBREMOTE\_\-DEBUGLEVEL}='0'


\config{LPD\_SMBREMOTE\_N}{LPD\_SMBREMOTE\_N}{LPDSMBREMOTEN} (benötigt OPT\_LPD\_SMBREMOTE='yes')

    Über \var{LPD\_SMBREMOTE\_\-N} wird die Anzahl der zu konfigurierenden
    SMB-Remote-Drucker eingestellt. Damit ist es möglich, einen Druckauftrag von
    einem Client an fli4l zu schicken, der diesen Druckauftrag seinerseits an
    eine entfernten SMB-Druckerfreigabe weiterleitet.

    Das Ganze funktioniert auch im Zusammenspiel mit Samba. Wenn man einen
    SMB-Remote-Drucker über einen entfernten Windows- oder Samba-Rechner über
    fli4l ansprechen möchte, ist

\begin{example}
\begin{verbatim}
    LPD_SMBREMOTE_N='1'
\end{verbatim}
\end{example}

    einzutragen. Bei 2 entfernten Windows- oder Samba-Rechnern oder einem
    entfernten Windows- oder Samba-Rechner mit 2 Druckerfreigaben ist \var{LPD\_\-SMBREMOTE\_\-N}
    zu inkrementieren, also

\begin{example}
\begin{verbatim}
    LPD_SMBREMOTE_N='2'
\end{verbatim}
\end{example}

    Weiterhin müssen dann auch die korrespondierenden Einstellungen

\begin{itemize}
\item \var{LPD\_\-SMBREMOTE\_\-1\_\-SERVER}
\item \var{LPD\_\-SMBREMOTE\_\-1\_\-SERVICE}
\item \var{LPD\_\-SMBREMOTE\_\-1\_\-USER}
\item \var{LPD\_\-SMBREMOTE\_\-1\_\-PASSWORD}
\item \var{LPD\_\-SMBREMOTE\_\-1\_\-IP}
\item \var{LPD\_\-SMBREMOTE\_\-2\_\-SERVER}
\item \var{LPD\_\-SMBREMOTE\_\-2\_\-SERVICE}
\item \var{LPD\_\-SMBREMOTE\_\-2\_\-USER}
\item \var{LPD\_\-SMBREMOTE\_\-2\_\-PASSWORD}
\item \var{LPD\_\-SMBREMOTE\_\-2\_\-IP}
\end{itemize}

    und, wenn zusätzlich Samba genutzt wird, auch

\begin{itemize}
\item \var{LPD\_\-SMBREMOTE\_\-1\_\-SAMBA\_\-NAME}
\item \var{LPD\_\-SMBREMOTE\_\-1\_\-SAMBA\_\-NET}
\item \var{LPD\_\-SMBREMOTE\_\-2\_\-SAMBA\_\-NAME}
\item \var{LPD\_\-SMBREMOTE\_\-2\_\-SAMBA\_\-NET}
\end{itemize}

    vorhanden und korrekt gefüllt sein.

    Standard-Einstellung: \var{LPD\_\-SMBREMOTE\_\-N}='0'


\config{LPD\_SMBREMOTE\_x\_SERVER}{LPD\_SMBREMOTE\_x\_SERVER}{LPDSMBREMOTExSERVER}

    Mit \var{LPD\_\-SMBREMOTE\_\-x\_\-SERVER} wird der NETBIOS-Name des Rechners
    mit der x'ten Druckerfreigabe eingestellt. Dieser Name ist notwendig, da per
    smbclient gedruckt wird.

    In der Standardeinstellung wird von einem NT-Rechner mit dem NETBIOS-Namen
    ``ente'' ausgegangen.

    Standard-Einstellung: \var{LPD\_\-SMBREMOTE\_\-1\_\-SERVER}='ente'


\config{LPD\_SMBREMOTE\_x\_SERVICE}{LPD\_SMBREMOTE\_x\_SERVICE}{LPDSMBREMOTExSERVICE}

    Mit \var{LPD\_\-SMBREMOTE\_\-x\_\-SERVICE} wird der Name der x'ten
    Druckerfreigabe des SMB-Remote-Druckers eingestellt.
    In der Standardeinstellung wird von dem Drucker-Freigabenamen ``pr2''
    ausgegangen.

    Standard-Einstellung: \var{LPD\_\-SMBREMOTE\_\-1\_\-SERVICE}='pr2'


\config{LPD\_SMBREMOTE\_x\_USER}{LPD\_SMBREMOTE\_x\_USER}{LPDSMBREMOTExUSER}

    Mit \var{LPD\_\-SMBREMOTE\_\-x\_\-USER} wird der Username mit Zugriff auf
    die x'te Druckerfreigabe eingestellt.
    In der Standardeinstellung wird von dem Usernamen ``king'' ausgegangen.

    Standard-Einstellung: \var{LPD\_\-SMBREMOTE\_\-1\_\-USER}='king'


\config{LPD\_SMBREMOTE\_x\_PASSWORD}{LPD\_SMBREMOTE\_x\_PASSWORD}{LPDSMBREMOTExPASSWORD}

    Mit \var{LPD\_\-SMBREMOTE\_\-x\_\-PASSWORD} wird das Passwort des x'ten
    Users mit Zugriff auf die Druckerfreigabe eingestellt.
    In der Standardeinstellung wird von dem Passwort ``kong'' ausgegangen.

    Standard-Einstellung: \var{LPD\_\-SMBREMOTE\_\-1\_\-PASSWORD}='kong'


\config{LPD\_SMBREMOTE\_x\_IP}{LPD\_SMBREMOTE\_x\_IP}{LPDSMBREMOTExIP}

    Mit \var{LPD\_\-SMBREMOTE\_\-x\_\-IP} wird die IP des Windows- oder
    Samba-Rechners mit der x'ten Druckerfreigabe eingestellt.
    In der Standardeinstellung wird von einem NT-Rechner ausgegangen, der unter
    der IP 192.168.0.6 erreichbar ist.

    Standard-Einstellung: \var{LPD\_\-SMBREMOTE\_\-1\_\-IP}='192.168.0.6'


\end{description}

\marklabel{sec:OPTSAMBAPOINTANDPRINT}{
\subsection{\var{OPT\_SAMBA\_POINT\_AND\_PRINT} -- Serverseitige Verwaltung von Windows-Druckertreibern}}

Point'n'Print ist eine Windows-Technologie zur serverseitigen Verwaltung von
Druckertreibern. Die Idee ist einfach: Ist ein Windows-Server gleichzeitig ein
Druckerserver, bietet also Druckdienste an, dann ist es auch vernünftig, dass
er auch die passenden Druckertreiber für den bzw. die angeschlossenen Drucker
anbietet. Denn es ist unnötig aufwändig, auf jedem einzelnen Windows-Client die
passenden Druckertreiber zu installieren. Point'n'Print ermöglicht genau dies:
Zuerst lädt ein Administrator beliebig viele Treiber für beliebig viele Drucker
und Rechnerarchitekturen auf den Druckerserver hoch. Ein normaler Benutzer kann
sich später bei Bedarf mit einem Drucker auf diesem Server verbinden (also
analog zu Dateifreigaben eine Druckerfreigabe nutzen), und der Windows-Client
holt sich automatisch die passenden Druckertreiber vom Druckerserver und
installiert diese lokal auf dem Client. Somit ist eine Nutzung des im Netzwerk
freigegebenen Druckers schnell und ohne viel Installationsaufwand möglich.

Im \jump{sec:OPTSAMBAPOINTANDPRINT:XP}{Anhang} wird erläutert, wie man die
Point'n'Print-Konfiguration mit Hilfe eines Windows XP-Clients einrichtet.

\begin{description}

\config{OPT\_SAMBA\_POINT\_AND\_PRINT}{OPT\_SAMBA\_POINT\_AND\_PRINT}{OPTSAMBAPOINTANDPRINT}

Mit dieser Variable wird die Point'n'Print-Funktio\-nalität aktiviert. Eine
Aktivierung erfordert \verb+OPT_SAMBA='yes'+ und \verb+OPT_LPD='yes'+.

Standard-Einstellung: \verb+OPT_SAMBA_POINT_AND_PRINT='no'+

Beispiel: \verb+OPT_SAMBA_POINT_AND_PRINT='yes'+

\config{SAMBA\_PRINT\_ADMIN\_NAME}{SAMBA\_PRINT\_ADMIN\_NAME}{SAMBAPRINTADMINNAME}

Damit nicht jeder Benutzer beliebig Druckertreiber installieren oder
deinstallieren kann (gerade letzteres ist i.\,d.\,R.\ unerwünscht), ist dies
nur einem \emph{Drucker-Administrator} gestattet. Der Name des Windows-Accounts,
der eine solche Funktion übernimmt, ist hier vermerkt.

Beispiel: \verb+SAMBA_PRINT_ADMIN_NAME='pradmin'+

\config{SAMBA\_PRINT\_ADMIN\_PASSWORD}{SAMBA\_PRINT\_ADMIN\_PASSWORD}{SAMBAPRINTADMINPASSWORD}

Hier ist das Passwort des Windows-Accounts vermerkt, der die Funktion des
Drucker-Administrators übernimmt.

Beispiel: \verb+SAMBA_PRINT_ADMIN_PASSWORD='geheim'+

\end{description}

\marklabel{sec:DRUCKEREINRICHTUNG}{\subsection {Druckereinrichtung auf den Clients}}

    Die Einrichtung der fli4l-Drucker auf den Clients richtet sich erheblich
    danach, ob \smalljump{sec:OPTSAMBA}{\var{OPT\_SAMBA}} aktiviert wurde oder nicht
    und ob bei aktiviertem \smalljump{sec:OPTSAMBA}{\var{OPT\_SAMBA}} auch noch \smalljump{sec:OPTNMBD}{\var{OPT\_NMBD}}
    aktiviert wurde oder nicht.
    Ausserdem gibt es auch noch Unterschiede der Client-Betriebssysteme und
    ihrer Möglichkeiten zu beachten.
    Daher gibt es für jede Konfigurationsmöglichkeit einen eigenen Abschnitt.



\subsubsection{Einrichtung OPT\_SAMBA deaktiviert}
\begin{enumerate}
\item \textbf{Einrichtung NT}

    Wenn Samba nicht genutzt wird, ist unter Windows NT 4.0/2000/XP für den
    Zugriff auf den LPD von fli4l die Installation der Druckdienste für Unix
    notwendig, da beim Drucken über den Standard-TCP/IP-Port von Windows
    ungeeignete Ports benutzt werden.

    Die Druckdienste für Unix können über

    Start/Einstellungen/Systemsteuerung/Software/Windows-Komponenten
    hinzufügen/Weitere Datei- und Druckdienste für das Netzwerk/Details/
    Druckdienste für UNIX

    hinzugefügt werden.

    Damit ist ein neuer Druckerport mit dem Namen ``LPR Port'' verfügbar. Nun
    richtet man mit dem Druckerassistenten unter Windows NT 4.0/2000/XP einen
    neuen Drucker mit dem Treiber des an fli4l hängenden Druckers ein. Dazu
    geht man auf

    Start/Einstellungen/Drucker

    und macht einen Doppelklick auf ``Neuer Drucker''. Hier bestätigt man die
    Einleitung mit ``Weiter'', wählt ``Lokaler Drucker'' aus, deaktiviert
    ``Automatische Druckererkennung und Installation von Plug \& Play-Druckern''
    und bestätigt mit ``Weiter''. Unter ``Druckeranschluss auswählen'' aktiviert man
    ``Einen neuen Anschluss erstellen'' und wählt unter ``Typ'' den oben erstellten
    ``LPR Port''. Nachdem man diese Einstellungen mit dem Drücken von ``Weiter''
    bestätigt hat, trägt man in das Feld ``Name oder Adresse des Servers für
    LPD'' die richtige IP-Adresse des fli4l-Rechners ein und schreibt in das
    Feld ``Name des Druckers oder der Druckerwarteschlange auf dem Server'' den
    Namen der richtigen Druckerqueue. Dabei ist ``prx'' für lokale Drucker an
    parallelen Ports, ``usbprx'' für lokale Drucker an USB-Ports, ``reprx'' für
    Remote-Drucker und ``smbprx'' für SMB-Remote-Drucker anzugeben, wobei das
    ``x'' für 1, 2, 3 für den ersten, zweiten, dritten Anschluss usw. steht.
    Auf dem nächsten Konfigurationsbildschirm wählt man auf der
    linken Seite den Hersteller des an fli4l hängenden Druckers und auf der
    rechten Seite den entsprechenden Typ aus und bestätigt abermals mit
    ``Weiter''. Im Feld ``Druckername'' kann man nun einen Namen für den Drucker
    festlegen. Unter Druckerfreigabe wählt man ``Diesen Drucker nicht
    freigeben'', da der Drucker am fli4l-Rechner freigegeben ist. Nach dem Klick
    auf ``Weiter'' verneint man die Frage, ob eine Testseite gedruckt werden
    soll, da noch nicht alle Einstellungen vorgenommen worden sind und
    bestätigt wieder mit ``Weiter''. Nun erscheint ein Fenster mit der
    Zusammenfassung der bisherigen Konfiguration. Wenn alles korrekt eingegeben
    wurde, drückt man ``Fertig stellen''. Nach dem Kopieren des Druckertreibers
    erscheint ein neues Icon für diesen Drucker im Druckerordner. Das Icon für
    den fli4l-Drucker klickt man mit der rechten Maustaste an und wählt aus dem
    Kontextmenü ``Eigenschaften''. Auf der Lasche ``Anschlüsse'' deaktiviert man
    ``Bidirektionale Unterstützung aktivieren''. Auf der Lasche ``Erweitert''
    betätigt man die Schaltfläche ``Druckprozessor'' und stellt unter
    ``Druckprozessor'' ``WinPrint'', unter ``Standarddatentyp'' ``RAW'' ein und verlässt
    diese Dialogbox mit ``OK'' (bei Windows NT 4.0 ist hier noch ein Häckchen bei
    ``Raw-Datentyp immer spoolen'' zu setzen). Wieder auf der Lasche ``Erweitert''
    aktiviert man ``Über Spooler drucken, um Druckvorgänge schneller
    abzuschliessen'' und ``Drucken beginnen, nachdem letzte Seite gespoolt wurde''.
    Bei ``Erweiterte Druckfunktionen aktivieren'' entfernt man den Haken, damit
    diese Funktionen nicht genutzt werden. Jetzt übernimmt man alle bisher
    gemachten Einstellungen mit der Schaltfläche ``Übernehmen'' und verlässt das
    komplette Konfigurationsfenster über ``OK'', da Windows NT 4.0/2000 die
    Einstellungen sonst nicht korrekt abspeichert.


\item \textbf{Einrichtung 9x}

    Wenn nicht über Samba gedruckt werden soll, kann mit \var{OPT\_\-LPD}
    lediglich ein Druckerserver für Unix-, Linux- und Windows-NT-Clients
    eingerichtet werden, da nur diese Betriebssysteme geeignete Client-Software
    mitbringen.

    Inzwischen ist es aber auch von Windows9x/Me aus möglich, mit der Freeware-
    Version eines LPR-Clients zu drucken, ohne das den Platz auf dem Medium arg
    strapazierende SAMBA installieren zu müssen.

    Download der LPR-Clients für Windows (es kann nicht garantiert werden, dass
    die Seiten noch erreichbar sind):\\
%   \altlink{http://utep.el.utwente.nl/diensten/ftd/pdf/instlpr.exe}
    \altlink{ftp://ftp.informatik.uni-hamburg.de/pub/os/unix/utils/LPRng/WINDOWS/acitsplr/instlpr.exe}

    Dies ist die letzte freie (kostenlose) Version 3.4f des lpr-Clients für
    Privatanwender. Die aktuelle Version kostet Geld und findet man hier:

        \altlink{http://www.utexas.edu/academic/otl/software/lpr/}

    Die Installation und Konfiguration dieser Software wird für Windows 9x/Me
    und Windows NT 4.0/2000/XP in der Dokumentation zu OPT\_LPDSRV beschrieben
    und wird hier nicht thematisiert.
    In diesem Abschnitt beschränken wir uns auf Betriebssysteme, die die
    LPR-Client-Funktionalität schon beinhalten.



\end{enumerate}
\subsubsection{Einrichtung OPT\_SAMBA aktiviert}

    Die Einrichtung eines Windows-Clients für den Druck über Samba läuft
    unterschiedlich ab, je nachdem, ob \smalljump{sec:OPTNMBD}{\var{OPT\_NMBD}}='no'
    oder \smalljump{sec:OPTNMBD}{\var{OPT\_NMBD}}='yes' gewählt wurde.

\begin{enumerate}
\item \textbf{OPT\_NMBD='no'}

    Bei \smalljump{sec:OPTNMBD}{\var{OPT\_NMBD}}='no' sind die fli4l-Drucker in der Netzwerkumgebung eines
    Windows-PCs nicht zu sehen. Trotzdem kann man sie über ihren UNC-Pfad
    anmelden.

    Dazu ist es notwendig, einen Eintrag für den Router in der Datei hosts
    vorzunehmen. Für diese Datei findet sich bei Windows 95, Windows 98 und
    Windows Me ein Beispiel als host.sam im Windows-Verzeichnis, bei
    Standard-Installationen also in \verb+C:\WINDOWS+, wobei die Endung sam für
    Sample wie Beispiel steht. Unter Windows NT 4.0/2000/XP befindet sich die
    Datei im Verzeichnis von Windows und dort im Verzeichnis
    \verb+SYSTEM32\DRIVERS\ETC+, bei Standard-Installationen also in
    \verb+C:\WINNT\SYSTEM32\DRIVERS\ETC+.

    Hier der Inhalt der Datei von Windows 2000:
\begin{example}
\begin{verbatim}
    # Copyright (c) 1993-1999 Microsoft Corp.
    #
    # Dies ist eine HOSTS-Beispieldatei, die von Microsoft TCP/IP
    # für Windows 2000 verwendet wird.
    #
    # Diese Datei enthält die Zuordnungen der IP-Adressen zu Hostnamen.
    # Jeder Eintrag muss in einer eigenen Zeile stehen. Die IP-
    # Adresse sollte in der ersten Spalte gefolgt vom zugehörigen
    # Hostnamen stehen.
    # Die IP-Adresse und der Hostname müssen durch mindestens ein
    # Leerzeichen getrennt sein.
    #
    # Zusätzliche Kommentare (so wie in dieser Datei) können in
    # einzelnen Zeilen oder hinter dem Computernamen eingefügt werden,
    # aber müssen mit dem Zeichen '#' eingegeben werden.
    #
    # Zum Beispiel:
    #
    #      102.54.94.97     rhino.acme.com          # Quellserver
    #       38.25.63.10     x.acme.com              # x-Clienthost

    127.0.0.1       localhost
\end{verbatim}
\end{example}


    Hier setzt man unter der letzten Zeile den Eintrag für den Router dazu.
    Wenn in der base.txt die IP-Adresse der Netzwerkkarte für das interne
    Netzwerk von fli4l z.B. so konfiguriert wurde

\begin{example}
\begin{verbatim}
    IP_NET_1='192.168.6.1/24'
\end{verbatim}
\end{example}

    und als Name des fli4l-Routers in

\begin{example}
\begin{verbatim}
    HOST_1='192.168.6.1 fli4l'
\end{verbatim}
\end{example}

    fli4l hinterlegt wurde, dann muss folgender Eintrag für die IP-Adresse
    und den Namen von fli4l vorgenommen werden:

    192.168.6.1     fli4l

    Nun speichert man die Datei unter dem Namen \textbf{hosts}
    ab. Dabei wird bei Benutzung von Notepad die Datei hosts.txt erzeugt.
    (Zur Kontrolle ist es notwendig, die Option ``Dateinamenserweiterung bei
    bekannten Dateitypen ausblenden'' unter Windows abzuschalten, da man
    diese nervige Eigenart von Notepad sonst nicht mitbekommt.) Damit die
    Datei wie erforderlich ``hosts''
    heisst, muss man sie in ``hosts'' umbenennen. Nach einem Neustart von Windows
    sind die Vorbereitungen abgeschlossen.

    Bei Erstellung eines neuen Druckers (Start/Einstellungen/Drucker/ Neuer
    Drucker) ist ``Netzwerkdrucker'' auszuwählen. Bei ``Netzwerkpfad oder
    Warteschlangenname'' gibt man
    \verb+\\FLI4LNAME\DRUCKERNAME+
    an. Dabei ist ``FLI4LNAME'' durch den Namen des fli4l-Routers zu ersetzen
    und DRUCKERNAME durch den Namen des Druckers. DRUCKERNAME ist dabei
    je nach Art des Anschlusses (parallel, USB, Remote) unterschiedlich.
    Hierbei gilt generell:
    ``prx'' steht für lokale Drucker an parallelen Ports, ``usbprx'' steht für
    lokale Drucker an USB-Ports, ``reprx'' für Remote-Drucker und ``smbreprx''
    für SMB-Remote-Drucker, wobei das x für 1, 2, 3, also für den ersten,
    zweiten, dritten Anschluss usw. steht.
    Um den ersten lokalen Drucker an einem parallelen Port auszuwählen, wäre also
    \verb+\\fli4l\pr1+
    einzugeben, wenn Euer fli4l wirklich fli4l heisst.
    Möglicherweise habt Ihr aber auch mittels \var{LPD\_\-PARPORT\_\-x\_\-SAMBA\_\-NAME},
    \var{LPD\_\-USBPORT\_\-x\_\-SAMBA\_\-NAME}, \var{LPD\_\-REMOTE\_\-x\_\-SAMBA\_\-NAME}
    und \var{LPD\_\-SMBREMOTE\_\-x\_\-SAMBA\_\-NAME} eigene Windows-Druckernamen
    vergeben, dann sind diese Namen für DRUCKERNAME einzugeben.
    Bei bereits installierten Druckern kann man in den Druckereigenschaften
    auf der Lasche ``Details'' analog zur vorher beschriebenen Vorgehensweise
    den neuen Anschluss eintragen, der hinterher unter ``Anschluss für die Druckausgabe''
    zuzuordnen ist. Die weiteren Einstellungen sind vom Betriebssystem abhängig:

    Weiter für Windows 9x/Me:

    Auf der Lasche ``Details'' sind ausserdem die ``Spool-Einstellungen'' zu
    bearbeiten, man setzt dort ``Druckaufträge in Warteschlange stellen (
    Druckvorgang schneller )'' und ``Druck nach letzter Seite beginnen''.
    Unter Datenformat wählt man ``RAW'' und setzt ausserdem ``Bidirektionale
    Unterstützung deaktivieren''.

    Weiter für Windows NT 4.0/2000/XP:

    Auf der Lasche ``Anschlüsse'' deaktiviert man ``Bidirektionale
    Unterstützung aktivieren''. Auf der Lasche ``Erweitert'' betätigt man die
    Schaltfläche ``Druckprozessor'' und stellt unter ``Druckprozessor''
    ``WinPrint'', unter ``Standarddatentyp'' ``RAW'' ein und verlässt diese
    Dialogbox mit ``OK'' (bei Windows NT 4.0 ist hier noch ein Häckchen bei
    ``Raw-Datentyp immer spoolen'' zu setzen). Wieder auf der Lasche
    ``Erweitert'' aktiviert man ``Über Spooler drucken, um Druckvorgänge
    schneller abzuschliessen'' und ``Drucken beginnen, nachdem letzte Seite
    gespoolt wurde''. Bei ``Erweiterte Druckfunktionen aktivieren'' entfernt
    man den Haken, damit diese Funktionen nicht genutzt werden. Jetzt
    übernimmt man alle bisher gemachten Einstellungen mit der Schaltfläche
    ``Übernehmen'' und verlässt das komplette Konfigurationsfenster über ``OK'',
    da Windows NT 4.0/2000/XP die Einstellungen sonst nicht korrekt
    abspeichert.



\item \textbf{OPT\_NMBD='yes'}


    Bei \smalljump{sec:OPTNMBD}{\var{OPT\_NMBD}}='yes' sind die fli4l-Drucker in der Netzwerkumgebung des
    Windows-PCs sichtbar.

    Bei Erstellung eines neuen Druckers (Start/Einstellungen/Drucker/ Neuer
    Drucker) ist ``Netzwerkdrucker'' auszuwählen. Bei ``Netzwerkpfad oder
    Warteschlangenname'' kann man den ``Durchsuchen''-Button benutzen. Hier
    findet man unter dem in der base.txt definierten Namen für den fli4l-
    Router (HOSTNAME='fli4l') Freigaben wie ``prx'', ``usbprx'', ``reprx'' oder
    ``smbreprx''.
    prx steht für lokale Drucker an parallelen Ports, usbprx steht für
    lokale Drucker an USB-Ports, reprx für Remote-Drucker und smbreprx für
    SMB-Remote-Drucker, wobei das x für 1, 2, 3, also für den ersten, zweiten,
    dritten Anschluss usw. steht.
    Möglicherweise habt Ihr aber auch mittels \var{LPD\_\-PARPORT\_\-x\_\-SAMBA\_\-NAME},
    \var{LPD\_\-USBPORT\_\-x\_\-SAMBA\_\-NAME} und \var{LPD\_\-REMOTE\_\-x\_\-SAMBA\_\-NAME}
    eigene Windows-Druckernamen vergeben, dann sind diese Namen hier zu sehen.
    Bei bereits installierten Druckern kann man in den Druckereigenschaften auf
    der Lasche ``Details'' analog zur vorher beschriebenen Vorgehensweise den
    neuen Anschluss auswählen, der hinterher unter ``Anschluss für die
    Druckausgabe'' zuzuordnen ist.

    Weiter für Windows 9x/Me:

    Auf der Lasche ``Details'' sind ausserdem die ``Spool-Einstellungen'' zu
    bearbeiten, man setzt dort ``Druckaufträge in Warteschlange stellen
    (Druckvorgang schneller)'' und ``Druck nach letzter Seite beginnen''.
    Unter Datenformat wählt man ``RAW'' und setzt ausserdem ``Bidirektionale
    Unterstützung deaktivieren''.

    Weiter für Windows NT 4.0/2000/XP:

    Auf der Lasche ``Anschlüsse'' deaktiviert man ``Bidirektionale
    Unterstützung aktivieren''. Auf der Lasche ``Erweitert'' betätigt man die
    Schaltfläche ``Druckprozessor'' und stellt unter ``Druckprozessor''
    ``WinPrint'', unter ``Standarddatentyp'' ``RAW'' ein und verlässt diese
    Dialogbox mit ``OK'' (bei Windows NT 4.0 ist hier noch ein Häckchen bei
    ``Raw-Datentyp immer spoolen'' zu setzen). Wieder auf der Lasche
    ``Erweitert'' aktiviert man ``Über Spooler drucken, um Druckvorgänge
    schneller abzuschliessen'' und ``Drucken beginnen, nachdem letzte Seite
    gespoolt wurde''. Bei ``Erweiterte Druckfunktionen aktivieren'' entfernt
    man den Haken, damit diese Funktionen nicht genutzt werden. Jetzt
    übernimmt man alle bisher gemachten Einstellungen mit der Schaltfläche
    ``Übernehmen'' und verlässt das komplette Konfigurationsfenster über ``OK'',
    da Windows NT 4.0/2000/XP die Einstellungen sonst nicht korrekt
    abspeichert.


    Noch ein Hinweis dazu:

    Auf dem Windows-Rechner muss das Netzwerkprotokoll TCP/IP installiert
    und konfiguriert sein. Als Standardeinstellung muss unter Windows
    ``NETBIOS over TCP/IP'' aktiviert sein, das Protokoll, welches Samba benutzt.


\end{enumerate}
\subsubsection{Einrichtung eines Linux-LPR-Clients}

    Auf einem Linux-Rechner kann der fli4l-Netzwerkdrucker in der Datei
    /etc/printcap eingetragen werden. Für neuere Drucksysteme wie CUPS siehe unten.

    Beispiel (Name des Druckers: ``drucker''):

\begin{example}
\begin{verbatim}
    drucker:\
            :lp=:\
            :rm=fli4l:\
            :rp=pr1:\
            :sd=/var/spool/lpd/drucker:\
            :sh:mx#0:
\end{verbatim}
\end{example}

    Dabei wird mit ``rm=fli4l'' der Rechnername des fli4l-Routers angegeben.
    Dieser ist gegebenenfalls anzupassen. Soll die Linux-Drucker-Queue anders
    heissen, ist ``drucker'' ebenfalls anzupassen.

    Der Remote-Warteschlangenname in ``rp=pr1'' lautet wie folgt:

    \begin{description}
    \item[:rp=pr1:\ ] für den ersten an fli4l parallel angeschlossenen Drucker
    \item[:rp=pr2:\ ] für den zweiten an fli4l parallel angeschlossenen Drucker

    \item[:rp=usbpr1:\ ] für den ersten an fli4l per USB angeschlossenen Drucker
    \item[:rp=usbpr2:\ ] für den zweiten an fli4l per USB angeschlossenen Drucker

    \item[:rp=repr1:\ bzw. :rp=repr2:\ ] für die konfigurierten Remote-
    Printerserveranschlüsse

    \item[:rp=smbrepr1:\ bzw. :rp=smbrepr2:\ ] für die konfigurierten SMB-
    Remote-Printerserveranschlüsse

    \end{description}


    \wichtig{Nach Einfügen des Eintrages in der Datei /etc/printcap
    muss das Verzeichnis /var/spool/lpd/drucker mit dem mkdir-Kommando
    selbst eingerichtet werden.}

    Mit dem Kommando ``lpr -P drucker DATEINAME'' kann man nun
    Dateien vom Linux-Rechner über fli4l ausdrucken.

Viele neuere Distributionen verwenden alternative Drucksysteme und eigene
Konfigurationstools, bei denen die oben beschriebene Konfiguration misslingt.
Peter Schöne hat aus diesem Grund eine Beschreibung für die in Deutschland weit
verbreitete Distribution SuSE (in der Version 8.1) beigesteuert:

Unter Suse 8.1 mit dem Standarddrucksystem CUPS ist die Einrichtung sehr viel
komfortabler.

Unter YAST2 wählt man aus der Rubrik Hardware die Druckerkonfiguration.
Falls man die lokalen Drucker bereits eingerichtet hat, kann man die
automatische Erkennung getrost überspringen ;-)
Im Fenster ``Druckereinrichtung'' wählt man den Button ``Konfigurieren...'', im
darauffolgenden die Rubrik ``Mehr Anschlussmöglichkeiten zeigen...'' und
bestätigt durch ``Weiter''.
Es werden nun verschiedene Druckertypen angezeigt.
Da es sich hier um ein LPD-kompatibles Paket handelt, wählt man gleich den
ersten Eintrag ``LPD-Vorfilter- und -Weiterleitungs-Warteschlange''.
Nach einer erneuten Bestätigung mit ``Weiter'' gelangt man zur eigentlichen
Konfiguration:
Man kann sich hier, wenn man unsicher ist, den Namen des Routers durch den
Button ``Lookup''-``LPD-Servers'' automatisch eintragen lassen, oder die
IP-Adresse des Routers direkt eingeben.
In das zweite Feld kommt der Name der Druckerwarteschlange.
Dabei ist beim ersten parallel an fli4l angeschlossenen Drucker ``pr1'', beim
zweiten ``pr2'', beim dritten ``pr3'', bei per USB an fli4l
angeschlossenen Druckern ``usbpr1'', ``usbpr2'' usw., bzw. für die remote von
fli4l angesteuerten Drucker ``repr1'', ``repr2'' usw. bzw. ``smbrepr1'',
``smbrepr2'' für die SMB-Remote-Drucker einzutragen.
Ein Klick auf den Button ``Entfernten LPD-Zugang testen'' zeigt, ob die
Einstellungen korrekt sind. Wird dies bestätigt, kann man im Dialog mit
``Weiter'' fortfahren.
Im folgenden Fenster wird ein Namen, unter dem der Drucker aus Anwendungen
heraus druckt, vergeben. Die Felder ``Beschreibung des Druckers'' und ``Standort
des Druckers'' bleiben leer. Es geht noch ``Weiter'' ...
Man wählt nun den am Router angeschlossenen Drucker aus, bestätigt die Auswahl,
entscheidet sich für die richtigen Treiber und schließt die gesamte
Konfiguration mit einem Klick auf den Button ``Beenden'' und einer Bestätigung
mit ``Ja'' ab.
Der Drucker ist nunmehr vollständig eigerichtet und sollte aus den meisten
Anwendungen heraus drucken.

\subsubsection{Einrichtung eines Mac-Clients (MacOSX 10.3 und höher)}


    Hier öffnet man in den ``Systemeinstellungen'' das ``Drucker Dienstprogramm''
    und drückt ``Hinzufügen''. Dann wird ``TCP/IP - Drucker'' ausgewählt und
    als Druckertyp ``LPD/LPR''. Unter ``Druckeradresse'' wird die IP-Adresse des
    Routers eingetragen. Nun ist der ``Name der Warteliste'' anzugeben.
    Dieser heisst beim ersten parallel an fli4l angeschlossenen Drucker ``pr1'',
    beim zweiten ``pr2'' und beim dritten ``pr3'' bzw. bei per USB an fli4l
    angeschlossenen Druckern ``usbpr1'', ``usbpr2'' usw., bzw. bei den
    Remote-Druckern ``repr1'', ``repr2'' usw. bzw. bei den SMB-Remote-Druckern
    ``smbrepr1'', ``smbrepr2'' usw.
    Dann wählt man das Druckermodell aus der Auswahlliste und klickt zum Schluss
    auf ``Hinzufügen''.

