% Synchronized to r30458
\section {OPT\_YAPS - Sending Of SMS Short Messages}
\configlabel{OPT\_YAPS}{OPTYAPS}

\subsection {Introduction}

With YAPS SMS messages can be sent sent via an ISDN card or a modem with
the fli4l router and depending on the provider lower costs arise as if
sending SMS with a mobile phone. In addition, the typing of an
SMS on the PC keyboard is more comfortable.

This package adapts the OPT\_YAPS package by Ralf Dausend to
fli4l 3.x. It is based on the Linux program ``yaps'' by Ulrich Dessauer
and was originally developed by Stefan Rupprecht. Ralf Dausend has expanded
it and wrote the original version of the manual.

\subsection {Preconditions}

YAPS requires a fli4l router with an ISDN card or a modem. YAPS is configured
via the file \texttt{config/yaps.txt}. The file \texttt{/etc/yaps.rc} required by YAPS
is created by the script \texttt{rc500.yaps} when booting the router.

\wichtig{YAPS does not work over DSL or Internet! YAPS needs an ISDN card or
a modem in any case!}

\subsection {Mobile Phone Networks}

For Germany the following services for sending SMS messages are set:
\begin{itemize}
\item D1 with the prefixes 0151, 0160, 0161, 0170, 0171, 0175
\item D2 with the prefixes 0152, 0162, 0172, 0173, 0174
\item E-Plus with the prefixes 0163, 0177, 0178
\end{itemize}

Viag Intercom or O2 offered no SMS gateway for a long time and now have
only a poorly functioning and slow gateway (as of 2005), so
O2 is configured in a way that D2 is used as the gateway. The prefixes
are 0176 and 0179.

Peter Egli enhanced YAPS for Switzerland with the following services: Swisscom,
Orange, Sunrise and Tele2 (Swisscom). Thank you, Peter!

YAPS is configured without international prefixes so users from Germany and
Switzerland may reach their respective networks directly without further
configuration. If someone from Germany wants to send over the Swiss network
or vice versa, in  \texttt{rc500.yaps} the relevant international prefixes have
to be added for the respective networks.

\subsection {Costs}
Since the costs in the mobile sector are in constant motion and also depend
very much on the chosen rates each user must decide for himself
what the service of OPT\_YAPS is worth and what it costs.

\wichtig{Some time ago, several users have reported, that the connection
to the SMS gateway is kept for several hours when using YAPS. Since this
is a connection to a mobile phone number this can be very expensive.
The fault was most likely the SMS gateway and not with YAPS, yet the
people had to pay a high phone bill.
Therefore, the Note: When using YAPS always check that after sending the
SMS the connection to the gateway was closed. Sending usually takes three
to four seconds, in exceptional cases, ten to twenty seconds.}

\achtung{Use YAPS at your own risk! The FLI4L team can and will not pay for any expenses incurred!}

\subsection {Configuration}

\begin{description}
\config{OPT\_YAPS}{OPT\_YAPS}{OPTYAPS}{
If this is set to ``yes'', the package will be activated.
}

\config{YAPS\_USE\_CID}{YAPS\_USE\_CID}{YAPSUSECID}{
This variable must be set to ``True'' or ``False''. If ``True'' is specified,
the Caller-ID will be activated when sending an SMS (see next option).
The Caller ID is the originating address which is displayed on the phone
that receives the SMS. Unfortunately, this seems to work only with D2, D1
always display the phone number (MSN) from which the SMS was sent.
With E-Plus the number of the SMS center is displayed.}

\config{YAPS\_CID}{YAPS\_CID}{YAPSCID}{
This variable contains the sender's number, without spaces or other
separators.}

\config{YAPS\_USE\_SIG}{YAPS\_USE\_SIG}{YAPSUSESIG}{
This variable must be set to ``True'' or ``False''. If ``True''
is selected, a signature is appended to the SMS (see next option).}

\config{YAPS\_SIG}{YAPS\_SIG}{YAPSSIG}{
The signature to be used, please respect the max. length of 160 characters per SMS
note. If the signature is 50 characters long, only 110 characters are left for
the SMS. If a text message (including signature) is longer than 160 characters,
YAPS will split it into several SMS of max 160 characters. So if you, for example,
create a text 500 characters long, YAPS simply sends 4 SMS messages,
one after the other.}

\config{YAPS\_CBC}{YAPS\_CBC}{YAPSCBC}{
If you want to use a call-by-call provider, you may enter its prefix here. If your
provider charges per minute, a CbC provider is useful if you have one that
charges per second, because a text message is normally sent in three to
five seconds.}

\config{YAPS\_MSN}{YAPS\_MSN}{YAPSMSN}{
Here the MSN of your own telephone line must be entered (without prefix).
This field must not be empty, but setting a wrong number is not a problem
since the telephone switch of your provider will simply set one of the available
MSNs for the connection.}

\config{YAPS\_VERBOSE}{YAPS\_VERBOSE}{YAPSVERBOSE}{
This variable sets YAPS's debug level, choose  a value of ``1'' to ``4''. The
debug level specifies how many information YAPS returns. Normally ``1'' should
suffice here. If something does not work, you can set a higher value for debugging.}

\config{YAPS\_LOG}{YAPS\_LOG}{YAPSLOG}{
This variable contains the path and name of the log file created by YAPS.
The default is \texttt{/var/log/yaps.log}. The directory \texttt{/var/log} is situated
in the ramdisk ususally and is therefore only available as long as the router is
running, i.e. after a reboot it is gone. If you want to keep the log file
you may, i.e. specify a file on the hard disk.}

\end{description}

\subsection {Operation}

YAPS may be executed on the console or via Telnet/SSH:
\begin{small}
\begin{example}
\begin{verbatim}
yaps <number> "<text>"
\end{verbatim}
\end{example}
\end{small}

i.e. \verb|yaps 0171xxx "Hello, how are you?"|.

If you execute YAPS without any parameters, a little help on possible
parameters is displayed.

There is also an input interface for fli4l's httpd. This was originally
written by Felix Eckhofer and may be activated with the variable
\var{OPT\_YAPSGUI}. It is documented in the next section.