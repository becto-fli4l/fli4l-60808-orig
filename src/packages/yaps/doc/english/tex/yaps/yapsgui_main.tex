% Synchronized to r29817
\section {OPT\_YAPSGUI - Web-GUI For YAPS}
\configlabel{OPT\_YAPSGUI}{OPTYAPSGUI}

\subsection {Introduction}

This package offers a Web-GUI for operating YAPS. It is based on the
OPT package ``YAPSGUI'' originally developed by Felix Eckhofer
(\email{felix@fli4l.de}). It requires the use of the Mini HTTPD
from the httpd package.

\subsection {User Interface}

If the package is installed (see below), it may be accessed via the Web interface
of the fli4l. Simply pick an entry for ``sender'' and ``recipient'' from the phone
book (see below) or enter a number in the text box below manually, create a message,
click on submit (only once!) and hope all went well (wait for confirmation ...).
The drop-down menu ``Debuglevel'' controls the verbosity level of ''YAPS``, with
``0'' for the lowest and ``4'' for the highest verbosity.

With the checkbox ``Receipt'' you can determine whether an acknowledgment
should be obtained from the network operator for proper delivering of the message.
This unfortunately does not work with all providers.

\subsection {Phonebooks}

For ease of use YAPSGUI provides a phone book, both for senders and recipients.
This may also be managed on a per-user base. Whether, and if so, what phone book
files are used is specified in the configuration (see below). The structure of these files
corresponds to the structure of the file \texttt{/etc/phonebook} from Imonc:

\begin{small}
\begin{example}
\begin{verbatim}
Number1=Name1
Number2=Name2
...
\end{verbatim}
\end{example}
\end{small}

In names all characters except for the equal-sign ``='' are allowed.
An example:

\begin{small}
\begin{example}
\begin{verbatim}
0170123456=John Do-Don't
0162666555=Sabine v. und zu der Thann
\end{verbatim}
\end{example}
\end{small}

\wichtig{If the phone book files are edited outside of the web interface
it must be made sure that the last line ends with a hard return (Enter)!}

\subsection {Configuration}

\begin{description}
\config{OPT\_YAPSGUI}{OPT\_YAPSGUI}{OPTYAPSGUI}{
If this is set to ``yes'' the package will be activated. Precondition is 
\var{OPT\_YAPS} is set to ``yes'' and ``yaps'' is thus installed.
}

\config{YAPSGUI\_DEBUG}{YAPSGUI\_DEBUG}{YAPSGUIDEBUG}{
The default setting for the dropdown menu ``Debuglevel''. Possible
values are between ``0'' and ``4''. ``0'' is predefined as the default value.
}

\config{YAPSGUI\_SENDER\_TB\_COMMON}{YAPSGUI\_SENDER\_TB\_COMMON}{YAPSGUISENDERTBCOMMON}{
Here you can specify where the general sender phonebook is stored which is available
to all users. This should be a place that survives a reboot of the router, such as a
Hard drive, a USB flash drive or a CompactFlash card, since the entries otherwise
will vanish after the restart! By default \texttt{/data/sndbook-common} is specified.
}

\config{YAPSGUI\_RECIPIENT\_TB\_COMMON}{YAPSGUI\_RECIPIENT\_TB\_COMMON}{YAPSGUIRECIPIENTTBCOMMON}{
Here you can specify where the general receiver phonebook is stored which is available
to all users. This should be a place that survives a reboot of the router, such as a
Hard drive, a USB flash drive or a CompactFlash card, since the entries otherwise
will vanish after the restart! By default \texttt{/data/rcvbook-common} specified.
}

\config{YAPSGUI\_USER\_N}{YAPSGUI\_USER\_N}{YAPSGUIUSERN}{
The number of users who will have their own phone books. All other
users automatically use the general phone book. By default,
``0'' is set, so only one telephone directory exists for anyone to use.
}

\config{YAPSGUI\_USER\_x}{YAPSGUI\_USER\_x}{YAPSGUIUSERx}{
Set the user name of x'th user here for whom an own phonebook exists. The user
name must be absolutely identical with that from the configuration of the
Mini-HTTPD (config/httpd.txt), in particular case sensitive!
}

\config{YAPSGUI\_SENDER\_TB\_x}{YAPSGUI\_SENDER\_TB\_x}{YAPSGUISENDERTBx}{
This indicates where the sender phone book for the appropriate user should
to be stored. To the selection of the storage location the same applies as for
the general sender-directory (see the description of the variable
\var{YAPSGUI\_SENDER\_TB\_COMMON} above). By default,
\texttt{/data/sndbook-user}\emph{x} is used, the ``x'' is replaced by the
index number of the user.
}

\config{YAPSGUI\_SENDER\_STD\_x}{YAPSGUI\_SENDER\_STD\_x}{YAPSGUISENDERSTDx}{
This variable controls which sender entry is selected when starting the web interface,
the count starts at one. If ``0'' is specified (the default value), no sender is
initially selected.
}

\config{YAPSGUI\_RECIPIENT\_TB\_x}{YAPSGUI\_RECIPIENT\_TB\_x}{YAPSGUIRECIPIENTTBx}{
Here you can specify where the recipient phone book for the corresponding
users should be stored. To select the storage location, the same applies
as for the general recipient phone book (see above). By default,
\texttt{/data/rcvbook-user}\emph{x} is used, the ``x'' is replaced by the
index number of the user.
}

\config{YAPSGUI\_RECIPIENT\_STD\_x}{YAPSGUI\_RECIPIENT\_STD\_x}{YAPSGUIRECIPIENTSTDx}{
This variable controls which recipient entry is selected when starting the web interface,
the count starts at one. If ``0'' is specified (the default value), no recipient is initially
selected.
}
\end{description}

\subsection {Access Rights}

The permission level for the httpd can be assigned separately for sending and editing
of the phone books. In \var{HTTPD\_RIGHTS\_N} ``sms:send'' resp. ``sms:edittb'' has
to be specified in this case. A user with the rights ``all'' may, of course,
do everything :)
