% Last Update: $Id$
\section {OPT\_YAPSGUI - Web-Oberfläche für YAPS}
\configlabel{OPT\_YAPSGUI}{OPTYAPSGUI}

\subsection {Einleitung}

Dieses Paket bietet eine Web-Oberfläche zur Bedienung von YAPS. Es basiert auf
dem OPT-Paket ``YAPSGUI'', das ursprünglich von Felix Eckhofer
(\email{felix@fli4l.de}) entwickelt wurde. Es setzt den Einsatz des
Mini-HTTPDs aus dem HTTPD-Paket voraus.

\subsection {Oberfläche}

Ist das Paket installiert (siehe unten), ist es über die Web-Oberfläche des
fli4l zu erreichen. Unter ``Absender'' und ``Empfänger'' einfach einen Eintrag
aus dem Telefonbuch (s.u.) auswählen oder Nummer im Textfeld darunter von Hand
eingeben, Nachricht verfassen, auf Abschicken klicken (nur einmal!) und
hoffen, dass alles geklappt hat (Bestätigung abwarten...). Das Dropdown-Menü
``Debuglevel'' regelt, wie viele Meldungen ``yaps'' ausgibt. ``0'' gibt am
wenigsten und ``4'' am meisten Meldungen aus.

Mit der Checkbox ``Empfangsbestätigung'' können Sie bestimmen, ob eine
Empfangsbestätigung vom Netzbetreiber angefordert werden soll. Das funktioniert
aber leider nicht bei jedem Netzbetreiber.

\subsection {Telefonbücher}

YAPSGUI bietet zur einfacheren Bedienung ein Telefonbuch, sowohl für
Absender als auch für Empfänger. Dieses kann auch abhängig vom Benutzer
verwaltet werden. Ob und wenn ja, welche Telefonbuch-Dateien verwendet werden
sollen, wird in der Konfiguration angegeben (s.u.). Der Aufbau dieser Dateien
entspricht dem Aufbau der Telefonbuch-Datei \texttt{/etc/phonebook} vom Imonc:

\begin{small}
\begin{example}
\begin{verbatim}
Nummer1=Name1
Nummer2=Name2
...
\end{verbatim}
\end{example}
\end{small}

Im Namen ist jedes Zeichen außer dem Gleichheitszeichen ``='' erlaubt. Ein
Beispiel:

\begin{small}
\begin{example}
\begin{verbatim}
0170123456=Hans Müller-Lüdenscheidt
0162666555=Sabine v. und zu der Thann
\end{verbatim}
\end{example}
\end{small}

\wichtig{Wenn die Telefonbuch-Dateien außerhalb der Weboberfläche bearbeitet
werden, ist auf jeden Fall darauf zu achten, dass die letzte Zeile mit einem
harten Zeilenumbruch (Enter) abgeschlossen wird!}

\subsection {Konfiguration}

\begin{description}
\config{OPT\_YAPSGUI}{OPT\_YAPSGUI}{OPTYAPSGUI}{
Steht dieser Wert auf ``yes'', wird das Paket aktiviert. Das setzt natürlich
voraus, dass die Variable \var{OPT\_YAPS} auf ``yes'' steht und ``yaps'' somit
installiert wird.
}

\config{YAPSGUI\_DEBUG}{YAPSGUI\_DEBUG}{YAPSGUIDEBUG}{
Der Standardwert für das Dropdown-Menü ``Debuglevel''. Möglich sind
Werte zwischen ``0'' und ``4''. Standardmäßig ist ``0'' voreingestelt.
}

\config{YAPSGUI\_SENDER\_TB\_COMMON}{YAPSGUI\_SENDER\_TB\_COMMON}{YAPSGUISENDERTBCOMMON}{
Hier wird angegeben, wo das allgemeine Absender-Telefonbuch abgelegt werden
soll, das für alle Benutzer gleichermaßen verfügbar ist. Das sollte nach
Möglichkeit ein Ort sein, das einen Neustart des Routers überdauert, etwa eine
Festplatte, ein USB-Stick oder eine CompactFlash-Karte, da die Einträge sonst
nach dem Neustart wieder verschwunden sind! Standardmäßig ist
\texttt{/data/sndbook-common} vorgegeben.
}

\config{YAPSGUI\_RECIPIENT\_TB\_COMMON}{YAPSGUI\_RECIPIENT\_TB\_COMMON}{YAPSGUIRECIPIENTTBCOMMON}{
Hier wird angegeben, wo das allgemeine Empfän\-ger-Telefonbuch abgelegt werden
soll, das für alle Benutzer gleichermaßen verfügbar ist. Das sollte nach
Möglichkeit ein Ort sein, das einen Neustart des Routers überdauert, etwa eine
Festplatte, ein USB-Stick oder eine CompactFlash-Karte, da die Einträge sonst
nach dem Neustart wieder verschwunden sind! Standardmäßig ist
\texttt{/data/rcvbook-common} vorgegeben.
}

\config{YAPSGUI\_USER\_N}{YAPSGUI\_USER\_N}{YAPSGUIUSERN}{
Die Anzahl der Benutzer, die eigene Telefonbücher erhalten sollen. Alle anderen
Benutzer verwenden automatisch das allgemeine Telefonbuch. Standardmäßig ist
``0'' vorgegeben, d.h. dass nur ein Telefonbuch existiert, das für alle Benutzer
verwendet wird.
}

\config{YAPSGUI\_USER\_x}{YAPSGUI\_USER\_x}{YAPSGUIUSERx}{
Hier wird der Benutzername des x. Benutzers vermerkt, für den ein eigenes
Telefonbuch geführt werden soll. Der Benutzername muss absolut identisch mit
demjenigen aus der Konfiguration des Mini-HTTPDs sein (config/httpd.txt);
insbesondere muss die Groß-/Kleinschreibung beachtet werden!
}

\config{YAPSGUI\_SENDER\_TB\_x}{YAPSGUI\_SENDER\_TB\_x}{YAPSGUISENDERTBx}{
Hier wird angegeben, wo das Absender-Telefonbuch für den entsprechenden Benutzer
abgelegt werden soll. Zur Auswahl des Speicherortes gilt das Gleiche wie für
das allgemeine Absender-Telefonbuch (siehe hierzu die Beschreibung der Variable
\var{YAPSGUI\_SENDER\_TB\_COMMON} weiter oben). Standardmäßig ist
\texttt{/data/sndbook-user}\emph{x} vorgegeben, wobei das ``x'' durch die Nummer
des Benutzers ersetzt wird.
}

\config{YAPSGUI\_SENDER\_STD\_x}{YAPSGUI\_SENDER\_STD\_x}{YAPSGUISENDERSTDx}{
Diese Variable regelt, welcher Absender-Eintrag beim Aufruf der Web-Oberfläche
ausgewählt werden soll, wobei die Zählung bei Eins beginnt. Wird ``0''
angegeben (der voreingestellte Wert), wird anfangs gar kein Absender ausgewählt.
}

\config{YAPSGUI\_RECIPIENT\_TB\_x}{YAPSGUI\_RECIPIENT\_TB\_x}{YAPSGUIRECIPIENTTBx}{
Hier wird angegeben, wo das Empfänger-Telefonbuch für den entsprechenden
Benutzer abgelegt werden soll. Zur Auswahl des Speicherortes gilt das Gleiche
wie für das allgemeine Empfänger-Telefonbuch (siehe oben). Standardmäßig ist
\texttt{/data/rcvbook-user}\emph{x} vorgegeben, wobei das ``x'' durch die Nummer
des Benutzers ersetzt wird.
}

\config{YAPSGUI\_RECIPIENT\_STD\_x}{YAPSGUI\_RECIPIENT\_STD\_x}{YAPSGUIRECIPIENTSTDx}{
Diese Variable regelt, welcher Empfänger-Eintrag beim Aufruf der Web-Oberfläche
ausgewählt werden soll, wobei die Zählung bei Eins beginnt. Wird ``0''
angegeben (der voreingestellte Wert), wird anfangs gar kein Empfänger
ausgewählt.
}
\end{description}

\subsection {Zugriffsrechte}

Die Berechtigungsstufe für den httpd kann für das Senden und Ändern der
Telefonbücher seperat vergeben werden. Bei \var{HTTPD\_RIGHTS\_N} muss dann
``sms:send'' bzw. ``sms:edittb'' angegeben werden. Ein Benutzer mit den Rechten
``all'' darf natürlich alles :)
