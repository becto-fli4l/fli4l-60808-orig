% Last Update: $Id$
\section {OPT\_YAPS - Versenden von SMS-Nachrichten}
\configlabel{OPT\_YAPS}{OPTYAPS}

\subsection {Einleitung}

Mit YAPS können über eine ISDN-Karte oder ein Modem des fli4l-Routers
SMS-Nachrichten verschickt werden, wobei je nach Provider weniger Kosten
entstehen, als beim SMS-Versand mit einem Handy. Außerdem ist das Eintippen
der SMS an der PC-Tastatur angenehmer.

Dieses Paket ist eine Anpassung des OPT\_YAPS-Pakets von Ralf Dausend
an fli4l 3.x. Es beruht auf dem Linux Programm ``yaps'' von Ulrich Dessauer
und war ursprünglich von Stefan Rupprecht. Ralf Dausend hat es ausgebaut und
die ursprüngliche Fassung der Anleitung geschrieben.

\subsection {Voraussetzungen}

YAPS benötigt nur einen fli4l-Router mit ISDN-Karte oder einem Modem.
Konfiguriert wird YAPS über die Datei \texttt{config/yaps.txt}. Die von YAPS
benötigte Datei \texttt{/etc/yaps.rc} wird beim Booten des Routers vom Skript
\texttt{rc500.yaps} erstellt.

\wichtig{YAPS funktioniert nicht über DSL oder Internet! YAPS benötigt auf
jeden Fall eine ISDN-Karte oder ein Modem!}

\subsection {Mobilfunknetze}

Für Deutschland sind folgende Dienste für den SMS-Versand eingerichtet:
\begin{itemize}
\item D1 mit den Vorwahlen 0151, 0160, 0161, 0170, 0171, 0175
\item D2 mit den Vorwahlen 0152, 0162, 0172, 0173, 0174
\item E-Plus mit den Vorwahlen 0163, 0177, 0178
\end{itemize}

Viag Interkom bzw. O2 bot lange Zeit gar keinen SMS-Gateway an und hat jetzt
nur einen schlecht funktionierenden und langsamen Gateway (Stand 2005), deshalb
ist O2 so konfiguriert, dass über den D2-Gateway gesendet wird. Die Vorwahlen
sind 0176 und 0179.

Peter Egli hat YAPS für die Schweiz um folgende Dienste erweitert: Swisscom,
Orange, Sunrise und Tele2 (über Swisscom). Vielen Dank dafür, Peter!

YAPS ist ohne internationale Vorwahlen konfiguriert, so dass Nutzer aus
Deutschland und der Schweiz direkt ohne weitere Konfiguration jeweils ihre
landeseigenen Netze erreichen können. Sollte jemand aus Deutschland in ein
Schweizer Netz senden wollen oder umgekehrt, müssen in der \texttt{rc500.yaps}
noch die jeweiligen internationalen Vorwahlen für die jeweiligen Netze ergänzt
werden.

\subsection {Kosten}

Da die Kosten im Mobilfunksektor in ständiger Bewegung sind und auch stark vom
gewählten Tarif abhängen, muss jeder User selbst entscheiden, was ihm der Service
von OPT\_YAPS wert ist und was er kostet.

\wichtig{Vor einiger Zeit haben sich mehrere Nutzer gemeldet, bei denen die
Verbindung von YAPS zum SMS-Gateway über mehrere Stunden gehalten wurde. Da die
Verbindung zu einer Mobilfunknummer hergestellt wird, war das entsprechend
teuer. Der Fehler lag höchstwahrscheinlich beim SMS-Gateway und nicht bei
YAPS, trotzdem mussten die Leute aber die hohe Telefonrechnung zahlen.
Deswegen der Hinweis: Bei der Benutzung von YAPS sollte immer überprüft werden,
dass nach dem Versenden der SMS die Verbindung getrennt wurde. Das Versenden
dauert normalerweise drei bis vier Sekunden, in Ausnahmefällen auch mal zehn
bis zwanzig Sekunden.}

\achtung{Die Nutzung von YAPS erfolgt auf eigene Gefahr!\\
Das fli4l-Team kann nicht für eventuell entstehende Kosten aufkommen!}

\subsection {Konfiguration}

\begin{description}
\config{OPT\_YAPS}{OPT\_YAPS}{OPTYAPS}{
Steht dieser Wert auf ``yes'', wird das Paket aktiviert.
}

\config{YAPS\_USE\_CID}{YAPS\_USE\_CID}{YAPSUSECID}{
Diese Variable muss auf ``True'' oder ``False'' gesetzt werden. Wenn ``True''
gewählt wird, wird die Caller-ID beim Versand aktiviert (siehe nächste Option).
Die Caller-ID ist die Absendernummer, die auf dem Handy, das die SMS empfängt,
angezeigt wird. Leider funktioniert das scheinbar nur bei D2, bei D1 wird
immer die Telefonnummer (MSN) angezeigt, von der aus die SMS verschickt wurde.
Bei E-Plus wird immer die Nummer der SMS-Zentrale angezeigt.}

\config{YAPS\_CID}{YAPS\_CID}{YAPSCID}{
Diese Variable enthält die Absendernummer, ohne Leerstellen oder sonstige
Trennzeichen.}

\config{YAPS\_USE\_SIG}{YAPS\_USE\_SIG}{YAPSUSESIG}{
Diese Variable muss auf ``True'' oder ``False'' gesetzt werden. Wenn ``True''
gewählt wird, wird eine Signatur an die SMS angehängt (siehe nächste Option).}

\config{YAPS\_SIG}{YAPS\_SIG}{YAPSSIG}{
Die zu verwendende Signatur, dabei bitte die max. Länge von 160 Zeichen pro SMS
beachten. Ist die Signatur 50 Zeichen lang, bleiben nur noch 110 Zeichen für
die SMS. Ist eine SMS (inklusive Signatur) länger als 160 Zeichen, zerlegt
YAPS sie in mehrere SMS zu je max. 160 Zeichen, d.h. wenn man z.B. einen 500
Zeichen langen Text eingibt, schickt YAPS einfach 4 SMS-Nachrichten
hintereinander.}

\config{YAPS\_CBC}{YAPS\_CBC}{YAPSCBC}{
Will man einen Call-by-Call Provider nutzen, kann man hier dessen Vorwahl
eintragen. Wenn ihr Provider im Minutentakt abrechnet, ist ein CbC Provider schon
sinnvoll, wenn man einen hat, der bei Mobilfunknummern im Sekundentakt
abrechnet, weil eine SMS normalerweise in drei bis fünf Sekunden verschickt
wird.}

\config{YAPS\_MSN}{YAPS\_MSN}{YAPSMSN}{
Hier muss die MSN des eigenen Telefonanschlusses (ohne Vorwahl) eingetragen
werden. Das Feld darf nicht leer bleiben; wenn eine falsche Nummer angegeben
wird,ist das aber nicht schlimm, da die Vermittlungsstelle dann einfach eine
der verfügbaren MSNs des Anschlusses setzt.}

\config{YAPS\_VERBOSE}{YAPS\_VERBOSE}{YAPSVERBOSE}{
Diese Variable beschreibt mit einem Wert von eins bis vier den von YAPS zu
nutzenden Debuglevel. Der Debuglevel gibt an, wieviele Infos YAPS zurückgibt.
Normalerweise sollte 1 hier ausreichen. Wenn etwas nicht funktioniert, kann man
einen höheren angeben, um zu sehen, woran es liegt.}

\config{YAPS\_LOG}{YAPS\_LOG}{YAPSLOG}{
Diese Variable enthält den Pfad und Namen der Log-Datei, die von YAPS angelegt
wird. Standardmäßig wird \texttt{/var/log/yaps.log} als Log-Datei verwendet. Das
Verzeichnis \texttt{/var/log} liegt normalerweise in der Ramdisk und ist daher
nur solange verfügbar, wie der Router läuft; nach einem Reboot z.B. ist sie
weg. Soll die Logdatei nicht gelöscht werden, muss eine Datei z.B. auf der
Festplatte angegeben werden.}
\end{description}

\subsection {Bedienung}

YAPS kann über die Konsole oder über Telnet/SSH aufgerufen werden:
\begin{small}
\begin{example}
\begin{verbatim}
yaps <nummer> "<text>"
\end{verbatim}
\end{example}
\end{small}

z.B. \verb|yaps 0171xxx "Hallo, wie gehts?"|.

Ruft man YAPS ohne weitere Parameter auf, wird eine kleine Hilfe zu möglichen
Parametern angezeigt.

Es gibt auch eine Eingabeoberfläche für den Mini-HTTPD von fli4l. Diese wurde
usrpünglich von Felix Eckhofer geschrieben und kann über die Variable
\var{OPT\_YAPSGUI} aktiviert werden. Sie wird im nächsten Abschnitt dokumentiert.
