% Synchronized to r29817
\section{Remerciment}

L'équipe fli4l remercies explicitement Ralf Dausend (\email{dausend-ralf@gmx.de})
qui a créé ce paquetage et maintenu pendant un certain temps. Sans son travail
constructif et sa coopération, il n'aurait pas été possible d'adapter l'ensemble
convenablement pour la nouvelle version de fli4l.

\section{License}

Copyright \copyright  2003-2010 Ralf Dausend (\email{dausend-ralf@gmx.de}) \\
Copyright \copyright  2010-     Par l'équipe fli4l (\email{team@fli4l.de}) \\

Ce programme est un logiciel libre. Il est distribué selon les termes
de la GNU License General Public comme prévu par la Free Software Foundation.
Pour de plus amples informations sur la licence, reportez-vous
s'il vous plaît sur le site \altlink{http://www.gnu.org/licenses/gpl.txt}.

Ce programme est distribué dans l'espoir qu'il sera utilisé, mais SANS AUCUNE
GARANTIE~-. Sans même la garantie implicite de COMMERCIALISATION OU POUR
L'ADAPTATION À UN USAGE PARTICULIER les détails peuvent être trouvés dans
la GNU licence General Public.

Vous devriez avoir reçu une copie de la licence GNU General Public avec ce
programme. Sinon, écrivez à~:

\begin{verbatim}
		Free Software Foundation Inc.
		59 Temple Place
		Suite 330
		Boston MA 02111-1307 USA.
\end{verbatim}

Le texte de la licence GNU General Public est publié sur Internet
\altlink{http://www.gnu.org/licenses/gpl.txt} une traduction non officielle en
Allemand peut être trouvé ici \altlink{http://www.gnu.de/documents/gpl.de.html}
et une traduction non officielle en Français ici \altlink{http://org.rodage.com//gpl-3.0.fr.html}
avec ces traductions vous aurez une meilleure aide à la compréhension de la
licence GPL, pour les droits juridiques vous devez utiliser la version Anglaise.