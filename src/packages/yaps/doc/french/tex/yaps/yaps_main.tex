% Synchronized to r30458
\section {OPT\_YAPS - Envoi de messages SMS}
\configlabel{OPT\_YAPS}{OPTYAPS}

\subsection {Introduction}

Avec le paquetage YAPS sur routeur fli4l vous pouvez envoyer des messages
SMS via une carte ISDN ou un modem, le coût de la prestation sera moins cher
que l'envoi d'un SMS avec un téléphone mobile. En outre, en tapant le texte
du SMS sur le clavier du PC est plus confortable.

Le paquetage OPT\_YAPS a été adapté par Ralf Dausend pour fli4l 3.x. le programme
"yaps" a été écrit par Ulrich Dessauer pour Linux, à l'origine il a été développé
par Stefan Rupprecht. Ralf Dausend a amélioré la version originale et écrit le manuel.  

\subsection {Condition}

Pour faire fonctionner YAPS il est nécessaire d'avoir une carte ISDN ou un modem
sur le routeur fli4l. On configure YAPS via le fichier \texttt{config/yaps.txt}.
Le fichier \texttt{/etc/yaps.rc} est requis par YAPS, il est créé par le script
\texttt{rc500.yaps} lors du démarrage du routeur.

\wichtig{YAPS ne fonctionne pas sur la DSL ou Internet~! Dans tous les cas YAPS
a besoin d'une carte ISDN ou d'un modem~!}

\subsection {Réseau téléphone mobile}

Les services suivants pour l'envoi de messages SMS sont définies pour l'Allemagne~:
\begin{itemize}
\item D1 avec les préfixes 0151, 0160, 0161, 0170, 0171, 0175
\item D2 avec les préfixes 0152, 0162, 0172, 0173, 0174
\item E-Plus avec les préfixes 0163, 0177, 0178
\end{itemize}

Viag Intercom ou O2 ne proposaient aucune passerelle SMS depuis longtemps,
maintenant (depuis 2005) ils proposent les SMS, mais il fonctionne mal et
la passerelle est lente, O2 est configuré pour envoyer les messages par
la passerelle D2. Les préfixes sont 0176 et 0179.

Peter Egli à enforcée YAPS pour la Suisse avec les services suivants~:
Swisscom, Orange, Sunrise et Tele2 (de Swisscom). Merci, Peter~!

YAPS est configuré sans préfixe international afin que les utilisateurs
respectifs Allemand et Suisse se connecte sur leur réseaux de leur pays,
directement sans autre configuration. Si quelqu'un en Allemagne veut
envoyer un message sur le réseau Suisse ou vice versa, vous devez compléter
le fichier \texttt{rc500.yaps} les codes internationaux pour les réseaux
respectifs.

\subsection {Le coût}

Comme les coûts dans le secteur des téléphones mobiles sont en constant mouvement
et sont également fortement dépendant du tarif choisi, chaque utilisateur doit
décider si le service OPT\_YAPS en vaut la peine et à qu'elle coûte.

\wichtig{Il y a quelque temps, plusieurs utilisateurs ont rapporté que la connexion
à la passerelle SMS était maintenue pendant plusieurs heures lors de l'utilisation
de YAPS. Comme il s'agit d'une connexion à un numéro de téléphone mobile cela peut
être très coûteux. La faute était probablement la passerelle SMS et pas de YAPS, 
mais les gens ont dû payer une facture de téléphone élevée. Par conséquent, lors
de l'utilisation YAPS toujours vérifier qu'après l'envoi du SMS, la connexion à
la passerelle a été fermée. L'envoi prend habituellement trois à quatre secondes,
dans des cas exceptionnels, de dix à vingt secondes.}

\achtung{L'utilisation de YAPS est à vos risques et périls~!\\
L'équipe fli4l ne peut pas payer pour tous les frais encourus~!}

\subsection {Configuration}

\begin{description}
\config{OPT\_YAPS}{OPT\_YAPS}{OPTYAPS}{
Si vous indiquez  "yes" dans cette variable, le paquet sera activé.
}

\config{YAPS\_USE\_CID}{YAPS\_USE\_CID}{YAPSUSECID}{
Vous pouvez paramétrer cette variable "True" ou "False". Si vous avez
sélectionné "True", l'identification de l'appelant est activé lors de
l'expédition (voir l'option suivante). L'identification de l'appelant est
le numéro de téléphone, il sera affiché à la réception du SMS. Malheureusement,
cela semble fonctionner seulement avec D2, D1 le numéro de téléphone (MSN)
sera toujours afficher à partir duquelle le SMS a été envoyé. Avec E-Plus,
le numéro de la centrale SMS sera toujours affiché.}

\config{YAPS\_CID}{YAPS\_CID}{YAPSCID}{
Dans cette variable vous indiquez le numéro de l'expéditeur, sans espaces
ou d'autres délimiteur.}

\config{YAPS\_USE\_SIG}{YAPS\_USE\_SIG}{YAPSUSESIG}{
Vous pouvez paramétrer cette variable "True" ou "False". Si vous avez
sélectionné "True", une signature sera ajoutée au SMS (voir point suivant).}

\config{YAPS\_SIG}{YAPS\_SIG}{YAPSSIG}{
Si vous utilisez la signature, s'il vous plaît respecter la longueur du texte
du SMS, un max. de 160 caractères par SMS. Si la signature est de 50 caractères,
il ne reste que 110 caractères pour le SMS. Si un message (y compris la signature)
a plus de 160 caractères, YAPS décompose ce message en plusieurs SMS d'un max. de
160 caractères, si par exemple, vous avez entré un texte de 500 caractères, YAPS
enverra tout simplement 4 messages SMS les un derrière les autres.}

\config{YAPS\_CBC}{YAPS\_CBC}{YAPSCBC}{
Si vous souhaitez utiliser un fournisseur Call-by-Call, vous pouvez entrer son
préfixe ici. Si ce fournisseur vous facture les appels à la minute, il vaut mieux
en prendre un fournisseur CbC qui facture les appels à la seconde, car un message
est envoyé normalement entre trois et cinq secondes.}

\config{YAPS\_MSN}{YAPS\_MSN}{YAPSMSN}{
Dans cette variable vous indiquez le MSN de votre propre ligne de téléphone
il doit être saisi (sans préfixe). Ce champ ne doit pas être vide, vous pouvez
même indiquer un faux numéro, se n'est pas un problème puisque pour l'échange
votre fournisseur téléphonique utilise simplement le MSN qui est disponible pour
la connexion.}

\config{YAPS\_VERBOSE}{YAPS\_VERBOSE}{YAPSVERBOSE}{
Dans cette variable vous définissez le niveau de débogage de YAPS, le choix des
valeurs sont de un à quatre. Le niveau de débogage indique le nombre d'informations
YAPS retourne. Normalement la valeur "1" doit suffire. Si quelque chose ne
fonctionne pas, vous pouvez définir une valeur plus élevée pour le débogage.}

\config{YAPS\_LOG}{YAPS\_LOG}{YAPSLOG}{
Cette variable contient le chemin et le nom du fichier journal créé par YAPS.
Le paramètre par défaut est \texttt{/var/log/yaps.log}. Le répertoire \texttt{/var/log}
est situé normalement sur le disque virtuel et est donc uniquement disponible tant
que le routeur est en cours d'exécution, c'est à dire qu'après un redémarrage, tous
disparaît. Si vous souhaitez conserver le fichier journal, vous devez spécifier
le fichier sur un disque dur.}
\end{description}

\subsection {Fonctionnement}

YAPS peut être exécuté sur la console via Telnet/SSH~:
\begin{small}
\begin{example}
\begin{verbatim}
yaps <nummer> "<text>"
\end{verbatim}
\end{example}
\end{small}

Par exemple \verb|yaps 0171xxx "Hallo, wie gehts?"|.

Si vous exécutez YAPS sans aucun paramètre, une aide sur les paramètres possible
sera affiché.

Il y a aussi une interface pour les paramètres avec le Mini-httpd de fli4l. Il a
été initialement écrit par Felix Eckhofer et peut être activé avec la variable
\var{OPT\_YAPSGUI}. Il y a une documentation dans la section suivante.

