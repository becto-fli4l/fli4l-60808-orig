% Synchronized to r29817
\section {OPT\_YAPSGUI - WebGUI pour YAPS}
\configlabel{OPT\_YAPSGUI}{OPTYAPSGUI}

\subsection {Introduction}

Ce paquetage offre un WebGUI pour l'utilisation de YAPS. Il est basé sur
le paquetage OPT "YAPSGUI" d'origine développé par Felix Eckhofer
(\email{felix@fli4l.de}). Il nécessite l'utilisation du Mini-HTTPD du paquetage
HTTPD.

\subsection {Interface utilisateur}

Si le paquet est installé (voir ci-dessous), il peut être accessible via
l'interface Web de fli4l. Il suffit de sélectionner sous "expéditeur" et
"destinataire" dans l'annuaire téléphonique (voir ci-dessous) ou d'entrez
manuellement un numèro dans la zone de texte en dessous, puis créer un message,
cliquez sur soumettre (une seule fois~!) et si tous c'est bien passé (attendre
la confirmation ...). Le menu déroulant "DebugLevel" contrôle le niveau des
messages sorties de "yaps" , avec "0" pour normale et "4" pour le plus haut
niveau de détail.

Si vous cochez la case "réception", vous pouvez déterminer si un accusé de
réception doit être obtenu auprès de l'opérateur du réseau après une bonne
réception du message. Cela ne fonctionne malheureusement pas avec tous
les opérateurs.

\subsection {Annuaire téléphonique}

Pour la facilité d'utilisation de YAPSGUI un annuaire téléphonique est fournit,
à la fois pour les expéditeurs et les destinataires. Cela peut également être géré
par utilisateur. Si, et si vous avez indiquez yes, vous pouvez utiliser le fichier
de configuration pour gérer un annuaire téléphonique (voir ci-dessous). La structure
de ce fichier correspond à la structure du fichier \texttt{/etc/phonebook} de Imonc~:

\begin{small}
\begin{example}
\begin{verbatim}
Nummer1=Name1
Nummer2=Name2
...
\end{verbatim}
\end{example}
\end{small}

Pour les noms tous les caractères, sauf le signe égal "=" sont autorisés. Un exemple~:

\begin{small}
\begin{example}
\begin{verbatim}
0170123456=Hans Müller-Lüdenscheidt
0162666555=Sabine v. und zu der Thann
\end{verbatim}
\end{example}
\end{small}

\wichtig{Si les fichiers du répertoire téléphonique sont édités en dehors
de l'interface web, il faut s'assurer que la dernière ligne se termine par
un retour chariot avec la touche (Entrée)~!}

\subsection {Configuration}

\begin{description}
\config{OPT\_YAPSGUI}{OPT\_YAPSGUI}{OPTYAPSGUI}{
Si vous indiquez le paramètre "yes" le paquetage sera activé. Bien sûr,
cela suppose que la variable  \var{OPT\_YAPS} soit paramétrée sur "yes"
et que "yaps" est donc installé.
}

\config{YAPSGUI\_DEBUG}{YAPSGUI\_DEBUG}{YAPSGUIDEBUG}{
La valeur par défaut du menu déroulant "Debuglevel". Les valeurs possibles
sont de "0" à "4". La valeur par défaut prédéfinie est "0".
}

\config{YAPSGUI\_SENDER\_TB\_COMMON}{YAPSGUI\_SENDER\_TB\_COMMON}{YAPSGUISENDERTBCOMMON}{
Dans cette variable vous indiquez le répertoire général ou sont stockés les
messages envoyés, il est également disponible à tous les utilisateurs. Ce doit
être un endroit qui survit à un redémarrage du routeur, comme un disque dur,
une clé USB ou une carte CompactFlash, autremant les entrées auront disparu
après un redémarrage~! Paramètre par défaut \texttt{/data/sndbook-common}.
}

\config{YAPSGUI\_RECIPIENT\_TB\_COMMON}{YAPSGUI\_RECIPIENT\_TB\_COMMON}{YAPSGUIRECIPIENTTBCOMMON}{
Dans cette variable vous indiquez le répertoire général ou sont stockés les
messages reçus, il est également disponible à tous les utilisateurs. Ce doit
être un endroit qui survit à un redémarrage du routeur, comme un disque dur,
une clé USB ou une carte CompactFlash, autremant les entrées auront disparu
après un redémarrage~! Paramètre par défaut \texttt{/data/rcvbook-common}.
}

\config{YAPSGUI\_USER\_N}{YAPSGUI\_USER\_N}{YAPSGUIUSERN}{
Dans cette variable vous indiquez le nombre d'utilisateurs qui auront leurs
propres annuaires. Tous les autres utilisateurs utiliseront automatiquement
le répertoire général. Paramètre par défaut, "0", un seul répertoire téléphonique
existe pour tous les utilisateurs.
}

\config{YAPSGUI\_USER\_x}{YAPSGUI\_USER\_x}{YAPSGUIUSERx}{
Dans cette variable vous indiquez le nom d'utilisateur, multiplier par x nombre
d'utilisateur, pour que son propre répertoire existe. Le nom d'utilisateur doit
être absolument identique à celui de la configuration du Mini-HTTPD dans
(config/httpd.txt), il est notamment sensible à la casse~!
}

\config{YAPSGUI\_SENDER\_TB\_x}{YAPSGUI\_SENDER\_TB\_x}{YAPSGUISENDERTBx}{
Dans cette variable vous indiquez où le fichier téléphonique pour l'utilisateur
approprié doit être stocké pour l'envoi des messages. Pour sélectionner
l'emplacement de stockage, le même que pour le répertoire général des envois 
(voir la description de la variable \var{YAPSGUI\_SENDER\_TB\_COMMON} ci-dessus).
Le paramètre par défaut est \texttt{/data/sndbook-user}\emph{x}, le "x" est remplacé
par le numéro d'index du nombre d'utilisateur.
}

\config{YAPSGUI\_SENDER\_STD\_x}{YAPSGUI\_SENDER\_STD\_x}{YAPSGUISENDERSTDx}{
Dans cette variable vous indiquez une valeur, un nombre d'expédition sera alors
sélectionné au démarrage de l'interface web, le nombre commence à un. Si vous
indiquez "0" (la valeur par défaut), aucune expédition ne sera sélectionné initialement.
}

\config{YAPSGUI\_RECIPIENT\_TB\_x}{YAPSGUI\_RECIPIENT\_TB\_x}{YAPSGUIRECIPIENTTBx}{
Dans cette variable vous pouvez indiquer, où le fichier téléphonique des messages reçus
pour les utilisateurs correspondant doit être stocké. Pour sélectionner l'emplacement
de stockage, le même que pour le répertoire général des réceptions (voir la description
de la variable \var{YAPSGUI\_RECIPIENT\_TB\_COMMON} ci-dessus). Le paramètre par défaut
est \texttt{/data/rcvbook-user}\emph{x}, le "x" est remplacé par le numéro d'index du
nombre d'utilisateur.
}

\config{YAPSGUI\_RECIPIENT\_STD\_x}{YAPSGUI\_RECIPIENT\_STD\_x}{YAPSGUIRECIPIENTSTDx}{
Dans cette variable vous indiquez une valeur, un nombre de destinataire sera alors
sélectionné au démarrage de l'interface web, le nombre commence à un. Si vous
indiquez "0" (la valeur par défaut), aucun destinataire ne sera sélectionné initialement.
}
\end{description}

\subsection {Droits d'accès}

Le niveau d'autorisation pour le httpd peut être affecté séparément pour l'envoi
et la modification des annuaires téléphoniques. vous devez spécifier dans
\var{HTTPD\_RIGHTS\_N} soit "sms:send" soit "sms:edittb". Un utilisateur avec
les droits "all" peut, bien sûr, tout faire :)

