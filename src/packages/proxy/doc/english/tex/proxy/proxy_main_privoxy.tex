% Synchronized to r43697
\subsection{OPT\_PRIVOXY - A HTTP-Proxy Not Only For Ad Filtering}
\configlabel{OPT\_PRIVOXY}{OPTPRIVOXY}

    The Privoxy homepage (\altlink{http://www.privoxy.org/}) qualifies it as a
    ``Privacy Enhancing Proxy''. As a side-effect it replaces ad banners and popups
    with empty pictures, prevents saving of unwanted cookies (small files which make
    it possible for websites to track users) and so-called web-bugs (1x1 pixel sized
    pictures that are also used to track user behavior).

    Privoxy can be configured and deactivated via a web interface (as long as its running).
    This web interface is found at \altlink{http://config.privoxy.org/} or \altlink{http://p.p/}.
    This configuration does not survive reboots\ldots

    Privoxy is an enhancement of Internet Junkbuster which has been contained in this
    package till version 2.1.0 (\altlink{http://www.junkbuster.com/}).
    All filtering rules are defined in a central file \texttt{default.action}.
    On fli4l it can be found in the directory \texttt{/etc/privoxy}. The main
    advantage of this method is that new versions can be downloaded separately at\\
    \altlink{http://sourceforge.net/projects/ijbswa/files/}.
    Each user of fli4l can keep the file up-to-date in this way without
    the need for fli4l updates.

\begin{description}

\config{PRIVOXY\_MENU}{PRIVOXY\_MENU}{PRIVOXYMENU}

Adds a privoxy entry to the httpd menu.

\config{PRIVOXY\_N}{PRIVOXY\_N}{PRIVOXYN}

        Sets the number of privoxy instances that should be started.

\config{PRIVOXY\_x\_LISTEN}{PRIVOXY\_x\_LISTEN}{PRIVOXYxLISTEN}

        {Specify IP addresses or symbolic names including portnumber of
        the interface here on which Privoxy should listen to clients.
        It is a good idea to specify only trusted interfaces because
        all clients have full access to privoxy (and its activated
        configuration editor). Normally setting \var{IP\_NET\_1\_IPADDR:8118}
        makes most sense.

        Privoxy will listen to the addresses set here offering its
        services. The default port is 8118. This setting has to be used
        in the proxy configuration of your Internet browser. Additional
        informations on client configuration can be found at
        \altlink{http://www.privoxy.org/}

        Define your fli4l name (see \var{HOSTNAME} in base.txt) or its
        IP (i.e. 192.168.6.1) as a proxy in your client. With the port
        number set here all data necessary to configure a web browser
        for using privoxy is provided.}

\config{PRIVOXY\_x\_ALLOW\_N}{PRIVOXY\_x\_ALLOW\_N}{PRIVOXYxALLOWN}

        {Set the number of list entries.}

\config{PRIVOXY\_x\_ALLOW\_x}{PRIVOXY\_x\_ALLOW\_x}{PRIVOXYxALLOWx}

        List of nets and/or IP addresses for which the packet filter
        has to be opened. Default: \var{IP\_NET\_1}.

\config{PRIVOXY\_x\_ACTIONDIR}{PRIVOXY\_x\_ACTIONDIR}{PRIVOXYxACTIONDIR} {
  This variable sets the path to the privoxy rulesets (\emph{default.action}
  and \emph{user.action}) on the router.
  This variable can be used for two things:
  \begin{description}
  \item [Moving rulesets to permanent storage]
    If you specify a path outside of the RAM disk standard rulesets will be
    copied there on first boot and further on the new path is used. Changes to
    rulesets will survice reboots then. Please note that changed rulesets due
    to privoxy updates will be ignored.
  \item [Use of own rulesets]
    fli4l allows standard rulesets  to be overwritten by user defined rulesets.
    Create a subdirectory (i.e. \emph{etc/my\_privoxy} - but not \emph{etc/privoxy}!)
    in the \emph{config} directory to place your own rulesets there.
\end{description}
  Setting this variable is optional.
}

\config{PRIVOXY\_x\_HTTP\_PROXY}{PRIVOXY\_x\_HTTP\_PROXY}{PRIVOXYxHTTPPROXY}

        {If an additional http proxy should be used in conjunction with privoxy
        (i.e. as a webcache) it can be specified here. Privoxy will use this
        proxy then. An entry could be like this:

\begin{example}
\begin{verbatim}
        PRIVOXY_1_HTTP_PROXY='my.provider.de:8000'
\end{verbatim}
\end{example}
        This setting is optional.}

\config{PRIVOXY\_x\_SOCKS\_PROXY}{PRIVOXY\_x\_SOCKS\_PROXY}{PRIVOXYxSOCKSPROXY}

        {If another SOCKS proxy should be used in addition to Privoxy
        it can be specified here. To ensure privacy data traffic may be
        i.e. sent through the Tor network. For details see the
        \jump{OPTTOR}{Tor Documentation}.
        An entry for using Tor could look like this:

\begin{example}
\begin{verbatim}
        PRIVOXY_x_SOCKS_PROXY='127.0.0.1:9050'
\end{verbatim}
\end{example}
        This setting is optional.}

\config{PRIVOXY\_x\_TOGGLE}{PRIVOXY\_x\_TOGGLE}{PRIVOXYxTOGGLE}

        {This option activates toggling the proxy over the web interface.
        If Privoxy is switched off it will act as a simple forwarding
        proxy and won't change page content in any way. Please note
        that this setting affects ALL proxy users (if one user disables
        privoxy it acts as a forwarding proxy for all users).}

\config{PRIVOXY\_x\_CONFIG}{PRIVOXY\_x\_CONFIG}{PRIVOXYxCONFIG}

        {This option enables interactive configuration editing for proxy
        users using Privoxy's web interface. For further details please
        consult the Privoxy documentation.}

\config{PRIVOXY\_x\_LOGDIR}{PRIVOXY\_x\_LOGDIR}{PRIVOXYxLOGDIR}

        {This option specifies a log directory for Privoxy. This may be
        useful for i.e. logging user accesses to websites. If nothing is
        set here only important messages will be logged to console and
        \var{PRIVOXY\_\-LOGLEVEL} is ignored.}

        You may specify 'auto' to make fli4l use the path of the system
        directory for persistent storage. Please take care for \var{FLI4L\_UUID}
        being correctly configured as huge data amounts should be expected and
        /boot or even RAM disk may else overflow.

\config{PRIVOXY\_x\_LOGLEVEL}{PRIVOXY\_x\_LOGLEVEL}{PRIVOXYxLOGLEVEL}

        {This option specifies what kind of informations Privoxy should log.
        The following values are possible (they may be divided by spaces or
        simply added):

        \begin{tabular}[h!]{rl}

   Value & What will be logged? \\
    \hline
    1    & Each request (GET/POST/CONNECT) \\
    2    & Status of each connection \\
    4    & show I/O status \\
    8    & show header parsing \\
    16   & log \textbf{all} data \\
    32   & debug force feature \\
    64   & debug regular expression \\
    128  & debug fast forwarding \\
    256  & debug GIF de-animation \\
    512  & common log format (for logfile analysis) \\
    1024 & debug Popup-Kill function \\
    2048 & log access to the builtin web server \\
    4096 & Start messages and warnings \\
    8192 & Non-fatal errors \\
        \end{tabular}

        To create a log file in common logfile format ONLY 512 should be
        set otherwise the logfile would be 'polluted' by other messages
        that prevent proper analysis.}

\end{description}

        Privoxy offers lots of configuration options. Not all can be
        offered in fli4l's config files. Many of these options can be
        set through Privoxy's web interface. More informations on
        configuration files can be found on Privoxy's homepage.
        The configuration files are found at
        $<$fli4l-directory$>$/opt/etc/privoxy/. These are
        original files from Privoxy packages with all comments
        removed because of space limitations.
