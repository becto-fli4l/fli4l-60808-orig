% Synchronized to r60625
\subsection{OPT\_STUNNEL - Tunneling Connections Over SSL/TLS}
\configlabel{OPT\_STUNNEL}{OPTSTUNNEL}

The program ``stunnel'' allows to encapsulate connections otherwise unencrypted
in an encrypted SSL/TLS tunnel. This allows safe data exchange over otherwise
insecure cleartext protocols. Due to the possibilities of the SSL/TLS protocol,
various forms of Client/server validation are possible.

\subsubsection{Configuration}
\begin{description}

\config{OPT\_STUNNEL}{OPT\_STUNNEL}{}

This variable activates support for SSL/TLS tunnels.

Default setting: \verb+OPT_STUNNEL='no'+

Example: \verb+OPT_STUNNEL='yes'+

\config{STUNNEL\_DEBUG}{STUNNEL\_DEBUG}{STUNNELDEBUG}

This variable can be set to configure the logging settings for ``stunnel''.
Available settings are ``yes'' (everything is logged), ``no'' (warnings and
errors are logged) or a value between zero and seven indicating the severity of
messages whith zero for highest and seven for lowest severity. The setting ``yes''
corresponds to severity seven, while ``no'' corresponds to severity four.

Default setting: \verb+STUNNEL_DEBUG='no'+

Example 1: \verb+STUNNEL_DEBUG='yes'+

Example 2: \verb+STUNNEL_DEBUG='5'+

\config{STUNNEL\_N}{STUNNEL\_N}{STUNNELN}
This variable configures the number of tunnel instances. Each tunnel instance
``listens'' on a network port ``A'' and connects to another network port ``B''
when a connection is established (may as well be on a different machine), then
forwards all traffic from ``A'' to ``B''. Whether the data, that arrives at
``A'' encrypted via SSL/TLS will be decrypted by ``stunnel'' before forwarding
unencrypted to ``B'' or vice versa is decided by the variable setting in
\jump{STUNNELxCLIENT}{\var{STUNNEL\_x\_CLIENT}}.

Default setting: \verb+STUNNEL_N='0'+

Example: \verb+STUNNEL_N='2'+

\config{STUNNEL\_x\_NAME}{STUNNEL\_x\_NAME}{STUNNELxNAME}

The name of each tunnel. Must be unique for all configured tunnels.

Example: \verb+STUNNEL_1_NAME='imond'+

\config{STUNNEL\_x\_CLIENT}{STUNNEL\_x\_CLIENT}{STUNNELxCLIENT}

This variable configures which parts of the communication are encrypted via
SSL/TLS. There are two options:

\begin{itemize}
\item \emph{Client mode:} The tunnel expects unencrypted data from outside
and sends it encrypted to the other end of the tunnel. This corresponds to
the setting\\
\verb+STUNNEL_x_CLIENT='yes'+.
\item\emph{Server mode:} The tunnel expects data encrypted via SSL/TLS from
outside and will send it decrypted to the other end of the tunnel. This is
equivalent to setting \verb+STUNNEL_x_CLIENT='no'+.
\end{itemize}

Tunnels in client mode hence are particularly suitable for connections ``to
the outside'', i.e. to the (unprotected) Internet because data is encrypted
before leaving the local network. Of course the remote site must offer a server
that expects data encrypted via SSL/TLS. For example an e-mail client in the LAN
only supporting unencrypted POP3 can ``talk'' to a POP3 over SSL service on the
Internet \footnote{see \altlink{http://en.wikipedia.org/wiki/POP3S}}

Tunnels in server mode in reverse are for connections that come ``from the
outside'', i.e. from the (unprotected) Internet providing encrypted data.
If the actual service on the server side is not capable to understand
SSL/TLS the data must be decrypted previously. For example the access to
the fli4l web GUI can be accomplished via HTTP (HTTPS) encrypted via SSL/TLS
by configuring a tunnel on the fli4l receiving HTTP traffic encrypted via
SSL/TLS on port 443, then decrypting the data and forwarding it to the local
web server \texttt{mini\_httpd} listening on port 80.

Configurations for these use cases are presented later.

Example: \verb+STUNNEL_1_CLIENT='yes'+

\config{STUNNEL\_x\_ACCEPT}{STUNNEL\_x\_ACCEPT}{STUNNELxACCEPT}

This determines on which port (and address) the tunnel is ``listening''
for incoming connections. In principle two possibilities exist:

\begin{itemize}
\item The tunnel should listen on \emph{all} addresses (on all interfaces).
Use the setting ``any'' in this case.
\item The tunnel should only listen to defined addresses. Set this with
a reference corresponding to the IP-subnet configured, for example
\var{IP\_NET\_1\_IPADDR} (for IPv4) or \var{IPV6\_NET\_2\_IPADDR} (for IPv6).
\end{itemize}

At the end of the address part the port \emph{must} be added, separated
by a colon (``:'').

Example 1: \verb+STUNNEL_1_ACCEPT='any:443'+

Example 2: \verb+STUNNEL_1_ACCEPT='IP_NET_1_IPADDR:443'+

Example 3: \verb+STUNNEL_1_ACCEPT='IPV6_NET_2_IPADDR:443'+

Please note that using \var{IP\_NET\_x\_IPADDR} resp. \var{IPV6\_NET\_x\_IPADDR}
determines the Layer-3-Protocol (IPv4 or IPv6), the choice here \emph{must}
match with the settings in the variables \var{STUNNEL\_x\_ACCEPT\_IPV4} and
\var{STUNNEL\_x\_ACCEPT\_IPV6}. Hence you may not deactivate IPv6 for the tunnel
by using \verb+STUNNEL_1_ACCEPT_IPV6='no'+ and then listen on an IPv6 address using
\verb+STUNNEL_1_ACCEPT='IPV6_NET_2_IPADDR:443'+ or vice versa by using
(\verb+STUNNEL_1_ACCEPT_IPV4='no'+ and \var{IP\_NET\_x\_IPADDR}).
Furthermore, the meaning of ``any'' depends on the Layer 3 protocols activated
(IPv4 or IPv6): of course, the tunnel only listens on addresses belonging to the
Layer-3-Protocols activated via \var{STUNNEL\_x\_ACCEPT\_IPV4} and
\var{STUNNEL\_x\_ACCEPT\_IPV6}.

\config{STUNNEL\_x\_ACCEPT\_IPV4}{STUNNEL\_x\_ACCEPT\_IPV4}{STUNNELxACCEPTIPV4}

This variable controls if the IPv4 protocol is used for \emph{incoming} connections
to the tunnel. Typically this is the case and this variable should be set to ``yes''
while ``no'' ensures that the tunnel only accepts incoming IPv6 connections. However,
this requires a valid IPv6 configuration (refer to the documentation for the ipv6
package for more information).

Default setting: \verb+STUNNEL_x_ACCEPT_IPV4='yes'+

Example: \verb+STUNNEL_1_ACCEPT_IPV4='no'+

\config{STUNNEL\_x\_ACCEPT\_IPV6}{STUNNEL\_x\_ACCEPT\_IPV6}{STUNNELxACCEPTIPV6}

Like in \var{STUNNEL\_x\_ACCEPT\_IPV4} this variable controls whether the IPv6
protocol is used for incoming connections to the tunnel. Typically this is the case
if you use the the general IPv6 protocol by using \verb+OPT_IPV6='yes'+. Setting
``no'' here ensures that the tunnel only accepts incoming IPv4 connections.

Default setting: \verb+STUNNEL_x_ACCEPT_IPV6=<Values from OPT_IPV6>+

Example: \verb+STUNNEL_1_ACCEPT_IPV6='no'+

\config{STUNNEL\_x\_CONNECT}{STUNNEL\_x\_CONNECT}{STUNNELxCONNECT}

Sets the target of the SSL/TLS tunnel. There are basically three possibilities and all
must have the port  appended, separated by a colon (``:''):

\begin{itemize}
\item A numeric IPv4- or IPv6 address

Example 1: \verb+STUNNEL_1_CONNECT='192.0.2.2:443'+

\item The DNS name of an internal host

Example 2: \verb+STUNNEL_1_CONNECT='@webserver:443'+

\item The DNS name of an external host

Example 3: \verb+STUNNEL_1_CONNECT='@www.example.com:443'+
\end{itemize}

If an internal host is entered with both IPv4 and IPv6 address,
the IPv4 address is preferred. If an external host is entered with both
IPv4 and IPv6 address, then the Layer 3 protocol used depends on
which address is first returned by the DNS resolver.

\config{STUNNEL\_x\_OUTGOING\_IP}{STUNNEL\_x\_OUTGOING\_IP}{STUNNELxOUTGOINGIP}

With this optional variable, the \emph{local} address for the
\emph{outgoing} connection of the tunnel can be set. This is only
useful if the target of the tunnel can be reached over multiple interfaces
(routes), i.e. if two concurrent Internet connections are used.
Normally, this variable must not be set.

Example: \verb+STUNNEL_1_OUTGOING_IP='IP_NET_1_IPADDR'+

\config{STUNNEL\_x\_DELAY\_DNS}{STUNNEL\_x\_DELAY\_DNS}{STUNNELxDELAYDNS}

If this optional variable is set to ``yes'', an external DNS name used in
\var{STUNNEL\_x\_CONNECT} will not be converted to an address until the
\emph{outbound} tunnel is established, meaning the point when the first client
has connected locally with the incoming side of the tunnel. This is useful if
the target of the tunnel is a computer that can only be reached through a
dynamic DNS name and the address behind the name changes frequently, or if
an active dialin when starting ``stunnel'' should be prevented.

Default setting: \verb+STUNNEL_x_DELAY_DNS='no'+

Example: \verb+STUNNEL_1_DELAY_DNS='yes'+

\config{STUNNEL\_x\_CERT\_FILE}{STUNNEL\_x\_CERT\_FILE}{STUNNELxCERTFILE}

This variable contains the file name of the certificate for the tunnel to be used.
For server mode tunnels (\verb+STUNNEL_x_CLIENT='no'+) this is the server
certificate that the client validates against a ``Certificate Authority'' (CA)
if necessary. For client mode tunnels (\verb+STUNNEL_x_CLIENT='yes'+)
this is a (usually optional) client certificate that is validated by the
server against a CA if necessary.

The certificate must be provided in the so-called PEM format and must be saved below\\
\texttt{<config-directory>/etc/ssl/stunnel/}. Only the file name must be stored
in this variable, not the path.

For a server mode tunnel the certificate is mandatory!

Example: \verb+STUNNEL_1_CERT_FILE='myserver.crt'+

\config{STUNNEL\_x\_CERT\_CA\_FILE}{STUNNEL\_x\_CERT\_CA\_FILE}{STUNNELxCERTCAFILE}

This variable contains the file name of the CA certificate to be used for the
validation of the certificate of the remote station. Typically clients validate
the server's certificate, vice versa however, is also possible. For details on
the validation please refer to the description of the variable \jump{STUNNELxCERTVERIFY}
{\var{STUNNEL\_x\_CERT\_VERIFY}}.

The certificate must be provided in the so-called PEM format and must be saved below\\
\texttt{<config-directory>/etc/ssl/stunnel/}. Only the file name must be stored
in this variable, not the path.

Example: \verb+STUNNEL_1_CERT_CA_FILE='myca.crt'+

\config{STUNNEL\_x\_CERT\_VERIFY}{STUNNEL\_x\_CERT\_VERIFY}{STUNNELxCERTVERIFY}

This variable controls the validation of the certificate of the remote station.
There are five options possible:

\begin{itemize}
\item \emph{none:} The certificate of the remote station is not validated at all.
In this case the variable \var{STUNNEL\_x\_CERT\_CA\_FILE} is empty.

\item \emph{optional:} If the remote station provides a certificate it is checked
against the CA certificate configured using the variable \var{STUNNEL\_x\_CERT\_CA\_FILE}.
If the remote station does \emph{not} provide a certificate this is not an error and the
connection is still accepted. This setting is only useful for server mode tunnel because
the client tunnel \emph{always} obtain a certificate from the server.

\item \emph{onlyca:} The certificate of the remote station is validated against
the CA certificate configured in the variable \var{STUNNEL\_x\_CERT\_CA\_FILE}.
If the remote station does not provide a certificate or it does not match the
configured CA, the connection is rejected. This is useful when a private CA is used,
as then all potential peers are know.

\item \emph{onlycert:} The certificate of the remote station is compared with
the certificate configured in the variable \var{STUNNEL\_x\_CERT\_CA\_FILE}.
It is \emph{not} checked against a CA certificate, but it will be ensured that
the remote station sends \emph{exactly} the matching (server or client) certificate.
The file referenced with the help of the variable \var{STUNNEL\_x\_CERT\_CA\_FILE}
in this case does not contain a CA certificate, but a host certificate. This setting
ensures that really only a fixed and known peer may connect (server tunnel) or
a connection to only a known peer (client tunnel) is established. This is useful for
peer-to-peer connections between hosts both under your control, but for which no
own CA is used.

\item \emph{both:} The certificate of the remote station is compared  with the
certificate configured by the help of the variable \var{STUNNEL\_x\_CERT\_CA\_FILE}
\emph{and} it is also ensured that it matches a CA certificate. The file referenced
by the help of the variable \var{STUNNEL\_x\_CERT\_CA\_FILE} in this case contains
\emph{both} a CA \emph{and} a host certificate. It is therefore a combination of
the settings \emph{onlycert} and \emph{onlyca}. In comparison to the setting
\emph{onlycert} connections with expired CA certificate will be rejected (even if the
certificate of the peer matches).

\end{itemize}

Default setting: \verb+STUNNEL_x_CERT_VERIFY='none'+

Example: \verb+STUNNEL_1_CERT_VERIFY='onlyca'+

\end{description}

\subsubsection{Use Case 1: Accessing the fli4l-WebGUI via SSL/TLS}

This example enhances the fli4l-WebGUI with SSL/TLS access.

\begin{example}
\begin{verbatim}
OPT_STUNNEL='yes'
STUNNEL_N='1'

STUNNEL_1_NAME='http'
STUNNEL_1_CLIENT='no'
STUNNEL_1_ACCEPT='any:443'
STUNNEL_1_ACCEPT_IPV4='yes'
STUNNEL_1_ACCEPT_IPV6='yes'
STUNNEL_1_CONNECT='127.0.0.1:80'
STUNNEL_1_CERT_FILE='server.pem'
STUNNEL_1_CERT_CA_FILE='ca.pem'
STUNNEL_1_CERT_VERIFY='none'
\end{verbatim}
\end{example}

\subsubsection{Use Case 2: Controlling two remote
fli4l routers via imonc secured by SSL/TLS}

The known weaknesses of the imonc/imond protocol for WAN connections (sending
passwords in clear text) are bypassed with this example. (The LAN connection to
the tunnel of course is vulnerable!)\\

Configuration of the local fli4l in LAN (client tunnel):
\begin{example}
\begin{verbatim}
OPT_STUNNEL='yes'
STUNNEL_N='2'

STUNNEL_1_NAME='remote-imond1'
STUNNEL_1_CLIENT='yes'
STUNNEL_1_ACCEPT='any:50000'
STUNNEL_1_ACCEPT_IPV4='yes'
STUNNEL_1_ACCEPT_IPV6='yes'
STUNNEL_1_CONNECT='@remote1:50000'
STUNNEL_1_CERT_FILE='client.pem'
STUNNEL_1_CERT_CA_FILE='ca+server1.pem'
STUNNEL_1_CERT_VERIFY='both'

STUNNEL_2_NAME='remote-imond2'
STUNNEL_2_CLIENT='yes'
STUNNEL_2_ACCEPT='any:50001'
STUNNEL_2_ACCEPT_IPV4='yes'
STUNNEL_2_ACCEPT_IPV6='yes'
STUNNEL_2_CONNECT='@remote2:50000'
STUNNEL_2_CERT_FILE='client.pem'
STUNNEL_2_CERT_CA_FILE='ca+server2.pem'
STUNNEL_2_CERT_VERIFY='both'
\end{verbatim}
\end{example}

Configuration of the first remote fli4l (server tunnel):
\begin{example}
\begin{verbatim}
OPT_STUNNEL='yes'
STUNNEL_N='1'

STUNNEL_1_NAME='remote-imond'
STUNNEL_1_CLIENT='no'
STUNNEL_1_ACCEPT='any:50000'
STUNNEL_1_ACCEPT_IPV4='yes'
STUNNEL_1_ACCEPT_IPV6='yes'
STUNNEL_1_CONNECT='127.0.0.1:5000'
STUNNEL_1_CERT_FILE='server1.pem'
STUNNEL_1_CERT_CA_FILE='ca+client.pem'
STUNNEL_1_CERT_VERIFY='both'
\end{verbatim}
\end{example}

Configuration of the second remote fli4l (server tunnel):
\begin{example}
\begin{verbatim}
OPT_STUNNEL='yes'
STUNNEL_N='1'

STUNNEL_1_NAME='remote-imond'
STUNNEL_1_CLIENT='no'
STUNNEL_1_ACCEPT='any:50000'
STUNNEL_1_ACCEPT_IPV4='yes'
STUNNEL_1_ACCEPT_IPV6='yes'
STUNNEL_1_CONNECT='127.0.0.1:5000'
STUNNEL_1_CERT_FILE='server2.pem'
STUNNEL_1_CERT_CA_FILE='ca+client.pem'
STUNNEL_1_CERT_VERIFY='both'
\end{verbatim}
\end{example}

A connection to the remote ``imond'' is established by initiating a connection
to the local fli4l on port 50000 (first remote fli4l) resp. 50001 (second remote
fli4l). This fli4l then connects via SSL/TLS-Tunnel to each of the remote fli4l's
which in turn forward their data over a third (host internal) connection to the
remote ``imond'' in the end. The settings of the validation ensure that each fli4l
only accepts the other fli4l as the connecting counterpart.

