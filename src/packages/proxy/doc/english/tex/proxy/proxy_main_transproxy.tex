% Synchronized to r43697
\subsection{OPT\_TRANSPROXY (EXPERIMENTAL) - Transparent HTTP Proxy}
\configlabel{OPT\_TRANSPROXY}{OPTTRANSPROXY}

Transproxy is a ,,transparent'' Proxy - a program that catches all HTTP
requests going through the fli4l router and redirects them to a normal HTTP
proxy i.e. Privoxy. To achieve this the packet filters has to redirect HTTP
queries that should go to the Internet to Transproxy which will then redirect
them to another HTTP proxy. It uses iptables's ,,REDIRECT'' function to
accomplish this:

\begin{verbatim}
        PF_PREROUTING_1='tmpl:http IP_NET_1 REDIRECT:8081'
\end{verbatim}

This rule would redirect all HTTP packets from the first defined net (internal
LAN normally) to Transproxy on port 8081.

\begin{description}

\config{TRANSPROXY\_LISTEN\_N}{TRANSPROXY\_LISTEN\_N}{TRANSPROXYLISTENN}
\config{TRANSPROXY\_LISTEN\_x}{TRANSPROXY\_LISTEN\_x}{TRANSPROXYLISTENx}

        {Specify IP addresses or symbolic names including portnumber of
        the interface here on which Transproxy should listen to clients.
        All interfaces have to be specified here that should redirect
        their packets to Transproxy by the packet filter. With the default
        setting \var{any:8081} Transproxy listens on all interfaces.}

\config{TRANSPROXY\_TARGET\_IP}{TRANSPROXY\_TARGET\_IP}{TRANSPROXYTARGETIP}
\config{TRANSPROXY\_TARGET\_PORT}{TRANSPROXY\_TARGET\_PORT}{TRANSPROXYTARGETPORT}

        {With this options it is set to which service incoming HTTP queries should
        be redirected. This can be a standard HTTP proxy (Squid, Privoxy, Apache, a.s.o.)
        on a random PC (or fli4l itself). Please ensure that this proxy is not in the
        range of the HTTP queries redirected by the packet filter. This would
        cause an infinite loop otherwise.}

\config{TRANSPROXY\_ALLOW\_N}{TRANSPROXY\_ALLOW\_N}{TRANSPROXYALLOWN}
\config{TRANSPROXY\_ALLOW\_x}{TRANSPROXY\_ALLOW\_x}{TRANSPROXYALLOWx}

        {List of nets and/or IP addresses for which the packet filter
        has to be opened. It should cover the nets that should be redirected
        by the packet filter. If you don't set any ranges here they have to
        be entered manually in the configuration of the packet filter.}

\end{description}
