% Synchronized to r43697
\subsection{OPT\_TOR - An Anonymous Communication System For The Internet}
\configlabel{OPT\_TOR}{OPTTOR}

Tor is a tool for a lot of organisations and humans that want to improve
their protection and security while using the Internet. Using Tor aids
them to anonymise browsing, publishing to the web, instant messaging,
IRC, SSH and other TCP based services. Tor also is a platform for software
developers to create new tools for enhanced anonymity, security and
privacy protection.

\altlink{https://www.torproject.org/index.html.de}

\begin{description}

\config{TOR\_LISTEN\_N}{TOR\_LISTEN\_N}{TORLISTENN}

\config{TOR\_LISTEN\_x}{TOR\_LISTEN\_x}{TORLISTENx}

        {Specify IP addresses or symbolic names including portnumber of
        the interface here on which Tor should listen to clients.
        It is a good idea to specify only trusted interfaces because
        all clients have full access to Tor (and its activated
        configuration editor). Normally setting \var{IP\_NET\_1\_IPADDR:9050}
        makes most sense.

        Tor will listen to the addresses set here offering its
        services. The default port is 9050. This setting has to be used
        in the configuration of your programs.

        Define your fli4l name (see \var{HOSTNAME} in base.txt) or its
        IP (i.e. 192.168.6.1) as a proxy in your client. With the port
        number set here all data necessary to configure programs
        for using Tor is provided.}

\config{TOR\_ALLOW\_N}{TOR\_ALLOW\_N}{TORALLOWN}

        {Sets the number of list entries.}

\config{TOR\_ALLOW\_x}{TOR\_ALLOW\_x}{TORALLOWx}

        List of nets and/or IP addresses for which the packet filter
        has to be opened. Default: \var{IP\_NET\_1}.

\config{TOR\_CONTROL\_PORT}{TOR\_CONTROL\_PORT}{TORCONTROLPORT}

        Specify here on which TCP port Tor should open a control
        port using Tor control protocol. The setting is optional.
        If nothing is specified this function will be deactivated.

\config{TOR\_CONTROL\_PASSWORD}{TOR\_CONTROL\_PASSWORD}{TORCONTROLPASSWORD}

        Set a password for the control port here.

\config{TOR\_DATA\_DIR}{TOR\_DATA\_DIR}{TORDATADIR}

        This setting is optional. If not set the default directory /etc/tor
        will be used.

\config{TOR\_HTTP\_PROXY}{TOR\_HTTP\_PROXY}{TORHTTPPROXY}

        {If Tor should forward queries to a HTTP proxy set it here.
        Tor will use the proxy then to use its functions. A valid entry
        could look like this:

\begin{example}
\begin{verbatim}
        TOR_HTTP_PROXY='my.provider.de:8000'
\end{verbatim}
\end{example}
        This setting is optional.}

\config{TOR\_HTTP\_PROXY\_AUTH}{TOR\_HTTP\_PROXY\_AUTH}{TORHTTPPROXYAUTH}

        If the proxy needs authentification set it here. Use notation
        username:password.

\config{TOR\_HTTPS\_PROXY}{TOR\_HTTPS\_PROXY}{TORHTTPSPROXY}

        A HTTPS proxy can be specified here. See
        \smalljump{TORHTTPPROXY}{\var{TOR\_HTTP\_PROXY}}.

\config{TOR\_HTTPS\_PROXY\_AUTH}{TOR\_HTTPS\_PROXY\_AUTH}{TORHTTPSPROXYAUTH}

        See \smalljump{TORHTTPPROXYAUTH}{\var{TOR\_HTTP\_PROXY\_AUTH}}.

\config{TOR\_LOGLEVEL}{TOR\_LOGLEVEL}{TORLOGLEVEL}

        {This option sets Tor's logging behavior. Possible values are: \\
        debug, info, notice, warn or err. \\
        Don't use debug and info due to security concerns except when really needed.}

\config{TOR\_LOGFILE}{TOR\_LOGFILE}{TORLOGFILE}

        If Tor should write its log to a file instead of syslog
        you can set its name here.

        You may specify 'auto' to make fli4l use the path of the system
        directory for persistent storage. Please take care for \var{FLI4L\_UUID}
        being correctly configured as huge data amounts should be expected and
       /boot or even RAM disk may else overflow.

\end{description}
