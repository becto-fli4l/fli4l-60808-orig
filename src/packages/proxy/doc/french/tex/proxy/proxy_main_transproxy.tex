% Synchronized to r43697
\subsection{OPT\_TRANSPROXY (Expétimental) - Proxy HTTP transparent}
\configlabel{OPT\_TRANSPROXY}{OPTTRANSPROXY}

Transproxy est un proxy "transparent", c'est une application qui permet,
d'intercepter toutes les requêtes-HTTP qui passent par le routeur et de
les transmettre à un Proxy-HTTP normal, par ex. au Privoxy. Pour parvenir
à cette fonctionnalité, le filtre de paquets des requêtes HTTP qui doit
aller sur Internet, passent par le transproxy celui-ci les traite et les
transmet à un Proxy-HTTP. Iptables supporte cette fonction en utilisant
le paramètre "REDIRECT"~:

\begin{verbatim}
        PF_PREROUTING_1='tmpl:http IP_NET_1 REDIRECT:8081'
\end{verbatim}

Cette règle transmet tous les paquets-HTTP du premier réseau défini
(normalement c'est le LAN interne) au transproxy par le port 8081.

\begin{description}
\config{TRANSPROXY\_LISTEN\_N}{TRANSPROXY\_LISTEN\_N}{TRANSPROXYLISTENN}
\config{TRANSPROXY\_LISTEN\_x}{TRANSPROXY\_LISTEN\_x}{TRANSPROXYLISTENx}

        {Vous indiquez ici, les adresses IP ou les noms symboliques, ainsi
        que le numéro de port des interfaces, sur lesquels Transproxy doit
        écouter les connexions des clients. Si Toutes les interfaces spécifiées
        ici, utilise déjà un filtrage de paquets, Transproxy sera gêné
        par les paquets qui provient de ce filtrage. Avec la configuration
        par défaut \var{any:8081} Transproxy écoutera toutes les interfaces.}

\config{TRANSPROXY\_TARGET\_IP}{TRANSPROXY\_TARGET\_IP}{TRANSPROXYTARGETIP}
\config{TRANSPROXY\_TARGET\_PORT}{TRANSPROXY\_TARGET\_PORT}{TRANSPROXYTARGETPORT}

        {Grâce à cette option vous définissez le service pour lequel les requêtes
        HTTP doivent être redirigé. Cela peut être n'importe quel proxy standard
        HTTP (Squid, Privoxy, Apache, etc) et sur n'importe quel ordinateur (ou
        sur fli4l lui-même). Faire attention, que le Proxy ne se trouve pas dans
        le domaine du filtrage de paquet, dans lequel les requêtes http seront
        redirigées. Autrement un bouclage d'adresse apparaîtra.}

\config{TRANSPROXY\_ALLOW\_N}{TRANSPROXY\_ALLOW\_N}{TRANSPROXYALLOWN}
\config{TRANSPROXY\_ALLOW\_x}{TRANSPROXY\_ALLOW\_x}{TRANSPROXYALLOWx}

        {Liste des réseaux et/ou des adresses IP pour le filtrage de paquets
        ouvert. Cela devrait couvrir mêmes les réseaux, qui sont redirigés par
        le filtrage de paquets. Si aucun domaine n'est indiqué ici, vous devez
        indiquez les informations manuellement dans la configuration du filtrage
        de paquets.}
\end{description}
