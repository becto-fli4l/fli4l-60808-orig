% Synchronized to r43697
\subsection{OPT\_SS5 - Proxy Socks 4/5 }
\configlabel{OPT\_SS5}{OPTSS5}

Il est nécessaire d'installer le Proxy-Socks pour certains programmes, nous
mettons à disposition ici le protocole SS5. Voir le site Web.

\altlink{http://ss5.sourceforge.net/}

\begin{description}
\config{SS5\_LISTEN\_N}{SS5\_LISTEN\_N}{SS5LISTENN}

\config{SS5\_LISTEN\_x}{SS5\_LISTEN\_x}{SS5LISTENx}

        {Dans la première variable vous indiquez le nombre d'adresse réseau, dans
        la deuxième variable vous indiquez l'adresse-IP ou le nom symbolique,
        y compris le numéro de Port de l'interface, sur lequel SS5 doit écouter
        les connexions des Clients. C'est une bonne idée, d'indiquer ici seulement
        les adresses des interfaces que l'on fait confiance, car tous les
        ordinateurs auront un accès complet à travers SS5 (avec bien sur le
        navigateur configuré et activé). En règle générale il est judicieux
        d'indiquer, la valeur par défaut qui est \var{IP\_NET\_1\_IPADDR:8050}

        Avec l'adresse indiquée ici, SS5 écoute et offre ses services. Le port
        par défaut est 8050. Vous devez utiliser cette information pour configurer
        le proxy dans votre navigateurs.

        Vous devez indiquer dans chaque navigateur en tant que proxy
        l'ordinateur-fli4l, vous allez donc prendre le nom de la variable
        HOSTNAME='fli4l' ou l'adresse-IP (par ex. 192.168.6.1) de la variable
        \var{HOST\_\-x\_\-IP}='192.168.6.1' qui est dans le fichier config de
        fli4l. Avec le Port par défaut, on a ici tous les paramètres nécessaires,
        pour configurer votre navigateur Web, pour l'utilisation de SS5.}

\config{SS5\_ALLOW\_N}{SS5\_ALLOW\_N}{SS5ALLOWN}

        {Vous indiquez dans cette variable le nombre d'adresse réseau à installer.}

\config{SS5\_ALLOW\_x}{SS5\_ALLOW\_x}{SS5ALLOWx}

        Vous indiquez dans cette variable l'adresse réseau ou l'adresse-IP pour
        le quelle le filtrage de paquets doit être ouvert. Normalement il est
        logique d'indiquer ici le paramètre \var{IP\_NET\_1}.
\end{description}
