% Synchronized to r43697
\subsection{OPT\_TOR - Système de communication anonyme pour Internet}
\configlabel{OPT\_TOR}{OPTTOR}

Tor est l'outil d'un grand nombre d'organismes et de simples citoyens, qui
veulent améliorer leur protection et leur sécurité sur Internet. L'utilisation
de Tor vous aide à être anonymes lors de la navigation et de la publication sur
le Web, messagerie instantanée, IRC, SSH et autres applications basées sur TCP.
En outre, Tor fournit une plate-forme sur laquelle les développeurs de logiciels
peuvent créer de nouvelles applications pour plus d'anonymat, sur la sécurité et
la protection de la vie privée.

\altlink{https://www.torproject.org/index.html.fr}

\begin{description}
\config{TOR\_LISTEN\_N}{TOR\_LISTEN\_N}{TORLISTENN}

\config{TOR\_LISTEN\_x}{TOR\_LISTEN\_x}{TORLISTENx}

        {Dans la première variable, vous indiquez le nombre d'adresse réseau,
        dans la deuxième variable vous indiquez l'adresse-IP ou le nom symbolique,
        y compris le numéro de Port de l'interface, sur lequel Tor doit écouter
        les connexions des Clients. C'est une bonne idée, d'indiquer ici seulement
        les adresses des interfaces que l'on fait confiance, car tous les
        ordinateurs auront un accès complet à travers Tor (avec bien sur le
        navigateur configuré et activé). En règle générale il est judicieux
        d'indiquer, la valeur par défaut qui est \var{IP\_NET\_1\_IPADDR:9050}

        Avec l'adresse indiquée ici, Tor écoute et offre ses services. Le port
        par défaut est 9050. Vous devez utiliser cette information pour
        configurer le proxy dans votre navigateur.

        Vous devez indiquer dans chaque navigateur en tant que proxy
        l'ordinateur-fli4l, vous allez donc prendre le nom de la variable
        HOSTNAME='fli4l' ou l'adresse-IP (par ex. 192.168.6.1) de la variable
        \var{HOST\_\-x\_\-IP}='192.168.6.1' qui est dans le fichier config de
        fli4l. Avec le Port par défaut, on a ici tous les paramètres nécessaires,
        pour configurer votre navigateur Web, pour l'utilisation de Tor.}

\config{TOR\_ALLOW\_N}{TOR\_ALLOW\_N}{TORALLOWN}

        {Vous indiquez dans cette variable le nombre d'adresse réseau à installer.}

\config{TOR\_ALLOW\_x}{TOR\_ALLOW\_x}{TORALLOWx}

        Vous indiquez dans cette variable l'adresse réseau ou l'adresse-IP pour
        le quelle le filtrage de paquets doit être ouvert. Normalement il est
        logique d'indiquer ici le paramètre \var{IP\_NET\_1}.

\config{TOR\_CONTROL\_PORT}{TOR\_CONTROL\_PORT}{TORCONTROLPORT}

        Vous indiquez dans cette variable, le port TCP que Tor doit utiliser,
        pour le contrôle d'accès via le protocole Tor. Cette variable
        est optionnelle, si rien n'ai indiqué cette fonction sera désactivée.

\config{TOR\_CONTROL\_PASSWORD}{TOR\_CONTROL\_PASSWORD}{TORCONTROLPASSWORD}

        Vous spécifier dans cette variable, un mot de passe pour le contrôle
        d'accés.

\config{TOR\_DATA\_DIR}{TOR\_DATA\_DIR}{TORDATADIR}

        Cette variable est optionnelle. Si rien n'est indiqué, le dossier par
        défaut /etc/tor est utilisé.

\config{TOR\_HTTP\_PROXY}{TOR\_HTTP\_PROXY}{TORHTTPPROXY}

        {Si vous voulez utiliser en plus de Tor un autre Proxy http, Tor pourra
        alors utiliser ce proxy. Avec cette variable vous avez l'avantage utilisé
        plusieurs Proxys. Le paramètre peut ressembler à cela~:

\begin{example}
\begin{verbatim}
        TOR_HTTP_PROXY='mein.provider.de:8000'
\end{verbatim}
\end{example}

        Ce paramètre est optionnel.}

\config{TOR\_HTTP\_PROXY\_AUTH}{TOR\_HTTP\_PROXY\_AUTH}{TORHTTPPROXYAUTH}

        Une authentification peut-être nécessaire, vous devez la spécifier
        dans cette variable. Ainsi le mandataire sera enregistré sous la forme
        Nom d'utilisateur:Mot de passe.

\config{TOR\_HTTPS\_PROXY}{TOR\_HTTPS\_PROXY}{TORHTTPSPROXY}

        Vous pouvez enregistrer dans cette variable, un Proxy-HTTPS. Voir
        \smalljump{TORHTTPPROXY}{\var{TOR\_HTTP\_PROXY}}.

\config{TOR\_HTTPS\_PROXY\_AUTH}{TOR\_HTTPS\_PROXY\_AUTH}{TORHTTPSPROXYAUTH}

        Voir pour ce sujet \smalljump{TORHTTPPROXYAUTH}{\var{TOR\_HTTP\_PROXY\_AUTH}}.

\config{TOR\_LOGLEVEL}{TOR\_LOGLEVEL}{TORLOGLEVEL}

        {On indique dans cette variable les valeurs pour que Tor puisse
        enregistrer les événements dans le fichier log. Les valeurs suivantes
        sont possibles~: debug, info, notice, warn ou err. Les valeurs debug et
        info ne devraient pas si possible être utilisées, pour des raisons
        de sécurité.}

\config{TOR\_LOGFILE}{TOR\_LOGFILE}{TORLOGFILE}

        Si vous voulez utiliser un autre système que syslog pour enregistrer
        les événements de Tor, vous devez l'indiquer dans cette variable.

        Vous pouvez aussi indiquer 'auto', le chemin du fichier log sera alors
        déplacé dans le répertoire système, pour avoir des données persistantes.
        S'il vous plaît, assurez-vous que la variable \var{FLI4L\_UUID} soit dans
        ce cas configuré correctement. Comme on peut si attendre une grandes
        quantités de données sera enregistrées et le fichier log dans le /boot
        ou dans le Disque-RAM sera rempli rapidement.
\end{description}
