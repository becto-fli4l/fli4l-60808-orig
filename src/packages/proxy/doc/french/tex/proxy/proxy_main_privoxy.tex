% Synchronized to r43697
\subsection{OPT\_PRIVOXY - Filtrage de la publicité avec un proxy HTTP}
\configlabel{OPT\_PRIVOXY}{OPTPRIVOXY}

    Privoxy "Privacy Enhancing Proxy" (="filtrage avancé, pour la protection de
    la vie privée") voir le site Web officiel de Privoxy (\altlink{http://www.privoxy.org/}).
    Privoxy filtre le contenu des pages web sur votre navigateur, en remplaçant
    par des images vides les bannières publicitaires et les Popups, Il gére les
    cookies dans une mémoire cache (petit paquet de données avec lesquels un site
    web peut reconnaître certain surfer) et empêche l'affichage de ce que l'on
    appelle bugs-Web (ce sont de grandes images 1x1 pixels, qui sont utilisées,
    pour espionner le comportement des utilisateurs sur le Net).

    Pendant que Privoxy fonctionne, vous pouvez tout simplement configurer et
    activer les paramètres par l'intermédiaire de l'interface Web. L'interface
    Web se trouve à l'adresse \altlink{http://config.privoxy.org/} ou en abrégée
    \altlink{http://p.p/}.

    \marginpar{Une configuration ainsi activer ne survit pas à un redémarrage
    du routeur fli4l\ldots (tobig)}

    Privoxy Internet Junkbusters à eu une évolution conséquente à partir de la
    version 2.1.0, voir le site Web (\altlink{http://www.junkbuster.com/}).
    L'innovation la plus importante, est que toutes les règles de filtrage sont
    centralisées dans un fichier \texttt{default.action}. Celui-ci se trouve dans le
    répertoire fli4l \texttt{/etc/privoxy}. Le grand avantage de cette méthode c'est
    que les nouvelles versions de ce fichier peuvent être télécharger séparément
    à cette adresse \\
    \altlink{http://sourceforge.net/projects/ijbswa/files/}.\\
    Ainsi, chaque utilisateur fli4l peut tenir ce fichier à jour, sans mettre
    à jour le routeur-fli4l. (Actuellement, la version 1.8 de ce fichier est
    dans ce paquetage)

\begin{description}
\config{PRIVOXY\_MENU}{PRIVOXY\_MENU}{PRIVOXYMENU}

        Avec cette variable, vous pouvez ajouter la section Privoxy au menu-httpd.

\config{PRIVOXY\_N}{PRIVOXY\_N}{PRIVOXYN}

        Vous indiquez dans cette variable le nombre de Privoxy qui doit être
        enregistré pour chaque interface.

\config{PRIVOXY\_x\_LISTEN}{PRIVOXY\_x\_LISTEN}{PRIVOXYxLISTEN}

        {Vous indiquez dans cette variable, l'adresse-IP ou le nom symbolique,
        y compris le numéro de Port de l'interface, sur lequel le Privoxy doit
        écouter les connexions des clients. C'est une bonne idée d'indiquer
        ici, seulement les adresses des interfaces que l'on fait confiance, car
        tous les ordinateurs auront un accès complet à travers le Privoxy (avec
        bien sur le navigateur configuré et activé). En règle générale il est
        judicieux d'indiquer, la valeur par défaut qui est \var{IP\_NET\_1\_IPADDR:8118}

        Avec l'adresse indiquée ici, le Privoxy écoute et offre ses services.
        Le port par défaut est 8118. Vous devez utiliser cette information pour
        configurer le proxy dans votre navigateur. Pour plus de détail sur
        la configuration d'Internet Explorer et de Netscape Navigator, voir le
        site Web~:

        \altlink{http://www.privoxy.org/} \marginpar{URL précise}

        Vous devez enregistrer dans chaque navigateur, en tant que proxy
        l'ordinateur-fli4l, vous allez donc prendre le nom de la variable
        HOSTNAME='fli4l' ou l'adresse-IP (par ex. 192.168.6.1) de la variable
        \var{HOST\_\-x\_\-IP}='192.168.6.1' qui est dans le fichier config de
        fli4l. Avec le Port par défaut, on a ici tous les paramètres nécessaires,
        pour configurer votre navigateur Web, pour l'utilisation du Privoxy.}

\config{PRIVOXY\_x\_ALLOW\_N}{PRIVOXY\_x\_ALLOW\_N}{PRIVOXYxALLOWN}

        {Vous indiquez dans cette variable le nombre d'adresse réseau à installer.}

\config{PRIVOXY\_x\_ALLOW\_x}{PRIVOXY\_x\_ALLOW\_x}{PRIVOXYxALLOWx}

        Vous indiquez dans cette variable l'adresse réseau ou l'adresse-IP pour
        le quelle le filtrage de paquets doit être ouvert. Normalement il est
        logique d'indiquer ici le paramètre \var{IP\_NET\_1}.

\config{PRIVOXY\_x\_ACTIONDIR}{PRIVOXY\_x\_ACTIONDIR}{PRIVOXYxACTIONDIR} {

  Avec cette variable vous indiquez l'emplacement, ou vous pouvez paramétrer
  l'ensembles des règles Privoxy (pour les fichiers \emph{default.action} et
  \emph{user.action}) sur le routeur. Le chemin d'accès spécifié est évaluée par
  rapport au répertoire racine. Cette variable peut être utilisé pour deux
  choses différentes~:

\begin{description}
  \item [Le déplacement dans la mémoire permanente de l'ensemble des règles]
  Si vous spécifiez un répertoire dans un emplacement autre que le disque-RAM,
  au démarrage de fli4l l'ensemble des règles défini par défaut seront copiés et
  utilisé à partir de cet emplacement. Les modifications apportées à ces
  ensembles de règles survivront à un redémarrage du routeur. On doit aussi
  tenir compte du fait qu'après une mise à jour du paquetage Privoxy, ces règles
  seront toujours utilisées donc l'ensemble des règles du paquetage de mise à
  jour sera simplement ignoré.
  \item [l'utilisation de vos propres ensembles de règles]
  L'utilisateur fli4l permet d'écrire des règles spécifiques à la place des
  règles standard. Vous devez dans cette variable indiquer votre propre
  sous-répertoire qui sera dans le répertoire \emph{config} (par exemple
  \emph{etc/mon\_privoxy}, par contre vous ne devez pas indiquer \emph{etc/privoxy})
  ensuite vous placez dans ce sous-répertoire vos propres règles.
\end{description}

         Le paramétrage de cette variable est optionnel.}

\config{PRIVOXY\_x\_HTTP\_PROXY}{PRIVOXY\_x\_HTTP\_PROXY}{PRIVOXYxHTTPPROXY}

        {Si vous voulez utiliser en plus du Privoxy un autre Proxy HTTP,
        c'est-à-dire utiliser également des pages Web en cache, vous pouvez
        paramétrer cette variable. Le Privoxy utilise alors ce proxy. Avec cette
        variable vous avez l'avantage utilisé plusieurs Proxys. Le paramètre
        peut ressembler à cela~:

\begin{example}
\begin{verbatim}
        PRIVOXY_1_HTTP_PROXY='mon.provider.de:8000'
\end{verbatim}
\end{example}
        Ce paramètre est optionnel.}

\config{PRIVOXY\_x\_SOCKS\_PROXY}{PRIVOXY\_x\_SOCKS\_PROXY}{PRIVOXYxSOCKSPROXY}

        {Si vous voulez utiliser en plus du Privoxy un autre Proxy SOCKS. Pour
        augmenter la surveillance privée de la transmission de données du Privoxy,
        par exemple, envoyé les données par le réseau Tor, vous pouvez paramétrer
        cette variable. Pour plus de détails sur Tor, reportez-vous à la
        \jump{OPTTOR}{Documentation Tor}. Le paramètre pour utiliser Tor peut
        ressembler à cela~:

\begin{example}
\begin{verbatim}
        PRIVOXY_x_SOCKS_PROXY='127.0.0.1:9050'
\end{verbatim}
\end{example}
        Ce paramètre est optionnel.}

\config{PRIVOXY\_x\_TOGGLE}{PRIVOXY\_x\_TOGGLE}{PRIVOXYxTOGGLE}

        {Avec cette variable vous pouvez arrêter le proxy par l'interface Web.
        Si le Privoxy est mis hors circuit, il réagira simplement comme
        Proxy-Forwarding et ne modifiera plus le contenu des pages Web transférées.
        Vous devez considérer, que ce réglage vaut pour TOUS les utilisateurs du
        Proxy, c.-à-d. que si un utilisateur arrête Privoxy, le Privoxy sera coupé
        pour tous les autres utilisateurs Web qui transfert par le Proxy.}

\config{PRIVOXY\_x\_CONFIG}{PRIVOXY\_x\_CONFIG}{PRIVOXYxCONFIG}

        {Avec cette variable, les utilisateurs ont la possibilité de configurer
        le proxy par l'interface web. Pour plus de détails, je vous demande
        de consulter la documentation Privoxy qui est ici.}

\config{PRIVOXY\_x\_LOGDIR}{PRIVOXY\_x\_LOGDIR}{PRIVOXYxLOGDIR}

        {Dans cette variable vous pouvez indiquer le répertoire du fichier log
        (ou journal) pour le privoxy. Cela peut être utile, par ex. lorsque
        l'utilisateur veut enregistrer les accès des sites Web. Si rien n'est
        spécifié (par défaut), les principaux messages seront enregistrés sur l
        console, de plus la variable \var{PRIVOXY\_\-LOGLEVEL} sera ignorée.}

        Vous pouvez aussi indiquer 'auto', le chemin du fichier log sera alors
        déplacé dans le répertoire système, pour avoir des données persistantes.
        S'il vous plaît, assurez-vous que la variable \var{FLI4L\_UUID} soit dans
        ce cas configuré correctement. Comme on peut si attendre une grandes
        quantités de données sera enregistrées et le fichier log dans le /boot
        ou dans le Disque-RAM sera rempli rapidement.

\config{PRIVOXY\_x\_LOGLEVEL}{PRIVOXY\_x\_LOGLEVEL}{PRIVOXYxLOGLEVEL}

        {On indique dans cette variable les valeurs, pour que Privoxy puisse
        enregistrer les événements dans le fichier log. Il est possible d'ajouter
        plusieurs valeurs à la suite, vous devez les séparer par un espace. Les
        valeurs suivantes peuvent être ajoutées.

        \begin{tabular}[h!]{rl}

    Valeur & ce qui sera enregistré~? \\
    \hline
    1    & Chaque Requête (GET/POST/CONNECT). \\
    2    & Le statut de chaque connexion \\
    4    & Le statut-I/O \\
    8    & Header-Parsing \\
    16   & \textbf{Toutes} les données \\
    32   & Debug force-feature \\
    64   & Debug regular expression filters \\
    128  & Debug redirects \\
    256  & Debug GIF animation \\
    512  & Common Log Format (Analyse fichier-log) \\
    1024 & Debug kill pop-ups \\
    2048 & CGI (server web) user interface \\
    4096 & Startup banner and warnings \\
    8192 & Non-fatal errors \\
        \end{tabular}

        Pour produire un fichier log (ou journal) avec Common Log Format, vous
        devez indiquer SEULEMENT la valeur 512, si vous indiquez d'autres
        valeurs le fichier log sera "pollué" par d'autres enregistrements
        et vous aurez des problèmes pour l'analyser.}
\end{description}

        Privoxy offre de très nombreuses options de configurations. Cependant
        pour des raisons compréhensibles nous ne pouvons pas développer toutes
        ces options dans le fichier de configuration de fli4l. Beaucoup de ces
        options peuvent être paramétrées sur l'interface Web de Privoxy. Vous
        trouverez des infos plus précises pour la configuration de ces fichiers
        sur la page d'accueil de Privoxy. Les fichiers de configuration de
        Privoxy se trouvent dans le répertoire $<$fli4l-Version$>$/opt/etc/privoxy/.
        Ce sont des fichiers originaux du Paquetage-Privoxy, toutefois, pour
        gagner de la place, tous les commentaires ont été supprimés.
