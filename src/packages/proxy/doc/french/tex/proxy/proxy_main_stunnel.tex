% Synchronized to r60625
\subsection{OPT\_STUNNEL - Tunnel avec une connexion SSL/TLS}
\configlabel{OPT\_STUNNEL}{OPTSTUNNEL}

Le programme "stunnel" permet d'encapsuler les connexions non cryptées dans un tunnel
SSL/TLS crypté. Ce protocole permet d'échanger du texte clair en toute sécurisée. Grâce
aux possibilités du protocole SSL/TLS, vous pouvez paramétrer les différents modes
Client/Serveur.

\subsubsection{Configuration}
\begin{description}

\config{OPT\_STUNNEL}{OPT\_STUNNEL}{}

Cette variable vous permez d'utiliser un tunnel SSL/TLS.

Configuration par défaut~: \verb+OPT_STUNNEL='no'+

Exemple~: \verb+OPT_STUNNEL='yes'+

\config{STUNNEL\_DEBUG}{STUNNEL\_DEBUG}{STUNNELDEBUG}

Avec cette variable vous enregistrez paramètres de fonctionnement du "stunnel" les valeurs
disponibles sont "yes" (tout est enregistrés), "no" (les avertissements et les erreurs sont
enregistrés), vous pouvez aussi indiquer une valeur comprise entre zéro et sept, cela spécifie
la gravité maximale d'enregistrement des messages, si vous indiquez zéro vous enregistrez tous
les messages d'urgences et sept tous les messages de débogage. Le paramètre "yes" correspond
à sept pour la gravité maximale et "no" correspond à quatre pour la gravité maximale.

Configuration par défaut~: \verb+STUNNEL_DEBUG='no'+

Exemple 1~: \verb+STUNNEL_DEBUG='yes'+

Exemple 2~: \verb+STUNNEL_DEBUG='5'+

\config{STUNNEL\_N}{STUNNEL\_N}{STUNNELN}

Avec cette variable vous configurez le nombre de tunnel. Chaque instance de tunnel "écoute"
sur un port du réseau "A" et utilise une connexion entrante sur un autre port du réseau "B"
(qui peut également être situé sur une autre machine) avant tout trafic de "A" vers "B".
Les données qui partent de "A" via le SSL/TLS sont cryptées, ils passent par "stunnel",
les données sont décodées par "B" ou vice versa et ensuite il sont transmis, on paramétre cette
fonction avec la variable \jump{STUNNELxCLIENT}{\var{STUNNEL\_x\_CLIENT}}.

Configuration par défaut~: \verb+STUNNEL_N='0'+

Exemple~: \verb+STUNNEL_N='2'+

\config{STUNNEL\_x\_NAME}{STUNNEL\_x\_NAME}{STUNNELxNAME}

Dans cette variable vous indiquez le nom du tunnel. Ce nom doit être unique pour chaque
tunnel configuré.

Exemple~: \verb+STUNNEL_1_NAME='imond'+

\config{STUNNEL\_x\_CLIENT}{STUNNEL\_x\_CLIENT}{STUNNELxCLIENT}

Dans cette variable vous pouvez régler le tunnel qui communiquera en SSL/TLS crypté.
Il y a deux options~:

\begin{itemize}
\item \emph{Mode client~:} il reçoit les données non cryptées qui proviennent du tunnel
distant et enverra les données cryptées à l'autre extrémité du tunnel. Pour cela vous devez
indiquer \verb+STUNNEL_x_CLIENT='yes'+.
\item \emph{Mode serveur~:} il reçoit les données cryptées via le SSL/TLS, il procède
au décodage et renvoie le résultat à l'autre extrémité du tunnel. Pour cela vous devez
indiquer \verb+STUNNEL_x_CLIENT='no'+.
\end{itemize}

Le tunnel en mode client est particulièrement adapté pour des connexions qui vont
vers l'extérieur, par ex. pour un accès Internet (non protégé), les données seront
cryptées avant de quitter le réseau local. Le site incontournable distant doit bien
sûr avoir également un serveur avec le service SSL/TLS pour recevoir les données
chiffrées. Exemple d'un client e-Mail sur le LAN, le protocole POP3 "parle" en utilisant
des données non cryptées, vous pouvez utiliser un compte POP3 avec le service SSL sur
Internet. \footnote{Voir \altlink{http://fr.wikipedia.org/wiki/Post_Office_Protocol}}

Le tunnel en mode serveur est à l'inverse adapté pour des connexions qui "proviennent
de l'extérieur", par ex. des données qui viennent Internet (non protégé), les données
seront décodées lorsqu'ils l'arriverons. Si le serveur distant n'a pas de service SSL/TLS
les données doivent être décodées avant l'envoient. Exemple, pour accéder sur l'interface
web de fli4l via le HTTP (HTTPS) avec le service SSL/TLS crypté à travers le tunnel, fli4l est
configuré pour recevoir via le SSL/TLS les données cryptées sur le port 443, ils seront
décodées puis les transmet au serveur Web du \texttt{mini\_httpd}, qui écoute sur le port 80.

La configuration pour ces applications sont représentées un peut plus bas.

Exemple~: \verb+STUNNEL_1_CLIENT='yes'+

\config{STUNNEL\_x\_ACCEPT}{STUNNEL\_x\_ACCEPT}{STUNNELxACCEPT}

Dans cette variable vous pouvez régler le port (et l'adresse) du tunnel pour "écouter"
les connexions entrantes. Il y a deux possibilités~:

\begin{itemize}
\item Si le tunnel écouter \emph{toutes} les adresses (sur toutes les interfaces). Vous devez
indiquer "any" dans cette variable.
\item Si le tunnel écoute que sur des adresses spécifiques. Vous devez indiquer la référence
appropriée avec l'IP du sous-réseau configurée, par exemple \var{IP\_NET\_1\_IPADDR} (pour IPv4)
ou \var{IPV6\_NET\_2\_IPADDR} (pour IPv6).
\end{itemize}

En outre, vous \emph{devez} indiquer derrière l'adresse le numéro de port, pour cela vous devez
placer les deux-points (":") entre l'adresse et le port.

Exemple 1~: \verb+STUNNEL_1_ACCEPT='any:443'+

Exemple 2~: \verb+STUNNEL_1_ACCEPT='IP_NET_1_IPADDR:443'+

Exemple 3~: \verb+STUNNEL_1_ACCEPT='IPV6_NET_2_IPADDR:443'+

Il convient de rappeler lors de l'utilisation d'une valeur dans la variable \var{IP\_NET\_x\_IPADDR}
ou dans \var{IPV6\_NET\_x\_IPADDR} elle est de couche 3 du protocole (IPv4 ou IPv6). Le choix \emph{doit}
correspondre au affectation des variables \var{STUNNEL\_x\_ACCEPT\_IPV4} et \var{STUNNEL\_x\_ACCEPT\_IPV6}.
Donc vous ne pouvez pas déactiver un tunnel IPv6 \verb+STUNNEL_1_ACCEPT_IPV6='no'+ et écouter
sur une adresse IPV6 avec \verb+STUNNEL_1_ACCEPT='IPV6_NET_2_IPADDR:443'+; Ceci s'applique
également pour la configuration inverse (\verb+STUNNEL_1_ACCEPT_IPV4='no'+ en utilisant \var{IP\_NET\_x\_IPADDR}).
En outre, cela dépend aussi du sens du paramètre "any", si vous utilisez la couche 3 du protocole
(IPv4 ou IPv6), il écoutera seulement sur des adresses activée, via \var{STUNNEL\_x\_ACCEPT\_IPV4}
et \var{STUNNEL\_x\_ACCEPT\_IPV6} du protocoles de couche 3.

\config{STUNNEL\_x\_ACCEPT\_IPV4}{STUNNEL\_x\_ACCEPT\_IPV4}{STUNNELxACCEPTIPV4}

Dans cette variable vous pouvez indiquer le protocole IPv4 qui peut être utilisé pour la connexion
\emph{entrante} du tunnel. Généralement, cette variable devrait contenir la valeur "yes". Si vous
indiquez "no" le protocole IPv6 sera utilisé pour la connexion entrante du tunnel. Cependant, cela
nécessite une configuration IPv6 valide (reportez-vous à la documentation du paquetage IPv6).

Configuration par défaut~: \verb+STUNNEL_x_ACCEPT_IPV4='yes'+

Exemple~: \verb+STUNNEL_1_ACCEPT_IPV4='no'+

\config{STUNNEL\_x\_ACCEPT\_IPV6}{STUNNEL\_x\_ACCEPT\_IPV6}{STUNNELxACCEPTIPV6}

Cette variable est analogue à la variable \var{STUNNEL\_x\_ACCEPT\_IPV4}, si le protocole IPv6
est utilisé pour la connexion entrante vous devez indiquer la même valeur pour activerer
le protocole IPv6 dans le paquetage avec \verb+OPT_IPV6='yes'+. Si vous indiquez "no" le protocole
IPv4 sera utilisé pour la connexion entrante du tunnel.

Configuration par défaut~: \verb+STUNNEL_x_ACCEPT_IPV6=<valeur de OPT_IPV6>+

Exemple~: \verb+STUNNEL_1_ACCEPT_IPV6='no'+

\config{STUNNEL\_x\_CONNECT}{STUNNEL\_x\_CONNECT}{STUNNELxCONNECT}

Dans cette variable vous indiquez le nom ou l'adresse SSL/TLS du tunnel distant. Cela donne en
principe trois possibilités, vous devez ajouter deux points ":" à la suite de ces trois possibilités
et indiquer le numéro de port~:

\begin{itemize}
\item L'adresse IPv4 ou IPv6 numérique.

Exemple 1~: \verb+STUNNEL_1_CONNECT='192.0.2.2:443'+

\item Le nom du DNS d'un hôte interne.

Exemple 2~: \verb+STUNNEL_1_CONNECT='@webserver:443'+

\item Le nom du DNS d'un hôte externe.

Exemple 3~: \verb+STUNNEL_1_CONNECT='@www.example.com:443'+
\end{itemize}

Si un hôte interne est indiqué, à la fois pour une adresse IPv4 et IPv6, l'adresse IPv4 sera
prévilégiée. Si un hôte externe est indiqué, à la fois pour une adresse IPv4 et IPv6, cela dépend
du retour de l'adresse du protocole de couche 3, le premier DNS résolu sera utilisé.

\config{STUNNEL\_x\_OUTGOING\_IP}{STUNNEL\_x\_OUTGOING\_IP}{STUNNELxOUTGOINGIP}

Dans cette variable optionnelle vous pouvez indiquer l'adresse locale pour la connexion sortante
du tunnel. Cette variable est utilisé que si le tunnel distant a de multiples interfaces (ou routes)
actif, par exemple si la cible utilise deux connexions Internet. Normalement, cette variable ne
doit pas être configurée.

Exemple~: \verb+STUNNEL_1_OUTGOING_IP='IP_NET_1_IPADDR'+

\config{STUNNEL\_x\_DELAY\_DNS}{STUNNEL\_x\_DELAY\_DNS}{STUNNELxDELAYDNS}

Si cette variable optionnelle est activée "yes", le nom de DNS externe utilisé dans \var{STUNNEL\_x\_CONNECT} 
sera converti en une adresse IP si la connexion \emph{sortante} du tunnel est construite. Ainsi
client local sera connecté à la première adresse qui travers le tunnel. Cette variable est utile
que si le tunnel distant, est un ordinateur qui ne peut être atteint que par le nom du DNS dynamique
et si les changements d'adresse sont fréquent derrière ce nom ou si une connexion active empêche
le démarrage du "stunnel".

Configuration par défaut~: \verb+STUNNEL_x_DELAY_DNS='no'+

Exemple~: \verb+STUNNEL_1_DELAY_DNS='yes'+

\config{STUNNEL\_x\_CERT\_FILE}{STUNNEL\_x\_CERT\_FILE}{STUNNELxCERTFILE}

Dans cette variable vous indiquez le nom du fichier du certificat pour le tunnel. Tunnel
en mode serveur, la variable (\verb+STUNNEL_x_CLIENT='no'+) est déactivée, c'est le certificat
du serveur qui est validé par le client contre un "Certificate Authority" (CA) si nécessaire.
Tunnel en mode client, la variable (\verb+STUNNEL_x_CLIENT='yes'+) est activée, c'est le
(généralement en option) certificat du client qui est validé par le serveur contre un CA si
nécessaire.

Le certificat doit être dans le format dit PEM et doit être stocké sous le répertoire
\texttt{<répertoire-config>/etc/ssl/stunnel/}. Seul le nom du fichier doit être stocké dans
cette variable, pas le chemin.

Pour un tunnel en mode serveur le certificat est obligatoire~!

Exemple~: \verb+STUNNEL_1_CERT_FILE='myserver.crt'+

\config{STUNNEL\_x\_CERT\_CA\_FILE}{STUNNEL\_x\_CERT\_CA\_FILE}{STUNNELxCERTCAFILE}

Dans cette variable vous indiquez le nom du fichier du certificat CA à utiliser pour valider
le certificat de la station distante. Les clients valident généralement le certificat du serveur,
cependant il est également possible de contourner la validation. Quelques détails pour
la validation, s'il vous plaît lire la description de la variable \jump{STUNNELxCERTVERIFY}
{\var{STUNNEL\_x\_CERT\_VERIFY}} ci-dessous.

Le certificat doit être dans le format dit PEM et doit être stocké sous le répertoire
\texttt{<répertoire-config>/etc/ssl/stunnel/}. Seul le nom du fichier doit être stocké dans
cette variable, pas le chemin.

Exemple~: \verb+STUNNEL_1_CERT_CA_FILE='myca.crt'+

\config{STUNNEL\_x\_CERT\_VERIFY}{STUNNEL\_x\_CERT\_VERIFY}{STUNNELxCERTVERIFY}

Dans cette variable vous commandez la validation du certificat de la station distante. Il y a
cinq options~:

\begin{itemize}
\item \emph{none~:} le certificat de la station distante n'est pas du tout validé. Dans ce
cas, la variable \var{STUNNEL\_x\_CERT\_CA\_FILE} peut rester vide.

\item \emph{optional~:} le certificat de la station distante est disponible, il sera
vérifié en utilisant le certificat CA qui est configuré dans la variable \var{STUNNEL\_x\_CERT\_CA\_FILE}.
Si la station distante n'a \emph{pas} de certificat disponible, cela ne sera pas indiqué
comme une erreur et la connexion sera toujours accepté. Ce paramètre est utile que pour un
tunnel en mode server, car le tunnel en mode client doit \emph{toujours} obtenir un certificat
à partir du serveur.

\item \emph{onlyca~:} le certificat de la station distante est vérifié en utilisant le certificat
CA, qui est configuré dans la variable \var{STUNNEL\_x\_CERT\_CA\_FILE}. Si la station distante
ne diffuse aucun certificat ou s'il ne correspond pas au CA configuré, la connexion sera chargée.
Ce paramètre est utile si vous utilisez votre propre CA et que vous connaissez tous
les stations distantes potentiels.

\item \emph{onlycert~:} le certificat de la station distante est comparé avec le certificat
qui est configuré dans la variable \var{STUNNEL\_x\_CERT\_CA\_FILE}. Il ne sera
\emph{pas} vérifié en utilisant le certificat CA, mais il veillera que le certificat de
la station distante \emph{envoyé} est exactement le même entre le certificat du (serveur ou
du client). Le fichier qui est référencé dans la variable \var{STUNNEL\_x\_CERT\_CA\_FILE}
ne contient pas de CA, mais un certificat hôte. Ce paramètre garantit que seul un serveur
déterminé connu (tunnel en mode serveur) peut être construite et connecté à un terminal connu
(tunnel en mode client). Ce paramètre est utile pour les connexions peer-to-peer entre hôtes
les deux sont sous contrôle et vous n'utilisez pas votre propre CA.

\item \emph{both~:} le certificat de la station distante est comparé avec le certificat
qui est configuré dans la variable \var{STUNNEL\_x\_CERT\_CA\_FILE} \emph{et} sera également
vérifié en utilisant le certificat CA qui est également dans cette variable. Les \emph{deux}
fichiers qui sont référencés dans la variable \var{STUNNEL\_x\_CERT\_CA\_FILE}, contient un CA
\emph{et} un certificat hôte. C'est une combinaison des options \emph{onlycert} et \emph{onlyca}.
En comparaison avec le réglage \emph{onlycert} la connexion sera refusée, si le certificat CA
a expiré (même si le certificat de la station distante est validé).

\end{itemize}

Configuration par défaut~: \verb+STUNNEL_x_CERT_VERIFY='none'+

Exemple~: \verb+STUNNEL_1_CERT_VERIFY='onlyca'+

\end{description}

\subsubsection{1ère exemple d'application~: connexion au WebGUI de fli4l via le SSL/TLS}

Avec cet exemple, le WebGUI de fli4l est prolongé par une connexion SSL/TLS.

\begin{example}
\begin{verbatim}
OPT_STUNNEL='yes'
STUNNEL_N='1'

STUNNEL_1_NAME='http'
STUNNEL_1_CLIENT='no'
STUNNEL_1_ACCEPT='any:443'
STUNNEL_1_ACCEPT_IPV4='yes'
STUNNEL_1_ACCEPT_IPV6='yes'
STUNNEL_1_CONNECT='127.0.0.1:80'
STUNNEL_1_CERT_FILE='server.pem'
STUNNEL_1_CERT_CA_FILE='ca.pem'
STUNNEL_1_CERT_VERIFY='none'
\end{verbatim}
\end{example}

\subsubsection{2ère exemple d'application~: contrôle de deux routeurs fli4l distant via
le SSL/TLS sécurisé, par le programme imonc}

Avec cet exemple, vous contournez les faiblesses connues du Protocole imonc/imond (envoi
des mots de passe en clair) pour une connexion WAN. (Naturellement, la connexion au LAN
avec le tunnel peut continuer à être exploité~!)

Configuration local de fli4l dans le LAN (tunnel en mode client)~:

\begin{example}
\begin{verbatim}
OPT_STUNNEL='yes'
STUNNEL_N='2'

STUNNEL_1_NAME='remote-imond1'
STUNNEL_1_CLIENT='yes'
STUNNEL_1_ACCEPT='any:50000'
STUNNEL_1_ACCEPT_IPV4='yes'
STUNNEL_1_ACCEPT_IPV6='yes'
STUNNEL_1_CONNECT='@remote1:50000'
STUNNEL_1_CERT_FILE='client.pem'
STUNNEL_1_CERT_CA_FILE='ca+server1.pem'
STUNNEL_1_CERT_VERIFY='both'

STUNNEL_2_NAME='remote-imond2'
STUNNEL_2_CLIENT='yes'
STUNNEL_2_ACCEPT='any:50001'
STUNNEL_2_ACCEPT_IPV4='yes'
STUNNEL_2_ACCEPT_IPV6='yes'
STUNNEL_2_CONNECT='@remote2:50000'
STUNNEL_2_CERT_FILE='client.pem'
STUNNEL_2_CERT_CA_FILE='ca+server2.pem'
STUNNEL_2_CERT_VERIFY='both'
\end{verbatim}
\end{example}

Configuration du premier fli4l distant (tunnel en mode serveur)~:
\begin{example}
\begin{verbatim}
OPT_STUNNEL='yes'
STUNNEL_N='1'

STUNNEL_1_NAME='remote-imond'
STUNNEL_1_CLIENT='no'
STUNNEL_1_ACCEPT='any:50000'
STUNNEL_1_ACCEPT_IPV4='yes'
STUNNEL_1_ACCEPT_IPV6='yes'
STUNNEL_1_CONNECT='127.0.0.1:5000'
STUNNEL_1_CERT_FILE='server1.pem'
STUNNEL_1_CERT_CA_FILE='ca+client.pem'
STUNNEL_1_CERT_VERIFY='both'
\end{verbatim}
\end{example}

Configuration du second fli4l distant (tunnel en mode serveur)~:
\begin{example}
\begin{verbatim}
OPT_STUNNEL='yes'
STUNNEL_N='1'

STUNNEL_1_NAME='remote-imond'
STUNNEL_1_CLIENT='no'
STUNNEL_1_ACCEPT='any:50000'
STUNNEL_1_ACCEPT_IPV4='yes'
STUNNEL_1_ACCEPT_IPV6='yes'
STUNNEL_1_CONNECT='127.0.0.1:5000'
STUNNEL_1_CERT_FILE='server2.pem'
STUNNEL_1_CERT_CA_FILE='ca+client.pem'
STUNNEL_1_CERT_VERIFY='both'
\end{verbatim}
\end{example}

La connexion à "imond" distant est construite à travers une connexion fli4l local sur
le port 5000 (première fli4l distant) et sur le port 5001 (seconde fli4l distant). fli4l
se connecte via le tunnel SSL/TLS au fli4l distant respectif, le démon "imond" transmettra
à son tour les données via un tiers (hôte interne) vers une connexion distante. Les réglages
garantit la validation de chaque fli4l et acceptera uniquement l'autre fli4l comme partenaire
de connexion.
