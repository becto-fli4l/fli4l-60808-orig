% Last Update: $Id$
\subsection{OPT\_PRIVOXY - Ein Werbung-filternder HTTP-Proxy}
\configlabel{OPT\_PRIVOXY}{OPTPRIVOXY}

    Privoxy wird auf der offiziellen Privoxy-Homepage
(\altlink{http://www.privoxy.org/})
    als ``Privacy Enhancing Proxy'' (=''privatsphärenerweiternder Proxy'')
beworben.
    Als sichbarer und erwünschenwerter Nebeneffekt ersetzt Privoxy Werbe-Banner
und
    -Popups durch leere Bilder, verhindert das Speichern von ungewollten Cookies
    (kleine Datenpakete, mit denen eine Website einen bestimmten Surfer
wiedererkennen
    kann) und verhindert die Anzeige von sogenannten Web-Bugs (das sind 1x1
Pixel große
    Bilder, die dazu benutzt werden, um das Surfverhalten von Benutzern
auszuspähen).

    Privoxy kann, während es läuft, ganz einfach über ein Webinterface
konfiguriert
    und (de)ak\-ti\-viert werden. Dieses Webinterface findet sich unter
    \altlink{http://config.privoxy.org/} oder \altlink{http://p.p/}.
    \marginpar{Eine so getätige Konfiguration überlebt keinen Neustart\ldots
(tobig)}

    Privoxy ist eine konsequente Weiterentwicklung des Internet Junkbusters, der
    bis Version 2.1.0 in diesem Paket (\altlink{http://www.junkbuster.com/})
enthalten
    war.
    Die wichtigste Neuerung ist, dass alle Regeln für die Filterung in der
zentralen
    Datei \texttt{default.action} definiert werden. Diese befindet sich bei fli4l
im
    Verzeichnis \texttt{/etc/privoxy}. Der große Vorteil an dieser Methode ist,
dass
    sich neue Versionen dieser Datei seperat von \\
    \altlink{http://sourceforge.net/projects/ijbswa/files/}
    herunterladen lassen. So kann jeder fli4l-Benutzer die Datei selbst auf dem
    neusten Stand halten, ohne auf Updates von fli4l angewiesen zu sein.
(Momentan
    befindet sich die Version 1.8 dieser Datei im Paket.)

\begin{description}

\config{PRIVOXY\_MENU}{PRIVOXY\_MENU}{PRIVOXYMENU}

Fügt dem httpd-Menü einen Privoxy-Abschnitt hinzu.

\config{PRIVOXY\_N}{PRIVOXY\_N}{PRIVOXYN}

        Gibt die Anzahl der Privoxy-Instanzen an, die gestartet werden sollen.

\config{PRIVOXY\_x\_LISTEN}{PRIVOXY\_x\_LISTEN}{PRIVOXYxLISTEN}

        {Hier werden die IP-Adressen oder symbolischen Namen inklusive
        der Portnummer der Interfaces angegeben, auf denen Privoxy auf
        Verbindungen von Clients horchen soll. Es ist eine gute Idee,
        hier nur die Adressen der Interfaces anzugeben, denen man
        vertraut, da alle Rechner vollen Zugriff auf Privoxy haben
        (und auf den eventuell aktivierten Konfigurations-Editor). In
        der Regel ist die Vorgabe \var{IP\_NET\_1\_IPADDR:8118}
        sinnvoll.

        Auf hier angegebenen Adressen lauscht Privoxy und bietet seine
        Dienste an.  8118 ist der Standard-Port. Die Angabe hier muss
        man dann bei der Proxy-Konfiguration des jeweils verwendeten
        Internet-Browsers benutzen.  Weitere Informationen zur
        Konfiguration von Internet Explorer und Netscape Navigator auf

        \altlink{http://www.privoxy.org/} \marginpar{Genaue URL}

        Als Proxy beim jeweiligen Browser muss der fli4l-Rechner
        angegeben werden, also das, was man bei HOSTNAME='fli4l'
        angegeben hat bzw.  dessen IP (z.B 192.168.6.1), die man bei
        \var{HOST\_\-x\_\-IP}='192.168.6.1' angegeben hat. Zusammen
        mit dieser Port-Angabe hier hat man dann alle nötigen Daten,
        um seinen Webbrowser für die Nutzung von Privoxy zu
        konfigurieren.}

\config{PRIVOXY\_x\_ALLOW\_N}{PRIVOXY\_x\_ALLOW\_N}{PRIVOXYxALLOWN}

        {Gibt die Anzahl der Listeneinträge an.}

\config{PRIVOXY\_x\_ALLOW\_x}{PRIVOXY\_x\_ALLOW\_x}{PRIVOXYxALLOWx}

        Die Liste der Netze und/oder IP-Adressen für die der
        Paketfilter geöffnet wird. Sinnvoll ist hier auch wieder die
        Vorgabe \var{IP\_NET\_1}.

\config{PRIVOXY\_x\_ACTIONDIR}{PRIVOXY\_x\_ACTIONDIR}{PRIVOXYxACTIONDIR} {
  Diese Variable gibt den Ort an, an dem die Privoxy-Regelsätze (die Dateien
\emph{default.action} und \emph{user.action}) auf dem Router liegen sollen. Der
angegebene Pfad wird relativ zum Wurzelverzeichnis ausgewertet.
  Diese Variable kann für zwei Dinge verwendet werden:
  \begin{description}
  \item [Verlagern der Standardregeln auf permanenten Speicher]
    Gibt man als Verzeichnis einen Ort ausserhalb der Ram-Disk an, werden die
Standardregelsätze beim erstmaligen Booten dorthin kopiert und dann von diesem
Ort aus genutzt. Änderungen an diesen Regelsätzen überleben dann ein Reboot des
Routers. Zu beachten ist, dass nach einem Update des Privoxy-Paketes diese
Regeln immer noch Verwendung finden und evtl. mit dem aktuellen Paket kommende
neuere Regelsätze ignoriert werden.
  \item [Verwenden eigener Regelsätze]
    fli4l gestattet das Überschreiben der Standardregeln mit nutzerspezifischen
Regeln. Dazu legt man im \emph{config}-Verzeichnis ein Unterverzeichnis an (z.B.
\emph{etc/my\_privoxy}; es darf nicht \emph{etc/privoxy} heissen) und legt dort
die eigenen Regeln ab.
\end{description}
  Das Setzen dieser Variable ist optional.
}

\config{PRIVOXY\_x\_HTTP\_PROXY}{PRIVOXY\_x\_HTTP\_PROXY}{PRIVOXYxHTTPPROXY}

        {Möchte man zusätzlich zu Privoxy einen weiteren HTTP Proxy
        verwenden, der dann z.B. auch Webseiten zwischenspeichert, so
        kann man den hier angeben. Privoxy bedient sich dann dieses
        Proxys. So kann man die Vorteile mehrerer Proxys nutzen. Ein
        Eintrag könnte so aussehen:

\begin{example}
\begin{verbatim}
        PRIVOXY_1_HTTP_PROXY='mein.provider.de:8000'
\end{verbatim}
\end{example}
        Die Angabe ist optional.}

\config{PRIVOXY\_x\_SOCKS\_PROXY}{PRIVOXY\_x\_SOCKS\_PROXY}{PRIVOXYxSOCKSPROXY}

        {Möchte man zusätzlich zu Privoxy einen weiteren SOCKS Proxy
        verwenden, kann man den hier angeben.
        Um die Privatspähre weiter zu erhöhen kann der Datenverkehr
        vom Privoxy beispielsweise durch das Tor Netzwerk geschickt werden.
        Für weitere Details zu Tor lesen Sie in der
        \jump{OPTTOR}{Tor Dokumentation} weiter.
        Ein Eintrag für die Nutzung von Tor könnte so aussehen:

\begin{example}
\begin{verbatim}
        PRIVOXY_x_SOCKS_PROXY='127.0.0.1:9050'
\end{verbatim}
\end{example}
        Die Angabe ist optional.}

\config{PRIVOXY\_x\_TOGGLE}{PRIVOXY\_x\_TOGGLE}{PRIVOXYxTOGGLE}

        {Diese Option aktiviert die Möglichkeit, den Proxy über das
        Webinterface ein- bzw. auszuschalten. Wird Privoxy
        ausgeschaltet, wirkt er als einfacher Forwarding-Proxy und
        ändert den Inhalt der übertragenen Seiten in keinster
        Weise. Es ist zu beachten, dass diese Einstellung für ALLE
        Benutzer des Proxys gilt, d.h. wenn ein Benutzer Privoxy
        abschaltet, ist Privoxy auch für die anderen Nutzer nur noch
        ein Weiterleitungs-Proxy.}

\config{PRIVOXY\_x\_CONFIG}{PRIVOXY\_x\_CONFIG}{PRIVOXYxCONFIG}

        {Diese Option ermöglicht den Benutzern des Proxys die
        interaktive Bearbeitung der Konfiguration über das
        Privoxy-Webinterface. Für weitere Details bitte ich auch hier,
        die Privoxy-Dokumentation zu konsultieren.}

\config{PRIVOXY\_x\_LOGDIR}{PRIVOXY\_x\_LOGDIR}{PRIVOXYxLOGDIR}

        {Mit dieser Option kann ein Logverzeichnis für Privoxy
        angegeben werden.  Dies kann z.B. dann sinnvoll sein, wenn
        Website-Zugriffe der Benutzer geloggt werden sollen. Wird hier
        nichts angegeben (Standard), dann loggt nur die wichtigsten
        Meldungen auf die Konsole und ignoriert
        \var{PRIVOXY\_\-LOGLEVEL}.}

        Hier kann auch 'auto' eingetragen werden, was den Log-Pfad auf
        das System-Verzeichnis für persistente Daten verlegt.
        Bitte darauf achten, daß in diesem Fall \var{FLI4L\_UUID} korrekt
        konfiguriert wird, da mit großen Datenmengen zu rechnen ist und sonst
        /boot oder gar die Ram-Disk gefüllt wird.

\config{PRIVOXY\_x\_LOGLEVEL}{PRIVOXY\_x\_LOGLEVEL}{PRIVOXYxLOGLEVEL}

        {Diese Option gibt an, was Privoxy in die Logdatei schreiben
        soll. Folgende Werte sind möglich, sie können durch Leerstelle
        getrennt angegeben werden, man kann sie aber auch addieren.

        \begin{tabular}[h!]{rl}

    Wert & Was wird geloggt? \\
    \hline
    1    & Jeden Request (GET/POST/CONNECT) ausgeben. \\
    2    & Status jeder Verbindung ausgeben \\
    4    & I/O-Status anzeigen \\
    8    & Header-Parsing anzeigen \\
    16   & \textbf{Alle} Daten loggen \\
    32   & Force-Feature debuggen \\
    64   & reguläre Ausdrücke debuggen \\
    128  & schnelle Weiterleitungen debuggen \\
    256  & GIF De-Animation debuggen \\
    512  & Common Log Format (zur Logfile-Analyse) \\
    1024 & Popup-Kill-Funktion debuggen \\
    2048 & Zugriffe auf den eingebauten Webserver loggen \\
    4096 & Startmeldungen und Warnungen \\
    8192 & Nicht-fatale Fehler \\
        \end{tabular}

        Um eine Logdatei im Common Logfile Format zu erstellen, sollte
        NUR der Wert 512 angegeben werden, da sonst die Logdatei durch
        andere Meldungen ``verschmutzt'' wird und somit nicht mehr
        problemlos ausgewertet werden kann.}

\end{description}

        Privoxy bietet sehr viele Konfigurationsmöglichkeiten. Diese
        können aber aus verständlichen Gründen nicht alle durch die
        Konfigurations-Datei von fli4l abgedeckt werden. Sehr viele
        dieser Optionen können im Webinterface von Privoxy eingestellt
        werden. Genauere Infos über den Aufbau dieser Dateien gibt es
        auf der Privoxy-Homepage. Die Konfigurationsdateien von
        Privoxy liegen unter
        $<$fli4l-Verzeichnis$>$/opt/etc/privoxy/. Es handelt sich
        hierbei um die Orginal-Dateien aus dem Privoxy-Paket,
        allerdings wurden, um Platz zu sparen, alle Kommentare
        entfernt.
