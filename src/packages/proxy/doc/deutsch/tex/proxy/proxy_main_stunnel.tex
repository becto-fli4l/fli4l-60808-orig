% Last Update: $Id$
\subsection{OPT\_STUNNEL - Tunneln von Verbindungen über SSL/TLS}
\configlabel{OPT\_STUNNEL}{OPTSTUNNEL}

Das Programm ``stunnel'' erlaubt es, ansonsten unverschlüsselte Verbindungen
in einem verschlüsselten SSL/TLS-Tunnel zu kapseln. Dies ermöglicht sicheren
Datenaustausch über sonst unsichere Klartext-Protokolle. Auf Grund der
Möglichkeiten des SSL/TLS-Protokolls sind verschiedene Formen der
Client/Server-Validierung möglich.

\subsubsection{Konfiguration}
\begin{description}

\config{OPT\_STUNNEL}{OPT\_STUNNEL}{}

Diese Variable aktiviert die Unterstützung für SSL/TLS-Tunnel.

Standard-Einstellung: \verb+OPT_STUNNEL='no'+

Beispiel: \verb+OPT_STUNNEL='yes'+

\config{STUNNEL\_DEBUG}{STUNNEL\_DEBUG}{STUNNELDEBUG}

Mit dieser Variable kann eingestellt werden, wie sehr ``stunnel'' seine
Arbeitsweise protokolliert. Mögliche Einstellungen umfassen ``yes'' (alles wird
protokolliert), ``no'' (Warnungen und Fehler werden protokolliert) oder ein
Wert zwischen null und sieben, der die maximale Schwere der zu protokollierenden
Meldungen angibt, wobei null für allerdringendste Meldungen und sieben für
Debug-Meldungen steht. Die Einstellung ``yes'' entspricht der maximalen Schwere
sieben, die Einstellung ``no'' entspricht der maximalen Schwere vier.

Standard-Einstellung: \verb+STUNNEL_DEBUG='no'+

Beispiel 1: \verb+STUNNEL_DEBUG='yes'+

Beispiel 2: \verb+STUNNEL_DEBUG='5'+

\config{STUNNEL\_N}{STUNNEL\_N}{STUNNELN}
Diese Variable konfiguriert die Anzahl der Tunnel-Instanzen. Jede Tunnel-Instanz
``horcht'' auf einem Netzwerkport ``A'', verbindet sich bei einer eingehenden
Verbindung mit einem anderen Netzwerkport ``B'' (der durchaus auch auf einer
ganz anderen Maschine liegen kann) und leitet jeglichen Datenverkehr von ``A''
nach ``B'' weiter.  Ob die Daten, die bei ``A'' ankommen, via SSL/TLS
verschlüsselt sind, von ``stunnel'' entschlüsselt und dann nach ``B''
unverschlüsselt weitergeleitet werden oder umgekehrt, entscheidet sich durch
die Variable \jump{STUNNELxCLIENT}{\var{STUNNEL\_x\_CLIENT}}.

Standard-Einstellung: \verb+STUNNEL_N='0'+

Beispiel: \verb+STUNNEL_N='2'+

\config{STUNNEL\_x\_NAME}{STUNNEL\_x\_NAME}{STUNNELxNAME}

Der Name des jeweiligen Tunnels. Er muss unter allen konfigurierten Tunneln
eindeutig sein.

Beispiel: \verb+STUNNEL_1_NAME='imond'+

\config{STUNNEL\_x\_CLIENT}{STUNNEL\_x\_CLIENT}{STUNNELxCLIENT}

Über diese Variable kann eingestellt werden, welche Teile der Kommunikation via
SSL/TLS verschlüsselt werden. Es gibt zwei Möglichkeiten:

\begin{itemize}
\item \emph{Client-Modus:} Der Tunnel erwartet von außen unverschlüsselte
Daten und schickt diese verschlüsselt an das andere Ende des Tunnels. Dies
entspricht der Einstellung \verb+STUNNEL_x_CLIENT='yes'+.
\item \emph{Server-Modus:} Der Tunnel erwartet von außen via SSL/TLS
verschlüsselte Daten und schickt diese entschlüsselt an das andere Ende des
Tunnels. Dies entspricht der Einstellung \verb+STUNNEL_x_CLIENT='no'+.
\end{itemize}

Tunnel im Client-Modus eignen sich also vor allem für Verbindungen, die ``nach
außen'', also z.\,B.\ ins (ungeschützte) Internet gehen, da die Daten vor dem
Verlassen des lokalen Netzwerks verschlüsselt werden. Dafür muss die Gegenstelle
aber natürlich auch einen Server anbieten, der via SSL/TLS verschlüsselte Daten
erwartet. Beispielsweise kann so ein E-Mail-Client im LAN, der nur
unverschlüsseltes POP3 ``spricht'', einen POP3-over-SSL-Dienst im Internet
nutzen.\footnote{vgl. \altlink{http://de.wikipedia.org/wiki/POP3S}}

Tunnel im Server-Modus eignen sich umgekehrt für Verbindungen, die ``von
außen'', also z.\,B.\ aus dem (ungeschützten) Internet kommen, bei denen die
Daten verschlüsselt ankommen. Wenn der eigentliche Dienst auf Server-Seite
jedoch kein SSL/TLS versteht, müssen die Daten vorher entsprechend entschlüsselt
werden. Beispielsweise kann so der Zugriff auf die Web-GUI des fli4l über
via SSL/TLS verschlüsseltes HTTP (HTTPS) erfolgen, indem man auf dem fli4l
einen Tunnel konfiguriert, der via SSL/TLS verschlüsselte HTTP-Daten auf Port
443 empfängt, diese entschlüsselt und dann an den lokalen Web-Server
\texttt{mini\_httpd}, der auf Port 80 horcht, weiterleitet.

Die Konfigurationen für diese Anwendungsfälle werden weiter hinten vorgestellt.

Beispiel: \verb+STUNNEL_1_CLIENT='yes'+

\config{STUNNEL\_x\_ACCEPT}{STUNNEL\_x\_ACCEPT}{STUNNELxACCEPT}

Hiermit wird festgelegt, auf welchem Port (und an welcher Adresse) der Tunnel
auf eingehende Verbindungen ``lauscht''. Es gibt prinzipiell zwei Möglichkeiten:

\begin{itemize}
\item Der Tunnel soll an \emph{allen} Adressen (auf allen Schnittstellen)
lauschen. Hierfür muss ``any'' verwendet werden.
\item Der Tunnel soll nur an bestimmten Adressen lauschen. Dies wird mit Hilfe
einer entsprechenden Referenz auf das konfigurierte IP-Subnetz eingestellt,
beispielsweise \var{IP\_NET\_1\_IPADDR} (für IPv4) oder
\var{IPV6\_NET\_2\_IPADDR} (für IPv6).
\end{itemize}

Des Weiteren \emph{muss} hinter die Adresse der Port stehen, wobei der Port von
der Adresse mit Hilfe eines Doppelpunktes (``:'') abgetrennt ist.

Beispiel 1: \verb+STUNNEL_1_ACCEPT='any:443'+

Beispiel 2: \verb+STUNNEL_1_ACCEPT='IP_NET_1_IPADDR:443'+

Beispiel 3: \verb+STUNNEL_1_ACCEPT='IPV6_NET_2_IPADDR:443'+

Zu bedenken ist, dass die Verwendung von \var{IP\_NET\_x\_IPADDR} bzw.
\var{IPV6\_NET\_x\_IPADDR} das Layer-3-Protokoll (IPv4 oder IPv6) festlegt;
diese Wahl \emph{muss} zu den Belegungen der Variablen
\var{STUNNEL\_x\_ACCEPT\_IPV4} und \var{STUNNEL\_x\_ACCEPT\_IPV6} passen. Sie
können also nicht IPv6 für den Tunnel mit Hilfe von
\verb+STUNNEL_1_ACCEPT_IPV6='no'+ deaktivieren und dann mit Hilfe von
\verb+STUNNEL_1_ACCEPT='IPV6_NET_2_IPADDR:443'+ an einer IPv6-Adresse lauschen;
dies gilt analog für die umgekehrte Konstellation
(\verb+STUNNEL_1_ACCEPT_IPV4='no'+ und die Verwendung von
\var{IP\_NET\_x\_IPADDR}). Des Weiteren hängt die Bedeutung von ``any'' von den
aktivierten Layer-3-Protokollen (IPv4 oder IPv6) ab: Es wird natürlich nur auf
Adressen gelauscht, die zu den via \var{STUNNEL\_x\_ACCEPT\_IPV4} und
\var{STUNNEL\_x\_ACCEPT\_IPV6} aktivierten Layer-3-Protokollen gehören.

\config{STUNNEL\_x\_ACCEPT\_IPV4}{STUNNEL\_x\_ACCEPT\_IPV4}{STUNNELxACCEPTIPV4}

Mit dieser Variable wird eingestellt, ob das IPv4-Protokoll für
\emph{eingehende} Verbindungen des Tunnels genutzt werden soll. Typischerweise
ist dies der Fall, und diese Variable sollte den Wert ``yes'' enthalten. Die
Belegung mit ``no'' stellt sicher, dass der Tunnel nur eingehende
IPv6-Verbindungen akzeptiert. Dies erfordert jedoch eine valide
IPv6-Konfiguration (siehe hierzu die Dokumentation des ipv6-Pakets).

Standard-Einstellung: \verb+STUNNEL_x_ACCEPT_IPV4='yes'+

Beispiel: \verb+STUNNEL_1_ACCEPT_IPV4='no'+

\config{STUNNEL\_x\_ACCEPT\_IPV6}{STUNNEL\_x\_ACCEPT\_IPV6}{STUNNELxACCEPTIPV6}

Analog zu \var{STUNNEL\_x\_ACCEPT\_IPV4} wird mit dieser Variable eingestellt,
ob das IPv6-Protokoll für eingehende Verbindungen des Tunnels genutzt werden
soll. Typischerweise ist das der Fall, wenn Sie die generelle Nutzung des
IPv6-Protokolls mit Hilfe von \verb+OPT_IPV6='yes'+ aktiviert haben. Die
Belegung mit ``no'' stellt sicher, dass der Tunnel nur eingehende
IPv4-Verbindungen akzeptiert.

Standard-Einstellung: \verb+STUNNEL_x_ACCEPT_IPV6=<Wert von OPT_IPV6>+

Beispiel: \verb+STUNNEL_1_ACCEPT_IPV6='no'+

\config{STUNNEL\_x\_CONNECT}{STUNNEL\_x\_CONNECT}{STUNNELxCONNECT}

Hiermit wird festgelegt, welches Ziel der SSL/TLS-Tunnel hat. Es gibt
prinzipiell drei Möglichkeiten, wobei bei jeder der drei Möglichkeiten noch ein
mit ``:'' abgetrennter Port angehängt werden muss:

\begin{itemize}
\item Eine numerische IPv4- oder IPv6-Adresse.

Beispiel 1: \verb+STUNNEL_1_CONNECT='192.0.2.2:443'+

\item Der DNS-Name von einem internen Host.

Beispiel 2: \verb+STUNNEL_1_CONNECT='@webserver:443'+

\item Der DNS-Name von einem externen Host.

Beispiel 3: \verb+STUNNEL_1_CONNECT='@www.example.com:443'+
\end{itemize}

Wird ein interner Host eingetragen, der sowohl eine IPv4- als auch eine
IPv6-Adresse besitzt, dann wird die IPv4-Adresse bevorzugt. Wird ein externer
Host eingetragen, der sowohl eine IPv4- als auch eine IPv6-Adresse besitzt,
dann hängt das verwendete Layer-3-Protokoll davon ab, welche Adresse als erstes
vom DNS-Resolver zurückgegeben wird.

\config{STUNNEL\_x\_OUTGOING\_IP}{STUNNEL\_x\_OUTGOING\_IP}{STUNNELxOUTGOINGIP}

Mit dieser optionalen Variable kann die \emph{lokale} Adresse für die
\emph{ausgehende} Verbindung des Tunnels angegeben werden. Dies ist nur dann
sinnvoll, wenn das Ziel des Tunnels über mehrere Schnittstellen (Routen)
erreicht werden kann, also wenn man z.\,B.\ zwei Internet-Anbindungen nutzt.
Normalerweise muss diese Variable nicht gesetzt werden.

Beispiel: \verb+STUNNEL_1_OUTGOING_IP='IP_NET_1_IPADDR'+

\config{STUNNEL\_x\_DELAY\_DNS}{STUNNEL\_x\_DELAY\_DNS}{STUNNELxDELAYDNS}

Wird diese optionale Variable auf ``yes'' gesetzt, so wird ein in
\var{STUNNEL\_x\_CONNECT} verwendeter externer DNS-Name erst dann in eine
Adresse umgewandelt, wenn die \emph{ausgehende} Tunnelverbindung aufgebaut wird,
also wenn der erste Client sich lokal mit der eingehenden Seite des Tunnels
verbunden hat. Dies ist dann nützlich, wenn das Ziel des Tunnels ein Rechner
ist, der nur über einen dynamischen DNS-Namen erreicht werden kann und die
Adresse hinter diesem Namen häufiger wechselt, oder auch wenn eine aktive
Einwahl bereits beim Starten von ``stunnel'' verhindert werden soll.

Standard-Einstellung: \verb+STUNNEL_x_DELAY_DNS='no'+

Beispiel: \verb+STUNNEL_1_DELAY_DNS='yes'+

\config{STUNNEL\_x\_CERT\_FILE}{STUNNEL\_x\_CERT\_FILE}{STUNNELxCERTFILE}

Diese Variable enthält den Dateinamen des Zertifikats, das für den
Tunnel verwendet werden soll. Für Server-Tunnel (\verb+STUNNEL_x_CLIENT='no'+)
ist dies das Server-Zertifikat, das vom Client ggfs. gegen eine ``Certificate
Authority'' (CA) validiert wird. Für Client-Tunnel
(\verb+STUNNEL_x_CLIENT='yes'+) ist dies ein (in der Regel optionales)
Client-Zertifikat, das vom Server ggfs. gegen eine CA validiert wird.

Das Zertifikat muss im so genannten PEM-Format vorliegen und muss unterhalb von
\texttt{<config-Verzeichnis>/etc/ssl/stunnel/} abgespeichert werden. Nur
der Dateiname muss in dieser Variable gespeichert werden, nicht der Pfad.

Für einen Server-Tunnel ist ein Zertifikat zwingend erforderlich!

Beispiel: \verb+STUNNEL_1_CERT_FILE='myserver.crt'+

\config{STUNNEL\_x\_CERT\_CA\_FILE}{STUNNEL\_x\_CERT\_CA\_FILE}{STUNNELxCERTCAFILE}

Diese Variable enthält den Dateinamen des CA-Zertifikats, das für die
Validierung des Zertifikats der Gegenstelle verwendet werden soll.
Typischerweise validieren Clients das Zertifikat des Servers, andersherum ist
dies jedoch genauso möglich. Einzelheiten zur Validierung lesen Sie bitte in
der Beschreibung der Variable \jump{STUNNELxCERTVERIFY}
{\var{STUNNEL\_x\_CERT\_VERIFY}} nach.

Das Zertifikat muss im so genannten PEM-Format vorliegen und muss unterhalb von
\texttt{<config-Verzeichnis>/etc/ssl/stunnel/} abgespeichert werden. Nur
der Dateiname muss in dieser Variable gespeichert werden, nicht der Pfad.

Beispiel: \verb+STUNNEL_1_CERT_CA_FILE='myca.crt'+

\config{STUNNEL\_x\_CERT\_VERIFY}{STUNNEL\_x\_CERT\_VERIFY}{STUNNELxCERTVERIFY}

Diese Variable steuert die Validierung des Zertifikats der Gegenstelle. Es gibt
fünf Möglichkeiten:

\begin{itemize}
\item \emph{none:} Das Zertifikat der Gegenstelle wird überhaupt nicht
validiert. In diesem Falle kann die Variable \var{STUNNEL\_x\_CERT\_CA\_FILE}
leer bleiben.

\item \emph{optional:} Stellt die Gegenstelle ein Zertifikat zur Verfügung, so
wird es gegen das CA-Zertifikat geprüft, das mit Hilfe der Variable
\var{STUNNEL\_x\_CERT\_CA\_FILE} konfiguriert wird. Stellt die Gegenstelle
\emph{kein} Zertifikat zur Verfügung, so ist dies kein Fehler und die
Verbindung wird dennoch akzeptiert. Diese Einstellung ist nur sinnvoll für
Server-Tunnel, weil Client-Tunnel \emph{immer} ein Zertifikat vom Server
erhalten.

\item \emph{onlyca:} Das Zertifikat der Gegenstelle wird gegen das
CA-Zertifikat geprüft, das mit Hilfe der Variable
\var{STUNNEL\_x\_CERT\_CA\_FILE} konfiguriert wird. Sendet die Gegenstelle kein
Zertifikat oder passt es nicht zur konfigurierten CA, wird die Verbindung
abgewiesen. Dies ist nützlich, wenn eine eigene CA verwendet wird, da man dann
alle potentiellen Gegenstellen kennt.

\item \emph{onlycert:} Das Zertifikat der Gegenstelle wird mit dem Zertifikat
verglichen, das mit Hilfe der Variable \var{STUNNEL\_x\_CERT\_CA\_FILE}
konfiguriert wird. Es wird \emph{nicht} gegen ein CA-Zertifikat geprüft,
sondern es wird sichergestellt, dass die Gegenstelle \emph{genau} das passende
(Server- oder Client-)Zertifikat sendet. Die Datei, die mit Hilfe der Variable
\var{STUNNEL\_x\_CERT\_CA\_FILE} referenziert wird, enthält in diesem Fall also
kein CA-, sondern ein Host-Zertifikat. Diese Einstellung stellt sicher, dass
sich wirklich nur eine bestimmte und bekannte Gegenstelle verbinden darf
(Server-Tunnel) bzw. eine Verbindung nur zu einer bekannten Gegenstelle
(Client-Tunnel) aufgebaut wird. Dies ist für Peer-to-Peer-Verbindungen zwischen
Hosts nützlich, die man beide unter Kontrolle hat, für die man aber keine
eigene CA verwendet.

\item \emph{both:} Das Zertifikat der Gegenstelle wird mit dem Zertifikat
verglichen, das mit Hilfe der Variable \var{STUNNEL\_x\_CERT\_CA\_FILE}
konfiguriert wird, \emph{und} es wird zusätzlich sichergestellt, dass es zu
einem CA-Zertifikat passt. Die Datei, die mit Hilfe der Variable
\var{STUNNEL\_x\_CERT\_CA\_FILE} referenziert wird, enthält in diesem Fall also
\emph{sowohl} ein CA- \emph{als auch} ein Host-Zertifikat. Es handelt sich also
um eine Kombination der Einstellungen \emph{onlycert} und \emph{onlyca}. Im
Vergleich zur Einstellung \emph{onlycert} werden somit Verbindungen abgelehnt,
falls das CA-Zertifikat abgelaufen ist (auch wenn das Zertifikat der
Gegenstelle ansonsten passt).

\end{itemize}

Standard-Einstellung: \verb+STUNNEL_x_CERT_VERIFY='none'+

Beispiel: \verb+STUNNEL_1_CERT_VERIFY='onlyca'+

\end{description}

\subsubsection{Anwendungsbeispiel 1: Zugang zur fli4l-WebGUI via SSL/TLS}

Mit diesem Beispiel wird die fli4l-WebGUI um einen SSL/TLS-Zugang erweitert.

\begin{example}
\begin{verbatim}
OPT_STUNNEL='yes'
STUNNEL_N='1'

STUNNEL_1_NAME='http'
STUNNEL_1_CLIENT='no'
STUNNEL_1_ACCEPT='any:443'
STUNNEL_1_ACCEPT_IPV4='yes'
STUNNEL_1_ACCEPT_IPV6='yes'
STUNNEL_1_CONNECT='127.0.0.1:80'
STUNNEL_1_CERT_FILE='server.pem'
STUNNEL_1_CERT_CA_FILE='ca.pem'
STUNNEL_1_CERT_VERIFY='none'
\end{verbatim}
\end{example}

\subsubsection{Anwendungsbeispiel 2: Via SSL/TLS gesicherte Kontrolle von zwei
entfernten fli4l-Routern via imonc}

Mit diesem Beispiel werden die bekannten Schwachstellen des
imonc/imond-Protokolls (Senden von Passwörtern im Klartext) für WAN-Verbindungen
umgangen. (Die LAN-Verbindung zum Tunnel kann natürlich weiterhin abgehört
werden!)

Konfiguration des lokalen fli4l im LAN (Client-Tunnel):
\begin{example}
\begin{verbatim}
OPT_STUNNEL='yes'
STUNNEL_N='2'

STUNNEL_1_NAME='remote-imond1'
STUNNEL_1_CLIENT='yes'
STUNNEL_1_ACCEPT='any:50000'
STUNNEL_1_ACCEPT_IPV4='yes'
STUNNEL_1_ACCEPT_IPV6='yes'
STUNNEL_1_CONNECT='@remote1:50000'
STUNNEL_1_CERT_FILE='client.pem'
STUNNEL_1_CERT_CA_FILE='ca+server1.pem'
STUNNEL_1_CERT_VERIFY='both'

STUNNEL_2_NAME='remote-imond2'
STUNNEL_2_CLIENT='yes'
STUNNEL_2_ACCEPT='any:50001'
STUNNEL_2_ACCEPT_IPV4='yes'
STUNNEL_2_ACCEPT_IPV6='yes'
STUNNEL_2_CONNECT='@remote2:50000'
STUNNEL_2_CERT_FILE='client.pem'
STUNNEL_2_CERT_CA_FILE='ca+server2.pem'
STUNNEL_2_CERT_VERIFY='both'
\end{verbatim}
\end{example}

Konfiguration des ersten entfernten fli4l (Server-Tunnel):
\begin{example}
\begin{verbatim}
OPT_STUNNEL='yes'
STUNNEL_N='1'

STUNNEL_1_NAME='remote-imond'
STUNNEL_1_CLIENT='no'
STUNNEL_1_ACCEPT='any:50000'
STUNNEL_1_ACCEPT_IPV4='yes'
STUNNEL_1_ACCEPT_IPV6='yes'
STUNNEL_1_CONNECT='127.0.0.1:5000'
STUNNEL_1_CERT_FILE='server1.pem'
STUNNEL_1_CERT_CA_FILE='ca+client.pem'
STUNNEL_1_CERT_VERIFY='both'
\end{verbatim}
\end{example}

Konfiguration des zweiten entfernten fli4l (Server-Tunnel):
\begin{example}
\begin{verbatim}
OPT_STUNNEL='yes'
STUNNEL_N='1'

STUNNEL_1_NAME='remote-imond'
STUNNEL_1_CLIENT='no'
STUNNEL_1_ACCEPT='any:50000'
STUNNEL_1_ACCEPT_IPV4='yes'
STUNNEL_1_ACCEPT_IPV6='yes'
STUNNEL_1_CONNECT='127.0.0.1:5000'
STUNNEL_1_CERT_FILE='server2.pem'
STUNNEL_1_CERT_CA_FILE='ca+client.pem'
STUNNEL_1_CERT_VERIFY='both'
\end{verbatim}
\end{example}

Eine Verbindung zu dem entfernten ``imond'' wird aufgebaut, indem eine
Verbindung zum lokalen fli4l auf Port 50000 (erster entfernter fli4l) bzw.
50001 (zweiter entfernter fli4l) initiiert wird. Dieser fli4l verbindet sich
dann via SSL/TLS-Tunnel mit dem jeweiligen entfernten fli4l, der wiederum
die Daten über eine dritte (Host-interne) Verbindung letztlich an den
entfernten ``imond'' weiterleitet. Die Einstellungen für die Validierung stellen
sicher, dass jeder fli4l jeweils nur den anderen fli4l als Verbindungspartner
akzeptiert.
