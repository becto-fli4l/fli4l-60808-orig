% Last Update: $Id$
\subsection{OPT\_TRANSPROXY (EXPERIMENTELL) - Transparenter HTTP-Proxy}
\configlabel{OPT\_TRANSPROXY}{OPTTRANSPROXY}

Transproxy ist ein ,,transparenter'' Proxy, also ein Programm, dass es
ermöglicht, alle HTTP-Abfragen, die über den Router laufen, abzufangen und an
einen normalen HTTP-Proxy, z.B. Privoxy, weiterzuleiten. Um diese
Funktionalität zu erreichen, muss der Paketfilter HTTP-Anfragen, die eigentlich
ins Internet gehen sollen, an Transproxy weiterreichen, welcher diese weiter
aufbereitet und an den anderen HTTP-Proxy weitergibt. iptables bietet zur
Unterstützung dieser Funktion die Aktion ,,REDIRECT'':

\begin{verbatim}
        PF_PREROUTING_1='tmpl:http IP_NET_1 REDIRECT:8081'
\end{verbatim}

Diese Regel würde alle HTTP-Pakete aus dem ersten definierten Netz
(normalerweise das interne LAN) an Transproxy auf Port 8081 weiterleiten.

\begin{description}

\config{TRANSPROXY\_LISTEN\_N}{TRANSPROXY\_LISTEN\_N}{TRANSPROXYLISTENN}
\config{TRANSPROXY\_LISTEN\_x}{TRANSPROXY\_LISTEN\_x}{TRANSPROXYLISTENx}

        {Hier werden die IP-Adressen oder symbolischen Namen inklusive
        der Portnummer der Interfaces angegeben, auf denen Transproxy auf
        Verbindungen von Clients horchen soll. Hier müssen alle Interfaces
        angegeben werden, für die im Paketfilter Pakete auf Transproxy
        umgelenkt werden. Mit der Vorgabeeinstellung \var{any:8081} hört
        Transproxy auf allen Interfaces.}

\config{TRANSPROXY\_TARGET\_IP}{TRANSPROXY\_TARGET\_IP}{TRANSPROXYTARGETIP}
\config{TRANSPROXY\_TARGET\_PORT}{TRANSPROXY\_TARGET\_PORT}{TRANSPROXYTARGETPORT
}

        {Mit diesen Optionen wird festgelegt, an welchen Dienst eingehende
        HTTP-Anfragen umgeleitet werden. Dies kann ein beliebiger
        Standard-HTTP-Proxy (Squid, Privoxy, Apache, etc.) auf einem beliebigen
        anderen Rechner (oder auch auf fli4l selbst) sein. Hier ist darauf zu
achten,
        dass der Proxy sich nicht im Bereich der durch den Paketfilter
        umgeleiteten HTTP-Anfragen befindet, da sonst eine Schleife entsteht.}

\config{TRANSPROXY\_ALLOW\_N}{TRANSPROXY\_ALLOW\_N}{TRANSPROXYALLOWN}
\config{TRANSPROXY\_ALLOW\_x}{TRANSPROXY\_ALLOW\_x}{TRANSPROXYALLOWx}

        {Die Liste der Netze und/oder IP-Adressen für die der
        Paketfilter geöffnet wird. Dies sollte die gleichen Netze abdecken, die
        auch im Paketfilter umgeleitet werden. Werden hier keine Bereiche
        angegeben, müssen die Angaben von Hand in der Paketfilter-Konfiguation
        vorgenommen werden.}

\end{description}
