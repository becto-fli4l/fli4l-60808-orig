% Last Update: $Id$
\subsection{OPT\_SS5 - Ein Socks4/5 Proxy}
\configlabel{OPT\_SS5}{OPTSS5}

Für einige Programme wird ein Socks-Proxy benötigt. SS5 stellt diese
Funktionalität bereit.

\altlink{http://ss5.sourceforge.net/}

\begin{description}

\config{SS5\_LISTEN\_N}{SS5\_LISTEN\_N}{SS5LISTENN}

\config{SS5\_LISTEN\_x}{SS5\_LISTEN\_x}{SS5LISTENx}

        {Hier werden die IP-Adressen oder symbolischen Namen inklusive
        der Portnummer der Interfaces angegeben, auf denen SS5 auf
        Verbindungen von Clients horchen soll. Es ist eine gute Idee,
        hier nur die Adressen der Interfaces anzugeben, denen man
        vertraut, da alle Rechner vollen Zugriff auf SS5 haben
        (und auf den eventuell aktivierten Konfigurations-Editor). In
        der Regel ist die Vorgabe \var{IP\_NET\_1\_IPADDR:8050}
        sinnvoll.

        Auf hier angegebenen Adressen lauscht SS5 und bietet seine
        Dienste an.  8050 ist der Standard-Port. Die Angabe hier muss
        man dann bei der Proxy-Konfiguration des jeweils verwendeten
        Programms benutzen.

        Als Proxy beim jeweiligen Programm muss der fli4l-Rechner
        angegeben werden, also das, was man bei HOSTNAME='fli4l'
        angegeben hat bzw.  dessen IP (z.B 192.168.6.1), die man bei
        \var{HOST\_\-x\_\-IP}='192.168.6.1' angegeben hat. Zusammen
        mit dieser Port-Angabe hier hat man dann alle nötigen Daten,
        um sein Programm für die Nutzung von SS5 zu
        konfigurieren.}

\config{SS5\_ALLOW\_N}{SS5\_ALLOW\_N}{SS5ALLOWN}

        {Gibt die Anzahl der Listeneinträge an.}

\config{SS5\_ALLOW\_x}{SS5\_ALLOW\_x}{SS5ALLOWx}

        Die Liste der Netze und/oder IP-Adressen für die der
        Paketfilter geöffnet wird. Sinnvoll ist hier auch wieder die
        Vorgabe \var{IP\_NET\_1}.

\end{description}


