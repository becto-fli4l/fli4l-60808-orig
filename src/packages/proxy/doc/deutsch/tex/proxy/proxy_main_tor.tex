% Last Update: $Id$
\subsection{OPT\_TOR - Ein anonymes Kommunikationssystem für das Internet}
\configlabel{OPT\_TOR}{OPTTOR}

Tor ist ein Werkzeug für eine Vielzahl von Organisationen und
Menschen, die ihren Schutz und ihre Sicherheit im Internet verbessern
wollen. Die Nutzung von Tor hilft Ihnen, das Browsen und
Veröffentlichen im Web, Instantmessaging, IRC, SSH und anderen TCP
basierende Anwendungen zu anonymisieren. Weiterhin bietet Tor eine
Plattform auf der Softwareentwickler neue Anwendungen schaffen können
die zu mehr Anonymität, Sicherheit und zum Schutz der Privatsphäre
beitragen.

\altlink{https://www.torproject.org/index.html.de}

\begin{description}

\config{TOR\_LISTEN\_N}{TOR\_LISTEN\_N}{TORLISTENN}

\config{TOR\_LISTEN\_x}{TOR\_LISTEN\_x}{TORLISTENx}

        {Hier werden die IP-Adressen oder symbolischen Namen inklusive
        der Portnummer der Interfaces angegeben, auf denen Tor auf
        Verbindungen von Clients horchen soll. Es ist eine gute Idee,
        hier nur die Adressen der Interfaces anzugeben, denen man
        vertraut, da alle Rechner vollen Zugriff auf Tor haben
        (und auf den eventuell aktivierten Konfigurations-Editor). In
        der Regel ist die Vorgabe \var{IP\_NET\_1\_IPADDR:9050}
        sinnvoll.

        Auf hier angegebenen Adressen lauscht Tor und bietet seine
        Dienste an.  9050 ist der Standard-Port. Die Angabe hier muss
        man dann bei der Proxy-Konfiguration des jeweils verwendeten
        Programms benutzen.

        Als Proxy beim jeweiligen Programm muss der fli4l-Rechner
        angegeben werden, also das, was man bei HOSTNAME='fli4l'
        angegeben hat bzw.  dessen IP (z.B 192.168.6.1), die man bei
        \var{HOST\_\-x\_\-IP}='192.168.6.1' angegeben hat. Zusammen
        mit dieser Port-Angabe hier hat man dann alle nötigen Daten,
        um sein Programm für die Nutzung von Tor zu
        konfigurieren.}

\config{TOR\_ALLOW\_N}{TOR\_ALLOW\_N}{TORALLOWN}

        {Gibt die Anzahl der Listeneinträge an.}

\config{TOR\_ALLOW\_x}{TOR\_ALLOW\_x}{TORALLOWx}

        Die Liste der Netze und/oder IP-Adressen für die der
        Paketfilter geöffnet wird. Sinnvoll ist hier auch wieder die
        Vorgabe \var{IP\_NET\_1}.

\config{TOR\_CONTROL\_PORT}{TOR\_CONTROL\_PORT}{TORCONTROLPORT}

        Hier kann angegeben werden auf welchem TCP Port Tor einen
        Kontrollzugang über das Tor Control Protocol öffnen soll.
        Die Angabe ist optional. Wird nichts angegeben wird diese
        Funktion deaktiviert.

\config{TOR\_CONTROL\_PASSWORD}{TOR\_CONTROL\_PASSWORD}{TORCONTROLPASSWORD}

        Hier kann ein Passwort für den Kontrollzugang angegeben werden.

\config{TOR\_DATA\_DIR}{TOR\_DATA\_DIR}{TORDATADIR}

        Diese Angabe ist optional. Wird nichts angegeben, wird der
Standardordner
        /etc/tor verwendet

\config{TOR\_HTTP\_PROXY}{TOR\_HTTP\_PROXY}{TORHTTPPROXY}

        {Soll Tor die Anfragen an einen HTTP-Proxy weiterleiten,
        kann man den hier angeben. Tor bedient sich dann dieses
        Proxys. So kann man die Vorteile mehrerer Proxys nutzen. Ein
        Eintrag könnte so aussehen:

\begin{example}
\begin{verbatim}
        TOR_HTTP_PROXY='mein.provider.de:8000'
\end{verbatim}
\end{example}
        Die Angabe ist optional.}

\config{TOR\_HTTP\_PROXY\_AUTH}{TOR\_HTTP\_PROXY\_AUTH}{TORHTTPPROXYAUTH}

        Eine eventuell notwendige Authentifizierung für den Proxy kann
        hier in der Form Benutzername:Passwort eingetragen werden.

\config{TOR\_HTTPS\_PROXY}{TOR\_HTTPS\_PROXY}{TORHTTPSPROXY}

        Hier kann ein HTTPS-Proxy eingetragen werden. Siehe dazu auch
        \smalljump{TORHTTPPROXY}{\var{TOR\_HTTP\_PROXY}}.

\config{TOR\_HTTPS\_PROXY\_AUTH}{TOR\_HTTPS\_PROXY\_AUTH}{TORHTTPSPROXYAUTH}

        Siehe dazu \smalljump{TORHTTPPROXYAUTH}{\var{TOR\_HTTP\_PROXY\_AUTH}}.

\config{TOR\_LOGLEVEL}{TOR\_LOGLEVEL}{TORLOGLEVEL}

        {Diese Option gibt an, was Tor in die Logdatei schreiben
        soll. Folgende Werte sind möglich:
        debug, info, notice, warn oder err
        Die Werte debug und info sollten aus Sicherheitsgründen möglichst
        nicht verwendet werden.}

\config{TOR\_LOGFILE}{TOR\_LOGFILE}{TORLOGFILE}

        Falls Tor statt ins syslog in eine Datei loggen soll,
        kann diese hier angegeben werden.

        Hier kann auch 'auto' eingetragen werden, was den Log-Pfad auf
        das System-Verzeichnis für persistente Daten verlegt.
        Bitte darauf achten, daß in diesem Fall \var{FLI4L\_UUID} korrekt
        konfiguriert wird, da mit großen Datenmengen zu rechnen ist und sonst
        /boot oder gar die Ram-Disk gefüllt wird.

\end{description}
