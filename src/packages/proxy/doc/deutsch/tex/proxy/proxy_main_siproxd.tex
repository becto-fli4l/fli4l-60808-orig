% Last Update: $Id$
\subsection{OPT\_SIPROXD (experimentell) -- Proxy für Session Initiation Protocol}
\configlabel{OPT\_SIPROXD}{OPTSIPROXD}

Möchte man mehrere SIP-Anwendungen (egal ob Ekiga, x-lite oder Hardware
SIP-Telefone) hinter einem Router betreiben, so kann es vorkommen,
dass Netzwerk-Ports speziell weitergeleitet werden müssen, da sonst
die Verbindungen nicht so funktionieren, wie sie sollten.

Um dies zu vermeiden, kann man einen speziellen SIP Proxy-Server nutzen.
Es werden derzeit (fli4l V4.0.0) mehrere solche Proxys evaluiert.
Sollte jemand über eine Empfehlung verfügen so darf er sich gerne
an das Team wenden!

% TODO  Den Einleitungstext an die tatsächlichen Gegebenheiten anpassen
%       (Siproxd implementiert, Kamailio nicht).

\subsubsection{Konfiguration}

\begin{description}

\config{SIPROXD\_N}{SIPROXD\_N}{SIPROXDN}

Gibt die Anzahl der Siproxd-Instanzen an, die gestartet werden sollen.
Mehrere Instanzen können nötig sein, wenn man SIP-Clients aus
verschiedenen hinter dem fli4l-Router liegenden Subnetzen benutzen will.

Standard-Einstellung: \verb+SIPROXD_N='0'+

Beispiel: \verb+SIPROXD_N='2'+

\config{SIPROXD\_x\_DEV\_IN}{SIPROXD\_x\_DEV\_IN}{SIPROXDxDEVIN}

Hier ist die Netzwerk-Schnittstelle zum lokalen Netz bzw. LAN hin
anzugeben, vom dem aus sich die VoIP-Clients mit dem SIP-Server
verbinden.  Erlaubt sind sowohl Namen von Netzwerkinterfaces (inklusive
VLAN-Schnittstellen) als auch fli4l-Variablen wie bspw.
\verb+IP_NET_1_DEV+.

Beispiel: \verb+SIPROXD_1_DEV_IN='br0'+

\config{SIPROXD\_x\_DEV\_OUT}{SIPROXD\_x\_DEV\_OUT}{SIPROXDxDEVOUT}

Die Netzwerkschnittstelle zum Internet hin, über die der SIP-Server
erreicht wird, also beispielsweise das selbe Interface wie der default
circuit.

Beispiel: \verb+SIPROXD_1_DEV_OUT='eth0.23'+

\config{SIPROXD\_x\_TRANSPARENT}{SIPROXD\_x\_TRANSPARENT}{SIPROXDxTRANSPARENT}

Ist diese Option gesetzt, werden dem Paketfilter automatisch Regeln
hinzugefügt, die den Betrieb als transparenten Proxy erlauben.  Das
heißt die SIP-Client Software oder das VoIP-Gerät müssen nicht explizit
die Konfiguration eines Proxy zulassen, sondern funktionieren direkt so,
als wären sie ohne NAT mit dem SIP-Server verbunden.  Ermöglicht
beispielsweise die Verwendung der FRITZ!App Fon, wenn der fli4l-Router
hinter einer FRITZ!Box betrieben wird.

Beispiel: \verb+SIPROXD_1_TRANSPARENT='yes'+

Diese Variable ist optional. Ist sie nicht gesetzt, bleibt die Funktion
deaktiviert.

\config{SIPROXD\_x\_ALLOW\_REG\_N}{SIPROXD\_x\_ALLOW\_REG\_N}{SIPROXDxALLOWREGN}

Anzahl der lokalen IP-Netze, aus denen Client-Registrierung erlaubt
werden soll.

Beispiel: \verb+SIPROXD_1_ALLOW_REG_N='2'+

\config{SIPROXD\_x\_ALLOW\_REG\_x}{SIPROXD\_x\_ALLOW\_REG\_x}{SIPROXDxALLOWREGx}

Ein lokales Netz, aus dem der Zugriff auf den Proxy für VoIP-Clients
erlaubt wird.  Im Gegensatz zu \verb+SIPROXD_x_DEV_IN+ werden hier
IP-Netze angegeben, die an das lokale Interface gebunden sind bzw. über
dieses erreicht werden können.  Erlaubt sind neben direkten Angaben wie
\verb+192.168.42.0/24+ auch Variablen der Form \verb+IP_NET_1+.

Beispiel: \verb+SIPROXD_1_ALLOW_REG_1='IP_NET_1'+

\config{SIPROXD\_x\_SIP\_PORT}{SIPROXD\_x\_SIP\_PORT}{SIPROXDxSIPPORT}

Port auf dem der SIP-Proxy auf eingehende SIP-Nachrichten lauscht.  Der
Standard-Port muss hier nur geändert werden, wenn mehr als eine Instanz
des siproxd betrieben wird.

Beispiel: \verb+SIPROXD_1_SIP_PORT='5060'+

\config{SIPROXD\_x\_RTP\_PORT\_MIN}{SIPROXD\_x\_RTP\_PORT\_MIN}{SIPROXDxRTPPORTMIN}

Unteres Ende des Port-Bereichs, der für eingehende
RTP-Stream-Verbindungen benutzt wird.

Beispiel: \verb+SIPROXD_1_RTP_PORT_MIN='7070'+

\config{SIPROXD\_x\_RTP\_PORT\_MAX}{SIPROXD\_x\_RTP\_PORT\_MAX}{SIPROXDxRTPPORTMAX}

Oberes Ende des Port-Bereichs für eingehenden RTP-Verkehr.

Beispiel: \verb+SIPROXD_1_RTP_PORT_MAX='7089'+

\config{SIPROXD\_x\_USER\_N}{SIPROXD\_x\_USER\_N}{SIPROXDxUSERN}

Ist dieser Wert größer null, müssen sich alle Nutzer gegenüber dem
siproxd mit Nutzername und Passwort authentifizieren.

Beispiel: \verb+SIPROXD_1_USER_N='1'+

Diese Variable ist optional. Ist sie nicht gesetzt, wird der Wert '0'
angenommen.

\config{SIPROXD\_x\_USER\_x\_NAME}{SIPROXD\_x\_USER\_x\_NAME}{SIPROXDxUSERxNAME}

Nutzername des jeweiligen Nutzerkontos, das sich am siproxd
authentifizieren muss.

Beispiel: \verb+SIPROXD_1_USER_1_NAME='alice'+

Diese Variable ist optional.

\config{SIPROXD\_x\_USER\_x\_PASS}{SIPROXD\_x\_USER\_x\_PASS}{SIPROXDxUSERxPASS}

Passwort des jeweiligen Nutzerkontos, das sich am siproxd
authentifizieren muss.

Beispiel: \verb+SIPROXD_1_USER_1_PASSWD='batteryhorsestaple'+

Diese Variable ist optional.

\end{description}

\subsubsection{Beispiel}

\begin{example}
\begin{verbatim}
OPT_SIPROXD='yes'
…
\end{verbatim}
\end{example}
