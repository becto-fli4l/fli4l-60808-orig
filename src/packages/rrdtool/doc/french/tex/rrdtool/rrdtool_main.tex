% Do not remove the next line
% Synchronized to r32928

\section {RRDTOOL~- Capture de données et affichage graphique}

\subsection {Description}

Le paquetage RRDTOOL recueille les données du système en utilisant le démon
"collectd" et les stocke dans une base de données RRD. Les données sont ensuite
récupérées et traitées à l'aide de rrdtool, ils produiront des graphiques dans
l'interface web du routeur fli4l.
\\
Vous pouvez par exemple, enregistrer et afficher les données suivantes~:
 \begin{itemize}
  \item Voir l'état du système avec
  \begin{itemize}
   \item Utilisation du CPU
   \item Charge du système
   \item Temps d'exécution du système
   \item Utilisation de la mémoire
   \item Nombre de processus
  \end{itemize}
  \item Voir l'état du disque dur avec
   \begin{itemize}
   \item L'utilisation de la partition /
   \item L'utilisation de la partition /boot
   \item L'utilisation de la partition /data (s'il existe)
   \item L'utilisation de la partition /opt (s'il existe)
  \end{itemize}
  \item Voir l'état du réseau avec
  \begin{itemize}
   \item Les quantités de données transmises et reçues pour chaque interface réseau
  \end{itemize}
  \item Voir les services d'interruptions avec
  \begin{itemize}
   \item Le nombre de chaque service d'interruption
  \end{itemize}
  \item Voir les zones de connexions actives avec
  \begin{itemize}
   \item Le nombre de connexions
  \end{itemize}
 \end{itemize}

En option, il est également possible en fonction de la configuration et des
paquetages installés, d'afficher les températures de sortie et les tensions
sur la carte mère, les informations du WLAN (ou sans fil), les valeurs de
l'onduleur APC, les valeurs de PING sur l'hôte ou bien les critères
d'évaluation du VPN, etc.

\subsection {Remarque sur la version RRDTOOL}

  Le fichier de la base de données RRD qui a été créé avec l'ancienne version
  rrdtool, ne peut pas être utilisé avec la version courante. Car le nouveau démon
  utilise un format de données différent, les fichiers ne sont pas compatibles.

\subsection {Remarque sur l'utilisation de RRDTOOL avec les différentes architectures}

  Si vous modifiez l'architecture du processeur de fli4l (par exemple de 32 bits
  à 64 bits) vous devez également mettre à jour le fichier de base de données RRDTOOL.
  Une conversion directe n'est pas possible.
  
  Pour cela, l'ancienne base de données doit être exporté dans un fichier XML, ensuite il
  doit être importé vers la nouvelle architecture. Il est important que l'exportation de
  l'ancienne architecture doit toujours être créée avec un fichier XML.
  
  Vous trouverez un article sur ce sujet dans cette HowTo
  \altlink{https://ssl.nettworks.org/wiki/display/f/rrdtool-Datenbanken}.

\subsection {Configuration du paquetage RRDTOOL}

  La configuration se fait, comme pour tous les autres paquetages fli4l en
  personnalisant\\
  le fichier \var{Pfad/fli4l-\version/$<$config$>$/rrdtool.txt} selon vos besoins.

\begin{description}

\config {OPT\_RRDTOOL}{OPT\_RRDTOOL}{OPTRRDTOOL}

  Si vous indiquez \var{'no'} dans cette variable vous désactivez
  complètement l'OPT\_RRDTOOL. Il n'y aura aucun changement dans l'archive
  \var{rootfs.img} ou l'archive \var{opt.img} de fli4l. De plus le paquetage
  OPT\_RRDTOOL n'écrase aucune partie de l'installation fli4l.\\
  Si vous voulez activez OPT\_RRDTOOL, vous devez indiquer \var{'yes'} dans
  cette variable.

\config {RRDTOOL\_DB\_PATH}{RRDTOOL\_DB\_PATH}{RRDTOOLDBPATH}

  Configuration par défaut~: \var{RRDTOOL\_DB\_PATH='/data/rrdtool/db'}

  Vous indiquez ici le chemin d'accès du fichier de la base de données RRDTOOL.
  Ce fichier doit toujours être enregistré sur un disque persistant. Car
  un mécanisme de mise en cache approprié est utilisé dans le paquetage RRDTOOL,
  ainsi le nombre d'accès sera réduit sur le disque, ce n'est pas un problème
  pour stocker les données sur une carte CompactFlash. Notez également, lorsque vous
  utilisez le paquetage OPT\_QOS les systèmes de fichiers par ex. ext2/ext3/ext4
  supportent les caractères utilisés dans le nom des répertoires et du fichier.

\config {RRDTOOL\_CACHETIME}{RRDTOOL\_CACHETIME}{RRDTOOLCACHETIME}

  Vous indiquez dans cette variable optionnelle un nombre en seconde, ce temp est
  utilisé par le démon pour écrire les données qui sont dans le cache RRD vers le
  fichier de la base de données RRD du disque. Si vous indiquez une petite valeur,
  les données du fichier cache dans le diskRam seront faible, cela est recommandé
  pour les routeurs qui on peut le RAM, cependant, il y aura un accès plus fréquent
  sur le disque. Sans activation de la variable l'écriture se fera après 3600 secondes
  ou lorsque vous arrêtez fli4l.

  Voici les valeurs possibles que vous pouvez configurer~:
  \begin{itemize}
    \item 3600
    \item 1800
    \item 1200
    \item 900
    \item 600
    \item 450
    \item 300
  \end{itemize}

\config {RRDTOOL\_NET}{RRDTOOL\_NET}{RRDTOOLNET}

  Si vous indiquez \var{'yes'} dans cette variable vous activez collectd avec le
  plugin-réseau. Ainsi il sera possible de transférer les données capturées/recueillies
  par collectd vers un autre ordinateur du réseau qui exécute collectd avec le
  plugin-réseau activé en mode serveur.

\config {RRDTOOL\_NET\_HOST}{RRDTOOL\_NET\_HOST}{RRDTOOLNETHOST}

  Dans cette variable vous indiquez le nom de domaine complet ou l'adresse IP
  de l'ordinateur exécutant collectd avec le plugin-réseau en mode serveur.

\config {RRDTOOL\_NET\_PORT}{RRDTOOL\_NET\_PORT}{RRDTOOLNETPORT}

  Dans cette variable optionnelle vous pouvez configurer le port sur lequel
  le serveur sera à l'écoute des connexions entrantes.

\config {RRDTOOL\_UNIXSOCK}{RRDTOOL\_UNIXSOCK}{RRDTOOLUNIXSOCK}

  Si vous indiquez \var{'yes'} dans cette variable vous activez collectd avec
  le plugin-unixsock. Avec ce socket d'autres services/processus peuvent
  utiliser les données puis les remettre dans collectd.

\config {RRDTOOL\_PING\_N}{RRDTOOL\_PING\_N}{RRDTOOLPINGN}

  Dans cette variable vous indiquez le nombre d'hôtes à paramétrer, cela servira
  à déterminer par un ping, la durée de vie de l'hôte dans le réseau.

\config {RRDTOOL\_PING\_x}{RRDTOOL\_PING\_x}{RRDTOOLPINGx}

  Dans cette variable vous indiquez l'hôte qui sera enregistré pour sa durée de vie
  dans le réseau. Les valeurs possibles sont le nom de domaine complet
  ou l'adresse IP.

\config {RRDTOOL\_PING\_x\_LABEL}{RRDTOOL\_PING\_x\_LABEL}{RRDTOOLPINGxLABEL}

  Vous pouvez éventuellement indiquer dans cette variable une autre description pour
  le ping de destination.

\config {RRDTOOL\_PING\_x\_GRPNR}{RRDTOOL\_PING\_x\_GRPNR}{RRDTOOLPINGxGRPNR}

  Dans cette variable vous attribuez un numéro d'index pour un groupe qui sera
  défini dans la variable \var{RRDTOOL\_PINGGROUP\_x\_LABEL}, cela servira à
  effectuer un ping de destination.

\config {RRDTOOL\_PINGGROUP\_N}{RRDTOOL\_PINGGROUP\_N}{RRDTOOLPINGGROUPN}

  Dans cette variable vous indiquez le nombre de groupe, qui servira à regrouper
  le ping de destination. Chaque groupe défini sera affiché dans un onglet séparé de
  l'interface web du routeur.

\config {RRDTOOL\_PINGGROUP\_x\_LABEL}{RRDTOOL\_PINGGROUP\_x\_LABEL}{RRDTOOLPINGGROUPxLABEL}

  Dans cette variable vous indiquez le nom du groupe pour un ping de destination.

\config {RRDTOOL\_APCUPS}{RRDTOOL\_APCUPS}{RRDTOOLAPCUPS}

  Dans cette variable vous activez ou désactivez la collecte de données
  d'un onduleur APC. Pour que la collecte des données fonctionne, le démon
  apcupsd doit être actif sur un ordinateur et accessible par le réseau.

\config {RRDTOOL\_APCUPS\_HOST}{RRDTOOL\_APCUPS\_HOST}{RRDTOOLAPCUPSHOST}

  Dans cette variable vous indiquez l'hôte qui exécute le démon apcupsd.

\config {RRDTOOL\_APCUPS\_PORT}{RRDTOOL\_APCUPS\_PORT}{RRDTOOLAPCUPSPORT}

  Dans cette variable vous indiquez le port réseau sous lequel le démon apcupsd
  est accessible. Normalement, ce port est 3351.

\config{RRDTOOL\_PEERPING\_N}{RRDTOOL\_PEERPING\_N}{RRDTOOLPEERPINGN}

  Dans cette variable vous indiquez le nombre de peer ping de destination (ou
  un ping pour tout le monde). L'objectif du peer ping de destination est par
  exemple de tester la destination d'un tunnels VPN.

\config {RRDTOOL\_PEERPING\_x}{RRDTOOL\_PEERPING\_x}{RRDTOOLPEERPINGx}

  Dans cette variable vous indiquez le peer ping de destination.\\
  Les destinations possibles sont par exemple tun0, tun1, pppoe, etc. On peut
  également indiquer le nom du circuit ou l'alias associé.

\config {RRDTOOL\_PEERPING\_x\_LABEL}{RRDTOOL\_PEERPING\_x\_LABEL}{RRDTOOLPEERPINGxLABEL}

  Vous pouvez éventuellement indiquer dans cette variable une autre description
  pour le peer ping de destination.

\config {RRDTOOL\_OWFS}{RRDTOOL\_OWFS}{RRDTOOLOWFS}

  Dans cette variable vous activez ou désactivez la collecte de données et
  la sortie graphique, qui sera envoyées par le paquetage OW.

\config {RRDTOOL\_OWFS\_HOST}{RRDTOOL\_OWFS\_HOST}{RRDTOOLOWFSHOST}

  Dans cette variable vous indiquez l'hôte sur lequel le service OWFS est
  exécuté. Normalement, il est exécuté sur le routeur, la valeur du domaine
  sera alors '127.0.0.1'.

\config {RRDTOOL\_OWFS\_PORT}{RRDTOOL\_OWFS\_PORT}{RRDTOOLOWFSPORT}

  Dans cette variable vous indiquez le port réseau sur lequel le service OWFS
  est accessibles. Normalement, ce port est 4304.

\config {RRDTOOL\_NTP}{RRDTOOL\_NTP}{RRDTOOLNTP}

  Dans cette variable vous activez ou désactivez la collecte de données et
  la sortie graphique, qui sera envoyées par le paquetage NTP.

\end{description}
