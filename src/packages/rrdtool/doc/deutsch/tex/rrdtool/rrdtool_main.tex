% Last Update: $Id$
\section {RRDTOOL - Daten erfassen und graphisch anzeigen}

\subsection {Beschreibung}
Das Paket RRDTOOL sammelt mit Hilfe des Daemons 'collectd' Systemdaten und 
speichert diese in rrd-Datenbanken ab.
Im Webinterface des fli4l-Routers sind dann die daraus mit Hilfe von rrdtool 
erzeugten Grafiken abruf- bzw. einsehbar.
\\
Es werden zum Beispiel die folgenden Daten erfasst und dargestellt:
 \begin{itemize}
  \item im Bereich Systemstatus
  \begin{itemize}
   \item Auslastung der CPU
   \item Systemlast
   \item Systemlaufzeit
   \item Abeitsspeicherauslastung
   \item Anzahl der Prozesse
  \end{itemize}
  \item im Bereich Festplattenstatus
   \begin{itemize}
   \item Auslastung der Partition /
   \item Auslastung der Partition /boot
   \item Auslastung der Partition /data (falls diese existiert)
   \item Auslastung der Partition /opt (falls diese existiert)
  \end{itemize}
  \item im Bereich Netzwerkstatus
  \begin{itemize}
   \item für jedes Netzwerkinterface die gesendeten und empfangenen Datenmengen
  \end{itemize}
  \item im Bereich Interrupts
  \begin{itemize}
   \item die Anzahl der einzelnen Interrupts
  \end{itemize}
  \item Im Bereich Aktive Verbindungen
  \begin{itemize}
   \item die Anzahl der Verbindungen
  \end{itemize}
 \end{itemize} 
 
Optional sind in Abhängigkeit der Konfiguration bzw. der installierten 
Pakete auch noch die Erfassung und Ausgabe von Temperaturen und Spannungen des 
Mainboards, WLAN-Informationen, die Werte einer APC-USV, PING-Werte von Hosts 
bzw. VPN-Endpunkten, usw. möglich. 

\subsection {Hinweis zur RRDTOOL-Version}

  RRD-Datenbank Dateien, die mit der alten Version von RRDTOOL erstellt wurden,
  können nicht mit der aktuellen Version verwendet werden. Da der hier verwendete
  Daemon ein anderes Datenformat verwendet, sind die Dateien inkompatibel.

\subsection {Hinweis zur Verwendung von RRDTOOL auf unterschiedlichen
Architekturen}

  Sofern die Prozessorarchitektur des fli4l gewechselt wird (z.B. von 32 Bit auf
  64 Bit) müssen auch die RRDTOOL Datenbank-Dateien angepasst werden. Eine
  direkte Konvertierung ist nicht möglich.
   
  Stattdessen müssen die alten Datenbanken in XML-Dateien exportiert und von
  dort aus auf der neuen Architektur wieder importiert werden. Wichtig ist, dass
  der Export in die XML-Dateien noch auf der alten Architektur stattfindet.

  Ein HowTo Artikel zu diesem Thema findet sich unter
  \altlink{https://ssl.nettworks.org/wiki/display/f/rrdtool-Datenbanken}.

\subsection {Konfiguration des Paketes RRDTOOL}
     
  Die Konfiguration erfolgt, wie bei allen fli4l Paketen, durch Anpassung der \\
  Datei \var{Pfad/fli4l-\version/$<$config$>$/rrdtool.txt} 
  an die eigenen Anforderungen.

\begin{description}

\config {OPT\_RRDTOOL}{OPT\_RRDTOOL}{OPTRRDTOOL}

  Die Einstellung \var{'no'} deaktiviert das OPT\_RRDTOOL vollständig. Es werden
  keine Änderungen am fli4l Archiv \var{rootfs.img} bzw. dem Archiv \var{opt.img}
  vorgenommen. Weiterhin überschreibt das OPT\_RRDTOOL grundsätzlich keine anderen
  Teile der fli4l Installation.\\
  Um OPT\_RRDTOOL zu aktivieren, ist die Variable \var{OPT\_RRDTOOL} auf 
  \var{'yes'} zu setzen.
  
\config {RRDTOOL\_DB\_PATH}{RRDTOOL\_DB\_PATH}{RRDTOOLDBPATH}

  Standard-Einstellung: \var{RRDTOOL\_DB\_PATH='/data/rrdtool/db'}
  
  Pfad zu den Datenbank-Dateien von RRDTOOL. Diese Dateien sollten immer auf 
  einem persistenten Datenträger liegen. Da im Paket RRDTOOL entsprechende 
  Caching-Mechanismen zum Einsatz kommen, wird die Anzahl der Zugriffe auf den 
  Datenträger minimiert, wodurch es kein Problem darstellt, die Daten auf einer 
  CompactFlash-Karte abzulegen.
  Weiterhin ist zu beachten, das beim Einsatz des OPT\_QOS das Dateisystem des 
  Paths z.B. ext2/ext3/ext4 ist, da nur diese Dateisystem die Zeichen in Dateinamen
  unterstützen die benötigt werden.

\config {RRDTOOL\_CACHETIME}{RRDTOOL\_CACHETIME}{RRDTOOLCACHETIME}

  Mit diesem optionalen Konfigurationsparameter kann festgelegt werden, nach 
  wievielen Sekunden der rrdcached-Daemon die gecached Werte in die RRD-Datenbank-
  Dateien schreibt. Durch kleiner Werte wird die Cache-Datei in der Ramdisk kleiner
  was sich bei Routern mit eher kleinem RAM empfiehlt, jedoch erfolgt dann häufiger
  ein Zugriff auf den Datenträger.
  Ohne Aktivierung der Variable erfolgt dies nach 3600 Sekunden oder beim
  Herunterfahren des fli4l.

  Folgende möglichen Werte können konfiguriert werden:
  \begin{itemize}
    \item 3600
    \item 1800
    \item 1200
    \item 900
    \item 600
    \item 450
    \item 300
  \end{itemize}


\config {RRDTOOL\_NET}{RRDTOOL\_NET}{RRDTOOLNET}

  Die Einstellung \var{'yes'} aktiviert das network-plugin des collectd.
  Dadurch ist es dann möglich, die durch den collectd erfassten/gesammelten Daten
  an einen anderen Rechner auf dem der collectd mit aktivem network-plugin im
  Server-Mode läuft zu übertragen.
  
\config {RRDTOOL\_NET\_HOST}{RRDTOOL\_NET\_HOST}{RRDTOOLNETHOST}

  FQDN oder IP-Adresse des Rechners, auf dem der collectd mit network-plugin im
  Server-Mode läuft.
  
\config {RRDTOOL\_NET\_PORT}{RRDTOOL\_NET\_PORT}{RRDTOOLNETPORT}
  
  In dieser optionalen Variablen kann der Port, auf dem der Server auf eingehende 
  Verbindungen wartet konfiguriert werden.
  
\config {RRDTOOL\_UNIXSOCK}{RRDTOOL\_UNIXSOCK}{RRDTOOLUNIXSOCK}
  
  Die Einstellung \var{'yes'} aktiviert das unixsock-Plugin des collectd.
  Über diesen Socket können weitere Dienste/Prozesse die Daten erfassen und an 
  den collectd übergeben.
  
\config {RRDTOOL\_PING\_N}{RRDTOOL\_PING\_N}{RRDTOOLPINGN}

  Legt die Anzahl der Hosts fest, bei dennen via Ping die Netzwerklaufzeiten
  ermittelt werden.
  
\config {RRDTOOL\_PING\_x}{RRDTOOL\_PING\_x}{RRDTOOLPINGx}
  
  Definiert den Host von dem die Netzwerklaufzeit erfasst werden soll.
  Die Angabe kann als FQDN oder mit der IP-Adresse erfolgen.
  
\config {RRDTOOL\_PING\_x\_LABEL}{RRDTOOL\_PING\_x\_LABEL}{RRDTOOLPINGxLABEL}

  Legt optional eine anderere Beschreibung für das Ping-Ziel fest. 
  
\config {RRDTOOL\_PING\_x\_GRPNR}{RRDTOOL\_PING\_x\_GRPNR}{RRDTOOLPINGxGRPNR}

  Ordnet dieses Ping-Ziel der in \var{RRDTOOL\_PINGGROUP\_x\_LABEL} definierten 
  Gruppe über die Nummer des Indexes zu.
  
\config {RRDTOOL\_PINGGROUP\_N}{RRDTOOL\_PINGGROUP\_N}{RRDTOOLPINGGROUPN}
  
  Anzahl der Gruppen in die die Ping-Ziele guppiert werden können. Jede definierte
  Gruppe wird im Webinterface in einem extra Reiter dargestellt.
  
\config {RRDTOOL\_PINGGROUP\_x\_LABEL}{RRDTOOL\_PINGGROUP\_x\_LABEL}{RRDTOOLPINGGROUPxLABEL}

   Name der Gruppe der Pingziele. 

\config {RRDTOOL\_APCUPS}{RRDTOOL\_APCUPS}{RRDTOOLAPCUPS}

  Aktiviert bzw. deaktiviert die Erfassung von Daten einer APC-USV.
  Zur Datenerfassung muss auf einem über Netzwerk erreichbaren Rechner der
  apcupsd-Daemon aktiv sein.

\config {RRDTOOL\_APCUPS\_HOST}{RRDTOOL\_APCUPS\_HOST}{RRDTOOLAPCUPSHOST}

  Host auf dem der apcupsd-Daemon ausgeführt wird.
  
\config {RRDTOOL\_APCUPS\_PORT}{RRDTOOL\_APCUPS\_PORT}{RRDTOOLAPCUPSPORT}

  Netzwerkport unter dem der apcupsd-Daemon erreichbar ist.
  Im Normalfall ist dies der Port 3351.

\config{RRDTOOL\_PEERPING\_N}{RRDTOOL\_PEERPING\_N}{RRDTOOLPEERPINGN}

  Legt die Anzahl der Peer-Ping-Ziele fest. Ein Peer-Ping-Ziel ist z.B. das Ziel 
  eines VPN-Tunnels.
  
\config {RRDTOOL\_PEERPING\_x}{RRDTOOL\_PEERPING\_x}{RRDTOOLPEERPINGx}

  Definiert das Peer-Ping-Ziel. \\
  Mögliche Ziele sind z.B. tun0, tun1, pppoe, usw. Es können auch die zugehörigen 
  Alias- bzw. Circuit-Namen verwendet werden.
  
\config {RRDTOOL\_PEERPING\_x\_LABEL}{RRDTOOL\_PEERPING\_x\_LABEL}{RRDTOOLPEERPINGxLABEL}

    Legt optional eine anderere Beschreibung für das Peer-Ping-Ziel fest. 
    
\config {RRDTOOL\_OWFS}{RRDTOOL\_OWFS}{RRDTOOLOWFS}

  Aktiviert bzw. Deaktiviert die Erfassung sowie graphische Ausgabe 
  der Daten, welche durch das Paket OW bereitgestellt werden.
  
\config {RRDTOOL\_OWFS\_HOST}{RRDTOOL\_OWFS\_HOST}{RRDTOOLOWFSHOST}

  Host auf dem der OWFS-Dienst läuft. Im Normalfall ist dies der Router
  selbst. Somit ist der Wert '127.0.0.1' einzutragen.

\config {RRDTOOL\_OWFS\_PORT}{RRDTOOL\_OWFS\_PORT}{RRDTOOLOWFSPORT}

  Netzwerkport auf dem der OWFS-Dienst erreichbar ist.
  Im Normalfall ist dies der Port 4304.
  
\config {RRDTOOL\_NTP}{RRDTOOL\_NTP}{RRDTOOLNTP}

  Aktiviert bzw. Deaktiviert die Erfassung sowie graphische Ausgabe 
  der Daten, welche durch das Paket NTP bereitgestellt werden.
  
\end{description} 

