% Last Update: $Id$
\section {OIDENTD - Ident Daemon}

OPT\_OIDENTD stellt einen Ident Daemon bereit, wie er im
\jump{url:rfc1413}{RFC 1413} \jump{url:rfc1413e}{(deutche Übersetzung)} 
spezifiziert wird. Ident (veraltet AUTH) ist ein Dienst, mit dem einem 
anderem System Benutzerdaten zugestellt werden. Manche \mbox{E-Mail}-, News- und
IRC-Server verwenden diesen Dienst, um die Identität eines Benutzers zu
überprüfen, bevor ein Zugriff erlaubt wird. Ident verwendet den TCP-Port 113. 
Zum Einsatz kommt \jump{url:oidentdsource}{oidentd} in
der Version 2.0.8. 


\textbf{Disclaimer: }\emph{Der Autor gibt weder eine Garantie auf die
Funktionsfähigkeit des OPT\_\-OIDENTD, noch haftet er für Schäden, z.B.
Datenverlust, die durch den Einsatz von OPT\_\-OIDENTD entstehen.}


\marklabel{sec:konfigoidentd}{
\subsection {Konfiguration des OPT\_OIDENTD}
}

Die Konfiguration erfolgt, wie bei allen fli4l Opts, durch Anpassung der Datei
\var{Pfad/fli4l-\version/$<$config$>$/oidentd.txt} an die eigenen Anforderungen. Im
weiteren erfolgt nun die Beschreibung der einzelnen Variablen: 


\begin{description}

\config {OPT\_OIDENTD}{OPT\_OIDENTD}{OPTOIDENTD}

  Default: \var{OPT\_OIDENTD='no'}

  Die Einstellung \var{'no'} deaktiviert das OPT\_OIDENTD vollständig. Es werden
  keine Änderungen am fli4l Bootmedium bzw. dem Archiv \var{opt.img}
  vorgenommen. Weiter überschreibt das OPT\_OIDENTD grundsätzlich keine anderen
  Teile der fli4l Installation.\\
  Um OPT\_OIDENTD zu aktivieren, ist die Variable \var{OPT\_OIDENTD} auf 
  \var{'yes'} zu setzen. 

  \wichtig:{Damit oidentd überhaupt Ident Anfragen beantworten kann, muß
  der INPUT Port 113 TCP geöffnet sein! Ab Version 2.1.12 wird der Port
  automatisch geöffnet!}
  
\config {OIDENTD\_FORWARD}{OIDENTD\_FORWARD}{OIDENTDFORWARD}

  Default: \var{OIDENTD\_FORWARD='no'}

  Die Variable \var{OIDENTD\_FORWARD} steuert, ob \var{oidentd} Ident Anfragen
  an die Clients hinterm fli4l Router weiterleitet oder anhand der eigenen
  Datenbank beantwortet. In der default Einstellung werden die Anfragen nicht
  weitergeleitet.

\config {OIDENTD\_DEFAULT}{OIDENTD\_DEFAULT}{OIDENTDDEFAULT}

  Default: \var{OIDENTD\_DEFAULT='unkown'}

  Läßt sich weder anhand der internen Datenbank, noch durch Forward (falls
  aktiviert), eine gültige Antwort ermitteln, sendet \var{oidentd} den Inhalt
  von \var{OIDENTD\_DEFAULT} als Antwort.

\config {OIDENTD\_HOST\_N}{OIDENTD\_HOST\_N}{OIDENTDHOSTN}

  Default: \var{OIDENTD\_HOST\_N='0'}

  \var{OIDENTD\_HOST\_N} legt die Anzahl der Einträge in der lokalen
  Datenbank fest. Für jeden Eintrag ist nachfolgender Satz an 
  \var{OIDENTD\_HOST\_x\_...} Variablen anzulegen. Der Index \var{x} muß
  fortlaufend bis zur Gesamtanzahl der Einträge heraufgezählt werden. 
  

\config {OIDENTD\_HOST\_x\_IP}{OIDENTD\_HOST\_x\_IP}{OIDENTDHOSTxIP}

  Über \var{OIDENTD\_HOST\_x\_IP} wird der Client bzw. das Subnet festgelegt,
  für den ein Eintrag generiert werden soll. Es kann sowohl der Hostname
  (DNS-Name), die IP-Adresse oder das Subnet angegeben werden.

  Beispiele: 

  \begin{example}
  OIDENTD\_HOST\_x\_IP='192.168.6.1'\\
  OIDENTD\_HOST\_x\_IP='192.168.6.0/255.255.255.0'\\
  OIDENTD\_HOST\_x\_IP='192.168.6.0/24'\\
  OIDENTD\_HOST\_x\_IP='client.lan.fli4l'\\
  OIDENTD\_HOST\_x\_IP='@client'
  \end{example}
  
    
\config {OIDENTD\_HOST\_x\_USERNAME}{OIDENTD\_HOST\_x\_USERNAME}{OIDENTDHOSTxUSERNAME}

  Den Inhalt von \var{OIDENTD\_HOST\_x\_USERNAME} sendet \var{oidentd} als 
  Antwort. Hier kann ein Benutzername (=Username), der echte Name, eine 
  \mbox{E-Mail}-Adresse oder was auch immer hinterlegt werden. Jedoch sind keine
  Leerstellen (Blanks, Spaces) erlaubt. Diese bitte durch einen Unterstrich
  \_ ersetzen.

\config {OIDENTD\_HOST\_x\_SYSTEM}{OIDENTD\_HOST\_x\_SYSTEM}{OIDENTDHOSTxSYSTEM}
 
  Die Antwort auf einen Ident Request (=Anfrage) beinhaltet nicht nur den 
  Benutzernamen (\var{OIDENTD\_HOST\_x\_USERNAME}), sondern auch das zugehörige
  Betriebssystem des Benutzers. Die entsprechenden Kürzel sind im
  \jump{url:rfc1340}{RFC 1340} festgelegt. Opt\_oidentd läßt aber nur eine
  begrenzte Auswahl zu: \texttt{DOS, ELF, MACOS, MSDOS, OS/2, PC-DOS,
  SCO-XENIX/386, SUN, UNIX, UNIX-BSD, UNIX-PC, UNKNOWN, WIN32, XENIX} und 
  \texttt{OTHERS}. Sollte bedarf an Ergänzungen bestehen, nehmen Sie bitte Kontakt
  mit dem Autor unter Berücksichtigung des 
  \jump{sec:oisupport}{Supporthinweises} auf.


\end{description}
  
\marklabel{sec:oisupport}{
\subsection{Support}
}
Der Autor leistet Support nur im Rahmen der \jump{url:oifli4lnews}{fli4l
Newsgroups}. Anfragen per \mbox{E-Mail} werden zu 100\% nicht beachtet. Einzig Hinweise
auf Fehler sind per \mbox{E-Mail} willkommen. Da die Adresse \texttt{<arno@fli4l.de>}
leider auch der massiven Spam-/Virenflut unterliegt, filtert der Autor diese
Adresse.  Es werden nur Mails akzeptiert, die:
\begin{itemize}
 \item auch den Realname des Autors im \texttt{To:} enthalten: \\
       \texttt{To: Arno Behrends <arno@fli4l.de>}
 \item im Subject/\/Betreff/\/Thema den Tag \texttt{[oidentd]} führen: \\
       \texttt{Subject: [oidentd] Fehler in der Doku}
 \item kein HTML-Code enthalten.
 \item keine Attachments/Dateianhänge beinhalten.
\end{itemize}
Bitte auch in den Newsgroup den Tag \texttt{[oidentd]} im Subject benutzen. Dies
erhöht deutlich die Chance, vom Autor gelesen zu werden.


\subsection{Literatur}

\marklabel{url:oidentdsource}{
 Homepage von oidentd: \altlink{http://dev.ojnk.net/}
 }

\marklabel{url:oidentdmanpage}{
 man page zu oidentd: \altlink{http://linux.die.net/man/8/oidentd}
 }

\marklabel{url:rfc1413}{
 RFC 1413 - Identification Protocol: \altlink{http://www.faqs.org/rfcs/rfc1413.html}
 }

\marklabel{url:rfc1413e}{
 Identification Protocol (deutsch): \altlink{http://kefk.net/Security/Misc/marc_ruef/rfc1413de/rfc1413de.txt}
 }

\marklabel{url:rfc1340}{
 RFC 1340 - Assigned Numbers: \altlink{http://www.faqs.org/rfcs/rfc1340.html}
 }

\marklabel{url:oifli4lnews}{
 fli4l Newsgroups und ihre Spielregeln: \altlink{http://www.fli4l.de/hilfe/newsgruppen/}
}
