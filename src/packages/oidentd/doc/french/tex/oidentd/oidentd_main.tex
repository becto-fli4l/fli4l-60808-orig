% Synchronized to r30389
\section {OIDENTD - Démon Ident}

Le paquetage OPT\_OIDENTD vous offre le démon Ident (ou Identification), qui est
documenté dans \jump{url:rfc1413}{RFC 1413} ou \jump{url:rfc1413e}{(traduction française)}.
Ident (anciennement AUTH) est un service qui fournit des données de l'utilisateur
à un autre système. Certains \mbox{E-Mail}, News et serveurs IRC utiliser ce
service pour vérifier l'identité de l'utilisateur avant d'autoriser l'accès.
Ident utilise le port TCP 113. Document pour l'utilisation de \jump{url:oidentdsource}{oidentd}
version 2.0.8


\textbf{Disclaimer: }\emph{L'auteur ne garantie pas le bon fonctionnement
du paquetage OPT\_\-OIDENTD, il est ni responsable de tout dommage
(c.-à-d. de la perte de données) causés par l'utilisation de OPT\_\-OIDENTD.}


\marklabel{sec:konfigoidentd}{
\subsection {Configuration de OPT\_OIDENTD}
}

La configuration se fait comme pour tous les autres opts- sur fli4l, en paramétrant
le fichier \var{Pfad/fli4l-\version/$<$config$>$/oidentd.txt}. Vous pouvez lire dans
les paragraphes ci-dessous une description des variables.


\begin{description}

\config {OPT\_OIDENTD}{OPT\_OIDENTD}{OPTOIDENTD}

  Par défaut~: \var{OPT\_OIDENTD='no'}

  Le paramètre \var{'no'} désactive complètement le paquetage OPT\_OIDENTD.
  Aucune modification ne sera apportée au média de boot de fli4l, ni à
  archive \var{opt.img}. Aucune autres parties de l'installation ne sera modifié
  par OPT\_OIDENTD.\\ 
  Pour activer le paquetage OPT\_OIDENTD, vous devez définir dans la variable
  \var{OPT\_OIDENTD} le paramètre \var{'yes'}. 

  \wichtig{Pour que oidentd réponde au demande d'Ident, il est essentiel
  que le port TCP 113 soit ouvert dans la variable INPUT~! Depuis la version 2.1.12
  le port est ouvert automatiquement~!}

\config {OIDENTD\_FORWARD}{OIDENTD\_FORWARD}{OIDENTDFORWARD}

  Par défaut~: \var{OIDENTD\_FORWARD='no'}

  Avec la variable \var{oidentd\_forward} vous définissez si les requêtes ident de
  \var{oidentd} seront transmises aux clients derrière le routeur fli4l ou si la
  réponce de la requête est sur sa propre base de données. Dans le réglage par défaut
  des requêtes ne seront pas transmises.

\config {OIDENTD\_DEFAULT}{OIDENTD\_DEFAULT}{OIDENTDDEFAULT}

  Par défaut~: \var{OIDENTD\_DEFAULT='unkown'}

  Si ni la base de données interne, ni forward (si activé) renvoie une réponse
  valable, \var{oidentd} enverra le contenu de la variable \var{oidentd\_DEFAULT}
  comme réponse.

\config {OIDENTD\_HOST\_N}{OIDENTD\_HOST\_N}{OIDENTDHOSTN}

  Par défaut~: \var{OIDENTD\_HOST\_N='0'}

  Avec la variable \var{OIDENTD\_HOST\_N} vous définissez le nombre d'entrées
  dans la base de données locale. Pour chaque entrée, la variable
  \var{OIDENTD\_HOST\_x\_...} doit être créée. L'indice \var{x} doit être incrémenté
  jusqu'à concurrence du nombre total d'entrées.

\config {OIDENTD\_HOST\_x\_IP}{OIDENTD\_HOST\_x\_IP}{OIDENTDHOSTxIP}

  Au sujet de la variable \var{OIDENTD\_HOST\_x\_IP} vous indiquez dans celle-ci
  un client ou un sous-réseau, pour lequel une entrée a été généré. Vous pouvez
  spécifier un nom d'hôte (DNS-Name), une adresse IP ou un sous-réseau.

  Exemple~:

  \begin{example}
  OIDENTD\_HOST\_x\_IP='192.168.6.1'\\
  OIDENTD\_HOST\_x\_IP='192.168.6.0/255.255.255.0'\\
  OIDENTD\_HOST\_x\_IP='192.168.6.0/24'\\
  OIDENTD\_HOST\_x\_IP='client.lan.fli4l'\\
  OIDENTD\_HOST\_x\_IP='@client'
  \end{example}


\config {OIDENTD\_HOST\_x\_USERNAME}{OIDENTD\_HOST\_x\_USERNAME}{OIDENTDHOSTxUSERNAME}

  Cette variable \var{OIDENTD\_HOST\_x\_USERNAME} est utilisé pour envoyer la réponce
  de \var{oidentd}. C'est peut-être un nom d'utilisateur,  un nom réel, une adresse
  \mbox{E-Mail} ou quelque chose d'autre. L'espace ou blancs n'est pas autorisé dans
  le paramètre. Merci de remplacer ce blanc par un trait de soulignement \_.

\config {OIDENTD\_HOST\_x\_SYSTEM}{OIDENTD\_HOST\_x\_SYSTEM}{OIDENTDHOSTxSYSTEM}

  La réponse à une requête(=demande) ident contient non seulement le nom de
  l'utilisateur (\var{OIDENTD\_HOST\_x\_USERNAME}), mais aussi le système
  d'exploitation utilisé. Les acronymes correspondants sont décrits dans
  \jump{url:rfc1340}{RFC 1340}. Opt\_oidentd renvoie un choix limité~: 
  \texttt{DOS, ELF, MACOS, MSDOS, OS/2, PC-DOS, SCO-XENIX/386, SUN, UNIX,
  UNIX-BSD, UNIX-PC, UNKNOWN, WIN32, XENIX} et \texttt{OTHERS}. Si des
  ajouts sont nécessaires s'il vous plaît contacter l'auteur sur
  \jump{sec:oisupport}{référence de soutien}

\end{description}


\marklabel{sec:oisupport}{
\subsection{support technique}
}
L'aide technique ne sera possible que sur les \jump{url:oifli4lnews}{Newsgroups fli4l}.
L'auteur ne répondra pas aux demandes de renseignements par \mbox{E-Mail}. Par contre
les rapports d'erreur via \mbox{E-Mail} seront les bienvenus depuis l'adresse
\texttt{<arno@fli4l.de>}. Malheureusement cette adresse fait l'objet d'abus de
spam/virus massif, donc l'auteur a installé des filtres sur cette adresse~:

\begin{itemize}
 \item L'adresse doit contenir un vrai nom de l'auteur dans \texttt{To:} par ex.\\
       \texttt{To: Arno Behrends <arno@fli4l.de>}
 \item le sujet doit contenir la balise \texttt{[oidentd]} par ex.\\
       \texttt{Subject: [oidentd] Erreur dans Docu}
 \item Ne doit pas contenir de code HTML.
 \item Ne doit pas contenir de pièces jointes.
\end{itemize}
S'il vous plaît utiliser la balise \texttt{[oidentd]} dans le sujet des groupes
de discussion. Cela augmente considérablement les chances d'être lu par l'auteur.


\subsection{Document}

\marklabel{url:oidentdsource}{
 Page d'accueil de oidentd~: \altlink{http://dev.ojnk.net/}
 }

\marklabel{url:oidentdmanpage}{
 Page du manuel de oidentd~: \altlink{http://linux.die.net/man/8/oidentd}
 }

\marklabel{url:rfc1413}{
 Identification du protocole - RFC 1413~: \altlink{http://www.faqs.org/rfcs/rfc1413.html}
 }

\marklabel{url:rfc1413e}{
 Identification du protocole (en français)~: \altlink{http://www.normes-internet.com/normes.php?rfc=rfc1413&lang=fr}
 }

\marklabel{url:rfc1340}{
 Numéros assignés - RFC 1340~: \altlink{http://www.faqs.org/rfcs/rfc1340.html}
 }

\marklabel{url:oifli4lnews}{
 Groupes de discussion et règlement de fli4l~: \altlink{http://www.fli4l.de/hilfe/newsgruppen/}
}
