% Synchronized to r30389
\section {OIDENTD - Ident Daemon}

OPT\_OIDENTD provides an Ident Daemon as specified in
\jump{url:rfc1413}{RFC 1413}. Ident (former AUTH) is a service that
delivers user data to other systems. Some \mbox{E-Mail}, News and
IRC Servers use this service to check the identity of a user
before allowing access. Ident uses TCP port 113.
\jump{url:oidentdsource}{oidentd} is used in particular.


\textbf{Disclaimer: }\emph{The author neither makes any guarantees for the proper
function of OPT\_\-OIDENTD, nor is he liable for any damage (i.e.
data loss) caused by the use of OPT\_\-OIDENTD.}


\marklabel{sec:konfigoidentd}{
\subsection {Configuration Of OPT\_OIDENTD}
}

Configuration is done by adapting the file
\var{path/fli4l-\version/$<$config$>$/oidentd.txt} to your needs.
Following is the description of the relevant variables:


\begin{description}

\config {OPT\_OIDENTD}{OPT\_OIDENTD}{OPTOIDENTD}

  Default: \var{OPT\_OIDENTD='no'}

  The setting \var{'no'} deactivates OPT\_OIDENTD completely. No changes
  are made to the fli4l boot medium resp. to the archive \var{opt.img}.
  No other parts of the installation will be changed by OPT\_OIDENTD.\\
  To activate OPT\_OIDENTD set the variable \var{OPT\_OIDENTD} to
  \var{'yes'}.

  \wichtig{For proper operation of oidentd it is essential to open
  INPUT port 113 TCP! As of version 2.1.12 the port is opened automatically!}

\config {OIDENTD\_FORWARD}{OIDENTD\_FORWARD}{OIDENTDFORWARD}

  Default: \var{OIDENTD\_FORWARD='no'}

  \var{OIDENTD\_FORWARD} sets if \var{oidentd} ident queries
  will be forwarded to the clients behind fli4l or will be answered
  from the database on fli4l. In the default setting queries will not
  be forwarded.

\config {OIDENTD\_DEFAULT}{OIDENTD\_DEFAULT}{OIDENTDDEFAULT}

  Default: \var{OIDENTD\_DEFAULT='unkown'}

  If neither the internal database nor forward (if activated)
  returns a valid answer \var{oidentd} will send the content of
  \var{OIDENTD\_DEFAULT} as an answer.

\config {OIDENTD\_HOST\_N}{OIDENTD\_HOST\_N}{OIDENTDHOSTN}

  Default: \var{OIDENTD\_HOST\_N='0'}

  \var{OIDENTD\_HOST\_N} sets the number of entries in the local
  database. For each entry the following \var{OIDENTD\_HOST\_x\_...}
  variables have to be created. The index \var{x} has to be incremented
  up to the total number of entries.


\config {OIDENTD\_HOST\_x\_IP}{OIDENTD\_HOST\_x\_IP}{OIDENTDHOSTxIP}

  By \var{OIDENTD\_HOST\_x\_IP} the client resp. the subnet is specified
  for which an entry should be generated. The hostname (DNS name) as well
  as the IP address or the subnet may be specified.

  Example:

  \begin{example}
  OIDENTD\_HOST\_x\_IP='192.168.6.1'\\
  OIDENTD\_HOST\_x\_IP='192.168.6.0/255.255.255.0'\\
  OIDENTD\_HOST\_x\_IP='192.168.6.0/24'\\
  OIDENTD\_HOST\_x\_IP='client.lan.fli4l'\\
  OIDENTD\_HOST\_x\_IP='@client'
  \end{example}


\config {OIDENTD\_HOST\_x\_USERNAME}{OIDENTD\_HOST\_x\_USERNAME}{OIDENTDHOSTxUSERNAME}

  The content of \var{OIDENTD\_HOST\_x\_USERNAME} is the answer sent by \var{oidentd}.
  This may be a user name, real name, an \mbox{E-Mail} address
  or something else. There are no blanks or spaces allowed. Please
  replace those by an underscore \_ .

\config {OIDENTD\_HOST\_x\_SYSTEM}{OIDENTD\_HOST\_x\_SYSTEM}{OIDENTDHOSTxSYSTEM}

  The answer to an ident request contains not only the user name (\var{OIDENTD\_HOST\_x\_USERNAME}),
  but also the operating system in use. The corresponding acronyms are described in
  \jump{url:rfc1340}{RFC 1340}. Opt\_oidentd only returns a limited
  selection: \texttt{DOS, ELF, MACOS, MSDOS, OS/2, PC-DOS,
  SCO-XENIX/386, SUN, UNIX, UNIX-BSD, UNIX-PC, UNKNOWN, WIN32, XENIX} and
  \texttt{OTHERS}. If additions are needed please contact the author at
  \jump{sec:oisupport}{Support}.


\end{description}

\marklabel{sec:oisupport}{
\subsection{Support}
}
Support will only be possible on the newsgroups \jump{url:oifli4lnews}{fli4l Newsgroups}.
\mbox{E-Mail} will not be answered. Error reports via \mbox{E-Mail} are welcome.
The address \texttt{<arno@fli4l.de>} is subject of massive spam abuse an thus is
filtered. Only mails are accepted that:
\begin{itemize}
 \item Contain the real name of the author in \texttt{To:}: \\
       \texttt{To: Arno Behrends <arno@fli4l.de>}
 \item Subject must contain the tag \texttt{[oidentd]}: \\
       \texttt{Subject: [oidentd] Error in Docu}
 \item Do not contain HTML code.
 \item Do not contain attachments.
\end{itemize}
Please use the tag \texttt{[oidentd]} in the subject also in the newsgroups. This
significantly increases the chance of being read by the author.


\subsection{Literature}

\marklabel{url:oidentdsource}{
 Homepage oidentd: \altlink{http://dev.ojnk.net/}
 }

\marklabel{url:oidentdmanpage}{
 man page oidentd: \altlink{http://linux.die.net/man/8/oidentd}
 }

\marklabel{url:rfc1413}{
 RFC 1413 - Identification Protocol: \altlink{http://www.faqs.org/rfcs/rfc1413.html}
 }

\marklabel{url:rfc1340}{
 RFC 1340 - Assigned Numbers: \altlink{http://www.faqs.org/rfcs/rfc1340.html}
 }

\marklabel{url:oifli4lnews}{
 fli4l Newsgroups and rules: \altlink{http://www.fli4l.de/hilfe/newsgruppen/}
}
