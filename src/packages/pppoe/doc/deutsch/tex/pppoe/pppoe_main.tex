% Last Update: $Id$
\section{PPPOE - PPP über Ethernet}

Dieses Paket erlaubt es, PPP-Verbindungen über ein Ethernet-Netzwerk aufzubauen.
Dies wird in Deutschland vor allem für die Internet-Anbindung via DSL genutzt,
wobei ein externes, über Ethernet angeschlossenes DSL-Modem genutzt wird.
Zwischen dem fli4l-Router und dem DSL-Modem wird das Protokoll PPP benutzt,
jedoch hier speziell über Ethernet (PPPoE\footnote{``Point-to-Point Protocol
over Ethernet'', siehe RFC 2516}).

\subsection{Ausgehende PPPoE-Verbindungen}

Ein DSL-Zugang via PPPoE oder auch beliebige andere ausgehende
PPPoE-Verbindungen werden generell als PPP-Circuit konfiguriert
(siehe \jump{sect:ppp-circuits}{Circuits vom Typ ``ppp''}), d.\,h. es gilt:

\begin{example}
\begin{verbatim}
    CIRC_x_TYPE='ppp'
\end{verbatim}
\end{example}

Zusätzlich muss das \verb+OPT_PPP_ETHERNET+ aktiviert werden:

\begin{description}
\config{OPT\_PPP\_ETHERNET}{OPT\_PPP\_ETHERNET}{OPTPPPETHERNET}

Diese Variable aktiviert die Unterstützung für ausgehende PPPoE-Verbindungen.
Damit auch tatsächlich eine PPPoE-Verbindung genutzt werden kann, muss
mindestens ein PPP-Circuit den Typ ``ethernet'' besitzen, d.\,h. es muss
zusätzlich gelten

\begin{example}
\begin{verbatim}
    CIRC_x_TYPE='ppp'
    CIRC_x_PPP_TYPE='ethernet'
\end{verbatim}
\end{example}

(wobei ``x'' einen gültigen Circuit-Index darstellt).

Standard-Einstellung: \verb+OPT_PPP_ETHERNET='no'+

Beispiel: \verb+OPT_PPP_ETHERNET='yes'+
\end{description}

Zu den allgemeinen Circuit-Variablen kommen die folgenden, für PPP-Circuits des
Typs ``ethernet'' spezifischen Variablen hinzu:

\begin{description}
\config{CIRC\_x\_PPP\_ETHERNET\_DEV}{CIRC\_x\_PPP\_ETHERNET\_DEV}{CIRCxPPPETHERNETDEV}

Diese Variable enthält den Namen der Netzwerk-Schnitt\-stelle, über die PPPoE
genutzt werden soll. Dabei entspricht \texttt{eth0} der ersten Ethernet-Karte,
\texttt{eth1} der zweiten Ethernet-Karte u.\,s.\,w.

Hat man nur eine Netzwerkkarte im Router, kann man das DSL-Modem ins LAN
hängen, etwa indem man das DSL-Modem an den LAN-Switch mit anschließt.
Dies ist jedoch nur die zweitbeste Lösung, weil die normalen IP-Pakete
dann zusammen mit den PPPoE-Paketen im selben Netzsegment auftauchen und dies
die maximale Übertragungsgeschwindigkeit negativ beeinflussen kann. Die beste
Lösung ist die Nutzung einer separaten Netzwerkkarte für die Kommunikation mit
dem DSL-Modem.

Beispiel: \verb+CIRC_1_PPP_ETHERNET_DEV='eth1'+

\config{CIRC\_x\_PPP\_ETHERNET\_TYPE}{CIRC\_x\_PPP\_ETHERNET\_TYPE}{CIRCxPPPETHERNETTYPE}

PPPoE steht für die Übertragung von PPP-Paketen über
Ethernet-Leitungen. Das bedeutet, dass die zu übertragenden Daten im ersten
Schritt vom ppp-Dämon in PPP-Pakete und dann in einem zweiten Schritt für die
Übertragung übers Ethernet nochmals in PPPoE-Pakete verpackt werden, um dann
ans DSL-Modem geschickt zu werden. Das zweite Verpacken kann durch den
pppoe-Dämon oder durch den Kern erfolgen. Mittels
\var{CIRC\_x\_PPP\_ETHERNET\_TYPE} wird die Art und Weise der
PPPoE-Paketerzeugung definiert, siehe hierzu Tabelle \ref{tab:pppoe-type}.

\begin{table}[h!]
  \centering
  \begin{tabular}{|l|p{10cm}|}
    \hline
    Wert & Beschreibung \\
    \hline
    kernel & Die PPP-Pakete werden direkt an den Linux-Kern
    gereicht, der daraus PPPoE-Pakete macht. Dadurch entfällt die
    Kommunikation mit einem zweiten Prozess und damit eine Menge
    Kopier- und Scheduling-Aufwand, was wiederum zu geringerer Prozessorlast
    führt. Dies ist die Standard-Methode, wenn nichts angegeben wurde.\\
    daemon & Die Pakete werden durch den \texttt{pppoe}-Dämon erzeugt; die
    Kommunikation zwischen \texttt{pppd} und \texttt{pppoe} erfolgt asynchron.
    Das bedeutet, dass der Datenstrom mit Anfang- und Ende-Markern versehen
    wird, damit der \texttt{pppoe}-Dämon die einzelnen Pakete auseinanderhalten
    kann. Aufgrund des zweiten Prozesses und der zusätzlichen Markierungen ist
    diese Methode aufwändiger als die Methode ``kernel'' und sollte in der
    Regel nicht verwendet werden.\\
    \hline
  \end{tabular}
  \caption{Arten der PPPoE-Paketerzeugung}
  \label{tab:pppoe-type}
\end{table}

Standard-Einstellung: \verb+CIRC_x_PPP_ETHERNET_TYPE='kernel'+

Beispiel: \verb+CIRC_1_PPP_ETHERNET_TYPE='daemon'+

\end{description}

\subsection{Eingehende PPPoE-Verbindungen}

Der fli4l kann auch konfiguriert werden, \emph{eingehende} PPPoE-Verbindungen
anzunehmen, also als ein Server zu fungieren. Solche PPPoE-Verbindungen werden
ebenfalls als PPP-Circuit konfiguriert (siehe
\jump{sect:ppp-circuits}{Circuits vom Typ ``ppp''}), d.\,h. es gilt:

\begin{example}
\begin{verbatim}
    CIRC_x_TYPE='ppp'
\end{verbatim}
\end{example}

Zusätzlich muss das \verb+OPT_PPP_ETHERNET_SERVER+ aktiviert werden:

\begin{description}
\config{OPT\_PPP\_ETHERNET\_SERVER}{OPT\_PPP\_ETHERNET\_SERVER}{OPTPPPETHERNETSERVER}

Diese Variable aktiviert die Unterstützung für eingehende PPPoE-Verbindungen.
Damit auch tatsächlich PPPoE-Verbindungen angenommen werden können, muss
mindestens ein PPP-Circuit den Typ ``ethernet-server'' besitzen, d.\,h. es muss
zusätzlich gelten

\begin{example}
\begin{verbatim}
    CIRC_x_TYPE='ppp'
    CIRC_x_PPP_TYPE='ethernet-server'
\end{verbatim}
\end{example}

(wobei ``x'' einen gültigen Circuit-Index darstellt).

Standard-Einstellung: \verb+OPT_PPP_ETHERNET_SERVER='no'+

Beispiel: \verb+OPT_PPP_ETHERNET_SERVER='yes'+
\end{description}

Zu den allgemeinen Circuit-Variablen kommen die folgenden, für PPP-Circuits
des Typs ``ethernet-server'' spezifischen Variablen hinzu:

\begin{description}
\config{CIRC\_x\_PPP\_ETHERNET\_SERVER\_DEV}{CIRC\_x\_PPP\_ETHERNET\_SERVER\_DEV}{CIRCxPPPETHERNETSERVERDEV}

Diese Variable enthält den Namen der Netz\-werk-Schnittstelle, über die PPPoE
genutzt werden soll. Dabei entspricht \texttt{eth0} der ersten Ethernet-Karte,
\texttt{eth1} der zweiten Ethernet-Karte u.\,s.\,w.

Beispiel: \verb+CIRC_1_PPP_ETHERNET_SERVER_DEV='eth1'+

\config{CIRC\_x\_PPP\_ETHERNET\_SERVER\_TYPE}{CIRC\_x\_PPP\_ETHERNET\_SERVER\_TYPE}{CIRCxPPPETHERNETSERVERTYPE}

PPPoE steht für die Übertragung von PPP-Paketen über Ethernet-Leitungen. Das
bedeutet, dass die zu übertragenden Daten im ersten Schritt vom ppp-Dämon in
PPP-Pakete und dann in einem zweiten Schritt für die Übertragung übers Ethernet
nochmals in PPPoE-Pakete verpackt werden. Das zweite Verpacken kann durch den
pppoe-Dämon oder durch den Kern erfolgen. Mittels
\var{CIRC\_x\_PPP\_ETHERNET\_SERVER\_TYPE} wird die Art und Weise der
PPPoE-Paketerzeugung definiert, siehe hierzu Tabelle \ref{tab:pppoe-type}.

Standard-Einstellung: \verb+CIRC_x_PPP_ETHERNET_SERVER_TYPE='kernel'+

Beispiel: \verb+CIRC_1_PPP_ETHERNET_SERVER_TYPE='daemon'+

\config{CIRC\_x\_PPP\_ETHERNET\_SERVER\_SESSIONS}{CIRC\_x\_PPP\_ETHERNET\_SERVER\_SESSIONS}{CIRCxPPPETHERNETSERVERSESSIONS}

Diese Variable enthält die Anzahl der Verbindungen, die dieser PPPoE-Server
maximal gleichzeitig verwalten kann.

Standard-Einstellung: \verb+CIRC_x_PPP_ETHERNET_SERVER_SESSIONS='64'+

Beispiel: \verb+CIRC_1_PPP_ETHERNET_SERVER_SESSIONS='100'+

\end{description}

\subsection{Beispiele}

\noindent
Beispiel 1 (Internet-Zugang über PPPoE):

\begin{example}
\begin{verbatim}
    OPT_PPP='yes'                       # PPP-Circuits aktivieren
    OPT_PPP_ETHERNET='yes'              # PPPoE-Client-Circuits aktivieren
    #
    CIRC_N='1'
    CIRC_1_NAME='DSL-Telekom'           # beliebig, aber eindeutig
    CIRC_1_TYPE='ppp'                   # das ist ein PPP-Circuit
    CIRC_1_ENABLED='yes'
    CIRC_1_NETS_IPV4_N='1'
    CIRC_1_NETS_IPV4_1='0.0.0.0/0'      # Default-Route ins Internet
    CIRC_1_CLASS_N='1'
    CIRC_1_CLASS_1='internet'           # Klasse für Internet-Anbindung
    CIRC_1_UP='yes'                     # beim Booten aktivieren
    CIRC_1_TIMES='Mo-Su:00-24:0.0:Y'
    CIRC_1_USEPEERDNS='yes'             # DNS-Server des Providers nutzen
    CIRC_1_PPP_TYPE='ethernet'          # PPPoE-Client
    CIRC_1_PPP_USERID='anonymer'        # Benutzername zur Authentifizierung
    CIRC_1_PPP_PASSWORD='surfer'        # Passwort zur Authentifizierung
    CIRC_1_PPP_ETHERNET_TYPE='kernel'   # Kernel soll PPPoE-Pakete packen
    CIRC_1_PPP_ETHERNET_DEV='eth1'      # Karte, an der das DSL-Modem hängt
    #
    CIRC_CLASS_N='1'
    CIRC_CLASS_1='internet'             # Klasse aller Internet-Circuits
\end{verbatim}
\end{example}

\noindent
Beispiel 2 (PPPoE-Server):

\begin{example}
\begin{verbatim}
    OPT_PPP='yes'                       # PPP-Circuits aktivieren
    OPT_PPP_ETHERNET_SERVER='yes'       # PPPoE-Server-Circuits aktivieren
    OPT_PPP_PEERS='yes'                 # zum Speichern der Anmeldedaten
    PPP_PEER_N='1'                      # 1x Anmeldedaten hinterlegen
    PPP_PEER_1_USERID='user'            # Benutzername vom Client
    PPP_PEER_1_PASSWORD='pass'          # Passwort vom Client
    PPP_PEER_1_CIRCUITS='pppoe-eth1'    # Anmeldedaten gelten für PPPoE-Circuit
    #
    CIRC_N='1'
    CIRC_1_NAME='pppoe-eth1'            # beliebig, aber eindeutig
    CIRC_1_TYPE='ppp'                   # das ist ein PPP-Circuit
    CIRC_1_ENABLED='yes'
    CIRC_1_UP='yes'                     # beim Booten aktivieren
    CIRC_1_TIMES='Mo-Su:00-24:0.0:Y'
    CIRC_1_PROTOCOLS='ipv4'             # IPv4 soll über die Verbindung laufen
    CIRC_1_PPP_TYPE='ethernet-server'   # PPPoE-Server
    CIRC_1_PPP_PEER_AUTH='yes'          # Client-Authentifizierung ist Pflicht
    CIRC_1_PPP_LOCALIP='192.168.222.1'  # IP-Adresse des Servers
    CIRC_1_PPP_REMOTEIP='192.168.222.2' # Start-IP-Adresse der Clients
    CIRC_1_PPP_ETHERNET_SERVER_TYPE='kernel'  # Kernel soll PPPoE-Pakete packen
    CIRC_1_PPP_ETHERNET_SERVER_DEV='eth1'     # Ethernet-Karte für PPPoE
    CIRC_1_PPP_ETHERNET_SERVER_SESSIONS='10'  # max. 10 Clients
\end{verbatim}
\end{example}

In der Dokumentation des PPP-Pakets im Abschnitt
\jump{sect:multilink-ppp}{``Multilink PPP''} finden Sie ein Beispiel gebündelter
PPPoE-Server-Verbindungen.
