% Last Update: $Id$
\section{DYNDNS}
\subsection{Hinzufügen von neuen Providern}

Das Hinzufügen von neuen Providern ist eigentlich sehr leicht, da die
Update-Scripts komplett von den Provider-Daten getrennt sind. Für einen neuen
Provider müssen folgende Dateien angepasst werden:

\subsubsection{Datei opt/etc/dyndns/provider.NAME}

Dies ist die Datei, in der definiert wird, wie ein Update bei diesem
speziellen Provider funktioniert. Meistens besteht sie nur aus einer Liste
von Variablen, da es aber ein ganz normales Shell-Script ist, können hier
auch komplexere Operationen durchgeführt werden, das sollte aber selten
nötig sein. In dieser Datei können folgende Variablen benutzt werden:

\begin{description}
\item[\$ip] Die IP des Interfaces, das den dynamischen Hostnamen erhalten
	soll.
\item[\$host] Der komplette Hostname, wie ihn der Benutzer in seiner
	Konfiguration angegeben hat.
\item[\$subdom] Alle Komponenten des Hostnamen bis zum vorletzten Punkt
	(\textbf{name}.provider.dom)
\item[\$domain] Die letzten beiden Komponenten des Hostnamen
	(name.\textbf{provider.dom})
\item[\$provider] Der symbolische Name des Providers, wie ihn der Benutzer
	in seiner Konfgurationsdatei angegeben hat.
\item[\$user] Der Benutzername für diesen Dienst.
\item[\$pass] Das dazugehörige Passwort.
\end{description}

Diese Variablen können zur klareren Abgrenzung gegenüber anderem Text mit
geschweiften Klammern geschrieben werden, aus \texttt{\$ip} wird z.B. \texttt{\$\{ip\}}.
Bei Verwendung von Anführungszeichen ist zu beachten, das
innerhalb von einfachen Anführungszeichen die oben genannten Variablen
\emph{nicht} expandiert werden, bei doppelten Anführungszeichen schon.
Als Faustregel kann man also sagen: Immer einfache Anführungszeichen
benutzen, aber sobald man Variablen benutzt, doppelte Anführungszeichen
benutzen.

Die folgenden Variablen müssen in dieser Datei definiert werden, damit das
Paket weiß, wie ein Update bei dem entsprechenden Provider funktioniert:

\begin{description}
\item[provider\_update\_type] Dies bestimmt die Art der Anfrage, die an
	den Server des Providers geschickt wird. Momentan werden unterstützt:
	\begin{description}
	\item[http] Es wird automatisiert eine bestimmte Webseite des
		Providers abgerufen und so der DynDNS-Eintrag aktualisiert.
	\item[netcat] Es wird einfach ein bestimmter Text an den Server des
		Providers geschickt, der das Update auslöst.
	\item[gnudip] Ein relativ einfaches aber sicheres Updateverfahren,
		welches über zwei HTTP-Anfragen ausgeführt wird.
	\end{description}
\item[provider\_host] Der Hostname des Providers, der bei einem Update
	kontaktiert wird.
\item[provider\_port] Der Port auf dem Host des Providers, der
	angesprochen werden soll. Der Standard-Port für HTTP ist 80.
\end{description}

Je nach Update-Typ müssen weitere Variablen angegeben werden:

\begin{description}
\item[HTTP]

\begin{description}
\item[provider\_url] Hier wird die relative URL (ohne Hostname, aber mit / am
	Anfang zum Script des Providers abgelegt. Für Beispiele bitte die Dateien
	der anderen Provider ansehen.
\item[provider\_auth] (optional) Benötigt der Provider eine Anmeldung per
	Basic Authentication, so sind hier die entsprechenden Daten anzugeben. Das
	Format ist \texttt{"{}USER:PASSWORD"{}}.
\end{description}

\item[Netcat]

\begin{description}
\item[provider\_data] Dies ist der Text, der an den Server des Providers
	geschickt wird. Siehe z.B. \texttt{provider.DYNEISFAIR}.
\end{description}

\item[GNUDip]

\begin{description}
\item[provider\_script] Der Pfad zum GNUDip-Script auf dem Server, dies ist
	meist etwas wie z.B. \texttt{'/cgi-bin/gdipupdt.cgi'}.
\end{description}
\end{description}

\subsubsection{Datei opt/dyndns.txt}

Hier müssen eine oder mehr Zeilen für den neuen Provider eingefügt werden.
Meistens reicht eine Zeile in der Art:

\begin{verbatim}
	dyndns_%_provider   NAME   etc/dyndns/provider.NAME
\end{verbatim}

Wird für den Provider HTTP und Basic Authentication benutzt, so braucht man
noch das base64-Tool:

\begin{verbatim}
	dyndns_%_provider   NAME   files/usr/local/bin/base64
\end{verbatim}

Sollten noch andere Tools benötigt werden, bitte mir vorher eine Mail
schicken, damit geprüft werden kann, ob das für das OPT\_DYNDNS geeignet ist.

\subsubsection{Datei check/dyndns.exp}

In dieser Datei muss an der langen Zeile, die mit \texttt{DYNPROVIDER = }
beginnt, der Providername mit einem senkrechten Strich von den anderen
abgetrennt, hinten angefügt werden.

\subsubsection{Datei doc/$<$Sprache$>$/tex/dyndns/dyndns\_main.tex}

In der Dokumentation einen neuen Abschnitt eintragen. Auch hier sind die
Provider alphabetisch nach dem Kurznamen, den der Benutzer in der
Config-Datei eingibt, sortiert. Das \\prov-Makro ist am Anfang der Datei
dokumentiert, genug Beispiele sollten vorhanden sein.

\subsection{Dank}

Als allererstes möchten wir dem danken, der dieses Paket ins Leben gerufen
hat und lange Zeit dieses Paket betreut hat:
Thomas Müller (\email{opt\_dyndns@s2h.cx}) hat hier hervorragende
Arbeit geleistet, ohne ihn wäre das Paket in der heutigen Form nicht
möglich gewesen.

Desweiteren möchten wir auch Marcel Döring (\email{m@rcel.to}) danken,
der das Paket einige Zeit lang gepflegt hat.

Bei der Entwicklung des Paketes haben sehr viele Leute geholfen und
Ideen beigetragen. Mein Dank gilt allen diesen fleißigen Helfern.

Außerdem danken wir Frank Meyer und dem Rest des fli4l-Tems für ihre
unermüdliche Arbeit, um den besten Router der Welt zu
basteln (Bitte nicht ganz Ernst nehmen ;-).

Weiterhin möchten wir folgenden Leuten danken, die sich mit Tips, neuen
Providern, Fehlerberichten, etc. an dem Paket beteiligt haben:

\begin{itemize}
\item Paul Bischof für den Provider AFRAID.
\item Jens Fischer schrieb das Paket opt\_dtdns, welches mich erst auf die
	Idee brachte, ein Paket für DynDNS.org zu schreiben.
\item Till Jäger schrieb das Paket opt\_cjb, welches in in opt\_dyndns
	übernommen habe.
\item Tobias Gruetzmacher hat auf \altlink{http://portfolio16.de/index.de} Informationen zu
	weiteren DynDNS-Anbietern zusammengetragen, die hier verwendet werden.
\item Die Anbieter dynamischer DNS, die auf ihren Webseiten zum Teil sehr
	gute, zum Teil weniger gute Beschreibungen des zu verwendenden Protokolls
	veröffentlicht haben.
\item Die Programmierer diverser Update-Programme für DynDNS Anbieter, aus
	deren Code schamlos geklaut wurde. ;-)
\item Heiko Ambos von dynaccess.de hat mich bei der Entwicklung der
	Unterstützung für diesen Anbieter unterstützt.
\item Dennis Neuhäuser, der die Idee hatte, die Antworten der Dienste per
	Webserver verfügbar zu machen statt sie auf der Konsole auszugeben
	und auch gleich eine erste Implementation dafür geschickt hat.
\item Lars Winkler der freundlicherweise die Änderungen, um das Paket unter
	2.0pre2 zum Laufen zu bringen zur Verfügung gestellt hat.
\item Markus Kraft und Tobias Gruetzmacher haben die Grundlage für die
	Anpassung an fli4l 2.0 gelegt.
\item Georg Bärwald für die Daten zu Selfhost.de
\item Mark C. Storck für die Daten zu Storck.org
\item Arne Biermann für den Hinweis auf den Anbieter hn.org
\item Detlef Paschke für die Daten zu dyn.ee und dyndns.dk
\item Martin Kisser für seine Idee zum Vermeiden von Updates, wenn die
	IP sich nicht geändert hat.
\item Björn Hoffmann für die Daten von DnsArt.com
\item Christian Busch für die Daten von no-ip.com.
\item Ralf Gill für das Update der Daten von selfhost.de.
\item Michael (HeinB) für eine weitere Möglichkeit sich mit fli4l selbst
	in den Fuss zu schiessen. ;-)
\item Marcus Mönnig, dito.
\end{itemize}

\subsection{Lizenz}

Copyright \copyright  2001-2002 Thomas Müller (\email{opt\_dyndns@s2h.cx}) \\
Copyright \copyright  2002-2003 Tobias Gruetzmacher (\email{fli4l@portfolio16.de}) \\
Copyright \copyright  2004-201x fli4l-Team (\email{team@fli4l.de}) \\

Dieses Programm ist freie Software. Sie können es unter
den Bedingungen der GNU General Public License, wie von der
Free Software Foundation herausgegeben, weitergeben und/oder
modifizieren, entweder unter Version 2 der Lizenz oder (wenn
Sie es wünschen) jeder späteren Version.

Die Veröffentlichung dieses Programms erfolgt in der
Hoffnung, dass es Ihnen von Nutzen sein wird, aber OHNE JEDE
GEWÄHRLEISTUNG - sogar ohne die implizite Gewährleistung
der MARKTREIFE oder der EIGNUNG FÜR EINEN BESTIMMTEN ZWECK.
Details finden Sie in der GNU General Public License.

Sie sollten eine Kopie der GNU General Public License zusammen
mit diesem Programm erhalten haben. Falls nicht, schreiben Sie
an die

\begin{verbatim}
		Free Software Foundation Inc.
		59 Temple Place
		Suite 330
		Boston MA 02111-1307 USA.
\end{verbatim}

Der Text der GNU General Public License ist auch im Internet unter
\altlink{http://www.gnu.org/licenses/gpl.txt} veröffentlicht. Eine
inoffizielle deutsche Übersetzung findet sich unter
\altlink{http://www.gnu.de/documents/gpl.de.html}
Diese Übersetzung soll jedoch nur zu einem besseren Verständnis
der GPL verhelfen, rechtsverbindlich ist die englischsprachige Version.

% vi: set ts=4 sw=4 tw=78:
