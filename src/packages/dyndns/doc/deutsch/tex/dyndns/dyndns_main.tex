% Last Update: $Id$
% Kommando für einen Tabelleneintrag
% \prov{Anbieter}{DYNDNS\_x\_PROVIDER}{Homepage}

\newcommand{\prov}[3]{
	\begin{tabular}{|l|l|l|l}
		\hline
		Anbieter & #1 \\
		\hline
		DYNDNS\_x\_PROVIDER & #2 \\
		\hline
		Homepage & #3 \\
		\hline
	\end{tabular}
	\newline\newline
}
\marklabel{sec:dyndns}{
  \section{DYNDNS - Dynamische Updates für Domain Name Services}
  }
	Dieses Paket ist dafür gedacht, automatisch bei jeder Einwahl
	einen dynamischen Hostname zu aktualisieren. Folgende Dienste werden
	unterstützt:\\

	  \prov{FreeDNS (afraid.org)}{AFRAID}{\altlink{http://freedns.afraid.org}}
	  
	  \wichtig{Als Passwort ist hier der letzte Teil (hinter dem Fragezeichen)
	  der URL anzugeben, die man auf der Homepage von Afraid.org
	  abrufen kann (Einloggen \pfeil ,,Dynamic DNS'' \pfeil Die URL,
	  die sich hinter dem Link ,,Direct URL'' versteckt). Alle anderen
	  Angaben werden ignoriert.}\\
	  
	  \prov{Companity}{COMPANITY}{\altlink{http://www.staticip.de/}}
    
    \prov{DDNSS}{DDNSS}{\altlink{http://www.ddnss.de/}}
	  
	  \prov{DHS International}{DHS}{\altlink{http://www.dhs.org/}}
	  
	  \prov{DNS2Go}{DNS2GO}{\altlink{http://dns2go.com/}}
	  
	  \prov{DNS-O-Matic}{DNSOMATIC}{\altlink{http://www.dnsomatic.com}}
	  
	  \prov{DtDNS}{DTDNS}{\altlink{http://www.dtdns.com/}}
	  
	  \prov{DynAccess}{DYNACCESS}{\altlink{http://dynaccess.de/}}
	  
	  \wichtig{DynAccess bietet im Rahmen der fli4l-DynAccess-Kooperation
	  für die Subdomains *.dyn-fli4l.de, *.dyn-fli4l.info und
	  *.dyn-eisfair.de Sondertarife an. Informationen hierzu gibt es
	  auf der Internet-Seite \altlink{http://www.dyn-fli4l.de/} bzw.
	  \altlink{http://www.dyn-eisfair.de/}.}\\
	  
	  \prov{DynDNS.org}{DYNDNS}{\altlink{http://dyn.com/}}

	  \prov{DynDNS.org (custom)}{DYNDNSC}{\altlink{http://dyn.com/standard-dns/}}
	  
	  \prov{DynDNS DK}{DYNDNSDK}{\altlink{http://dyndns.dk/}}
	  
	  \prov{dyndns:free}{DYNDNSFREE}{\altlink{http://dyndnsfree.de/}}
	  
	  \prov{eisfair.net}{DYNEISFAIR}{\altlink{http://www.intersales.de/it-infrastruktur/dyneisfair.html}}
	  
	  \wichtig{Mit der Benutzung dieses Dienstes unterstützt man die Arbeit der fli4l- und eisfair-Entwickler.}\\
	  
	  \prov{DyNS}{DYNSCX}{\altlink{http://www.dyns.cx/}}
	  
	  \prov{GnuDIP Dynamic DNS}{GNUDIP}{\altlink{http://gnudip2.sourceforge.net/}}
	  
	  \prov{Provider Hurricane Electric}{HE}{\altlink{https://dns.he.net/}}
	  
	  \prov{IN-Berlin e.V.}{INBERLIN}{\altlink{http://www.in-berlin.de}}
	  
	  \prov{INWX GmbH & Co. KG}{INWX}{\altlink{https://www.inwx.de/}}
	  
	  \prov{KONTENT}{KONTENT}{\altlink{http://www.kontent.de/}}
	  
	  \prov{Nerdcamp.net}{NERDCAMP}{\altlink{http://nerdcamp.net/dynamic/dns.cgi}}
	  
	  \prov{No-IP.com}{NOIP}{\altlink{http://www.no-ip.com/}}
	  
	  \prov{noxaDynDNS}{NOXA}{\altlink{http://www.noxa.de/}}
	  
	  \prov{OVH.DE}{OVHDE}{\altlink{http://www.ovh.de/}}
	  
	  \prov{PHPDYN}{PHPDYN}{\altlink{http://www.webnmail.de/phpdyn/}}
	  
	  \wichtig{diese Lösung muß man selber hosten}\\
	  
	  \prov{Regfish.com}{REGFISH}{\altlink{http://www.regfish.de/}}
	  
	  \prov{SelfHost.de}{SELFHOST}{\altlink{http://selfhost.de/cgi-bin/selfhost}}

	  \prov{Securepoint Dynamic DNS Service}{SPDNS}{\altlink{http://www.spdns.de/}}
	  
	  \prov{Strato}{STRATO}{\altlink{http://www.strato.de/}}
	  
	  \prov{T-Link.de}{TLINK}{\altlink{http://www.t-link.de/}}
      
      \prov{twodns.de}{TWODNS}{\altlink{http://www.twodns.de/}}
				  
	  \prov{ZoneEdit.com}{ZONEEDIT}{\altlink{http://zoneedit.com/}}
	  

	Wir versuchen diese Daten aktuell zu halten. Trotzdem übernehmen wir
	keine Haftung für die Richtigkeit dieser Daten. Wer einen Fehler oder
	eine Änderung entdeckt sollte eine Mail an das fli4l-Team
	(\email{team@fli4l.de}) schicken.

	Diese Liste ist komplett, andere Provider werden ohne Änderung nicht
	unterstützt. Wie man das Paket um eigene Anbieter erweitern kann, steht
	im Anhang.

	Der dynamische Hostname wird automatisch bei jeder Einwahl ins Internet
	aktualisiert. Das Paket beinhaltet eine Sperre, die das mehrmalige
	aktualisieren der gleichen IP verhindert, da dies bei einigen
	DynDNS-Anbietern nicht gerne gesehen wird und im Extremfall zur Sperrung
	des Accounts führen kann.

	Hinweis: Es kann einige Minuten dauern, bis die Änderung des dynamischen
	Hostnamens wirksam wird.

	Bevor man mit der Einrichtung dieses Paketes beginnen kann, muss man sich
	bei einem der oben genannten Anbietern einen Account holen. Falls man
	das schon hat, kann man sofort loslegen. Hat man noch keinen Account,
	so kann man sich an obiger Tabelle orientieren, um einen Hostname zu
	finden, der den Ansprüchen genügt und den persönlichen Geschmack trifft.

	Für die nun folgende Konfiguration benötigt man folgende Daten:

	\begin{itemize}
		\item Name des Anbieters
		\item Benutzername
		\item Passwort
		\item Den DynDNS-Hostnamen
	\end{itemize}

	Die benötigten Angaben können je nach Anbieter variieren, es wird versucht
	eine möglichst konsistene Konfiguration zu bieten. Manchmal ist z.B.
	der Hostname gleich dem Benutzernamen, in so einem Fall werden wir 
	versuchen, immer das Host-Feld zu benutzen und den Benutzernamen einfach
	ignorieren. Jetzt aber los:

\begin{description}

\config{OPT\_DYNDNS}{OPT\_DYNDNS}{OPTDYNDNS}

    {Steht dieser Parameter auf \verb*?'yes'?, wird \var{OPT\_DYNDNS} aktiviert.}

\config{DYNDNS\_SAVE\_OUTPUT}{DYNDNS\_SAVE\_OUTPUT}{DYNDNSSAVEOUTPUT}

	{Wird dieser Parameter auf \verb*?'yes'? gestellt, wird das Ergebnis
	der DynDNS-Anfrage(n) in einer Datei gespeichert und kann über
	den Webserver\footnote{OPT\_HTTPD im Paket \jump{OPTHTTPD}{HTTPD} auf
	\altlink{http://www.fli4l.de/download/stabile-version/}} abgefragt werden.}

\config{DYNDNS\_N}{DYNDNS\_N}{DYNDNSN}

	{Hat man bei mehreren DynDNS-Anbietern einen Account und will deswegen
	bei jeder Einwahl mehrere Namen updaten, so ist dieser Wert
	entsprechend anzupassen.}

\config{DYNDNS\_x\_PROVIDER}{DYNDNS\_x\_PROVIDER}{DYNDNSxPROVIDER}

	{Hier wird der Name des zu benutzenden Providers angegeben (siehe
	Tabelle weiter oben und Hinweis in der Config-Datei).}

\config{DYNDNS\_x\_USER}{DYNDNS\_x\_USER}{DYNDNSxUSER}

	{Benutzername bei dem DynDNS-Anbieter. Häufig ist dies eine
	E-Mail-Adresse, ein selbstgewählter Name oder gleich dem
	DynDNS-Hostname.}

\config{DYNDNS\_x\_PASSWORD}{DYNDNS\_x\_PASSWORD}{DYNDNSxPASSWORD}

	{Hier ist das Passwort des DynDNS-Accounts anzugeben. Aufpassen,
	dass niemand anderes beim Editieren der Config-Datei zusieht!}

\config{DYNDNS\_x\_HOSTNAME}{DYNDNS\_x\_HOSTNAME}{DYNDNSxHOSTNAME}

	{Hier ist der \emph{komplette} DynDNS-Hostname des Accounts
	einzutragen. Beispielsweise könnte hier folgendes stehen:

	\begin{itemize}
		\item \texttt{cool.nerdcamp.net}
		\item \texttt{user.dyndns.org}
		\item \texttt{fli4luser.fli4l.net}
	\end{itemize}

	}

\config{DYNDNS\_x\_UPDATEHOST}{DYNDNS\_x\_UPDATEHOST}{DYNDNSxUPDATEHOST}

	{Hier wird für den Provider PHPDYN angegeben, auf welchem Host der
	Updater installiert ist. Dies ist nötig, da dies kein herkömmlicher
	Provider ist sondern nur ein Script, welches einen PowerDNS Server
	mit MySQL aktualisiert und welches unter der GPL steht.
	}

\config{DYNDNS\_x\_CIRCUIT}{DYNDNS\_x\_CIRCUIT}{DYNDNSxCIRCUIT}

	{Hier kann angegeben werden, bei welchen Circuits dieser Hostname
	aktualisiert wird. Die einzelnen Circuits werden mit Leerzeichen
	voneinander getrennt. Es kann z.B. erwünscht sein, den Hostnamen nur bei
	der DSL-Einwahl zu benutzen. Hier ein paar Beispiele:

\begin{example}
\begin{verbatim}
        DYNDNS_1_CIRCUIT='1 2 3'           # Nur ISDN: Circuits 1 bis 3
        oder
        DYNDNS_1_CIRCUIT='pppoe'           # Nur DSL: pppoe-Circuit
        oder
        DYNDNS_1_CIRCUIT='dhcp'            # Update bei DHCP-Providern
                                           # (opt_dhcp wird benötigt)
        oder
        DYNDNS_1_CIRCUIT='pppoe 1'         # DSL und ISDN
\end{verbatim}
\end{example}
	}
\config{DYNDNS\_x\_RENEW}{DYNDNS\_x\_RENEW}{DYNDNSxRENEW}
Manche Provider erwarten, dass alle n Tage ein Update ausgeführt wird,
auch wenn sich die IP nicht verändert hat. Dieses Intervall kann man
hier angeben. Gibt man keinen Wert an, wird nach 29 Tagen ein Update
durchgeführt.

Zu beachten ist hierbei, dass ein Update nur bei einer Einwahl
angestoßen wird - also bei einer Einwahl über DSL oder ISDN oder einer
Erneuerung einer Lease bei einem via DHCP konfigurierten Interface,
wie man es bei einem Kabelmodem findet. Findet über längere Zeit keine
Einwahl statt, muß man das Update auf andere Weise anstoßen.

\config{DYNDNS\_x\_EXT\_IPV4}{DYNDNS\_x\_EXT\_IPV4}{DYNDNSxEXTIPV4}
\config{DYNDNS\_x\_EXT\_IPV6}{DYNDNS\_x\_EXT\_IPV6}{DYNDNSxEXTIPV6}

Mit dieser Variable wird die Methode, mit der die externe IP Adresse
ermittelt wird, konfiguriert. Im Moment gibt es die Möglichkeit mit
\verb*?'none'? überhaupt keinen externen Dienst nach der IP Adresse zu 
befragen sondern direkt die externe IP Adresse anhand des WAN Interfaces
zu bestimmen. Das funktioniert in der Regel aber nur bei WAN Verbindungen,
die direkt auf dem fli4l terminieren, wie z.B. DSL via PPPoE. Mit der
Einstellung \verb*?'dyndns'? wird die beim Update verwendete IP Adresse
über den externen Dienst von checkip.dyndns.org ermittelt. Wird die
Einstellung \verb*?'stun'? benutzt wird die Liste der STUN Server der
Reihe nach abgefragt bis eine erfolgreiche Antwort geliefert wird. Die
Nutzung eines externen Dienstes zur Ermittlung der IP Adresse ist notwendig,
wenn der  Router selbst nicht derjenige ist, der die externe IP erhält.
Dabei ist zu beachten, dass der Router in diesem Falle momentan eine Änderung
der externen IP nicht mitbekommt, den dyndns-Namenseintrag also nicht zeitnah
aktualisieren kann.

\config{DYNDNS\_x\_LOGIN}{DYNDNS\_x\_LOGIN}{DYNDNSxLOGIN}

Manche Provider erfordern, dass sich der Benutzer regelmäßig auf ihrer
Internet-Seite unter seinem Benutzerkonto anmeldet, damit der Dienst nicht
deaktiviert wird. Wenn diese Variable auf \verb*?'yes'? gesetzt ist, erledigt
das der fli4l für Sie. Allerdings funktioniert das nur, wenn das dyndns-Paket
für den jeweiligen Provider vorbereitet wurde. Momentan ist eine solche
regelmäßige Anmeldung nur für den Provider ``DYNDNS'' erforderlich und möglich.
Bedenken Sie bitte auch, dass die Nutzung dieser Funktion
\var{OPT\_EASYCRON='yes'} im Paket easycron erfordert.

\config{DYNDNS\_LOGINTIME}{DYNDNS\_LOGINTIME}{DYNDNSLOGINTIME}

Nutzen Sie einen Provider, bei dem sich der fli4l regelmäßig anmelden soll,
um eine Deaktivierung des Dienstes zu verhindern (s.o.), dann können Sie mit
dieser Variable einstellen, wann diese Anmeldung stattfinden soll. Benötigt
wird eine Zeitangabe im Cron-Format; zu Details lesen Sie sich bitte die
Dokumentation des easycron-Pakets durch. Die Standardbelegung lautet
\texttt{0 8 * * *}, was einer täglichen Anmeldung um acht Uhr morgens
entspricht.

\config{DYNDNS\_ALLOW\_SSL}{DYNDNS\_ALLOW\_SSL}{DYNDNSALLOWSSL}

Ist diese Variable auf \verb*?'yes'? gesetzt, wird das Update wenn möglich
über SSL (verschlüsselte Verbindung) durchgeführt.

\config{DYNDNS\_LOOKUP\_NAMES}{DYNDNS\_LOOKUP\_NAMES}{DYNDNSLOOKUPNAMES}
Ein Update der IP sollte eigentlich nur erfolgen, wenn sich die IP
geändert hat. Viele fli4l-Router haben jedoch keinen permanenten
Speicher, auf der die Information über die registierte IP gesichert
werden kann, daher steht diese Information direkt nach dem Booten dort
nicht zur Verfügung. Um trotzdem unnötige Updates zu vermeiden, kann
fli4l in dieser Situation (und nur in dieser Situation) beim
Namensdienst nach der aktuell registrierten IP fragen. Die ermittelte
IP wird dann zwischengespeichert und für jedes weitere Update genutzt.

Zu beachten ist dabei, dass nach einem Reboot das Update-Intervall neu
beginnt, wenn fli4l den Namensdienst zur Ermittlung der IP nutzt.

\config{DYNDNS\_DEBUG\_PROVIDER}{DYNDNS\_DEBUG\_PROVIDER}{DYNDNSDEBUGPROVIDER}

Ist   diese  Variable   auf  \verb*?'yes'?   gesetzt,  wird   ein   trace  des
Update-Vorgangs  aufgezeichnet, so  dass man  im Nachhinein  bei einem
Problem prüfen kann, was schief gegangen ist. 
Default: \verb*?DYNDNS_DEBUG_PROVIDER='no'?

\config{OPT\_STUN}{OPT\_STUN}{OPTSTUN}

Mit \verb*?'yes'? wird die Funktionalität zur Ermittlung der externen
IP-Adresse über STUN-Server aktiviert

\config{STUN\_SERVER\_N}{STUN\_SERVER\_N}{STUNSERVERN}

Mit dieser Variable wird die Anzahl der STUN-Server definiert.

\config{STUN\_SERVER\_x}{STUN\_SERVER\_x}{STUNSERVERx}

FQDN des STUN-Server, optional kann der FQDN um den zu verwendenden Port
ergänzt werden

\begin{example}
\begin{verbatim}
       STUN_SERVER_1='stun.l.google.com:19302'
       STUN_SERVER_2='stun1.l.google.com:19302'
       STUN_SERVER_3='stun2.l.google.com:19302'
       STUN_SERVER_4='stun3.l.google.com:19302'
       STUN_SERVER_5='stun4.l.google.com:19302'
       STUN_SERVER_6='stun01.sipphone.com'
       STUN_SERVER_7='stun.ekiga.net'
       STUN_SERVER_8='stun.fwdnet.net'
       STUN_SERVER_9='stun.ideasip.com' 
\end{verbatim}
\end{example}

\end{description}

% vim: tabstop=4
