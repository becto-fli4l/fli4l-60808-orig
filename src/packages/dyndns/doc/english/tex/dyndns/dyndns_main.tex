% Synchronized to r44713
% Kommando für einen Tabelleneintrag
% \prov{Provider}{DYNDNS\_x\_PROVIDER}{Homepage}

\newcommand{\prov}[3]{
	\begin{tabular}{|l|l|l|l}
		\hline
		Provider & #1 \\
		\hline
		DYNDNS\_x\_PROVIDER & #2 \\
		\hline
		Homepage & #3 \\
		\hline
	\end{tabular}
	\newline
}

\marklabel{sec:dyndns}{
  \section{DYNDNS - Dynamic Update For Domain Name Services}
  }
	This package is used to update a dynamic hostname at every dial-in process.
	The following services are supported:\\
	
	  \prov{FreeDNS (afraid.org)}{AFRAID}{\altlink{http://freedns.afraid.org}}
	  
	  \wichtig{Provide the last part of the URL downloadable on Afraid.org's 
	  homepage (behind the question mark) as password 
	  (Login \pfeil ,,Dynamic DNS'' \pfeil The URL behind the Link ,,Direct URL''). 
	  All other informations will be ignored.}\\
	  
	  \prov{Companity}{COMPANITY}{\altlink{http://www.staticip.de/}}
    
    \prov{DDNSS}{DDNSS}{\altlink{http://www.ddnss.de/}}
	  
	  \prov{DHS International}{DHS}{\altlink{http://www.dhs.org/}}
	  
	  \prov{DNS2Go}{DNS2GO}{\altlink{http://dns2go.com/}}
	  
	  \prov{DNS-O-Matic}{DNSOMATIC}{\altlink{http://www.dnsomatic.com}}
	  
	  \prov{DtDNS}{DTDNS}{\altlink{http://www.dtdns.com/}}
	  
	  \prov{DynAccess}{DYNACCESS}{\altlink{http://dynaccess.de/}}
	  
	  \wichtig{DynAccess offers special charges for the subdomains *.dyn-fli4l.de, 
	  *.dyn-fli4l.info and *.dyn-eisfair.de because of a fli4l-DynAccess cooperation. 
	  Informations can be found here: \altlink{http://www.dyn-fli4l.de/} or 
	  \altlink{http://www.dyn-eisfair.de/}.}\\
	  
	  \prov{DynDNS.org}{DYNDNS}{\altlink{http://dyn.com/}}

	  \prov{DynDNS.org (custom)}{DYNDNSC}{\altlink{http://dyn.com/standard-dns/}}
	  
	  \prov{DynDNS DK}{DYNDNSDK}{\altlink{http://dyndns.dk/}}
	  
	  \prov{dyndns:free}{DYNDNSFREE}{\altlink{http://dyndnsfree.de/}}
	  
	  \prov{eisfair.net}{DYNEISFAIR}{\altlink{http://www.intersales.de/it-infrastruktur/dyneisfair.html}}
	  
	  \wichtig{By using this service you support the work of the fli4l- and eisfair-developers.}\\
	  
	  \prov{DyNS}{DYNSCX}{\altlink{http://www.dyns.cx/}}
	  
	  \prov{GnuDIP Dynamic DNS}{GNUDIP}{\altlink{http://gnudip2.sourceforge.net/}}

	  \prov{Provider Hurricane Electric}{HE}{\altlink{https://dns.he.net/}}

	  \prov{IN-Berlin e.V.}{INBERLIN}{\altlink{http://www.in-berlin.de}}
	  
	  \prov{INWX GmbH & Co. KG}{INWX}{\altlink{https://www.inwx.de/}}
      
	  \prov{KONTENT}{KONTENT}{\altlink{http://www.kontent.de/}}
	  
	  \prov{Nerdcamp.net}{NERDCAMP}{\altlink{http://nerdcamp.net/dynamic/dns.cgi}}
	  
	  \prov{No-IP.com}{NOIP}{\altlink{http://www.no-ip.com/}}
	  
	  \prov{noxaDynDNS}{NOXA}{\altlink{http://www.noxa.de/}}
	  
	  \prov{OVH.DE}{OVHDE}{\altlink{http://www.ovh.de/}}
	  
	  \prov{PHPDYN}{PHPDYN}{\altlink{http://www.webnmail.de/phpdyn/}}
	  
	  \wichtig{you have to host this type for yourself}\\
	  
	  \prov{Regfish.com}{REGFISH}{\altlink{http://www.regfish.de/}}
	  
	  \prov{SelfHost.de}{SELFHOST}{\altlink{http://selfhost.de/cgi-bin/selfhost}}

	  \prov{Securepoint Dynamic DNS Service}{SPDNS}{\altlink{http://www.spdns.de/}}
    
	  \prov{Strato}{STRATO}{\altlink{http://www.strato.de/}}
	  
	  \prov{T-Link.de}{TLINK}{\altlink{http://www.t-link.de/}}
      
      \prov{twodns.de}{TWODNS}{\altlink{http://www.twodns.de/}}
				  
	  \prov{ZoneEdit.com}{ZONEEDIT}{\altlink{http://zoneedit.com/}}
  
	We try to keep this list up-to-date but do not rely too closely on it. There
	is no liability at all for the correctness of this data. You may report errors
	or changes detected by sending a mail to the fli4l team (\email{team@fli4l.de}).

	This is a complete list. Other providers are not supported without changes. How 
	to expand the package to support new providers can be found in the appendix.\\

	The dynamic hostname will be actualized with every internet dial-in. 
	The package includes a lock that prevents repeated updates of the 
	same IP as this is frowned with some DynDNS providers and in extreme 
	cases can lead to account blocking.\\
	
	Note: The changing of the dynamic hostname may take some minutes.\\
	
	Before using the package you have to aquire an account with one of the 
	providers named above. If you already have an account you are ready to 
	start. If you have no account yet, you can be guided by the table above 
	to find a host name which fulfills the requirements and meets the personal 
	taste.\\
	
	For the configuration you will need the following data:
	
	\begin{itemize}
		\item Name of the provider
		\item Username
		\item Password
		\item DynDNS-hostname
	\end{itemize}

	The Data my vary with the provider while we try to provide a 
	consistent configuration. Sometimes the hostname equals to the 
	username, in such cases we will try to use the Host-field and 
	ignore the username.

\begin{description}

\config{OPT\_DYNDNS}{OPT\_DYNDNS}{OPTDYNDNS}

    {Setting this to \verb*?'yes'? activates \var{OPT\_DYNDNS}.}

\config{DYNDNS\_SAVE\_OUTPUT}{DYNDNS\_SAVE\_OUTPUT}{DYNDNSSAVEOUTPUT}

	{By activating this with \verb*?'yes'? the result of the DynDNS query 
	will be saved to a file shown by the webserver
	\footnote{OPT\_HTTPD}}

\config{DYNDNS\_N}{DYNDNS\_N}{DYNDNSN}

	{Change this value if you have accounts with more DynDNS providers 
	and therefore want to update several names with every dial-in.}

\config{DYNDNS\_x\_PROVIDER}{DYNDNS\_x\_PROVIDER}{DYNDNSxPROVIDER}

	{Specify the name of the provider (see table above and 
	hints in the config file).}

\config{DYNDNS\_x\_USER}{DYNDNS\_x\_USER}{DYNDNSxUSER}

	{Your username for the DynDNS provider. Mostly an email address, 
	a name you choose for yourself when registering or the DynDNS hostname.}

\config{DYNDNS\_x\_PASSWORD}{DYNDNS\_x\_PASSWORD}{DYNDNSxPASSWORD}

	{Your passwort for the DynDNS account. Be aware to keep your secret!}

\config{DYNDNS\_x\_HOSTNAME}{DYNDNS\_x\_HOSTNAME}{DYNDNSxHOSTNAME}

	{The \emph{complete} DynDNS hostname for the account. 
	Examples:

	\begin{itemize}
		\item \texttt{cool.nerdcamp.net}
		\item \texttt{user.dyndns.org}
		\item \texttt{fli4luser.fli4l.net}
	\end{itemize}

	}

\config{DYNDNS\_x\_UPDATEHOST}{DYNDNS\_x\_UPDATEHOST}{DYNDNSxUPDATEHOST}

	{For the provider PHPDYN specify here on which host the updater
	is installed. You need this because of PHPDYN not being a real 
	provider but only a gpl'd script actualizing a PowerDNS Server with 
	MySQL.
	}

\config{DYNDNS\_x\_CIRCUIT}{DYNDNS\_x\_CIRCUIT}{DYNDNSxCIRCUIT}

	{Set the circuits for which dialing in should update the hostname. 
	Circuits are separated by spaces. It may be useful to update
	hostnames only when dialing in via DSL.
	Some examples:

\begin{example}
\begin{verbatim}
        DYNDNS_1_CIRCUIT='1 2 3'           # Only ISDN: Circuits 1 to 3
        or
        DYNDNS_1_CIRCUIT='pppoe'           # Only DSL: pppoe-Circuit
        or
        DYNDNS_1_CIRCUIT='dhcp'            # Update with DHCP providers
                                           # (opt_dhcp needed)
        or
        DYNDNS_1_CIRCUIT='pppoe 1'         # DSL and ISDN

        or
        DYNDNS_1_CIRCUIT='static'          # fli4l i.e. behind a LTE Router

\end{verbatim}
\end{example}
	}
\config{DYNDNS\_x\_RENEW}{DYNDNS\_x\_RENEW}{DYNDNSxRENEW}
Some providers expect an update every n days even without your IP 
having changed. This interval may be set here. If no value is 
given an update will be forced every 29 days.

It should be noted that an update is triggered only when dialing. 
This means: DSL or ISDN connections or a renewing of a lease 
of an interface configured via DHCP like in a cable modem. If 
no dialing occurs for a longer time period you have to trigger 
the update in other ways.

\config{DYNDNS\_x\_EXT\_IPV4}{DYNDNS\_x\_EXT\_IPV4}{DYNDNSxEXTIPV4}
\config{DYNDNS\_x\_EXT\_IPV6}{DYNDNS\_x\_EXT\_IPV6}{DYNDNSxEXTIPV6}

This variable configures the method by which the external IP address
is detected. By setting this to \verb*?'none'? no external service will be
queried for the IP address and the IP address of the WAN interface will
be used instead. This usually works only with WAN connections
which terminate directly on the fli4l, such as PPPoE via DSL.
If specifying \verb*?'dyndns'? the IP address used when updating
is determined by using the external service of checkip.dyndns.org.
When using \verb*?'stun'? the list of STUN servers is queried one by
one until a valid response is returned. The use of an external service
to determine the IP address is necessary if the router itself is not
getting the external IP. It should be noted that the router will not note
a change of the external IP in this case and thus can't update the dyndns
name entry immedeately.

\config{DYNDNS\_x\_LOGIN}{DYNDNS\_x\_LOGIN}{DYNDNSxLOGIN}

Some providers require the user to log in to their Web page regularly 
by using her/his user account to prevent the service from being deactivated. 
If this variable is set to \verb*?'yes'?, the FLI4L will do that for you. 
However, this only works if the dyndns package is prepared for this 
provider. Such a regular registration is currently only necessary and 
possible for the provider ``DYNDNS''. Please remember also that the 
use of this function requires \var{OPT\_EASYCRON='yes'} in package 
easycron.

\config{DYNDNS\_LOGINTIME}{DYNDNS\_LOGINTIME}{DYNDNSLOGINTIME}

If you use a provider FLI4L should log into regularly to prevent 
deactivation of the service (see above), you can use this variable 
to configure at what time this login process is intended to take place. A time 
value in Cron format is required here, for details please read the
Documentation of package easycron. The default setting is 
\texttt{0 8 * * *}, meaning a daily login at eight o'clock 
in the morning.

\config{DYNDNS\_ALLOW\_SSL}{DYNDNS\_ALLOW\_SSL}{DYNDNSALLOWSSL}

Setting this variable to \verb*?'yes'? will update over an encrypted 
SSL connection if possible.

\config{DYNDNS\_LOOKUP\_NAMES}{DYNDNS\_LOOKUP\_NAMES}{DYNDNSLOOKUPNAMES}
The IP should only be updated if it really changed. Many fli4l routers don't 
have a permanent data storage like a harddisk where this information 
could be saved to be present at boot time. To prevent unnecessary 
updates flil4 may query name servers (only in this case) for its 
actually registrated IP. The information will be saved and used 
for further updates.

Note that after a reboot a new update interval will start if fli4l 
uses a name server to detect its IP.

\config{DYNDNS\_DEBUG\_PROVIDER}{DYNDNS\_DEBUG\_PROVIDER}{DYNDNSDEBUGPROVIDER}

Setting this variable to \verb*?'yes'? will record a trace of the update 
process for debugging purposes. 
Default: \verb*?DYNDNS_DEBUG_PROVIDER='no'?

\config{OPT\_STUN}{OPT\_STUN}{OPTSTUN}

Setting \verb*?'yes'? enables the functionality for determining the external
IP address via a STUN server.

\config{STUN\_SERVER\_N}{STUN\_SERVER\_N}{STUNSERVERN}

This variable defines the number of STUN servers.

\config{STUN\_SERVER\_x}{STUN\_SERVER\_x}{STUNSERVERx}

FQDN of the STUN server, optionally the FQDN may be expanded by the port used.

\begin{example}
\begin{verbatim}
       STUN_SERVER_1='stun.l.google.com:19302'
       STUN_SERVER_2='stun1.l.google.com:19302'
       STUN_SERVER_3='stun2.l.google.com:19302'
       STUN_SERVER_4='stun3.l.google.com:19302'
       STUN_SERVER_5='stun4.l.google.com:19302'
       STUN_SERVER_6='stun01.sipphone.com'
       STUN_SERVER_7='stun.ekiga.net'
       STUN_SERVER_8='stun.fwdnet.net'
       STUN_SERVER_9='stun.ideasip.com' 
\end{verbatim}
\end{example}

\end{description}

% vim: tabstop=4
