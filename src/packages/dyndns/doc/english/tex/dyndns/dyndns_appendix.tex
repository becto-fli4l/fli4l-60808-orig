% Synchronized to r29817
\section{DYNDNS}
\subsection{Adding Of New Providers}

Adding new providers is easy because update-scripts are separated
from provider data completely. For a new provider
adapt the following files:

\subsubsection{opt/etc/dyndns/provider.NAME}

This file defines how an update is working with this provider. It mostly 
consists only of a list of variables but is a normal shell script that even 
allows complex operations to be done. This should not be necessary in most 
cases. These variables can be used in the file:

\begin{description}
\item[\$ip] The IP of the interface that should get the dynamic hostname.
\item[\$host] The complete hostname the user specified in his configuration.
\item[\$subdom] All components of the hostname ending with the dot next to last
	(\textbf{name}.provider.dom)
\item[\$domain] the both last components of the hostname
	(name.\textbf{provider.dom})
\item[\$provider] The symbolic name of the provider the user specified in his configuration file.
\item[\$user] The username for this service.
\item[\$pass] The password.
\end{description}

These variables can be put in curly brackets to be cleary distinguishable 
from normal text, \texttt{\$ip} i.e. becomes \texttt{\$\{ip\}}. 
If using quotation marks it should be noted that
within single quotes the variables mentioned above
are \emph{not} expanded while this works with double quotes.
As a rule of thumb: Always use single quotes but when using 
variables double quotes are needed.

The following variables must be defined in this file in order 
to get an update working with the provider:

\begin{description}
\item[provider\_update\_type] This determines the type of query sent to the 
	provider's server. These types are supported at the moment:
	\begin{description}
	\item[http] A predefined website of the provider will be loaded to 
		update the DynDNS-entry.
	\item[netcat] A predefined text will be sent to the provider's server 
		triggering an update.
	\item[gnudip] An update process relatively easy and secure done 
		by two HTTP-queries.
	\end{description}
\item[provider\_host] The provider's hostname that is to be contacted
	during an update.
\item[provider\_port] The port to be contacted on the provider's host. 
	Standard-port for HTTP is 80.
\end{description}

Depending on the update type further variables have to be specified:

\begin{description}
\item[HTTP]

\begin{description}
\item[provider\_url] The relative URL (without the hostname, but with an / at
	the beginning) for the provider script. For examples please have a look at 
	the files for other providers.
\item[provider\_auth] (optional) If the provider needs a login via 
	basic authentication provide the information needed here. 
	The format is \texttt{"{}USER:PASSWORD"{}}.
\end{description}

\item[Netcat]

\begin{description}
\item[provider\_data] The text to be sent to the provider's server. 
	See \texttt{provider.DYNEISFAIR} as an example.
\end{description}

\item[GNUDip]

\begin{description}
\item[provider\_script] The path tot he GNUDip-script on the server, mostly 
	something like \texttt{'/cgi-bin/gdipupdt.cgi'}.
\end{description}
\end{description}

\subsubsection{opt/dyndns.txt}

One or more lines for the new provider have to be inserted here.
Usually a line like that is enough:

\begin{verbatim}
	dyndns_%_provider   NAME   etc/dyndns/provider.NAME
\end{verbatim}

If HTTP and Basic Authentication are used by the provider you will 
need the base64 executable:

\begin{verbatim}
	dyndns_%_provider   NAME   files/usr/local/bin/base64
\end{verbatim}

If other tools are needed sent an email
so we can validate if this is suitable for OPT\_DYNDNS.

\subsubsection{check/dyndns.exp}

In this file the provider name has to be added at the end of the 
long line starting with \texttt{DYNPROVIDER = }, seperated by a '|'.

\subsubsection{doc/$<$Language$>$/tex/dyndns/dyndns\_main.tex}

Add a new paragraph to the documentation. The providers have to be sorted 
alphabetically by the short name given by the user in the config file. The 
prov-macro is documented at the beginning of the file,
enough examples should be present.

\subsection{Note Of Thanks}

At first I wish to thank Thomas Müller (\email{opt\_dyndns@s2h.cx}) 
who originally developed this package and maintained it for a long time.
He has done exceptional work here, without him this packages would not 
have been possible.

I would like to thank as well Marcel Döring (\email{m@rcel.to}),
who maintained the package for quite some time.

A lot of people have been helping and providing ideas at the 
development of the package. Many thanks also to all those 
hard-working people.

Further thanks got to all the people contributing to the package by 
providing tips, new providers, bug reports and so on:

Last but not least my thanks go to Frank Meyer and the rest of the fli4l 
team for their tireless work to tinker with the best router in the world 
(;-) Please do not take this too serious).

%\begin{itemize}
%\item Paul Bischof for the provider AFRAID.
%\end{itemize}

%Bevor ich das Paket übernommen habe, haben schon sehr viele Leute an diesem
%Paket mitgewirkt. Natürlich soll ihr Beitrag nicht in Vergessenheit geraten,
%daher zitiere ich hier die Dankesliste, die Thomas Müller (d.h. ich = Thomas)
%in seinem Paket hatte:

%\begin{itemize}
%\item Jens Fischer schrieb das Paket opt\_dtdns, welches mich erst auf die
%	Idee brachte, ein Paket für DynDNS.org zu schreiben.
%\item Till Jäger schrieb das Paket opt\_cjb, welches in in opt\_dyndns
%	übernommen habe.
%\item Tobias Gruetzmacher hat auf \altlink{http://portfolio16.de/index.de} Informationen zu
%	weiteren DynDNS-Anbietern zusammengetragen, die ich hier verwendet habe.
%\item Die Anbieter dynamischer DNS, die auf ihren Webseiten zum Teil sehr
%	gute, zum Teil weniger gute Beschreibungen des zu verwendenden Protokolls
%	veröffentlicht haben.
%\item Die Programmierer diverser Update-Programme für DynDNS Anbieter, aus
%	deren Code ich schamlos geklaut habe. ;-)
%\item Heiko Ambos von dynaccess.de hat mich bei der Entwicklung der
%	Unterstützung für diesen Anbieter unterstützt.
%\item Dennis Neuhäuser, der die Idee hatte, die Antworten der Dienste per
%	Webserver verfügbar zu machen statt sie auf der Konsole auszugeben
%	(geniale Idee, wieso bin ich nicht selbst darauf gekommen?), und mir
%	auch gleich eine Implementation dafür geschickt hat (die ich dann
%	umgehend so weit aufgebohrt habe, dass er sie vermutlich nicht
%	mehr wiedererkennt).
%\item Lars Winkler der freundlicherweise die Änderungen, um das Paket unter
%	2.0pre2 zum Laufen zu bringen zur Verfügung gestellt hat.
%\item Markus Kraft und Tobias Gruetzmacher haben die Grundlage für die
%	Anpassung an fli4l 2.0 gelegt.
%\item Diverse andere Leute haben mir ihre jeweilige Portierung auf fli4l 2.0
%	geschickt. Ich muss zu meiner Schande gestehen, dass ich mir die wenigsten 
%	davon angesehen habe.
%\item Georg Bärwald für die Daten zu Selfhost.de
%\item Mark C. Storck für die Daten zu Storck.org
%\item Arne Biermann für den Hinweis auf den Anbieter hn.org
%\item Detlef Paschke für die Daten zu dyn.ee und dyndns.dk
%\item Martin Kisser für seine Idee zum Vermeiden von Updates, wenn die
%	IP sich nicht geändert hat.
%\item Björn Hoffmann für die Daten von DnsArt.com
%\item Christian Busch für die Daten von no-ip.com.
%\item Ralf Gill für das Update der Daten von selfhost.de.
%\item Michael (HeinB) für eine weitere Möglichkeit sich mit fli4l selbst
%	in den Fuss zu schiessen. ;-)
%\item Marcus Mönnig, dito.
%\end{itemize}

\subsection{Licence}

Copyright \copyright  2001-2002 Thomas Müller (\email{opt\_dyndns@s2h.cx}) \\
Copyright \copyright  2002-2003 Tobias Gruetzmacher (\email{fli4l@portfolio16.de}) \\
Copyright \copyright  2004-201x fli4l-Team (\email{team@fli4l.de}) \\

This program is free software. It is distributed under the terms 
of the GNU General Public License as provided by the Free 
Software Foundation. For further informations on the licence please 
refer to \altlink{http://www.gnu.org/licenses/gpl.txt}.

% vi: set ts=4 sw=4 tw=78:
