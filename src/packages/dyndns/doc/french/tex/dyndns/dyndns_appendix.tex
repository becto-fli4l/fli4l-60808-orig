% Do not remove the next line
% Synchronized to r29817

\section{DYNDNS}
\subsection{Ajouter un nouveau fournisseur DynDNS}

L'ajout de nouveaux fournisseurs est en fait très facilement, étant donné
que les scripts de mise à jour des fournisseurs d'accés sont séparées. Pour
installer un nouveau fournisseur vous devez modifier le fichier suivant~:

\subsubsection{Fichier opt/etc/dyndns/provider.NAME}

Ce fichier dans lequel est défini les paramètres, est utilisé pour la  mise à
jour de fournisseur d'accès spécial. Le plus souvent, le fichier ce compose
seulement d'une liste de variables, il s'agit d'un script-shell tout a fait
normal, cependant, des opérations plus complexes peuvent être exécutées,
mais cela est rarement nécessaire. Dans ce fichier, les variables suivants
peuvent être utilisés~:

\begin{description}
\item[\$ip] Adresse-IP de l'interface qui doit recevoir un nom d'hôte
  dynamique.
\item[\$host] Nom d'hôte complet, que l'utilisateur a donné dans sa
  configuration.
\item[\$subdom] Composants du nom d'hôte avec le point suivant
  (\textbf{name}.provider.dom)
\item[\$domain] Les deux dernières composantes du nom d'hôte
  (name.\textbf{provider.dom})
\item[\$provider] Nom symbolique du fournisseur d'accès, que l'utilisateur a
  spécifié dans son fichier de configuration.
\item[\$user] Nom d'utilisateur pour ce service.
\item[\$pass] Mot de passe pour ce service.
\end{description}

Ces variables peuvent être écrites entre deux accolades, pour une séparation plus
claire par rapport au texte, c'est à dire \texttt{\$ip} cela devient \texttt{\$\{ip\}}.
Lors de l'application de guillemets faite attention, de ne \emph{pas} utiliser
les guillemets simples avec les extentions de variables ci-dessus, mais utilisez
les guillemets doubles. En règle générale, on peut donc dire~: toujours utiliser
des guillemets simples, mais dès que l'on utilise des extentions de variables,
utiliser les guillemets doubles.

Les variables suivantes doivent être définis dans ce fichier, de manière à ce
que le paquet sache comment mettre à jour le fournisseur d'accès correspondant~:

\begin{description}
\item[provider\_update\_type] Cela détermine la nature de la demande, qui est
    adressée au serveur du fournisseur d'accès. Pour le monent on prend en charge~:
  \begin{description}
  \item[http] On détermine l'appelle automatiquement une page Web, du fournisseur
    d'accès et ainsi on récupére la mise à jour de DynDNS actualisé.
  \item[netcat] On détermine un simple texte, qui est envoyé au serveur du
    fournisseur d'accès pour déclencher la mise à jour.
  \item[gnudip] Une simple authentification pour une mise à jour des procédures,
    ce lequel est exécuté plus de deux demandes HTTP.
  \end{description}
\item[provider\_host] Nom de l'hôte du fournisseur d'accès, qui est contacté
  pour la  mise à jour.
\item[provider\_port] Port de l'hôte du fournisseur d'accès, qui est concerné.
  Le port par défaut pour HTTP est 80.
\end{description}
  
Selon le type de mise à jour d'autres variables doivent être indiquées~:

\begin{description}
\item[HTTP]

\begin{description}
\item[provider\_url] Ici on met URL relative (sans le nom d'hôte, mais avec
  un / au début du Script du fournisseur d'accès. Pour les exemples voir
  s'il vous plaît les fichiers des autres fournisseurs d'accès enregistrés.
\item[provider\_auth] (optionnel) Les fournisseurs d'accès ont besoin pour
  l'ouverture de la session une Authentication basic, le format est
  \texttt{"{}USER:PASSWORD"{}}.
\end{description}

\item[Netcat]

\begin{description}
\item[provider\_data] On indique ici le texte qui sera envoyé au serveur du
  fournisseur d'accès. On peut indiquer par ex. \texttt{provider.DYNEISFAIR}.
\end{description}

\item[GNUDip]

\begin{description}
\item[provider\_script] On indique ici le chemin d'accès au GNUDip script du
  serveur, ce qui ressemble généralement à quelque chose comme ça
  \texttt{'/cgi-bin/gdipupdt.cgi'}.
\end{description}

\end{description}
\subsubsection{Fichier opt/dyndns.txt}

Dans ce fichier une ou plusieurs lignes doivent être insérées pour le nouveau fournisseur
d'accès. Le plus souvent une ligne suffit comme indiqué ci-dessous~:

\begin{verbatim}
  dyndns_%_provider   NAME   etc/dyndns/provider.NAME
\end{verbatim}

Si l'Authentication Basic est utilisé pour le fournisseur d'accès HTTP, on a
encore besoin de l'outil base64~:

\begin{verbatim}
  dyndns_%_provider   NAME   files/usr/local/bin/base64
\end{verbatim}

Si d'autres outils sont demandés, s'il vous plaît envoyer moi plus tôt un Mail,
pour que je puisse les examiner et voir si ils conviennent au paquetage OPT\_DYNDNS.

\subsubsection{Fichier check/dyndns.exp}

Vous devez indiquer dans ce fichier, à la ligne \texttt{DYNPROVIDER = } le nom du
fournisseur d'accès, vous devez ajouter un trait vertical derrière des autres
paramètre, de manière séparée le nouveau nom.

\subsubsection{Fichier doc/$<$langue$>$/tex/dyndns/dyndns\_main.tex}

Enregistrer un nouvelle section dans la documentation. Là encore, les
fournisseurs d'accès sont trié par ordre alphabétique selon le nom abrégé,
c'est celui-ci que les utilisateurs paramètre dans le fichier de Config. \\
Un macro-tableau est présent au début de la documentation, qui est
suffisament explicite.

\subsection{Remerciment}

Je tiens à remercier tout le monde qui a participés au lancement de ce
paquet et une longue vie à ce paquet:

Thomas Müller (\email{opt\_dyndns@s2h.cx}) il a fait un excellent travail,
sans lui il n'était pas possible de proposé ce paquet dans la forme actuelle.

Je voudrais remercier Marcel Döring (\email {m@rcel.to}) qui a longtemps
maintenu ce paquet.

Lors de l'élaboration du paquet de très nombreuses personnes m'ont aidé et
ont trouvées des idées. Je tiens à remercier tous ces courageux volontaires.

En outre, je remercie Frank Meyer et le reste de l'équipe-fli4l de leurs
inlassable travail qui ont bricolage l'un des meilleurs routeur-disquette
du monde (exusez c'est pas très sérieux ;-).

En outre, je tiens à remercier les personnes suivantes, pour leurs conseils,
les rapports d'erreur des nouveaux fournisseurs, etc, ont participé au paquet~:

\begin{itemize}
\item Paul Bischof pour le fournisseur AFRAID.
\item Jens Fischer schrieb das Paket opt\_dtdns, welches mich erst auf die
  Idee brachte, ein Paket für DynDNS.org zu schreiben.
\item Till Jäger schrieb das Paket opt\_cjb, welches in in opt\_dyndns
  übernommen habe.
\item Tobias Gruetzmacher hat auf \altlink{http://portfolio16.de/index.de} Informationen zu
  weiteren DynDNS-Anbietern zusammengetragen, die hier verwendet werden.
\item Die Anbieter dynamischer DNS, die auf ihren Webseiten zum Teil sehr
  gute, zum Teil weniger gute Beschreibungen des zu verwendenden Protokolls
  veröffentlicht haben.
\item Die Programmierer diverser Update-Programme für DynDNS Anbieter, aus
  deren Code schamlos geklaut wurde. ;-)
\item Heiko Ambos von dynaccess.de hat mich bei der Entwicklung der
  Unterstützung für diesen Anbieter unterstützt.
\item Dennis Neuhäuser, der die Idee hatte, die Antworten der Dienste per
  Webserver verfügbar zu machen statt sie auf der Konsole auszugeben
  und auch gleich eine erste Implementation dafür geschickt hat.
\item Lars Winkler der freundlicherweise die Änderungen, um das Paket unter
  2.0pre2 zum Laufen zu bringen zur Verfügung gestellt hat.
\item Markus Kraft und Tobias Gruetzmacher haben die Grundlage für die
  Anpassung an fli4l 2.0 gelegt.
\item Diverse andere Leute haben mir ihre jeweilige Portierung auf fli4l 2.0
  geschickt. Ich muss zu meiner Schande gestehen, dass ich mir die wenigsten
  davon angesehen habe.
\item Georg Bärwald für die Daten zu Selfhost.de
\item Mark C. Storck für die Daten zu Storck.org
\item Arne Biermann für den Hinweis auf den Anbieter hn.org
\item Detlef Paschke für die Daten zu dyn.ee und dyndns.dk
\item Martin Kisser für seine Idee zum Vermeiden von Updates, wenn die
  IP sich nicht geändert hat.
\item Björn Hoffmann für die Daten von DnsArt.com
\item Christian Busch für die Daten von no-ip.com.
\item Ralf Gill für das Update der Daten von selfhost.de.
\item Michael (HeinB) für eine weitere Möglichkeit sich mit fli4l selbst
  in den Fuss zu schiessen. ;-)
\item Marcus Mönnig, dito.
\end{itemize}

\subsection{Licence}

Copyright \copyright  2001-2002 Thomas Müller (\email{opt\_dyndns@s2h.cx}) \\
Copyright \copyright  2002-2003 Tobias Gruetzmacher (\email{fli4l@portfolio16.de}) \\
Copyright \copyright  2004-201x L'équipe fli4l (\email{team@fli4l.de}) \\

Ce programme est un logiciel libre. Il est distribué selon les termes
de la GNU License General Public comme prévu par la Free
Software Foundation. Pour de plus amples informations sur la licence, reportez-vous
s'il vous plaît à \altlink{http://www.gnu.org/licenses/gpl.txt}.

Ce programme est distribué dans l'espoir qu'il sera utile, mais SANS AUCUNE
GARANTIE~-. Sans même la garantie implicite de COMMERCIALISATION ou
D'ADAPTATION À UN USAGE PARTICULIER Les détails peuvent être trouvés dans
la GNU licence General Public.

Vous devriez avoir reçu une copie de la licence GNU General Public avec ce
programme. Sinon, écrivez à~:

\begin{verbatim}
    Free Software Foundation Inc.
    59 Temple Place
    Suite 330
    Boston MA 02111-1307 USA.
\end{verbatim}

Le texte de la licence GNU General Public est également publié sur Internet
\altlink{http://www.gnu.org/licenses/gpl.txt}. Une traduction non officielle en
Allemand peut être trouvé ici \altlink{http://www.gnu.de/documents/gpl.de.html}
et une traduction non officielle en Français ici \altlink{http://org.rodage.com//gpl-3.0.fr.html}
Ces traductions sont cependant, les meilleures compréhensions de l'aide GPL,
pour les droits juridiques vous devez utiliser la version anglaise.

% vi: set ts=4 sw=4 tw=78:
