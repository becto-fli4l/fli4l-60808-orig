% Synchronized to r44713
% Kommando für einen Tabelleneintrag
% \prov{Anbieter}{DYNDNS\_x\_PROVIDER}{Homepage}

\newcommand{\prov}[3]{
  \begin{tabular}{|l|l|l|l}
    \hline
    Fournisseur & #1 \\
    \hline
    DYNDNS\_x\_PROVIDER & #2 \\
    \hline
    Page d'accueil & #3 \\
    \hline
  \end{tabular}
  \newline\newline
  }

\marklabel{sec:dyndns}{
  \section{DYNDNS~- Mise à jour dynamiques des services de noms de domaine}
  }

  Ce paquetage est conçu pour actualiser automatiquement la connexion du
  nom d'hôte dynamique. Voici les services pris en charge par fli4l~:\\

    \prov{FreeDNS (afraid.org)}{AFRAID}{\altlink{http://freedns.afraid.org}}

    \wichtig{Le mot de passe est la dernière partie à indiquer (après le point
    d'interrogation) dans l'URL, que l'on peut obtenir sur la page d'accueil
    du site Afraid.org (votre loggin dans \pfeil "Dynamic DNS" \pfeil de l'URL,
    est caché derrière le lien "Direct URL"). Toutes les autres données
    sont ignorées.}\\

    \prov{Companity}{COMPANITY}{\altlink{http://www.staticip.de/}}
    
    \prov{DDNSS}{DDNSS}{\altlink{http://www.ddnss.de/}}

    \prov{DHS International}{DHS}{\altlink{http://www.dhs.org/}}

    \prov{DNS2Go}{DNS2GO}{\altlink{http://dns2go.com/}}

    \prov{DNS-O-Matic}{DNSOMATIC}{\altlink{http://www.dnsomatic.com}}

    \prov{DtDNS}{DTDNS}{\altlink{http://www.dtdns.com/}}

    \prov{DynAccess}{DYNACCESS}{\altlink{http://dynaccess.de/}}

    \wichtig{DynAccess vous offre dans le cadre de la coopération avec
    DynAccess-fli4l des sous-domaines. *.dyn-fli4l.de, *.dyn-fli4l.info
    et *.dyn-eisfair.de à des tarifs spéciaux. Si vous voulez plus d'informations
    à ce sujet, voir le site Internet \altlink{http://www.dyn-fli4l.de/} ou
    \altlink{http://www.dyn-eisfair.de/}.}\\

    \prov{DynDNS.org}{DYNDNS}{\altlink{http://dyn.com/}}

    \prov{DynDNS.org (custom)}{DYNDNSC}{\altlink{http://dyn.com/standard-dns/}}

    \prov{DynDNS DK}{DYNDNSDK}{\altlink{http://dyndns.dk/}}

    \prov{dyndns:free}{DYNDNSFREE}{\altlink{http://dyndnsfree.de/}}

    \prov{eisfair.net}{DYNEISFAIR}{\altlink{http://www.intersales.de/it-infrastruktur/dyneisfair.html}}

    \wichtig{En utilisant ce service, vous soutenez le travail des développeurs de eisfair et fli4l.}\\

    \prov{DyNS}{DYNSCX}{\altlink{http://www.dyns.cx/}}

    \prov{GnuDIP Dynamic DNS}{GNUDIP}{\altlink{http://gnudip2.sourceforge.net/}}

    \prov{Provider Hurricane Electric}{HE}{\altlink{https://dns.he.net/}}

    \prov{IN-Berlin e.V.}{INBERLIN}{\altlink{http://www.in-berlin.de}}

    \prov{INWX GmbH & Co. KG}{INWX}{\altlink{https://www.inwx.de/}}

    \prov{KONTENT}{KONTENT}{\altlink{http://www.kontent.de/}}

    \prov{Nerdcamp.net}{NERDCAMP}{\altlink{http://nerdcamp.net/dynamic/dns.cgi}}

    \prov{No-IP.com}{NOIP}{\altlink{http://www.no-ip.com/}}

    \prov{noxaDynDNS}{NOXA}{\altlink{http://www.noxa.de/}}

    \prov{OVH.DE}{OVHDE}{\altlink{http://www.ovh.de/}}

    \prov{PHPDYN}{PHPDYN}{\altlink{http://www.webnmail.de/phpdyn/}}

    \wichtig{Vous devez héberger ce type pour votre self}\\

    \prov{Regfish.com}{REGFISH}{\altlink{http://www.regfish.de/}}

    \prov{SelfHost.de}{SELFHOST}{\altlink{http://selfhost.de/cgi-bin/selfhost}}

    \prov{Securepoint Dynamic DNS Service}{SPDNS}{\altlink{http://www.spdns.de/}}

    \prov{Strato}{STRATO}{\altlink{http://www.strato.de/}}

    \prov{T-Link.de}{TLINK}{\altlink{http://www.t-link.de/}}

    \prov{twodns.de}{TWODNS}{\altlink{http://www.twodns.de/}}

    \prov{ZoneEdit.com}{ZONEEDIT}{\altlink{http://zoneedit.com/}}

    Nous essayons de garder ces informations à jour. Néanmoins, nous déclinons
    toute responsabilité quand à l'exactitude de ces données. Si vous découvrez
    une erreur ou un changement vous pouvez envoyer un courriel à l'équipe fli4l
    (\email{team@fli4l.de}).

    Cette liste est complète, Les autres fournisseurs d'accès ne seront pas
    supportés sans modification du programme. Dans annexe vous avez une explication
    pour ajouter sont propre fournisseur, utilisable pour les développeurs de
    paquetage.

    Le nom d'hôte dynamique est automatiquement mis à jour, à chaque connexion
    Internet. Le paquetage comprend une barrière, qui empêche de mettre à jour
    plusieurs fois la même IP, car se n'est pas bien vu par certains
    fournisseurs-DynDNS et peut mener dans les cas extrêmes, au blocage du compte.

    Remarque: cela peut prendre quelques minutes avant que la modification du
    nom d'hôte dynamique prenne effet.

    Avant de commencer la configuration de ce paquetage, il faut ouvrir un
    compte à l'un des fournisseurs mentionnés ci-dessus. Si cela est déjà fait,
    vous pouvez commencer immédiatement. Si vous n'avez pas encore de compte,
    vous pouvez vous diriger vers le tableau ci-dessus et trouver un nom d'hôte,
    pour cela, il suffit de répondre aux exigences du fournisseur et à votre
    goût personnel.

    Pour la configuration, on a besoin des données suivantes~:

  \begin{itemize}
    \item Le nom du fournisseur
    \item Le nom d'utilisateur
    \item Un mot de passe
    \item Le nom d'hôte DynDNS
  \end{itemize}

    Les informations nécessaires peuvent varier selon le fournisseur, nous
    tenterons d'offrir autant que possible une configuration cohérente.
    Par exemple le nom d'hôte est parfois semblable au nom d'utilisateur, dans
    ce cas, nous essaierons d'utiliser toujours le champ hôte et ignorent
    simplement le nom d'utilisateur. Voyons à présent la suite~:

\begin{description}

\config{OPT\_DYNDNS}{OPT\_DYNDNS}{OPTDYNDNS}

    {Si cette variable est paramétrée sur \verb*?'yes'?, alors \var{OPT\_DYNDNS} est activé.}

\config{DYNDNS\_SAVE\_OUTPUT}{DYNDNS\_SAVE\_OUTPUT}{DYNDNSSAVEOUTPUT}

    {Si cette variable est paramétrée sur \verb*?'yes'?, les demandes DynDNS
    peut être stockées dans un fichier sur le serveur web\footnote{OPT\_HTTPD il
    faut installer le paquetage \jump{OPTHTTPD}{HTTPD} pour les visualiser, voir
    \altlink{http://www.fli4l.de/fr/telechargement/version-stable/}}.}

\config{DYNDNS\_N}{DYNDNS\_N}{DYNDNSN}

    {Si vous avez ouvert un compte auprès de plusieurs fournisseurs DynDNS,
    vous pouvez avoir pour chaque connexion plusieurs noms, on adapte ce
    paramètre en conséquence.}

\config{DYNDNS\_x\_PROVIDER}{DYNDNS\_x\_PROVIDER}{DYNDNSxPROVIDER}

    {on indique dans cette variable le nom du fournisseur à utiliser (voir
    tableau ci-dessus et les instructions dans le fichier de configuration).}

\config{DYNDNS\_x\_USER}{DYNDNS\_x\_USER}{DYNDNSxUSER}

    {On indique dans cette variable le nom d'utilisateur chez le fournisseur
    DynDNS. Souvent il s'agit, d'une adresse e-mail, on peut choisir un nom ou
    également un nom d'hôte DynDNS.}

\config{DYNDNS\_x\_PASSWORD}{DYNDNS\_x\_PASSWORD}{DYNDNSxPASSWORD}

    {Vous indiquer dans cette variable le mot de passe du compte DynDNS. Prenez
    garde, que personne ne surveille lorsque vous éditez le fichier de
    configuration~!}

\config{DYNDNS\_x\_HOSTNAME}{DYNDNS\_x\_HOSTNAME}{DYNDNSxHOSTNAME}

    {Vous indiquez ici le nom d'hôte DynDNS \emph{complet} du compte indiqué.
    Par exemple, cela pourrait ressembler à ce qui suit~:

  \begin{itemize}
    \item \texttt{cool.nerdcamp.net}
    \item \texttt{user.dyndns.org}
    \item \texttt{fli4luser.fli4l.net}
  \end{itemize}
  }

\config{DYNDNS\_x\_UPDATEHOST}{DYNDNS\_x\_UPDATEHOST}{DYNDNSxUPDATEHOST}

    {Vous indiquez ici l'hôte qui sera mise à jour, cette variable est utilisée
    pour les fournisseurs qui utilisent un PHPDYN. Ce ne sont pas des fournisseurs
    classiques, un script est utilisé pour mettre à jour une base MySQL du
    serveur PowerDNS. Ce système est sous licence GPL.
  }

\config{DYNDNS\_x\_CIRCUIT}{DYNDNS\_x\_CIRCUIT}{DYNDNSxCIRCUIT}

    {vous pouvez spécifié ici, avec quel circuit le nom d'hôte sera mis à jour.
    Les différents circuits doivent être séparés par un espace. Il est souhaitable,
    d'utiliser un nom hôte seulement pour une connexion DSL. Voici quelques exemples~:

\begin{example}
\begin{verbatim}
        DYNDNS_1_CIRCUIT='1 2 3'           # ISDN seul: Circuits 1 à 3
        ou
        DYNDNS_1_CIRCUIT='pppoe'           # DSL seul: pppoe-Circuit
        ou
        DYNDNS_1_CIRCUIT='dhcp'            # Mise à jour avec un fournisseur DHCP
                                           # (opt_dhcp est nécessaire)
        ou
        DYNDNS_1_CIRCUIT='pppoe 1'         # DSL et ISDN
        ou
        DYNDNS_1_CIRCUIT='static'          # fli4l derrière un routeur par ex. LTE 
\end{verbatim}
\end{example}
  }

\config{DYNDNS\_x\_RENEW}{DYNDNS\_x\_RENEW}{DYNDNSxRENEW}

    Certains fournisseurs prévoient que tous les n jours une mise à jour
    sera exécutée, même si l'adresse IP n'a pas changé. On peut indiquer ici
    cet intervalle. Si l'on n'indique aucune valeur dans cette variable,
    le 29e jour la mise à jour sera exécutée.

    Il faut noter à cet égard, que, seulement une mise à jour est initié lors de
    la connexion~- C'est-à-dire lors d'une connexion DSL ou ISDN ou un renouvellement
    du bail, sur l’interface de configuration via le DHCP, comme si vous le feriez
    avec un modem-câble. Lors d’une longue période sans connexion, vous devez utiliser
    une autre solution pour la mise à jour de la connexion.

\config{DYNDNS\_x\_EXT\_IPV4}{DYNDNS\_x\_EXT\_IPV4}{DYNDNSxEXTIPV4}
\config{DYNDNS\_x\_EXT\_IPV6}{DYNDNS\_x\_EXT\_IPV6}{DYNDNSxEXTIPV6}

	Avec cette variable vous configurez la méthode avec laquelle l'adresse IP
	externe sera détectée. Avec le paramètre \verb*?'none'? vous n'avez aucune
	possibilité d'interroger un service extérieur pour mettre à jour l'adresse
	IP externe, mais, il est possible d'utiliser directement l'interface WAN pour
	récupéter l'adresse IP externe. Cela fonctionne en général seulement quand une
	connexion WAN est directement installé sur fli4l, par ex. avec la DSL via le
	protocole PPPoE. Si vous indiquezle paramètre \verb*?'dyndns'?, le service
	checkip.dyndns.org sera utilisé pour la mise à jour de l'adresse IP externe. Si
	vous utilisez le paramètre \verb*?'stun'?, la liste des serveurs STUN seront
	interrogées un par un jusqu'à obtenir une réponse positive. Pour recevoir une
	adresse IP externe l'utilisation d'un service est nécessaire, si votre routeur
	ne peut pas recevoir d'adresse IP externe par un autre moyen. Vous devez faire
	attention que le routeur n'est pas en train de changer d'adresse IP externe,
	lorsque vous utilisez un service, si c'est le cas votre nom d'inscription dyndns
	ne sera pas mise à jour rapidement.

\config{DYNDNS\_x\_LOGIN}{DYNDNS\_x\_LOGIN}{DYNDNSxLOGIN}

    Certains fournisseurs exigent que l'utilisateur se connecte régulièrement
    à l'aide de leur compte utilisateur sur leur site internet, afin que le
    service ne soit pas désactivé. Si cette variable est paramétrée sur \verb*?'yes'?,
    fli4l le fera pour vous. Toutefois, cela fonctionne que si le paquetage
    dyndns a été préparé pour le fournisseur respectif. Actuellement, une telle
    activité régulière est seulement possible et nécessaire que pour le
    fournisseur "DYNDNS". S'il vous plaît, garder à l'esprit que cette fonction
    exige également de l'activation de la variable \var{OPT\_EASYCRON='yes'}
    du paquetage easycron.

\config{DYNDNS\_LOGINTIME}{DYNDNS\_LOGINTIME}{DYNDNSLOGINTIME}

    Utilisez-vous un fournisseur avec qui fli4l vous connecte régulièrement pour
    éviter la désactivation du service (voir ci-dessus), alors vous pouvez
    paramétrer cette variable, pour que cette application à lieu. Ce qui est
    nécessaire est de définir une valeur de temps au format Cron; Pour plus de
    détails s'il vous plaît lisez la documentation du paquetage easycron. Le
    réglage par défaut est \texttt{0 8 * * *}, ce qui correspond à une notification
    journalière à huit heures du matin.

\config{DYNDNS\_ALLOW\_SSL}{DYNDNS\_ALLOW\_SSL}{DYNDNSALLOWSSL}

    Si cette variable est paramétrée sur \verb*?'yes'?, la mise à jour est
    effectuée lorsque cela est possible en utilisant le SSL (connexion sécurisée).

\config{DYNDNS\_LOOKUP\_NAMES}{DYNDNS\_LOOKUP\_NAMES}{DYNDNSLOOKUPNAMES}

    La mise à jour de l'adresse-IP devrait être faite uniquement si l'adresse-IP
    change. Tous les routeur-fli4l n'ont pas de mémoire permanente, pour sauvegardée
    les informations de l’adresse-IP enregistrée, juste après le démarrage ces
    informations ne sont pas disponible. Pour éviter tout de même des mises à jour
    inutiles, fli4l peut dans cette situation (et seulement dans cette situation),
    demander au service de nom l'adresse-IP enregistrée. L'adresse-IP est
    alors sauvegardée dans le cache et sera vérifiée avant chaque mise à jour
    de celle-ci.

    À noter, après un reboot (ou redémarrage), un nouvel intervalle de mise à
    jour commence, puis fli4l recherche dans le service de nom l’adresse-IP.

\config{DYNDNS\_DEBUG\_PROVIDER}{DYNDNS\_DEBUG\_PROVIDER}{DYNDNSDEBUGPROVIDER}

    Si cette variable est paramétrée sur 'yes', une trace du processus de mise
    à jour est enregistrée, vous pourrez par la suite examiner le problème qui
    c'est peut être produit.

    Configuration par défaut~: \var{DYNDNS\_DEBUG\_PROVIDER}='no'

	Si vous avez indiquer \verb*?'yes'? vous pourrez voir aussi l'adresse IP externe
	du serveur STUN s'il est paramétré.

\config{STUN\_SERVER\_N}{STUN\_SERVER\_N}{STUNSERVERN}

	Dans cette variable, vous indiquez le nombre de serveur STUN.

\config{STUN\_SERVER\_x}{STUN\_SERVER\_x}{STUNSERVERx}

	Dans cette variable, vous indiquez le FQDN (ou nom de domaine complètement qualifié)
	pour le serveur STUN. Vous pouvez également ajouter en option le numéro de port du FQDN.

\begin{example}
\begin{verbatim}
       STUN_SERVER_1='stun.l.google.com:19302'
       STUN_SERVER_2='stun1.l.google.com:19302'
       STUN_SERVER_3='stun2.l.google.com:19302'
       STUN_SERVER_4='stun3.l.google.com:19302'
       STUN_SERVER_5='stun4.l.google.com:19302'
       STUN_SERVER_6='stun01.sipphone.com'
       STUN_SERVER_7='stun.ekiga.net'
       STUN_SERVER_8='stun.fwdnet.net'
       STUN_SERVER_9='stun.ideasip.com'
\end{verbatim}
\end{example}
\end{description}

% vim: tabstop=4
