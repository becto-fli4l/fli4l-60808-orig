% Do not remove the next line
% Synchronized to r43133

\section {SQUID~- Serveur proxy HTTP}

\subsection{Description~:}

Squid est un proxy HTTP qui permet de mettre en cache les pages Web sur le
disque dur local, si vous ouvrez à nouveau la page Web, vous ouvrez la copie de
la page qui est enregistrée dans le cache, plutôt que de la retéléchargée à
chaque fois sur le site Internet. De ce fait, la connexion Internet sera
libérée et la structure de la page sera affiché plus rapidement. L'utilisation
de ce serveur proxy est particulièrement recommandée, si plusieurs personnes
souhaitent surfer en même temps, et si vous avez seulement à votre disposition
une connexion Internet à faible débit.

\subsection{Conditions~:}

Voici les besoins nécessaires pour faire fonctionner le proxy sur un système de
base fli4l (www.fli4l.de). Il faut aussi que fli4l soit installé sur un disque dur.

Les conditions minimum sont les suivantes~:
CPU~: 100 MHz
RAM~:  32 Mo
HD~:  100 Mo

\wichtig{Squid est un processus à par entière sur le routeur fli4l~! Donc chaque système
est automatique, la nouvelle configuration du filtrage de paquets dans la liste
Input à pour effet d'avoir un accés sur le routeur sans restriction et aussi
un accés Internet via Squid sans restriction. Cela affectera les protocoles
HTTP, HTTPS et FTP~!}



\section{Les variables du fichier squid.txt~:}

\begin{description}
\config{SQUID\_MANAGER}{SQUID\_MANAGER}{SQUIDMANAGER}

                Vous indiquez ici l'adresse \mbox{E-Mail} de l'administrateur
                local du proxy. Cette adresse est également utilisée lorsque
                les messages d'erreur doivent être adressées à l'administrateur.
                En outre, l'adresse \mbox{E-Mail} est utilisée comme login
                pour le gestionnaire de cache.

                Configuration par défaut~:
\begin{verbatim}
  SQUID_MANAGER='squid@fli4l'
\end{verbatim}


\config{SQUID\_PASSWORD}{SQUID\_PASSWORD}{SQUIDPASSWORD}

                Dans cette variable vous indiquez le mot de passe pour
                le gestionnaire de cache.

                Configuration par défaut~:
\begin{verbatim}
  SQUID_PASSWORD='fli4l'
\end{verbatim}


\config{SQUID\_TRANSPARENT\_CACHING}{SQUID\_TRANSPARENT\_CACHING}{SQUIDTRANSPARENTCACHING}
\config{SQUID\_TRANSPARENT\_FORWARDING}{SQUID\_TRANSPARENT\_FORWARDING}{SQUIDTRANSPARENTFORWARDING}

                Si vous activez ces fonctions, Squid filtrera toutes les "demandes
				de surf" à partir du réseau interne vers Internet et les mettra en
				cache. Cela peut être utile si vous souhaitez mettre en cache les
				clients qui n'ont pas explicitement configuré le proxy dans leur
				navigateur. Le client internaute ne remarquera pas la présense du
				proxy. Les avantages~:
\begin{enumerate}
	\item{ Vous pouvez également mettre en cache la connexion
           Internet si vous ne pouvez pas configurer les
           clients dans les règles ou si les clients ne
           peuvent/veulent pas sans occupés.}

	\item{ Les programmes qui ne sont pas pris en charge par
           le proxy seront interceptés, ils profiterons
           eux aussi d'une "connexion plus rapide".}
\end{enumerate}

                Configuration par défaut~:
\begin{verbatim}
  SQUID_TRANSPARENT_CACHING='no'
  SQUID_TRANSPARENT_FORWARDING='no'
\end{verbatim}

                 \wichtig{ Si ces options sont définies sur 'yes', on ne pourra
                                plus utiliser la redirection de port sur
                                le port 80. Si c'est le cas, cela conduirait à
                                un conflit et Squid s'arrêtera de fonctionner.}

                 \emph{Si ces options sont définies sur 'yes', les accés
                                aux adresses IP individuelles seront bloquets
                                et inefficace dans le fichier base.txt. Enfin
                                fli4l pourra toujours accèder à Internet.}

                 \emph{Si vous avez activé le cache transparent, un autre port
                         sera ouvert pour que Squid affiche certaines pages web
                         (par exemple, les pages d'erreur) pour pouvoir corriger
                         les erreurs. Vous indiquez une valeur dans la variable
                         \var{SQUID\_HTTP\_PORT} (ci-dessous), le port suppérieur
                         sera automatiquement ouvert. Si vous spécifiez le port
                         \var{SQUID\_HTTP\_PORT='3128'} le port 3129 sera
                         également ouvert et donc ne peut être utilisé pour
                         une autre application sur fli4l.}


\config{SQUID\_HTTP\_PORT}{SQUID\_HTTP\_PORT}{SQUIDHTTPPORT}

                Dans cette variable vous indiquez le port TCP sur lequel Squid
                attendra les connexions Internet.

                Configuration par défaut~:
\begin{verbatim}
  SQUID_HTTP_PORT='3128'
\end{verbatim}

                \achtung{Si le cache transparent est activé avec
                (\var{SQUID\_TRANSPARENT\_CACHING='yes'} (voir ci-dessus), un
                autre port sera ouvert, à savoir un port dans \var{SQUID\_HTTP\_PORT}
                plus un autre port. Ce port ne doit pas être attribué à une
                autre application~!}


\config{SQUID\_MEM\_CACHE\_SIZE}{SQUID\_MEM\_CACHE\_SIZE}{SQUIDMEMCACHESIZE}

                Dans cette variable vous allez déterminer la quantité d'espace
                maximum autorisé à prendre dans la RAM pour le cache. La taille
                est en Mo.

                La quantité recommandée est de 1/4 de la mémoire principale
                disponible du routeur.


\config{SQUID\_DISK\_CACHE\_SIZE}{SQUID\_DISK\_CACHE\_SIZE}{SQUIDDISKCACHESIZE}

                Dans cette variable vous allez déterminer la quantité d'espace
                maximum autorisé à prendre dans le disque dur pour le cache. La
                taille est en Mo. Si la mémoire du cache est plus grand que celle
                indiquée, Squid supprimera automatiquement les anciens fichiers,
                de sorte à ne pas dépasser la taille maximale. Vous devez au moins
                laissez 25 % d'espace libre sur le disque dur, car si le disque
                est plein, Squid fonctionnera très lentement. Si Squid n'a pas
                de place sur le disque pour les écriture des fichiers, il se
                produira une erreur.

                En outre, il faut noter que Squid a besoin d'environ 6 Mo de
                mémoire principale pour les processus + une quantitée de mémoire
                RAM dans SQUID\_MEM\_CACHE\_SIZE pour la mise en cache + 6 Mo
                de mémoire par Go de disque pour le cache. Les processus Squid
                ne peuvent être sous-traités, il doit y avoir une quantité de
                mémoire physique correspondant au disque dur~!

                Donc, pour ne pas avoir de problème avec SQUID\_DISK\_CACHE\_SIZE
                choisissez, un Squid avec un cache plus petit il sera plus rapide
                qu'un Squid avec un grand cache et avec pas assez de mémoire
                principale.


\config{SQUID\_MAX\_OBJECT\_SIZE}{SQUID\_MAX\_OBJECT\_SIZE}{SQUIDMAXOBJECTSIZE}

                Dans cette variable vous indiquez une valeur exprimée en
                kilo-octets. Les objets qui seront plus grands que la valeur
                indiquée ne seront pas stockées sur le disque. Si vous voulez
                gagner en trafic sur Internet au détriment de la vitesse, vous
                devez définir cette valeur avec un taux élevée. Si vous voulez
                avoir une plus grande vitesse au détriment du trafic, vous devez
                définir cette valeur à 10000 ko et même moins que cela.

                Configuration par défaut~:
\begin{verbatim}
  SQUID_MAX_CACHE_SIZE='65536'
\end{verbatim}


\config{SQUID\_WORK\_DIR}{SQUID\_WORK\_DIR}{SQUIDWORKDIR}

                Dans cette variable vous indiquez le répertoire Squid pour stocker
                les données du cache, les fichiers Log (ou journal) et d'autres
                fichiers pertinents pour les opérations.

                Le cache sera alors dans~:
                                SQUID\_WORK\_DIR/cache

                De même, les fichiers log seront dans~:
                                SQUID\_WORK\_DIR/logs

                La variable SQUID\_WORK\_DIR doit donc se référer à un disque
                dur d'une capacité suffisante. Il est important que le disque
                est un accés en lecture/écriture et doit être formaté avec le
                système de fichiers ext3. En faite vous avez besoin d'un accès
                au disque et que celui-ci soit formaté. Vous pouvez trouver plus
                d'information sur la préparation du disque et son utilisation
                dans le paquetage HD de fli4l.

                Configuration par défaut~:
\begin{verbatim}
  SQUID_WORK_DIR='/data/squid'
\end{verbatim}

                Vous pouvez aussi spécifié dans la variable 'auto', ainsi le
                chemin pour les fichiers système sera utilisé pour les fichiers
                persistente. Mais vous devez vous assurer que le média soit
                configuré avec \var{FLI4L\_UUID} et qu'il soit suffisamment
                grand, si non /boot ne sera pas accessible en écriture même si
                le disque Ram est utilisé.


\config{SQUID\_CYCLE\_LOG\_N}{SQUID\_CYCLE\_LOG\_N}{SQUIDCYCLELOGN}

                Dans cette variable vous indiquez le nombre de rotations des
                fichiers log (ou journal). Si vous indiquez la valeur "0", les
                fichiers log ne seront pas fermées et peuvent être ouvert. Cela
                vous permet de renommer les fichiers log.

                Configuration par défaut~:
\begin{verbatim}
  SQUID_CYCLE_LOG_N='10'
\end{verbatim}

                Cela correspond à une rotation de 10 fichiers logs.


\config{SQUID\_CYCLE\_LOG\_TIME}{SQUID\_CYCLE\_LOG\_TIME}{SQUIDCYCLELOGTIME}

                Une fois que Squid est exécuté, on l'oublie parfois, il est alors
                très pratique d'effacer les fichiers LOG régulièrement et
                automatiquement. Cela évite que le disque soit rempli de
                ces fichiers. La valeur qui sera indiquée dans cette variable
                est exprimée en secondes. 1 heure correspond à 3600 secondes.

                Configuration par défaut~:
\begin{verbatim}
  SQUID_CYCLE_LOG_TIME='172800'
\end{verbatim}

                Cela correspond à 48 heures.

\section{Variables pour la configuration du réseau en dehors de base.txt}

\config{SQUID\_AUTO\_CONFIG}{SQUID\_AUTO\_CONFIG}{SQUIDAUTOCONFIG}

                Dans cette variable vous indiquez, si les réseaux connectés à
                fli4l sont automatiquement récupérées à partir du fichier
                base.txt ou si les réseaux sont configurés manuellement en
                utilisant les champs suivants.

                Vous pouvez utiliser les variables suivantes pour des réseaux
                importants, avec plusieurs plages d'adresses IP et avec d'autres
                routeurs, cela peut être combiné automatiquement.


\config{SQUID\_ACCESS\_NET\_N}{SQUID\_ACCESS\_NET\_N}{SQUIDACCESSNETN}

                Vous indiquez dans cette variable le nombre de réseaux qui seront
                autorisés à accéder au service Squid. Normalement, la valeur est
                sur '0' parce que tous les droit de configuration des réseaux
                sont déjà couverts avec la variable SQUID\_AUTO\_CONFIG.


\config{SQUID\_ACCESS\_NET\_x}{SQUID\_ACCESS\_NET\_x}{SQUIDACCESSNETx}

                Vous indiquez dans cette variable les réseaux pouvant accéder
                au service Squid. Ce paramètre sera utilisé en interne par Squid,
				il contrôlera ainsi des requètes de l'utilisateur pour accèder
				au page web. Vous pouvez si nécessaire ajouter des règles de filtrage
				de paquets avec la variable (\var{PF\_INPUT\_x} dans base.txt, ou avec
				\var{PF6\_INPUT\_x} dans ipv6.txt).

                Par exemple~:
\begin{verbatim}
  SQUID_ACCESS_NET_1='192.168.0.0/24'
\end{verbatim}

                Maintenant tous les PCs avec les adresses IP 192.168.0.XXX
                peuvent utilisés Squid.


\section{Variables optionnelles elles ne sont pas nécessaire pour le fonctionnement}

\config{SQUID\_CONNECT\_TIMEOUT}{SQUID\_CONNECT\_TIMEOUT}{SQUIDCONNECTTIMEOUT}

                Vous indiquez dans cette variable un temps de connexion pour
                que Squid puisse se connecter au réseau TCP du serveur. Durant
                cette période (par défaut 120 secondes) la connexion doit être
                établie avec le serveur.

\config{SQUID\_CACHE\_N}{SQUID\_CACHE\_N}{SQUIDCACHEN}

                Dans cette variable vous pouvez activer l'accés au cache sur
                plusieurs disque. En outre, vous peuvez spécifier plusieurs
                répertoires supplémentaires pour le cache. Pour ce faire, vous
                devez indiquer ici le nombre souhaité pour les accés au cache.

                Maintenant, vous pouvez configurer les variables
\config{SQUID\_CACHE\_1\_DIR}{SQUID\_CACHE\_1\_DIR}{SQUIDCACHE1DIR}
\config{SQUID\_CACHE\_1\_SIZE}{SQUID\_CACHE\_1\_SIZE}{SQUIDCACHE1SIZE}
                avec les valeurs souhaitées. Voici un exemple, vous devez monter
                un deuxième disque dur /disk2, avec une capacité de 4 Go~:

\begin{verbatim}
  SQUID_CACHE_1_DIR='/disk2'
  SQUID_CACHE_1_SIZE='3000'
\end{verbatim}
                Ce disque dur devrait désormais être utilisé pour le cache
                de Squid.

                Configuration par défaut~: pas d'autre disque pour le cache\_dir.


\config{SQUID\_NEXT\_PROXY}{SQUID\_NEXT\_PROXY}{SQUIDNEXTPROXY}

                Si l'accès à Internet n'est possible que via un proxy, ou pour
                des raisons de performances, l'accés sera accessible depuis un
                proxy amont, vous devez indiquer le proxy avec le format suivant
                'name.domain', par exemple.
\begin{verbatim}
  SQUID_NEXT_PROXY='www-proxy.t-online.de'
\end{verbatim}
                En cas échec du proxy, Squid essaira de rechercher
                directement la page web.

                Configuration par défaut~: pas de proxy amont


\config{SQUID\_NEXT\_PROXY\_PORT}{SQUID\_NEXT\_PROXY\_PORT}{SQUIDNEXTPROXYPORT}

                Dans cette variable vous indiquez le port du proxy amont.\\
                Uniquement si la variable SQUID\_NEXT\_PROXY est configurée,
                par exemple.
\begin{verbatim}
  SQUID_NEXT_PROXY_PORT='80'
\end{verbatim}

                Configuration par défaut~: variable vide


\section{Seulement pour les experts~!}

Le fichier opt/etc/rc.d/rc455.squid contient toutes les variable de squid.conf.
Pour ceux qu'ils veulent faire des changements dans squid.conf, vous pouvez les
faire à cette emplacement. Au démarrage de fli4l les modifications apportées
seront prises en charge dans le squid.conf généré dynamiquement.

\section{Liens}

Pour d'autres questions, je vous recommande le site du projet Squid avec en
connexe les FAQs~: \altlink{http://wiki.squid-cache.org/SquidFaq} ou aussi
\altlink{http://www.Squid-handbuch.de/hb/}.

\section{Remerciment}

Auteur à origine~: Hermann Strassner (hermann.strassner@web.de)

Extension et modification par~: Ingo Winiarski (iwiniarski@gmx.de)

\end{description}
