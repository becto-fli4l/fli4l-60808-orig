% Last Update: $Id$

\subsection{Beschreibung:}
Squid ist ein HTTP Proxy, der es ermöglicht, Internet Seiten auf einer lokalen
Festplatte zwischenzuspeichern und bei erneutem Aufruf der Seite, eine
aktuelle lokale Kopie sendet, anstatt die Seite jedesmal neu zu übertragen.
Dadurch wird die Leitung in das Internet entlastet und der Aufbau der Seite
wird schneller. Besonders ist der Einsatz zu empfehlen, wenn mehrere Personen
gleichzeitig surfen wollen, aber nur eine geringe Bandbreite ins Internet zur
Verfügung steht.

\subsection{Voraussetzungen:}
Es wird ein lauffähiges Basissystem von fli4l (www.fli4l.de) benötigt. fli4l
muß dabei auf einer Festplatte installiert sein.

Minimalanforderungen sind:\\
CPU: 100 MHz\\
RAM:  32 MB\\
HD:  100 MB

\wichtig{Squid ist ein Prozess auf dem fli4l-Router! Damit hat automatisch jedes System,
welches über die Input-Liste der Paketfilter-Konfiguration uneingeschränkten
Zugriff auf den Router hat auch gleichzeitig uneingeschränkten Internet-Zugriff
über den Squid. Dies betrifft die Protokolle HTTP, HTTPS und FTP!}

\section{Variablen in der squid.txt:}

\begin{description}
\config{SQUID\_MANAGER}{SQUID\_MANAGER}{SQUIDMANAGER}

                Gibt die \mbox{E-Mail}-Adresse des lokalen Proxy Administrators an.
                Diese Adresse wird auch bei Fehlermeldungen als Ansprechpartner
                ausgegeben. Ebenso dient diese \mbox{E-Mail}-Adresse als Login für den
                Cachemanager.

                Standardeinstellung ist:
\begin{verbatim}
	SQUID_MANAGER='squid@fli4l'
\end{verbatim}


\config{SQUID\_PASSWORD}{SQUID\_PASSWORD}{SQUIDPASSWORD}

                Passwort für den Cachemanager.

                Standardeinstellung ist:
\begin{verbatim}
	SQUID_PASSWORD='fli4l'
\end{verbatim}


\config{SQUID\_TRANSPARENT\_CACHING}{SQUID\_TRANSPARENT\_CACHING}{SQUIDTRANSPARENTCACHING}
\config{SQUID\_TRANSPARENT\_FORWARDING}{SQUID\_TRANSPARENT\_FORWARDING}{SQUIDTRANSPARENTFORWARDING}

                Wenn man diese Funktionen aktiviert, filtert Squid alle
                \dq{}Surf-Zugriffe\dq{} vom internen Netz ins Internet heraus und
                cached sie somit. Das kann nützlich sein, wenn man auch die
                Clients cachen möchte, die in ihrem Browser nicht explizit
                einen Proxy angegeben haben. Der Surfer am Client bekommt
                davon allerdings nichts mit. Der Vorteil:
\begin{enumerate}
    \item{ Man kann die Leitung ins Internet auch dann cachen,
           wenn man sich nicht um die ordnungsgemäße
           Konfiguration der Clients kümmern kann/will.}
    \item{ Programme die keinen Proxy unterstützen, werden
           auch abgefangen und kommen ebenfalls in den Genuß
           der \dq{}schnelleren Verbindung\dq{}}
\end{enumerate}

                Standardeinstellung ist:
\begin{verbatim}
	SQUID_TRANSPARENT_CACHING='no'
	SQUID_TRANSPARENT_FORWARDING='no'
\end{verbatim}

                \wichtig{Wenn diese Optionen auf 'yes' gesetzt wurden, kann man
                                kein portforwarding mehr auf PORT 80 verwenden.
                                Das würde zu einem Konflikt führen und Squid
                                würde seinen Dienst verweigern.}

                \emph{Weiterhin sind Zugangssperren von einzelnen IPs in der
                                base.txt wirkungslos! Schließlich hat der
                                fli4l selbst immer vollen Zugriff auf das
                                Internet.}

                \emph{Wenn transparentes Caching aktiviert ist, wird ein
                         weiterer Port geöffnet, damit Squid gewisse Seiten
                         (z.B. Fehlerseiten) ordentlich ausliefern kann. Dies
                         ist der Wert von \var{SQUID\_HTTP\_PORT} (siehe unten)
                         plus eins. Wenn also \var{SQUID\_HTTP\_PORT = 3128}
                         gesetzt ist, wird der Port 3129 ebenfalls geöffnet und
                         darf somit keiner weiteren Anwendung auf dem fli4l
                         zugeordnet sein.}


\config{SQUID\_HTTP\_PORT}{SQUID\_HTTP\_PORT}{SQUIDHTTPPORT}

                Hier wird angegeben auf welchen TCP Port Squid auf Verbindungen
                warten soll.

                Standardeinstellung ist:
\begin{verbatim}
	SQUID_HTTP_PORT='3128'
\end{verbatim}

                \achtung{Wenn transparentes Caching aktiviert ist
                (\var{SQUID\_TRANSPARENT\_CACHING='yes'}, siehe oben), wird ein
                weiterer Port geöffnet, nämlich \var{SQUID\_HTTP\_PORT} plus
                eins. Dieser Port darf somit keiner anderen Anwendung zugeordnet
                sein!}


\config{SQUID\_MEM\_CACHE\_SIZE}{SQUID\_MEM\_CACHE\_SIZE}{SQUIDMEMCACHESIZE}

                Diese Einstellung bestimmt, wieviel Platz der Cache maximal
                im RAM einnehmen darf. Man gibt die Größe in MB an.

                Empfohlen wird 1/4 des vorhandenen Hauptspeicher des Routers.


\config{SQUID\_DISK\_CACHE\_SIZE}{SQUID\_DISK\_CACHE\_SIZE}{SQUIDDISKCACHESIZE}

                Diese Einstellung bestimmt, wieviel Platz der Cache maximal
                auf der Festplatte einnehmen darf. Man gibt die Größe in MB
                an. Wird der Cache größer, löscht Squid automatisch alte
                Dateien, so daß die Maximalgröße nicht überschritten wird.
                Auf der Platte sollte bei vollem Cache noch mindestens 25\%
                frei sein, weil Squid sehr langsam wird, wenn die Platte
                annähernd voll ist. Wenn Squid keinen Platz auf der Platte
                hat, um Dateien zu schreiben, beendet er sich mit einem
                Fehler.

                Auch sollte man beachten, daß Squid ca. 6 MB Hauptspeicher
                für den Prozess + den unter SQUID\_MEM\_CACHE\_SIZE gewählten
                Hauptspeicher für den Cache im RAM + ca. 6 MB Hauptspeicher
                pro GB Cache auf der Platte benötigt. Der Squid Prozess darf
                nicht ausgelagert werden, es muß entsprechend viel
                physikalischer Speicher vorhanden sein!

                Also bei Problemen SQUID\_DISK\_CACHE\_SIZE klein genug
                wählen, ein Squid mit kleinerem Cache ist schneller als ein
                Squid mit großem Cache aber zuwenig Hauptspeicher.


\config{SQUID\_MAX\_OBJECT\_SIZE}{SQUID\_MAX\_OBJECT\_SIZE}{SQUIDMAXOBJECTSIZE}

                Der Wert wird in kB angegeben. Objekte, die größer als
                dieser Wert sind, werden nicht auf Festplatte gespeichert.
                Wenn man viel Traffic einsparen will auf Kosten hoher
                Geschwindigkeit, sollte man diesen Wert hochsetzen. Wenn man
                eine höhere Geschwindigkeit auf Kosten von mehr Traffic
                erreichen will, sollte man diesen Wert auf 10000 kB und
                darunter setzen.

                Standardeinstellung ist:
\begin{verbatim}
	SQUID_MAX_CACHE_SIZE='65536'
\end{verbatim}


\config{SQUID\_WORK\_DIR}{SQUID\_WORK\_DIR}{SQUIDWORKDIR}

                In diesem Verzeichnis legt Squid seinen Cache, die Log Dateien,
                sowie weitere für den Betrieb benötigte Dateien ab.

                Der Cache befindet sich dann in:
                                SQUID\_WORK\_DIR/cache

                Analog dazu die Logs:
                                SQUID\_WORK\_DIR/logs

                SQUID\_WORK\_DIR sollte also auf eine Festplatte mit genügend
                Kapazität verweisen. Wichtige Voraussetzungen sind, daß man
                einen Read/Write Zugriff hat und daß die Festplatte mit dem
                ext Dateisystem formatiert ist. Genauere Informationen zur
                Festplattenvorbereitung sowie -nutzung unter fli4l findet man
                im Paket HD.

                Standardeinstellung ist:
\begin{verbatim}
	SQUID_WORK_DIR='/data/squid'
\end{verbatim}

                Alternativ kann hier auch 'auto' angegeben werden, so daß der System-Pfad
                für persistente Files benutzt wird. Hier ist aber darauf zu achten, daß
                dieser mittels \var{FLI4L\_UUID} auch auf ein ausreichend großes Medium
                konfiguriert wird, da sonst /boot oder wenn nicht beschreibbar sogar die
                Ramdisk benutzt wird.

\config{SQUID\_CYCLE\_LOG\_N}{SQUID\_CYCLE\_LOG\_N}{SQUIDCYCLELOGN}

                Diese Option bestimmt die Anzahl von Logdatei-Rotationen. Wird
                hier der Wert \verb+'0'+ angegeben, werden die Logdateien nur
                geschlossen und wieder geöffnet. Dies ermöglicht Ihnen, die Log-Dateien
                selbst umzubenennen.

                Der Standardwert ist:
\begin{verbatim}
	SQUID_CYCLE_LOG_N='10'
\end{verbatim}

                Das entspricht 10 rotierenden Logdateien.


\config{SQUID\_CYCLE\_LOG\_TIME}{SQUID\_CYCLE\_LOG\_TIME}{SQUIDCYCLELOGTIME}

                Da man Squid, wenn er einmal läuft, meistens unbeachtet läßt,
                ist es ganz praktisch, die Log-Dateien regelmäßig automatisch
                zu löschen. Es vermeidet, daß die Festplatte voll geschrieben
                wird. Der Wert wird in Sekunden angegeben. 1 Stunde entspricht
                3600 Sekunden.

                Der Standardwert ist:
\begin{verbatim}
	SQUID_CYCLE_LOG_TIME='172800'
\end{verbatim}

                Das entspricht 48 Stunden.

\section{Variablen für Netzwerkkonfigurationen außerhalb der base.txt}

\config{SQUID\_AUTO\_CONFIG}{SQUID\_AUTO\_CONFIG}{SQUIDAUTOCONFIG}

                Mit dieser Option wird festgelegt, ob die Netzwerke mit
                Zugriffsrecht auf die Dienste von Squid automatisch aus der
                base.txt geholt werden, oder ob die Netzwerke manuell über die
                nachfolgenden Felder konfiguriert werden.

                Für große Netzwerke mit mehreren IP-Bereichen und weiteren
                Routern kann diese Automatik mit den nachfolgenden Optionen
                kombiniert werden.


\config{SQUID\_ACCESS\_NET\_N}{SQUID\_ACCESS\_NET\_N}{SQUIDACCESSNETN}

                Anzahl der Netzwerke, die auf die Dienste von Squid zugreifen
                dürfen. Im Normalfall ist dieser Wert '0', da alle direkt am
                fli4l konfigurierten Netzwerke über SQUID\_AUTO\_CONFIG bereits
                erfasst sind.


\config{SQUID\_ACCESS\_NET\_x}{SQUID\_ACCESS\_NET\_x}{SQUIDACCESSNETx}

                Hier gibt man das Netzwerk an, das auf die Dienste von Squid
                zugreifen darf. Diese Angaben werden Squid-intern für die
                Steuerung von Zugriffen verwendet, um zu regeln, wer worauf
                mittels Squid zugreifen darf. Der Zugriff auf Squid selbst muss
                ggf. mit zusätzlichen Regeln des Paketfilters
                (\var{PF\_INPUT\_x} in base.txt oder \var{PF6\_INPUT\_x} in
                ipv6.txt) erlaubt werden.

                Ein Beispiel wäre:
\begin{verbatim}
	SQUID_ACCESS_NET_1='192.168.0.0/24'
\end{verbatim}

                Jetzt dürften alle PC's mit den IP Adressen 192.168.0.XXX Squid
                benutzten.


\section{Otionale und nicht für den Betrieb notwendige Variablen}

\config{SQUID\_CONNECT\_TIMEOUT}{SQUID\_CONNECT\_TIMEOUT}{SQUIDCONNECTTIMEOUT}

                Diese Option bestimmt, wie lange Squid auf einen vollständigen
                TCP-Verbindungsaufbau zu einem Server wartet. Innerhalb dieser
                Zeit (Standard: 120 Sekunden) muß die Verbindung zum Server
                aufgebaut sein.


\config{SQUID\_CACHE\_N}{SQUID\_CACHE\_N}{SQUIDCACHEN}

                Falls man den Cache über mehrere Festplatte verteilen will,
                dann kann man den Zugriff darauf hier aktivieren. Dazu kann
                man mehrere zusätzliche Cache Verzeichnisse angeben. Um das zu
                machen, muß der Zähler auf die gewünschte Anzahl der
                zusätzlichen Verzeichnisse gesetzt werden.

                Jetzt kann man die Variablen
\config{SQUID\_CACHE\_1\_DIR}{SQUID\_CACHE\_1\_DIR}{SQUIDCACHE1DIR}
\config{SQUID\_CACHE\_1\_SIZE}{SQUID\_CACHE\_1\_SIZE}{SQUIDCACHE1SIZE}
                mit den gewünschten Werten füllen. Ein Beispiel wäre, wenn
                eine zweite Festplatte unter /disk2 gemountet ist und sie eine
                Kapazität von 4 GB hat:
\begin{verbatim}
	SQUID_CACHE_1_DIR='/disk2'
	SQUID_CACHE_1_SIZE='3000'
\end{verbatim}
                Diese Festplatte würde jetzt nur für den Squid Cache genutzt
                werden.

                Standardeinstellung ist: keine weiteren cache\_dir.


\config{SQUID\_NEXT\_PROXY}{SQUID\_NEXT\_PROXY}{SQUIDNEXTPROXY}

                Falls der Internetzugang nur über einen Proxy möglich ist,
                oder aus Performancegründen auf einen Upstream Proxy
                zugegriffen werden soll, wird der hier im Format
                'name.domain' eingetragen, also z. B.
\begin{verbatim}
	SQUID_NEXT_PROXY='www-proxy.t-online.de'
\end{verbatim}
                Bei Ausfall dieses Proxy versucht Squid die Seiten direkt
                aufzulösen.

                Standardeinstellung ist: kein Upstream Proxy


\config{SQUID\_NEXT\_PROXY\_PORT}{SQUID\_NEXT\_PROXY\_PORT}{SQUIDNEXTPROXYPORT}

                Port des Upstream Proxy\\
                Wird nur ausgewertet, wenn
                SQUID\_NEXT\_PROXY nicht leer ist, z. B.
\begin{verbatim}
	SQUID_NEXT_PROXY_PORT='80'
\end{verbatim}

                Standardeinstellung ist: leer


\section{NUR FÜR EXPERTEN!}
In der Datei opt/etc/rc.d/rc455.squid stehen alle in der squid.conf enthaltenen
Variablen. Wer an der squid.conf selbst Änderungen vornehmen will kann dies an
dieser Stelle machen. Beim Start des fli4l werden die gemachten Änderungen in
die dynamisch erzeugte squid.conf übernommen.

\section{Links}
Für weitergehende Fragen kann ich die Homepage des Squid Projects und die
dazugehörige FAQs empfehlen: \altlink{http://wiki.squid-cache.org/SquidFaq},\\
oder auch: \altlink{http://www.Squid-handbuch.de/hb/}.

\section{Credits}
Ursprünglicher Autor: Hermann Strassner (hermann.strassner@web.de)

Erweitert und geändert durch: Ingo Winiarski (iwiniarski@gmx.de)
\end{description}
