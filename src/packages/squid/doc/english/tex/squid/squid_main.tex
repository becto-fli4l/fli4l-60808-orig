% Synchronized to r43133

\subsection{Description:}
Squid is a HTTP proxy that allows to cache web pages to a local disk and if
the page is opened again will send the current local copy, instead of loading
the page again from the web. This reduces Internet traffic and the display of
the page is faster. Particularly, the use is recommended if more than one person
at the same time wants to surf, but only a small bandwidth to the Internet is
available.

\subsection{Preconditions:}

A running base flil4l system installed on a harddisk is needed.
Minimum requirements are:\\
CPU: 100 MHz\\
RAM:  32 MB\\
HD:  100 MB

\wichtig{Squid is a process on the fli4l! Therefore every system that has
unrestricted access on the input list of the packet filter configuration
also has the same unrestricted Internet access over Squid. This relates to the
protocols HTTP, HTTPS and FTP!}

\section{Variables in squid.txt:}

\begin{description}
\config{SQUID\_MANAGER}{SQUID\_MANAGER}{SQUIDMANAGER}

                Specifies the \mbox{E-Mail}-address of the local proxy administrator.
		This address is also the contact address in error message outputs.
		Likewise the {E-Mail}-address is used as the login for the cache manager.
                Default Setting is:
\begin{verbatim}
	SQUID_MANAGER='squid@fli4l'
\end{verbatim}


\config{SQUID\_PASSWORD}{SQUID\_PASSWORD}{SQUIDPASSWORD}

                Password for the cache manager.

                Default Setting is:
\begin{verbatim}
	SQUID_PASSWORD='fli4l'
\end{verbatim}


\config{SQUID\_TRANSPARENT\_CACHING}{SQUID\_TRANSPARENT\_CACHING}{SQUIDTRANSPARENTCACHING}
\config{SQUID\_TRANSPARENT\_FORWARDING}{SQUID\_TRANSPARENT\_FORWARDING}{SQUIDTRANSPARENTFORWARDING}

    If activating this function, Squid filters all ``surf accesses'' from the
    internal net to the Internet and thus caches them. This can be of use if you
    want to cache those clients not explicitely having a proxy configured in their
    bowsers. The client does not recognize the redirection. The advantage:
\begin{enumerate}
    \item{You can cache the line to the Internet
	  even if you do not want to take care of the proper
	  configuration of the clients.}
    \item{Programs not supporting a proxy will be cached as well
	  and can make use of the ``faster connection''}
\end{enumerate}
    Default Setting is:
\begin{verbatim}
	SQUID_TRANSPARENT_CACHING='no'
	SQUID_TRANSPARENT_FORWARDING='no'
\end{verbatim}

                \wichtig{If these options are set to 'yes', you can
                                no longer use port forwarding on port 80.
                                That would lead to a conflict and Squid
                                would stop working.}

                \emph{Furthermore, access blocking
                                 for individual IP's in the base.txt is not
                                 working anymore! The fli4l itself always has
                                 full access to the Internet.}
        
                \emph{If transparent caching is enabled, an additional port
                          will be opened to allow Squid to deliver certain pages
                          (i.e. error pages) seamlessly. This is the value
                          of \var{SQUID\_HTTP\_Port} (see below) plus one. So if
                          \var{SQUID\_HTTP\_Port=3128} is set, the port 3129 is
                          also opened and hence may not be assigned to other applications
                          on the flil4l.}

\config{SQUID\_HTTP\_PORT}{SQUID\_HTTP\_PORT}{SQUIDHTTPPORT}

		Here you can specify on which TCP port Squid should wait for connections.
                Default Setting is:
\begin{verbatim}
	SQUID_HTTP_PORT='3128'
\end{verbatim}

                \achtung{If transparent caching is enabled
                (\var{SQUID\_TRANSPARENT\_CACHING='yes'}, see above), an additional port
                will be openend, \var{SQUID\_HTTP\_PORT} plus one. This port may not be
                assigned to other applications on the flil4l.}


\config{SQUID\_MEM\_CACHE\_SIZE}{SQUID\_MEM\_CACHE\_SIZE}{SQUIDMEMCACHESIZE}

		This setting determines how much space the cache
                may take in RAM in maximum. Set the size in MB.

                Recommended is 1/4 of the available main memory of the router.

\config{SQUID\_DISK\_CACHE\_SIZE}{SQUID\_DISK\_CACHE\_SIZE}{SQUIDDISKCACHESIZE}

		This setting determines how much space the cache is allowed
		to take on the hard disk as a maximum. Set the size in MB.
		While the cache gets bigger, Squid automatically deletes old
		files so that the maximum size is not exceeded.
		On the disk at least 25\% should be still free at full cache
		because Squid becomes very slow when the disk is nearly full.
		If Squid has no space on the disk left to write files, it exits
		with an error.

		Also note that Squid needs about 6 MB of RAM for itself plus
		the RAM specified at \var{SQUID\_MEM\_CACHE\_SIZE} plus about
		6 MB of RAM per GB of cache on the disk. The Squid process may
		not be swapped, physical memory must be available in corresponding amount!

		When hitting problems set SQUID\_DISK\_CACHE\_SIZE smaller.
		Squid with small cache is faster than Squid with big cache and
		not enough RAM.


\config{SQUID\_MAX\_OBJECT\_SIZE}{SQUID\_MAX\_OBJECT\_SIZE}{SQUIDMAXOBJECTSIZE}

		The value is specified in kB. Objects that are larger than
		this value are not saved to disk. If you want to save a lot
		of traffic at the expense of high speed, you should set this
		value high. If you want higher speed at the expense of more traffic
		this value should be set to 10000 kB or less.

                Default Setting is:
\begin{verbatim}
	SQUID_MAX_CACHE_SIZE='65536'
\end{verbatim}


\config{SQUID\_WORK\_DIR}{SQUID\_WORK\_DIR}{SQUIDWORKDIR}

		In this directory Squid stores its cache, the log files,
		and other files relevant for operating.

		The cache is stored in:
				SQUID\_WORK\_DIR/cache

		Similarly, the logs:
				SQUID\_WORK\_DIR/logs

		SQUID\_WORK\_DIR should therefore point to a hard disk with enough
		capacity. Important requirement is having read/write access
		and that the hard drive is formatted with the ext file system.
		More detailed information on disk preparation and use under fli4l
		can be found in the HD package.

                Default Setting is:
\begin{verbatim}
	SQUID_WORK_DIR='/data/squid'
\end{verbatim}

		Alternatively, you may use 'auto' here so that the system path
		for persistent files is used. Pay attention to \var{FLI4L\_UUID}
		pointing to a medium with sufficient space, otherwise /boot or if
		this is not writable even the RAMdisk will be used.

\config{SQUID\_CYCLE\_LOG\_N}{SQUID\_CYCLE\_LOG\_N}{SQUIDCYCLELOGN}

		This option determines the number of logfile rotations. If
		setting this to the value "0", the logfiles will only be
		closed and reopened. This allows you to rename the log files
		yourself.

                Default Setting is:
\begin{verbatim}
	SQUID_CYCLE_LOG_N='10'
\end{verbatim}

                This equals 10 rotating log files.


\config{SQUID\_CYCLE\_LOG\_TIME}{SQUID\_CYCLE\_LOG\_TIME}{SQUIDCYCLELOGTIME}

		Since Squid, once it runs, mostly is performing without problems,
		it comes in handy that the log files regularly may be deleted automatically.
		This avoids that the hard drive will overflow. The value is specified in seconds.
		1 hour corresponds to 3600 seconds.

                Default Setting is:
\begin{verbatim}
	SQUID_CYCLE_LOG_TIME='172800'
\end{verbatim}

                This equals 48 hours.


\section{Variables Important Only For Network Configurations Outside base.txt}

\config{SQUID\_AUTO\_CONFIG}{SQUID\_AUTO\_CONFIG}{SQUIDAUTOCONFIG}

		This option determines whether the networks with rights to access the services
		of Squid should automatically be taken from the base.txt, or whether the networks
		are configured manually by the following fields.

		For large networks with multiple IP ranges and other
		routers, this automatic may be combined with the following options.

\config{SQUID\_ACCESS\_NET\_N}{SQUID\_ACCESS\_NET\_N}{SQUIDACCESSNETN}

		Number of networks that are allowed to access the services of Squid.
		Normally this value is '0 'because all networks configured directly on the
		flil4 are already detected by SQUID\_AUTO\_CONFIG.

\config{SQUID\_ACCESS\_NET\_x}{SQUID\_ACCESS\_NET\_x}{SQUIDACCESSNETx}

                Specify networks here that are allowed to access Squid.
                This information will be used only internally by Squid
                to control the user's web access through Squid. Access to Squid itself
                has to be allowed with additional packet filter rules if necessary
                (see \var{PF\_INPUT\_x} in base.txt or \var{PF6\_INPUT\_x} in
                ipv6.txt).

                An example would be:
\begin{verbatim}
	SQUID_ACCESS_NET_1='192.168.0.0/24'
\end{verbatim}

                All computers with IP addresses 192.168.0.XXX are allowed to use Squid.

\section{Optional Variables Not Necessary For Operation}

\config{SQUID\_CONNECT\_TIMEOUT}{SQUID\_CONNECT\_TIMEOUT}{SQUIDCONNECTTIMEOUT}

		This option determines how long Squid waits for a
		full TCP connection to a server. Within this
		time (default: 120 seconds) the connection to the server must
		be established.

\config{SQUID\_CACHE\_N}{SQUID\_CACHE\_N}{SQUIDCACHEN}

		If you want to spread the cache over multiple disks
		you may activate this here. With this you can specify
		several additional cache directories. In order to
		do this the index must be set to the desired number of
		additional directories.

                Now fill the varable
\config{SQUID\_CACHE\_1\_DIR}{SQUID\_CACHE\_1\_DIR}{SQUIDCACHE1DIR}
\config{SQUID\_CACHE\_1\_SIZE}{SQUID\_CACHE\_1\_SIZE}{SQUIDCACHE1SIZE}
                with the values desired. If, for example a second harddisk with
                a capacity of 4 GB is mounted under /disk2:
\begin{verbatim}
	SQUID_CACHE_1_DIR='/disk2'
	SQUID_CACHE_1_SIZE='3000'
\end{verbatim}
                This harddisk would be used only for Squid cache now.

                Default Setting is: no additional cache\_dir.


\config{SQUID\_NEXT\_PROXY}{SQUID\_NEXT\_PROXY}{SQUIDNEXTPROXY}

		If the internet access is only possible via a proxy,
		or for performance reasons an upstream proxy should
		be accessed, is has to be entered here in the format
		'name.domain', eg
\begin{verbatim}
	SQUID_NEXT_PROXY='www-proxy.t-online.de'
\end{verbatim}
		In case of failure of this proxy Squid tries to resolve the pages directly.

                Default Setting is: no upstream proxy


\config{SQUID\_NEXT\_PROXY\_PORT}{SQUID\_NEXT\_PROXY\_PORT}{SQUIDNEXTPROXYPORT}

                Port of the upstream proxy. Only taken into account if
                SQUID\_NEXT\_PROXY is not empty, i.e.
\begin{verbatim}
	SQUID_NEXT_PROXY_PORT='80'
\end{verbatim}

                Default Setting is: empty

\section{ONLY FOR EXPERTS!}

In file opt/etc/rc.d/rc455.squid all variables from squid.conf can be
found. Those in need of changes to squid.conf may incorporate them there.
When booting fli4l the changes made are taken over to the dynamically
generated squid.conf.

\section{Links}
For further questions I advice to check the homepage of the Squid project and
the FAQs there: \altlink{http://wiki.squid-cache.org/SquidFaq}\\
or the (german) manual at \altlink{http://www.Squid-handbuch.de/hb/}

\section{Credits}
Original author: Hermann Strassner (hermann.strassner@web.de)

Extended and changed by: Ingo Winiarski (iwiniarski@gmx.de)
\end{description}
