% Last Update: $Id$
Das OPT\_C3SURF\_VOUCHER ermöglicht einen anonymen Zugang in das Internet. Es werden Gutscheine
erstellt, die in unterschiedlichen Kategorien eingerichtet werden können. Das Opt kann dann über das
Webinterface manuell oder automatisch verwaltet werden.

\section {Konfiguration von OPT\_VOUCHER}

\begin{description}
\config {OPT\_C3SURF\_VOUCHER}{OPT\_C3SURF\_VOUCHER}{OPTC3SURFVOUCHER}

   \var{OPT\_C3SURF\_VOUCHER='no'}
   
  Das Gutscheinsystem von Opt C3SURF\_VOUCHER aktivieren ('yes'), Standard ist 'no'. Gutscheine sind
  anonyme aber sichere Einmal-Accounts, die zur Anmeldung verwendet werden können.
  Voraussetzung ist die Einstellung \jump{OPTLOGINUSR}{\var{OPT\_LOGINUSR='yes'}}.

  Die Erzeugung und Löschung der Gutscheine übernehmen zwei nächtliche cron JOBs, welche jederzeit auch
  manuell(Admin-Interface) gestartet werden können. Im Folgenden wird erklärt wie diese Jobs verwaltet
  werden.

  Alle neu generierten Gutscheine werden an eine Druckliste angehängt. Nur in der Druckliste ist das zum
  Gutschein gehörende Kennwort im Klartext gespeichert. Du kannst diese Liste jederzeit Herunterladen,
  Drucken oder Löschen. Nach dem Löschen der Liste kann das Kennwort nicht wieder ermittelt werden.
  Normal wird die Liste zuerst gedruckt und dann gelöscht. Es darf immer nur ein gedrucktes Exemplar eines
  Gutscheines geben. Die Druckfunktion ist in html implementiert, Seitenumbrüche werden leider nicht beachtet.
  Gutscheine, die durch Seitenwechsel nicht gedruckt werden, sollten vernichtet werden (allerdings verfallen
  sie sowieso durch Zeitablauf). Listen, die nicht gedruckt, sondern heruntergeladen wurden, können mit anderen
  Programmen ein eigenes Layout erhalten, welches dann auch Seitenwechsel sauber berücksichtigt.

\config {C3SURF\_VOUCHER\_N}{C3SURF\_VOUCHER\_N}{C3SURFVOUCHERN}

   \var{C3SURF\_VOUCHER\_N='n'}
    
    Wertebereich: 0 und Natürliche Zahlen
    
  Wie viele verschiedene Gutscheinkategorien sollen erzeugt werden? Wichtigstes Kriterium
  für Gutscheine ist die Laufzeit. Daneben gibt man noch die Anzahl der
  Gutscheine und deren Gültigkeit in Tagen an. Siehe auch die folgenden Variablen.

\config {C3SURF\_VOUCHER\_x\_TIME}{C3SURF\_VOUCHER\_x\_TIME}{C3SURFVOUCHERxTIME}
   
   \var{C3SURF\_VOUCHER\_x\_TIME='30'}
    
    Wertebereich: Natürliche Zahlen
    
  Laufzeit in Minuten (hier: 30) für einen Gutschein dieser Kategorie ('n' siehe oben).

\config{C3SURF\_VOUCHER\_x\_COUNT}{C3SURF\_VOUCHER\_x\_COUNT}{C3SURFVOUCHERxCOUNT}

   \var{C3SURF\_VOUCHER\_x\_COUNT='3'}
   
   Wertebereich: Natürliche Zahlen
  
  Wie viele Gutscheine dieser Kategorie (in diesm Fall 3) sollen erzeugt werden? 
  
\config{C3SURF\_VOUCHER\_x\_DAYS}{C3SURF\_VOUCHER\_x\_DAYS}{C3SURFVOUCHERxDAYS}
   
   \var{C3SURF\_VOUCHER\_x\_DAYS='90'}
   
   Wertebereich: 0 und Natürliche Zahlen
   
  Wie viele Tage soll der Gutschein ab seiner Erzeugung gültig sein (hier:90).
  Damit wird ein Verfallsdatum für diesen Gutschein erzeugt. Die Löschung
  erfolgt dann entweder manuell oder per cron-Job. Der Gutschein erlischt,
  wenn er das erste Mal verwendet wird.
  
  \wichtig{'0' bedeutet, dass Gutscheine dieser Kategorie kein Verfallsdatum haben. Sie werden
           erst mit Benutzung oder wenn die Zeit komplett verbraucht wurde ungültig (wird auch durch
           C3SURF\_VOUCHER\_LIVES\_N beeinflusst). Sie können jedoch jederzeit im Admin-Interface
	   des fli4l gelöscht werden.}

\end{description}

\subsection {Optionale Parameter OPT\_VOUCHER}
  
\begin{description}

\config{C3SURF\_VOUCHER\_x\_LIVES}{C3SURF\_VOUCHER\_x\_LIVES}{C3SURFVOUCHERxLIVES}
   
   \var{C3SURF\_VOUCHER\_x\_LIVES='n'}
   
   Wertebereich(n): -1, 0, Natürliche Zahlen
   
  Angabe von Stunden, die der Voucher nach der ersten Anmeldung noch gültig ist.

\achtung{Sonderfälle:}

\begin{itemize}
   \item{C3SURF\_VOUCHER\_x\_LIVES='-1'} 
         
         bis zum ursprünglich generierten Verfallsdatum aus C3SURF\_VOUCHER\_DAYS gültig
   \item{C3SURF\_VOUCHER\_x\_LIVES='0'}
         
         (Standard), bedeutet Voucher wird mit erster Anmeldung ungültig.
    \item{C3SURF\_VOUCHER\_x\_LIVES='Natürl. Zahl'}
          
          Anzahl der Stunden, die der Voucher nach erster Anmeldung weiter gültig ist - ggf. ein neues
          Verfallsdatum berechnen.
\end{itemize}
\parskip 12pt

  Diese Gutscheine werden nicht mit der ersten Anmeldung ungültig, sondern gelten 'n' Stunden
  weiter. Sobald der Gutschein benutzt wird, wird daraus ein zeitlich limitierter LOGINUSR-Account generiert
  oder es wird das Verfallsdatum des Vouchers neu berechnet. Dieser Account / Voucher darf sich beliebig oft
  an und wieder abmelden. Es wird das von LOGIN\_USR gewohnte Quota-System für diesen Account verwendet. Erst
  wenn die gesamte Zeit verbraucht ist oder wenn das Verfallsdatum (C3SURF\_VOUCHER\_DAYS\_N) erreicht wurde,
  wird dieser Account automatisch von C3SURF gelöscht.


\config{C3SURF\_VOUCHER\_DEL\_CRON}{C3SURF\_VOUCHER\_DEL\_CRON}{C3SURFVOUCHERDELCRON}

   \var{C3SURF\_VOUCHER\_DEL\_CRON='0 4 * * *'}
    
   Wertebereich: 'cron-Syntax' oder 'never'
   
  Der oben angegebene Wert ist der Standard, wenn diese Variable in der config-Datei
  'c3surf.txt' fehlt.
  Standard: lösche täglich morgens um 4 Uhr alle verfallenen Guscheine.

  Die cron-Syntax ist einzuhalten und wird nicht geprüft. Zusätzlich kann der Wert 'never' verwendet werden.
  Dann wird der Job vom System überhaupt nicht eingeplant. Man kann im Admin-Interface jederzeit manuell
  alle verfallenen Gutscheine löschen lassen.

\config{C3SURF\_VOUCHER\_GEN\_CRON}{C3SURF\_VOUCHER\_GEN\_CRON}{C3SURFVOUCHERGENCRON}

   \var{C3SURF\_VOUCHER\_GEN\_CRON='15 4 * * *'}
   
   Wertebereich: 'cron-Syntax' oder 'never'
  
  Der oben angegebene Wert ist der Standard, wenn diese Variable in der config-Datei
  'c3surf.txt' fehlt.
  Standard: generiere täglich morgens um 4:15 Uhr neue Gutscheine, falls weniger als C3SURF\_VOUCHER\_COUNT
  vorhanden sind.
  
  Die cron-Syntax ist einzuhalten und wird nicht geprüft. Zusätzlich kann der Wert 'never' verwendet werden.
  Dann wird der Job vom System überhaupt nicht eingeplant. Man kann im Admin-Interface jederzeit manuell
  neue Gutscheine bis zur Menge\\
  \jump{C3SURFVOUCHERxCOUNT}{\var{C3SURF\_VOUCHER\_x\_COUNT}} erzeugen lassen.
  
  Alle neu generierten Gutscheine werden an eine Druckliste angehängt. Nur in der Druckliste ist das zum
  Gutschein gehörende Kennwort im Klartext gespeichert. Jeder Gutschein sollte nur einmal gedruckt werden.
  Die Liste sollte sofort nach dem Ausdruck oder Herunterladen gelöscht werden.

\config{C3SURF\_VOUCHER\_PRTUPDATE}{C3SURF\_VOUCHER\_PRTUPDATE}{C3SURFVOUCHERPRTUPDATE}
   
   Standard-Einstellung: \var{C3SURF\_VOUCHER\_PRTUPDATE='no'}
   
   Wertebereich: 'yes' oder 'no'
   
   Aktualisierung der Druckdatei. Meine Empfehlung: 'no'. Wer wenige Gutscheine im System hält und
   die Druckdatei nach dem Ausdruck oder Herunterladen nicht löschen möchte, kann mit 'yes' eine
   Aktualisierung der Druckdatei beim Verbrauch von Gutscheinen einstellen. Bei der Wahl von 'yes' werden
   benutzte Gutscheine auch aus der Druckdatei gelöscht. Das benötigt Ressourcen auf dem Router.
   
\config{C3SURF\_VOUCHER\_USRLEN}{C3SURF\_VOUCHER\_USRLEN}{C3SURFVOUCHERUSRLEN}
   
   Standard-Einstellung: \var{C3SURF\_VOUCHER\_USRLEN='12'}
   
   Wertebereich: '1-16'
   
   Zeichenlänge für Gutscheinaccount festlegen, ab 8 Zeichen werden '-' als Trenner eingebaut,
   die auch eingegeben werden müssen. Es werden immer vier Zeichen gruppiert. Der Maximalwert ist 16.
   
\config{C3SURF\_VOUCHER\_USRCAP}{C3SURF\_VOUCHER\_USRCAP}{C3SURFVOUCHERUSRCAP}
   
   Standard-Einstellung: \var{C3SURF\_VOUCHER\_USRCAP='random'}
   
   \begin{tabular}{rlrl}
    -&'yes'&:&nur Großbuchstaben \\
    -&'no'&:&nur Kleinbuchstaben \\
    -&'random'&:&zufällige Wechsel von Groß- Kleinschreibung (Empfehlung) \\
   \end{tabular}

    Mit dieser Variable wird festgelegt ob Groß- oder Kleinbuchstaben im Benutzernamen verwendet
    werden sollen. Der Wert \var{'random'} (empfohlen) bewirkt eine zufällige Auswahl. 

\config{C3SURF\_VOUCHER\_PWDLEN}{C3SURF\_VOUCHER\_PWDLEN}{C3SURFVOUCHERPWDLEN}

   Standard-Einstellung: \var{C3SURF\_VOUCHER\_PWDLEN='6'}
   
   Wertebereich: 1-12
   
   Zeichenlänge für das Gutschein-Password.
   
\config{C3SURF\_VOUCHER\_PWDMOD}{C3SURF\_VOUCHER\_PWDMOD}{C3SURFVOUCHERPWDMOD}

   Standard-Einstellung: \var{C3SURF\_VOUCHER\_PWDMOD='3'}
   
   Wertebereich: 1-5
   
   Modulo für zufällige Verlängerung des Passwortes. Max: 5 (die Werte 0, 1, 2, 3, 4), 
   Min 1 (der Wert 0). Es wird bei der Password-Generierung das Password zufällig um die
   möglichen Werte verlängert. Damit ergeben sich per default Password-Längen zwischen 6 und 8.
   Maximal einstellbar sind Password-Längen zwischen 12 und 16, das ist sicher genug mit zufälliger
   Groß- und Kleinschreibung.

\config{C3SURF\_VOUCHER\_PWDCAP}{C3SURF\_VOUCHER\_PWDCAP}{C3SURFVOUCHERPWDCAP}

   Standard-Einstellung: \var{C3SURF\_VOUCHER\_PWDCAP='random'}
   
   \begin{tabular}{rlrl}
    -&'yes'&:&nur Großbuchstaben \\
    -&'no'&:&nur Kleinbuchstaben \\
    -&'random'&:&zufälliger Wechsel von Groß- und Kleinschreibung (Empfehlung) \\
   \end{tabular}
   
    Mit dieser Variablen wird festgelegt ob Groß- oder Kleinbuchstaben im Passwort verwendet
    werden sollen. Der Wert \var{'random'} (empfohlen) bewirkt eine zufällige Auswahl.
   
\end{description}
