% Last Update: $Id$
\section {fli4l Webinterface}

\achtung{Rechte:}
c3surf:view, admin
\begin{itemize}
    \item view: Den Eintrag im Admin Menü anzeigen
    \item admin: Für die Nutzung der Funktionen im Web-Interface.

    Httpd-User mit dem Recht ``all'' haben auch hier alle Rechte.

    \item Das OPT C3SURF trägt sich in der Weboberfläche unter ``Opt'' als ``c3Surf'' ein.
\end{itemize}


\section {Funktionsweise}

Mit c3surf können Netze oder einzelne Hosts spezifiziert werden, die nach
dem Systemstart zunächst gesperrt sind. Durch eine formlose Registrierung
über das Webinterface können Benutzern Freischaltungen auf Zeit vornehmen.
\parskip 12pt

Ist die LOGINUSR - Option aktiviert, dann können sich nur Personen anmelden,
die einen gültigen Zugang mit Benutzernamen und Passworte besitzen.
\parskip 12pt

Über das Admin-Interface des Routers können Benutzer oder MAC-Adressen angezeigt,
ausgeloggt sowie temporär oder dauerhaft gesperrt werden. Die Sperrung betrifft
immer nur die Anmeldung über c3Surf und wird von c3Surf verwaltet. Kommt der
Rechner über ein anderes interface an den Router, hat die Sperre keine Wirkung.
\parskip 12pt

Die Anmeldung erfolgt durch Vornamen, Namen und E-Mail-Adresse oder durch
User/Passwort. Nach Ablauf einer eingestellten Zeit wird der Zugang wieder
gesperrt und kann durch neue Anmeldung wieder freigeschaltet werden.
\parskip 12pt

Die Anmeldeseite kann auch für die Benutzer gesperrt werden (siehe FreeSurf bzw.
LoginUsr im OPT-Menue des Webinterfaces).
\parskip 12pt

Das alles kann im Admin-Teil des Webinterfaces nachvollzogen werden.
\parskip 12pt

Will man bestimmte Rechner dauerhaft freischalten, so kannst dies im
Web-Interface erfolgen. Siehe dazu die ARP-Liste oder die DHCP-Leases.
\parskip 12pt

Jede Nutzung kann mit Quotas versehen werden. Damit kann die Nutzung in der
Zeit eingeschränkt werden. Mit den Parametern ``-TIME'', ``-BLOCKTIME'' und
``-COUNTER'' lässt sich dabei sehr viel einstellen, auch benutzerbezogen.

\underline{Beispiele:}\\

\begin{tabular}{|c|c|c|p{9cm}|}
 \hline
   Time & Blocktime & Counter & Quota (\var{C3SURF\_QUOTA='yes'})\\
 \hline
 \hline
   60 & -1 & 0 &  60 Min Zeit, keine Sperre nach Ablauf, mit jeder Anmeldung läuft die Zeit (Parkuhrprinzip)\\
 \hline 
   60 & 240 & 0 & 60 Min Zeit, danach für 240 gesperrt, mit jeder Anmeldung läuft die Zeit 
   (Parkuhrprinzip = Geld rein, Zeit läuft ohne Unterbrechungsmöglichkeit)\\
 \hline
   60 & 0 & -1 &  60 Min Zeit, nach Ablauf der Zeit wird der Zugang bis 00:00 Uhr gesperrt,
   beliebige An- Abmeldungszahl (kein Parkuhrprinzip) \\
 \hline
   60 & -1 & 1 &   60 Min Zeit, keine Sperre nach Ablauf der Zeit, die Zeit kann 1x unterbrochen werden \\
 \hline
   60 & -1 & -1 &  60 Min Zeit, keine Sperre nach Ablauf der Zeit, beliebig viele Unterbrechungen möglich \\
 \hline
   600 & 10080 & -2 & 10 Stunden innerhalb einer Woche mit beliebig vielen Unterbrechungen \\
 \hline
   0 & -1 & 0 &   Unendliche Zeit mit jeder Anmeldung, keine Sperre nach Ablauf der Zeit \\
 \hline
\end{tabular}

\section{Namensauflösung - DNS}

\achtung{Wichtig: Der fli4l muss als DNS-Server bei den Clients eingetragen sein und muss befähigt sein Namen
              aufzulösen. Dazu}\\    
          \begin{itemize}
                \item \emph{benötigt er einen ``Forward'' auf den DNS-Server des Netzes oder}
                \item \emph{er ist selbst der DNS-Server und kann ggf. automatisch Verbindugen aufbauen.}
          \end{itemize}
               
               \emph{Sonst gibt es Probleme automatisch auf die Anmeldeseite umzuleiten. Die kann aber immer
               noch manuell durch die Eingabe ihrer URL aufgerufen werden.}
