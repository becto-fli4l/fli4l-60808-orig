% Last Update: $Id$
  Das OPT\_C3SURF\_TRAFFIC ermöglicht ``Power User'' auszubremsen und in ihre Schranken zu weisen.
  Es wird das Datenvolumen in einem definierbaren Zeitintervall überwacht und ausgewertet. Die 
  Kofiguration kann nach eigenen Bedürfnissen angepasst werden.

\section {Konfiguration OPT\_C3SURF\_TRAFFIC}

\begin{description}

\config{OPT\_C3SURF\_TRAFFIC}{OPT\_C3SURF\_TRAFFIC}{OPTC3SURFTRAFFIC}
   
  Standard-Einstellung: \var{OPT\_C3SURF\_TRAFFIC='no'}
   
  Die Angabe von 'yes' hier aktiviert das Traffic-Modul. Die weitere Variablen sind im
  Folgenden beschrieben. Die Defaultwerte sind für einen DSL-6000 Anschluss ausgelegt
   
  Mit den folgenden Variablen kann eingestellt werden in welcher Zeit welches Datenvolumen nicht überschritten
  werden darf. Dabei wird nicht zwischen Up- und Download unterschieden. Die Logik dieses Moduls ist so konzipiert,
  dass bei zweimaliger Überschreitung des Volumens in Folge die Abmeldung des Verursachers erfolgt und dieser mit
  der definierten Zeitstrafe (Blockzeit) belegt wird. Diese Einstellungen wirken global für alle C3SURF-Benutzer.
  Die Wahl der richtigen Parameter sollte von der vor Ort verfügbaren Bandbreite abhängen. Da keine Sperre bei
  einmaliger Überschreitung erfolgt, kann auch mal ein Betriebssystem-Update oder das normale Herunterladen größerer
  Datenmengen erfolgen. Sollte der Verbrauch der Bandbreite aber als ``dauerhaft'' erkannt werden wird eine
  Sperre aktiv.
   
  Will man beispielsweise das gelegentliche Herunterladen von größeren Datenmengen erlauben, so muss bei
  der Wahl der Parameter aus der erlaubten Datenmenge und der verfügbaren Bandbreite eine Zeit kalkuliert
  werden, in der die Datenmenge herunterladbar sein soll.

\achtung{Beispiel:}

   Das Herunterladen einer Distributions-CD (700MB) würde im günstigsten
   Fall folgende Zeiten in Anspruch nehmen:
   
   \begin{tabular}{lrlrl}
    
    DSL-&1000 & ca.& 93 & Minuten \\
    DSL-&2000 & ca.& 47 & Minuten \\
    DSL-&6000 & ca.& 16 & Minuten \\
    DSL-&16000 & ca.& 6 & Minuten \\
    
   \end{tabular}

   Die erlaubte Datenmenge (hier 700MB) sollte durch einen Wert kleiner (aber nahe an) 2 und größer 1 geteilt werden.
 
   \achtung{Beispiel (konservativ):} 700MB / 1,9 = 386317473 Bytes \\
   Das wäre die Anzahl Bytes, die in der oben kalkulierten Zeit maximal heruntergeladen werden darf. Ob es
   sinnvoll ist, solch ein hohes Volumen bei DSL-1000 oder DSL-2000 pro Nutzer zuzulassen, hängt auch von
   der Anzahl der erwarteten Nutzer ab.

   Will man solche Datenmengen nicht zulassen, sondern maximal z. B. das Musikhören eines mp3-Streams oder
   einen kontinuierlichen Datenstrom von 128 kBit/s erlauben, so sollte man unabhängig vom DSL-Anschluss
   folgende Werte wählen: 16220160 Bytes pro 15 Minuten (ergibt sich aus 128kBit/s * 1024 / 8Bit = 16384
   Bytes/s * 60 = 983040 Bytes/min * 15min = 14745600 Bytes * 1,1 = 16220160 Bytes (pro 15 min)).
   Da es sich um einen kontinuierlichen Datenstrom handelt darf nicht geteilt werden, denn diese Last ist
   ja immer zulässig. Hier ist es sinnvoll noch 10\% Sicherheitsaufschlag zu kalkulieren, da neben der
   reinen Datenmenge auch noch andere Informationen transportiert werden müssen. Daher wurde der errechnete
   Wert von 14745600 Bytes noch mit 1,1 multipliziert.

   Im Folgenden werden die Variablen vorgestellt mit Default-Werten für das hier angeführte Beispiel mit dem
   gelegentlichen CD-Download an einem DSL-6000 Anschluss.
   
\config{C3SURF\_TRAFFIC\_BYTES}{C3SURF\_TRAFFIC\_BYTES}{C3SURFTRAFFICBYTES}
   
   \var{C3SURF\_TRAFFIC\_BYTES='386317473'}
   
   Wertebereich: natürliche Zahlen
   
   Legt die Anzahl der Bytes fest, die maximal in der Zeit \jump{C3SURFTRAFFICMINUTES}
   {\var{C3SURF\_TRAFFIC\_MINUTES}} heruntergeladen werden darf. Hier als Beispiel der 1,9te Teil einer
   700MB CD. Für das Beispiel mp3-Hören mit 128kBit ist hier 16220160 einzutragen.
   
\config{C3SURF\_TRAFFIC\_MINUTES}{C3SURF\_TRAFFIC\_MINUTES}{C3SURFTRAFFICMINUTES}
   
   \var{C3SURF\_TRAFFIC\_BYTES='16'}
   
   Wertebereich: natürliche Zahlen

   Legt die Zeit in Minuten fest, die zwischen zwei Messungen des Datenvolumens verstreicht. Wird nach der
   hier abgelaufenen Zeit eine Überschreitung festgestellt, so wird der Verursacher zunächst temporär
   gespeichert. Wird bei der nächsten Messung wieder eine Überschreitung festgestellt, so erfolgt automatisch
   die Abmeldung und Sperrung (für \jump{C3SURFTRAFFICBLOCKTIME}{\var{C3SURF\_TRAFFIC\_BLOCKTIME}} Minuten).
   Wird bei der zweiten Messung keine Überschreitung mehr festgestellt, wird die temporäre Speicherung wieder
   gelöscht.

   Für das mp3-Beispiel ist hier '15' einzustellen.

\config{C3SURF\_TRAFFIC\_BLOCKTIME}{C3SURF\_TRAFFIC\_BLOCKTIME}{C3SURFTRAFFICBLOCKTIME}
   
   \var{C3SURF\_TRAFFIC\_BLOCKTIME='60'}
   
   Wertebereich: ganze Zahlen

   Legt die Zeit in Minuten fest, die ein Zugang nach Überschreitung des Traffic-Limits gesperrt wird.
   
   
\end{description}
