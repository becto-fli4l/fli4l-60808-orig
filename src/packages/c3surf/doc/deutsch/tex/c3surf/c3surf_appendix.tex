% Last Update: $Id$
\begin{tabular}{rlcl}
  Opt und Doku: & 07. Januar 2008 & Frank Saurbier & \altlink{mailto:c3surf@arcor.de}\\
  Doku-TexSatz: & 01. April 2009 & Helmut Backhaus & \altlink{mailto:helmut.backhaus@gmx.de}\\
  Übergabe: & 01. Mai 2010 & fli4l-Team & \altlink{mailto:team@fli4l.de}\\
\end{tabular}


\section{Tips zu anderen Opts}

\subsection{cpmvrmlog Config}

\marklabel{C3SURFLOG}{ }
\underline{Beispiel für das C3SURF-Logverzeichnis, mit restart des mini\_httpd}
\begin{example}
\begin{verbatim}
# archive C3SURF log dir
# einmal im Monat am 1. um 01:30
# maximal 12 Archive aufbewahren
CPMVRMLOG_n_ACTION='move'
CPMVRMLOG_n_SOURCE='/var/log/c3surf/c3surf_*.log'
CPMVRMLOG_n_DESTINATION='/data/Archive/log/c3surf'
CPMVRMLOG_n_CUSTOM='/usr/local/bin/c3surf_kill_httpd.sh'
CPMVRMLOG_n_MAXCOUNT='12'
CPMVRMLOG_n_CRONTIME='30 1 1 * *'
\end{verbatim}
\end{example}

\subsection{Samba ohne Anmeldung erlauben}

\underline{Man nehme das opt\_usercmd und trage dort folgendes ein.}

\begin{example}
\begin{verbatim}
USERCMD_BOOT_N='3'
USERCMD_BOOT_1='/sbin/iptables -I c3surf\_control 1 -v -p udp --dport 
                137:138 -j RETURN' # samba thru c3surf
USERCMD_BOOT_2='/sbin/iptables -I c3surf\_control 1 -v -p tcp --dport
                455 -j RETURN'     # samba thru c3surf
USERCMD_BOOT_3='/sbin/iptables -I c3surf\_control 1 -v -p tcp --dport
                139 -j RETURN'     # samba thru c3surf
\end{verbatim}
\end{example}
Durch hinzufügen der Option \var{-d IPsambaHOST} kann die
jeweilige Regel noch um den Zielrechner erweitert werden.
\parskip 12pt

Damit werden die Samba Ports normal durch die Forward-Chain geleitet und
nicht mehr von C3SURF geblockt. Sollten in der Forward-Chain samba
Weiterleitungen verboten sein, so ändern diese Eintragungen nichts daran.

Es gelten also immer noch die Einstellungen in base.txt.

\subsection{Migration aus Vorgängerversionen}

\item Migration zur Version 2.3.1 (von 2.3.0)
     \begin{itemize}
     \item Neue Variablen. Da es sich nur um optionale Variablen handelt,
           ist die Bearbeitung nicht zwingend erforderlich. Der neue Bereich ist in der config.txt so
     \item ``\# $+$ new 2.3.1 $+$ begin ------------------ delete this line'' gekennzeichnet.
     \item Das Voucherformat hat sich geändert, alte Voucher können weiter verbraucht werden,
           sie werden aber bei der Generierung von neuen Vouchern nicht als existent erkannt.
           Wer es sauber haben will, sollte alle Voucher löschen und dann neu erzeugen.
     \end{itemize}

\item Migration zur Version 2.3.0 (von 2.2.2)
    \begin{itemize}
    \item Will man keine Gutscheine nutzen, sind keine Änderungen an der Konfiguration nötig.
    \item Bearbeite die neue Variablen für das OPT\_C3SURF\_VOUCHER, sofern man die Gutscheinfunktion nutzen möchte.
    \item Der neue Bereich ist in der config.txt so
    \item ``\# $+$ new 2.3.0 $+$ begin ------------------ delete this line'' gekennzeichnet.
\end{itemize}

\item Migration zur Version 2.2.2 (von 2.2.1)
    \begin{itemize}
    \item Bearbeite die neue Variablen. Sie sind in der config.txt so 
    \item ``\# $+$ new 2.2.2 $+$ begin ------------------ delete this line'' gekennzeichnet.
    \item C3SURF\_CONTROL\_SQUID: optional zur Kontrolle von squid, da squid nicht den Konventionen entspricht ist
    es vorläufig.
    \item Die Variablen zum überschreiben der Quota-Defaults bei LOGINUSR\_ACCOUNT sind jetzt optional
\end{itemize}

\item Migration zur Version 2.2.1 (von 2.2.0)
\begin{itemize}
   \item Bearbeite die neuen Variablen. Sie sind in der config.txt so
   \item ``\# $+$ new 2.2.1 $+$ begin ------------------ delete this line''  gekennzeichnet.
   \item C3SURF\_WORKON\_TMP: Empfehlung für Festplattenschlaf 'yes' sonst 'no' auch bei FLASH Medien (s.u.).
   \item C3SURF\_SAVE\_QUOTA: Empfehlung 'yes' (s.u.).
\end{itemize}


\item Migration zur Version 2.2.0 (von 2.1.0)
\begin{itemize}
   \item Bearbeite die neue Variable ``C3SURF\_CHECK\_ARP'' in der config nach
   (Empfehlung: 'yes', s.u.). Sie ist in der config.txt so
   \item ``\# $+$ new 2.2.0 $+$ begin ------------------ delete this line'' gekennzeichnet.
\end{itemize}


\item Migration zur Version 2.1.0 (von früheren Versionen)
\begin{itemize} 
   \item Bearbeite die neuen Variablen. Sie sind in der config.txt so
   \item ``\# $+$ new 2.1.0 $+$ begin ------------------ delete this line'' gekennzeichnet.
   \item Die MAC-Blackliste (so Du eine gepflegt hast) musst Du manuell ins Verzeichnis\\
   ``C3SURF\_PERSISTENT\_PATH'' kopieren.
   \item Das Format der c3surf\_login.log wurde um eine Spalte erweitert.
   Am besten die alte log sichern und in C3SURF\_LOG\_PATH löschen.
\end{itemize}
\end{itemize}
