% Last Update: $Id$
Stellt eine Anmeldung für Benutzer bereit. Damit kann nicht mehr
jeder das Internet oder die Dienste des Routers nutzen. Eine Umschaltung
im laufenden Bertrieb ist technisch möglich aber derzeit nicht implementiert.

\wichtig{Die ist keine echte Benutzeranmeldung, die Software
substituiert jeden Benutzer auf eine Rechneradresse. Nach Ablauf
der Quota wird dann nicht der Rechner (IP-Adresse) sondern der
Benutzer geblockt.}

\section{Konfiguration OPT\_LOGINUSER für C3SURF}

\begin{description}
\config {OPT\_LOGINUSR}{OPT\_LOGINUSR}{OPTLOGINUSR}

  Standard-Einstellung: \var{OPT\_LOGINUSR}='no'

  OPT\_LOGINUSR='yes': echte Anmeldung verwenden (wird empfohlen)

  LOGINUSR stellt eine echte Anmeldung (User/Password) zur Verfügung. Die Account Verwaltung erfolgt
  in der Config-Datei, Passwörter werden nur verschlüsselt übertragen.

\config {LOGINUSR\_DELETE\_PERSISTENT\_DATA}{LOGINUSR\_DELETE\_PERSISTENT\_DATA}{LOGINUSRDELETEPERSISTENTDATA}\ \\
  Standard-Einstellung: \var{LOGINUSR\_DELETE\_PERSISTENT\_DATA='no'}

  LOGINUSR\_DELETE\_PERSISTENT\_DATA\\
  Benutzerdaten auf einer Platte bleiben erhalten.
  Der Standardwert 'no' stellt dies für die Accountdaten sicher.

  \achtung{Mit der Eingabe von 'yes' werden alle Benutzer-Accounts bei jedem Neustart gelöscht.
  Danach erfolgt eine Neuanlage der Accounts, wie unten definiert.}

  Es wird empfohlen hier 'no' beizubehalten. Dann bleiben die Daten zu den
  Accounts erhalten, als da sind:

\begin{itemize}
\item Benutzer-Accounts
\item Quota-Daten, wenn \jump{C3SURFSAVEQUOTA}{\var{C3SURF\_SAVE\_QUOTA='yes'}} gewählt ist
(s.o.) (für einen einzelnen Account siehe: \jump{LOGINUSRACCOUNTxOVERWRITE}{\var{LOGINUSR\_ACCOUNT\_x\_OVERWRITE}})
\end{itemize}

\config {LOGINUSR\_ACCOUNT\_N}{LOGINUSR\_ACCOUNT\_N}{LOGINUSRACCOUNTN}

 \var{LOGINUSR\_ACCOUNT\_N='0'}

  LOGINUSR\_ACCOUNT\_N\\
  Anzahl Accounts, Wert: Ganze Zahl.\\
  Gibt die Anzahl der User-Accounts an.

\config {LOGINUSR\_ACCOUNT\_x\_USER}{LOGINUSR\_ACCOUNT\_x\_USER}{LOGINUSRACCOUNTxUSER}

   \var{LOGINUSR\_ACCOUNT\_x\_USER='user1'}

  LOGINUSR\_ACCOUNT\_x\_USER\\
  Username für die Anmeldung (muss zwingend definiert werden).

\config {LOGINUSR\_ACCOUNT\_x\_PWD}{LOGINUSR\_ACCOUNT\_x\_PWD}{LOGINUSRACCOUNTxPWD}

   \var{LOGINUSR\_ACCOUNT\_x\_PWD='user1\_secret'}

  LOGINUSR\_ACCOUNT\_x\_PWD\\
  Passwort für die Anmeldung (muss zwingend definiert werden)

\config {LOGINUSR\_ACCOUNT\_x\_FORENAME}{LOGINUSR\_ACCOUNT\_x\_FORENAME}{LOGINUSRACCOUNTxFORENAME}

   \var{LOGINUSR\_ACCOUNT\_x\_FORENAME='Vorname'}

  LOGINUSR\_ACCOUNT\_x\_FORENAME\\
  Vorname des Nutzers für die bessere Verwaltung (Optional, leer lassen erlaubt).
  Dieser Inhalt wird im Log und Admin-Interface angezeigt, so kann der Admin
  besser erkennen, wer gerade online ist.

\config {LOGINUSR\_ACCOUNT\_x\_SURNAME}{LOGINUSR\_ACCOUNT\_x\_SURNAME}{LOGINUSRACCOUNTxSURNAME}

   \var{LOGINUSR\_ACCOUNT\_x\_SURNAME='Nachname'}

  LOGINUSR\_ACCOUNT\_x\_SURNAME\\
  Nachname des Nutzers für die bessere Verwaltung (Optional, leer lassen erlaubt).
  Dieser Inhalt wird im Log und Admin-Interface angezeigt, so kann der Admin
  besser erkennen, wer gerade online ist.

\config {LOGINUSR\_ACCOUNT\_x\_EMAIL}{LOGINUSR\_ACCOUNT\_x\_EMAIL}{LOGINUSRACCOUNTxEMAIL}

   \var{LOGINUSR\_ACCOUNT\_x\_EMAIL='usr1@home.de'}

  LOGINUSR\_ACCOUNT\_x\_EMAIL\\
  E-Mail des Nutzers für die bessere Verwaltung (Optional, leer lassen erlaubt).
  Dieser Inhalt wird im Log und Admin-Interface angezeigt, so kann der Admin
  besser erkennen, wer gerade online ist.

\config {LOGINUSR\_ACCOUNT\_x\_OVERWRITE}{LOGINUSR\_ACCOUNT\_x\_OVERWRITE}{LOGINUSRACCOUNTxOVERWRITE}

   \var{LOGINUSR\_ACCOUNT\_x\_OVERWRITE='yes'}

  Optional:LOGINUSR\_ACCOUNT\_x\_OVERWRITE\\
  Überschreibe persistente Nutzerdaten beim Router-Neustart.

   Es kann ein Verzeichnis für persistente Daten angegeben
   werden. Dort werden die Daten für die Accounts gespeichert. Damit stehen
   diese Daten unverändert nach einem Reboot zur Verfügung. Mit dieser Option
   können der Benutzer-Account und alle zugehörigen persistenen Daten
   (Statistiken) überschrieben werden.
\end{description}

\subsection {Optionale Parameter OPT\_LOGINUSR}

\begin{description}

\config {LOGINUSR\_ACCOUNT\_x\_TIME}{LOGINUSR\_ACCOUNT\_x\_TIME}{LOGINUSRACCOUNTxTIME}

   \var{LOGINUSR\_ACCOUNT\_x\_TIME='60'}

  Anzahl der Minuten nur für diesen Nutzer.

  Fehlt dieser Parameter, so gilt \jump{C3SURFTIME}{\var{C3SURF\_TIME}}.
  Das Überschreiben macht natürlich nur Sinn, wenn \jump{C3SURFQUOTA}{\var{C3SURF\_QUOTA='yes'}} eingestellt ist.

\config {LOGINUSR\_ACCOUNT\_x\_BLOCKTIME}{LOGINUSR\_ACCOUNT\_x\_BLOCKTIME}{LOGINUSRACCOUNTxBLOCKTIME}

   \var{LOGINUSR\_ACCOUNT\_x\_BLOCKTIME='240'}

  Sperrzeit nur für diesen Nutzer.

  Fehlt dieser Parameter, so gilt \jump{C3SURFBLOCKTIME}{\var{C3SURF\_BLOCKTIME}}. 
  Das Überschreiben macht natürlich nur Sinn, wenn
  \jump{C3SURFQUOTA}{\var{C3SURF\_QUOTA='yes'}} eingestellt ist.

\config {LOGINUSR\_ACCOUNT\_x\_COUNTER}{LOGINUSR\_ACCOUNT\_x\_COUNTER}{LOGINUSRACCOUNTxCOUNTER}

   \var{LOGINUSR\_ACCOUNT\_x\_COUNTER='1'}

  Anzahl der Anmeldungen nur für diesen Nutzer.

  Fehlt dieser Parameter, so gilt \jump{C3SURFCOUNTER}{\var{C3SURF\_COUNTER}}. 
  Das Überschreiben macht natürlich nur Sinn,
  wenn \jump{C3SURFQUOTA}{\var{C3SURF\_QUOTA='yes'}} eingestellt ist.

\config {LOGINUSR\_ACCOUNT\_x\_CURFEW}{LOGINUSR\_ACCOUNT\_x\_CURFEW}{LOGINUSRACCOUNTxCURFEW}

   \var{LOGINUSR\_ACCOUNT\_x\_CURFEW='Liste Sperrstunden'}

   Format: (Liste von Sperrstunden 0-23 durch Leerzeichen getrennt)
  \begin{example}
  \begin{verbatim}
  Beispiel: LOGINUSR_ACCOUNT_x_CURFEW='0 1 2 3 4 5 6 7 21 22 23'
  \end{verbatim}
  \end{example}
  Bedeutung: Eine Anmeldung ist nur zwischen 8:00-20:59 Uhr erlaubt. Die Anmeldung wird immer unterbunden,
  wenn der Nutzer sich innerhalb der Stunde (plus 0-59 Minuten), die in der Liste steht, versucht anzumelden.

  Ist der Benutzer angemeldet und läuft in die Sperrstunde, so wird er ohne Warnung automatisch abgemeldet.
  Das Abmeldeverhalten kann durch den optionalen Parameter \jump{C3SURFCHECKCURFEW}
  {\var{C3SURF\_CHECK\_CURFEW}}='no' unterbunden werden.
   
  Mit dieser Liste kann ein Zugang sehr flexibel eingeschränkt werden. Die Liste kann auch wie gewohnt im
  Webinterface verwaltet werden. Bei der Eingabe findet keine Prüfung der Liste statt. \achtung{Es sind nur die
  Zahlen von 0 bis 23 sinnvoll!}
   
  Zugehöriger OPT\_C3SURF Parameter:\\
  \jump{C3SURFCHECKCURFEW}{\var{C3SURF\_CHECK\_CURFEW}}='no'
    \begin{itemize}
        \item{C3SURF\_CHECK\_CURFEW='no'}: schaltet automatisches Abmelden beim Erreichen der Sperrstunde ab.
        \item{C3SURF\_CHECK\_CURFEW='yes'} (Standard): wer bei Erreichen der Sperrstunde angemeldet ist, wird abgemeldet.
    \end{itemize}
\end{description}
