% Last Update: $Id$
\section{Einleitung}

Mit C3SURF kann man ein offenes, nicht verschlüsseltes Netz/WLAN betreiben. Schon
aus rechtlichen Gründen sollte man jedoch kontrollieren, wer das Netz nutzt.
C3SURF ermöglicht eine formlose Registrierung im Netz. Das Paket basiert
auf ``opt\_onco'' (Copyright (c) 2001-2007 Michael Mattes). Durch die
Nutzung von OPT\_LOGINUSR kann eine ``fast'' echte Anmeldung nachgestellt werden.
C3SURF kann auch Gutscheine generieren und eine rudimentäre (experimentelle)
Steuerungsfunktion zur Unterbindung von übermäßigem Download-Volumen ist
ebenfalls integriert.

Man definiert, welche Hosts oder kompletten Netze von C3SURF verwaltet
werden. Diese sind nach einem Routerstart zunächst gesperrt und
deren http Anfragen werden auf die C3SURF Anmeldeseite geleitet.
Nach der Registrierung eines Benutzers auf der Anmeldeseite wird
eine Nutzung auf Zeit erlaubt. Alles wird protokolliert und über
das Webinterface von C3SURF gesteuert.

\section {Hinweise zur Installation}

\begin{itemize}
\item Wie immer bei Opt-Paketen für den fli4l:
  \begin{itemize}
     \item opt\_c3surf\_$<$versionsinfo$>$.tar.gz in das fli4l Verzeichnis (Buildrechner) entpacken.
     \item c3surf.txt den eigenen Bedürfnissen anpassen.
     \item ggf. in der httpd.txt die Rechte 'c3surf:view,admin' hinzufügen.
     \item Build erstellen.\\
\end{itemize}
\achtung{Wichtig: Der fli4l muss als DNS-Server bei den Clients eingetragen sein und muss befähigt sein Namen
         aufzulösen. Dazu}\\    
          \begin{itemize}

                \item \emph{benötigt er einen ``Forward'' auf den DNS-Server des Netzes oder}
                \item \emph{er ist selbst der DNS-Server und kann ggf. automatisch Verbindugen aufbauen.}
          \end{itemize}
               
               \emph{Sonst gibt es Probleme auf die Anmeldeseite umzuleiten. Diese kann aber immer
               noch manuell durch die Eingabe ihrer URL aufgerufen werden.}

\section {Konfiguration OPT\_C3SURF}
\begin{description}

\config {OPT\_C3SURF}{OPT\_C3SURF}{OPTC3SURF}

  Standard-Einstellung: \var{OPT\_\-C3SURF='no'}

  Paket aktivieren oder deaktivieren.

\config {C3SURF\_LOG\_PATH}{C3SURF\_LOG\_PATH}{C3SURFLOGPATH}

  Standard-Einstellung: \var{C3SURF\_LOG\_PATH='/var/log/c3surf'}

  Definiert das Verzeichnis für log-Dateien von C3SURF. Beim
  Herunterfahren sollten die Logdateien auf ein permanentes
  Medium gesichert oder gleich der Pfad dorthin eingestellt werden,
  wenn man die Dateien behalten will. Der Pfad muss auf dem
  permanenten Medium existieren.

  \wichtig{Die 'c3surf\_mac.blacklist' liegt im persistenten Verzeichnis\\
  \jump{C3SURFPERSISTENTPATH}{\var{C3SURF\_PERSISTENT\_PATH}}. Eine eigene 
  Blacklist muss dorthin kopiert werden. Der Umfang des Protokolls
  wird im Foldenden definiert.}

\config {C3SURF\_DOLOG\_LOGIN}{C3SURF\_DOLOG\_LOGIN}{C3SURFDOLOGLOGIN}

  Standard-Einstellung: \var{C3SURF\_DOLOG\_LOGIN='yes'}

  Zeichne Login/Logout auf: c3surf\_login.log (default: 'yes')

\config {C3SURF\_DOLOG\_INVALID}{C3SURF\_DOLOG\_INVALID}{C3SURFDOLOGINVALID}

  Standard-Einstellung: \var{C3SURF\_DOLOG\_INVALID='yes'}

  Zeichne ungültige Logins auf: c3surf\_invalid.log (default: 'yes'). Wenn
  \jump{OPTLOGINUSR}{\var{OPT\_LOGINUSR}} ='yes' gesetzt ist, kann eine fehlerhafte
  Anmeldung nicht aufgezeichnet werden.

\config {C3SURF\_DOLOG\_PAGE}{C3SURF\_DOLOG\_PAGE}{C3SURFDOLOGPAGE}

  Standard-Einstellung: \var{C3SURF\_DOLOG\_PAGE='no'}

  Zeichne den Aufruf der html-Seite auf: c3surf\_page.log (default: 'no').
  Jeder Zugriff auf die Anmeldeseite wird gelogged. Das Page-Log wächst
  schnell und ist daher nur für ``Neugierige'' empfohlen.

\config {C3SURF\_DOLOG\_HTTPD}{C3SURF\_DOLOG\_HTTPD}{C3SURFDOLOGHTTPD}

  Standard-Einstellung: \var{C3SURF\_DOLOG\_HTTPD='no'}

  Zeichne alle mini\_httpd Anfragen auf: c3surf\_httpd.log (default: 'no').

  \wichtig{Zusätzlich die Mini-httpd Logfunktion starten (bitte nur für Test oder 
  Debuggingzwecke verwenden). Wenn eingeschaltet, empfiehlt es sich regelmäßig
  das Protokoll zu prüfen und zu löschen, das es sehr schnell groß wird.}


  opt\_cpmvrmlog: \altlink{http://extern.fli4l.de/fli4l_opt-db3/search.pl?pid=427}
  kann zum regelmäßigen Sichern benutzt werden. Damit danach wieder korrekt gelogged
  wird, muss der mini\_httpd neu gestartet werden.\\
  Dazu gibt es das Script \var{/usr/local/bin/c3surf\_kill\_httpd.sh}
  \jump{C3SURFLOG}{\var{(Config Beispiel im Anhang).}}

\config {C3SURF\_PERSISTENT\_PATH}{C3SURF\_PERSISTENT\_PATH}{C3SURFPERSISTENTPATH}

  Unbedingt anpassen, Empfehlung: '/var/lib/persistent/c3surf'

  Definiert das Verzeichnes für Dateien, die nach dem Ausschalten
  oder nach einem Reboot erhalten bleiben sollen. Idealerweise liegt dies auf
  einer Festplatte oder CF-Karte ('/var/lib/persistent/c3surf'). Es kann auch ein Verzeichnis
  in der RAM-Disk gewählt werden (z.B. um das Medium zu schonen). Dann sollte das
  Verzeichnis sporadisch auf die Platte kopiert werden (z.B. opt\_cpmvrmlog),
  da die Daten sonst nach einem Reboot, Absturz oder Stromverlust verloren wären.

  \achtung{Was wird hier gespeichert:}
\newpage
  \achtung{MAC-Blackliste:}

  'c3surf\_mac.blacklist',wird bei Bedarf angelegt (siehe Admin
  Interface). Die Sperrung für die Mac-Adresse wurde über eine eigene Datei
  und nicht über den Paketfilter gelöst, da eine größeren Zahl von Einträgen
  Probleme auslöen kann. Nicht vergessen: Geblockte MAC-Adressen halten
  Standardbenutzer vom Netz fern, was im normalen Anwendungsfall ausreicht,
  nicht jedoch die Profis. Diese MAC-Blacklist verhindert nur die
  Anmeldung über C3SURF / loginusr, weil nichts direkt in der Firewall
  gesperrt wird.

  \achtung{Benutzerdaten:}

  $<$userloginname$>$.data (z. B. 'frank.data'), diese Dateien 
  enthalten Daten über die Benutzer, wie Vorname, Name und E-mail-Adresse,
  Statistiken und Quotas. Die Persistenz der Benutzerdaten ermöglicht,
  dass die Daten aus der Konfigurationsdatei nur beim ersten Mal erzeugt werden.
  Das heißt: Ist für den Benutzer ``frank'' eine 'frank.data' beim Systemstart
  vorhanden, so werden die Einstellungen in der Config-Datei ignoriert.
\parskip 12pt

  Mittels \jump{LOGINUSRACCOUNTxOVERWRITE}{\var{LOGINUSR\_ACCOUNT\_x\_OVERWRITE='yes'}}
  kann das überschreiben der entsprechenden Benutzerdaten erzwungen werden.\\
  Mittels \jump{LOGINUSRDELETEPERSISTENTDATA}{\var{LOGINUSR\_DELETE\_PERSISTENT\_DATA='yes'}},
  werden alle ``*.data'' Dateien beim Reboot gelöscht.

\config {C3SURF\_WORKON\_TMP}{C3SURF\_WORKON\_TMP}{C3SURFWORKONTMP}

  Standard-Einstellung: \var{C3SURF\_WORKON\_TMP='no'}

  Wer \jump{C3SURFPERSISTENTPATH}{\var{C3SURF\_PERSISTENT\_PATH}}
  eingestellt hat, kann hier 'yes' eintragen. Persistenten Daten werden
  dann beim Systemstart von der Festplatte auf das Verzeichnis C3SURF\_TMP\_PATH
  kopiert und nur noch von dort gelesen. Auf die Festplatte wird dann von
  C3SURF nur noch zugegriffen, wenn durch den Admin Daten in die
  persistenten Dateien geschrieben werden.

 \wichtig{Persistente Daten sind:}
  \begin{itemize}
  \item \emph{Benutzer-Accounts}
  \item \emph{MAC-Blackliste}
  \item \emph{System Lock Datei (Verhindern jeder Anmeldung)}\\
  \end{itemize}
  
  \emph{Für FLASH-Speicher kann hier 'no' stehen, da ja im normalen Betrieb von
  C3SURF nur gelesen wird. Schreibzugriffe verursacht nur der Admin.}

\config {C3SURF\_QUOTA}{C3SURF\_QUOTA}{C3SURFQUOTA}

  Standard-Einstellung: \var{C3SURF\_QUOTA='no'}

  Soll der Zugang limitiert werden, wird hier 'yes' eingetragen. So wird
  der Zugang für eine IP-Adresse für \jump{C3SURFBLOCKTIME}{\var{C3SURF\_BLOCKTIME}} Minuten nach
  Zeitablauf oder Überschreitung des Anmeldezählers blockiert. Standardwert ist 'yes'.

  \wichtig{Auch die individuellen -TIME, -BLOCKTIME und -COUNTER bei den Accounts zum
   LOGIN\_USR werden durch diese Variable aktiviert ('yes') oder deaktiviert ('no').}

\config {C3SURF\_COUNTER}{C3SURF\_COUNTER}{C3SURFCOUNTER}

  Standard-Einstellung: \var{C3SURF\_COUNTER='0'}

  Gibt die Anzahl der möglichen Unterbrechungen innerhalb der
  Freiminuten an.

  \wichtig{Es kann eine Anzahl von Unterbrechungen
  (Logout/Login) definiert werden. Wird hier z. B. '1' eingetragen, so kann
  man sich innerhalb der Freiminuten einmal abmelden und dann wieder anmelden,
  was 2 Anmeldungen in der Zeit entspricht. Bei der folgenden Anmeldung erhält
  man die von \jump{C3SURFTIME}{\var{C3SURF\_TIME}} noch verbliebene Differenz
  von der Anmeldung davor.}

  \emph{Ist zudem \jump{C3SURFBLOCKTIME}{\var{C3SURF\_BLOCKTIME='0'}} gesetzt wird der
  \jump{C3SURFCOUNTER}{\var{C3SURF\_COUNTER}} erst nach 0:00 Uhr des Folgetages zurückgesetzt.}

  \begin{itemize}
    \item{Mit C3SURF\_COUNTER='0'}
    
    entspricht der Wert dem Parkuhrprinzip (Geld rein, Geld weg, Zeit läuft: also keine
    Unterbrechungen möglich).
  
    \item{Mit C3SURF\_COUNTER= '-1'}
    
    wird diese Funktion abgeschaltet = beliebig viele Unterbrechungen der Freiminuten
    sind möglich.

    \item{Mit C3SURF\_COUNTER='-2'}
    
    gibt es beliebig viele Unterbrechungen (wie '-1'), aber die Blockzeit wird
    bereits mit der ersten Anmeldung heruntergezählt. Im Gegensatz zur '-1', wo die
    Blockzeit erst nach dem Verbrauch der gesamten Zeit eingestellt wird.
    Da hier die Blockzeit gleichzeitig heruntergezählt wird, wird der Benutzer nur geblockt,
    wenn er sein Kontingent zu schnell verbraucht.
  \end{itemize}
  
   Erläuterungen zum Langzeitkontingent (C3SURF\_COUNTER='-2'):\\
   Damit kann man z. B. 10 Stunden Onlinezeit \jump{C3SURFTIME}{\var{(C3SURF\_TIME='600')}} mit
   einer Blockzeit von einer Woche \jump{C3SURFBLOCKTIME}{\var{(C3SURF\_BLOCKTIME='10080'}} : 60sec x 24h x 7Tage)
   kombinieren. Dann können die 10 Stunden innerhalb einer Woche verbraucht werden. Wer sie am ersten Tag am Stück 
   verbraucht, der wartet dann eben den Rest der Woche. Nach Ablauf der Blockzeit werden wieder 10 Stunden
   bereitgestellt.\\
   Kurzfassung: Der Nutzer hat ein Wochenkontingent von 10 Stunden, welches er sich selbst sinnvoll auf
   die Woche verteilen kann. Verbraucht er das Kontingent nicht innerhalb einer Woche, dann wird er nicht mit einem 
   ``Quota-Block'' belegt. Es entsteht dann keine Wartezeit. Verbraucht er es hingegen am ersten Tag, dann
   ist er die restlichen 6 Tage der Woche geblockt.\\
   Gilt auch für \jump{LOGINUSRACCOUNTxCOUNTER}{\var{LOGINUSR\_ACCOUNT\_x\_COUNTER}}.
   
   Empfehlung: \jump{C3SURFSAVEQUOTA}{\var{C3SURF\_SAVE\_QUOTA='yes'}}, dann bleiben die Werte auch nach einem
   normalen Neustart erhalten.\\
   Bei Stromausfall gehen die Werte verloren.
   
   Ist \jump{C3SURFQUOTA}{\var{C3SURF\_QUOTA='yes'}}, so wird nach der Überschreitung des Zählers die
   Sperre entsprechend \jump{C3SURFBLOCKTIME}{\var{C3SURF\_BLOCKTIME}} aktiviert.

\config {C3SURF\_TIME}{C3SURF\_TIME}{C3SURFTIME}

  Standard-Einstellung: \var{C3SURF\_TIME='60'}

  Anzahl der Minuten, die eine Freischaltung gilt. 

  Wobei der Wert '0' ein unlimitiertes Login bedeutet (gilt auch für \\
  LOGINUSR\_ACCOUNT\_x\_TIME).

  \underline{Sonderfall:}
    \begin{itemize}
      \item{C3SURF\_TIME='0'}
      
      Bedeutet eine unlimitierte Onlinezeit. Der Nutzer sollte sich selbst abmelden.
      Das System (C3SURF) meldet ihn nur ab, wenn der Rechner abgeschaltet wird und
      \jump{C3SURFCHECKARP}{\var{C3SURF\_CHECK\_ARP='yes'}} (Standardeinstellung) gewählt wurde.
    \end{itemize}

\config {C3SURF\_BLOCKTIME}{C3SURF\_BLOCKTIME}{C3SURFBLOCKTIME}

  Standard-Einstellung: \var{C3SURF\_BLOCKTIME='240'}

  Anzahl der Minuten, die eine IP geblockt wird, wenn die Freiminuten
  abgelaufen sind oder wenn der Admin dies über das Web-Interface
  veranlasst. So kann ein Rechner für diese Zeit aus dem Netz ferngehalten
  und die Nutzung eingeschränkt werden. Damit bei Zeitablauf die
  Sperrung erfolgt, muss \jump{C3SURFQUOTA}{\var{C3SURF\_QUOTA='yes'}} eingestellt sein.

  \underline{Sonderfälle:}\\
    \begin{itemize}
      \item{C3SURF\_BLOCKTIME='0'}
            
            es erfolgt eine Sperrung der Adresse, bzw. des Nutzers bis 00:00 Uhr
            des Folgetages.
       \item{C3SURF\_BLOCKTIME='-1'}
             
             es erfolgt keine Sperrung.
     \end{itemize}
     
    \wichtig{Die Aufhebung der Sperre ist mit einem mittleren Fehler von einer Minute
             behaftet.}

\config {C3SURF\_SAVE\_QUOTA}{C3SURF\_SAVE\_QUOTA}{C3SURFSAVEQUOTA}

  Standard-Einstellung: \var{C3SURF\_SAVE\_QUOTA='yes'}

  Sichert die Quota-Werte beim Herunterfahren und lädt sie beim
  Start des Routers. Damit werden bei einem normalen Reboot des Routers die
  temporären Dateien der Quota-Verwaltung nach \jump{C3SURFPERSISTENTPATH}
  {\var{C3SURF\_PERSISTENT\_PATH}} geschrieben und beim Neustart von dort wieder
  in das temporäre Verzeichnis kopiert. So bleiben die momentanen
  Verbrauchsdaten der Benutzer erhalten. Ein plötzlicher Ausfall ist damit
  nicht abgedeckt.

  \wichtig{\jump{LOGINUSRDELETEPERSISTENTDATA}{\var{LOGINUSR\_DELETE\_PERSISTENT\_DATA='no'}},
  sollte eingestellt sein, weil diese Option sonst beim Neustart alle Benutzer-Accounts
  und die zugehörigen Quota-Daten löscht.}

\config {C3SURF\_CHECK\_ARP}{C3SURF\_CHECK\_ARP}{C3SURFCHECKARP}

  Standard-Einstellung: \var{C3SURF\_CHECK\_ARP='yes'}

  Prüfe im Countdown Modul, ob die IP eines Rechners aus der ARP Tabelle
  verschwunden ist. So kann ein abgeschalteter Rechner erkannt
  werden, jedoch manchmal mit erheblichem Zeitversatz.

\config {C3SURF\_CONTROL\_HOST\_OR\_NET\_N}{C3SURF\_CONTROL\_HOST\_OR\_NET\_N}{C3SURFCONTROLHOSTORNETN}
 
 \var{C3SURF\_CONTROL\_HOST\_OR\_NET\_N='0'}

  Wert: Ganze Zahl.\\
  Wieviele und welche IP-Bereiche oder Hosts sollen von c3Surf kontrolliert
  werden? Dies betrifft die Weiterleitung in ein anderes Netz (FORWARD Chain).

\config {C3SURF\_CONTROL\_HOST\_OR\_NET\_x}{C3SURF\_CONTROL\_HOST\_OR\_NET\_x}{C3SURFCONTROLHOSTORNETx}\ \\ 
 \var{C3SURF\_CONTROL\_HOST\_OR\_NET\_x='Netzwerk OR Host OR IP-Adresse'}

  Kontrolliert alle Clients.

  \wichtig{Hier kann zur Vereinfachung ein komplettes Netz angegeben werden, z.B.
           das WLAN. Dann müssen alle WLAN-Nutzer die Anmeldeseite benutzen. Es
           können auch eine Referenz auf einen Host (@Host) oder eine IP-Adresse
           angegeben werden. Wer oder was hier eingetragen ist, wird auf die
           Anmeldeseite umgeleitet und es gelten die weiter unten zu definierenden
           Sperrregeln.}

\underline{Beispiel:}

\begin{example}
\begin{verbatim}
C3SURF_CONTROL_HOST_OR_NET_1='IP_NET_3'       # Das Netz angeben IP/MASK
C3SURF_CONTROL_HOST_OR_NET_2='@T8200'         # oder den Host @HOST
C3SURF_CONTROL_HOST_OR_NET_3='192.168.13.11'  # oder eine IP-Adresse
\end{verbatim}
\end{example}

Das nächsten Beispiel ist vom Prinzip her gleich mit dem oben bereits
dargestellten (IP\_NET\_3). Wenn in der "base.txt" die IP-Adresse so gesetzt
wurde.

\begin{example}
\begin{verbatim}
C3SURF_CONTROL_HOST_OR_NET_1='192.168.0.1/24' # kontrolliert alle Clients
\end{verbatim}
\end{example}

Soll ein Rechner ausgenommen sein, kann man entweder alle IP-Adressen
einzeln in die C3SURF.txt aufnehmen (also eine Liste aller 256 Adressen
erstellen, wobei man die eine weglässt), oder man verwendet die CIDR Notation
(wie oben), dann sind es IP-Gruppen und damit weniger Schreibarbeit
(8 Zeilen, statt 255).

\underline{Das sieht dann so aus:}

\begin{example}
\begin{verbatim}
C3SURF_CONTROL_HOST_OR_NET_N='8'                # Die Anzahl der Hosts
						# oder Netze
C3SURF_CONTROL_HOST_OR_NET_1='192.168.0.0/31'   # 0-1
C3SURF_CONTROL_HOST_OR_NET_2='192.168.0.3'      # only 3 not 2
C3SURF_CONTROL_HOST_OR_NET_3='192.168.0.4/30'   # 4-7
C3SURF_CONTROL_HOST_OR_NET_4='192.168.0.8/29'   # 8-15
C3SURF_CONTROL_HOST_OR_NET_5='192.168.0.16/28'  # 16-31
C3SURF_CONTROL_HOST_OR_NET_6='192.168.0.32/27'  # 32-63
C3SURF_CONTROL_HOST_OR_NET_7='192.168.0.64/26'  # 64-127
C3SURF_CONTROL_HOST_OR_NET_8='192.168.0.128/25' # 128-255
\end{verbatim}
\end{example}

Der Rechner mit der IP '192.168.0.2' kann ohne Anmeldung alles, was
in der firewall des fli4l erlaubt ist.

\config {C3SURF\_CONTROL\_PORT\_N}{C3SURF\_CONTROL\_PORT\_N}{C3SURFCONTROLPORTN}

   \var{C3SURF\_CONTROL\_PORT\_N='0'}

  Wert: Ganze Zahl.\\
  Wieviele TCP-Ports des Routers sollen gesteuert werden? 

  Wieviele und welche explizit benannten Ports sollen von c3Surf kontrolliert
  werden? Betroffen sind die IP-Bereiche und die Hosts von oben\\
  \jump{C3SURFCONTROLHOSTORNETN}{\var{\dq C3SURF\_CONTROL\_HOST\_OR\_NET\_N\dq\ }}. c3Surf
  steuert diese Ports und gibt diese nach einer erfolgreichen Anmeldung frei,
  so dass die über diese Ports existierenden Services des Routers genutzt werden können
  (betrifft die INPUT-Chain).

\config {C3SURF\_CONTROL\_PORT\_x}{C3SURF\_CONTROL\_PORT\_x}{C3SURFCONTROLPORTx}

   \var{C3SURF\_CONTROL\_PORT\_x='port\_nr'}

  Angabe der Portnummer und der Zugriff auf die dahinter stehenden Dienste
  des Routers (fli4l) sind bis zur Anmeldung gesperrt. Nach
  erfolgter Anmeldung wird der Dienst dann für die Zeit der Freischaltung
  zur Verfügung gestellt.

\underline{Beispiele:}
\begin{example}
\begin{verbatim}
C3SURF_CONTROL_PORT_1='515' # z.B. lpdsrv (Drucker benutzbar, nach Anmeldung)
C3SURF_CONTROL_PORT_2='21'  # z.B. ftp - (wohl gemerkt ftp auf dem router!)
\end{verbatim}
\end{example}

\begin{example}
\begin{verbatim}
Weitere mögliche Portadressen:
  21=ftp
  22=ssh
  5000=imonc
  5001=telmod
  8118=privoxy
  9050=tor
  3128=squid
  20000=mtgcapri
  80=http(Admin)
  515=lpdsrv
\end{verbatim}
\end{example}

        Entscheidend ist die eigene Konfiguration. Es gelten für alle Ports,
        die nicht angegeben sind immer die Regeln aus der 'base.txt'.
        Nach einer Anmeldung gelten immer noch die Regeln aus der
        'base.txt'. c3Surf ist diesen Regeln bis zur Anmeldung durch den Benutzer
        nur vorgeschaltet. Es werden also nach der Anmeldung immer noch alle Regeln
        beachtet. So kann man z. B. in der 'base.txt' den Zugriff von WLAN auf das
        kabelgebundene Netz verbieten. Dieses Verbot gilt dann auch für
        die über c3Surf im WLAN angemeldeten Benutzer.

\config {C3SURF\_BLOCK\_PORT\_N}{C3SURF\_BLOCK\_PORT\_N}{C3SURFBLOCKPORTN}

   \var{C3SURF\_BLOCK\_PORT\_N='0'}

  Wert: Ganze Zahl.\\
  Wieviele TCP-Ports des Routers sollen geblockt werden? 

   \underline{Hinweise:}\\
   Permanentes Blocken von Diensten für oben benannte Netze und Hosts\\
   \jump{C3SURFCONTROLHOSTORNETN}{\var{\dq C3SURF\_CONTROL\_HOST\_OR\_NET\_N\dq }}. Wieviele
   und welche explizit benannten Ports sollen von c3Surf permanent geblockt werden?
   Es gibt dann keinen Zugriff auf die dahinter stehenden Dienste des Routers
   (fli4l) für die Hosts und/oder Rechner der gesperrten Netze, auch nach dem Anmelden
   nicht. Dies betrifft die INPUT-Chain. Wer bestimmte Dienste dauerhaft sperren will, sollte
   dies allerdings besser in der 'base.txt' mit den Parametern zur INPUT Chain tun.

   \underline{Warum:}\\
   Weil diese Regeln hier nicht mehr gelten, sobald man den Parameter\\
   \jump{OPTC3SURF}{\var{OPT\_C3SURF='no'}} setzt. Wer also C3SURF deaktiviert, muss zuvor die hier
   definierten Regeln in die 'base.txt' übertragen, wenn ihm die Sperre
   für die oben benannten Hosts oder Netze weiter wichtig sind.


\config {C3SURF\_BLOCK\_PORT\_x}{C3SURF\_BLOCK\_PORT\_x}{C3SURFBLOCKPORTx}

   \var{C3SURF\_BLOCK\_PORT\_x='port\_nr'}

\underline{Beispiele:}
\begin{example}
\begin{verbatim}
C3SURF_BLOCK_PORT_1='5000'           # z.B. imonc
C3SURF_BLOCK_PORT_2='5001'           # z.B. telmond
C3SURF_BLOCK_PORT_3='20000'          # z.B. mtgcapri (OPT_MTGCAPRI)
C3SURF_BLOCK_PORT_4='22'             # z.B. ssh
C3SURF_BLOCK_PORT_5='8118'           # z.B. privoxy (PROXY)
C3SURF_BLOCK_PORT_6='9050'           # z.B. tor (PROXY)
C3SURF_BLOCK_PORT_7='80'             # z.B. httpd Admin interface (HTTPD)
C3SURF_BLOCK_PORT_8='7437'           # z.B. caiviar (OPT_CAIVIAR)
\end{verbatim}
\end{example}

\config {C3SURF\_HTTPD\_PORT}{C3SURF\_HTTPD\_PORT}{C3SURFHTTPDPORT}

  Standard-Einstellung: \var{C3SURF\_HTTPD\_PORT='8080'}

  Auf welchem Port und welcher IP-Adresse soll der mini\_httpd für
  die Benutzeranmeldung lauschen? http Anfragen von Rechnern werden auf 
  diese Adresse und diesen Port umgeleitet. Port 8080 ist hier Default.

  \achtung{Folgendes ist bei der Wahl der Portnummer zu beachten:}
  \begin{itemize}
  \item Es sollte nicht der Port aus dem httpd-Paket sein (normal ist das 80).
  \item Der httpd für den Web-Admin des fli4l bindet sich im Standard an alle lokalen IP's.
  \item auch keine Portnummer benutzen, die bereits von einem anderen Dienst genutzt wird.\\
  \end{itemize}
  Sollte hier versehentlich ein bereits verwendeten Port definiert sein,
  versucht der fli4l diesen httpd immer wieder zu starten. Das schlägt fehl,
  weil der Port schon vom Admin-Interface oder einem anderen Dienst belegt ist.
  Das kann nur auf der Konsole oder in einem eingeschalteten Log gesehen werden.
  Man merkt es daran, dass C3SURF nicht funktionieren wird, fli4l hohe CPU-Last
  erzeugt und langsam zu laufen scheint.

\config {C3SURF\_HTTPD\_LISTENIP}{C3SURF\_HTTPD\_LISTENIP}{C3SURFHTTPDLISTENIP}

  Standard-Einstellung: \var{C3SURF\_HTTPD\_LISTENIP='Host OR IP\-Adresse'}

  Gibt die lokale IP an, an die sich das Interface für die Anmeldung binden soll,
  entweder IP-Adresse oder @hostname. Hierhin werden http Anfragen der Clients bei
  Bedarf (also wenn sie nicht angemeldet sind) umgeleitet. So
  kommen die Anwender dann schnell auf die Anmeldeseite.

\underline{Beispiele:}
\begin{example}
\begin{verbatim}
C3SURF_HTTPD_LISTENIP='@wifi-router'    # Angabe eines Hostnamens
C3SURF_HTTPD_LISTENIP='192.168.11.3'    # Angabe einer IP-Adresse
C3SURF_HTTPD_LISTENIP='IP_NET_1_IPADDR' # Angabe einer IP-Adress-Variablen
\end{verbatim}
\end{example}

Der http-Dienst für C3SURF wird immer an genau eine IP-Adresse gebunden.

\end{description}

\subsection {Optionale Parameter OPT\_C3SURF}

\begin{description}

\config {C3SURF\_CONTROL\_SQUID}{C3SURF\_CONTROL\_SQUID}{C3SURFCONTROLSQUID}

  Standard-Einstellung: \var{C3SURF\_CONTROL\_SQUID='no'}

  Mit dem Einfügen der Variablen C3SURF\_CONTROL\_SQUID='yes' wird die
  Kontrolle über squid erzwungen. Damit wird die C3SURF Portumleitung an den
  Anfang gesetzt, was auch Auswirkungen auf andere Pakete hat (z. B. openvpn).

  Die Empfehlung ist 'no', wer z.B. squid verwendet sollte prüfen, ob nicht
  ungewollt noch andere Funktionen dadurch beeinflusst werden.

\config {C3SURF\_SLOPPY\_MAC}{C3SURF\_SLOPPY\_MAC}{C3SURFSLOPPYMAC}
  
  Standard-Einstellung: \var{C3SURF\_SLOPPY\_MAC='no'}
  
   \begin{itemize}
      \item{C3SURF\_SLOPPY\_MAC='no'}
      
            (Standard) - wenn dieser Parameter nicht angegeben wurde, 
            lasse Login nur mit MAC-Adressermittlung aus der ARP-Tabelle zu.
  
       \item{C3SURF\_SLOPPY\_MAC='yes'}
       
            C3SURF akzeptiert auch MAC-Adressen die fehlen und nicht über die ARP-Tabelle ermittelbar sind.
            Wer hier 'yes' wählt sollte \jump{C3SURFCHECKARP}{\var{C3SURF\_CHECK\_ARP='no'}} setzen. Sonst
            erfolgt die automatische Abmeldung (im Mittel nach einer Minute), weil der ``countdown''-Prozess
            wegen fehlendem Eintrag in der ARP-Tabelle zur Abmeldung aufgefordert wird.
   \end{itemize}
       
\config {C3SURF\_CHECK\_CURFEW}{C3SURF\_CHECK\_CURFEW}{C3SURFCHECKCURFEW}

  Standard-Einstellung: \var{C3SURF\_CHECK\_CURFEW='yes'}
  
  Schalte automatisches Abmelden beim Erreichen der Sperrstunde an (\var{'yes'}) oder ab (\var{'no'}).

\config {C3SURF\_PORTAL\_DEFAULT\_LANG}{C3SURF\_PORTAL\_DEFAULT\_LANG}{C3SURFPORTALDEFAULTLANG}

  Standard-Einstellung: \var{C3SURF\_PORTAL\_DEFAULT\_LANG='de'}
  
  Wertebereich: ein zweistelliger Ländercode (z. B. 'de', 'fr', 'en').
  
  Legt die Standard-Sprache für die Anmeldeseite fest. Wird diese Variable weggelassen, wird 'de' angenommen.
  
  Es sollte unter \textasciitilde/opt/files/srv/www/c3surf/lang/ eine Datei namens
  c3surf.$<$ländercode$>$ existieren. Derzeit werden 'de', 'fr', 'en' und
  'it' mitgeliefert. Wer daraus für eine andere Sprache eine weitere Datei erstellt,
  möge diese an das fli4l-Team schicken.

\config {C3SURF\_PORTAL\_LANGUAGES}{C3SURF\_PORTAL\_LANGUAGES}{C3SURFPORTALLANGUAGES}\ \\
  
  Standard-Einstellung: \var{C3SURF\_PORTAL\_DEFAULT\_LANG='de fr en it'}
  
  Wertebereich: eine mit Leerzeichen getrennte Liste von immer zwei Buchstaben.
  
  Legt fest, welche Sprachdateien für die Anmeldeseite auf das System übertragen werden sollen.
  Sollten hier zweistellige Kürzel stehen, zu denen keine Sprachdatei existiert, so wird bei der
  Generierung der Router-Images eine Warnung ausgegeben, dass zu dem Kürzel in der Liste keine Datei
  gefunden wurde und somit auch nichts kopiert wurde. Der Build-Prozess wird nicht abgebrochen.
  
\end{description}
