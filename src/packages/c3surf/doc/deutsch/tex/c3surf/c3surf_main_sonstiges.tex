% Last Update: $Id$
\subsection{Warnung}
\wichtig{Ohne \jump{OPTLOGINUSR}{\var{OPT\_LOGINUSR='yes'}} ist es jeder Person
möglich, die für ihren Rechner eine IP-Adresse vom Router zugewiesen
bekommen hat (z. B. aus einem offenen WLAN), einen freien Zugriff auf das
Internet und die nicht geblockten Dienste des Routers zu
bekommen. c3Surf unterstützt beim Blocken der Dienste, ist aber kein
Ersatz für eine ordentliche Konfiguration der Firewall in base.txt.}

\subsection{Empfehlung}
\wichtig{Der Router sollte für alle Fälle über eine Recovery-Version verfügen. 
Mit einer unglücklichen Konfiguration kann man sich komplett aussperren.}

\subsection{Fehler}
Fehlerbeschreibungen können mit Config-Info in einer der fli4l-Newsgruppen
\altlink{http://www.fli4l.de/hilfe/newsgruppen-forum/} gepostet werden.

\subsection{Lizenz}
Diese Software wird unter den Bedingungen der GNU General Public License
in Version 2 oder folgende veröffentlicht. Damit ist diese Software frei
für Benutzer, Entwickler und Firmen. Es ist guter Stil, wenn Urheber
in einer weiteren Verwertung oder Veröffentlichung genannt werden.
Das gilt besonders für freie Werke.

\subsection{Literatur}
Wer gerne sein Netz für andere zur Verfügung stellt, der sollte sich auch
einmal mit der rechtlichen Situation auseinander setzen.

Es gibt eine unter CC stehende Arbeit dazu:\\
Rechtsfragen offener Netze:\\
\altlink{http://digbib.ubka.uni-karlsruhe.de/volltexte/1000007749}\\
Autor: Mantz, Reto\\
Reihe: Schriften des Zentrums für Angewandte Rechtswissenschaft / ZAR\\
Zentrum für Angewandte Rechtswissenschaft, Universität Karlsruhe (TH)\\
Band: 8\\
Verlag: Universitätsverlag Karlsruhe\\
ISBN: 978-3-86644-222-1\\
Erschienen: 10.04.2008
