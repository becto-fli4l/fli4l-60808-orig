% Synchronized to r30494
OPT\_C3SURF\_VOUCHER permet d'avoir un accès anonyme à Internet. Des tickets sont créés pour chaque compte,
ils pourront être mise en place dans diverses catégories. L'opt peut ensuite les gérés automatiquement
ou manuellement via l'interface Web admin.

\section {Configuration de OPT\_VOUCHER}

\begin{description}
\config {OPT\_C3SURF\_VOUCHER}{OPT\_C3SURF\_VOUCHER}{OPTC3SURFVOUCHER}

   \var{OPT\_C3SURF\_VOUCHER='no'}

  Avec cette variable vous activez le système voucher C3SURF\_VOUCHER ('yes') par défaut la valeur est
  sur 'no'. Le système de ticket est utilisé par C3SURF. Des tickets anonymes à usage unique peuvent
  être utilisés à l'enregistrement des comptes.
  La condition est que la variable \jump{OPTLOGINUSR}{\var{OPT\_LOGINUSR='yes'}} soit activée.

  La création et la suppression des tickets se fait en deux tâches avec cron nocturne, ils peuvent également
  être lancé manuellement et à tout moment avec (interface admin). Ci-dessous vous découvrirez comment gérer
  le fonctionnement de voucher.

  Tous les tickets nouvellement générées sont attachés à une liste d'impression. Le mot de passe correspondant
  au ticket qui est stocké dans la liste d'impression est en texte brut. Vous pouvez télécharger, imprimer ou
  supprimer cette liste à tout moment. Après la suppression de la liste, le mot de passe ne peut plus être
  récupéré. Normalement, vous devez en premier imprimer la liste et ensuite, la supprimé. Il devrait exister
  seulement une copie imprimée des tickets. la fonction d'impression rudimentaire est implémentée dans le html,
  la page RSS n'est pas prise en compte. Vous pouvez détruire les tickets qui sont victime de pagination,
  (de toute façon, ils deviennent invalide en raison de l'expiration). Les listes qui n'ont pas été imprimés
  mais téléchargés peuvent avoir une mise en page avec un traitement de texte, la pagination sera alors prise
  en compte.

\config {C3SURF\_VOUCHER\_N}{C3SURF\_VOUCHER\_N}{C3SURFVOUCHERN}

   \var{C3SURF\_VOUCHER\_N='n'}

    Plage de valeur~: 0 et nombre entier naturel

  Dans cette variable vous indiquez le nombre de catégories de ticket différents qui seront
  produits. Le critère le plus important pour les tickets est le temps que vous pouvez rester
  connecté. De plus il faut faire attention au nombre de ticket et leur validité en jours. Voir
  également les variables suivantes.

\config {C3SURF\_VOUCHER\_x\_TIME}{C3SURF\_VOUCHER\_x\_TIME}{C3SURFVOUCHERxTIME}

   \var{C3SURF\_VOUCHER\_x\_TIME='30'}

    Plage de valeur~: nombre entier naturel

  Dans cette variable vous indiquez un temps en minute pour la catégorie de ticket ('n' voir ci-dessus).

  Exemple~: le(s) 'n' catégorie de ticket sont conçus pour 30 minutes.

\config{C3SURF\_VOUCHER\_x\_COUNT}{C3SURF\_VOUCHER\_x\_COUNT}{C3SURFVOUCHERxCOUNT}

   \var{C3SURF\_VOUCHER\_x\_COUNT='3'}

   Plage de valeur~: nombre entier naturel

  Dans cette variable vous indiquez le nombre de ticket qui doit être généré dans une catégorie.

  Exemple~: on produit un total de 3 tickets pour une catégorie de temps.

\config{C3SURF\_VOUCHER\_x\_DAYS}{C3SURF\_VOUCHER\_x\_DAYS}{C3SURFVOUCHERxDAYS}

   \var{C3SURF\_VOUCHER\_x\_DAYS='90'}

   Plage de valeur~: 0 et nombre entier naturel

  Dans cette variable vous indiquez le nombre de jours, c'est le temps de validité du ticket généré.
  C'est cette Date d'expiration que C3SURF utilisera pour le ticket généré. La suppression pourra
  alors être effectuée manuellement ou via cron (c'est son job). Le ticket disparaît plus tôt, à savoir quand il
  est utilisée pour la première fois.

  Exemple~: Ces tickets sont valables pendant 90 jours après leur création.

  \wichtig{La valeur '0', signifie que cette catégorie de ticket n'ont pas de date d'expiration.
		Ils perdent leur validité en les utilisant ou si le temps a été complètement consommé
		cette valeur est également (affecté par C3SURF\_VOUCHER\_LIVES\_N). Cependant,
		ils peuvent être supprimés à tout moment dans l'interface Web admin}

\end{description}

\subsection {Paramètre Optionnel de OPT\_VOUCHER}

\begin{description}

\config{C3SURF\_VOUCHER\_x\_LIVES}{C3SURF\_VOUCHER\_x\_LIVES}{C3SURFVOUCHERxLIVES}

   \var{C3SURF\_VOUCHER\_x\_LIVES='n'}

   (n) plage de valeur~: -1, 0 et nombre entier naturel

  Dans cette variable vous indiquez le nombre d'heure, que le ticket peut vivre même après
  la première application.

\achtung{Cas particuliers~:}

\begin{itemize}
   \item{C3SURF\_VOUCHER\_x\_LIVES='-1'}

         Valable jusqu'à la date d'expiration initialement généré avec C3SURF\_VOUCHER\_DAYS.
   \item{C3SURF\_VOUCHER\_x\_LIVES='0'}

         (Par défaut), cela signifie que le ticket ne sera plus valide après la première application.
    \item{C3SURF\_VOUCHER\_x\_LIVES='nombre entier naturel'}

          Nombre d'heures que vit le ticket après la première application - calculer une nouvelle date
		  d'expiration si nécessaire.
\end{itemize}
\parskip 12pt

  Ces tickets ne sont plus valables à la première application, mais sont valables pour 'n' heures de plus.
  Une fois que le ticket est utilisé, un compte LOGINUSR limitée dans le temps est généré, alors,
  la date d'expiration du ticket peut être recalculée. Ce compte / ticket peut être connecter et déconnecter
  pendant un certain nombre de fois. Le système de quotas de LOGIN\_USR est utilisé pour ce compte. Si
  la durée totale ou la date d'expiration avec (C3SURF\_VOUCHER\_DAYS\_n) est atteint, C3SURF supprimera
  automatiquement ce compte.

\config{C3SURF\_VOUCHER\_DEL\_CRON}{C3SURF\_VOUCHER\_DEL\_CRON}{C3SURFVOUCHERDELCRON}

   \var{C3SURF\_VOUCHER\_DEL\_CRON='0 4 * * *'}

   Plage de valeur~: 'cron-Syntax' ou 'never'

  La valeur ci-dessus est la valeur par défaut, si la variable est manquante dans le fichier
  config 'c3surf.txt'.
  Par défaut~: supprime tous les tickets expirés tous les matins à 4 heures.

  La syntaxe cron doit être respectée, elle ne sera pas vérifiée. La valeur 'never' peut être utilisé
  en complément. Le travail n'est pas planifié par le système. Dans l'interface Web admin tous
  les tickets périmés peuvent être supprimés manuellement à tout moment.

\config{C3SURF\_VOUCHER\_GEN\_CRON}{C3SURF\_VOUCHER\_GEN\_CRON}{C3SURFVOUCHERGENCRON}

   \var{C3SURF\_VOUCHER\_GEN\_CRON='15 4 * * *'}

   Plage de valeur~: 'cron-Syntax' ou 'never'

  La valeur ci-dessus est la valeur par défaut, si la variable est manquante dans le fichier
  config 'c3surf.txt'.
  Par défaut~: génére les nouveaux tickets tous les jours à 4h15 du matin, tout au moin si C3SURF\_VOUCHER\_COUNT
  est configuré.

  La syntaxe cron doit être respectée, elle ne sera pas vérifiée. La valeur 'never' peut être utilisé
  en complément. Le travail n'est pas planifié par le système. Dans l'interface Web admin tous
  les tickets périmés peuvent être supprimés manuellement à tout moment par rapport à la quantité
  définie dans \jump{C3SURFVOUCHERxCOUNT}{\var{C3SURF\_VOUCHER\_x\_COUNT}}.

  Tous les tickets nouvellement générées sont attachés à une liste d'impression. Seulement, le mot
  de passe correspondant au ticket est enregistré en texte brut dans la liste d'impression. Chaque
  ticket doit être imprimé qu'une seule fois. La liste devrait être supprimé immédiatement après
  l'impression ou téléchargement.

\config{C3SURF\_VOUCHER\_PRTUPDATE}{C3SURF\_VOUCHER\_PRTUPDATE}{C3SURFVOUCHERPRTUPDATE}

   Paramètre par défaut~: \var{C3SURF\_VOUCHER\_PRTUPDATE='no'}

   Plage de valeur~: 'yes' ou 'no'

   Avec cette variable vous paramétrez la mise à jour du fichier d'impression. Recommandation~:
   vous pouvez indiquer 'no' si quelques tickets sont enregistrés dans le système, le fichier
   d'impression ne doit pas être supprimé après l'impression ou le téléchargement, vous pouvez
   spécifier 'yes' pour la mise à jour du fichier d'impression, quand les tickets sont utilisés.
   Dans le cas ou vous avez indiqué 'yes' les tickets utilisés seront également supprimées du
   fichier d'impression. Ceci nécessite des ressources sur le routeur.

\config{C3SURF\_VOUCHER\_USRLEN}{C3SURF\_VOUCHER\_USRLEN}{C3SURFVOUCHERUSRLEN}

   Paramètre par défaut~: \var{C3SURF\_VOUCHER\_USRLEN='12'}

   Plage de valeur~: '1-16'

   Avec cette variable vous paramétrez la longueur du texte pour le ticket du compte, Dès
   le 8 ème caractères vous pouvez placer le signe '-' comme séparateurs, vous devez toujours
   avoir quatre caractères groupés. La valeur maximale est 16 caractères.

\config{C3SURF\_VOUCHER\_USRCAP}{C3SURF\_VOUCHER\_USRCAP}{C3SURFVOUCHERUSRCAP}

   Paramètre par défaut~: \var{C3SURF\_VOUCHER\_USRCAP='random'}

   \begin{tabular}{rlrl}
    -&'yes'&:&en majuscules \\
    -&'no'&:&tout en minuscules \\
    -&'random'&:&changement aléatoire des lettres majuscules et minuscules (recommandé) \\
   \end{tabular}

    Avec cette variable vous déterminez si vous utilisez les caractères majuscules, minuscules
	ou aléatoires pour écrire le nom des utilisateurs. La valeur \var{'random'} (est recommandé)
	c'est le mode aléatoire.

\config{C3SURF\_VOUCHER\_PWDLEN}{C3SURF\_VOUCHER\_PWDLEN}{C3SURFVOUCHERPWDLEN}

   Paramètre par défaut~: \var{C3SURF\_VOUCHER\_PWDLEN='6'}

   Plage de valeur~: 1-12

   Avec cette variable vous indiquez le nombre de caractères pour le mot de passe du ticket.

\config{C3SURF\_VOUCHER\_PWDMOD}{C3SURF\_VOUCHER\_PWDMOD}{C3SURFVOUCHERPWDMOD}

   Paramètre par défaut~: \var{C3SURF\_VOUCHER\_PWDMOD='3'}

   Plage de valeur~: 1-5

   Avec cette variable vous indiquez le Modulo pour les extensions aléatoires du mot de passe.
   Max: 5 (avec les valeurs 0, 1, 2, 3, 4), Min 1 (avec la valeur 0). Le mot de passe sera généré
   de façon aléatoire. Le résulta par défaut sera entre 6 et 8 caractaires c'est la longueur
   du mot de passe. La longueur maximum du mot de passe est entre 12 et 16 caractères, avec
   l'utilisation de majuscules et de minuscules cela donnera un mot de passe assez sûr.

\config{C3SURF\_VOUCHER\_PWDCAP}{C3SURF\_VOUCHER\_PWDCAP}{C3SURFVOUCHERPWDCAP}

   Paramètre par défaut~: \var{C3SURF\_VOUCHER\_PWDCAP='random'}

   \begin{tabular}{rlrl}
    -&'yes'&:&en majuscules \\
    -&'no'&:&tout en minuscules \\
    -&'random'&:&changement aléatoire des lettres majuscules et minuscules (recommandé) \\
   \end{tabular}

    Avec cette variable vous déterminez si vous utilisez les caractères majuscules, minuscules
	ou aléatoires pour écrire le mot de passe. La valeur \var{'random'} (est recommandé)
	c'est le mode aléatoire.

\end{description}
