% Synchronized to r30494
un enregistrement de l'utilisateur est met à la disposition. Il ne sera plus
possible pour tout les utilisateurs d'utiliser librement les services Internet
du routeur. La commutation des modes de fonctionnement est techniquement possible
mais pas actuellement mis en œuvre.

\wichtig{Ce n'est pas vrai un nom d'utilisateur, le logiciel substitue
l'adresse de l'ordinateur pour chaque utilisateur. Lorsque le quota est
atteint, l'utilisateur sera bloqué, pas l'ordinateur (adresse IP).}

\section{Configuration de OPT\_LOGINUSER pour C3SURF}

\begin{description}
\config {OPT\_LOGINUSR}{OPT\_LOGINUSR}{OPTLOGINUSR}

  Paramètre par défaut~: \var{OPT\_LOGINUSR}='no'

  OPT\_LOGINUSR='yes'~: Utilisez un login pour se connecter (recommandé)

  LOGINUSR fournit une véritable connexion par (User/Password). La gestion
  des comptes se fait dans le fichier de configuration, le mot de passe
  sera crypté avant le transfére.

\config {LOGINUSR\_DELETE\_PERSISTENT\_DATA}{LOGINUSR\_DELETE\_PERSISTENT\_DATA}{LOGINUSRDELETEPERSISTENTDATA}\ \\
  Paramètre par défaut~: \var{LOGINUSR\_DELETE\_PERSISTENT\_DATA='no'}

  LOGINUSR\_DELETE\_PERSISTENT\_DATA\\
  Les données de l'utilisateur sur un disque dur survivront au redémarre.
  La valeur par défaut est 'no' conserve les données de compte.

  \achtung{En entrant 'yes' tous les comptes utilisateur seront supprimés
  à chaque redémarrage. Ensuite, ils seront recréés comme décrit ci-dessous.
   Par la suite, un nouveau système de comptes, tel que défini ci-dessous.}

  Il est recommandé de garder par défaut 'no'. Toutes les données des comptes
  seront préservées, à savoir~:

\begin{itemize}
\item Les comptes utilisateurs
\item Le fichier Quotas, si \jump{C3SURFSAVEQUOTA}{\var{C3SURF\_SAVE\_QUOTA='yes'}} est sélectionné 
(voir ci-dessus) (pour un compte séparé voir~: \jump{LOGINUSRACCOUNTxOVERWRITE}{\var{LOGINUSR\_ACCOUNT\_x\_OVERWRITE}})
\end{itemize}

\config {LOGINUSR\_ACCOUNT\_N}{LOGINUSR\_ACCOUNT\_N}{LOGINUSRACCOUNTN}

 \var{LOGINUSR\_ACCOUNT\_N='0'}

  LOGINUSR\_ACCOUNT\_N\\
  Valeur pour le nombre de comptes~: nombre entier\\
  Vous indiquez ici le nombre de comptes utilisateurs.

\config {LOGINUSR\_ACCOUNT\_x\_USER}{LOGINUSR\_ACCOUNT\_x\_USER}{LOGINUSRACCOUNTxUSER}

   \var{LOGINUSR\_ACCOUNT\_x\_USER='user1'}

  LOGINUSR\_ACCOUNT\_x\_USER\\
  Vous indiquez ici le nom d'utilisateur pour la connexion (obligatoire).

\config {LOGINUSR\_ACCOUNT\_x\_PWD}{LOGINUSR\_ACCOUNT\_x\_PWD}{LOGINUSRACCOUNTxPWD}

   \var{LOGINUSR\_ACCOUNT\_x\_PWD='user1\_secret'}

  LOGINUSR\_ACCOUNT\_x\_PWD\\
  Vous indiquez ici le mot de passe de l'utilisateur (obligatoire).

\config {LOGINUSR\_ACCOUNT\_x\_FORENAME}{LOGINUSR\_ACCOUNT\_x\_FORENAME}{LOGINUSRACCOUNTxFORENAME}

   \var{LOGINUSR\_ACCOUNT\_x\_FORENAME='Vorname'}

  LOGINUSR\_ACCOUNT\_x\_FORENAME\\
  Vous indiquez ici le prénom de l'utilisateur pour une meilleure gestion
  (facultatif, peut être vide). Le contenu est affiché dans le journal et
  dans l'interface Web admin pour aider l'administrateur à mieux reconnaître
  les utilisateurs qui sont en ligne pour le moment.

\config {LOGINUSR\_ACCOUNT\_x\_SURNAME}{LOGINUSR\_ACCOUNT\_x\_SURNAME}{LOGINUSRACCOUNTxSURNAME}

   \var{LOGINUSR\_ACCOUNT\_x\_SURNAME='Nachname'}

  LOGINUSR\_ACCOUNT\_x\_SURNAME\\
  Vous indiquez ici le nom de l'utilisateur pour une meilleure gestion
  (facultatif, peut être vide). Le contenu est affiché dans le journal et
  dans l'interface Web admin pour aider l'administrateur à mieux reconnaître
  les utilisateurs qui sont en ligne pour le moment.

\config {LOGINUSR\_ACCOUNT\_x\_EMAIL}{LOGINUSR\_ACCOUNT\_x\_EMAIL}{LOGINUSRACCOUNTxEMAIL}

   \var{LOGINUSR\_ACCOUNT\_x\_EMAIL='usr1@home.de'}

  LOGINUSR\_ACCOUNT\_x\_EMAIL\\
  Vous indiquez ici l'E-Mail de l'utilisateur pour une meilleure gestion
  (facultatif, peut être vide). Le contenu est affiché dans le journal et
  dans l'interface Web admin pour aider l'administrateur à mieux reconnaître
  les utilisateurs qui sont en ligne pour le moment.

\config {LOGINUSR\_ACCOUNT\_x\_OVERWRITE}{LOGINUSR\_ACCOUNT\_x\_OVERWRITE}{LOGINUSRACCOUNTxOVERWRITE}

   \var{LOGINUSR\_ACCOUNT\_x\_OVERWRITE='yes'}

  Optionnel~: LOGINUSR\_ACCOUNT\_x\_OVERWRITE\\
  Vous pouvez écraser les données persistantes de l'utilisateur au redémarrage du système.

   Vous pouvez spécifié ici un pépertoire pour stocker les données persistante,
   vous pourrez conserver les données des compte utilisateurs. De cette façon,
   les données sont conservées après un redémarrage. Avec cette option les
   comptes utilisateur et toutes les autres données pour (des statistiques) peuvent
   être copiées et remplacées.
\end{description}

\subsection {Paramètre optionnel de OPT\_LOGINUSR}

\begin{description}

\config {LOGINUSR\_ACCOUNT\_x\_TIME}{LOGINUSR\_ACCOUNT\_x\_TIME}{LOGINUSRACCOUNTxTIME}

   \var{LOGINUSR\_ACCOUNT\_x\_TIME='60'}

  Vous indiquez ici le nombre de minutes uniquement pour cette utilisateur.

  Si vous avez oublier de paramétrer la variable \jump{C3SURFTIME}{\var{C3SURF\_TIME}}.
  L'écrasement n'a de sens que si la variable \jump{C3SURFQUOTA}{\var{C3SURF\_QUOTA='yes'}}
  a été défini.

\config {LOGINUSR\_ACCOUNT\_x\_BLOCKTIME}{LOGINUSR\_ACCOUNT\_x\_BLOCKTIME}{LOGINUSRACCOUNTxBLOCKTIME}

   \var{LOGINUSR\_ACCOUNT\_x\_BLOCKTIME='240'}

   Vous indiquez ici le temps pour le blocage uniquement pour cette utilisateur.

  Si vous avez oublier de paramétrer la variable \jump{C3SURFBLOCKTIME}{\var{C3SURF\_BLOCKTIME}}.
  L'écrasement n'a de sens que si la variable \jump{C3SURFQUOTA}{\var{C3SURF\_QUOTA='yes'}}
  a été défini.

\config {LOGINUSR\_ACCOUNT\_x\_COUNTER}{LOGINUSR\_ACCOUNT\_x\_COUNTER}{LOGINUSRACCOUNTxCOUNTER}

   \var{LOGINUSR\_ACCOUNT\_x\_COUNTER='1'}

  Vous indiquez ici le nombre de connexions uniquement pour cet utilisateur. 

  Si vous avez oublier de paramétrer la variable \jump{C3SURFCOUNTER}{\var{C3SURF\_COUNTER}}.
  L'écrasement n'a de sens que si la variable \jump{C3SURFQUOTA}{\var{C3SURF\_QUOTA='yes'}}
  a été défini.

\config {LOGINUSR\_ACCOUNT\_x\_CURFEW}{LOGINUSR\_ACCOUNT\_x\_CURFEW}{LOGINUSRACCOUNTxCURFEW}

   \var{LOGINUSR\_ACCOUNT\_x\_CURFEW='Liste couvre-feux'}

   Format~: (Liste des heures de couvre-feux 0-23, les séparés par un espace)
  \begin{example}
  \begin{verbatim}
  Beispiel: LOGINUSR_ACCOUNT_x_CURFEW='0 1 2 3 4 5 6 7 21 22 23'
  \end{verbatim}
  \end{example}
  Signification~: la connexion est autorisée seulement entre 08:00-20:59.
  Si l'utilisateur tente d'ouvrir une session, la connexion sera toujours
  refusée aux heures indiquées dans la liste à (plus 0-59 minutes).

  Si l'utilisateur est connecté et si le couvre-feu fonctionne, il sera
  automatiquement déconnecté sans avertissement. Le comportement de
  déconnexion peut être évité en spécifiant dans la variable
  \jump{C3SURFCHECKCURFEW}{\var{C3SURF\_CHECK\_CURFEW}}='no'

  Cette liste peut être très flexible pour restreindre l'accès. La liste
  peut également être géré dans l'interface web. Aucune vérification de
  la liste n'est effectuée. \achtung{Seul les chiffres de 0 à 23 sont logique~!}

  Paramètres associés à OPT\_C3SURF~:\\
  \jump{C3SURFCHECKCURFEW}{\var{C3SURF\_CHECK\_CURFEW}}='no'
    \begin{itemize}
        \item{C3SURF\_CHECK\_CURFEW='no'}~: pas de déconnexion automatique lorsque
		le couvre-feu est en fonction.
        \item{C3SURF\_CHECK\_CURFEW='yes'} (Par défaut)~: les utilisateurs seront déconnectés
		automatiquement lorsque l'heure du couvre-feu est atteint.
    \end{itemize}
\end{description}
