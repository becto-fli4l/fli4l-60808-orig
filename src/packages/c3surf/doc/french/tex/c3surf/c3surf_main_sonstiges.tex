% Synchronized to r30831
\subsection{Mise en garde}

\wichtig{Si vous n'avez pas activé la variable \jump{OPTLOGINUSR}{\var{OPT\_LOGINUSR='yes'}},
tous les utilisateurs ayant une adresse IP sur leur ordinateur attribuée à
par le routeur, (par exemple à partir d'un Wi-Fi ouvert), dispose d'un accès
Internet libre et des services débloqués sur le routeur. c3Surf est utile pour
bloquer les services, mais ne remplace pas une bonne configuration du pare-feu
dans le fichier "base.txt".}

\subsection{Recommendation}

\wichtig{Assurez-vous que vous avez installé une "version Recovery" sur
votre routeur. En cas de problème, mieux vaut prévenir que guérir.
Avec une configuration incorrecte vous serait complètement exclus du routeur.}

\subsection{Problème et erreur}

Si vous avez des problèmes des erreurs ou un bug, vous pouvez envoyer
une description de l'erreur, avec les informations de configuration
sur le(s) newsgroups fli4l.

Site Web des newsgroups et réglement~:\\
\altlink{http://www.fli4l.de/fr/aide/newsgroup-forum/}

\subsection{License}

Ce logiciel est distribué sous les termes de la Licence Publique Générale GNU
publiée en version 2 ou ultérieur. Selon elle, ce logiciel est gratuit pour
les utilisateurs, les développeurs et les entreprises. Il est de bon style,
de mentionner les développeurs originaux lorsque vous utilisez, modifiez ou
publiez ce logiciel. Cela est particulièrement vrai pour les \oe{}uvres
du domaine public.

\subsection{Litérature}

Si vous mettez à disposition votre réseau pour d'autres, vous devriez
vous intéresser sur la situation juridique.
Wer gerne sein Netz für andere zur Verfügung stellt, der sollte sich auch
einmal mit der rechtlichen Situation auseinander setzen.

Un travail CC sous licence (en allemand) peut être trouvé ici~:\\
Rechtsfragen offener Netze:\\
\altlink{http://digbib.ubka.uni-karlsruhe.de/volltexte/1000007749}\\
Autor: Mantz, Reto\\
Reihe: Schriften des Zentrums für Angewandte Rechtswissenschaft / ZAR\\
Zentrum für Angewandte Rechtswissenschaft, Universität Karlsruhe (TH)\\
Band: 8\\
Verlag: Universitätsverlag Karlsruhe\\
ISBN: 978-3-86644-222-1\\
Erschienen: 10.04.2008
