% Synchronized to r30843
\section {Introduction}

Avec C3SURF vous pouvez créer un système de processus ouvert, un réseau/WiFi non crypté.
Cependant, pour des raisons juridiques vous devez contrôler qui utilise le réseau.
C3SURF permet un enregistrement informelle sur le réseau. Le paquetage est basé sur
"opt\_onco" (Copyright (c) 2001-2007 Michael Mattes). En utilisant PT\_LOGINUSR une
"presque" réelle connexion peut être réalisée. C3SURF peut générer des tickets et
un contrôle rudimentaire (expérimental) de la fonction de suppression, lors de
téléchargements excessifs.


Vous pouvez définir des hôtes ou des réseaux entiers qui seront gérés par C3SURF.
Les paramètres de configuration sont d'abord bloqués au démarrage du routeur.
Les requêtes HTTP de ces hôtes ou des clients réseau sont redirigés vers une
page de connexion C3SURF. Après avoir enregistré un compte sur la page de
connexion, votre réseau pourrat être accésible par l'utilisation de C3SURF. Tout
est consigné - Vous pouvez tout contrôler via l'interface web de l'administrateur
C3SURF.

\section {Remarques sur l'installation}

\begin{itemize}
\item Comme d'abitude avec les paquetages-opt pour fli4l~:
  \begin{itemize}
	\item Décompresser le fichier opt\_c3surf\_$<$versionsinfo$>$.tar.gz dans
        le répertoire fli4l (de l'ordinateur de construction).
	\item Paramétrer le fichier c3surf.txt selon vos besoins (décrit plus bas).
	\item Dans le cas échéant ajouter dans le fichier httpd.txt
        les droits d'accés 'c3surf:view,admin'.
	\item Créer une nouvelle construction du routeur.\\
\end{itemize}
\achtung{Important~: fli4l doit être enregistré comme serveur DNS pour les clients
        du réseau, il doit être capable de résoudre les noms d'hôtes. De plus}\\
          \begin{itemize}
                \item \emph{fli4l a besoin du "Forward" pour le serveur DNS du réseau ou}
                \item \emph{utiliser fli4l comme le serveur DNS, il pourra éventuellement
				établir une connexion automatiquement.}
          \end{itemize}

               \emph{Sinon, il y aura des problèmes de redirection automatique vers
               la page de connexion. Cependant il sera toujours possible d'accéder
               à la page de connexion en entrant manuellement URL de l'hôte.}

\section {Configuration OPT\_C3SURF}
\begin{description}

\config {OPT\_C3SURF}{OPT\_C3SURF}{OPTC3SURF}

  Paramètre par défaut~: \var{OPT\_\-C3SURF='no'}

  Avec cette variable vous activez ou déactiver le paquetage

\config {C3SURF\_LOG\_PATH}{C3SURF\_LOG\_PATH}{C3SURFLOGPATH}

  Paramètre par défaut~: \var{C3SURF\_LOG\_PATH='/var/log/c3surf'}

  Avec cette variable vous définissez le répertoire pour les fichiers
  log (ou journal) de C3SURF. Lorsque vous arrêtez le routeur les fichiers
  (log) doivent être enregistrés sur un support persistant si vous souhaitez
  les conserver le chemin doit exister sur le support permanent.

  \wichtig{Le fichier 'c3surf\_mac.blacklist' se trouve dans le répertoire
  persistant\\
  \jump{C3SURFPERSISTENTPATH}{\var{C3SURF\_PERSISTENT\_PATH}}. Une liste noire
  distincte doit être copié ici. Le champ d'application du protocole est défini
  ci-dessous.}

\config {C3SURF\_DOLOG\_LOGIN}{C3SURF\_DOLOG\_LOGIN}{C3SURFDOLOGLOGIN}

  Paramètre par défaut~: \var{C3SURF\_DOLOG\_LOGIN='yes'}

  Pour enregistrer les Connexion/Déconnexion~: c3surf\_login.log (par défaut~: 'yes')

\config {C3SURF\_DOLOG\_INVALID}{C3SURF\_DOLOG\_INVALID}{C3SURFDOLOGINVALID}

  Paramètre par défaut~: \var{C3SURF\_DOLOG\_INVALID='yes'}

  Pour enregistrer les connexions non valides~: c3surf\_invalid.log (par défaut~: 'yes').
  Si la variable est sur \jump{OPTLOGINUSR}{\var{OPT\_LOGINUSR}} ='yes',
  les connexions non valides ne peuvent pas être enregistrés.

\config {C3SURF\_DOLOG\_PAGE}{C3SURF\_DOLOG\_PAGE}{C3SURFDOLOGPAGE}

  Paramètre par défaut~: \var{C3SURF\_DOLOG\_PAGE='no'}

  Pour enregistrer des accès à la page html~: c3surf\_page.log (par défaut~: 'no').
  Chaque accès à la page de connexion sera enregistré. Le journal page html
  devient rapidement volumineux, c'est donc recommandée que par "curiosité".

\config {C3SURF\_DOLOG\_HTTPD}{C3SURF\_DOLOG\_HTTPD}{C3SURFDOLOGHTTPD}

  Paramètre par défaut~: \var{C3SURF\_DOLOG\_HTTPD='no'}

  Pour enregistrer tous les accès sur le mini\_httpd~: c3surf\_httpd.log (par défaut~: 'no').

  \achtung{De plus si vous activez la fonction journal Mini-httpd (est utiliser
  uniquement pour des tests ou le débogage). Lorsqu'elle est activée, il est
  recommandé de vérifier régulièrement le journal et de le supprimer, car
  il devient rapidement volumineux.}

  L'opt\_cpmvrmlog~: \altlink{http://extern.fli4l.de/fli4l_opt-db3/search.pl?pid=427}
  peut être utilisé pour une sauvegarde régulière. Le mini\_httpd doit être redémarré
  pour un enregistrement correcte.\\
  Vous avez également le script \var{/usr/local/bin/c3surf\_kill\_httpd.sh}
  \jump{C3SURFLOG}{\var{(Exemple de configuration dans l'annexe).}}

\config {C3SURF\_PERSISTENT\_PATH}{C3SURF\_PERSISTENT\_PATH}{C3SURFPERSISTENTPATH}

  A paramétrer dans tous les cas, recommandé~: '/var/lib/persistent/c3surf'

  Dans cette variable vous définissez le répertoire pour conserver les fichiers après
  un redémarrage ou un poweroff. L'idéal, serait peut être un disque dur ou une carte CF
  ('/var/lib/persistent/c3surf'). Vous pouvez aussi choisir de définir le répertoire sur
  le disque RAM (par exemple, pour minimiser l'accès au support). Si vous vous utilisez
  le répertoire sur un disque RAM, vous devez copié les données sur le disque persistent
  de temps en temps en utilisant (par exemple opt\_cpmvrmlog), car les données seront
  perdues après un redémarrage accidentel ou une coupure de courant.

  \achtung{Ce qui sera sauvegardé~:}
\newpage
  \achtung{MAC blacklists (ou liste noire)~:}

  'c3surf\_mac.blacklist', sera créé en cas de besoin (voir l'interface
  administration). Le blocage d'une adresse MAC sera résolu par ce fichier
  et non pas par le filtrage de paquets, car si vous entrez de grandes quantités
  de données dans le filtrage de paquets cela peut causer des problèmes. N'oubliez
  pas~: que les adresses MAC bloquent les utilisateurs par défaut sur votre réseau,
  ce qui est suffisant pour une utilisation normale, mais n'est pas professionnels.
  MAC-Blacklist empêche seulement la connexion via C3SURF / loginusr, car elle n'est
  pas configurée directement dans le pare-feu.

  \achtung{Données de l'utilisateur~:}

  $<$userloginname$>$.data (par ex. 'frank.data'), ce fichier contient les données
  sur l'utilisateur, telles que le Nom et Prénom, l'adresse E-mail, les statistiques et
  les contrôles de quotas. Les données utilisateurs doivent être persistantes, cela évite
  de recréer le fichier de données utilisateur à chaque démarrage. Cela signifie~: que si
  l'utilisateur "frank" a déjà un fichier 'frank.data' au démarrage du système, les paramètres
  du fichier de configuration seront ignorés.
\parskip 12pt

  L'écrasement des données de l'utilisateur peut être renforcée avec l'activation de
  la variable \jump{LOGINUSRACCOUNTxOVERWRITE}{\var{LOGINUSR\_ACCOUNT\_x\_OVERWRITE='yes'}}.
  et de \jump{LOGINUSRDELETEPERSISTENTDATA}{\var{LOGINUSR\_DELETE\_PERSISTENT\_DATA='yes'}},
  tous les données du fichier "*.data"  seront effacées lors du redémarrage.

\config {C3SURF\_WORKON\_TMP}{C3SURF\_WORKON\_TMP}{C3SURFWORKONTMP}

  Paramètre par défaut~: \var{C3SURF\_WORKON\_TMP='no'}

  Si vous avez activer la variable \jump{C3SURFPERSISTENTPATH}{\var{C3SURF\_PERSISTENT\_PATH}}
  avec la valeur 'yes'. Le fichier de données persistantes sera copié au
  démarrage du système dans le répertoire C3SURF\_TMP\_PATH à partir du disque
  dur, ensuite ils seront lu. L'accès au disque dur par C3SURF ne se fera que
  si l'administrateur a écrit des données dans des fichiers persistants.

 \wichtig{Les données persistantes sont~:}
  \begin{itemize}
  \item \emph{Les comptes utilisateurs}
  \item \emph{La MAC-Blackliste}
  \item \emph{Le fichier de verrouillage du système (blocage des connexions)}\\
  \end{itemize}

  \emph{Si vous utilisez la mémoire flash vous pouvez indiquez 'no' parce que dans
  une utilisation normale C3SURF est en lecture seule. L'accès en ècrire
  est produit que par l'administrateur.}

\config {C3SURF\_QUOTA}{C3SURF\_QUOTA}{C3SURFQUOTA}

  Paramètre par défaut~: \var{C3SURF\_QUOTA='no'}

  Si vous limité les accès, indiquez ici 'yes'. Ainsi, l'adresse IP ou le compte
  enregistré sera bloqué après le temps dépassé qui est paramétré dans la variable
  \jump{C3SURFBLOCKTIME}{\var{C3SURF\_BLOCKTIME}} en minutes.
  La valeur par défaut est 'yes'.

  \wichtig{Individuellement les comptes LOGIN\_USR pour -TIME, -BLOCKTIME et -COUNTER
  sont activés ('yes') ou désactivée ('no') par cette variable.}

\config {C3SURF\_COUNTER}{C3SURF\_COUNTER}{C3SURFCOUNTER}

  Paramètre par défaut~: \var{C3SURF\_COUNTER='0'}

  Dans cette variable vous indiquez un nombre de connexion possible dans un forfait.

  \wichtig{Vous pouvez définir un certain nombre de connexion pour (Logout/Login).
  par exemple si vous indiquez '1' l'utilisateur pourra se déconnecter et se
  connecter une fois dans le forfait, cela correspond à deux de connexions
  dans un temps donné. Lors de la  connexion suivant, l'utilisateur obtiendra la
  différence du temps écoulé, par rapport à la valeur de la variable
  \jump{C3SURFTIME}{\var{C3SURF\_TIME}}.}

   \emph{Si en plus, la variable est paramétré sur \jump{C3SURFBLOCKTIME}{\var{C3SURF\_BLOCKTIME='0'}},
   la variable \jump{C3SURFCOUNTER}{\var{C3SURF\_COUNTER}} sera réinitialisé à 0:00 heures
   le jour suivante.}

  \begin{itemize}
    \item{Avec C3SURF\_COUNTER='0'}

    la valeur correspond comme un horodateur (de l'argent dedans, l'argent disparait,
	le temps passe~: pas de connexion possible).

    \item{Avec C3SURF\_COUNTER='-1'}

    cette fonction est désactivée = à de nombreuses connexions comme avec un forfait
	gratuit.

    \item{Avec C3SURF\_COUNTER='-2'}

    De nombreuses connexions sont possibles comme avec un forfait gratuit (ressemble à
	la valeur '-1'), mais le compte à rebours du temps de blocage commence avec
	la première connexion. Contrairement à '-1', le temps de blocage commence quand
	vous utilisatez le temps de connexion.
  \end{itemize}

   Remarque pour une quantité de connexion avec une longue durée (C3SURF\_COUNTER='-2')~:\\
   Ainsi, vous pouvez par exemple indiquer 10 heures de temps de connexion dans
   \jump{C3SURFTIME}{\var{(C3SURF\_TIME='600')}} et combiner avec un délai de bloquage
   d'une semaine dans \jump{C3SURFBLOCKTIME}{\var{(C3SURF\_BLOCKTIME='10080'}}~:
   60sec x 24h x 7jours). Ainsi les 10 heures peuvent être consommés dans la semaine.\\
   En résumé~: L'utilisateur a une semaine avec un quota de 10 heures, dont il peut répartir
   les connexions dans la semaine. S'il ne consomme pas le quota dans la semaine, il n'y aura
   pas de "quota bloqué". il n'y a pas aussi de temps d'attente. Si le quota est utilisé
   le premier jour, le compte sera bloqué les 6 autres jours de la semaine.\\
   S'applique également à la variable \jump{LOGINUSRACCOUNTxCOUNTER}{\var{LOGINUSR\_ACCOUNT\_x\_COUNTER}}

   Recommandation~: la variable \jump{C3SURFSAVEQUOTA}{\var{C3SURF\_SAVE\_QUOTA='yes'}}
   conserve les valeurs même après un redémarrage normal. Sur une panne de courant les
   valeurs seront perdues.

   Si la variable \jump{C3SURFQUOTA}{\var{C3SURF\_QUOTA='yes'}} est activé, le blocage se sera
   en fonction de la variable \jump{C3SURFBLOCKTIME}{\var{C3SURF\_BLOCKTIME}} après le décompte
   du compteur.

\config {C3SURF\_TIME}{C3SURF\_TIME}{C3SURFTIME}

  Paramètre par défaut~: \var{C3SURF\_TIME='60'}

  Dans cette variable vous indiquez le temps en minute que vous avez besoin pour la connexion

  Si vous indiquez la valeur '0' la connexion sera illimitée (cela est valable aussi
  pour la variable LOGINUSR\_ACCOUNT\_x\_TIME).

  \underline{Cas particulier~:}
    \begin{itemize}
      \item{C3SURF\_TIME='0'}

      Cela signifie une connexion illimitée. L'utilisateur doit se déconnecter.
	  Le système (C3SURF) déconnecte l'utilisateur que si l'ordinateur a été arrêté et
	  que la variable\\
	  \jump{C3SURFCHECKARP}{\var{C3SURF\_CHECK\_ARP='yes'}} (a été paramétré par défaut)
    \end{itemize}

\config {C3SURF\_BLOCKTIME}{C3SURF\_BLOCKTIME}{C3SURFBLOCKTIME}

  Paramètre par défaut~: \var{C3SURF\_BLOCKTIME='240'}

  Dans cette variable vous indiquez le temps en minute, après lequel l'adresse IP sera
  bloqué, l'administrateur peut aussi effectue un bloquage via l'interface Web. Un ordinateur
  peut être bloqué sur le réseau pour une période définie. La variable
  \jump{C3SURFQUOTA}{\var{C3SURF\_QUOTA='yes'}} doit être activée afin de réaliser le bloquage.

  \underline{Cas particulier~:}\\
    \begin{itemize}
      \item{C3SURF\_BLOCKTIME='0'}

            Le blocage de l'adresse ou de l'utilisateur est effectuée à 00:00 heure
			le jour suivant.

       \item{C3SURF\_BLOCKTIME='-1'}

             il n'y a pas de blocage.
     \end{itemize}

    \wichtig{Le blocage est effectué avec une précision d'une minute.}

\config {C3SURF\_SAVE\_QUOTA}{C3SURF\_SAVE\_QUOTA}{C3SURFSAVEQUOTA}

  Paramètre par défaut~: \var{C3SURF\_SAVE\_QUOTA='yes'}

  Si vous voulez sauvegarder les valeurs du quota à l'arrêt et recharger lorsque vous
  démarrez le routeur vous pouvez activez cette variable. Ainsi, les fichiers
  temporaires de la gestion des quotas qui sont écrits dans
  \jump{C3SURFPERSISTENTPATH}{\var{C3SURF\_PERSISTENT\_PATH}} seront copiés dans le
  répertoire temporaire lorsque vous redémarrez le routeur, bien sûr avec un redémarrage
  normal. Les données actuelles de consommation de l'utilisateur seront conservées.
  Lors d'un arrêt accidentel les données ne seront pas récupérables.

  \wichtig{Si cette variable est paramétrée sur \jump{LOGINUSRDELETEPERSISTENTDATA}{\var{LOGINUSR\_DELETE\_PERSISTENT\_DATA='no'}},
   tous les comptes utilisateur et les données de quotas associés de cette fonction seront
   supprimés au redémarrage.}

\config {C3SURF\_CHECK\_ARP}{C3SURF\_CHECK\_ARP}{C3SURFCHECKARP}

  Paramètre par défaut~: \var{C3SURF\_CHECK\_ARP='yes'}

  Avec cette variable vous pouvez vérifier si l'adresse IP d'un ordinateur à partir
  de la table ARP est déconnecté, en utilisant un module de compte à rebours. Par exemple,
  si un ordinateur se déconnecte, il sera parfois détecté avec un décalage de temps
  considérable.

\config {C3SURF\_CONTROL\_HOST\_OR\_NET\_N}{C3SURF\_CONTROL\_HOST\_OR\_NET\_N}{C3SURFCONTROLHOSTORNETN}

 \var{C3SURF\_CONTROL\_HOST\_OR\_NET\_N='0'}

  La valeur~: est un nombre entier.\\
  Avec cette variable vous indiquez le nombre de plages IP ou d'hôtes qui seront
  contrôlés par c3Surf. Cela affecte le transfert des adresses vers un autre réseau
  (Chaîne FORWARD).

\config {C3SURF\_CONTROL\_HOST\_OR\_NET\_x}{C3SURF\_CONTROL\_HOST\_OR\_NET\_x}{C3SURFCONTROLHOSTORNETx}\ \\
 \var{C3SURF\_CONTROL\_HOST\_OR\_NET\_x='Réseau ou Hôte ou Adresse-IP'}

  Contrôle de tous les clients.

  \wichtig{Pour simplifier la configuration d'un réseau complet vous paramétrez,
		par ex. WLAN, tous les utilisateurs du sans fil devront utiliser la page de
		connexion. Vous pouvez également spécifier un hôte (@host) ou une adresse IP.
		Tous se qui est indiqué ici sera dirigé vers la page de connexion et les règles
		de bloquage spécifier ci-dessous seront appliquées}

\underline{Exemple~:}

\begin{example}
\begin{verbatim}
C3SURF_CONTROL_HOST_OR_NET_1='IP_NET_3'       # Vous spécifiez un réseau IP/MASK
C3SURF_CONTROL_HOST_OR_NET_2='@T8200'         # ou un hôte @HOST
C3SURF_CONTROL_HOST_OR_NET_3='192.168.13.11'  # ou une adresse-IP
\end{verbatim}
\end{example}

L'exemple de la valeur (IP\_NET\_3) suivant est identique à celle du dessus,
si dans le fichier "base.txt" l'adresse IP a été réglé en conséquence.

\begin{example}
\begin{verbatim}
C3SURF_CONTROL_HOST_OR_NET_1='192.168.0.1/24' # Contrôle tous les clients
\end{verbatim}
\end{example}

Pour exclure des ordinateurs, vous pouvez indiquer toutes les adresses IP
individuellement dans C3SURF.txt (c'est à dire, créer une liste de 256 adresses,
sans en oublié un) ou vous pouvez utiliser la notation CIDR (comme ci-dessous),
des groupes d'adresse IP seront créés, on aura moins d'écriture à faire
(8 lignes au lieu de 255).

\underline{Cela ressemblera à ceci~:}

\begin{example}
\begin{verbatim}
C3SURF_CONTROL_HOST_OR_NET_N='8'                # Nombre d'hôtes ou de réseau
C3SURF_CONTROL_HOST_OR_NET_1='192.168.0.0/31'   # 0-1
C3SURF_CONTROL_HOST_OR_NET_2='192.168.0.3'      # Adresse 3 seul pas 2
C3SURF_CONTROL_HOST_OR_NET_3='192.168.0.4/30'   # 4-7
C3SURF_CONTROL_HOST_OR_NET_4='192.168.0.8/29'   # 8-15
C3SURF_CONTROL_HOST_OR_NET_5='192.168.0.16/28'  # 16-31
C3SURF_CONTROL_HOST_OR_NET_6='192.168.0.32/27'  # 32-63
C3SURF_CONTROL_HOST_OR_NET_7='192.168.0.64/26'  # 64-127
C3SURF_CONTROL_HOST_OR_NET_8='192.168.0.128/25' # 128-255
\end{verbatim}
\end{example}


L'ordinateur avec l'adresse IP '192.168.0.2 ' n'est pas enregistrée,
les autorisations de celui-ci sont paramétrée dans le pare-feu de fli4l.

\config {C3SURF\_CONTROL\_PORT\_N}{C3SURF\_CONTROL\_PORT\_N}{C3SURFCONTROLPORTN}

   \var{C3SURF\_CONTROL\_PORT\_N='0'}

  La valeur~: est un nombre entier.\\
  Avec cette variable vous indiquez le nombre de ports TCP à contrôler sur le routeur.

  Combien de ports seront ils contrôlés et qui sera explicitement désigné par c3Surf~?
  Seul les adresses IP et les hôtes qui sont affecté dans la variable\\
  \jump{C3SURFCONTROLHOSTORNETN}{\var{C3SURF\_CONTROL\_HOST\_OR\_NET\_N}} auront des
  ports contrôlés. c3Surf contrôle les ports et les ouvrent après une connexion réussie,
  de sorte que les services existants via ces ports peuvent être utilisés derrière le routeur
  (la chaîne INPUT est concernée).

\config {C3SURF\_CONTROL\_PORT\_x}{C3SURF\_CONTROL\_PORT\_x}{C3SURFCONTROLPORTx}

   \var{C3SURF\_CONTROL\_PORT\_x='port\_nr'}

  Dans cette variable vous pouvez spécifier le numéro de port pour accèder au
  service du routeur (fli4l), il sera bloqué jusqu'à l'enregistrementdu compte.
  Après l'inscription, les services peuvent être utilisés en fonction de la période
  d'activation.

\underline{Exemple~:}
\begin{example}
\begin{verbatim}
C3SURF_CONTROL_PORT_1='515' # par ex. lpdsrv (imprimante utilisable après l'enregistrement)
C3SURF_CONTROL_PORT_2='21'  # par ex. ftp - (ftp sur le routeur!)
\end{verbatim}
\end{example}

\begin{example}
\begin{verbatim}
Autres adresses de ports possibles~:
  21=ftp
  22=ssh
  5000=imonc
  5001=telmod
  8118=privoxy
  9050=tor
  3128=squid
  20000=mtgcapri
  80=http(Admin)
  515=lpdsrv
\end{verbatim}
\end{example}

        Tout dépend de votre propre configuration. les règles dans 'base.txt'
		s'appliquent à tous les ports même ceux qui ne sont pas mentionnés ici.
		Après l'enregistrement, les règles dans 'base.txt' s'appliquent toujours.
		c3Surf relie en amont l'application avec l'utilisateur uniquement par
		les règles. Ainsi, après un enregistrement, toutes les règles sont respectées.
		Vous pouvez, par exemple, dans 'base.txt' interdire l'accès du WLAN sur
		le réseau câblé local. Cette interdiction s'appliquera alors aux utilisateurs
		de c3Surf pour les connectés WLAN (ou sans fil).

\config {C3SURF\_BLOCK\_PORT\_N}{C3SURF\_BLOCK\_PORT\_N}{C3SURFBLOCKPORTN}

   \var{C3SURF\_BLOCK\_PORT\_N='0'}

  La valeur~: est un nombre entier.\\
  Avec cette variable vous indiquez le nombre de ports TCP à bloquer sur le routeur.

   \underline{Remarques~:}\\
   Combien de port seront ils bloqués de façon permanente et qui sera explicitement
   désigné par c3Surf~? Seul les adresses IP et les hôtes qui sont affecté dans la variable\\
   \jump{C3SURFCONTROLHOSTORNETN}{\var{"C3SURF\_CONTROL\_HOST\_OR\_NET\_N"}}
   auront les services bloqués en permanence. Les ports d'accès aux services derrière
   le routeur (fli4l) seront bloqués pour les hôtes et/ou des ordinateurs du réseau,
   même après un enregistrement sur c3Surf. La chaîne INPUT est concernée. Si vous voulez
   vraiment bloquer certains services de façon permanent, il est préférable de paramétrer
   la chaîne INPUT dans le fichier 'base.txt'.

   \underline{Pourquoi~:}\\
   Parce que les règles configurées ici ne s'appliqueront plus, dès que la variable\\
   \jump{OPTC3SURF}{\var{OPT\_C3SURF='no'}} est déactivé. Donc, si C3SURF est désactivé,
   les règles définies ici doivent être transférés dans le fichier 'base.txt', si vous
   voulez que les bloquages pour les hôtes ou les réseaux mentionnés ci-dessous persistent.

\config {C3SURF\_BLOCK\_PORT\_x}{C3SURF\_BLOCK\_PORT\_x}{C3SURFBLOCKPORTx}

   \var{C3SURF\_BLOCK\_PORT\_x='port\_nr'}

\underline{Exemple~:}
\begin{example}
\begin{verbatim}
C3SURF_BLOCK_PORT_1='5000'           # Par ex. imonc
C3SURF_BLOCK_PORT_2='5001'           # Par ex. telmond
C3SURF_BLOCK_PORT_3='20000'          # Par ex. mtgcapri (OPT_MTGCAPRI)
C3SURF_BLOCK_PORT_4='22'             # Par ex. ssh
C3SURF_BLOCK_PORT_5='8118'           # Par ex. privoxy (PROXY)
C3SURF_BLOCK_PORT_6='9050'           # Par ex. tor (PROXY)
C3SURF_BLOCK_PORT_7='80'             # Par ex. httpd Admin interface (HTTPD)
C3SURF_BLOCK_PORT_8='7437'           # Par ex. caiviar (OPT_CAIVIAR)
\end{verbatim}
\end{example}

\config {C3SURF\_HTTPD\_PORT}{C3SURF\_HTTPD\_PORT}{C3SURFHTTPDPORT}

  Paramètre par défaut~: \var{C3SURF\_HTTPD\_PORT='8080'}

  Avec quel port et quel adresse IP le mini\_httpd c3Surf écoute pour la connexion des
  utilisateurs~? Les requêtes http de l'ordinateurs seront redirigés vers cette adresse et
  ce port. Le port 8080 est la valeur par défaut.

  \achtung{Lors du choix du numéro de port, les points suivants doivent être pris en compte~:}
  \begin{itemize}
  \item Vous ne devez pas utiliser le port du paquetage httpd (c'est habituellement le port 80).
  \item L'administrateur Web httpd pour fli4l lie par défaut toutes les adresses IP locales.
  \item Vous ne devez utiliser aucun numéro de port qui est déjà utilisé par un autre service.\\
  \end{itemize}
  Si vous définissez par erreur un port qui est déjà utilisé, fli4l essaira de démarrer
  encore et encore le httpd. Le démarrage échoura parce que le port est déjà occupé par
  une interface admin ou un autre service. Cela peut être vu sur la console ou dans un
  fichier log. Vous le remarqué parce C3SURF ne fonctionne pas et une charge élevée du
  processeur sera génèrée par fli4l, il semblera fonctionner lentement.

\config {C3SURF\_HTTPD\_LISTENIP}{C3SURF\_HTTPD\_LISTENIP}{C3SURFHTTPDLISTENIP}

  Paramètre par défaut~: \var{C3SURF\_HTTPD\_LISTENIP='Hôte ou Adresse-IP'}

  Dans cette variable vous indiquez l'adresse IP locale où l'interface qui sera
  utilisé par l'utilisateur pour se connecté, les requêtes http des clients seront
  redirigés (si l'utilisateur n'est sont pas enregistré). Ainsi les utilisateurs
  seront plus rapidement sur la page de connexion.

\underline{Exemple~:}
\begin{example}
\begin{verbatim}
C3SURF_HTTPD_LISTENIP='@wifi-router'    # Un nom d'hôte
C3SURF_HTTPD_LISTENIP='192.168.11.3'    # Une adresse-IP
C3SURF_HTTPD_LISTENIP='IP_NET_1_IPADDR' # Une variable d'adresse-IP
\end{verbatim}
\end{example}

Le service http pour C3SURF est toujours attaché à une adresse IP.

\end{description}

\subsection {Paramètres optionnels de OPT\_C3SURF}

\begin{description}

\config {C3SURF\_CONTROL\_SQUID}{C3SURF\_CONTROL\_SQUID}{C3SURFCONTROLSQUID}

  Paramètre par défaut~: \var{C3SURF\_CONTROL\_SQUID='no'}

  Si vous activez cette variable C3SURF\_CONTROL\_SQUID='yes', le contrôle
  sera appliqué via squid. La redirection de port C3SURF sera établi au démarrage,
  il y aura également des répercussions sur d'autres paquets (par exemple openvpn).

  Die Empfehlung ist 'no', wer z.B. squid verwendet sollte prüfen, ob nicht
  ungewollt noch andere Funktionen dadurch beeinflusst werden.

\config {C3SURF\_SLOPPY\_MAC}{C3SURF\_SLOPPY\_MAC}{C3SURFSLOPPYMAC}

  Paramètre par défaut~: \var{C3SURF\_SLOPPY\_MAC='no'}

   \begin{itemize}
      \item{C3SURF\_SLOPPY\_MAC='no'}

            (Par défaut) - si ce paramètre n'est pas activé, à la connexion
			seul des adresses MAC enregistrées seront recherchées dans la table ARP.

       \item{C3SURF\_SLOPPY\_MAC='yes'}

            C3SURF accepte les adresses MAC qui manquent et les trouvera via la table ARP.
			Si vous avez sélectionné 'yes', vous devez déactiver la variable
			\jump{C3SURFCHECKARP}{\var{C3SURF\_CHECK\_ARP='no'}}. Sinon, la déconnexion automatique
			(en moyenne après une minute) sera activée, car le "processus de compte à rebours" se
			déclenche en raison d'un enregistrement manquant dans la table ARP.
   \end{itemize}

\config {C3SURF\_CHECK\_CURFEW}{C3SURF\_CHECK\_CURFEW}{C3SURFCHECKCURFEW}

  Paramètre par défaut~: \var{C3SURF\_CHECK\_CURFEW='yes'}

  Avec cette variable vous activez la déconnexion automatique en atteignant le couvre-feu,
  les valeurs sont (\var{'yes'}) ou (\var{'no'}).

\config {C3SURF\_PORTAL\_DEFAULT\_LANG}{C3SURF\_PORTAL\_DEFAULT\_LANG}{C3SURFPORTALDEFAULTLANG}

  Paramètre par défaut~: \var{C3SURF\_PORTAL\_DEFAULT\_LANG='de'}

  Valeur possible~: le code du pays est indiqué par deux caractères (par ex. 'de', 'fr', 'en').

  Avec cette variable vous définissez la langue par défaut de la page de connexion. Si vous
  n'indiquez aucune valeur, alors la langue 'de' sera activé.

  Il existe un fichier appelé c3surf.$<$codepays$>$ dans le répertoire
  \textasciitilde/opt/files/srv/www/c3surf/lang/. Actuellement les fichiers , 'en', 'fr',
  'en' et 'it' sont inclus dans ce répertoire. Vous pouvez traduire ce fichier pour
  une autre langue et l'envoyé par mail à l'équipe de fli4l.

\config {C3SURF\_PORTAL\_LANGUAGES}{C3SURF\_PORTAL\_LANGUAGES}{C3SURFPORTALLANGUAGES}\ \\

  Paramètre par défaut~: \var{C3SURF\_PORTAL\_DEFAULT\_LANG='de fr en it'}

  Valeurs possibles~: le code du pays est indiqué par deux caractères et séparé par un espace.

  Dans cette variable vous indiquez les fichiers de langues qui doivent être transférés au
  système pour la page de connexion. S'il n'y a pas de fichier langue correspondant au raccourci
  du code pays, un avertissement sera émis lors de la création de l'image du routeur, comme quoi
  que le fichier du code pays n'a pas été trouvé et rien a été copié. Le processus de construction
  continura il ne sera pas abandonné.

\end{description}