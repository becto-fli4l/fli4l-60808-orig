% Synchronized to r30467
\begin{tabular}{rlcl}
  Opt et Doc~: & 07. Janvier 2008 & Frank Saurbier & \altlink{mailto:c3surf@arcor.de}\\
  Doc-Tex~: & 01. Avril 2009 & Helmut Backhaus & \altlink{mailto:helmut.backhaus@gmx.de}\\
  Réédition~: & 01. Mai 2010 & L'équipe fli4l & \altlink{mailto:team@fli4l.de}\\
\end{tabular}


\section{Conseils et autres Opts}

\subsection{cpmvrmlog Config}

\marklabel{C3SURFLOG}{ }
\underline{Un exemple du fichier du log de C3SURF, avec redémarrage du mini\_httpd}
\begin{example}
\begin{verbatim}
# archive C3SURF log dir
# einmal im Monat am 1. um 01:30
# maximal 12 Archive aufbewahren
CPMVRMLOG_n_ACTION='move'
CPMVRMLOG_n_SOURCE='/var/log/c3surf/c3surf_*.log'
CPMVRMLOG_n_DESTINATION='/data/Archive/log/c3surf'
CPMVRMLOG_n_CUSTOM='/usr/local/bin/c3surf_kill_httpd.sh'
CPMVRMLOG_n_MAXCOUNT='12'
CPMVRMLOG_n_CRONTIME='30 1 1 * *'
\end{verbatim}
\end{example}

\subsection{Autoriser Samba sans enregistrement}

\underline{Utilisez le paquetage opt\_usercmd avec les entrées suivantes.}

\begin{example}
\begin{verbatim}
USERCMD_BOOT_N='3'
USERCMD_BOOT_1='/sbin/iptables -I c3surf\_control 1 -v -p udp --dport 
                137:138 -j RETURN' # samba thru c3surf
USERCMD_BOOT_2='/sbin/iptables -I c3surf\_control 1 -v -p tcp --dport
                455 -j RETURN'     # samba thru c3surf
USERCMD_BOOT_3='/sbin/iptables -I c3surf\_control 1 -v -p tcp --dport
                139 -j RETURN'     # samba thru c3surf
\end{verbatim}
\end{example}
En ajoutant l'option \var{-d IPsambaHOST} à la règle correspondante,
vous tiendrez compte de l'ordinateur cible.

\parskip 12pt

Ainsi, les ports de samba seront normalement transmis par la chaîne Forward
et ne seront plus bloqués par C3SURF. Même si dans la chaîne Forward de samba
d'autre paramètre sont enregistré pour une interdiction, ces enregistrements
ne changera rien au fonctionnement.

Les réglages dans le fichier base.txt seront toujours appliqués.

\subsection{Migration des versions précédentes}

\item Migration vers la Version 2.3.1 (de 2.3.0)
     \begin{itemize}
     \item Nouvelle variable, seulement en option. Les nouvelles variables sont
	  identifiées dans le config.txt comme ceci
     \item "\# $+$ new 2.3.1 $+$ begin ------------------ delete this line".
     \item Le format des tickets a changé, les anciens tickets peuvent être utilisé,
	 mais ils ne sont pas reconnus lors de la génération de nouveau ticket.
	 Si vous aimez un système propre, vous devrait enlever tous les tickets et
	 en créer de nouveaux.
     \end{itemize}

\item Migration vers Version 2.3.0 (de 2.2.2)
    \begin{itemize}
    \item Si vous souhaitez utiliser des tickets, aucun changement de configuration
    n'est nécessaires.
    \item Ajouter d'une  nouvelle variable avec OPT\_C3SURF\_VOUCHER, si vous souhaitez
    utiliser la fonction ticket.
    \item Les nouvelles variables sont identifiées dans le config.txt comme ceci
    \item "\# $+$ new 2.3.0 $+$ begin ------------------ delete this line".
\end{itemize}

\item Migration vers Version 2.2.2 (de 2.2.1)
    \begin{itemize}
    \item Les nouvelles variables sont identifiées dans le config.txt comme ceci
    \item "\# $+$ new 2.2.2 $+$ begin ------------------ delete this line".
    \item C3SURF\_CONTROL\_SQUID~: optionnelle pour le contrôle squid, parce que le squid
    n'est pas conforme aux conventions, c'est provisoire.
    \item La variable pour remplacer la valeur par défaut du quota dans LOGINUSR\_ACCOUNT
    sont maintenant en option
\end{itemize}

\item Migration vers Version 2.2.1 (de 2.2.0)
\begin{itemize}
   \item Les nouvelles variables sont identifiées dans le config.txt comme ceci
   \item "\# $+$ new 2.2.1 $+$ begin ------------------ delete this line".
   \item C3SURF\_WORKON\_TMP~: Recommander pour mettre en sommeil le disque dur 'yes' ou 'no',
   même avec un système FLASH.
   \item C3SURF\_SAVE\_QUOTA~: 'yes' recommander.
\end{itemize}


\item Migration vers Version 2.2.0 (de 2.1.0)
\begin{itemize}
   \item Ajouter la nouvelle variable "C3SURF\_CHECK\_ARP" dans la configuration ('yes' recommander)
   \item Les nouvelles variables sont identifiées dans le config.txt comme ceci
   \item "\# $+$ new 2.2.0 $+$ begin ------------------ delete this line".
\end{itemize}


\item Migration vers Version 2.1.0 (à partir de version précédente)
\begin{itemize} 
   \item Les nouvelles variables sont identifiées dans le config.txt comme ceci
   \item "\# $+$ new 2.1.0 $+$ begin ------------------ delete this line".
   \item La MAC-Blackliste (si vous l'avez un bien entretenu) doit être copié manuellement dans le répertoire\\
   "C3SURF\_PERSISTENT\_PATH".
   \item Une colonne a été ajouté dans le format de c3surf\_login.log. Le mieux est de supprimer
   les anciens fichiers dans C3SURF\_LOG\_PATH.
\end{itemize}
\end{itemize}
