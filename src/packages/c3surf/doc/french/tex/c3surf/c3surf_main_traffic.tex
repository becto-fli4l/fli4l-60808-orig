% Synchronized to r30467
   OPT\_C3SURF\_TRAFFIC permet de réguler un "Power User" (ou un utilisateur intensif).
   Il sera surveillé et évalue sur la quantité de données, dans un intervalle de temps
   prédéfini. La configuration peut être personnalisée en fonction de vos besoins.

\section {Configuration de OPT\_C3SURF\_TRAFFIC}

\begin{description}

\config{OPT\_C3SURF\_TRAFFIC}{OPT\_C3SURF\_TRAFFIC}{OPTC3SURFTRAFFIC}

   Paramètre par défaut~: \var{OPT\_C3SURF\_TRAFFIC='no'}

   Si vous indiquez 'yes' dans cette variable vous activez le module Traffic.
   Les variables de configuration sont décrites ci-dessous. Les valeurs par défaut
   ont été paramétrées avec une connexion DSL-6000, cela donne (6016 kbit/s descendant
   et 512 kbit/s montant).

   Vous devez définir les variables suivantes pour un volume de données qui ne doit pas
   être dépassée. Aucune distinction n'est faite entre l'upload et download. La logique
   de ce module est conçu de manière à ce que si le volume de données est dépassé deux
   fois de suite - le délinquant sera bloqué avec une pénalité de temps défini (temps de bloquage).
   Ces paramètres sont appliqués globalement à tous les utilisateurs C3SURF. Pour choisir
   les bons paramètres, cela devrait dépendre de la bande passante disponible sur le site.
   Aucun bloquage ne résultera lors d'un dépassement de téléchargement exceptionnel, vous pouvez
   même faire normalement une mise à jour du système d'exploitation ou de télécharger une grandes
   quantités de données. Mais si la consommation de bande passante est reconnu comme "permanente"
   le bloquage devient actif. Il s'agit avec ce module de blâmer ceux qui utilise longs temps
   la bande passante.

   Si vous voulez, par exemple, pour permettre le télécharger occasionnel de grandes quantités de données,
   un temps doit être calculée par rapport au montant de données autorisé et à la bande passante disponible
   avec laquel la quantité de données peut être téléchargé.

\achtung{Exemple~:}

   Si vous voulez télécharger occasionnelle d'un CD de distribution de (700Mo), au meilleur des cas cela
   prendra le temps suivant pour le terminer~:

   \begin{tabular}{lrlrl}

    DSL-&1000 & environ & 93 & Minutes \\
    DSL-&2000 & environ & 47 & Minutes \\
    DSL-&6000 & environ & 16 & Minutes \\
    DSL-&16000 & environ & 6 & Minutes \\

   \end{tabular}

   Le montant autorisé de données à télécharger est ici (de 700 Mo) il doit être divisé par une valeur
   supérieur à 1 mais inférieure à 2.

\achtung{Exemple (conservative)~:} 700Mo / 1,9 = 386317473 Octets \\
   C'est le nombre d'octets qui peut être téléchargés au maximum dans un temps, calcul ci-dessus.
   S'il est logique de permettre un tel volume de données élevé par utilisateur pour la DSL-1000
   et DSL-2000 cela dépend aussi du nombre d'utilisateurs.

   Si vous ne souhaitez pas une telle quantités de données à télécharger, mais par exemple permettre
   l'écoute de flux de musique mp3 ou de permettre un flux continu de données de 128 kBit/s, vous
   devez sélectionner les valeurs suivantes~: 16220160 octets pour 15 minutes
   (résultats pour 128kbit/s * 1024 / 8 bits = 16384 octets/s * 60 = 983040 octets/min * 15min = 14745600 octets * 1,1 = 16220160 octets (pour 15 min)).
   Comme il s'agit d'un flux de données continu qui ne peut pas être partagée, cette charge sera toujours
   autorisé. Ici, il est utile de calculer une marge supplémentaire de 10\% par sécurité, car, en plus
   de la quantité pure de données d'autres informations doivent aussi être transportés. Par conséquent,
   la valeur calculée de 14745600 octets sera multipliée par 1,1.

   Dans ce qui suit, les variables sont présentés avec des valeurs par défaut, avec l'exemple donné ici pour
   le téléchargement occasionnel d'un CD avec une connexion DSL 6000.

\config{C3SURF\_TRAFFIC\_BYTES}{C3SURF\_TRAFFIC\_BYTES}{C3SURFTRAFFICBYTES}

   \var{C3SURF\_TRAFFIC\_BYTES='386317473'}

   Plage de valeur~: nombre entier naturel

   Vous indiquez dans cette variable le nombre d'octets qui peut être téléchargé dans le temps maximum
   \jump{C3SURFTRAFFICMINUTES}{\var{C3SURF\_TRAFFIC\_MINUTES}}. Voici par exemple la 1,9 ème
   partie d'un CD de 700Mo. Pour l'exemple de flux de musique mp3 à 128kbit vous pouvez définir 16220160.

\config{C3SURF\_TRAFFIC\_MINUTES}{C3SURF\_TRAFFIC\_MINUTES}{C3SURFTRAFFICMINUTES}

   \var{C3SURF\_TRAFFIC\_BYTES='16'}

   Plage de valeur~: nombre entier naturel

   Vous indiquez dans cette variable le temps en minutes qui s'écoulera entre deux mesures de volume de données
   à télécharger. S'il est constaté que le temps est dépassement, le responsable sera enregistré (ou fiché)
   temporairement. Si se même responsable dépasse encore le temps lors de la prochaine mesure de téléchargement,
   il sera automatiquement déconnecté et bloqué (avec \jump{C3SURFTRAFFICBLOCKTIME}{\var{C3SURF\_TRAFFIC\_BLOCKTIME}} en minute).
   Si aucun excés n'est détectée dans la seconde mesure, l'enregistrement temporaire du responsable sera supprimé.

   Pour l'exemple du mp3, vous pouvez indiquer '15'.

\config{C3SURF\_TRAFFIC\_BLOCKTIME}{C3SURF\_TRAFFIC\_BLOCKTIME}{C3SURFTRAFFICBLOCKTIME}

   \var{C3SURF\_TRAFFIC\_BLOCKTIME='60'}

   Plage de valeur~: nombre entiers

   Vous indiquez dans cette variable le temps en minutes, pendant lequel l'utilisateur aura l'accès bloqué
   après avoir dépassé la limite du temps de téléchargement.

\end{description}