% Synchronized to r30467
\section {Interface Web fli4l}

\achtung{Les droits~:}
c3surf:view, admin
\begin{itemize}
    \item view~: Pour voir les entrées dans le menu admin
    \item admin~: Pour utiliser les fonctions de l'interface Web.

    Utilisateurs du httpd avec l'option "all" a tous les droits ici aussi.

    \item Vous pouvez trouver OPT C3SURF dans l'interface Web sous "Opt" et "c3Surf".
% HB 13.09.09
%     \item Zu finden in Web-Admin unter Opt:\\
%  \begin{itemize}
%     \item Als \dq c3Surf\dq , wenn \jump{OPTLOGINUSR}{\var{OPT\_LOGINUSR='no'}} (sprich: FreeSurf)
%     \item Als \dq LoginUsr\dq, wenn \jump{OPTLOGINUSR}{\var{OPT\_LOGINUSR='yes'}}
%  \end{itemize}
\end{itemize}

\section {Fonctionnement}

Vous pouvez spécifier un réseau ou des hôtes individuellement, ils sont bloqués
après le démarrage du système. les utilisateurs peuvent ensuite être activés et
utiliser l'interface web pour un laps de temps défini.

\parskip 12pt

Si l'option LOGINUSR est activée, seules les ordinateurs ayant un nom d'utilisateur 
et mot de passe valide seront autorisé à se connectés.
\parskip 12pt

Dans l'interface Web admin des utilisateurs du routeur ou les addesses MAC peut être
affichée, ils pourront être déconnecté ou bloqué, soit de façon permanente ou selon
un temps défini. Ces bloquages sont valable que s'il sont enregistrés et gérés par c3Surf.
Si l'ordinateur est connecté via une autre carte réseau sur votre routeur, le bloquage
n'a pas d'effet.
\parskip 12pt

L'enregistrement se fait par nom, prénom et adresse e-mail ou par User/Passwort.
Après un temps défini, l'accès sera bloqué, il sera réactivée que par une nouvelle
connexion.
\parskip 12pt

Vous pouvez voir la page de connexion des utilisateurs bloqués dans 
(voir FreeSurf et LoginUsr dans le menu à gauche de l'OPT de l'interface Web).
\parskip 12pt

Toutes ces opérations font partie de l'interface web admin.
\parskip 12pt

Vous pouvez aussi débloquer certain ordinateur de façon permanente,
via l'interface web. Voir la liste ARP, ou les baux du DHCP.

\parskip 12pt

Avec la version 2.1.0, tous les utilisateurs peuvent avoir une restriction avec
un quotas. Cela limite le temps d'utilisation avec les paramètres
"-TIME", "-BLOCKTIME" et "-COUNTER" plusieurs options sont disponibles.

\underline{Exemple~:}\\

\begin{tabular}{|c|c|c|p{12cm}|}
 \hline
   Time & BLocktime & Counter & Quota, si C3SURF\_QUOTA='yes' \\
 \hline
 \hline
   60 & -1 & 0 & 60 Min temps d'utilisation, pas de bloquage après, à chaque enregistrement
   le temps sera décompté (c'est le principe de l'horodateur) \\
 \hline 
   60 & 240 & 0 & 60 Min temps d'utilisation, sera bloqué dans 240 min, à chaque enregistrement
   le temps sera décompté (horodateur = argent, le temps passe sans possibilité de l'interrompre) \\
 \hline
   60 & 0 & -1 & 60 Min temps d'utilisation, après expiration sera bloqué jusqu'à 0h00, avec
   un forfait illimité de connexion et déconnexion (pas horodateur) \\
 \hline
   60 & -1 & 1 & 60 Min temps d'utilisation, pas de bloquage à l'expiration du temps,
   le temps peut être interrompu 1x \\
 \hline
   60 & -1 & -1 & 60 Min temps d'utilisation, pas de bloquage à l'expiration du temps
   un certain nombre d'interruptions est possibles \\
 \hline
   600 & 10080 & -2 & 10 Heures dans une semaine avec autant d'interruption que vous le souhaitez \\
 \hline
   0 & -1 & 0 & Le temps est infini à chaque ouverture de session, pas de bloquage après le temps \\
 \hline
\end{tabular}

\section{Résolution de noms - DNS}

\achtung{Important~: fli4l doit être configuré en tant que serveur DNS pour les clients,
		il doit être en mesure d'effectuer la résolution de noms. Pour cela}\\    
          \begin{itemize}
                \item \emph{il a besoin du "Forward" pour accéder à un serveur DNS sur le réseau ou}
                \item \emph{il est serveur DNS, il pourra établir une connexion si nécessaire.}
          \end{itemize}

               \emph{Sinon, il y aura des problèmes de redirection automatique vers la page de connexion.
                   Mais, elle peut être consultée manuellement en tapant leur URL.}
