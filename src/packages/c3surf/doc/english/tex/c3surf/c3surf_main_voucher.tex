% Synchronized to r30494
OPT\_C3SURF\_VOUCHER allows anonymous access to the Internet. There are vouchers
created which can be arranged in various categories. The opt can then be managed manually
or automatically on the Web Interface.

\section {Configuration Of OPT\_VOUCHER}

\begin{description}
\config {OPT\_C3SURF\_VOUCHER}{OPT\_C3SURF\_VOUCHER}{OPTC3SURFVOUCHER}

  \var{OPT\_C3SURF\_VOUCHER='no'}

  Activate the voucher system of opt C3SURF\_VOUCHER ('yes'), default is 'no'. Vouchers are
  anonymous yet secure single-use accounts that can be used for login.
  Requirement is the setting \jump{OPTLOGINUSR}{\var{OPT\_LOGINUSR='yes'}}.

  The creation and deletion of the vouchers is done by two nightly cron jobs which also can
  be started manually at any time (admin interface). The following explains how to manage these jobs.

  All newly generated vouchers are attached to a print list. Only in the print list the password
  corresponding to the voucher is stored in plain text. You can download, print or delete this
  list at any time. After deleting the list, the password can not be recovered again. Normally
  the list is printed first and then deleted. There should only exist one printed copy of a voucher.
  the print function is implemented in html, page feeds are not taken into account. Vouchers which
  are not printed caused by a page feed should be destroyed (though they get invalid due to expiration
  anyway). Lists that were not printed but downloaded may get an own layout with other programs which
  also avoids the page feed caveat.

\config {C3SURF\_VOUCHER\_N}{C3SURF\_VOUCHER\_N}{C3SURFVOUCHERN}

   \var{C3SURF\_VOUCHER\_N='n'}

  Value range: 0 and natural numbers

  How many different voucher categories should be produced? The most important criterion
  for vouchers is the runtime. Next to it are the number of vouchers and their validity in days.
  See also the following variables.

\config {C3SURF\_VOUCHER\_x\_TIME}{C3SURF\_VOUCHER\_x\_TIME}{C3SURFVOUCHERxTIME}

   \var{C3SURF\_VOUCHER\_x\_TIME='30'}

  Value range: natural numbers

  Duration in minutes (here: 30) for a voucher of this category ('n' see above).

\config{C3SURF\_VOUCHER\_x\_COUNT}{C3SURF\_VOUCHER\_x\_COUNT}{C3SURFVOUCHERxCOUNT}

   \var{C3SURF\_VOUCHER\_x\_COUNT='3'}

  Value range: natural numbers

  How many vouchers of this category (in this case 3) should be produced?

\config{C3SURF\_VOUCHER\_x\_DAYS}{C3SURF\_VOUCHER\_x\_DAYS}{C3SURFVOUCHERxDAYS}

   \var{C3SURF\_VOUCHER\_x\_DAYS='90'}

  Value range: 0 and natural numbers

  How many days do you want the voucher to be valid starting from its generation (here: 90).
  Thus, an expiration date for this coupon is generated. The deletion is then carried out
  either manually or via cron job. The voucher is void when it is first used.

  \wichtig{'0' means that vouchers of this category have no expiration date. They only
	   become invalid with use or if the time has been completely consumed (also affected by
           C3SURF\_VOUCHER\_LIVES\_N). However they may be deleted at any time in the admin interface.}

\end{description}

\subsection {Optional Parameters Of OPT\_VOUCHER}

\begin{description}

\config{C3SURF\_VOUCHER\_x\_LIVES}{C3SURF\_VOUCHER\_x\_LIVES}{C3SURFVOUCHERxLIVES}

   \var{C3SURF\_VOUCHER\_x\_LIVES='n'}

   Value range(s): -1, 0, natural numbers

   Number of hours, in which the voucher is still valid after your first login.

\achtung{Special cases:}

\begin{itemize}
   \item{C3SURF\_VOUCHER\_x\_LIVES='-1'}

         valid until the expiration date originally generated with C3SURF\_VOUCHER\_DAYS
   \item{C3SURF\_VOUCHER\_x\_LIVES='0'}

         (Standard), means voucher will become invalid with the first login.
    \item{C3SURF\_VOUCHER\_x\_LIVES='Natural number'}

	  Number of hours for which voucher is still valid after the first login - calculate a new
          expiration date if necesssary.
\end{itemize}
\parskip 12pt

  These vouchers will not become invalid with the first login, but are valid for 'n' more hours. Once the voucher
  is used, a time-limited LOGINUSR account is generated or the expiry date of the voucher is recalculated.
  This account / voucher may login and logout for any number of times. The quota system of LOGIN\_USR is used for
  this account. If the total time or the expiry date (C3SURF\_VOUCHER\_DAYS\_N) is reached, C3SURF
  will automatically delete this account.

\config{C3SURF\_VOUCHER\_DEL\_CRON}{C3SURF\_VOUCHER\_DEL\_CRON}{C3SURFVOUCHERDELCRON}

   \var{C3SURF\_VOUCHER\_DEL\_CRON='0 4 * * *'}

  Value range: 'cron-Syntax' or 'never'

  The above value is the default if the variable is missing in the config file 'c3surf.txt'.
  Default: delete all expired vouchers every morning at 4 o'clock.

  Cron syntax must be obeyed and will not be verified. The value 'never' may be used in addition.
  Then the job is not scheduled by the system. In the admin interface all expired vouchers may be
  deleted manually at any time.

\config{C3SURF\_VOUCHER\_GEN\_CRON}{C3SURF\_VOUCHER\_GEN\_CRON}{C3SURFVOUCHERGENCRON}

  \var{C3SURF\_VOUCHER\_GEN\_CRON='15 4 * * *'}

  Value range: 'cron-Syntax' or 'never'

  The above value is the default if the variable is missing in the config file 'c3surf.txt'.
  Default: generate new vouchers daily at 4:15 AM if less than\\ C3SURF\_VOUCHER\_COUNT exist.

  Cron syntax must be obeyed and will not be verified. The value 'never' may be used in addition.
  Then the job is not scheduled by the system. In the admin interface new vouchers may be generated
  manually at any time, up to the amount defined in \jump{C3SURFVOUCHERxCOUNT}{\var{C3SURF\_VOUCHER\_x\_COUNT}}.

  All newly generated vouchers are attached to a print list. Only in the print list the password
  corresponding to the voucher is stored in plain text. Each voucher should be printed only once.
  The list should be deleted immediately after printing or downloading.

\config{C3SURF\_VOUCHER\_PRTUPDATE}{C3SURF\_VOUCHER\_PRTUPDATE}{C3SURFVOUCHERPRTUPDATE}

  Default Setting: \var{C3SURF\_VOUCHER\_PRTUPDATE='no'}

  Value range: 'yes' or 'no'

  Update of the print file. My recommendation: 'no'. If only a few vouchers are held in the system and
  the print file should not be deleted after printing or downloading, an update of the print file can be
  specified with 'yes' when vouchers are used. In the case of 'yes' used vouchers are also deleted from
  the print file. This requires resources on the router.

\config{C3SURF\_VOUCHER\_USRLEN}{C3SURF\_VOUCHER\_USRLEN}{C3SURFVOUCHERUSRLEN}

  Default Setting: \var{C3SURF\_VOUCHER\_USRLEN='12'}

  Value range: '1-16'

  Set character length for voucher account, from 8 characters on '-' will be filled in as separators,
  which also must also be entered. There are always four characters grouped. The maximum value is 16.

\config{C3SURF\_VOUCHER\_USRCAP}{C3SURF\_VOUCHER\_USRCAP}{C3SURFVOUCHERUSRCAP}

  Default Setting: \var{C3SURF\_VOUCHER\_USRCAP='random'}

  \begin{tabular}{rlrl}
   -&'yes'&:&only capital letters \\
   -&'no'&:&all lowercase\\
   -&'random'&:&random change of upper and lower case (recommended)\\
  \end{tabular}

  This variable determines whether uppercase or lowercase letters should be used in the user name.
  The value \var{'random'} (recommended) causes a random selection.

\config{C3SURF\_VOUCHER\_PWDLEN}{C3SURF\_VOUCHER\_PWDLEN}{C3SURFVOUCHERPWDLEN}

  Default Setting: \var{C3SURF\_VOUCHER\_PWDLEN='6'}

  Value range: 1-12

  Character length for the voucher password.

\config{C3SURF\_VOUCHER\_PWDMOD}{C3SURF\_VOUCHER\_PWDMOD}{C3SURFVOUCHERPWDMOD}

  Default Setting: \var{C3SURF\_VOUCHER\_PWDMOD='3'}

  Value range: 1-5

  Modulo for random extensions of the password. Max: 5 (the values 0, 1, 2, 3, 4),
  Min 1 (the value 0). It is used randomly in the password generation to change the
  possible values. This results by default in password lengths from 6 to 8 characters.
  The maximum are password-lengths between 12 and 16 characters, in conjunction with
  random upper- and lowercase letters this is considered as safe.

\config{C3SURF\_VOUCHER\_PWDCAP}{C3SURF\_VOUCHER\_PWDCAP}{C3SURFVOUCHERPWDCAP}

   Default Setting: \var{C3SURF\_VOUCHER\_PWDCAP='random'}

  \begin{tabular}{rlrl}
   -&'yes'&:&only capital letters \\
   -&'no'&:&all lowercase\\
   -&'random'&:&random change of upper and lower case (recommended)\\
  \end{tabular}

  This variable determines whether uppercase or lowercase letters should be used in the password.
  The value \var{'random'} (recommended) causes a random selection.

\end{description}
