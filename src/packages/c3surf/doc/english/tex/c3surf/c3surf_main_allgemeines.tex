% Synchronized to r30467
\section {fli4l Web Interface}

\achtung{Rights:}
c3surf:view, admin
\begin{itemize}
    \item view: Show the entry in Admin menu
    \item admin: To use functions in the web interface.

    Httpd users with right ``all'' have all rights here too.

    \item OPT C3SURF can be found in the Web interface at ``Opt'' -> ``c3Surf''.
\end{itemize}


\section {Operation}
With c3surf single hosts or complete nets may be defined that are blocked
after system boot then. By an informal registration via the web interface
users can unlock circuits for a defined time span.

\parskip 12pt

If the option LOGINUSR is activated, only persons with a valid user name
and password will be allowed to log in.
\parskip 12pt

In the Admin interface of the router users or MAC addesses may be displayed,
logged out or blocked, either permanently or time based. These locks are only
vaild for the registration with c3Surf and are managed there. The lock is
meaningless if the PC in question gains access to the router via another interface.
\parskip 12pt

The registration is accomplished with first-, lastname and E-Mail address or via
user/password. After a defined time the access is blocked again and has to be
reactivated by a new login.
\parskip 12pt

The registration page may also be locked for the users (see FreeSurf resp.
LoginUsr in the OPT menue of the Web interface).
\parskip 12pt

All this operations are part of the Admin menu of the Web interface.
\parskip 12pt

If a defined host should be unblocked permanently this may be done in
the Web interface (see the ARP lists or DHCP leases).
\parskip 12pt

All usage can be restircted by Quotas. This restricts use time and by the
parameters ``-TIME'', ``-BLOCKTIME'' and ``-COUNTER'' a variety of
configuration options are available.

\underline{Examples:}\\

\begin{tabular}{|c|c|c|p{9cm}|}
 \hline
   Time & Blocktime & Counter & Quota (\var{C3SURF\_QUOTA='yes'})\\
 \hline
 \hline
   60 & -1 & 0 &  60 Min use time, No lock after timeout, With every registration the time is running (parking meter principle)\\
 \hline
   60 & 240 & 0 & 60 Min use time, after timeout locked for 240 mins., With every registration the time is running (parking
    meter principle= Money in, no chance of interception)\\
 \hline
   60 & 0 & -1 &  60 Min use time, after timeout locked until 00:00 o'clock,
   any number of logins and logouts possible (no parking meter principle) \\
 \hline
   60 & -1 & 1 &   60 Min use time, no lock after timeout, time may be intercepted once \\
 \hline
   60 & -1 & -1 &  60 Min use time, no lock after timeout, as many interceptions as you like \\
 \hline
   600 & 10080 & -2 & 10 hours within a week with as many interceptions as you like \\
 \hline
   0 & -1 & 0 &   Endless use time with each registration, no lock after timeout \\
 \hline
\end{tabular}

\section{Name Resolution - DNS}

\achtung{Important: fli4l has to be entered as the DNS-Server on all clients and must be able to perform
	      name resolution. This either}\\
          \begin{itemize}
                \item \emph{needs a ``Forward'' to the DNS server of the net or}
                \item \emph{fli4l itself is the DNS server and may establish connections if needed.}
          \end{itemize}

          \emph{Otherwise problems arise to redirect to the registration page automatically. But it may also be
	  accessed manually by specifying its URL in the browser.}
