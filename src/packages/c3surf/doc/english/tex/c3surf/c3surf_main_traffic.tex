% Synchronized to r30467
  OPT\_C3SURF\_TRAFFIC makes it possible to throttle ``Power Users''.
  The data volume in a defined time interval will be monitored and evaluated. The
  configuration can be customized according to your needs.

\section {Configuration Of OPT\_C3SURF\_TRAFFIC}

\begin{description}

\config{OPT\_C3SURF\_TRAFFIC}{OPT\_C3SURF\_TRAFFIC}{OPTC3SURFTRAFFIC}

  Default Setting: \var{OPT\_C3SURF\_TRAFFIC='no'}

  Specifying 'yes' here activates the traffic module. The variables are described
  below. The defaults have been chosen with a DSL-6000 connection in mind.

  With the following variables can be set in which time which data volume should not be exceeded.
  No distinction is made between upload and download. The logic of the module is designed in a way
  that if the volume is exceeded twice in a row - the offender will be blocked and this is done with
  the defined time penalty (block time). These settings are applied globally for all C3SURF users.
  Choosing the right parameters should depend on the bandwidth available on site. Since no lock at
  a single occurance is made, even an operating system update or the normal download of larger amounts
  of data is accepted. But if the bandwidth consumption is recognized as ``permanently'' a lock will
  become active.

  If you want, for example, to allow the occasional downloading of large amounts of data a time has to be
  calculated out of the allowed amount of data and the available bandwidth in which the amount of data
  can be downloaded.

\achtung{Example:}

   Downloading a distribution CD (700MB) would at best
   case take the following time to complete:

   \begin{tabular}{lrlrl}

    DSL-&1000 & approximately & 93 & Minutes \\
    DSL-&2000 & approximately & 47 & Minutes \\
    DSL-&6000 & approximately & 16 & Minutes \\
    DSL-&16000 & approximately & 6 & Minutes \\

   \end{tabular}

   The allowed amount of data (here 700MB) should be divided by a value smaller (but near to) 2 and bigger than 1.

   \achtung{Example (conservative):} 700MB / 1,9 = 386317473 Bytes \\
   That would be the number of bytes that may be downloaded as a maximum in the time calculated above. Whether it
   makes sense to allow such a high volume per user for DSL-1000 and DSL-2000 depends  also on the number of users
   expected.

   If you don't want to allow such amounts of data, but for example allow listening to mp3 streams of music or
   allow a continuous data stream of 128 kbit/s, you should select the following values: 16220160 bytes per 15
   minutes (results from 128kBit/s * 1024 / 8Bit = 16384 Bytes/s * 60 = 983040 Bytes/min * 15min = 14745600 Bytes
   * 1,1 = 16220160 Bytes (per 15 min)). Since this is a continuous data stream no devision should take place
   because this load is always permitted. Here it is useful to calculate a further 10\% safety margin, since in
   addition to the pure amount of data other information must be transported too. Therefore, the calculated
   Value of 14745600 bytes is multiplied by 1.1.

   In the following, the variables are presented with default values for the example given here for the
   occasional download of a CD with a DSL 6000 connection.

\config{C3SURF\_TRAFFIC\_BYTES}{C3SURF\_TRAFFIC\_BYTES}{C3SURFTRAFFICBYTES}

   \var{C3SURF\_TRAFFIC\_BYTES='386317473'}

   Value range: natural numbers

   Specifies the number of bytes which may be downloaded in the maximum time \jump{C3SURFTRAFFICMINUTES}
   {\var{C3SURF\_TRAFFIC\_MINUTES}}. Here for example the 1,9th part of a 700MB CD. For the example of mp3 music
   streams at 128kBit set this to 16220160.

\config{C3SURF\_TRAFFIC\_MINUTES}{C3SURF\_TRAFFIC\_MINUTES}{C3SURFTRAFFICMINUTES}

   \var{C3SURF\_TRAFFIC\_BYTES='16'}

   Value range: natural numbers

   Specifies the time in minutes that elapses between two data volume measurements. If, after the
   elapsed time an excess is found here, the responsible party is first temporarily stored. If again
   found to be exceeded during the next measurement, it is automatically logged out and blocked
   (for \jump{C3SURFTRAFFICBLOCKTIME}{\var{C3SURF\_TRAFFIC\_BLOCKTIME}} minutes). If no exceeding
   is detected in the second measurement, the temporary storage is deleted.

   For the mp3 example set '15' here.

\config{C3SURF\_TRAFFIC\_BLOCKTIME}{C3SURF\_TRAFFIC\_BLOCKTIME}{C3SURFTRAFFICBLOCKTIME}

   \var{C3SURF\_TRAFFIC\_BLOCKTIME='60'}

   Value range: natural numbers

   Specifies the time in minutes for that access is blocked after exceeding the traffic limits.

\end{description}
