% Synchronized to r30843
\section{Introduction}

With C3SURF you can create an open, unencrypted network/WiFi. For legal reasons,
however, you should control who uses the network. C3SURF allows an informal
registration to the network. The package is based on ``opt\_onco'' (Copyright
(c) 2001-2007 Michael Mattes). By using OPT\_LOGINUSR an ``almost'' real login
can be realized. C3SURF can generate vouchers and a rudimentary (experimental)
control function to suppress excessive downloads is integrated as well.

It may be defined, which hosts or entire networks are managed by C3SURF. These
are initially blocked on router startup and the http requests are sent to the C3SURF
login page. After registering an account on the login page, usage is allowed on a time basis.
Everything will be logged and can be controlled via the web interface of C3SURF.

\section {Notes On Installation}

\begin{itemize}
\item As always with opt-packets for the fli4l:
  \begin{itemize}
     \item unpack opt\_c3surf\_$<$version$>$.tar.gz to the fli4l directory (Build-PC).
     \item adapt c3surf.txt to your own needs.
     \item add the rights 'c3surf:view,admin' to httpd.txt (if needed).
     \item generate a new fli4l build.\\
\end{itemize}
\achtung{Important: fli4l has to be configured as the DNS server on all clients and should be
         able to perform name resolution. To accomplish this}\\
          \begin{itemize}
                \item \emph{a ``Forward'' to the DNS server of the net is needed or}
                \item \emph{fli4l itself is the DNS server and may establish connections automatically.}
          \end{itemize}

	\emph{Otherwise problems may arise to redirect requests to the login page. In all cases the
	page may also be accessed by entering its URL.}

\section {Configuration OPT\_C3SURF}
\begin{description}

\config {OPT\_C3SURF}{OPT\_C3SURF}{OPTC3SURF}

  Default Setting: \var{OPT\_\-C3SURF='no'}

  Activate or deactivate the package.

\config {C3SURF\_LOG\_PATH}{C3SURF\_LOG\_PATH}{C3SURFLOGPATH}

  Default Setting: \var{C3SURF\_LOG\_PATH='/var/log/c3surf'}

  Defines the directory for C3SURF's log files. On router shutdown
  the log files should be saved to a persistent medium or the path
  can be set here, if you want to keep the files. The path must exist
  on the permanent medium.

  \wichtig{'c3surf\_mac.blacklist' resides in the persistent directory\\
  \jump{C3SURFPERSISTENTPATH}{\var{C3SURF\_PERSISTENT\_PATH}}. An own
  blacklist has to be copied there. The scope of the protocol is defined
  below.}

\config {C3SURF\_DOLOG\_LOGIN}{C3SURF\_DOLOG\_LOGIN}{C3SURFDOLOGLOGIN}

  Default Setting: \var{C3SURF\_DOLOG\_LOGIN='yes'}

  Logging of Login/Logout: c3surf\_login.log (default: 'yes')

\config {C3SURF\_DOLOG\_INVALID}{C3SURF\_DOLOG\_INVALID}{C3SURFDOLOGINVALID}

  Default Setting: \var{C3SURF\_DOLOG\_INVALID='yes'}

  Logging of invalid logins: c3surf\_invalid.log (default: 'yes'). If
  \jump{OPTLOGINUSR}{\var{OPT\_LOGINUSR}} is set to 'yes', invalid logins
  can not be logged.

\config {C3SURF\_DOLOG\_PAGE}{C3SURF\_DOLOG\_PAGE}{C3SURFDOLOGPAGE}

  Default Setting: \var{C3SURF\_DOLOG\_PAGE='no'}

  Logging of accesses to the html page: c3surf\_page.log (default: 'no').
  Each access to the login page will be logged. The page log grows
  fast and thus is only recommended for the ``curious''.

\config {C3SURF\_DOLOG\_HTTPD}{C3SURF\_DOLOG\_HTTPD}{C3SURFDOLOGHTTPD}

  Default Setting: \var{C3SURF\_DOLOG\_HTTPD='no'}

  Logging of all accesses to mini\_httpd: c3surf\_httpd.log (default: 'no').

  \wichtig{In addition start the log function of the Mini-httpd (use only for tests or
  debugging). When turned on, it is advisable to regularly check the log and delete it,
  it quickly gets quite large.}

  opt\_cpmvrmlog: \altlink{http://extern.fli4l.de/fli4l_opt-db3/search.pl?pid=427}
  may be used for regular saving. The mini\_httpd has to be restarted afterwards
  for correct logging. The script \var{/usr/local/bin/c3surf\_kill\_httpd.sh}
  \jump{C3SURFLOG}{\var{(Config example in appendix) exists for this purpose.}}

\config {C3SURF\_PERSISTENT\_PATH}{C3SURF\_PERSISTENT\_PATH}{C3SURFPERSISTENTPATH}

  Adapt in any case, recommendation: '/var/lib/persistent/c3surf'

  Defines the directory for files that should be preserved after a reboot or poweroff.
  Ideally, this points to a hard disk or CF card ('/var/lib/persistent/c3surf'). A directory in
  the RAM disk may be selected as well (eg in order to minimize access to the medium).
  In this case the directory should be copied to the disk from time to time (eg by
  opt\_cpmvrmlog) because the data would be lost after a reboot, crash or power loss.

  \achtung{What is saved here:}
\newpage
  \achtung{MAC blacklists:}

  'c3surf\_mac.blacklist', will be created when needed (see Admin
  Interface). Blocking of a MAC address is solved via an own file
  and not with the packet filter, because large amounts of entries may
  cause problemes there. Don't forget: blocked MAC addresses keep
  average users away from your net, which is enough for normal usecases,
  but not professionals. The MAC blacklist only prevents the login via
  C3SURF / loginusr, there are no direct blocks in the firewall.

  \achtung{User data:}

  $<$userloginname$>$.data (i.e. 'frank.data'), these files contain data on the
  user such as first and lastname, E-mail address, statistics and quotas. Persistant
  user data allows to avoid recreating user data files on every startup.
  This means: if for user ``frank'' a file 'frank.data' exist on system start
  the settings in the config file will be ignored.
\parskip 12pt

  Overwriting of user data may be forced by \jump{LOGINUSRACCOUNTxOVERWRITE}{\var{LOGINUSR\_ACCOUNT\_x\_OVERWRITE='yes'}}.
  By \jump{LOGINUSRDELETEPERSISTENTDATA}{\var{LOGINUSR\_DELETE\_PERSISTENT\_DATA='yes'}},
  all ``*.data'' files will be deleted on reboot.

\config {C3SURF\_WORKON\_TMP}{C3SURF\_WORKON\_TMP}{C3SURFWORKONTMP}

  Default Setting: \var{C3SURF\_WORKON\_TMP='no'}

  If \jump{C3SURFPERSISTENTPATH}{\var{C3SURF\_PERSISTENT\_PATH}}
  is set, you may specify 'yes' here. On system start persistent data will
  then be copied from the harddisk to the directory C3SURF\_TMP\_PATH
  and only be accessed there. Accesses to the harddisk by C3SURF only will
  occur if the admin writes data to persistent files.

 \wichtig{Persistant data is:}
  \begin{itemize}
  \item \emph{User accounts}
  \item \emph{MAC blacklist}
  \item \emph{System lock files (blocking of logins)}\\
  \end{itemize}

  \emph{For FLASH memory specify 'no' here, because in normal use C3SURF will
  only read files. Write accesses are only caused by the admin.}

\config {C3SURF\_QUOTA}{C3SURF\_QUOTA}{C3SURFQUOTA}

  Default Setting: \var{C3SURF\_QUOTA='no'}

  If the access should be limited, enter 'yes' here. Access is blocked for an IP
  address for \jump{C3SURFBLOCKTIME}{\var{C3SURF\_BLOCKTIME}} minutes after
  reaching time limit or the maximum registration counter. Default value is 'yes'.

  \wichtig{Individual -TIME, -BLOCKTIME and -COUNTER for LOGIN\_USR accounts are
  activated ('yes') or deactivated ('no') by this variable.}

\config {C3SURF\_COUNTER}{C3SURF\_COUNTER}{C3SURFCOUNTER}

  Default Setting: \var{C3SURF\_COUNTER='0'}

  Specifies the number of possible interruptions within the
  surftime.

  \wichtig{A multitude of interruptions for (Logout/Login) may be defined. If i.e. '1'
  is specified here the user may logout and login once within the surftime which corresponds to
  two registrations in this time. On the following registration the user gets the
  time difference left from \jump{C3SURFTIME}{\var{C3SURF\_TIME}}.}

  \emph{If in addition \jump{C3SURFBLOCKTIME}{\var{C3SURF\_BLOCKTIME='0'}} is set the
  \jump{C3SURFCOUNTER}{\var{C3SURF\_COUNTER}} will be reset at 0:00 o'clock the following day.}

  \begin{itemize}
    \item{With C3SURF\_COUNTER='0'}

    the value corresponds to the parking meter principle (money in, money gone, time runs: no interruption
    possible).

    \item{With C3SURF\_COUNTER= '-1'}

    this function is deactivated = as many interruptions of surftime as you like are possible.

    \item{With C3SURF\_COUNTER='-2'}

    as many interruptions of surftime as you like are possible (like '-1'), but blocking time countdown
    starts with the first login. In contrary to '-1', where blocking time starts after using the whole time.
    Since blocking time countdown runs simultaneously, the user will only be blocked when he consumes his
    or her quota too quickly.

  \end{itemize}

  Notes to the long-term contingent (C3SURF\_COUNTER='-2'):\\
  Hence you may combine i.e. 10 hours of online time {C3SURFTIME}{\var{(C3SURF\_TIME='600')}} with
  a blocking time of a week \jump{C3SURFBLOCKTIME}{\var{(C3SURF\_BLOCKTIME='10080'}} : 60sec x 24h x 7days).
  This way the 10 hours may be used during one week. Those using all the time on the first day in one piece
  will have to wait for the rest of the week. After the blocking time 10 hours will be provided anew.
  Short: The user may use ten hours in one week, which he may spread in a meaningful manner over this timespan.

  If the quota is not used within a week, no ``Quota-Block'' will occur. Then there is no waiting time.
  If the quota is used on the first day then the account is blocked for the remaining 6 days of the week.
  Applies also to \jump{LOGINUSRACCOUNTxCOUNTER}{\var{LOGINUSR\_ACCOUNT\_x\_COUNTER}}.

  Recommendation: \jump{C3SURFSAVEQUOTA}{\var{C3SURF\_SAVE\_QUOTA='yes'}}, to retain the values also after
  a normal reboot. On a power failure the values will be lost.

  If \jump{C3SURFQUOTA}{\var{C3SURF\_QUOTA='yes'}}, after reaching the counter a block
  corresponding to \jump{C3SURFBLOCKTIME}{\var{C3SURF\_BLOCKTIME}} is activated.

\config {C3SURF\_TIME}{C3SURF\_TIME}{C3SURFTIME}

  Default Setting: \var{C3SURF\_TIME='60'}

  Number of minutes that an activation is valid.

  The value '0' means an unlimited login (also applies for LOGINUSR\_ACCOUNT\_x\_TIME).

\underline{Special case:}
\begin{itemize}
  \item{C3SURF\_TIME='0'}

  Means unlimitited online time. The user himself should perform the logout.
  C3SURF will only logout the user if the computer was shut down and\\
  \jump{C3SURFCHECKARP}{\var{C3SURF\_CHECK\_ARP='yes'}} (default) was set.
\end{itemize}

\config {C3SURF\_BLOCKTIME}{C3SURF\_BLOCKTIME}{C3SURFBLOCKTIME}

  Default Setting: \var{C3SURF\_BLOCKTIME='240'}

  Number of minutes an IP gets blocked if surftime was exceeded or the Admin performs
  a block via the Web interface. By this a computer may be blocked from the net for this
  time and thus usage is restricted. \jump{C3SURFQUOTA}{\var{C3SURF\_QUOTA='yes'}} has to
  be set in order to perform the block.

\underline{Special cases:}\\
\begin{itemize}
  \item{C3SURF\_BLOCKTIME='0'}

	a block of the addresse resp. the user is preformed until 00:00 o'clock
	of the following day.
    \item{C3SURF\_BLOCKTIME='-1'}

	no blocking will be performed.
  \end{itemize}

    \wichtig{Unblocking is performed with an accuracy of one minute.}

\config {C3SURF\_SAVE\_QUOTA}{C3SURF\_SAVE\_QUOTA}{C3SURFSAVEQUOTA}

  Default Setting: \var{C3SURF\_SAVE\_QUOTA='yes'}

  Saves Quota values on shutdown and restores them on systemstart of the router.
  The temporary files of the quota-management will be written to
  \jump{C3SURFPERSISTENTPATH}{\var{C3SURF\_PERSISTENT\_PATH}} on normal shutdown
  and will be restored to the temporary directory on system start again.
  All actual user data will be preserved this way. An accidental shutdown
  will not be recoverable this way.

  \wichtig{\jump{LOGINUSRDELETEPERSISTENTDATA}{\var{LOGINUSR\_DELETE\_PERSISTENT\_DATA='no'}},
  should be set because otherwise this setting will delete all user accounts und their quota data.}

\config {C3SURF\_CHECK\_ARP}{C3SURF\_CHECK\_ARP}{C3SURFCHECKARP}

  Default Setting: \var{C3SURF\_CHECK\_ARP='yes'}

  Check in the countdown module whether an IP of a computer has vanished from the ARP table.
  Shut down computers may be recognized this way, but sometimes with a massive time delay.

\config {C3SURF\_CONTROL\_HOST\_OR\_NET\_N}{C3SURF\_CONTROL\_HOST\_OR\_NET\_N}{C3SURFCONTROLHOSTORNETN}

 \var{C3SURF\_CONTROL\_HOST\_OR\_NET\_N='0'}

  Value: integer numbers.\\
  How much and which IP ranges or hosts should be controlled by c3Surf?
  This affects forwarding to another net (FORWARD Chain).

\config {C3SURF\_CONTROL\_HOST\_OR\_NET\_x}{C3SURF\_CONTROL\_HOST\_OR\_NET\_x}{C3SURFCONTROLHOSTORNETx}\ \\
 \var{C3SURF\_CONTROL\_HOST\_OR\_NET\_x='Netzwerk OR Host OR IP-Address'}

  Controls all clients.

  \wichtig{A complete net may be specified here for simplicity, e.g. WLAN.
	   Then all wireless users need to use the login page. Also a reference
	   to a host (@host) or an IP address may be specified. Who or what is
	   entered here is redirected to the login page and the blocking rules
	   defined below apply.}

\underline{Example:}

\begin{example}
\begin{verbatim}
C3SURF_CONTROL_HOST_OR_NET_1='IP_NET_3'       # Specify the net IP/MASK
C3SURF_CONTROL_HOST_OR_NET_2='@T8200'         # or host @HOST
C3SURF_CONTROL_HOST_OR_NET_3='192.168.13.11'  # or IP address
\end{verbatim}
\end{example}

The next example is basically the same as the one above (IP\_NET\_3) if in "base.txt" the
IP address has been set accordingly.

\begin{example}
\begin{verbatim}
C3SURF_CONTROL_HOST_OR_NET_1='192.168.0.1/24' # controls all clients
\end{verbatim}
\end{example}

For a computer to be excluded, you may either include all IP addresses
individually in C3SURF.txt (i.e. create a list of all 256 addresses
and leave one out), or you can use the CIDR notation (as above).
Then IP groups have to be used causing less writing (8 rows instead of 255).

\underline{This may look as follows:}

\begin{example}
\begin{verbatim}
C3SURF_CONTROL_HOST_OR_NET_N='8'                # Number of hosts or nets
C3SURF_CONTROL_HOST_OR_NET_1='192.168.0.0/31'   # 0-1
C3SURF_CONTROL_HOST_OR_NET_2='192.168.0.3'      # only 3 not 2
C3SURF_CONTROL_HOST_OR_NET_3='192.168.0.4/30'   # 4-7
C3SURF_CONTROL_HOST_OR_NET_4='192.168.0.8/29'   # 8-15
C3SURF_CONTROL_HOST_OR_NET_5='192.168.0.16/28'  # 16-31
C3SURF_CONTROL_HOST_OR_NET_6='192.168.0.32/27'  # 32-63
C3SURF_CONTROL_HOST_OR_NET_7='192.168.0.64/26'  # 64-127
C3SURF_CONTROL_HOST_OR_NET_8='192.168.0.128/25' # 128-255
\end{verbatim}
\end{example}

The computer with the IP '192.168.0.2' is able to do everything
allowed by fli4l's firewall without registration.

\config {C3SURF\_CONTROL\_PORT\_N}{C3SURF\_CONTROL\_PORT\_N}{C3SURFCONTROLPORTN}

   \var{C3SURF\_CONTROL\_PORT\_N='0'}

  Value: Integer numbers.\\
  How much TCP ports of the routers should be controlled?

  How much and which ports explicitely named should be controlled by c3Surf?
  IP ranges and hosts from above are affected.\\
  \jump{C3SURFCONTROLHOSTORNETN}{\var{C3SURF\_CONTROL\_HOST\_OR\_NET\_N}}. c3Surf
  controls these ports and frees them after successful login so that all services
  existing on this ports of the router may be used (INPUT-Chain).

\config {C3SURF\_CONTROL\_PORT\_x}{C3SURF\_CONTROL\_PORT\_x}{C3SURFCONTROLPORTx}

   \var{C3SURF\_CONTROL\_PORT\_x='port\_nr'}

  Port number and the access to the services of the router (fli4l) behind them
  are blocked until login. After successful registration, services can be used for
  the time provided.

\underline{Examples:}
\begin{example}
\begin{verbatim}
C3SURF_CONTROL_PORT_1='515' # i.e. lpdsrv (printer usable after login)
C3SURF_CONTROL_PORT_2='21'  # i.e. ftp - (note: ftp on the router!)
\end{verbatim}
\end{example}

\begin{example}
\begin{verbatim}
Other possible port adresses:
  21=ftp
  22=ssh
  5000=imonc
  5001=telmod
  8118=privoxy
  9050=tor
  3128=squid
  20000=mtgcapri
  80=http(Admin)
  515=lpdsrv
\end{verbatim}
\end{example}

All depends on your own configuration. To all ports not mentioned here
the rules from 'base.txt' apply. After registration, the rules of 'base.txt'
are still valid. c3Surf is only a pre-chain to these rules until the login
was performed successfully. So after registration all the rules are still
obeyed. So you may, for example, deny access from WLAN to the wired network
in 'base.txt'. This is also valid for users legitimated in WLAN by c3Surf.

\config {C3SURF\_BLOCK\_PORT\_N}{C3SURF\_BLOCK\_PORT\_N}{C3SURFBLOCKPORTN}

   \var{C3SURF\_BLOCK\_PORT\_N='0'}

  Value: Integer numbers.\\
  How much TCP ports of the routers should be blocked?

   \underline{Hints:}\\
   Permanent blocking of services for nets and hosts mentioned above\\
   \jump{C3SURFCONTROLHOSTORNETN}{\var{C3SURF\_CONTROL\_HOST\_OR\_NET\_N}}. How much
   and which ports explicitely named should be blocked permanently by c3Surf?
   No access to the router's services behind those ports for hosts and/or computers
   of the blocked nets even not after login. This affects the INPUT-Chain.
   If you want to block certain services permanently, you should better
   do this with the parameters for the INPUT chain in 'base.txt'.
   \underline{Why:}\\
   Because these rules are not valid anymore if the parameter
   \jump{OPTC3SURF}{\var{OPT\_C3SURF='no'}} is set. If you deactivate C3SURF
   the rules defined here have to be transferred to the 'base.txt' if you
   want your blocks for the hosts or nets mentioned above to persist.


\config {C3SURF\_BLOCK\_PORT\_x}{C3SURF\_BLOCK\_PORT\_x}{C3SURFBLOCKPORTx}

   \var{C3SURF\_BLOCK\_PORT\_x='port\_nr'}

\underline{Examples:}
\begin{example}
\begin{verbatim}
C3SURF_BLOCK_PORT_1='5000'           # z.B. imonc
C3SURF_BLOCK_PORT_2='5001'           # z.B. telmond
C3SURF_BLOCK_PORT_3='20000'          # z.B. mtgcapri (OPT_MTGCAPRI)
C3SURF_BLOCK_PORT_4='22'             # z.B. ssh
C3SURF_BLOCK_PORT_5='8118'           # z.B. privoxy (PROXY)
C3SURF_BLOCK_PORT_6='9050'           # z.B. tor (PROXY)
C3SURF_BLOCK_PORT_7='80'             # z.B. httpd Admin interface (HTTPD)
C3SURF_BLOCK_PORT_8='7437'           # z.B. caiviar (OPT_CAIVIAR)
\end{verbatim}
\end{example}

\config {C3SURF\_HTTPD\_PORT}{C3SURF\_HTTPD\_PORT}{C3SURFHTTPDPORT}

  Default Setting: \var{C3SURF\_HTTPD\_PORT='8080'}

  On which port and which IP address should the mini\_httpd listen for
  login attempts? http queries from computers will be redirected to
  this address and port. Port 8080 is the default here.

  \achtung{The following should be considered when choosing the port number:}
  \begin{itemize}
  \item You should not use the port from the httpd package (usually this is 80).
  \item The httpd for fli4l's web admin per default binds to all local IP's.
  \item also use no port number that is already used by another service.\\
  \end{itemize}
  If by mistake a port already in use is defined here, fli4l tries again and again
  to start httpd. This fails because the port is already occupied by the Admin
  Interface or another service. This can only be seen on the console or in the logs.
  You notice it because C3SURF will not work and fli4l generates high CPU load
  and appears to be running slowly.

\config {C3SURF\_HTTPD\_LISTENIP}{C3SURF\_HTTPD\_LISTENIP}{C3SURFHTTPDLISTENIP}

  Default Setting: \var{C3SURF\_HTTPD\_LISTENIP='Host OR IP\-Address'}

  Specifies the local IP to which the login interface will bind to, either IP
  address or @hostname. Http requests of clients will be redirected on demand
  (i.e., when they are not logged in). Hence, users come quickly to the login page.

\underline{Examples:}
\begin{example}
\begin{verbatim}
C3SURF_HTTPD_LISTENIP='@wifi-router'    # Hostname
C3SURF_HTTPD_LISTENIP='192.168.11.3'    # IP-address
C3SURF_HTTPD_LISTENIP='IP_NET_1_IPADDR' # IP-address-variable
\end{verbatim}
\end{example}

The http service for C3SURF always binds to exactly one IP address.

\end{description}

\subsection {Optional Parameters For OPT\_C3SURF}

\begin{description}

\config {C3SURF\_CONTROL\_SQUID}{C3SURF\_CONTROL\_SQUID}{C3SURFCONTROLSQUID}

  Default Setting: \var{C3SURF\_CONTROL\_SQUID='no'}

  By adding the variable C3SURF\_CONTROL\_SQUID='yes' the control over squid
  will be forced. The C3SURF port redirection will be set to the beginning
  which also affects other packages (i.e. openvpn).

  Recommendation is 'no', those using i.e. squid should check, if no
  other functions are affected inadvertently by it.

\config {C3SURF\_SLOPPY\_MAC}{C3SURF\_SLOPPY\_MAC}{C3SURFSLOPPYMAC}

  Default Setting: \var{C3SURF\_SLOPPY\_MAC='no'}

   \begin{itemize}
      \item{C3SURF\_SLOPPY\_MAC='no'}

            (Standard) - if this parameter is not specified only allow
            login with MAC addresses from the ARP table.

       \item{C3SURF\_SLOPPY\_MAC='yes'}

            C3SURF accepts missing MAC addresses or those not contained in the ARP table.
            If you set this to 'yes' you should set \jump{C3SURFCHECKARP}{\var{C3SURF\_CHECK\_ARP='no'}}.
            Else the automatic logout (in average after one minute) will be performed because the
            ``countdown'' process will fire due to a missing entry in the ARP table.
   \end{itemize}

\config {C3SURF\_CHECK\_CURFEW}{C3SURF\_CHECK\_CURFEW}{C3SURFCHECKCURFEW}

  Default Setting: \var{C3SURF\_CHECK\_CURFEW='yes'}

  Turn automatic logoff when reaching the curfew on (\var{'yes'}) or off (\var{'no'}).

\config {C3SURF\_PORTAL\_DEFAULT\_LANG}{C3SURF\_PORTAL\_DEFAULT\_LANG}{C3SURFPORTALDEFAULTLANG}

  Default Setting: \var{C3SURF\_PORTAL\_DEFAULT\_LANG='de'}

  Possible values: a two-characters country code (i.e. 'de', 'fr', 'en').

  Sets the default language for the login page. If omitted, 'de' is assumed.

  Under \textasciitilde/opt/files/srv/www/c3surf/lang/ a file named
  c3surf.$<$countrycode$>$ should exist. At the moment 'de', 'fr', 'en' and
  'it' are supported. If you want to create a file for another language you may
  send it to the fli4l team for inclusion.

\config {C3SURF\_PORTAL\_LANGUAGES}{C3SURF\_PORTAL\_LANGUAGES}{C3SURFPORTALLANGUAGES}\ \\

  Default Setting: \var{C3SURF\_PORTAL\_DEFAULT\_LANG='de fr en it'}

  Value range: a list of two characters each, separated by spaces.

  Specifies the language files that should be transferred to the system for the login page.
  If there is no language file corresponding to the two character shortcut here, a warning
  will be issued that no file was found for it and therefore nothing was copied. The build
  process is not aborted.

\end{description}
