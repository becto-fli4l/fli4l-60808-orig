% Last Update: $Id$
\section {HTTPD}

\subsection{Zusätzliche Einstellungen}

Diese Einstellungen stehen normalerweise nicht in der Konfigurationsdatei,
müssen also hinzugefügt werden, wenn sie benötigt werden.

\begin{description}
\config{HTTPD\_USER}{HTTPD\_USER}{HTTPDUSER}
    {Mit dieser Option ist es möglich, den Webserver mit den Rechten eines
    anderen Benutzers als ,,root'' laufen zu lassen. Dies ist besonders
    sinnvoll, wenn der Webserver benutzt wird, um andere Seiten als das
    Admin-Interface bereitzustellen. Achtung: Es kann sein, dass einige
    Scripts, die Zugriff auf Konfigurationsdateien brauchen, dann nicht mehr
    laufen. Die Standard-Scripts dieses Pakets laufen unter jedem Benutzer.}
\end{description}

\subsection{Allgemeine Bemerkungen}

    Wenn man TELMOND installiert hat, werden auf der Status- und der Calls-
    Seite die Telefonnummern der Anrufer angezeigt. Eine Namenszuordnung
    lässt sich in der Datei opt/etc/phonebook vornehmen. Diese Datei hat das
    gleiche Format wie die Telefonnummerndatei vom IMONC. Es können also
    Telefonbücher zwischen IMONC und Router ausgetauscht werden. Das Format
    jeder Zeile ist dabei ``Telefonnummer=Name[,WAV-Datei]'' (ohne die
    Anführungszeichen). Die WAV-Datei wird aber nur vom IMONC benutzt und
    vom Webserver ignoriert.\\

    Das komplette Webinterface ist seit der Version 2.1.12 auf ein Framefreies
    Design mit CSS umgestellt worden. Alte Browser könnten damit Probleme haben.
    Allerdings hat das den Vorteil, dass man das Aussehen der Oberfläche fast beliebig
    verändern kann, einfach indem man die CSS-Dateien (im wesentlichen 
    /opt/srv/www/css/main.css) anpasst.\\

    Das Webserver-Paket wurde von Thorsten Pohlmann (\email{pohlmann@tetronik.com})
    erstellt und wird zur Zeit von Tobias Gruetzmacher (\email{fli4l@portfolio16.de})
    gepflegt. Das neue Design (seit der Version 2.1.12) wurde von 
    Helmut Hummel (\email{hh@fli4l.de}) realisiert.

