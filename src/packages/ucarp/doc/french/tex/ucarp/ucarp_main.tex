% Do not remove the next line
% Synchronized to r29817

\marklabel{sec:opt-ucarp}
{
\section {OPT\_UCARP}
}
UCARP est un programme qui implante un espace utilisateur avec le protocole
CARP (Common Adress Redundancy Protocols) (ou Protocole Commun de Redondance
d'Adresse). CARP a été développé par l'équipe OpenBSD.

Pour utiliser Ucarp, vous allez avoir besoin d'au moins de deux systèmes. Ces
systèmes ont chacun une adresse IP unique. En outre, une adresse IP virtuelle
sera défini. Un des systèmes est le maître et le(s) autre(s) esclave(s). On
affectera l'adresse IP virtuelle au maître. Si le maître est défaillant
(ou en maintenance), l'esclave prendra le relais du maître. L'adresse IP virtuelle
est utilisée dans le réseau de la Gateway (ou passerelle).

Sur fli4l à la requête UCARP, il sera lancé un script IP-UP pour monter l'interface
virtuelle et un script IP-Down pour couper l'interface virtuelle. L'esclave
prioritère pourra intercepté non seulement lors de l'échec du routeur, mais
aussi de l'échec de la connexion Internet.

\subsection{Configuration}

\begin{description}

\config{OPT\_UCARP}{OPT\_UCARP}{OPTUCARP}

Si vous indiquez 'yes' vous activez le paquetage.

\config{UCARP\_N}{UCARP\_N}{UCARPN}

Vous indiquez ici le nombre de requête UCARP.

\config{UCARP\_x\_INTERFACE}{UCARP\_x\_INTERFACE}{UCARPxINTERFACE}

Vous indiquez ici l'interface sur laquelle la requête UCARP est attachée
(par ex. eth0, br0, eth0.7).

\config{UCARP\_x\_SRCIP}{UCARP\_x\_SRCIP}{UCARPxSRCIP}

Vous indiquez ici la véritable adresse IP et le masque de sous-réseau CIDR
(Classless Inter-Domain Routing), qui est associée à la variable
UCARP\_x\_INTERFACE (par ex. 192.168.6.1/24).

\config{UCARP\_x\_ADDR}{UCARP\_x\_ADDR}{UCARPxADDR}

Vous indiquez dans cette variable l'adresse IP virtuelle et le masque de
sous-réseau CIDR (Classless Inter-Domain Routing) (par ex. 192.168.6.10/24).

\config{UCARP\_x\_PASS}{UCARP\_x\_PASS}{UCARPxPASS}

Vous indiquez dans cette variable un mot de passe pour codé pour la requête
UCARP. il est utilisé par tous les routeurs qui participent au partage de
la même IP virtuelle.

\config{UCARP\_x\_PREEMPT}{UCARP\_x\_PREEMPT}{UCARPxPREEMPT}

Cette variable est optionnelle et peut être omise. Si vous indiquez 'yes'
dans cette variable, le routeur sera maître dès que possible.

\config{UCARP\_x\_ADVSKEW}{UCARP\_x\_ADVSKEW}{UCARPxADVSKEW}

Cette variable est optionnelle, et peut être omise. Dans cette variable
vous indiquez la priorité du routeur maître. Plus la valeur est faible, plus
vous avez de chance que le routeur esclave devient maître. Les valeurs possibles
sont de 0 à 254, la valeur par défaut est '0'

\end{description}
