% Synchronized to r29817
\marklabel{sec:opt-ucarp}
{
\section {OPT\_UCARP}
}

UCARP is a free Userland-Port of the Common Adress Redundancy Protocols
(CARP). CARP was developed by the OpenBSD-Team.

To use UCARP you need at least two systems. Those systems both get
a unique IP address. A virtual IP address is defined in addition.
One of these systems will become master and the other(s) slaves. The master
gets assigned the virtual IP address. If the master goes down, one of the
slaves becomes master.
The virtual IP address is used as the network gateway.

UCARP instances on fli4l are started by an IP-Up script and stopped
by an IP-Down script. This will handle both failure of the router
and failure of the Internet connection.

\subsection{Configuration}

\begin{description}

\config{OPT\_UCARP}{OPT\_UCARP}{OPTUCARP}
'yes' aktivates the package.

\config{UCARP\_N}{UCARP\_N}{UCARPN}
Number of UCARP instances.

\config{UCARP\_x\_INTERFACE}{UCARP\_x\_INTERFACE}{UCARPxINTERFACE}
The interface the UCARP instances should be bound to
(i.e. eth0, br0, eth0.7).

\config{UCARP\_x\_SRCIP}{UCARP\_x\_SRCIP}{UCARPxSRCIP}
The real IP address and the netmask in CIDR (Classless Inter-Domain Routing)
notation set for the UCARP\_x\_INTERFACE (i.e. 192.168.6.1/24)

\config{UCARP\_x\_ADDR}{UCARP\_x\_ADDR}{UCARPxADDR}
The virtual IP address and netmask in CIDR (Classless Inter-Domain
Routing) notation (i.e. 192.168.6.10/24)

\config{UCARP\_x\_PASS}{UCARP\_x\_PASS}{UCARPxPASS}
The encrypting password for this UCARP instance.

\config{UCARP\_x\_PREEMPT}{UCARP\_x\_PREEMPT}{UCARPxPREEMPT}
This setting is optional and may be omitted.
By specifiying 'yes' the router will become master as fast as possible.

\config{UCARP\_x\_ADVSKEW}{UCARP\_x\_ADVSKEW}{UCARPxADVSKEW}
This setting is optional and may be omitted.
By this setting the master may be set. The higher the value the lower
is its  seine right to take role of a master host. Valid values are 0 to
254. The default is 0.
\end{description}
