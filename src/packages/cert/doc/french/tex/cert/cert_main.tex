% Do not remove the next line
% Synchronized to r46350

\marklabel{sec:cert}
{
\section{CERT - TLS protocole de sécurité}
}

    Le paquetage CERT est utilisé pour gérer les certificats fli4l pour la sécurisation
	des échanges sur Internet. À l'heure actuelle, le paquetage fli4l est limité à un
	certificat TLS, il sert à établir une connexion HTTP sécurisées avec le routeur fli4l.
	Ce paquetage est nécessaire, par exemple, si on utilise le paquetage DYNDNS pour mettre
	à jour l'adresse IP externe de interface réseau, il est absolument nécessère d'utiliser
	une connexion HTTP sécurisée.

\begin{description}

\config{OPT\_CERT}{OPT\_CERT}{OPTCERT}

    Cette variable permet la gestion des certificats.

    Configuration par défaut : \verb+OPT_CERT='no'+

    Exemple : \verb+OPT_CERT='yes'+

\config{OPT\_CERT\_X509}{OPT\_CERT\_X509}{OPTCERTX509}

    Cette variable permet d'utiliser la norme X.509 pour des certificats.

	\wichtig{Faire attention, il faut également activer la variable \var{OPT\_OPENSSL='yes'}
	dans le paquetage TOOLS.}

    Configuration par défaut : \verb+OPT_CERT_X509='no'+

    Exemple : \verb+OPT_CERT_X509='yes'+

\config{CERT\_X509\_MOZILLA}{CERT\_X509\_MOZILLA}{CERTX509MOZILLA}

    En activant cette variable, vous installez la norme X.509 pour les certificats,
	elle est aussi livrée avec le navigateur Mozilla Firefox. Cette variable
	ne peut pas être utilisée conjointement avec la variable \var{OPT\_CERT\_X509='yes'}.

    Configuration par défaut : \verb+CERT_X509_MOZILLA='no'+

    Exemple : \verb+CERT_X509_MOZILLA='yes'+

\end{description}
