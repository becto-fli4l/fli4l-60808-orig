% Last Update: $Id$
\marklabel{sec:cert}
{
\section{CERT - Zertifikatsverwaltung}
}

    Das Paket CERT dient der Verwaltung von Zertifikaten auf dem fli4l. Zur
    Zeit beschränkt sich das Paket darauf, die nötigen TLS-Wurzelzertifikate
    auf den fli4l zu bringen, um via TLS gesicherte HTTP-Verbindungen auf dem
    fli4l aufbauen zu können. Dies wird beispielsweise vom DYNDNS-Paket
    benötigt, weil einige Schnittstellen zum Aktualisieren der externen
    IP-Adresse zwingend eine gesicherte HTTP-Verbindung erfordern.

\begin{description}

\config{OPT\_CERT}{OPT\_CERT}{OPTCERT}

    Diese Option aktiviert die Zertifikatsverwaltung.

    Standard-Einstellung: \verb+OPT_CERT='no'+

    Beispiel: \verb+OPT_CERT='yes'+

\config{OPT\_CERT\_X509}{OPT\_CERT\_X509}{OPTCERTX509}

    Diese Option aktiviert die Verwaltung von X.509-Zertifikaten.
    
    \wichtig{Beachten Sie, dass Sie auch \var{OPT\_OPENSSL='yes'} im
    TOOLS-Paket aktivieren müssen.}

    Standard-Einstellung: \verb+OPT_CERT_X509='no'+

    Beispiel: \verb+OPT_CERT_X509='yes'+

\config{CERT\_X509\_MOZILLA}{CERT\_X509\_MOZILLA}{CERTX509MOZILLA}

    Mit dieser Variable kann die X.509-Zertifikatssammlung installiert werden,
    die mit dem Mozilla Firefox-Browser ausgeliefert wird. Sie kann folglich
    nur zusammen mit \var{OPT\_CERT\_X509='yes'} verwendet werden.

    Standard-Einstellung: \verb+CERT_X509_MOZILLA='no'+

    Beispiel: \verb+CERT_X509_MOZILLA='yes'+

\end{description}
