% Last Update: $Id$
\marklabel{sec:cert}
{
\section{CERT - Certificate Management}
}

    The package CERT serves the purpose of managing certificates on the fli4l.
    Currently, the package is restricted to providing the necessary TLS root
    certificates in order to be able to establish HTTP connections secured by
    TLS on the fli4l. This is needed e.g. by the DYNDNS package, because some
    interfaces for updating the external IP address forcibly require a secured
    HTTP connection.

\begin{description}

\config{OPT\_CERT}{OPT\_CERT}{OPTCERT}

    This option enables the certificate manager.

    Default setting: \verb+OPT_CERT='no'+

    Example: \verb+OPT_CERT='yes'+

\config{OPT\_CERT\_X509}{OPT\_CERT\_X509}{OPTCERTX509}

    This option enables the management of X.509 certificates.

    \wichtig{Note that you also need to enable \var{OPT\_OPENSSL='yes'} in the
    TOOLS package.}

    Default setting: \verb+OPT_CERT_X509='no'+

    Example: \verb+OPT_CERT_X509='yes'+

\config{CERT\_X509\_MOZILLA}{CERT\_X509\_MOZILLA}{CERTX509MOZILLA}

    This variable controls the installation of the X.509 certificate collection
    which is deployed with the Mozilla Firefox browser. Consequently, it can
    only be used with \var{OPT\_CERT\_X509='yes'}.

    Default setting: \verb+CERT_X509_MOZILLA='no'+

    Example: \verb+CERT_X509_MOZILLA='yes'+

\end{description}
