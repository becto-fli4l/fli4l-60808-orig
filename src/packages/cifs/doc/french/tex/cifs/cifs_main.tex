% Synchronized to r39197
\marklabel{sec:cifs}
{
\section {CIFS}
}

\subsection {Description}
  Ce paquetage fournit un programme pour utiliser le système de fichier CIFS.
  Ce système de fichier vous permet de faire un montage pour le partage Windows
  et Samba. C'est surtout pour les utilisateurs ayant un routeurs fli4l sans disque
  dur, ces utilisateurs ont peut être besoin de stocker des données de façon
  permanente quelque part. Si vous avez déjà un NAS dans votre réseau, ce périphérique
  avec ce protocole pourra être monté pour stocker des données, en utilisant le bail
  DHCP ou une liste d'identifiants.

\subsection{Configuration}

\begin{description}

\config{OPT\_CIFS}{OPT\_CIFS}{OPTCIFS}{}
  En spécifiant \var{'no'} dans cette variable vous déactivez entièrement le paquetage.
  Pour l'activer vous devez indiquer la valeur \var{'yes'} dans la variable \var{OPT\_CIFS}.

\config{CIFS\_MOUNT\_N}{CIFS\_MOUNT\_N}{CIFSMOUNTN}{}
  Dans cette variable vous indiquez le nombre de partages CIFS à configurer, ils
  seront montés lors du démarrage du système.

\config{CIFS\_MOUNT\_x\_MOUNTPOINT}{CIFS\_MOUNT\_x\_MOUNTPOINT}{CIFSMOUNTxMOUNTPOINT}{}
  Dans la variable \var{CIFS\_MOUNT\_x\_MOUNTPOINT} vous indiquez le répertoire
  pour le partage, il sera monté par le système fli4l. Le répertoire sera créé au
  démarrage du système, le répertoire n'a pas besoin d'être configuré.

\config{CIFS\_MOUNT\_x\_SERVICE}{CIFS\_MOUNT\_x\_SERVICE}{CIFSMOUNTxSERVICE}{}
  Dans cette variable vous indiquez le chemin d'accès réseau pour le partage,
  il sera ensuite monté. Le paramètrage suit le schéma suivant,
  adresse\_réseau/nom\_partage. Exemple~:

  \begin{example}
  \begin{verbatim}
    CIFS_1_SERVICE='192.168.6.100/data'
    CIFS_2_SERVICE='synology/data'
  \end{verbatim}
  \end{example}

\config{CIFS\_MOUNT\_x\_DOMAIN}{CIFS\_MOUNT\_x\_DOMAIN}{CIFSMOUNTxDOMAIN}{}
  Dans la variable \var{CIFS\_MOUNT\_x\_DOMAIN} vous indiquez le nom de domaine
  pour se logger ou se connecté sur Windows ou sur un serveur samba. Cette
  variable est nécessaire uniquement lorsque vous utilisez un domaine, si vous
  n'avez pas de domaine vous pouvez passer cette variable.

\config{CIFS\_MOUNT\_x\_USER}{CIFS\_MOUNT\_x\_USER}{CIFSMOUNTxUSER}{}
  Dans la variable \var{CIFS\_MOUNT\_x\_USER} vous indiquez le nom d'utilisateur
  pour se connecté sur Windows ou sur un serveur samba.

\config{CIFS\_MOUNT\_x\_PASSWORD}{CIFS\_MOUNT\_x\_PASSWORD}{CIFSMOUNTxPASSWORD}{}
  Dans la variable \var{CIFS\_MOUNT\_x\_PASSWORD} vous indiquez un mot de passe
  pour l'authentification avec \var{CIFS\_MOUNT\_x\_USER}, ces valeurs seront utilisé
  pour se connecté sur Windows ou sur un serveur samba.

\config{CIFS\_MOUNT\_x\_SECURITY}{CIFS\_MOUNT\_x\_SECURITY}{CIFSMOUNTxSECURITY}
  Le paramètre que vous allez indiquer dans \var{CIFS\_MOUNT\_x\_SECURITY} sera transmis
  au kernel via "sec=...", il indique le mode de sécurité à utiliser pour l'échange de
  données. Actuellement, sept modes sont pris en charge~:

    \begin{tabular}{|p{1.5cm}|p{11.5cm}|}
    \hline
    Mode & Signification \\
    \hline
    none & pas d'authentification \\
    ntlm & NTLM (NT LAN Manager) Password-Hash (utilise un mot de passe haché)\\
    ntlmi & Comme "ntlm", à l'exception que les paquets de données sont tous signés \\
    ntlmv2 & NTLMv2 (NT LAN Manager Version 2) Password-Hash \\
    ntlmv2i & Comme "ntlmv2", à l'exception que les paquets de données sont tous signés \\
    ntlmssp & NTLMv2-SSP (NT LAN Manager Version 2 Security Support Provider) Password-Hash \\
    ntlmsspi & Comme "ntlmssp", à l'exception que les paquets de données sont tous signés \\
    \hline
  \end{tabular}

  Si l'option n'est pas indiquée, alors l'option par défaut du kernel est utilisée. À
  l'heure actuelle (la version 3.8 du kernel Linux ), utilise le mode de sécurité "ntlmssp".

\config{CIFS\_MOUNT\_x\_IOCHARSET}{CIFS\_MOUNT\_x\_IOCHARSET}{CIFSMOUNTxIOCHARSET}{}
  Dans la variable \var{CIFS\_MOUNT\_x\_IOCHARSET} vous indiquez un jeu de caractères
  il sera utilisé pour convertir les noms de chemin d'accès local dans le jeu de
  caractères unicode. Unicode est généralement utilisé dans la configuration des
  chemins d'accés du réseaux s'il est pris en charge par le serveur. Si ce paramètre
  n'est pas configuré le serveur utilisera le jeu de caractères par défaut, enregistré
  lors de la construction du kernel de fli4l. Si le serveur ne supporte pas unicode vous
  pouvez omettre ce paramètre. Les valeurs suivantes sont valides~:

  \begin{itemize}
    \item{cp1250}
    \item{cp1251}
    \item{cp1255}
    \item{cp437}
    \item{cp737}
    \item{cp775}
    \item{cp850}
    \item{cp852}
    \item{cp855}
    \item{cp857}
    \item{cp860}
    \item{cp861}
    \item{cp862}
    \item{cp863}
    \item{cp864}
    \item{cp865}
    \item{cp866}
    \item{cp869}
    \item{cp874}
    \item{cp932}
    \item{cp936}
    \item{cp949}
    \item{cp950}
    \item{euc-jp}
    \item{iso8859-13}
    \item{iso8859-14}
    \item{iso8859-15}
    \item{iso8859-1}
    \item{iso8859-2}
    \item{iso8859-3}
    \item{iso8859-4}
    \item{iso8859-5}
    \item{iso8859-6}
    \item{iso8859-7}
    \item{iso8859-9}
    \item{koi8-r}
    \item{koi8-ru}
    \item{koi8-u}
    \item{utf8}
    \item{ascii}
  \end{itemize}

\config{CIFS\_MOUNT\_x\_EXTRA\_OPTIONS}{CIFS\_MOUNT\_x\_EXTRA\_OPTIONS}{CIFSMOUNTxEXTRAOPTIONS}{}
  Dans la variable \var{CIFS\_MOUNT\_x\_EXTRA\_OPTIONS} vous pouvez indiquer des option supplémentaires
  (-o) elles seront utilisées par la fonction 'mount.cifs'. Elles seront ajoutées à la ligne de commande.

\end{description}
