% Synchronized to r39197
\marklabel{sec:cifs}
{
\section {CIFS}
}

\subsection {Description}
  This package provides programs to use the CIFS file system.
  This file system enables mounting of Windows and Samba shares.
  Especially for fli4l users having routers without hard drive
  it may be useful to store data permanently somewhere. If a NAS
  already exists hence a share on that device can be mounted
  to store i.e. protocols, DHCP leases or the caller id list. 

\subsection{Configuration}

\begin{description}

\config{OPT\_CIFS}{OPT\_CIFS}{OPTCIFS}{}
  Specifying \var{'no'} deactivates the package completely.
  To activate it set the variable \var{OPT\_CIFS} to 
  \var{'yes'}.

\config{CIFS\_MOUNT\_N}{CIFS\_MOUNT\_N}{CIFSMOUNTN}{}
  This variable contains the count of CIFS shares to configure which will be
  mounted on system boot.

\config{CIFS\_MOUNT\_x\_MOUNTPOINT}{CIFS\_MOUNT\_x\_MOUNTPOINT}{CIFSMOUNTxMOUNTPOINT}{}
  \var{CIFS\_MOUNT\_x\_MOUNTPOINT} specifies the directory where the share
  will be mounted to fli4l's file system. The directory will be created on
  system boot, so it does not have to exist already.

\config{CIFS\_MOUNT\_x\_SERVICE}{CIFS\_MOUNT\_x\_SERVICE}{CIFSMOUNTxSERVICE}{}
  This variable holds the network path to the share to be mounted. The setting
  follows the scheme network\_address/share\_name. Examples:
  \begin{example}
  \begin{verbatim}
    CIFS_1_SERVICE='192.168.6.100/data'
    CIFS_2_SERVICE='synology/data'
  \end{verbatim}
  \end{example}

\config{CIFS\_MOUNT\_x\_DOMAIN}{CIFS\_MOUNT\_x\_DOMAIN}{CIFSMOUNTxDOMAIN}{}
  \var{CIFS\_MOUNT\_x\_DOMAIN} configures the domain name needed for login
  on a Windows- or Samba server. Use this only if you really use domains and
  omit the setting in all other cases.

\config{CIFS\_MOUNT\_x\_USER}{CIFS\_MOUNT\_x\_USER}{CIFSMOUNTxUSER}{}
  \var{CIFS\_MOUNT\_x\_USER} sets the user name used for login on a Windows- or Samba server.

\config{CIFS\_MOUNT\_x\_PASSWORD}{CIFS\_MOUNT\_x\_PASSWORD}{CIFSMOUNTxPASSWORD}{}
  \var{CIFS\_MOUNT\_x\_PASSWORD} contains the password for \var{CIFS\_MOUNT\_x\_USER}
  needed for login on a Windows- or Samba server.

\config{CIFS\_MOUNT\_x\_SECURITY}{CIFS\_MOUNT\_x\_SECURITY}{CIFSMOUNTxSECURITY}
  The value specified in \var{CIFS\_MOUNT\_x\_SECURITY} will be transferred to the kernel
  via ``sec=...'' and determines which security mechanism should be used for file transfers.
  At the moment seven different values are supported:

  \begin{tabular}{|p{1.5cm}|p{11.5cm}|}
    \hline
    Mode & Meaning \\
    \hline
    none & no authentication \\
    ntlm & NTLM (NT LAN Manager) Password-Hash \\
    ntlmi & like ``ntlm'', but all packets will be signed \\
    ntlmv2 & NTLMv2 (NT LAN Manager Version 2) Password-Hash \\
    ntlmv2i & like ``ntlmv2'', but all packets will be signed \\
    ntlmssp & NTLMv2-SSP (NT LAN Manager Version 2 Security Support Provider) Password-Hash \\
    ntlmsspi & like ``ntlmssp'', but all packets will be signed \\
    \hline
  \end{tabular}

  If the option is not specified the kernel default will be used. At the moment
  (Linux kernel as of Version 3.8) this means ``ntlmssp''.

\config{CIFS\_MOUNT\_x\_IOCHARSET}{CIFS\_MOUNT\_x\_IOCHARSET}{CIFSMOUNTxIOCHARSET}{}
  The charset provided in \var{CIFS\_MOUNT\_x\_IOCHARSET} is used for conversion of
  local path's names to the unicode charset. Unicode is usually used in network
  path configuration if the server supports it. If this parameter is not set the
  server will use nls as the default charset used on kernel build. If the server
  does not support unicode the parameter may be omitted. Valid values are:
  \begin{itemize}
    \item{cp1250}
    \item{cp1251}
    \item{cp1255}
    \item{cp437}
    \item{cp737}
    \item{cp775}
    \item{cp850}
    \item{cp852}
    \item{cp855}
    \item{cp857}
    \item{cp860}
    \item{cp861}
    \item{cp862}
    \item{cp863}
    \item{cp864}
    \item{cp865}
    \item{cp866}
    \item{cp869}
    \item{cp874}
    \item{cp932}
    \item{cp936}
    \item{cp949}
    \item{cp950}
    \item{euc-jp}
    \item{iso8859-13}
    \item{iso8859-14}
    \item{iso8859-15}
    \item{iso8859-1}
    \item{iso8859-2}
    \item{iso8859-3}
    \item{iso8859-4}
    \item{iso8859-5}
    \item{iso8859-6}
    \item{iso8859-7}
    \item{iso8859-9}
    \item{koi8-r}
    \item{koi8-ru}
    \item{koi8-u}
    \item{utf8}
    \item{ascii}
  \end{itemize}

\config{CIFS\_MOUNT\_x\_EXTRA\_OPTIONS}{CIFS\_MOUNT\_x\_EXTRA\_OPTIONS}{CIFSMOUNTxEXTRAOPTIONS}{}
  With \var{CIFS\_MOUNT\_x\_EXTRA\_OPTIONS} additional options (-o) may be passed to the `mount.cifs`.
  These will be added to the command line.

\end{description}
