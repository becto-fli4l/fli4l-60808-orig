% Last Update: $Id$
\marklabel{sec:cifs}
{
\section {CIFS}
}

\subsection {Beschreibung}
  Dieses Paket stellt Programme zur Nutzung des CIFS-Dateisystems bereit.
  Mit diesem Dateisystem können unter anderen Windows- und Samba-Freigaben
  eingehängt werden. Insbesondere für Nutzer von fli4l-Routern ohne Festplatte
  kann es nützlich sein, trotzdem Daten permanent irgendwo zu speichern. Wenn
  bereits ein NAS existiert, kann somit eine Freigabe auf diesem Gerät
  eingebunden werden, auf der z.B. die Protokolle, die DHCP-Leases oder die
  Anrufliste gespeichert werden können.

\subsection{Konfiguration}

\begin{description}

\config{OPT\_CIFS}{OPT\_CIFS}{OPTCIFS}{}
  Die Einstellung \var{'no'} deaktiviert dieses Paket vollständig.
  Um das Paket zu aktivieren, ist die Variable \var{OPT\_CIFS} auf 
  \var{'yes'} zu setzen.

\config{CIFS\_MOUNT\_N}{CIFS\_MOUNT\_N}{CIFSMOUNTN}{}
  Diese Variable enthält die Anzahl der zu konfigurierenden CIFS-Freigaben, die
  beim Systemstart eingebunden werden sollen.

\config{CIFS\_MOUNT\_x\_MOUNTPOINT}{CIFS\_MOUNT\_x\_MOUNTPOINT}{CIFSMOUNTxMOUNTPOINT}{}
  Mit \var{CIFS\_MOUNT\_x\_MOUNTPOINT} wird das Verzeichnis konfiguriert, unter
  dem die Windows- bzw. Samba-Freigabe in den Verzeichnisbaum des fli4l
  eingehängt werden soll. Dieses Verzeichnis wird beim Booten erzeugt, es muss
  somit nicht bereits vorher existieren.

\config{CIFS\_MOUNT\_x\_SERVICE}{CIFS\_MOUNT\_x\_SERVICE}{CIFSMOUNTxSERVICE}{}
  Diese Variable enthält den Pfad zu der Windows- bzw. Samba-Freigabe, die in
  den lokalen Verzeichnisbaum eingehängt werden soll. Die Angabe erfolgt nach
  dem Schema Adresse/Freigabe. Beispiele:
  \begin{example}
  \begin{verbatim}
    CIFS_1_SERVICE='192.168.6.100/data'
    CIFS_2_SERVICE='synology/data'
  \end{verbatim}
  \end{example}

\config{CIFS\_MOUNT\_x\_DOMAIN}{CIFS\_MOUNT\_x\_DOMAIN}{CIFSMOUNTxDOMAIN}{}
  Mit \var{CIFS\_MOUNT\_x\_DOMAIN} wird der Domänenname konfiguriert, mit dem die
  Authentifizierung am Windows- oder Samba-Server durchgeführt wird. Dies ist
  nur bei Verwendung von Domänen notwendig und kann in allen anderen Fällen weg
  gelassen werden.

\config{CIFS\_MOUNT\_x\_USER}{CIFS\_MOUNT\_x\_USER}{CIFSMOUNTxUSER}{}
  Mit \var{CIFS\_MOUNT\_x\_USER} wird der Benutzername konfiguriert, mit dem die
  Authentifizierung am Windows- oder Samba-Server durchgeführt wird.

\config{CIFS\_MOUNT\_x\_PASSWORD}{CIFS\_MOUNT\_x\_PASSWORD}{CIFSMOUNTxPASSWORD}{}
  \var{CIFS\_MOUNT\_x\_PASSWORD} enthält das Passwort, das zusammen mit
  \var{CIFS\_MOUNT\_x\_USER} für die Authentifizierung am Windows- oder
  Samba-Server verwendet wird.

\config{CIFS\_MOUNT\_x\_SECURITY}{CIFS\_MOUNT\_x\_SECURITY}{CIFSMOUNTxSECURITY}
  Der unter \var{CIFS\_MOUNT\_x\_SECURITY} angegebene Wert wird an den Kernel
  via ``sec=...'' übergeben und gibt an, welcher Sicherheitsmodus für den
  Datenaustausch verwendet werden soll. Momentan werden sieben verschiedene
  Modi unterstützt:

  \begin{tabular}{|p{1.5cm}|p{11.5cm}|}
    \hline
    Modus & Bedeutung \\
    \hline
    none & keine Authentifizierung \\
    ntlm & NTLM (NT LAN Manager) Passwort-Hash \\
    ntlmi & wie ``ntlm'', nur dass die Datenpakete alle signiert werden \\
    ntlmv2 & NTLMv2 (NT LAN Manager Version 2) Passwort-Hash \\
    ntlmv2i & wie ``ntlmv2'', nur dass die Datenpakete alle signiert werden \\
    ntlmssp & NTLMv2-SSP (NT LAN Manager Version 2 Security Support Provider) Passwort-Hash \\
    ntlmsspi & wie ``ntlmssp'', nur dass die Datenpakete alle signiert werden \\
    \hline
  \end{tabular}

  Wird die Option nicht belegt, dann wird die Kernel-Standardeinstellung
  genommen. Zum aktuellen Zeitpunkt (Linux-Kernel ab Version 3.8) ist dies
  ``ntlmssp''.

\config{CIFS\_MOUNT\_x\_IOCHARSET}{CIFS\_MOUNT\_x\_IOCHARSET}{CIFSMOUNTxIOCHARSET}{}
  Der unter \var{CIFS\_MOUNT\_x\_IOCHARSET} angegebene Zeichensatz wird für die
  Konvertierung lokaler Pfadnamen in den Unicode-Zeichensatz benutzt. Der
  Unicode-Zeichensatz wird normalerweise für Netzwerkpfade verwendet, wenn der
  jeweilige Server das unterstützt. Wenn dieser Parameter nicht angegeben wird,
  nutzt der Server den nls-default-Zeichensatz, der beim fli4l-Kernelbuild
  eingetragen wurde. Falls der Server Unicode nicht unterstützt, kann dieser
  Parameter weggelassen werden. Gültige Werte sind:
  \begin{itemize}
    \item{cp1250}
    \item{cp1251}
    \item{cp1255}
    \item{cp437}
    \item{cp737}
    \item{cp775}
    \item{cp850}
    \item{cp852}
    \item{cp855}
    \item{cp857}
    \item{cp860}
    \item{cp861}
    \item{cp862}
    \item{cp863}
    \item{cp864}
    \item{cp865}
    \item{cp866}
    \item{cp869}
    \item{cp874}
    \item{cp932}
    \item{cp936}
    \item{cp949}
    \item{cp950}
    \item{euc-jp}
    \item{iso8859-13}
    \item{iso8859-14}
    \item{iso8859-15}
    \item{iso8859-1}
    \item{iso8859-2}
    \item{iso8859-3}
    \item{iso8859-4}
    \item{iso8859-5}
    \item{iso8859-6}
    \item{iso8859-7}
    \item{iso8859-9}
    \item{koi8-r}
    \item{koi8-ru}
    \item{koi8-u}
    \item{utf8}
    \item{ascii}
  \end{itemize}

\config{CIFS\_MOUNT\_x\_EXTRA\_OPTIONS}{CIFS\_MOUNT\_x\_EXTRA\_OPTIONS}{CIFSMOUNTxEXTRAOPTIONS}{}
  Mit \var{CIFS\_MOUNT\_x\_EXTRA\_OPTIONS} können dem Befehl `mount.cifs`
  weitere Optionen (-o) übergeben werden. Diese werden der Kommandozeile
  angehängt.

\end{description}
