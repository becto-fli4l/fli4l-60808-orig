% Synchronized to r43045
\marklabel{sec:advanced_networking}
{
\section {Advanced Networking}
}

The package 'advanced networking' provides bonding and bridging 
capabilities for the fli4l-router.
EBTables (\altlink{http://ebtables.sourceforge.net/}) support
can be enabled as well. This allows to build a transparent 
packet filter. To all options of the advanced\_networking 
package generally applies:
\smallskip

\achtung{This package is only for users with profund knowledge
about networks and routing.}

Very unusual problems can appear especially using EBTables
without perfectly knowing the diverse operational modes of layer
2 and 3. Some filtering rules of the packet filter will work 
completely different with EBTables support enabled.

\subsection{Broadcast Relay - Forwarding of IP Broadcasts}

Using a Broadcast Relay, IP broadcasts can surpass interface boundaries.
This is necessary for applications which determine network devices
using broadcasts (eg QNAP Finder). Broadcasts are normally not passed
across network boundaries by the router. This problem can be circumvented
by using a broadcast relay.

Within a Broadcast Relay broadcasts are always forwarded to all connected
interfaces. This means that setting up a second Broadcast Relay with interfaces
swapped is not necessary. In addition, multiple broadcast relays including the same
interface are not allowed.

\begin{description}

\config{OPT\_BCRELAY}{OPT\_BCRELAY}{OPTBCRELAY} Broadcast Forwarding

  Default: \var{OPT\_BCRELAY='no'}

  Setting \var{'yes'} here activates the Broadcast Relay package. Specifying
  \var{'no'} deactivates the Broadcast Relay package completely.

\config{BCRELAY\_N}{BCRELAY\_N}{BCRELAYN}

  Default: \var{BCRELAY\_N='0'}

  The number of Broadcast Relays to configure.

\config{BCRELAY\_x\_IF\_N}{BCRELAY\_x\_IF\_N}{BCRELAYxIFN}

  Default: \var{BCRELAY\_x\_IF\_N='1'}

  Number of interfaces assigned to this broadcast relay.

\config{BCRELAY\_x\_IF\_x}{BCRELAY\_x\_IF\_x}{BCRELAYxIFx}

  Default: \var{BCRELAY\_x\_IF\_x=''}
  Name of the interface assigned to this broadcast relay.

\end{description}

For illustration an example follows where the computer with the application
(eg QNAP Finder) is located in the internal network (connected to \var{eth0})
and the NAS is in a different network (connected to \var{eth1}).

\begin{example}
\begin{verbatim}
OPT_BCRELAY='yes'
BCRELAY_N='1'
BCRELAY_1_IF_N='2'
BCRELAY_1_IF_1='eth0'
BCRELAY_1_IF_2='eth1'
\end{verbatim}
\end{example}

\subsection{Bonding - Combining Several Network Interface Cards In One Link}

Bonding refers to joining at least two network interface cards into
one link. The cards even may be of different type (ie 3Com and Intel) 
or speed (ie 10~Mbit/s or 100~Mbit/s). You can either connect linux 
computers directly or connect to a network switch using bonding.
In this way a 200~Mbit/s full duplex connection from a flil4-router 
to a switch can be used without much effort.
Everyone interested in using bonding should have read the documentation
in the kernel directory (bonding.txt). The names of the bonding settings
largely correspond to the names used there.

\begin{description}

\config{OPT\_BONDING\_DEV}{OPT\_BONDING\_DEV}{OPTBONDINGDEV}

  Default: \var{OPT\_BONDING\_DEV='no'}

  \var{'yes'} activates the bonding package, \var{'no'} deativates the 
  bonding package completely.

\config{BONDING\_DEV\_N}{BONDING\_DEV\_N}{BONDINGDEVN}

  Default: \var{BONDING\_DEV\_N='0'}

  Number of bonding devices to be configured.

\config{BONDING\_DEV\_x\_DEVNAME}{BONDING\_DEV\_x\_DEVNAME}{BONDINGDEVxDEVNAME}

  Default: \var{BONDING\_DEV\_x\_DEVNAME=''}

  Name of the bonding device to be created. It should consist of the 
  prefix 'bond' and a trailing number with out a leading '0'. The 
  numbers of the bonding devices don't have to start with '0' and 
  need not be consecutive. Possible values could be 'bond0', 'bond8'
  or 'bond99'.

  \config{BONDING\_DEV\_x\_MODE}{BONDING\_DEV\_x\_MODE}{BONDINGDEVxMODE}

  Default: \var{BONDING\_DEV\_x\_MODE=''}

  Specifies the bonding method. Default is round-robin
  'balance-rr'. Possible values are listed below: 

\begin{description}

\item [balance-rr]

  Round-robin method: Submit sequentially over all slaves from
  the first to the last. This method provides both load balancing
  and fault tolerance.

\item [active-backup]

  Active backup: Only one slave in the bond is active. The other 
  slaves are activated only when the active slave fails. The
  MAC address of the bond is only visible on one port (network 
  adapter) so it does not confuse the switch. This mode 
  provides fault tolerance.

\item [balance-xor]

  XOR method: Submit based on the formula [
  (Source-MAC-address XOR destination-MAC-address) modulo the 
  number of slaves]. This ensures that the same slave always is 
  used for the same destination-MAC-address. This method provides 
  both load balancing and fault tolerance.

\item [broadcast]

  Broadcast method: Transmits everything on all slave devices.
  This mode provides fault tolerance.

\item [802.3ad]

  IEEE 802.3ad dynamic link aggregation. Creates
  aggregation groups that share the same speed and duplex
  settings. Transmits on all slaves in the active aggregator.

  Requirements:
  \begin{itemize}

    \item ethtool support in the base device driver to retrieve
          speed and duplex status for each device.

    \item a switch that supports dynamic IEEE 802.3ad
          connection aggregation.

  \end{itemize}

\item [balance-tlb]

   Adaptive load balancing for outgoing data: channel bonding that
  does not need any special features in the switch. The outgoing
  network traffic is distributed on each slave according to the 
  current load. Incoming network traffic is received by the current
  slave. If the receiving slave fails, another slave takes over the
  MAC address of the slave gone down.
  
  Requirements:
  \begin{itemize}

    \item ethtool support in the base device driver to retrieve
          speed and duplex status for each device.
  \end{itemize}
  
\item [balance-alb]

  Adaptive load balancing: includes both balance-tlb, and inbound 
  load balancing (rlb) for IPV4 traffic and needs no special 
  requirements on the Switch. Load Balancing for incoming traffic 
  is achieved through ARP requests. The bonding driver catches ARP
  responses from the server on their way outside and overrides the 
  source hardware address with the unique hardware address of a slave
  in the bond. This way different clients use different hardware 
  addresses for the server.
  
  Incoming traffic from connections created by the server will also be 
  balanced. If the server sends ARP requests, the bonding driver 
  copies and stores the client IP from the ARP. At the time the ARP 
  response of the client arrives the bonding driver determines its 
  hardware address and creates an ARP reply to this client assigning 
  a client in the bond to it. A problematic effect of ARP arrangements 
  for load balancing is that every time an ARP request is sent the 
  hardware address of the bond is used. Clients learn the hardware 
  address of the bond and the incoming traffic on the current slave 
  collapses. This fact is countered in a way that updates (ARP Replies) 
  to all clients will be sent to their respective hardware addresses so 
  that the traffic is divided again. Incoming traffic will be newly 
  allocated even when a new slave is added to the bonding or an 
  inactive slave is re-activated. The receiving load is distributed
  sequentially (round robin) in the group of the slave with the largest
  network speed in the bond.
  
  When a connection is restored or a new slave joins the bond incoming 
  traffic will be distributed anew to all active Slaves in the bond 
  by sending ARP replies with the selected MAC addresses to each client. 
  The parameter 'updelay' must be set to a value greater than or equal 
  to the forwarding delay of the switch in order to avoid blocking of
  ARP responses to clients.
  
  Requirements:
  \begin{itemize}

    \item ethtool support in the base device driver to retrieve
          speed and duplex status for each device.
    
    \item support in the base device driver to set the hardware address
	  even when the device is open. This is neccessary for granting
	  that at every time at least one slave in the bond is carrying
	  the hardware address of the bond (curr\_active\_slave) although
	  every slave in the bond has its own unique hardware address.
          If curr\_active\_slave fails its hardware address will simply be 
          replaced with a new one.

  \end{itemize}

\end{description}

\config{BONDING\_DEV\_x\_DEV\_N}{BONDING\_DEV\_x\_DEV\_N}{BONDINGDEVxDEVN}

  Default: \var{BONDING\_DEV\_x\_DEV\_N='0'}

  Specifies the number of physical devices the bond consists of. E.g. for 
  a bond between 'eth0' and 'eth1' (two eth-devices) '2' has to be entered.
  
\config{BONDING\_DEV\_x\_DEV\_x}{BONDING\_DEV\_x\_DEV\_x}{BONDINGDEVxDEVx}

  Default: \var{BONDING\_DEV\_x\_DEV\_x=''}

  The name of a physical device which belongs to this bonding device. 
  An example would be the value 'eth0'. Please note that a physical 
  device that you use for a bond can't be used for anything else.
  So you can't use it additionally for a DSL modem, a bridge, a VLAN or 
  inclusion in base.txt.

\config{BONDING\_DEV\_x\_MAC}{BONDING\_DEV\_x\_MAC}{BONDINGDEVxMAC}

  Default: \var{BONDING\_DEV\_x\_MAC=''}

  This setting is optional and can also be completely omitted.

  A bonding device defaults to the MAC address of the first
  physical device which is used for bonding. If you do not
  want this it is possible to specify a MAC address the
  bonding device should use here.
  

\config{BONDING\_DEV\_x\_MIIMON}{BONDING\_DEV\_x\_MIIMON}{BONDINGDEVxMIIMON}

  Default: \var{BONDING\_DEV\_x\_MIIMON='100'}

  This setting is optional and can also be completely omitted.

  Specifies the interval (in milliseconds) in which the individual
  connections of a bonding device are checked for their link status. 
  The link status of each physical device in the bond will be checked
  every x milliseconds. Setting this to '0' will disable the miimon 
  monitoring.
  
\config{BONDING\_DEV\_x\_USE\_CARRIER}{BONDING\_DEV\_x\_USE\_CARRIER}{BONDINGDEVxUSECARRIER}

  Default: \var{BONDING\_DEV\_x\_USE\_CARRIER='yes'}

  This setting is optional and can also be completely omitted.

  If the link status check by miimon (see above) is enabled
  this setting can specify the function performing the check.
  
  \begin{itemize}

    \item \var{'yes'}: netif \_carrier\_ok() function 
    \item \var{'no}: direct calls to MII or ethtool ioctl() System calls
  
  \end{itemize}
  
  The netif\_carrier\_ok() method is more efficient, but not all 
  drivers do support this method.

\config{BONDING\_DEV\_x\_UPDELAY}{BONDING\_DEV\_x\_UPDELAY}{BONDINGDEVxUPDELAY}

  Default: \var{BONDING\_DEV\_x\_UPDELAY='0'}

  This setting is optional and can also be completely omitted.

  The value of this setting multiplied by the setting of 
  \var{BONDING\_DEV\_x\_MIIMON} specifies the time a in which a
  connection of bonding devices is activated after the corresponding 
  link (for example 'eth0') is up. This way the connection of the
  bonding device is activated until the link status switches to 
  "not connected".
  
\config{BONDING\_DEV\_x\_DOWNDELAY}{BONDING\_DEV\_x\_DOWNDELAY}{BONDINGDEVxDOWNDELAY}

  Default: \var{BONDING\_DEV\_x\_DOWNDELAY='0'}

  This setting is optional and can also be completely omitted.

  The value of this setting multiplied by the setting of
  \var{BONDING\_DEV\_x\_MIIMON} specifies the time a in which a
  connection of bonding devices is deactivated if the appropriate
  link (iE an eth-device) fails. This will deactivate the connection
  of a bonding device temporarily until the link status is back to 'active'.
  
\config{BONDING\_DEV\_x\_LACP\_RATE}{BONDING\_DEV\_x\_LACP\_RATE}{BONDINGDEVxLACPRATE}

  Default: \var{BONDING\_DEV\_x\_LACP\_RATE='slow'}

  This setting is optional and can also be completely omitted.

  Specifiy how often link informations are exchanged between the 
  link partners (for example a switch or another linux PC) if 
  \var{BONDING\_DEV\_x\_MODE=''} is set to '802.3ad'.
  
  \begin{itemize}

    \item 'slow': every 30 seconds 
    \item 'fast': each second.
    
  \end{itemize}
  
\config{BONDING\_DEV\_x\_PRIMARY}{BONDING\_DEV\_x\_PRIMARY}{BONDINGDEVxPRIMARY}

  Default: \var{BONDING\_DEV\_x\_PRIMARY=''}

  This setting is optional and can also be completely omitted.

  Specify primary output device if mode is set to 'active-backup'.
  This is useful if the various devices have different speeds. 
  Provide a string (for example 'eth0') for the device to be used primarily.
  If a value is entered and the device is online it will be used
  as the first output medium. Only if the device is offline another 
  device will be used. If a failure is detected a new standard 
  output medium will be chosen. This comes in handy if one slave 
  has priority over another, for example if a slave is faster
  than another (1000 Mbit/s versus 100 Mbit/s).
  If the 1000 Mbit/s slave fails and later gets back up it may 
  be of advantage to set the faster slave active again without 
  having to cause a fail of the 100 Mbit/s slave artificially
  (for example by pulling the plug).
  
\config{BONDING\_DEV\_x\_ARP\_INTERVAL}{BONDING\_DEV\_x\_ARP\_INTERVAL}{BONDINGDEVxARPINTERVAL}

  Default: \var{BONDING\_DEV\_x\_ARP\_INTERVAL='0'}

  This setting is optional and can also be completely omitted.

  The interval in which IP-addresses specified in 
  \var{BONDING\_DEV\_x\_ARP\_IP\_TARGET\_x} are checked by using 
  their ARP responses (in milliseconds). If ARP monitoring 
  is used in load-balancing mode (mode 0 or 2) the switch 
  should be adjusted to distribute packets to all connections 
  equally (for example round robin). If the switch is set to
  distribute the packets according to the XOR method all responses 
  of the ARP targets will arrive on the same connection which could
  cause failure for all team members. ARP monitoring should not 
  be combined with miimon. Passing '0' will disable ARP monitoring.
  
\config{BONDING\_DEV\_x\_ARP\_IP\_TARGET\_N}{BONDING\_DEV\_x\_ARP\_IP\_TARGET\_N}{BONDINGDEVxARPIPTARGETN}

  Default: \var{BONDING\_DEV\_x\_ARP\_IP\_TARGET\_N=''}

  This setting is optional and can also be completely omitted.

  The number of IP-addresses which are used for ARP checking.
  A maximum of 16 IP-addresses can be checked.
  
\config{BONDING\_DEV\_x\_ARP\_IP\_TARGET\_x}{BONDING\_DEV\_x\_ARP\_IP\_TARGET\_x}{BONDINGDEVxARPIPTARGETx}

  Default: \var{BONDING\_DEV\_x\_ARP\_IP\_TARGET\_x=''}

  This setting is optional and can also be completely omitted.

  If \var{BONDING\_DEV\_x\_ARP\_INTERVAL} is $>$ 0, specify
  one IP address which is used as the target for ARP requests 
  to evaluate the quality of the connection. Enter values using
  format 'ddd.ddd.ddd.ddd'. To get ARP monitoring to work at least one 
  IP address has to be given here.
  
\end{description}

\marklabel{sec:vlan}
{
\subsection {VLAN - 802.1Q Support}
}

Support for 802.1Q VLAN is reasonable only in conjunction with
using appropriate switches. Port-based VLAN switches are
\emph{not} suitable. A general introduction to the subject
VLAN can be found at
\altlink{http://en.wikipedia.org/wiki/IEEE_802.1Q}.
At \altlink{http://de.wikipedia.org/wiki/VLAN} some additional
information can be found.

Please note that not any network card can handle VLANs. Some can not 
handle VLANs at all, others require a matching MTU and few cards work
without any problems. The author of the advanced\_networking package
uses Intel network cards with the 'e100' driver without any problem.
MTU adjustment is not necessary. 3COM's '3c59x' driver
requires MTU adjustment to 1496 otherwise the card won't work
correctly. The 'starfire' driver does not work properly if a VLAN 
device is added to a bridge. In this case no packets can be received.
Those who want to work with VLANs should ensure that the respective
Linux NIC drivers support VLANs correctly.

\begin{description}

\config{OPT\_VLAN\_DEV}{OPT\_VLAN\_DEV}{OPTVLANDEV}

  Default: \var{OPT\_VLAN\_DEV='no'}

  \var{'yes'} activates the VLAN package, \var{'no'} deactivates it.

\config{VLAN\_DEV\_N}{VLAN\_DEV\_N}{VLANDEVN}

  Default: \var{VLAN\_DEV\_N=''}

  Number of VLAN devices to configure.

\config{VLAN\_DEV\_x\_DEV}{VLAN\_DEV\_x\_DEV}{VLANDEVxDEV}

  Default: \var{VLAN\_DEV\_x\_DEV=''}

  Name of the device connected to a VLAN capable switch (iE \var{'eth0'}, \var{'br1'}, \var{'eth2'}...).

\config{VLAN\_DEV\_x\_VID}{VLAN\_DEV\_x\_VID}{VLANDEVxVID}

  Default: \var{VLAN\_DEV\_x\_VID=''}

  The VLAN ID for which the appropriate VLAN device should be created. 
  The name of the VLAN device consists of the prefix 'ethX'
  and the attached VLAN ID (without leading '0'). For example '42'
  creates a VLAN device \var{'eth0.42'} on the fli4l-router.
  
\end{description}

  VLAN devices on the fli4l-router are always named '$<$device$>$.$<$vid$>$'. 
  So if you have an eth-device connected to a VLAN-capable switch
  and you want to use VLANs 10, 11 and 23 on the fli4l-router you have to 
  configure 3 VLAN devices with the eth-device as \var{VLAN\_DEV\_x\_DEV='ethX'}
  and the respective VLAN ID in \var{VLAN\_DEV\_x\_VID=''}. 
  Example:
\begin{example}
\begin{verbatim}
OPT_VLAN_DEV='yes'
VLAN_DEV_N='3'
VLAN_DEV_1_DEV='eth0'
VLAN_DEV_1_VID='10'	# will create device: eth0.10
VLAN_DEV_2_DEV='eth0'
VLAN_DEV_2_VID='11'	# will create device: eth0.11
VLAN_DEV_3_DEV='eth0'
VLAN_DEV_3_VID='23'	# will create device: eth0.23
\end{verbatim}
\end{example}

\achtung{Please always remember to check the MTU of all units involved.
Caused by the VLAN header the frames will be 4 bytes longer. If
necessary the MTU must be changed to 1496 on the devices.}

\marklabel{sec:devmtu}
{
\subsection {Device MTU - Adjusting MTU Values}
}

In rare circumstances it may be necessary to adjust the MTU of a device.
E.G. some not 100\% VLAN-compatible network cards need to adjust the MTU. 
Please remember that few network cards are capable of processing Ethernet 
frames larger than the 1500 bytes!

\begin{description}

\config{DEV\_MTU\_N}{DEV\_MTU\_N}{DEVMTUN}

  Default: \var{DEV\_MTU\_N=''}

  This setting is optional and can also be completely omitted.

  Number of devices to change their MTU settings.

\config{DEV\_MTU\_x}{DEV\_MTU\_x}{DEVMTUx}

  Default: \var{DEV\_MTU\_x=''}

  This setting is optional and can also be completely omitted.

  Name of the device to change its MTU followed by the MTU to be set.
  Both statements have to be separated by a space.
  To set a MTU of \var{'1496'} for \var{'eth0'} 
  enter the following:

\begin{example}
\begin{verbatim}
DEV_MTU_N='1'
DEV_MTU_1='eth0 1496'
\end{verbatim}
\end{example}

\end{description}

\marklabel{sec:bridge}
{
\subsection {BRIDGE - Ethernet Bridging for fli4l}
}

This is a full-fledged ethernet-bridge using spanning tree protocol
on demand. For the user the Computer seems to work as a layer~3 
switch on configured ports.
\hfil\break
Further information on bridging can be found here:

\begin{itemize}
\item Homepage of the Linux Bridging Project:\hfil\break
      \altlink{http://bridge.sourceforge.net/}\hfil\break

\item The detailed and authoritative description of the bridging standards:\hfil\break
      \altlink{http://standards.ieee.org/getieee802/download/802.1D-2004.pdf}.\hfil\break
      (Mainly informations from page 153 on are interesting. Please note that the 
       Linux bridging code is working according to standards from 1998, allowing 
       only 16 bit Values for pathcost as an example.)

\item Calculation of different timing values for the spanning tree protocol:\hfil\break
      \altlink{http://www.dista.de/netstpclc.htm}\hfil\break

\item See how STP is working by looking at some nice examples:\hfil\break
      \altlink{http://web.archive.org/web/20060114052801/http://www.zyxel.com/support/supportnote/ves1012/app/stp.htm}\hfil\break
\end{itemize}

\begin{description}

\config{OPT\_BRIDGE\_DEV}{OPT\_BRIDGE\_DEV}{OPTBRIDGEDEV}

  Default: \var{OPT\_BRIDGE\_DEV='no'}

  \var{'yes'} activates the bridge package, \var{'no'} deactivates it.

\config{BRIDGE\_DEV\_BOOTDELAY}{BRIDGE\_DEV\_BOOTDELAY}{BRIDGEDEVBOOTDELAY}

  Default: \var{BRIDGE\_DEV\_BOOTDELAY='yes'}

  This setting is optional and can also be completely omitted.

  As a bridge needs at least $2\ \times$ \var{BRIDGE\_DEV\_x\_FORWARD\_DELAY} in
  seconds to become active this period has to be waited if devices are needed 
  at the startup of the fli4l-router. As an example consider sending syslog messages 
  or dialing in via DSL. 
  If the entry is set to \var{'yes'} $2\ \times$ \var{BRIDGE\_DEV\_x\_FORWARD\_DELAY} 
  is waited automatically. If the bridges are not required at startup-time
  \var{'no'} should be set to accelerate the startup process of fli4l router.
  
%  \var{BRIDGE\_DEV\_x\_MAX\_MESSAGE\_AGE}$ + (2\ \times$ \var{BRIDGE\_DEV\_x\_FORWARD\_DELAY}$)$

\config{BRIDGE\_DEV\_N}{BRIDGE\_DEV\_N}{BRIDGEDEVN}

  Default: \var{BRIDGE\_DEV\_N='1'}

  The number of independent bridges. Each bridge has to be considered 
  completely isolated. This applies in particular for the setting of 
  \var{BRIDGE\_\-DEV\_\-x\_\-STP}. There will be created one virtual device 
  by the name of \var{'br$<$nummer$>$'} per bridge.
  
\config{BRIDGE\_DEV\_x\_NAME}{BRIDGE\_DEV\_x\_NAME}{BRIDGEDEVxNAME}

  Default: \var{BRIDGE\_DEV\_x\_NAME=''}

  The symbolic name of the bridge. This name can be used by other packages 
  in order to use the bridge regardless of its device name.
  
\config{BRIDGE\_DEV\_x\_DEVNAME}{BRIDGE\_DEV\_x\_DEVNAME}{BRIDGEDEVxDEVNAME}

  Default: \var{BRIDGE\_DEV\_x\_DEVNAME=''}

  Each bridge device needs a name in the form of
  \var{'br$<$number$>$'}. $<$number$>$ can be a number between '0' and
  '99' without leading '0'. Possible entries could be  
  \var{'br0'}, \var{'br9'} or \var{'br42'}. Names can be chosen 
  arbitrary, the first bridge may be \var{'br3'} and the second \var{'br0'}.

\config{BRIDGE\_DEV\_x\_DEV\_N}{BRIDGE\_DEV\_x\_DEV\_N}{BRIDGEDEVxDEVN}

  Default: \var{BRIDGE\_DEV\_x\_DEV\_N='0'}

  How many network devices belong to the bridge? The count of devices 
  that should be connected to the bridge. It can even be \var{'0'} if the 
  bridge is only a placeholder for an IP-address that should be taken 
  over by a VPN-tunnel connected to the bridge.

\config{BRIDGE\_DEV\_x\_DEV\_x\_DEV}{BRIDGE\_DEV\_x\_DEV\_x\_DEV}{BRIDGEDEVxDEVxDEV}

  Specifies which device can be connected to the bridge. You can fill in 
  an eth-device (ie \var{'eth0'}), a bonding device (iE \var{'bond0'}) or 
  also a VLAN-device (iE \var{'vlan11'}).
  A device connected here may not be used in other places and is not 
  allowed to get an IP-address assigned.
  
\begin{example}
\begin{verbatim}
BRIDGE_DEV_1_DEV_N='3'
BRIDGE_DEV_1_DEV_1_DEV='eth0.11'	#VLAN 11 on eth0
BRIDGE_DEV_1_DEV_2_DEV='eth2'
BRIDGE_DEV_1_DEV_3_DEV='bond0'
\end{verbatim}
\end{example}

\config{BRIDGE\_DEV\_x\_AGING}{BRIDGE\_DEV\_x\_AGING}{BRIDGEDEVxAGING}

  Default: \var{BRIDGE\_DEV\_x\_AGING='300'}

  This setting is optional and can also be completely omitted.

  Specifies the time after which old entries in the bridges' MAC 
  table will be deleted. If in this amount of time in seconds
  no data is received or transmitted by the computer with the network 
  card the corresponding MAC address will be deleted in the bridges' MAC table.
  
\config{BRIDGE\_DEV\_x\_GARBAGE\_COLLECTION\_INTERVAL}{BRIDGE\_DEV\_x\_GARBAGE\_COLLECTION\_INTERVAL}{BRIDGEDEVxGARBAGECOLLECTIONINTERVAL}
\hfil 

  Default: \var{BRIDGE\_DEV\_x\_GARBAGE\_COLLECTION\_INTERVAL='4'}
 
  This setting is optional and can also be completely omitted.

  Specifies the time after which \glqq{}garbage collection\grqq{} 
  will be done. All dynamic entries will be checked. Entries not 
  longer valid and outdated will get deleted. In particular old 
  invalid connections will be deleted.

\config{BRIDGE\_DEV\_x\_STP}{BRIDGE\_DEV\_x\_STP}{BRIDGEDEVxSTP}

  Default: \var{BRIDGE\_DEV\_x\_STP='no'}

  This setting is optional and can also be completely omitted.

  Spanning tree protocol allows to manage multiple connections to 
  different switches. This results in redundancy ensuring network 
  functionality in case of line failures. Without the use of STP 
  redundant lines between switches aren't possible and networking 
  may fail. STP tries to use the fastest connection between two 
  switches. This way even connections with different speeds are 
  reasonable. You may iE use a 1~Gbit/s connection as main  
  and a second 100~Mbit/s as a fallback.

  A good source of background informations can be found here:
  \hfil\break
  \altlink{http://en.wikipedia.org/wiki/Spanning_Tree_Protocol}.

\config{BRIDGE\_DEV\_x\_PRIORITY}{BRIDGE\_DEV\_x\_PRIORITY}{BRIDGEDEVxPRIORITY}

  Default: \var{BRIDGE\_DEV\_x\_PRIORITY=''}

  This setting is optional and can also be completely omitted.

  Only valid if \var{BRIDGE\_DEV\_x\_STP='yes'} is set!

  Which priority has this bridge? The bridge with the lowest 
  priority wins the main bridge election. Each bridge should have 
  a different priority. Please note that the bridge with the lowest 
  priority should also have the biggest available bandwith because in 
  addition to the complete data traffic control packets will be sent 
  by it every 2 seconds. (See also: \var{BRIDGE\_DEV\_x\_HELLO})
  
  Valid Values are from \var{'0'} to \var{'61440'} in steps of 4096.

\config{BRIDGE\_DEV\_x\_FORWARD\_DELAY}{BRIDGE\_DEV\_x\_FORWARD\_DELAY}{BRIDGEDEVxFORWARDDELAY}

  Default: \var{BRIDGE\_DEV\_x\_FORWARD\_DELAY='15'}

  This setting is optional and can also be completely omitted.

  Only valid if \var{BRIDGE\_DEV\_x\_STP='yes'} is set!

  If one connection of the bridge was deactivated or if a connection 
  is added to the bridge it takes the given time in seconds $\times$ 2 
  until the connection can send data. This parameter is crucial for the 
  time the bridge needs to recognize a dead connection. The time period 
  is calculated in seconds with this formula:
  
  \textbf{\var{BRIDGE\_DEV\_x\_MAX\_MESSAGE\_AGE}$ + (2\ \times$ \var{BRIDGE\_DEV\_x\_FORWARD\_DELAY}$)$}

  In standard values this means: $20 + (2 \times 15) = 50$
  seconds. The time to recognize a dead connection can be minimized
  if \var{BRIDGE\_DEV\_x\_HELLO} is set to 1 second and 
  \var{BRIDGE\_DEV\_x\_FORWARD\_DELAY} is set to 4 seconds. 
  In addition \var{BRIDGE\_DEV\_x\_MAX\_MESSAGE\_AGE} has to set to 4
  seconds. This leads to: $4 + (2 \times 4) = 12$ seconds. This is as 
  fast as it can get.

\config{BRIDGE\_DEV\_x\_HELLO}{BRIDGE\_DEV\_x\_HELLO}{BRIDGEDEVxHELLO}

  Default: \var{BRIDGE\_DEV\_x\_HELLO='2'}

  This setting is optional and can also be completely omitted.

  Only valid if \var{BRIDGE\_DEV\_x\_STP='yes'} is set!

  The time mentioned in \var{BRIDGE\_DEV\_x\_HELLO} is the time 
  in seconds in which the so-called 'Hello-message' is sent by the 
  main bridge. These messages are necessary for STP's automatic configuration.

\config{BRIDGE\_DEV\_x\_MAX\_MESSAGE\_AGE}{BRIDGE\_DEV\_x\_MAX\_MESSAGE\_AGE}{BRIDGEDEVxMAXMESSAGEAGE}

  Default: \var{BRIDGE\_DEV\_x\_MAX\_MESSAGE\_AGE='20'}

  This setting is optional and can also be completely omitted.

  Only valid if \var{BRIDGE\_DEV\_x\_STP='yes'} is set!

  The maximum time period the last 'Hello-message' stays valid. If 
  no new 'Hello-message' is received during this period a new main bridge 
  election will be triggered. This is why this value should \textbf{never} 
  be lower than $2\ \times$ \var{BRIDGE\_DEV\_x\_HELLO}.

\config{BRIDGE\_DEV\_x\_DEV\_x\_PORT\_PRIORITY}{BRIDGE\_DEV\_x\_DEV\_x\_PORT\_PRIORITY}{BRIDGEDEVxDEVxPORTPRIORITY}

  Default: \var{BRIDGE\_DEV\_x\_DEV\_x\_PORT\_PRIORITY='128'}

  This setting is optional and can also be completely omitted.

  Only valid if \var{BRIDGE\_DEV\_x\_STP='yes'} is set!
  
  Only relevant if multiple connections with the same
  \var{BRIDGE\_DEV\_x\_DEV\_x\_PATHCOST} have the same destination. 
  If this is the case the connection with lowest priority will 
  be chosen.

  Valid values are '0' to '240' in steps of '16'.

\config{BRIDGE\_DEV\_x\_DEV\_x\_PATHCOST}{BRIDGE\_DEV\_x\_DEV\_x\_PATHCOST}{BRIDGEDEVxDEVxPATHCOST}

  Default: \var{BRIDGE\_DEV\_x\_DEV\_x\_PATHCOST='100'}

  This setting is optional and can also be completely omitted.

  Only valid if \var{BRIDGE\_DEV\_x\_STP='yes'} is set!

  Indirectly specifies the bandwidth for this connection. The lower the 
  value the higher is the bandwidth and therefore the connection 
  gets a higher priority.
    
  The calculation base proposed is 1000000~kbit/s which leads to
  the traffic costs listed in table \ref{tab:traffic-costs}. Please note 
  to use the actual usable bandwidth in the formula when calculating. As a result 
  this leads to significantly lower values than you would expect, especially on 
  wireless lan.
  
  Note: The current IEEE standard from 2004 uses 32~bit numbers for 
  bandwidth calculation which is not supported on Linux yet.

\begin{table}[htbp]
\centering
\begin{tabular}{r|l}
Bandwidth & Setting of \var{BRIDGE\_DEV\_x\_DEV\_x\_PATHCOST} \\
\hline
 64 kbit/s & 15625\\
128 kbit/s &  7812\\
256 kbit/s &  3906\\
 10 Mbit/s &   100\\
 11 Mbit/s &   190\\
 54 Mbit/s &    33\\
100 Mbit/s &    10\\
  1 Gbit/s &     1\\
\end{tabular}
\caption{Values for BRIDGE\_DEV\_x\_DEV\_x\_PATHCOST as a function of bandwidth}
\label{tab:traffic-costs}
\end{table}

\end{description}

\subsection{Notes}

A bridge will forward any type of Ethernet data - thus e.g. a 
regular DSL modem can be used over WLAN as if it had
a WLAN interface. No packets that pass the bridge will be 
examined for any undesirable activities (ie the fli4l packet 
filter is not active!). Use only after careful consideration of
security risks (ie as a WLAN access point). There is also 
the possibility to activate EBTables support however.

\marklabel{sec:ebtables}
{
\subsection {EBTables - EBTables for fli4l}
}
\configlabel{OPT\_EBTABLES}{OPTEBTABLES}
As of Version 2.1.9 fli4l has rudimental EBTables support.
By setting \var{OPT\_EBTABLES='yes'} EBTables support will get activated. 

This means that all ebtables kernel modules get loaded and the 
ebtables program on the fli4l-routers will get available.
In contrast to the much simplified netfilter configuration
through the different filter lists of fli4l it then is
necessary to write an ebtables script of your own. This
means you have to write the complete ebtables script yourself.

For background informations about EBTables support please read
the EBTables documentation at \altlink{http://ebtables.sourceforge.net}.

There is the possibility of issuing ebtables commands on the router 
before and after setting up the netfilter (\var{PF\_INPUT\_x}, \var{PF\_FORWARD\_x} etc).
To do so, create the files ebtables.pre und ebtables.post 
in the directory config/ebtables. Ebtables.pre  will get executed before 
and ebtables.post after configuring the netfilter. Please remember that
an error in the ebtables scripts may interrupt the boot process 
of the fli4l-router!

\achtung{Before using EBTables you should definitely read the complete 
documentation. By using EBTables the complete behavior of the router may 
change! Especially filtering by mac: in PF\_FORWARD will not work as before.}

Have a look at this page giving a small glimpse about how the
the ebtables support works:
\altlink{http://ebtables.sourceforge.net/br_fw_ia/br_fw_ia.html}.

% #N.A.# ##TRANSLATE## : translate new section (FFL-1447)
\marklabel{sec:switch}
{
\subsection {SWITCH - Switch configuration}
}

This OPT allows for configuration of a switch built into the fli4l hardware.
Some platforms like the Banana Pi R-1 provide more network ports, grouped
internally to a switch. Only by an appropriate switch configuration the individual
network ports (or groups thereof) can be assigned to individual networks (VLANs).

\begin{description}
\config{OPT\_SWITCH}{OPT\_SWITCH}{OPTSWITCH}

This variable activates switch configuration.

Default setting: \verb+OPT_SWITCH='no'+

Example: \verb+OPT_SWITCH='yes'+

\config{SWITCH\_N}{SWITCH\_N}{SWITCHN}

This variable specifies the number of switch devices to configure. 
Typically, only one switch is available.

Default setting: \verb+SWITCH_N='0'+

Example: \verb+SWITCH_N='1'+

\config{SWITCH\_x\_DEV}{SWITCH\_x\_DEV}{SWITCHxDEV}

Specify the network interface the switch is connected to. Note that via the 
variable \var{NET\_DRV\_x} the matching switch driver has to be loaded. For the 
Banana Pi R-1 i.e. use ``b53\_mdio.ko''.

Example: \verb+SWITCH_1_DEV='eth0'+

\config{SWITCH\_x\_VLAN\_N}{SWITCH\_x\_VLAN\_N}{SWITCHxVLANN}

Here the number of logical networks is given, in which the switch should 
be partitioned. These networks are internally mapped to VLANs.

Example: \verb+SWITCH_1_VLAN_N='2'+

\config{SWITCH\_x\_VLAN\_y\_ID}{SWITCH\_x\_VLAN\_y\_ID}{SWITCHxVLANyID}

This variable contains the IDs of the VLANs to be defined. A VLAN must
be defined by the VLAN-OPT, so the fli4l may gain access to it.

Example: \verb+SWITCH_1_VLAN_1_ID='100'+

\config{SWITCH\_x\_VLAN\_y\_PORT\_N}{SWITCH\_x\_VLAN\_y\_PORT\_N}{SWITCHxVLANyPORTN}

Here the number of network ports is specified, which are part of this VLAN.

Example: \verb+SWITCH_1_VLAN_1_PORT_N='3'+

\config{SWITCH\_x\_VLAN\_y\_PORT\_z\_ID}{SWITCH\_x\_VLAN\_y\_PORT\_z\_ID}{SWITCHxVLANyPORTzID}

This variable contains the ID of the network port to be used. These IDs
are hardware-specific and may not be generally derived.
The Banana Pi R-1 numbers its ports as follows from left to right:
2 1 0 4 3.

Example:

\begin{example}
\begin{verbatim}
    SWITCH_1_VLAN_1_PORT_1_ID='0'
    SWITCH_1_VLAN_1_PORT_2_ID='1'
    SWITCH_1_VLAN_1_PORT_3_ID='2'
\end{verbatim}
\end{example}

\config{SWITCH\_x\_VLAN\_y\_PORT\_z\_MODE}{SWITCH\_x\_VLAN\_y\_PORT\_z\_MODE}{SWITCHxVLANyPORTzMODE}

Here you can specify whether in network packets of this (!) VLAN the
VLAN tag should be removed (mode ``untagged'') or preserved (mode ``tagged''). 
Typically one wants to remove the VLAN tags, thus the connected devices have 
no knowledge about the configured VLANs.

It should be noted that a network port can only be configured for \emph{one} 
VLAN to be ``untagged'', otherwise assignment of untagged incoming network 
packets to a VLAN is no longer possible. However, a network port can be
additionally (or exclusively) configured as ``tagged'' in multiple VLANs.

Example:

\begin{example}
\begin{verbatim}
    SWITCH_1_VLAN_1_PORT_1_MODE='untagged'
    SWITCH_1_VLAN_1_PORT_2_MODE='untagged'
    SWITCH_1_VLAN_1_PORT_3_MODE='untagged'
\end{verbatim}
\end{example}

\end{description}

\subsubsection{Use case}

The following configuration for the Banana Pi R-1 creates three logical
Networks. The first two ports form VLAN 100, the next two VLAN
101. The fifth port is the WAN port and the only port of VLAN 102. All
network ports work in ``untagged'' mode because the VLANs are only used 
internally for packet assignment.

\begin{example}
\begin{verbatim}
    NET_DRV_N='1'
    NET_DRV_1='b53_mdio'
    NET_DRV_1_OPTION=''
    [...]
    OPT_VLAN_DEV='yes'
    VLAN_DEV_N='3'
    VLAN_DEV_1_DEV='eth0'
    VLAN_DEV_1_VID='100'
    VLAN_DEV_2_DEV='eth0'
    VLAN_DEV_2_VID='101'
    VLAN_DEV_3_DEV='eth0'
    VLAN_DEV_3_VID='102'
    [...]
    OPT_SWITCH='yes'
    SWITCH_N='1'
    SWITCH_1_DEV='eth0'
    SWITCH_1_VLAN_N='3'
    SWITCH_1_VLAN_1_ID='100'
    SWITCH_1_VLAN_1_PORT_N='2'
    SWITCH_1_VLAN_1_PORT_1_ID='2'
    SWITCH_1_VLAN_1_PORT_1_MODE='untagged'
    SWITCH_1_VLAN_1_PORT_2_ID='1'
    SWITCH_1_VLAN_1_PORT_2_MODE='untagged'
    SWITCH_1_VLAN_2_ID='101'
    SWITCH_1_VLAN_2_PORT_N='2'
    SWITCH_1_VLAN_2_PORT_1_ID='0'
    SWITCH_1_VLAN_2_PORT_1_MODE='untagged'
    SWITCH_1_VLAN_2_PORT_2_ID='4'
    SWITCH_1_VLAN_2_PORT_2_MODE='untagged'
    SWITCH_1_VLAN_3_ID='102'
    SWITCH_1_VLAN_3_PORT_N='1'
    SWITCH_1_VLAN_3_PORT_1_ID='3'
    SWITCH_1_VLAN_3_PORT_1_MODE='untagged'
\end{verbatim}
\end{example}

\subsection{ETHTOOL - Settings for Ethernet Network Adapters}
\configlabel{OPT\_ETHTOOL}{OPTETHTOOL}

By setting \var{OPT\_ETHTOOL='yes'} the ethtool program will be copied 
to the fli4l router in order to be used by other packages.
By the help of this program, various settings of Ethernet network 
cards and drivers can be displayed and changed.

\begin{description}

\config{ETHTOOL\_DEV\_N}{ETHTOOL\_DEV\_N}{ETHTOOLDEVN}{
Specify the number of settings to set at boot time.

Default: \var{ETHTOOL\_DEV\_N='0'}
}

\config{ETHTOOL\_DEV\_x}{ETHTOOL\_DEV\_x}{ETHTOOLDEVx}{
\var{ETHTOOL\_DEV\_x} indicates for which network device the settings
should apply.

Example: \var{ETHTOOL\_DEV\_1='eth0'}
}

\config{ETHTOOL\_DEV\_x\_OPTION\_N}{ETHTOOL\_DEV\_x\_OPTION\_N}{ETHTOOLDEVxOPTIONN}{
\var{ETHTOOL\_DEV\_x\_OPTION\_N} indicates the number of settings for the device.
}

\config{ETHTOOL\_DEV\_x\_OPTION\_x\_NAME}{ETHTOOL\_DEV\_x\_OPTION\_x\_NAME}{ETHTOOLDEVxOPTIONxNAME}{}
\config{ETHTOOL\_DEV\_x\_OPTION\_x\_VALUE}{ETHTOOL\_DEV\_x\_OPTION\_x\_VALUE}{ETHTOOLDEVxOPTIONxVALUE}{

The variable \var{ETHTOOL\_DEV\_x\_OPTION\_x\_NAME} gives the name and
\var{ETHTOOL\_DEV\_x\_OPTION\_x\_VALUE} the value of the setting to be changed.

Following is a list of options and possible values activated by now:
\begin{itemize}
  \item speed 10|100|1000|2500|10000 expandable by HD or FD (default FD = full duplex)
  \item autoneg on|off
  \item advertise \%x
  \item wol p|u|m|b|a|g|s|d
\end{itemize}
}
\end{description}

Example:

\begin{example}
\begin{verbatim}
OPT_ETHTOOL='yes'
ETHTOOL_DEV_N='2'
ETHTOOL_DEV_1='eth0'
ETHTOOL_DEV_1_OPTION_N='1'
ETHTOOL_DEV_1_OPTION_1_NAME='wol'
ETHTOOL_DEV_1_OPTION_1_VALUE='g'
ETHTOOL_DEV_2='eth1'
ETHTOOL_DEV_2_OPTION_N='2'
ETHTOOL_DEV_2_OPTION_1_NAME='wol'
ETHTOOL_DEV_2_OPTION_1_VALUE='g'
ETHTOOL_DEV_2_OPTION_2_NAME='speed'
ETHTOOL_DEV_2_OPTION_2_VALUE='100hd'
\end{verbatim}
\end{example}

Further informations about ethtool can be found here:
\altlink{http://linux.die.net/man/8/ethtool}
  
\subsection{Example}

For understanding a simple example is certainly helpful. In our example 
we assume 2 parts of a building which are connected by 2~x~100~Mbit/s lines.
Four separate networks should be routed from one building to the other.

To achieve this a combination of bonding (joining the two physical lines) 
VLAN (to transport several separate networks on the bond) and bridging 
(to link the different nets to the bond/VLAN) is used. This has been tested 
successful on 2 Intel e100 cards and 1 Adaptec 4-port card ANA6944 in each 
building's router.
The two e100 have the device names \var{'eth0'} and \var{'eth1'}. They 
are used for connecting the building. Intel e100's are the only cards known 
to work flawlessly with VLAN by now. Gigabit-cards should work too.
The 4 ports of the multiport-card are used for the networks and have device
names \var{'eth2'} to \var{'eth5'}.

At first the two 100~Mbit/s lines will be bonded:

\begin{example}
\begin{verbatim}
OPT_BONDING_DEV='yes'
BONDING_DEV_N='1'
BONDING_DEV_1_DEVNAME='bond0'
BONDING_DEV_1_MODE='balance-rr'
BONDING_DEV_1_DEV_N='2'
BONDING_DEV_1_DEV_1='eth0'
BONDING_DEV_1_DEV_2='eth1'
\end{verbatim}
\end{example}

This creates the device \var{'bond0'}. Now the two VLANs will 
be built on this bond. We use VLAN-IDs 11, 22, 33 und 44:

\begin{example}
\begin{verbatim}
OPT_VLAN_DEV='yes'
VLAN_DEV_N='4'
VLAN_DEV_1_DEV='bond0'
VLAN_DEV_1_VID='11'
VLAN_DEV_2_DEV='bond0'
VLAN_DEV_2_VID='22'
VLAN_DEV_3_DEV='bond0'
VLAN_DEV_3_VID='33'
VLAN_DEV_4_DEV='bond0'
VLAN_DEV_4_VID='44'
\end{verbatim}
\end{example}

Over this two VLAN connections the bridge into the networks segments will be 
built. Routing is not necessary this way.

\begin{example}
\begin{verbatim}
OPT_BRIDGE_DEV='yes'
BRIDGE_DEV_N='4'
BRIDGE_DEV_1_NAME='_VLAN11_'
BRIDGE_DEV_1_DEVNAME='br11'
BRIDGE_DEV_1_DEV_N='2'
BRIDGE_DEV_1_DEV_1='bond0.11'
BRIDGE_DEV_1_DEV_2='eth2'
BRIDGE_DEV_2_NAME='_VLAN22_'
BRIDGE_DEV_2_DEVNAME='br22'
BRIDGE_DEV_2_DEV_N='2'
BRIDGE_DEV_2_DEV_1='bond0.22'
BRIDGE_DEV_2_DEV_2='eth3'
BRIDGE_DEV_3_NAME='_VLAN33_'
BRIDGE_DEV_3_DEVNAME='br33'
BRIDGE_DEV_3_DEV_N='2'
BRIDGE_DEV_3_DEV_1='bond0.33'
BRIDGE_DEV_3_DEV_2='eth4'
BRIDGE_DEV_4_NAME='_VLAN44_'
BRIDGE_DEV_4_DEVNAME='br44'
BRIDGE_DEV_4_DEV_N='2'
BRIDGE_DEV_4_DEV_1='bond0.44'
BRIDGE_DEV_4_DEV_2='eth5'
\end{verbatim}
\end{example}

As a result all 4 Nets are connected with each other absolutely 
transparent and share the 200~Mbit/s connection. Even with a failure
of one 100~Mbit/s line the connection will not fail. If necessary
EBTables support can also be activated e.g. to activate certain packet filter.

This configuration is set up on two fli4l routers. I think this is an 
impressive example what the advanced\_networking package can do.

