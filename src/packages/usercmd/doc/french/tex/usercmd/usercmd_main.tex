% Do not remove the next line
% Synchronized to r31455

\marklabel{sec:opt-usercmd}
{
\section {OPT\_USERCMD~- Exécution de commandes personnalisées au début/fin du système}
}

\subsection{Configuration}

Avec le paquetage opt\_usercmd vous pouvez exécuter des commandes et des
scripts après le démarrage ou à l'arrêt du système selon votre choix.
Voici des exemples possibles~:

\begin{itemize}
    \item Un montage supplémentaire de fichier système (en particulier pour CD bootable).
    \item L'ajout d'une route supplémentaire en dehors du cadre de base.
    \item L'envoie d'un ping à un hôte pour déconnecter (automatiquement un système
      en ligne avec \var{DIALMODE}='auto').
\end{itemize}

En outre, il est possible d'utiliser vos propres fichiers depuis le répertoire
de configuration, ils seront intégrés dans l'image de fli4l.

Pour la configuration on utilise les variables suivantes~:

\begin{description}

\config{OPT\_USERCMD}{OPT\_USERCMD}{OPTUSERCMD}

        Configuration par défaut~: \var{OPT\_USERCMD}='no'

    Si vous paramétrez la variable \var{OPT\_USERCMD}='yes' le paquetage sera
    activé. Vous pourrez alors paramétrer les commandes et les scripts
    supplémentaires au démarrage ou à l'arrêt du système.

\config{USERCMD\_BOOT\_N}{USERCMD\_BOOT\_N}{USERCMDBOOTN}

        Configuration par défaut~: \var{USERCMD\_BOOT\_N}='0'

    Vous indiquez dans cette variable le nombre de commande/script qui sera
    activé au démarrage du système (boot). Les variables peuvent être traitées
    consécutivements.

\config{USERCMD\_BOOT\_x}{USERCMD\_BOOT\_x}{USERCMDBOOTx}

        Configuration par défaut~: \var{USERCMD\_BOOT\_x}='echo user-defined boot-command here'

    Instruction à exécuter au démarrage du système.

    \wichtig{Normalement une seule commande peut être saisie dans chaque
    variable. Si vous indiquez plusieurs commandes à la suite, ils doivent être
    séparés par un point-virgule. (Voir les exemples)}

\config{USERCMD\_HALT\_N}{USERCMD\_HALT\_N}{USERCMDHALTN}

        Configuration par défaut~: \var{USERCMD\_HALT\_N}='0'

    Vous indiquez dans cette variable le nombre de commande/script qui sera
    activé à l'arrêt du système (halt, shutdown, reboot). Les variables peuvent
    être traitées consécutivement.

\config{USERCMD\_HALT\_x}{USERCMD\_HALT\_x}{USERCMDHALTx}

        Configuration par défaut~: \var{USERCMD\_HALT\_x}='echo user-defined halt-command here'

    Commande à exécuter lors de l'arrêt du système.

    \wichtig{Normalement une seule commande peut être saisie dans chaque
    variable. Si vous indiquez plusieurs commandes à la suite, ils doivent être
    séparés par un point-virgule. (Voir les exemples)}

\config{USERCMD\_FILE\_N}{USERCMD\_FILE\_N}{USERCMDFILEN}

        Configuration par défaut~: \var{USERCMD\_FILE\_N}='0'

   Parfois, il est nécessaire d'inclure vos propres fichiers dans l'image
   opt.img de fli4l (il n'est pas possible d'intégrer des fichiers dans l'image
   rootfs.img), sans créer un paquetage-opt suppémentaire. Dans ce cas, un petit
   nombre de fichiers peut être inclus directement dans l'image fli4l ils seront
   placés dans le répertoire \var{$<$config$>$/etc/usercmd}.
   
   Dans la variable \var{USERCMD\_FILE\_N}='x' vous indiquez le nombre de fichiers
   à intégrer dans l'image fli4l.

\config{USERCMD\_FILE\_x\_SRC}{USERCMD\_FILE\_x\_SRC}{USERCMDFILExSRC}

	Dans cette variable vous indiquez le nom du fichier source qui sera placé dans
	le répertoire de configuration \var{$<$config$>$/etc/usercmd}. Seuls les fichiers
	résidant dans ce répertoire seront intégrés~!

\config{USERCMD\_FILE\_x\_DST}{USERCMD\_FILE\_x\_DST}{USERCMDFILExDST}

	Dans cette variable vous indiquez le nom du fichier absolu, comme il doit être
	exactement inclus dans l'image fli4l qui sera ensuite générée. Vous pouvez par
	exemple spécifié \var{USERCMD\_BOOT\_x}='/usr/bin/mystuff.sh', ce fichier sera
	alors appelé lors du démarrage de fli4l.

\config{USERCMD\_FILE\_x\_MODE}{USERCMD\_FILE\_x\_MODE}{USERCMDFILExMODE}

    Dans cette variable vous indiquez le droits d'accès du fichier qui sera intégré
	au fichier image de fli4l. Les droits d'accès des fichier Unix sont en fonction
	des conventions habituelles de Unix. Vous pouvez trouver des détails dans la
	documentation pour développeur (voir la section \ref{sec:opt_txt}) ou consulter
	le site Wikipedia \altlink{http://fr.wikipedia.org/wiki/Permissions_UNIX}.

\config{USERCMD\_FILE\_x\_FLAGS}{USERCMD\_FILE\_x\_FLAGS}{USERCMDFILExFLAGS}

    Dans cette variable vous indiquez le flags= du fichier conformément au 
	opt/<package>.txt, voir aussi \ref{sec:Dateiformate}.
	
	Les options suivantes existent pour convertir le format du fichier texte avant
	de l'inclusion dans l'image~:
    \newline\newline
    \begin{tabular}{lp{6cm}}
        \emph{utxt} & Conversion au format Unix\\
        \emph{dtxt} & Conversion au format DOS\\
        \emph{sh}   & Shell-Skript: Conversion au format Unix, avec la suppression des
		caractères inutiles, recommandé pour les scripts shell
    \end{tabular}

\end{description}

\subsection{Exemples}

\begin{verbatim}
USERCMD_BOOT_x='fli4lctrl dial pppoe'
\end{verbatim}

Active une connexion Internet DSL à la fin du processus de boot. 

\begin{verbatim}
USERCMD_BOOT_x='sleep 60; ip link set tr0 down; ip link set tr0 up'
\end{verbatim}

On définit au départ du processus de boot une pause d'une minute, puis on monte
l'interface tr0, ensuite on descend l'interface, on remonte à nouveau l'interface.
Peut être utilisé pour laisser du temps au démarrage un switch Token Ring et
ensuite au démarrage du réseau.

\begin{verbatim}
USERCMD_BOOT_N='1'
USERCMD_BOOT_1='cp /data/log/imond.log /var/log/'
USERCMD_HALT_N='2'
USERCMD_HALT_1='cp /var/log/sys.log /data/log/sys`date +%Y%m%d`.log'
USERCMD_HALT_2='fli4lctrl hangup pppoe ; sleep 2 ; cp /var/log/imond.log /data/log/'
\end{verbatim}

On copie le fichier imond.log du répertoire de sauvegarder vers un répertoire à
la fin du processus de boot. Ensuite on copie le fichier sys.log avec la date
actuelle dans un répertoire de sauvegarder avant l'arrêt du système, pour
finir la connexion DSL est coupé et on sauvegarde le fichier imond.log

\subsection{Support}

Un soutien est accordé sur ce paquetage uniquement dans le groupe
de discussion fli4l.

\subsection{Observation}

\begin{itemize}
    \item L'utilisation du caractère \flqq{}'\frqq{} n'est pas possible dans
      une ligne de commande.
    \item Une ligne de commande ne peut pas commencer par le caractère \flqq{}-\frqq{}.
\end{itemize}


% Diese Dokumentation wurde dankenswerterweise bereitgestellt von Oliver Musch (23 Feb. 2005).
% Überarbeitet und erweitert hat sie Michael Köhler (3. Jan. 2005).
