% Synchronized to r56820
\section {HWSUPP - Hardware support}

\subsection {Description}

This package supplies the support for special hardware components.

Supported are:
\begin{itemize}
\item Temperature sensors
\item LEDs
\item Voltage sensors
\item Fan speed
\item Buttons
\item Watchdog
\item VPN cards
\end{itemize}

The following systems, mainboards and VPN cards are supported:
\begin{itemize}
  \item Standard PC hardware
  \begin{itemize}
    \item PC keyboard LEDs
  \end{itemize}
  \item ACPI Hardware
  \item Embedded systems
  \begin{itemize}
    \item AEWIN SCB6971
    \item Fujitsu Siemens Futro S200
    \item Fujitsu Siemens Futro S900
    \item PC Engines ALIX
    \item PC Engines APU
    \item PC Engines WRAP
    \item Soekris net4801
    \item Soekris net5501
  \end{itemize}
  \item Mainboards
  \begin{itemize}
    \item Commell LE-575
    \item GigaByte GA-M521-S3
    \item GigaByte GA-N3150N-D3V
    \item LEX CV860A
    \item MSI MS-9803
    \item SuperMicro PDSME
    \item SuperMicro X7SLA
    \item Tyan S5112
    \item WinNet PC640
    \item WinNet PC680
  \end{itemize}
    \item VPN cards (PCI, miniPCI and miniPCIe)
  \begin{itemize}
    \item vpn1401 vpn1411
  \end{itemize}
\end{itemize}


\subsection {Configuration of the HWSUPP package}

  The configuration is made, as for all fli4l packages, by adjusting the file\\
  \var{path/fli4l-\version/$<$config$>$/hwsupp.txt} to meet your own demands.

\begin{description}

\config {OPT\_HWSUPP}{OPT\_HWSUPP}{OPTHWSUPP}

  The setting \var{'no'} deactivates the OPT\_HWSUPP package completely. There
  will be no changes made to the fli4l boot medium or the archive \var{opt.img}.
  OPT\_HWSUPP does not overwrite any other parts of the fli4l installation.
  To activate OPT\_HWSUPP set the variable \var{OPT\_HWSUPP} to \var{'yes'}.

\config {HWSUPP\_TYPE}{HWSUPP\_TYPE}{HWSUPPTYPE}

  This configuration variable sets the type of supported hardware.
  Following values can be used:
  \begin{itemize}
    \item sim
    \item generic-pc
    \item generic-acpi
    \item generic-acpi-coretemp
    \item aewin-scb6971
    \item commell-le575
    \item fsc-futro-s200
    \item fsc-futro-s900
    \item gigabyte-ga-m52l-s3
    \item gigabyte-ga-n3150n-d3v
    \item lex-cv860a
    \item msi-ms-9803
    \item pcengines-alix
    \item pcengines-apu (APU-1)
    \item pcengines-apu2 (APU-2)
    \item pcengines-wrap
    \item rpi
    \item soekris-net4801
    \item soekris-net5501
    \item supermicro-pdsme
    \item supermicro-x7sla
    \item tyan-s5112
    \item winnet-pc640
    \item winnet-pc680
  \end{itemize}

\config {HWSUPP\_WATCHDOG}{HWSUPP\_WATCHDOG}{HWSUPPWATCHDOG}
  The setting \var{'yes'} activates the watchdog daemon if the hardware
  contains a watchdog. The watchdog will automatically restart a non responding system.

\config {HWSUPP\_CPUFREQ}{HWSUPP\_CPUFREQ}{HWSUPPCPUFREQ}
  The setting \var{'yes'} activates CPU frequency adjustment controls.

\config {HWSUPP\_CPUFREQ\_GOVERNOR}{HWSUPP\_CPUFREQ\_GOVERNOR}{HWSUPPCPUFREQGOVERNOR}
  Selection of CPU frequency governor. The selected governor controls the frequency
  adjustment behaviour. It's a selection of one of: 

  \begin{itemize}
    \item performance\\
      The CPU allways runs with the highest available frequency.
    \item ondemand\\
      The CPU frequency will be adjusted depending on the current CPU usage.
      The frequency can change very quickly. 
    \item conservative\\
      The CPU frequency will be adjusted depending on the current CPU usage.
      The frequency is changed step by step. 
    \item powersave\\
      The CPU allways runs with the lowest available frequency.
    \item userspace\\
      The CPU frequence kann be set manually or by an user script via the
      sysfs variable \var{/devices/system/cpu/cpu\emph{<n>}/cpufreq/scaling\_setspeed}.
  \end{itemize}

\config {HWSUPP\_LED\_N}{HWSUPP\_LED\_N}{HWSUPPLEDN}
  Defines the number of LEDs. The number of LEDs of the hardware in use
  should be entered here.

\config {HWSUPP\_LED\_x}{HWSUPP\_LED\_x}{HWSUPPLEDx}
  Defines the information indicated by the LED.
  The following informations are possible:
  \begin{itemize}
    \item ready - the fli4l router ist ready for operation\footnotemark
    \item online - the fli4l router has an active internet connection
    \item trigger - LED is controlled by a kernel trigger
    \item user - LED is controlled by an user script
  \end{itemize}

  \footnotetext{If \var{HWSUPP\_LED\_x='ready'} is set, the boot progress
  is indicated by a blink sequence (see appendix \ref{sec:blink}).}

  The list of possible indications can be extended by other packages.
  For example, if the WLAN package is loaded the information 
  \begin{itemize}
    \item wlan - WLAN is active
  \end{itemize}
  is possible.

  In apppendix \ref{sec:developer} package developers can get some hints 
  on how to create such extensions. 
  
\config {HWSUPP\_LED\_x\_DEVICE}{HWSUPP\_LED\_x\_DEVICE}{HWSUPPLEDxDEVICE}
  Specifies the LED device.

  Here you either have to enter a LED device (to be found at \var{/sys/class/leds/}
  in the router's file system) or a GPIO\footnotemark number.

  \footnotetext{A GPIO (General Purpose Input/Output) is a generic pin on an integrated
  circuit whose behavior can be programmed at run time, including whether it is an
  input or output pin.}

  A list of valid LED device names for a specific \var{HSUPP\_TYPE} can be
  found in the appendix \jump{sec:leddevices}{``Available LED devices''}.

  The GPIO number has to be enterd in the format \var{gpio::x}.
  If a GPIO is entered, the corresponding LED device will be created
  automatically. By preceding the char \var{/} the GPIO functionality may be inverted.

  Examples:
  \begin{verbatim}
    HWSUPP_LED_1_DEVICE='apu::1'      # LED 1 on PC engines APU
    HWSUPP_LED_2_DEVICE='gpio::237'   # GPIO 237
    HWSUPP_LED_3_DEVICE='/gpio::245'  # inverted GPIO 245
    HWSUPP_LED_4_DEVICE='led0'
  \end{verbatim}

\config {HWSUPP\_LED\_x\_PARAM}{HWSUPP\_LED\_x\_PARAM}{HWSUPPLEDxPARAM}
  Defines parameters for the selected LED information.

  Depending on the selection in in \var{HWSUPP\_LED\_x}, 
  in \var{HWSUPP\_LED\_x\_PARAM} different settings are possible.

  If \var{HWSUPP\_LED\_x='trigger'} is set, the trigger name has to be specified
  in \var{HWSUPP\_LED\_x\_PARAM}.

  Available triggers can be displayed with the shell command 
  \var{cat /sys/class/leds/*/trigger}. 

  Besides triggers created by e.g. netfilter or hardware drivers like ath9k,
  further trigger modules can be loaded via \var{HWSUPP\_DRIVER\_x}.

  Examples:
  \begin{verbatim}
    HWSUPP_LED_1='trigger'
    HWSUPP_LED_1_TRIGGER='heartbeat'
    HWSUPP_LED_2='trigger'
    HWSUPP_LED_2_TRIGGER='netfilter-ssh'
  \end{verbatim}

  If \var{'HWSUPP\_LED\_x'} has the value \var{'user'} 
  in \var{HWSUPP\_LED\_PARAM} a valid script name including path has to be entered. 

  Example:
  \begin{verbatim}
    HWSUPP_LED_1='user'
    HWSUPP_LED_1_PARAM='/usr/local/bin/myledscript'
  \end{verbatim}

  When \var{HWSUPP\_LED\_x='wlan'} is set, the WLAN devices have to be entered in 
  \var{HWSUPP\_LED\_x\_PARAM}.
  
  Defines one ore more WLAN devices, whose state shall be displayed.
  Multiple WLAN devices have to be separated by spaces.

  When the state of multiple WLAN devices should be indicated by a single LED, 
  the LED has the following meaning:
  \begin{itemize}
    \item off - all WLAN devices are inactiv
    \item blinking - some WLAN device(s) is/are active
    \item on - all WLAN devices are active
  \end{itemize} 

  Example:
  \begin{verbatim}
    HWSUPP_LED_1='wlan'
    HWSUPP_LED_1_WLAN='wlan0 wlan1'
  \end{verbatim}


\config {HWSUPP\_BOOT\_LED}{HWSUPP\_BOOT\_LED}{HWSUPPBOOTLED}
  Definies a LED to indicate the boot progress by a blink sequence. 
  
  When \var{HWSUPP\_LED\_x='ready'} is set for any LED, this setting is used
  and \var{HWSUPP\_BOOT\_LED} will be ignored.

\config {HWSUPP\_BUTTON\_N}{HWSUPP\_BUTTON\_N}{HWSUPPBUTTONN}
  Defines the number of buttons.\\ The number of buttons of the hardware in use 
  should be entered here.

\config {HWSUPP\_BUTTON\_x}{HWSUPP\_BUTTON\_x}{HWSUPPBUTTONx}
  Defines the action which should be executed on button press.\\
  The following actions are supported:
  \begin{itemize}
    \item reset - restart the fli4l router
    \item online - causes an internet dialin or terminates an internet connection.
    \item user - an user script will be executed
  \end{itemize}

  Th
e list of possible actions can be extended by other packages.
  If the WLAN package is loaded, eg. the action 
  \begin{itemize}
    \item wlan - activate or deactivate WLAN
  \end{itemize}
  is possible.

\config {HWSUPP\_BUTTON\_x\_DEVICE}{HWSUPP\_BUTTON\_x\_DEVICE}{HWSUPPBUTTONxDEVICE}
  Specifies the button device. Here two possibilities exist. Either GPIO numbers 
  can be entered or a so-called input path is specified, that gets detected via 
  the evdev subsystem later on. The second is, if possible, always to be preferred 
  because the GPIO indexes may not remain the same for different kernels.

  The GPIO number must be entered in the format \var{gpio::x}. Usually it is 
  assumed that the GPIO pins are ``active-low'', meaning that they the have value 
  0 assigned when a button is pressed, and 1 if the button is not. In the reversed 
  case ``active-high'' (i.e. 1 when button is pressed and 0 when not), this can be 
  indicated by prefixing with \texttt{/}.

  The input path has the format \var{evdev:<Path>}. In this case,
  \var{HWSUPP\_BUTTON\_x\_DEVICE\_KEY} indicates which key is to be processed. This is 
  necessary, because in contrast to the connection via GPIO pin one evdev device may 
  incorporate many different buttons. An overview on the keys can be found in appendix 
  \jump{sec:hwsupp:evdev-keys}{``Keycodes''}

  A list of predefined input paths for the \var{HSUPP\_TYPE} in question 
  can be found in appendix \jump{sec:buttondevices}{``Available Button Devices''}.

  Examples:
  \begin{verbatim}
    HWSUPP_BUTTON_1_DEVICE='/gpio::237' # GPIO line #237, active-high
    HWSUPP_BUTTON_2_DEVICE='evdev:isa0060/serio0/input0' # AT keyboard,
    HWSUPP_BUTTON_2_DEVICE_KEY='88'                      # F12 key
  \end{verbatim}

\config {HWSUPP\_BUTTON\_x\_PARAM}{HWSUPP\_BUTTON\_x\_PARAM}{HWSUPPBUTTONxPARAM}
  Defines parameters for the action selected in \var{HWSUPP\_BUTTON\_x}.
  
  Depending on the action \var{HWSUPP\_BUTTON\_x\_PARAM} has different meanings.

  If \var{HWSUPP\_BUTTON\_x='user'} is set, 
  \var{HWSUPP\_BUTTON\_x\_PARAM} defines a script to be executed 
  on button press.
  
  Example:
  \begin{verbatim}
    HWSUPP_BUTTON_1='user'
    HWSUPP_BUTTON_2_WLAN='/usr/local/bin/myscript'
  \end{verbatim}

% ##TODO## ? move to wlan doc begin
  If  \var{HWSUPP\_BUTTON\_x} is set to \var{'wlan'}, the 
  \var{HWSUPP\_BUTTON\_x\_PARAM} defines one ore more WLAN devices, 
  which shall be activated or deactivated on  button press.
  Multiple WLAN devices have to be separated by spaces.
  
  Example:
  \begin{verbatim}
    HWSUPP_BUTTON_2='wlan'
    HWSUPP_BUTTON_2_WLAN='wlan0 wlan1'
  \end{verbatim}
% ##TODO## ? move to wlan doc end

\subsection{Expert settings}
  The following settings should only be touched if you know exactly
  \begin{itemize}
    \item which hardware you have,
    \item which additional drivers it needs and
    \item the addresses and types of I\textsuperscript{2}C\footnotemark  devices.
    \footnotetext{An I\textsuperscript{2}C bus or SMBus is a serial bus used in
    PCs eg. to read temperature sensor values.
    In many cases an I\textsuperscript{2}C bus or SMBus is available on a pin
    header and can be used for own hardware extensions.}

  \end{itemize}
  Activating the expert settings will issue a warning during the mkfli4l build.

\config {HWSUPP\_DRIVER\_N}{HWSUPP\_DRIVER\_N}{HWSUPPDRIVERN}
  Number of additional drivers.
  The drivers in \var{HWSUPP\_DRIVER\_x} will be loaded in the denoted order.

\config {HWSUPP\_DRIVER\_x}{HWSUPP\_DRIVER\_x}{HWSUPPDRIVERx}
  Driver name (without file extension \var{.ko}).

Example:
\begin{verbatim}
HWSUPP_DRIVER_N='2'
HWSUPP_DRIVER_1='i2c-piix4'     # I2C bus driver
HWSUPP_DRIVER_2='gpio-pcf857x'  # I2C GPIO expander
\end{verbatim}

\config {HWSUPP\_I2C\_N}{HWSUPP\_I2C\_N}{HWSUPPI2CN}
  Number of I\textsuperscript{2}C devices to be loaded.

  I\textsuperscript{2}C doesn't support any PnP mechanismn.
  Hence for each I\textsuperscript{2}C device the bus number,
  the device address and the device type have to be specified.

\config {HWSUPP\_I2C\_x\_BUS}{HWSUPP\_I2C\_x\_BUS}{HWSUPPI2CxBUS}
  I\textsuperscript{2}C bus number the device is attached to.

  The bus number has to be entered as \var{i2c-x}.

\config {HWSUPP\_I2C\_x\_ADDRESS}{HWSUPP\_I2C\_x\_ADDRESS}{HWSUPPI2CxADDRESS}
  The device's I\textsuperscript{2}C address.

  The address has to be entered as a hex number in the range between \var{0x03} and \var{0x77}.

\config {HWSUPP\_I2C\_x\_DEVICE}{HWSUPP\_I2C\_x\_DEVICE}{HWSUPPI2CxDEVICE}
  The type of I\textsuperscript{2}C device which is supported by an already
  loaded driver.

Example:
\begin{verbatim}
HWSUPP_I2C_N='1'
HWSUPP_I2C_1_BUS='i2c-1'
HWSUPP_I2C_1_ADDRESS='0x38'
HWSUPP_I2C_1_DEVICE='pcf8574a' # supported by gpio-pcf857x driver
\end{verbatim}

\subsection {Support for VPN cards}

\config {OPT\_VPN\_CARD}{OPT\_VPN\_CARD}{OPTVPNCARD}
  The setting \var{'no'} deactivates the OPT\_VPN\_CARD package completely. There
  will be no changes made to the fli4l boot mediums or the archive \var{opt.img}.
  OPT\_VPN\_CARD does not overwrite any other parts of the fli4l installation.
  To activate OPT\_VPN\_CARD set the variable \var{OPT\_VPN\_CARD} to \var{'yes'}.

\config {VPN\_CARD\_TYPE}{VPN\_CARD\_TYPE}{VPNCARDTYPE}

  This configuration variable defines the type of the VPN accelerator.
  The following values are supported:
  \begin{itemize}
    \item hifn7751 - Soekris vpn1401 and vpn1411
    \item hifnhipp
  \end{itemize}

\end{description}
