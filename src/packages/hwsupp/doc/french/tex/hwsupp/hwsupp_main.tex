% Do not remove the next line
% Synchronized to r56820
\section {HWSUPP~- Supporte du matériel spécifique}

\subsection {Description}

Ce paquetage fournit un support pour l'utilisation de composants et de matériels
spécifiques.

Matériels et composants prisent en charges~:
\begin{itemize}
\item Capteurs de température
\item LEDs
\item Capteurs de tension
\item Vitesse des ventilateurs
\item Bouton poussoir
\item Watchdog (ou chien de garde)
\item Carte VPN
\end{itemize}

Il prend en charge aussi les systémes/cartes mère/cartes VPN~:
\begin{itemize}
  \item Matériel pour PC standard
  \begin{itemize}
    \item LEDs et bouton d'ordinateur
  \end{itemize}
  \item Matériel ACPI
  \item Système embarqué
  \begin{itemize}
    \item AEWIN SCB6971
    \item Fujitsu Siemens Futro S200
    \item Fujitsu Siemens Futro S900
    \item PC Engines ALIX
    \item PC Engines APU
    \item PC Engines WRAP
    \item Soekris net4801
    \item Soekris net5501
  \end{itemize}
  \item Carte mère
  \begin{itemize}
    \item Commell LE-575
    \item GigaByte GA-M521-S3
    \item GigaByte GA-N3150N-D3V
    \item LEX CV860A
    \item MSI MS-9803
    \item SuperMicro PDSME
    \item SuperMicro X7SLA
    \item Tyan S5112
    \item WinNet PC640
    \item WinNet PC680
  \end{itemize}
    \item Carte VPN (PCI, miniPCI et miniPCIe)
  \begin{itemize}
    \item vpn1401 vpn1411
  \end{itemize}
\end{itemize}

\subsection {Configuration du paquetage HWSUPP}

  La configuration se fait comme les autres paquetages fli4l, en paramètrant\\
  le fichier \var{Pfad/fli4l-\version/$<$config$>$/hwsupp.txt} selon votre
  propre configuration.

\begin{description}

\config {OPT\_HWSUPP}{OPT\_HWSUPP}{OPTHWSUPP}

  La valeur \var{'no'} dans cette variable désactive complètement le paquetage
  OPT\_HWSUPP. Il n'y aura aucun changement sur le support de boot de l'archive
  fli4l \var{rootfs.img} n'y dans l'archive \var{opt.img}. Pour finir OPT\_HWSUPP
  n'écrase aucune partie de l'installation fli4l.\\
  Pour activer la variable OPT\_HWSUPP du paquetage \var{OPT\_HWSUPP} vous devez
  placer la valeur sur \var{'yes'}.

\config {HWSUPP\_TYPE}{HWSUPP\_TYPE}{HWSUPPTYPE}

  Dans cette variable vous indiquez le matériel à configurer.
  Les valeurs suivantes sont disponibles~:
  \begin{itemize}
    \item sim
    \item generic-pc
    \item generic-acpi
    \item generic-acpi-coretemp
    \item aewin-scb6971
    \item commell-le575
    \item fsc-futro-s200
    \item fsc-futro-s900
    \item gigabyte-ga-m52l-s3
    \item gigabyte-ga-n3150n-d3v
    \item lex-cv860a
    \item msi-ms-9803
    \item pcengines-alix
    \item pcengines-apu (APU-1)
    \item pcengines-apu2 (APU-2)
    \item pcengines-wrap 
    \item rpi
    \item soekris-net4801
    \item soekris-net5501
    \item supermicro-pdsme
    \item supermicro-x7sla
    \item tyan-s5112
    \item winnet-pc640
    \item winnet-pc680 
  \end{itemize}

\config {HWSUPP\_WATCHDOG}{HWSUPP\_WATCHDOG}{HWSUPPWATCHDOG}

  La valeur \var{'yes'} dans cette variable active le démon Watchdog, si
  le matériel sélectionné est équipé du Watchdog. le Watchdog redémarre
  automatiquement le système qui a été monentanémente arrêté pour une
  raison quelconque.

\config {HWSUPP\_CPUFREQ}{HWSUPP\_CPUFREQ}{HWSUPPCPUFREQ}

  Si vous avez indiquez \var{'yes'} dans cette variable, vous pouvez gérer la fréquence
  du processeur en fonction de la demande en ressources du système et des applications.

\config {HWSUPP\_CPUFREQ\_GOVERNOR}{HWSUPP\_CPUFREQ\_GOVERNOR}{HWSUPPCPUFREQGOVERNOR}

  Dans cette variable vous sélectionnez le gouverneur pour la fréquence du processeur.
  Le gouverneur sélectionné contrôle le comportement et le réglage de la fréquence.
  Vous pouvez sélectionner l'un d'entre eux~:

  \begin{itemize}
    \item performance

      Le CPU fonctionne toujours avec la fréquence la plus élevée.

    \item ondemand

      La fréquence du CPU sera ajustée en fonction de l'utilisation du CPU.
      La fréquence peut changer très rapidement.

    \item conservative

      La fréquence du CPU sera ajustée en fonction de l'utilisation du CPU.
      La fréquence est modifiée étape par étape.

    \item powersave

      Le CPU fonctionne toujours avec la fréquence la plus basse.

    \item userspace

      La fréquence du CPU peut être réglée manuellement ou par un script utilisateur via 
      la variable sysfs dans \var{/devices/system/cpu/cpu\emph{<n>}/cpufreq/scaling\_setspeed}.
  \end{itemize}

\config {HWSUPP\_LED\_N}{HWSUPP\_LED\_N}{HWSUPPLEDN}

  Vous indiquez dans cette variable le nombre de LEDs, les appareils sont
  différents et peuvent avoir un nombre de LED différente.

\config {HWSUPP\_LED\_x}{HWSUPP\_LED\_x}{HWSUPPLEDx}

  Vous indiquez ici la valeur qui sera affichée par la LED, les valeurs
  suivantes sont disponibles~:

  \begin{itemize}
    \item ready~- Le routeur fli4l est prêt
    \item online~- Le routeur fli4l est connecté à l'Internet
    \item trigger~- La LED est commandée par un déclencheur du kernel
    \item user~- L'affichage est commandé par un script utilisateur
  \end{itemize}

  \footnotetext{Si vous avez paramétré la variable \var{HWSUPP\_LED\_x='ready'}
  avec cette valeur ready, la progression du boot sera indiquée par une séquence
  de clignotement de la LED (voir appendix \ref{sec:blink}).}

  La liste des valeurs peut être prolongée en utilisant d'autres paquetages.
  Si vous chargez le paquetage WLAN l'affichage pour le sans fil sera possible.
  \begin{itemize}
    \item wlan~- Le WLAN est activé
  \end{itemize}

  Pour créer une telle extension vous avez dans l'annexe \ref{sec:developer}
  des conseils pour les développeurs de paquetages.

\config {HWSUPP\_LED\_x\_DEVICE}{HWSUPP\_LED\_x\_DEVICE}{HWSUPPLEDxDEVICE}

  Dans cette variable vous indiquez un périphérique pour la LED du matériel utilisé. Soit
  vous paramétrez un périphérique pour la LED qui se trouve dans le répertoire \var{/sys/class/leds/}
  du routeur, soit vous paramétrez un nombre GPIO \footnote{GPIO (General Purpose Input/Output,
  c'est-à-dire entrée/sortie pour un usage général) le nombre correspond à la position
  physique de la broche sur le circuit intégré, le comportement de la broche peut être
  programmé au moment de son exécution, notamment s'il s'agit d'une entrée ou d'une sortie.}

  Une liste de noms des périphériques pour la LED peut être trouvées dans la documentation
  annexe \jump{sec:leddevices}{\og{}Périphérique disponible pour la LED\fg{}},
  selon le matériel spécifique dans \var{HSUPP\_TYPE}.

  Si vous indiquez un nombre GPIO il doit être indiqué dans le format \var{gpio::x}. Si vous
  avez indiqué un GPIO, le périphérique correspondant à la LED sera crée automatiquement.
  Si vous placez le caractère \var{/} devant GPIO, son fonctionnement sera inversé.

  Exemple~:
  \begin{verbatim}
    HWSUPP_LED_1_DEVICE='apu::1'      # LED 1 sur PC engines APU
    HWSUPP_LED_2_DEVICE='gpio::237'   # GPIO 237
    HWSUPP_LED_3_DEVICE='/gpio::245'  # GPIO 245 inversé
    HWSUPP_LED_4_DEVICE='led0'
  \end{verbatim}

\config {HWSUPP\_LED\_x\_PARAM}{HWSUPP\_LED\_x\_PARAM}{HWSUPPLEDxPARAM}

  Dans cette variable vous indiquez les paramètres pour l'affichage de la LED.

  Selon la valeur de la variable \var{HWSUPP\_LED\_x}, la variable
  \var{HWSUPP\_LED\_x\_PARAM} aura une signification différente.

  Si vous avez paramétré la variable \var{HWSUPP\_LED\_x='trigger'}, avec la valeur trigger
  (ou déclencheur), le contrôle d'activation de la LED doit être paramétré dans la variable
  \var{HWSUPP\_LED\_x\_PARAM}.

  Dans cette variable vous définissez le trigger (ou déclencheur) qui commande la LED.
  Les déclencheurs disponibles peuvent être affichés avec la commande
  \var{cat /sys/class/leds/*/trigger}.

  Parmi les déclencheurs créés, par exemple netfilter ou les pilotes de matériels comme ath9k,
  d'autres modules de déclencheurs peuvent être chargés via la variable \var{HWSUPP\_DRIVER\_x}.

  Exemple~:
  \begin{verbatim}
    HWSUPP_LED_1='trigger'
    HWSUPP_LED_1_TRIGGER='heartbeat'
    HWSUPP_LED_2='trigger'
    HWSUPP_LED_2_TRIGGER='netfilter-ssh'
  \end{verbatim}

  Si dans la variable \var{'HWSUPP\_LED\_x'} vous avez indiquez la valeur \var{'user'},
  vous devez indiquer dans la variable \var{HWSUPP\_LED\_PARAM} le nom du script pour la LED
  ainsi que le chemin de celui-ci.

  Exemple~:
  \begin{verbatim}
    HWSUPP_LED_1='user'
    HWSUPP_LED_1_PARAM='/usr/local/bin/myledscript'
  \end{verbatim}

% ##TODO## ? move to wlan doc begin
  Si vous avez indiquer dans la variable la valeur \var{HWSUPP\_LED\_x='wlan'},
  vous devez indiquer dans la variable \var{HWSUPP\_LED\_x\_PARAM} le nom du périphérique.

  Vous pouvez indiquer dans cette variable un ou plusieurs périphériques WLAN (ou sans fil).
  Si vous avez indiquez plusieurs périphériques WLAN ils seront séparés par un espace.

  Si vous avez indiqué plusieurs WLAN et si vous avez activé la LED, elle aura
  la signification suivante~:

  \begin{itemize}
    \item Éteinte~- Tous les périphériques sans fil sont inactifs
    \item Clignotante~- Une partie des périphériques sans fil est actif
    \item Allumée~- Tous les périphériques sans fil sont actifs
  \end{itemize}

  Exemple~:
  \begin{verbatim}
    HWSUPP_LED_1='wlan'
    HWSUPP_LED_1_WLAN='wlan0 wlan1'
  \end{verbatim}
% ##TODO## ? move to wlan doc end

\config {HWSUPP\_BOOT\_LED}{HWSUPP\_BOOT\_LED}{HWSUPPBOOTLED}

  Dans cette variable vous pouvez indiquer la séquence de clignotement de la LED par rapport
  à la progression du boot.

  Si vous avez indiquer dans la variable la valeur \var{HWSUPP\_LED\_x='ready'},
  la variable \var{HWSUPP\_BOOT\_LED} sera ignorée.

\config {HWSUPP\_BUTTON\_N}{HWSUPP\_BUTTON\_N}{HWSUPPBUTTONN}

  Dans cette variable vous indiquez le nombre de boutons. Le nombre de boutons
  dépend du matériel utilisé.

\config {HWSUPP\_BUTTON\_x}{HWSUPP\_BUTTON\_x}{HWSUPPBUTTONx}

  Dans cette variable vous définissez l'action qui doit être exécutée lorsque
  vous pressez le bouton. Les actions suivantes sont possibles~:

  \begin{itemize}
    \item reset~- Réinitialise le routeur fli4l
    \item online~- Active ou désactive la connexion Internet
    \item user~- Le script utilisateur sera exécutée
  \end{itemize}

  La liste des valeurs peut être prolongée en utilisant d'autres paquetages.
  Si vous chargez le paquetage WLAN une action sur le sans fil sera possible.

  \begin{itemize}
    \item wlan~- Active ou désactive le sans fil (ou WLAN)
  \end{itemize}

\config {HWSUPP\_BUTTON\_x\_DEVICE}{HWSUPP\_BUTTON\_x\_DEVICE}{HWSUPPBUTTONxDEVICE}

  Dans cette variable vous pouvez indiquer un bouton de commande. Il y a deux
  possibilités. La première, vous pouvez indiquer un nombre GPIO. La deuxième,
  vous pouvez indiquer le chemin Input, qui sera lu par la suite en utilisant
  le sous-système evdev. La deuxième option, est si possible, toujours préférable
  aux index GPIO ils ne sont pas forcément stable, car le Kernel est parfois limité.
  
  Le nombre GPIO doit être saisi dans le format \var{gpio::x}. le brochage GPIO soit
  est normalement sur "active-low" c'est à dire qu'il prend la valeur 0 quand le bouton
  est pressé, et 1 quand le bouton n'est pas enfoncé ou soit "active-high" c'est
  l'inverse (c'est à dire 1 quand le bouton est pressé et 0 si le bouton n'est pas
  pressé), en préfixant le paramètre avec le caractère \texttt{/} le fonctionnement
  du bouton sera inversé.
  
  Le chemin Input que vous devez configurer est dans le format \var{evdev:<chemin>}.
  Dans ce cas, la variable \var{HWSUPP\_BUTTON\_x\_DEVICE\_KEY} doit est configurée
  pour activer une touche clavier. Cette configuration est nécessaire contrairement
  à la configuration du brochage GPIO, le dispositif evdev peut encapsuler de nombreuses
  touches différents. Vous trouverez une vue d'ensemble de ces touches dans
  l'annexe au chapitre \jump{sec:hwsupp:evdev-keys}{"Code pour les touches du clavier}.
  
  Une liste prédéfinis des chemins Input pour chaque matériel configuré dans \var{HSUPP\_TYPE}
  est indiqué dans l'annexe au chapitre \jump{sec:buttondevices}{"Périphérique disposant de bouton"}.

  Exemples~:
  \begin{verbatim}
    HWSUPP_BUTTON_1_DEVICE='/gpio::237' # GPIO line #237, active-high
    HWSUPP_BUTTON_2_DEVICE='evdev:isa0060/serio0/input0' # clavier AT,
    HWSUPP_BUTTON_2_DEVICE_KEY='88'                      # touche F12
  \end{verbatim}

\config {HWSUPP\_BUTTON\_x\_PARAM}{HWSUPP\_BUTTON\_x\_PARAM}{HWSUPPBUTTONxPARAM}

  Dans cette variable vous indiquez les paramètres pour l'action du bouton.

  Selon la valeur de la variable \var{HWSUPP\_BUTTON\_x}, la variable
  \var{HWSUPP\_BUTTON\_x\_PARAM} aura une signification différente.

  Si vous avez indiquez dans la variable la valeur \var{HWSUPP\_BUTTON\_x='user'}, 

  Vous devez placez dans \var{HWSUPP\_BUTTON\_x\_PARAM} le nom du fichier script à
  exécuter lorsque le bouton sera enfoncé.

  Exemple~:
  \begin{verbatim}
    HWSUPP_BUTTON_1='user'
    HWSUPP_BUTTON_2_WLAN='/usr/local/bin/myscript'
  \end{verbatim}

% ##TODO## ? move to wlan doc begin
  Si vous avez indiquer dans la variable la valeur \var{HWSUPP\_BUTTON\_x\_ACTION='wlan'},
  vous devez indiquer dans la variable \var{HWSUPP\_BUTTON\_x\_PARAM} le nom du périphérique
  sans fil.

  Si dans cette variable vous indiquez un ou plusieurs périphériques WLAN, ils seront surveillés
  par le systéme. Si vous indiquez plusieurs périphériques sans fil, vous devez les séparés par
  un espace. Grâce à la pression sur la bouton le périphérique sera activé ou désactivé.

  Exemple~:
  \begin{verbatim}
    HWSUPP_BUTTON_2='wlan'
    HWSUPP_BUTTON_2_WLAN='wlan0 wlan1'
  \end{verbatim}
% ##TODO## ? move to wlan doc end

\subsection{Paramètre pour expert}

  Les paramètres suivants doivent être utilisés uniquement lorsque vous savez exactement
  \begin{itemize}
    \item quel matériel que vous avez et quel pilote supplémentaire est nécessaires.
    \item l'adresse et le type de périphérique I\textsuperscript{2}C\footnotemark.

    \footnotetext{Un bus I\textsuperscript{2}C ou un SMBus est un bus série sur PC, il est
	utilisé par exemple pour la lecture de la température avec un capteur.
	Dans de nombreux cas, le bus I\textsuperscript{2}C ou le SMBus est disponible par
	l'intermédiaire un connecteur à broche il peut être utilisé pour une extension de
	matériel personnel.}
  \end{itemize}

  En activant les paramètres experts, vous aurez un message d'avertissement lors de
  la construction de fli4l avec la commande mkfli4l.

\config {HWSUPP\_DRIVER\_N}{HWSUPP\_DRIVER\_N}{HWSUPPDRIVERN}

  Dans cette variable vous indiquez le nombre de pilotes supplémentaires.
  Les pilotes dans la variable \var{HWSUPP\_DRIVER\_x} seront chargés dans l'ordre
  enregistré.

\config {HWSUPP\_DRIVER\_x}{HWSUPP\_DRIVER\_x}{HWSUPPDRIVERx}

  Dans cette variable vous indiquez le nom du pilote (sans l'extension du fichier \var{.ko}).

Exemple~:
\begin{verbatim}
HWSUPP_DRIVER_N='2'
HWSUPP_DRIVER_1='i2c-piix4'     # pilote du bus I2C
HWSUPP_DRIVER_2='gpio-pcf857x'  # I2C extension GPIO
\end{verbatim}

\config {HWSUPP\_I2C\_N}{HWSUPP\_I2C\_N}{HWSUPPI2CN}

  Dans cette variable vous indiquez le nombre de périphérique
  I\textsuperscript{2}C à charger.

I\textsuperscript{2}C ne supporte pas du tout le mécanisme PnP.
  Par conséquent, vous devez avoir un numéro de bus pour chaque
  périphérique I\textsuperscript{2}C, l'adresse du périphérique et
  le type de périphérique pour la configuration.

\config {HWSUPP\_I2C\_x\_BUS}{HWSUPP\_I2C\_x\_BUS}{HWSUPPI2CxBUS}

  Dans cette variable vous indiquez le numéro de bus I\textsuperscript{2}C
  donc le périphérique sera associé.

  Le numéro de bus doit être saisi dans le format suivant \var{i2c-x}.

\config {HWSUPP\_I2C\_x\_ADDRESS}{HWSUPP\_I2C\_x\_ADDRESS}{HWSUPPI2CxADDRESS}

  Dans cette variable vous indiquez l'adresse du périphérique I\textsuperscript{2}C.

  L'adresse doit être indiquée comme un nombre hexadécimal et dans la plage de
  \var{0x03} à \var{0x77}.

\config {HWSUPP\_I2C\_x\_DEVICE}{HWSUPP\_I2C\_x\_DEVICE}{HWSUPPI2CxDEVICE}

  Dans cette variable vous indiquez le type de périphérique I\textsuperscript{2}C
  qui sera supporté par le pilote précédemment chargé.

Exemple~:
\begin{verbatim}
HWSUPP_I2C_N='1'
HWSUPP_I2C_1_BUS='i2c-1'
HWSUPP_I2C_1_ADDRESS='0x38'
HWSUPP_I2C_1_DEVICE='pcf8574a' # supporté par le pilote gpio-pcf857x
\end{verbatim}

\subsection {Prise en charge des cartes VPN}

\config {OPT\_VPN\_CARD}{OPT\_VPN\_CARD}{OPTVPNCARD}

  La valeur \var{'no'} dans cette variable désactive complètement le paquetage
  OPT\_VPN\_CARD. Il n'y aura aucun changement sur le support de boot de
  l'archive fli4l \var{rootfs.img} n'y dans l'archive \var{opt.img}. Pour finir
  OPT\_VPN\_CARD n'écrase aucune partie de l'installation fli4l.\\
  Pour activer la variable OPT\_VPN\_CARD du paquetage \var{OPT\_VPN\_CARD}
  vous devez placer la valeur sur \var{'yes'}.

\config {VPN\_CARD\_TYPE}{VPN\_CARD\_TYPE}{VPNCARDTYPE}

  Dans cette variable vous pouvez indiquez l'accélérateur VPN. La carte VPN
  décharge le CPU des tâches informatiques intensives de cryptage et de la
  fonction de hachage. Les valeurs suivantes sont disponibles~:

  \begin{itemize}
    \item hifn7751~- Pour carte Soekris vpn1401 et vpn1411
    \item hifnhipp
  \end{itemize}

\end{description}
